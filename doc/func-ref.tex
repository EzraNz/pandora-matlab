\newcommand{\fidxl}[1]{{\small \texttt{#1}}}
\newcommand{\fidxlb}[1]{{\small \bf \texttt{#1}}}
\newcommand{\methodline}{%
  {\normalsize \vspace{1ex} \hrule width \columnwidth \vspace{1ex}}%
}%
\subsection{Class \texttt{cip\_trace}}%
\index[funcref]{cip_trace@\fidxlb{cip\_trace}}%
\label{ref_cip_trace}%
\hypertarget{ref_cip_trace}{}%
\subsubsection[Constructor \texttt{cip\_trace}]{Constructor \texttt{cip\_trace/cip\_trace}}%
\index[funcref]{cip_trace@\fidxlb{cip\_trace}!cip_trace@\fidxl{cip\_trace}}%
\label{ref_cip_trace__cip_trace}%
\hypertarget{ref_cip_trace__cip_trace}{}%
\begin{description}
\item[Summary:]A trace with a current injection pulse (CIP).
%
\item[Usage:]~%
\begin{lyxcode}%
obj = cip\_trace(datasrc, dt, dy,
		  pulse\_time\_start, pulse\_time\_width, id, props)
%
\end{lyxcode}%
%
%
\item[Parameters:]~
\begin{description}%
\item[\texttt{datasrc}:]
 A vector of data points containing the spike shape.
\item[\texttt{dt}:]
 Time resolution [s].
\item[\texttt{dy}:]
 y-axis resolution [ISI (V, A, etc.)]
\item[\texttt{pulse\_time\_start, pulse\_time\_width}:]


Start and width of the pulse [dt]\item[\texttt{id}:]
 Identification string.
\item[\texttt{props}:]
 A structure with any optional properties, such as:
\begin{description}%
\item[\texttt{trace\_time\_start}:]
 Samples in the beginning to discard [dt]

(see trace for more)\end{description}%
\end{description}%
%
\item[Returns a structure object with the following fields:]~

	trace, pulse\_time\_start, pulse\_time\_width, props.
%
%
\item[See also:]%
\hyperlink{ref_trace}{\texttt{trace}}%
\ (p.~\pageref{ref_trace})%
\index[funcref]{@\fidxl{trace}}%
, \hyperlink{ref_spikes}{\texttt{spikes}}%
\ (p.~\pageref{ref_spikes})%
\index[funcref]{@\fidxl{spikes}}%
, \hyperlink{ref_spike_shape}{\texttt{spike\_shape}}%
\ (p.~\pageref{ref_spike_shape})%
\index[funcref]{@\fidxl{spike\_shape}}%
, \hyperlink{ref_period}{\texttt{period}}%
\ (p.~\pageref{ref_period})%
\index[funcref]{@\fidxl{period}}%
%
\item[Author:]%
Cengiz Gunay <cgunay@emory.edu>, 2004/07/30%
\end{description}
\methodline%
\subsubsection[Method \texttt{periodPulseIni50ms}]{Method \texttt{cip\_trace/periodPulseIni50ms}}%
\index[funcref]{cip_trace@\fidxlb{cip\_trace}!periodPulseIni50ms@\fidxl{periodPulseIni50ms}}%
\label{ref_cip_trace__periodPulseIni50ms}%
\hypertarget{ref_cip_trace__periodPulseIni50ms}{}%
\begin{description}
\item[Summary:]Returns the first 50ms of the CIP period of 
			cip\_trace, t. 
%
\item[Usage:]~%
\begin{lyxcode}%
the\_period = periodPulseIni50ms(t)
%
\end{lyxcode}%
%
%
\item[Parameters:]~
\begin{description}%
\item[\texttt{t}:]
 A trace object.
\end{description}%
%
\item[Returns:]~

	the\_period: A period object.
%
%
\item[See also:]%
\hyperlink{ref_period}{\texttt{period}}%
\ (p.~\pageref{ref_period})%
\index[funcref]{@\fidxl{period}}%
, \hyperlink{ref_cip_trace}{\texttt{cip\_trace}}%
\ (p.~\pageref{ref_cip_trace})%
\index[funcref]{@\fidxl{cip\_trace}}%
, \hyperlink{ref_trace}{\texttt{trace}}%
\ (p.~\pageref{ref_trace})%
\index[funcref]{@\fidxl{trace}}%
%
\item[Author:]%
Cengiz Gunay <cgunay@emory.edu>, 2004/08/25%
\end{description}
\methodline%
\subsubsection[Method \texttt{display}]{Method \texttt{cip\_trace/display}}%
\index[funcref]{cip_trace@\fidxlb{cip\_trace}!display@\fidxl{display}}%
\label{ref_cip_trace__display}%
\hypertarget{ref_cip_trace__display}{}%
\begin{description}
%
%
%
%
%
%
%
\item[Author:]%
Cengiz Gunay <cgunay@emory.edu>, 2004/08/04%
\end{description}
\methodline%
\subsubsection[Method \texttt{periodRecSpontRestPeriod}]{Method \texttt{cip\_trace/periodRecSpontRestPeriod}}%
\index[funcref]{cip_trace@\fidxlb{cip\_trace}!periodRecSpontRestPeriod@\fidxl{periodRecSpontRestPeriod}}%
\label{ref_cip_trace__periodRecSpontRestPeriod}%
\hypertarget{ref_cip_trace__periodRecSpontRestPeriod}{}%
\begin{description}
%
\item[Usage:]~%
\begin{lyxcode}%
the\_period = periodRecSpont(t)
%
\end{lyxcode}%
%
%
\item[Parameters:]~
\begin{description}%
\item[\texttt{t}:]
 A trace object.
\item[\texttt{iniPeriod}:]
 the time following pulse offset that is ignored. The rest of

the time is kept\end{description}%
%
\item[Returns:]~

	the\_period: A period object.
%
%
\item[See also:]%
\hyperlink{ref_period}{\texttt{period}}%
\ (p.~\pageref{ref_period})%
\index[funcref]{@\fidxl{period}}%
, \hyperlink{ref_cip_trace}{\texttt{cip\_trace}}%
\ (p.~\pageref{ref_cip_trace})%
\index[funcref]{@\fidxl{cip\_trace}}%
, \hyperlink{ref_trace}{\texttt{trace}}%
\ (p.~\pageref{ref_trace})%
\index[funcref]{@\fidxl{trace}}%
%
\item[Author:]%
Cengiz Gunay <cgunay@emory.edu>,Tom Sangrey 2006/01/26%
\end{description}
\methodline%
\subsubsection[Method \texttt{get}]{Method \texttt{cip\_trace/get}}%
\index[funcref]{cip_trace@\fidxlb{cip\_trace}!get@\fidxl{get}}%
\label{ref_cip_trace__get}%
\hypertarget{ref_cip_trace__get}{}%
\begin{description}
\item[Summary:]Defines generic attribute retrieval for objects.
%
%
%
%
%
%
%
\item[Author:]%
Cengiz Gunay <cgunay@emory.edu>, 2004/09/14%
\end{description}
\methodline%
\subsubsection[Method \texttt{set}]{Method \texttt{cip\_trace/set}}%
\index[funcref]{cip_trace@\fidxlb{cip\_trace}!set@\fidxl{set}}%
\label{ref_cip_trace__set}%
\hypertarget{ref_cip_trace__set}{}%
\begin{description}
\item[Summary:]Generic method for setting object attributes.
%
%
%
%
%
%
%
\item[Author:]%
Cengiz Gunay <cgunay@emory.edu>, 2004/10/08%
\end{description}
\methodline%
\subsubsection[Method \texttt{getBurstResults}]{Method \texttt{cip\_trace/getBurstResults}}%
\index[funcref]{cip_trace@\fidxlb{cip\_trace}!getBurstResults@\fidxl{getBurstResults}}%
\label{ref_cip_trace__getBurstResults}%
\hypertarget{ref_cip_trace__getBurstResults}{}%
\begin{description}
\item[Summary:]Calculate test results related to Burst behavior.
%
\item[Usage:]~%
\begin{lyxcode}%
results = getRateResults(a\_cip\_trace, a\_spikes)
%
\end{lyxcode}%
%
%
\item[Parameters:]~
\begin{description}%
\item[\texttt{a\_cip\_trace}:]
 A cip\_trace object.
\item[\texttt{a\_spikes}:]
 A spikes object.
\end{description}%
%
\item[Returns:]~

	results: A structure associating test names with result values.
%
%
\item[See also:]%
\hyperlink{ref_cip_trace}{\texttt{cip\_trace}}%
\ (p.~\pageref{ref_cip_trace})%
\index[funcref]{@\fidxl{cip\_trace}}%
, \hyperlink{ref_spikes}{\texttt{spikes}}%
\ (p.~\pageref{ref_spikes})%
\index[funcref]{@\fidxl{spikes}}%
, \hyperlink{ref_spike_shape}{\texttt{spike\_shape}}%
\ (p.~\pageref{ref_spike_shape})%
\index[funcref]{@\fidxl{spike\_shape}}%
%
\item[Author:]%
Cengiz Gunay <cgunay@emory.edu>, 2004/08/30, Tom Sangrey%
\end{description}
\methodline%
\subsubsection[Method \texttt{periodIniSpont}]{Method \texttt{cip\_trace/periodIniSpont}}%
\index[funcref]{cip_trace@\fidxlb{cip\_trace}!periodIniSpont@\fidxl{periodIniSpont}}%
\label{ref_cip_trace__periodIniSpont}%
\hypertarget{ref_cip_trace__periodIniSpont}{}%
\begin{description}
\item[Summary:]Returns the initial spontaneous activity period of 
		cip\_trace, t. 
%
\item[Usage:]~%
\begin{lyxcode}%
the\_period = periodIniSpont(t)
%
\end{lyxcode}%
%
%
\item[Parameters:]~
\begin{description}%
\item[\texttt{t}:]
 A trace object.
\end{description}%
%
\item[Returns:]~

	the\_period: A period object.
%
%
\item[See also:]%
\hyperlink{ref_period}{\texttt{period}}%
\ (p.~\pageref{ref_period})%
\index[funcref]{@\fidxl{period}}%
, \hyperlink{ref_cip_trace}{\texttt{cip\_trace}}%
\ (p.~\pageref{ref_cip_trace})%
\index[funcref]{@\fidxl{cip\_trace}}%
, \hyperlink{ref_trace}{\texttt{trace}}%
\ (p.~\pageref{ref_trace})%
\index[funcref]{@\fidxl{trace}}%
%
\item[Author:]%
Cengiz Gunay <cgunay@emory.edu>, 2004/08/25%
\end{description}
\methodline%
\subsubsection[Method \texttt{spikes}]{Method \texttt{cip\_trace/spikes}}%
\index[funcref]{cip_trace@\fidxlb{cip\_trace}!spikes@\fidxl{spikes}}%
\label{ref_cip_trace__spikes}%
\hypertarget{ref_cip_trace__spikes}{}%
\begin{description}
\item[Summary:]Convert cip\_trace to spikes object for spike timing calculations.
%
\item[Usage:]~%
\begin{lyxcode}%
obj = spikes(trace, plotit)
%
\end{lyxcode}%
%
\item[Description:]%
Creates a spikes object by finding the spikes in the three 
 separate periods, initial spontaneous activity period, CIP period, and
 final recovery period.
%%
\item[Parameters:]~
\begin{description}%
\item[\texttt{trace}:]
 A trace object.
\item[\texttt{plotit}:]
 If non-zero, a plot is generated for showing spikes found

(optional).\end{description}%
%
%
%
\item[See also:]%
\hyperlink{ref_spikes}{\texttt{spikes}}%
\ (p.~\pageref{ref_spikes})%
\index[funcref]{@\fidxl{spikes}}%
, \hyperlink{ref_period}{\texttt{period}}%
\ (p.~\pageref{ref_period})%
\index[funcref]{@\fidxl{period}}%
%
\item[Author:]%
Cengiz Gunay <cgunay@emory.edu>, 2004/08/25%
\end{description}
\methodline%
\subsubsection[Method \texttt{getCIPResults}]{Method \texttt{cip\_trace/getCIPResults}}%
\index[funcref]{cip_trace@\fidxlb{cip\_trace}!getCIPResults@\fidxl{getCIPResults}}%
\label{ref_cip_trace__getCIPResults}%
\hypertarget{ref_cip_trace__getCIPResults}{}%
\begin{description}
\item[Summary:]Calculate test results about CIP protocol.
%
\item[Usage:]~%
\begin{lyxcode}%
results = getCIPResults(a\_cip\_trace, a\_spikes)
%
\end{lyxcode}%
%
%
\item[Parameters:]~
\begin{description}%
\item[\texttt{a\_cip\_trace}:]
 A cip\_trace object.
\item[\texttt{a\_spikes}:]
 A spikes object.
\end{description}%
%
\item[Returns:]~

	results: A structure associating test names with result values.
%
%
\item[See also:]%
\hyperlink{ref_cip_trace}{\texttt{cip\_trace}}%
\ (p.~\pageref{ref_cip_trace})%
\index[funcref]{@\fidxl{cip\_trace}}%
, \hyperlink{ref_spikes}{\texttt{spikes}}%
\ (p.~\pageref{ref_spikes})%
\index[funcref]{@\fidxl{spikes}}%
, \hyperlink{ref_spike_shape}{\texttt{spike\_shape}}%
\ (p.~\pageref{ref_spike_shape})%
\index[funcref]{@\fidxl{spike\_shape}}%
%
\item[Author:]%
Cengiz Gunay <cgunay@emory.edu>, 2004/08/30%
\end{description}
\methodline%
\subsubsection[Method \texttt{periodRecSpont1}]{Method \texttt{cip\_trace/periodRecSpont1}}%
\index[funcref]{cip_trace@\fidxlb{cip\_trace}!periodRecSpont1@\fidxl{periodRecSpont1}}%
\label{ref_cip_trace__periodRecSpont1}%
\hypertarget{ref_cip_trace__periodRecSpont1}{}%
\begin{description}
\item[Summary:]Returns the first half of the recovery spontaneous
		 activity period of cip\_trace, t. 
%
\item[Usage:]~%
\begin{lyxcode}%
the\_period = periodRecSpont1(t)
%
\end{lyxcode}%
%
%
\item[Parameters:]~
\begin{description}%
\item[\texttt{t}:]
 A trace object.
\end{description}%
%
\item[Returns:]~

	the\_period: A period object.
%
%
\item[See also:]%
\hyperlink{ref_period}{\texttt{period}}%
\ (p.~\pageref{ref_period})%
\index[funcref]{@\fidxl{period}}%
, \hyperlink{ref_cip_trace}{\texttt{cip\_trace}}%
\ (p.~\pageref{ref_cip_trace})%
\index[funcref]{@\fidxl{cip\_trace}}%
, \hyperlink{ref_trace}{\texttt{trace}}%
\ (p.~\pageref{ref_trace})%
\index[funcref]{@\fidxl{trace}}%
%
\item[Author:]%
Cengiz Gunay <cgunay@emory.edu>, 2004/08/25%
\end{description}
\methodline%
\subsubsection[Method \texttt{periodRecSpont2}]{Method \texttt{cip\_trace/periodRecSpont2}}%
\index[funcref]{cip_trace@\fidxlb{cip\_trace}!periodRecSpont2@\fidxl{periodRecSpont2}}%
\label{ref_cip_trace__periodRecSpont2}%
\hypertarget{ref_cip_trace__periodRecSpont2}{}%
\begin{description}
\item[Summary:]Returns the second half of the recovery spontaneous
		 activity period of cip\_trace, t. 
%
\item[Usage:]~%
\begin{lyxcode}%
the\_period = periodRecSpont2(t)
%
\end{lyxcode}%
%
%
\item[Parameters:]~
\begin{description}%
\item[\texttt{t}:]
 A trace object.
\end{description}%
%
\item[Returns:]~

	the\_period: A period object.
%
%
\item[See also:]%
\hyperlink{ref_period}{\texttt{period}}%
\ (p.~\pageref{ref_period})%
\index[funcref]{@\fidxl{period}}%
, \hyperlink{ref_cip_trace}{\texttt{cip\_trace}}%
\ (p.~\pageref{ref_cip_trace})%
\index[funcref]{@\fidxl{cip\_trace}}%
, \hyperlink{ref_trace}{\texttt{trace}}%
\ (p.~\pageref{ref_trace})%
\index[funcref]{@\fidxl{trace}}%
%
\item[Author:]%
Cengiz Gunay <cgunay@emory.edu>, 2004/08/25%
\end{description}
\methodline%
\subsubsection[Method \texttt{getProfileAllSpikes}]{Method \texttt{cip\_trace/getProfileAllSpikes}}%
\index[funcref]{cip_trace@\fidxlb{cip\_trace}!getProfileAllSpikes@\fidxl{getProfileAllSpikes}}%
\label{ref_cip_trace__getProfileAllSpikes}%
\hypertarget{ref_cip_trace__getProfileAllSpikes}{}%
\begin{description}
\item[Summary:]Creates a cip\_trace\_allspikes\_profile object by collecting test results of a cip\_trace, analyzing each individual spike.
%
\item[Usage:]~%
\begin{lyxcode}%
profile\_obj = getProfileAllSpikes(a\_cip\_trace)
%
\end{lyxcode}%
%
\item[Description:]%
Analyzes the spontaneous (periodIniSpont), pulse (periodPulse) and the
 recovery (periodRecSpont) periods separately and produces spike shape
 distribution results. Rate and CIP measurements are added to these.
%%
\item[Parameters:]~
\begin{description}%
\item[\texttt{a\_cip\_trace}:]
 A cip\_trace object.
\end{description}%
%
\item[Returns:]~

	profile\_obj: A cip\_trace\_allspikes\_profile object.
%
%
\item[See also:]%
\hyperlink{ref_cip_trace}{\texttt{cip\_trace}}%
\ (p.~\pageref{ref_cip_trace})%
\index[funcref]{@\fidxl{cip\_trace}}%
, \hyperlink{ref_cip_trace_allspikes_profile}{\texttt{cip\_trace\_allspikes\_profile}}%
\ (p.~\pageref{ref_cip_trace_allspikes_profile})%
\index[funcref]{@\fidxl{cip\_trace\_allspikes\_profile}}%
%
\item[Author:]%
Cengiz Gunay <cgunay@emory.edu>, 2005/04/26%
\end{description}
\methodline%
\subsubsection[Method \texttt{periodPulse}]{Method \texttt{cip\_trace/periodPulse}}%
\index[funcref]{cip_trace@\fidxlb{cip\_trace}!periodPulse@\fidxl{periodPulse}}%
\label{ref_cip_trace__periodPulse}%
\hypertarget{ref_cip_trace__periodPulse}{}%
\begin{description}
\item[Summary:]Returns the CIP period of cip\_trace, t. 
%
\item[Usage:]~%
\begin{lyxcode}%
the\_period = periodPulse(t)
%
\end{lyxcode}%
%
%
\item[Parameters:]~
\begin{description}%
\item[\texttt{t}:]
 A trace object.
\end{description}%
%
\item[Returns:]~

	the\_period: A period object.
%
%
\item[See also:]%
\hyperlink{ref_period}{\texttt{period}}%
\ (p.~\pageref{ref_period})%
\index[funcref]{@\fidxl{period}}%
, \hyperlink{ref_cip_trace}{\texttt{cip\_trace}}%
\ (p.~\pageref{ref_cip_trace})%
\index[funcref]{@\fidxl{cip\_trace}}%
, \hyperlink{ref_trace}{\texttt{trace}}%
\ (p.~\pageref{ref_trace})%
\index[funcref]{@\fidxl{trace}}%
%
\item[Author:]%
Cengiz Gunay <cgunay@emory.edu>, 2004/08/25%
\end{description}
\methodline%
\subsubsection[Method \texttt{calcRecSpontPotAvg}]{Method \texttt{cip\_trace/calcRecSpontPotAvg}}%
\index[funcref]{cip_trace@\fidxlb{cip\_trace}!calcRecSpontPotAvg@\fidxl{calcRecSpontPotAvg}}%
\label{ref_cip_trace__calcRecSpontPotAvg}%
\hypertarget{ref_cip_trace__calcRecSpontPotAvg}{}%
\begin{description}
\item[Summary:]Calculates the average potential value of the 
			recovery period of the cip\_trace, t. 
%
\item[Usage:]~%
\begin{lyxcode}%
avg\_val = calcRecSpontPotAvg(t)
%
\end{lyxcode}%
%
%
\item[Parameters:]~
\begin{description}%
\item[\texttt{t}:]
 A cip\_trace object.
\end{description}%
%
\item[Returns:]~

	avg\_val: The avg value [dy].
%
%
\item[See also:]%
\hyperlink{ref_period}{\texttt{period}}%
\ (p.~\pageref{ref_period})%
\index[funcref]{@\fidxl{period}}%
, \hyperlink{ref_trace}{\texttt{trace}}%
\ (p.~\pageref{ref_trace})%
\index[funcref]{@\fidxl{trace}}%
, \hyperlink{ref_trace__calcAvg}{\texttt{trace/calcAvg}}%
\ (p.~\pageref{ref_trace__calcAvg})%
\index[funcref]{trace@\fidxlb{trace}!calcAvg@\fidxl{calcAvg}}%
%
\item[Author:]%
Cengiz Gunay <cgunay@emory.edu>, 2004/08/25%
\end{description}
\methodline%
\subsubsection[Method \texttt{periodPulseHalf1}]{Method \texttt{cip\_trace/periodPulseHalf1}}%
\index[funcref]{cip_trace@\fidxlb{cip\_trace}!periodPulseHalf1@\fidxl{periodPulseHalf1}}%
\label{ref_cip_trace__periodPulseHalf1}%
\hypertarget{ref_cip_trace__periodPulseHalf1}{}%
\begin{description}
\item[Summary:]Returns the first half of the CIP period of cip\_trace, t. 
%
\item[Usage:]~%
\begin{lyxcode}%
the\_period = periodPulseHalf1(t)
%
\end{lyxcode}%
%
%
\item[Parameters:]~
\begin{description}%
\item[\texttt{t}:]
 A trace object.
\end{description}%
%
\item[Returns:]~

	the\_period: A period object.
%
%
\item[See also:]%
\hyperlink{ref_period}{\texttt{period}}%
\ (p.~\pageref{ref_period})%
\index[funcref]{@\fidxl{period}}%
, \hyperlink{ref_cip_trace}{\texttt{cip\_trace}}%
\ (p.~\pageref{ref_cip_trace})%
\index[funcref]{@\fidxl{cip\_trace}}%
, \hyperlink{ref_trace}{\texttt{trace}}%
\ (p.~\pageref{ref_trace})%
\index[funcref]{@\fidxl{trace}}%
%
\item[Author:]%
Cengiz Gunay <cgunay@emory.edu>, 2004/08/25%
\end{description}
\methodline%
\subsubsection[Method \texttt{periodPulseIni50msRest1}]{Method \texttt{cip\_trace/periodPulseIni50msRest1}}%
\index[funcref]{cip_trace@\fidxlb{cip\_trace}!periodPulseIni50msRest1@\fidxl{periodPulseIni50msRest1}}%
\label{ref_cip_trace__periodPulseIni50msRest1}%
\hypertarget{ref_cip_trace__periodPulseIni50msRest1}{}%
\begin{description}
\item[Summary:]Returns the first half of the rest after the 
			first 50ms of the CIP period of cip\_trace, t. 
%
\item[Usage:]~%
\begin{lyxcode}%
the\_period = periodPulseIni50msRest1(t)
%
\end{lyxcode}%
%
%
\item[Parameters:]~
\begin{description}%
\item[\texttt{t}:]
 A trace object.
\end{description}%
%
\item[Returns:]~

	the\_period: A period object.
%
%
\item[See also:]%
\hyperlink{ref_period}{\texttt{period}}%
\ (p.~\pageref{ref_period})%
\index[funcref]{@\fidxl{period}}%
, \hyperlink{ref_cip_trace}{\texttt{cip\_trace}}%
\ (p.~\pageref{ref_cip_trace})%
\index[funcref]{@\fidxl{cip\_trace}}%
, \hyperlink{ref_trace}{\texttt{trace}}%
\ (p.~\pageref{ref_trace})%
\index[funcref]{@\fidxl{trace}}%
%
\item[Author:]%
Cengiz Gunay <cgunay@emory.edu>, 2004/08/25%
\end{description}
\methodline%
\subsubsection[Method \texttt{periodPulseIni50msRest2}]{Method \texttt{cip\_trace/periodPulseIni50msRest2}}%
\index[funcref]{cip_trace@\fidxlb{cip\_trace}!periodPulseIni50msRest2@\fidxl{periodPulseIni50msRest2}}%
\label{ref_cip_trace__periodPulseIni50msRest2}%
\hypertarget{ref_cip_trace__periodPulseIni50msRest2}{}%
\begin{description}
\item[Summary:]Returns the second half of the rest after the 
			first 50ms of the CIP period of cip\_trace, t. 
%
\item[Usage:]~%
\begin{lyxcode}%
the\_period = periodPulseIni50msRest2(t)
%
\end{lyxcode}%
%
%
\item[Parameters:]~
\begin{description}%
\item[\texttt{t}:]
 A trace object.
\end{description}%
%
\item[Returns:]~

	the\_period: A period object.
%
%
\item[See also:]%
\hyperlink{ref_period}{\texttt{period}}%
\ (p.~\pageref{ref_period})%
\index[funcref]{@\fidxl{period}}%
, \hyperlink{ref_cip_trace}{\texttt{cip\_trace}}%
\ (p.~\pageref{ref_cip_trace})%
\index[funcref]{@\fidxl{cip\_trace}}%
, \hyperlink{ref_trace}{\texttt{trace}}%
\ (p.~\pageref{ref_trace})%
\index[funcref]{@\fidxl{trace}}%
%
\item[Author:]%
Cengiz Gunay <cgunay@emory.edu>, 2004/08/25%
\end{description}
\methodline%
\subsubsection[Method \texttt{getPulseSpike}]{Method \texttt{cip\_trace/getPulseSpike}}%
\index[funcref]{cip_trace@\fidxlb{cip\_trace}!getPulseSpike@\fidxl{getPulseSpike}}%
\label{ref_cip_trace__getPulseSpike}%
\hypertarget{ref_cip_trace__getPulseSpike}{}%
\begin{description}
\item[Summary:]Convert a spike in the CIP period to a spike\_shape object.
%
\item[Usage:]~%
\begin{lyxcode}%
obj = getPulseSpike(trace, spikes, spike\_num, props)
%
\end{lyxcode}%
%
\item[Description:]%
Creates a spike\_shape object from desired spike. Calls trace/getSpike method.
%%
\item[Parameters:]~
\begin{description}%
\item[\texttt{trace}:]
 A trace object.
\item[\texttt{spikes}:]
 (Optional) A spikes object obtained from trace, 

calculated automatically if given as [].\item[\texttt{spike\_num}:]
 The index of spike to extract.
\item[\texttt{props}:]
 A structure with any optional properties passed to getSpike.
\end{description}%
%
%
\item[Example:]~
\begin{lyxcode} Get 2nd pulse spike and plot it:\\%
 >> plotFigure(plotResults(getPulseSpike(t, [], 2)))\\%
\end{lyxcode}
%
\item[See also:]%
\hyperlink{ref_spike_shape}{\texttt{spike\_shape}}%
\ (p.~\pageref{ref_spike_shape})%
\index[funcref]{@\fidxl{spike\_shape}}%
, \hyperlink{ref_trace__getSpike}{\texttt{trace/getSpike}}%
\ (p.~\pageref{ref_trace__getSpike})%
\index[funcref]{trace@\fidxlb{trace}!getSpike@\fidxl{getSpike}}%
%
\item[Author:]%
Cengiz Gunay <cgunay@emory.edu>, 2005/04/19%
\end{description}
\methodline%
\subsubsection[Method \texttt{getRateResults}]{Method \texttt{cip\_trace/getRateResults}}%
\index[funcref]{cip_trace@\fidxlb{cip\_trace}!getRateResults@\fidxl{getRateResults}}%
\label{ref_cip_trace__getRateResults}%
\hypertarget{ref_cip_trace__getRateResults}{}%
\begin{description}
\item[Summary:]Calculate test results related to spike rate.
%
\item[Usage:]~%
\begin{lyxcode}%
results = getRateResults(a\_cip\_trace, a\_spikes)
%
\end{lyxcode}%
%
%
\item[Parameters:]~
\begin{description}%
\item[\texttt{a\_cip\_trace}:]
 A cip\_trace object.
\item[\texttt{a\_spikes}:]
 A spikes object.
\end{description}%
%
\item[Returns:]~

	results: A structure associating test names with result values.
%
%
\item[See also:]%
\hyperlink{ref_cip_trace}{\texttt{cip\_trace}}%
\ (p.~\pageref{ref_cip_trace})%
\index[funcref]{@\fidxl{cip\_trace}}%
, \hyperlink{ref_spikes}{\texttt{spikes}}%
\ (p.~\pageref{ref_spikes})%
\index[funcref]{@\fidxl{spikes}}%
, \hyperlink{ref_spike_shape}{\texttt{spike\_shape}}%
\ (p.~\pageref{ref_spike_shape})%
\index[funcref]{@\fidxl{spike\_shape}}%
%
\item[Author:]%
Cengiz Gunay <cgunay@emory.edu>, 2004/08/30%
\end{description}
\methodline%
\subsubsection[Method \texttt{periodPulseIni100msRest1}]{Method \texttt{cip\_trace/periodPulseIni100msRest1}}%
\index[funcref]{cip_trace@\fidxlb{cip\_trace}!periodPulseIni100msRest1@\fidxl{periodPulseIni100msRest1}}%
\label{ref_cip_trace__periodPulseIni100msRest1}%
\hypertarget{ref_cip_trace__periodPulseIni100msRest1}{}%
\begin{description}
%
\item[Usage:]~%
\begin{lyxcode}%
the\_period = periodPulseIni50msRest1(t)
%
\end{lyxcode}%
%
%
\item[Parameters:]~
\begin{description}%
\item[\texttt{t}:]
 A trace object.
\end{description}%
%
\item[Returns:]~

	the\_period: A period object.
%
%
\item[See also:]%
\hyperlink{ref_period}{\texttt{period}}%
\ (p.~\pageref{ref_period})%
\index[funcref]{@\fidxl{period}}%
, \hyperlink{ref_cip_trace}{\texttt{cip\_trace}}%
\ (p.~\pageref{ref_cip_trace})%
\index[funcref]{@\fidxl{cip\_trace}}%
, \hyperlink{ref_trace}{\texttt{trace}}%
\ (p.~\pageref{ref_trace})%
\index[funcref]{@\fidxl{trace}}%
%
\item[Author:]%
Cengiz Gunay <cgunay@emory.edu>, 2004/08/25%
\end{description}
\methodline%
\subsubsection[Method \texttt{periodPulseIni100msRest2}]{Method \texttt{cip\_trace/periodPulseIni100msRest2}}%
\index[funcref]{cip_trace@\fidxlb{cip\_trace}!periodPulseIni100msRest2@\fidxl{periodPulseIni100msRest2}}%
\label{ref_cip_trace__periodPulseIni100msRest2}%
\hypertarget{ref_cip_trace__periodPulseIni100msRest2}{}%
\begin{description}
%
\item[Usage:]~%
\begin{lyxcode}%
the\_period = periodPulseIni50msRest2(t)
%
\end{lyxcode}%
%
%
\item[Parameters:]~
\begin{description}%
\item[\texttt{t}:]
 A trace object.
\end{description}%
%
\item[Returns:]~

	the\_period: A period object.
%
%
\item[See also:]%
\hyperlink{ref_period}{\texttt{period}}%
\ (p.~\pageref{ref_period})%
\index[funcref]{@\fidxl{period}}%
, \hyperlink{ref_cip_trace}{\texttt{cip\_trace}}%
\ (p.~\pageref{ref_cip_trace})%
\index[funcref]{@\fidxl{cip\_trace}}%
, \hyperlink{ref_trace}{\texttt{trace}}%
\ (p.~\pageref{ref_trace})%
\index[funcref]{@\fidxl{trace}}%
%
\item[Author:]%
Cengiz Gunay <cgunay@emory.edu>, 2004/08/25%
\end{description}
\methodline%
\subsubsection[Method \texttt{periodRecSpontIniPeriod}]{Method \texttt{cip\_trace/periodRecSpontIniPeriod}}%
\index[funcref]{cip_trace@\fidxlb{cip\_trace}!periodRecSpontIniPeriod@\fidxl{periodRecSpontIniPeriod}}%
\label{ref_cip_trace__periodRecSpontIniPeriod}%
\hypertarget{ref_cip_trace__periodRecSpontIniPeriod}{}%
\begin{description}
%
\item[Usage:]~%
\begin{lyxcode}%
the\_period = periodRecSpont(t)
%
\end{lyxcode}%
%
%
\item[Parameters:]~
\begin{description}%
\item[\texttt{t}:]
 A trace object.
\item[\texttt{iniPeriod}:]
 the time following pulse offset that is kept, the rest of

the time is ignored.\end{description}%
%
\item[Returns:]~

	the\_period: A period object.
%
%
\item[See also:]%
\hyperlink{ref_period}{\texttt{period}}%
\ (p.~\pageref{ref_period})%
\index[funcref]{@\fidxl{period}}%
, \hyperlink{ref_cip_trace}{\texttt{cip\_trace}}%
\ (p.~\pageref{ref_cip_trace})%
\index[funcref]{@\fidxl{cip\_trace}}%
, \hyperlink{ref_trace}{\texttt{trace}}%
\ (p.~\pageref{ref_trace})%
\index[funcref]{@\fidxl{trace}}%
%
\item[Author:]%
Cengiz Gunay <cgunay@emory.edu>,Tom Sangrey 2006/01/26%
\end{description}
\methodline%
\subsubsection[Method \texttt{subsref}]{Method \texttt{cip\_trace/subsref}}%
\index[funcref]{cip_trace@\fidxlb{cip\_trace}!subsref@\fidxl{subsref}}%
\label{ref_cip_trace__subsref}%
\hypertarget{ref_cip_trace__subsref}{}%
\begin{description}
\item[Summary:]Defines generic indexing for objects.
%
%
%
%
%
%
%
\item[Author:]%
Cengiz Gunay <cgunay@emory.edu>, 2004/08/04%
\end{description}
\methodline%
\subsubsection[Method \texttt{calcPulsePotAvg}]{Method \texttt{cip\_trace/calcPulsePotAvg}}%
\index[funcref]{cip_trace@\fidxlb{cip\_trace}!calcPulsePotAvg@\fidxl{calcPulsePotAvg}}%
\label{ref_cip_trace__calcPulsePotAvg}%
\hypertarget{ref_cip_trace__calcPulsePotAvg}{}%
\begin{description}
\item[Summary:]Calculates the average potential value of the 
		CIP period of the cip\_trace, t. 
%
\item[Usage:]~%
\begin{lyxcode}%
avg\_val = calcPulsePotAvg(t)
%
\end{lyxcode}%
%
%
\item[Parameters:]~
\begin{description}%
\item[\texttt{t}:]
 A cip\_trace object.
\end{description}%
%
\item[Returns:]~

	avg\_val: The avg value [dy].
%
%
\item[See also:]%
\hyperlink{ref_period}{\texttt{period}}%
\ (p.~\pageref{ref_period})%
\index[funcref]{@\fidxl{period}}%
, \hyperlink{ref_trace}{\texttt{trace}}%
\ (p.~\pageref{ref_trace})%
\index[funcref]{@\fidxl{trace}}%
, \hyperlink{ref_trace__calcAvg}{\texttt{trace/calcAvg}}%
\ (p.~\pageref{ref_trace__calcAvg})%
\index[funcref]{trace@\fidxlb{trace}!calcAvg@\fidxl{calcAvg}}%
%
\item[Author:]%
Cengiz Gunay <cgunay@emory.edu>, 2004/08/25%
\end{description}
\methodline%
\subsubsection[Method \texttt{calcPulsePotSag}]{Method \texttt{cip\_trace/calcPulsePotSag}}%
\index[funcref]{cip_trace@\fidxlb{cip\_trace}!calcPulsePotSag@\fidxl{calcPulsePotSag}}%
\label{ref_cip_trace__calcPulsePotSag}%
\hypertarget{ref_cip_trace__calcPulsePotSag}{}%
\begin{description}
\item[Summary:]Calculates the minimal sag and sag amount of the CIP period of the cip\_trace, t. 
%
\item[Usage:]~%
\begin{lyxcode}%
[min\_val, min\_idx, sag\_val] = calcPulsePotSag(t)
%
\end{lyxcode}%
%
\item[Description:]%
The minimal sag is the minimal potential value of the 
 first half of the CIP period. The sag amount is calculated by 
 comparing this to the steady-state value at the end of the CIP period.
%%
\item[Parameters:]~
\begin{description}%
\item[\texttt{t}:]
 A cip\_trace object.
\end{description}%
%
\item[Returns:]~

	min\_val: The min value [dy].
	min\_idx: The index of the min value [dt].
	sag\_val: The sag amount [dy].
%
%
\item[See also:]%
\hyperlink{ref_period}{\texttt{period}}%
\ (p.~\pageref{ref_period})%
\index[funcref]{@\fidxl{period}}%
, \hyperlink{ref_trace}{\texttt{trace}}%
\ (p.~\pageref{ref_trace})%
\index[funcref]{@\fidxl{trace}}%
, \hyperlink{ref_trace__calcMin}{\texttt{trace/calcMin}}%
\ (p.~\pageref{ref_trace__calcMin})%
\index[funcref]{trace@\fidxlb{trace}!calcMin@\fidxl{calcMin}}%
%
\item[Author:]%
Cengiz Gunay <cgunay@emory.edu>, 2004/08/25%
\end{description}
\methodline%
\subsubsection[Method \texttt{periodPulseIni100ms}]{Method \texttt{cip\_trace/periodPulseIni100ms}}%
\index[funcref]{cip_trace@\fidxlb{cip\_trace}!periodPulseIni100ms@\fidxl{periodPulseIni100ms}}%
\label{ref_cip_trace__periodPulseIni100ms}%
\hypertarget{ref_cip_trace__periodPulseIni100ms}{}%
\begin{description}
\item[Summary:]Returns the first 100ms of the CIP period of 
			cip\_trace, t. 
%
\item[Usage:]~%
\begin{lyxcode}%
the\_period = periodPulseIni100ms(t)
%
\end{lyxcode}%
%
%
\item[Parameters:]~
\begin{description}%
\item[\texttt{t}:]
 A trace object.
\end{description}%
%
\item[Returns:]~

	the\_period: A period object.
%
%
\item[See also:]%
\hyperlink{ref_period}{\texttt{period}}%
\ (p.~\pageref{ref_period})%
\index[funcref]{@\fidxl{period}}%
, \hyperlink{ref_cip_trace}{\texttt{cip\_trace}}%
\ (p.~\pageref{ref_cip_trace})%
\index[funcref]{@\fidxl{cip\_trace}}%
, \hyperlink{ref_trace}{\texttt{trace}}%
\ (p.~\pageref{ref_trace})%
\index[funcref]{@\fidxl{trace}}%
%
\item[Author:]%
Cengiz Gunay <cgunay@emory.edu>, 2004/08/25%
\end{description}
\methodline%
\subsubsection[Method \texttt{periodRecSpont}]{Method \texttt{cip\_trace/periodRecSpont}}%
\index[funcref]{cip_trace@\fidxlb{cip\_trace}!periodRecSpont@\fidxl{periodRecSpont}}%
\label{ref_cip_trace__periodRecSpont}%
\hypertarget{ref_cip_trace__periodRecSpont}{}%
\begin{description}
\item[Summary:]Returns the recovery spontaneous activity period 
		of cip\_trace, t. 
%
\item[Usage:]~%
\begin{lyxcode}%
the\_period = periodRecSpont(t)
%
\end{lyxcode}%
%
%
\item[Parameters:]~
\begin{description}%
\item[\texttt{t}:]
 A trace object.
\end{description}%
%
\item[Returns:]~

	the\_period: A period object.
%
%
\item[See also:]%
\hyperlink{ref_period}{\texttt{period}}%
\ (p.~\pageref{ref_period})%
\index[funcref]{@\fidxl{period}}%
, \hyperlink{ref_cip_trace}{\texttt{cip\_trace}}%
\ (p.~\pageref{ref_cip_trace})%
\index[funcref]{@\fidxl{cip\_trace}}%
, \hyperlink{ref_trace}{\texttt{trace}}%
\ (p.~\pageref{ref_trace})%
\index[funcref]{@\fidxl{trace}}%
%
\item[Author:]%
Cengiz Gunay <cgunay@emory.edu>, 2004/08/25%
\end{description}
\methodline%
\subsubsection[Method \texttt{getResults}]{Method \texttt{cip\_trace/getResults}}%
\index[funcref]{cip_trace@\fidxlb{cip\_trace}!getResults@\fidxl{getResults}}%
\label{ref_cip_trace__getResults}%
\hypertarget{ref_cip_trace__getResults}{}%
\begin{description}
\item[Summary:]Calculate test results given a\_spikes object.
%
\item[Usage:]~%
\begin{lyxcode}%
results = getResults(a\_cip\_trace, a\_spikes)
%
\end{lyxcode}%
%
%
\item[Parameters:]~
\begin{description}%
\item[\texttt{a\_cip\_trace}:]
 A cip\_trace object.
\item[\texttt{a\_spikes}:]
 A spikes object.
\end{description}%
%
\item[Returns:]~

	results: A structure associating test names with result values.
%
%
\item[See also:]%
\hyperlink{ref_cip_trace}{\texttt{cip\_trace}}%
\ (p.~\pageref{ref_cip_trace})%
\index[funcref]{@\fidxl{cip\_trace}}%
, \hyperlink{ref_spikes}{\texttt{spikes}}%
\ (p.~\pageref{ref_spikes})%
\index[funcref]{@\fidxl{spikes}}%
, \hyperlink{ref_spike_shape}{\texttt{spike\_shape}}%
\ (p.~\pageref{ref_spike_shape})%
\index[funcref]{@\fidxl{spike\_shape}}%
%
\item[Author:]%
Cengiz Gunay <cgunay@emory.edu>, 2004/09/14%
\end{description}
\methodline%
\subsubsection[Method \texttt{getRecSpontSpike}]{Method \texttt{cip\_trace/getRecSpontSpike}}%
\index[funcref]{cip_trace@\fidxlb{cip\_trace}!getRecSpontSpike@\fidxl{getRecSpontSpike}}%
\label{ref_cip_trace__getRecSpontSpike}%
\hypertarget{ref_cip_trace__getRecSpontSpike}{}%
\begin{description}
\item[Summary:]Convert a spike in the CIP period to a spike\_shape object.
%
\item[Usage:]~%
\begin{lyxcode}%
obj = getRecSpontSpike(trace, spikes, spike\_num, props)
%
\end{lyxcode}%
%
\item[Description:]%
Creates a spike\_shape object from desired spike.
%%
\item[Parameters:]~
\begin{description}%
\item[\texttt{trace}:]
 A trace object.
\item[\texttt{spikes}:]
 A spikes object on trace.
\item[\texttt{spike\_num}:]
 The index of spike to extract.
\end{description}%
%
%
%
\item[See also:]%
\hyperlink{ref_spike_shape}{\texttt{spike\_shape}}%
\ (p.~\pageref{ref_spike_shape})%
\index[funcref]{@\fidxl{spike\_shape}}%
%
\item[Author:]%
Cengiz Gunay <cgunay@emory.edu>, 2005/05/08%
\end{description}
\methodline%
\subsection{Class \texttt{cip\_trace\_allspikes\_profile}}%
\index[funcref]{cip_trace_allspikes_profile@\fidxlb{cip\_trace\_allspikes\_profile}}%
\label{ref_cip_trace_allspikes_profile}%
\hypertarget{ref_cip_trace_allspikes_profile}{}%
\subsubsection[Constructor \texttt{cip\_trace\_allspikes\_profile}]{Constructor \texttt{cip\_trace\_allspikes\_profile/cip\_trace\_allspikes\_profile}}%
\index[funcref]{cip_trace_allspikes_profile@\fidxlb{cip\_trace\_allspikes\_profile}!cip_trace_allspikes_profile@\fidxl{cip\_trace\_allspikes\_profile}}%
\label{ref_cip_trace_allspikes_profile__cip_trace_allspikes_profile}%
\hypertarget{ref_cip_trace_allspikes_profile__cip_trace_allspikes_profile}{}%
\begin{description}
\item[Summary:]Creates and collects test results of a cip\_trace.
%
\item[Usage:]~%
\begin{lyxcode}%
obj = 
   cip\_trace\_allspikes\_profile(a\_cip\_trace, a\_spikes, a\_spont\_spike\_shape, 
				results, id, props)
%
\end{lyxcode}%
%
\item[Description:]%
This is a subclass of results\_profile. It is made to be used from 
 subclass constructors.
%%
\item[Parameters:]~
\begin{description}%
\item[\texttt{a\_cip\_trace}:]
 A cip\_trace object.
\item[\texttt{a\_spikes}:]
 A spikes object.
\item[\texttt{spont\_spikes\_db, pulse\_spikes\_db, recov\_spikes\_db}:]
 

tests\_dbs with spontaneous, pulse and recovery period spike info.\item[\texttt{results\_obj}:]
 A results\_profile object with test results.
\item[\texttt{id}:]
 Identification string.
\item[\texttt{props}:]
 A structure with any optional properties.
\end{description}%
%
\item[Returns a structure object with the following fields:]~

	trace, spikes, spont\_spikes\_db, 
	pulse\_spikes\_db, recov\_spikes\_db, props
%
%
\item[See also:]%
\hyperlink{ref_cip_trace}{\texttt{cip\_trace}}%
\ (p.~\pageref{ref_cip_trace})%
\index[funcref]{@\fidxl{cip\_trace}}%
, \hyperlink{ref_spikes}{\texttt{spikes}}%
\ (p.~\pageref{ref_spikes})%
\index[funcref]{@\fidxl{spikes}}%
, \hyperlink{ref_tests_db}{\texttt{tests\_db}}%
\ (p.~\pageref{ref_tests_db})%
\index[funcref]{@\fidxl{tests\_db}}%
%
\item[Author:]%
Cengiz Gunay <cgunay@emory.edu>, 2005/05/04%
\end{description}
\methodline%
\subsubsection[Method \texttt{display}]{Method \texttt{cip\_trace\_allspikes\_profile/display}}%
\index[funcref]{cip_trace_allspikes_profile@\fidxlb{cip\_trace\_allspikes\_profile}!display@\fidxl{display}}%
\label{ref_cip_trace_allspikes_profile__display}%
\hypertarget{ref_cip_trace_allspikes_profile__display}{}%
\begin{description}
%
%
%
%
%
%
%
\item[Author:]%
Cengiz Gunay <cgunay@emory.edu>, 2004/08/04%
\end{description}
\methodline%
\subsubsection[Method \texttt{get}]{Method \texttt{cip\_trace\_allspikes\_profile/get}}%
\index[funcref]{cip_trace_allspikes_profile@\fidxlb{cip\_trace\_allspikes\_profile}!get@\fidxl{get}}%
\label{ref_cip_trace_allspikes_profile__get}%
\hypertarget{ref_cip_trace_allspikes_profile__get}{}%
\begin{description}
\item[Summary:]Defines generic attribute retrieval for objects.
%
%
%
%
%
%
%
\item[Author:]%
Cengiz Gunay <cgunay@emory.edu>, 2004/09/14%
\end{description}
\methodline%
\subsubsection[Method \texttt{set}]{Method \texttt{cip\_trace\_allspikes\_profile/set}}%
\index[funcref]{cip_trace_allspikes_profile@\fidxlb{cip\_trace\_allspikes\_profile}!set@\fidxl{set}}%
\label{ref_cip_trace_allspikes_profile__set}%
\hypertarget{ref_cip_trace_allspikes_profile__set}{}%
\begin{description}
\item[Summary:]Generic method for setting object attributes.
%
%
%
%
%
%
%
\item[Author:]%
Cengiz Gunay <cgunay@emory.edu>, 2004/10/08%
\end{description}
\methodline%
\subsubsection[Method \texttt{plotRowSpontSpikeAnal}]{Method \texttt{cip\_trace\_allspikes\_profile/plotRowSpontSpikeAnal}}%
\index[funcref]{cip_trace_allspikes_profile@\fidxlb{cip\_trace\_allspikes\_profile}!plotRowSpontSpikeAnal@\fidxl{plotRowSpontSpikeAnal}}%
\label{ref_cip_trace_allspikes_profile__plotRowSpontSpikeAnal}%
\hypertarget{ref_cip_trace_allspikes_profile__plotRowSpontSpikeAnal}{}%
\begin{description}
\item[Summary:]Creates a row of plots that show spontaneous spikes, starting from the whole trace, zooming into the individual spike.
%
\item[Usage:]~%
\begin{lyxcode}%
a\_plot = plotRowSpontSpikeAnal(prof, title\_str)
%
\end{lyxcode}%
%
%
\item[Parameters:]~
\begin{description}%
\item[\texttt{prof}:]
 A cip\_trace\_allspikes\_profile object.
\item[\texttt{title\_str}:]
 (Optional) String to append to plot title.
\end{description}%
%
\item[Returns:]~

	a\_plot: A plot\_abstract object that can be visualized.
%
%
\item[See also:]%
\hyperlink{ref_trace}{\texttt{trace}}%
\ (p.~\pageref{ref_trace})%
\index[funcref]{@\fidxl{trace}}%
, \hyperlink{ref_cip_trace}{\texttt{cip\_trace}}%
\ (p.~\pageref{ref_cip_trace})%
\index[funcref]{@\fidxl{cip\_trace}}%
, \hyperlink{ref_spike_shape__plotCompareMethodsSimple}{\texttt{spike\_shape/plotCompareMethodsSimple}}%
\ (p.~\pageref{ref_spike_shape__plotCompareMethodsSimple})%
\index[funcref]{spike_shape@\fidxlb{spike\_shape}!plotCompareMethodsSimple@\fidxl{plotCompareMethodsSimple}}%
, \hyperlink{ref_plot_abstract}{\texttt{plot\_abstract}}%
\ (p.~\pageref{ref_plot_abstract})%
\index[funcref]{@\fidxl{plot\_abstract}}%
%
\item[Author:]%
Cengiz Gunay <cgunay@emory.edu>, 2005/05/23%
\end{description}
\methodline%
\subsection{Class \texttt{cip\_trace\_profile}}%
\index[funcref]{cip_trace_profile@\fidxlb{cip\_trace\_profile}}%
\label{ref_cip_trace_profile}%
\hypertarget{ref_cip_trace_profile}{}%
\subsubsection[Constructor \texttt{cip\_trace\_profile}]{Constructor \texttt{cip\_trace\_profile/cip\_trace\_profile}}%
\index[funcref]{cip_trace_profile@\fidxlb{cip\_trace\_profile}!cip_trace_profile@\fidxl{cip\_trace\_profile}}%
\label{ref_cip_trace_profile__cip_trace_profile}%
\hypertarget{ref_cip_trace_profile__cip_trace_profile}{}%
\begin{description}
\item[Summary:]Creates and collects test results of a cip\_trace.
%
%
\item[Description:]%
The first usage is fully customizable to be used from subclass constructors.
 The second usage generates the spikes and spont\_spike\_shape objects, and
 collects some generic test results from them. 
%%
\item[Parameters:]~
\begin{description}%
\item[\texttt{data\_src}:]
 The trace column OR the filename.
\item[\texttt{dt}:]
 Time resolution [s]
\item[\texttt{dy}:]
 y-axis resolution [ISI (V, A, etc.)]
\item[\texttt{pulse\_time\_start, pulse\_time\_width}:]


Start and width of the pulse [dt]\item[\texttt{id}:]
 Identification string.
\item[\texttt{props}:]
 See trace object.
\end{description}%
%
\item[Returns a structure object with the following fields:]~

	trace, spikes, spont\_spike\_shape, results, id, props.
%
%
\item[See also:]%
\hyperlink{ref_cip_trace}{\texttt{cip\_trace}}%
\ (p.~\pageref{ref_cip_trace})%
\index[funcref]{@\fidxl{cip\_trace}}%
, \hyperlink{ref_spikes}{\texttt{spikes}}%
\ (p.~\pageref{ref_spikes})%
\index[funcref]{@\fidxl{spikes}}%
, \hyperlink{ref_spike_shape}{\texttt{spike\_shape}}%
\ (p.~\pageref{ref_spike_shape})%
\index[funcref]{@\fidxl{spike\_shape}}%
%
\item[Author:]%
Cengiz Gunay <cgunay@emory.edu>, 2004/08/25%
\end{description}
\methodline%
\subsubsection[Method \texttt{display}]{Method \texttt{cip\_trace\_profile/display}}%
\index[funcref]{cip_trace_profile@\fidxlb{cip\_trace\_profile}!display@\fidxl{display}}%
\label{ref_cip_trace_profile__display}%
\hypertarget{ref_cip_trace_profile__display}{}%
\begin{description}
%
%
%
%
%
%
%
\item[Author:]%
Cengiz Gunay <cgunay@emory.edu>, 2004/08/04%
\end{description}
\methodline%
\subsubsection[Method \texttt{get}]{Method \texttt{cip\_trace\_profile/get}}%
\index[funcref]{cip_trace_profile@\fidxlb{cip\_trace\_profile}!get@\fidxl{get}}%
\label{ref_cip_trace_profile__get}%
\hypertarget{ref_cip_trace_profile__get}{}%
\begin{description}
\item[Summary:]Defines generic attribute retrieval for objects.
%
%
%
%
%
%
%
\item[Author:]%
Cengiz Gunay <cgunay@emory.edu>, 2004/09/14%
\end{description}
\methodline%
\subsubsection[Method \texttt{set}]{Method \texttt{cip\_trace\_profile/set}}%
\index[funcref]{cip_trace_profile@\fidxlb{cip\_trace\_profile}!set@\fidxl{set}}%
\label{ref_cip_trace_profile__set}%
\hypertarget{ref_cip_trace_profile__set}{}%
\begin{description}
\item[Summary:]Generic method for setting object attributes.
%
%
%
%
%
%
%
\item[Author:]%
Cengiz Gunay <cgunay@emory.edu>, 2004/10/08%
\end{description}
\methodline%
\subsubsection[Method \texttt{subsref}]{Method \texttt{cip\_trace\_profile/subsref}}%
\index[funcref]{cip_trace_profile@\fidxlb{cip\_trace\_profile}!subsref@\fidxl{subsref}}%
\label{ref_cip_trace_profile__subsref}%
\hypertarget{ref_cip_trace_profile__subsref}{}%
\begin{description}
\item[Summary:]Defines generic indexing for objects.
%
%
%
%
%
%
%
\item[Author:]%
Cengiz Gunay <cgunay@emory.edu>, 2004/08/04%
\end{description}
\methodline%
\subsubsection[Method \texttt{plot}]{Method \texttt{cip\_trace\_profile/plot}}%
\index[funcref]{cip_trace_profile@\fidxlb{cip\_trace\_profile}!plot@\fidxl{plot}}%
\label{ref_cip_trace_profile__plot}%
\hypertarget{ref_cip_trace_profile__plot}{}%
\begin{description}
\item[Summary:]Plots a cip\_trace\_profile object.
%
\item[Usage:]~%
\begin{lyxcode}%
h = plot(t)
%
\end{lyxcode}%
%
\item[Description:]%
Plots contents of this object.
%%
\item[Parameters:]~
\begin{description}%
\item[\texttt{t}:]
 A cip\_trace\_profile object.
\end{description}%
%
\item[Returns:]~

	h: Plot handle(s).
%
%
%
\item[Author:]%
Cengiz Gunay <cgunay@emory.edu>, 2004/09/15%
\end{description}
\methodline%
\subsection{Class \texttt{cip\_traces\_dataset}}%
\index[funcref]{cip_traces_dataset@\fidxlb{cip\_traces\_dataset}}%
\label{ref_cip_traces_dataset}%
\hypertarget{ref_cip_traces_dataset}{}%
\subsubsection[Constructor \texttt{cip\_traces\_dataset}]{Constructor \texttt{cip\_traces\_dataset/cip\_traces\_dataset}}%
\index[funcref]{cip_traces_dataset@\fidxlb{cip\_traces\_dataset}!cip_traces_dataset@\fidxl{cip\_traces\_dataset}}%
\label{ref_cip_traces_dataset__cip_traces_dataset}%
\hypertarget{ref_cip_traces_dataset__cip_traces_dataset}{}%
\begin{description}
\item[Summary:]Dataset of cip\_traces objects, each with varying cip magnitudes.
%
\item[Usage:]~%
\begin{lyxcode}%
obj = cip\_traces\_dataset(ts, cipmag, id, props)
%
\end{lyxcode}%
%
\item[Description:]%
This is a subclass of params\_tests\_fileset.
%%
\item[Parameters:]~
\begin{description}%
\item[\texttt{ts}:]
 A cell array of cip\_traces objects.
\item[\texttt{cipmag}:]
 A single cip magnitude to trace take from objects.
\item[\texttt{id}:]
 An identification string for the whole dataset.
\item[\texttt{props}:]
 A structure with any optional properties passed to cip\_trace\_profile.
\end{description}%
%
\item[Returns a structure object with the following fields:]~

	params\_tests\_dataset,
	cipmag, props (see above).
%
%
\item[See also:]%
\hyperlink{ref_cip_traces}{\texttt{cip\_traces}}%
\ (p.~\pageref{ref_cip_traces})%
\index[funcref]{@\fidxl{cip\_traces}}%
, \hyperlink{ref_params_tests_fileset}{\texttt{params\_tests\_fileset}}%
\ (p.~\pageref{ref_params_tests_fileset})%
\index[funcref]{@\fidxl{params\_tests\_fileset}}%
, \hyperlink{ref_params_tests_db}{\texttt{params\_tests\_db}}%
\ (p.~\pageref{ref_params_tests_db})%
\index[funcref]{@\fidxl{params\_tests\_db}}%
%
\item[Author:]%
Cengiz Gunay <cgunay@emory.edu>, 2004/11/30%
\end{description}
\methodline%
\subsubsection[Method \texttt{display}]{Method \texttt{cip\_traces\_dataset/display}}%
\index[funcref]{cip_traces_dataset@\fidxlb{cip\_traces\_dataset}!display@\fidxl{display}}%
\label{ref_cip_traces_dataset__display}%
\hypertarget{ref_cip_traces_dataset__display}{}%
\begin{description}
%
%
%
%
%
%
%
\item[Author:]%
Cengiz Gunay <cgunay@emory.edu>, 2004/08/04%
\end{description}
\methodline%
\subsubsection[Method \texttt{get}]{Method \texttt{cip\_traces\_dataset/get}}%
\index[funcref]{cip_traces_dataset@\fidxlb{cip\_traces\_dataset}!get@\fidxl{get}}%
\label{ref_cip_traces_dataset__get}%
\hypertarget{ref_cip_traces_dataset__get}{}%
\begin{description}
\item[Summary:]Defines generic attribute retrieval for objects.
%
%
%
%
%
%
%
\item[Author:]%
Cengiz Gunay <cgunay@emory.edu>, 2004/09/14%
\end{description}
\methodline%
\subsubsection[Method \texttt{set}]{Method \texttt{cip\_traces\_dataset/set}}%
\index[funcref]{cip_traces_dataset@\fidxlb{cip\_traces\_dataset}!set@\fidxl{set}}%
\label{ref_cip_traces_dataset__set}%
\hypertarget{ref_cip_traces_dataset__set}{}%
\begin{description}
\item[Summary:]Generic method for setting object attributes.
%
%
%
%
%
%
%
\item[Author:]%
Cengiz Gunay <cgunay@emory.edu>, 2004/10/08%
\end{description}
\methodline%
\subsubsection[Method \texttt{cip\_trace\_profile}]{Method \texttt{cip\_traces\_dataset/cip\_trace\_profile}}%
\index[funcref]{cip_traces_dataset@\fidxlb{cip\_traces\_dataset}!cip_trace_profile@\fidxl{cip\_trace\_profile}}%
\label{ref_cip_traces_dataset__cip_trace_profile}%
\hypertarget{ref_cip_traces_dataset__cip_trace_profile}{}%
\begin{description}
\item[Summary:]Loads a raw cip\_trace\_profile given a index 
		      to this dataset.
%
\item[Usage:]~%
\begin{lyxcode}%
a\_cip\_trace\_profile = cip\_trace\_profile(dataset, index)
%
\end{lyxcode}%
%
%
\item[Parameters:]~
\begin{description}%
\item[\texttt{dataset}:]
 A params\_tests\_dataset.
\item[\texttt{index}:]
 Index of file in dataset.
\end{description}%
%
\item[Returns:]~

	a\_cip\_trace\_profile: A cip\_trace\_profile object.
%
%
\item[See also:]%
\hyperlink{ref_cip_trace_profile}{\texttt{cip\_trace\_profile}}%
\ (p.~\pageref{ref_cip_trace_profile})%
\index[funcref]{@\fidxl{cip\_trace\_profile}}%
, \hyperlink{ref_params_tests_dataset}{\texttt{params\_tests\_dataset}}%
\ (p.~\pageref{ref_params_tests_dataset})%
\index[funcref]{@\fidxl{params\_tests\_dataset}}%
%
\item[Author:]%
Cengiz Gunay <cgunay@emory.edu>, 2004/09/14%
\end{description}
\methodline%
\subsubsection[Method \texttt{subsref}]{Method \texttt{cip\_traces\_dataset/subsref}}%
\index[funcref]{cip_traces_dataset@\fidxlb{cip\_traces\_dataset}!subsref@\fidxl{subsref}}%
\label{ref_cip_traces_dataset__subsref}%
\hypertarget{ref_cip_traces_dataset__subsref}{}%
\begin{description}
\item[Summary:]Defines generic indexing for objects.
%
%
%
%
%
%
%
\item[Author:]%
Cengiz Gunay <cgunay@emory.edu>, 2004/08/04%
\end{description}
\methodline%
\subsubsection[Method \texttt{paramNames}]{Method \texttt{cip\_traces\_dataset/paramNames}}%
\index[funcref]{cip_traces_dataset@\fidxlb{cip\_traces\_dataset}!paramNames@\fidxl{paramNames}}%
\label{ref_cip_traces_dataset__paramNames}%
\hypertarget{ref_cip_traces_dataset__paramNames}{}%
\begin{description}
\item[Summary:]Returns the only parameter, 'pAcip,' for this fileset.
%
\item[Usage:]~%
\begin{lyxcode}%
param\_names = paramNames(fileset)
%
\end{lyxcode}%
%
\item[Description:]%
Looks at the filename of the first file to find the parameter names.
%%
\item[Parameters:]~
\begin{description}%
\item[\texttt{fileset}:]
 A params\_tests\_fileset.
\end{description}%
%
\item[Returns:]~

	params\_names: Cell array with ordered parameter names.
%
%
\item[See also:]%
\hyperlink{ref_params_tests_fileset}{\texttt{params\_tests\_fileset}}%
\ (p.~\pageref{ref_params_tests_fileset})%
\index[funcref]{@\fidxl{params\_tests\_fileset}}%
, \hyperlink{ref_paramNames}{\texttt{paramNames}}%
\ (p.~\pageref{ref_paramNames})%
\index[funcref]{@\fidxl{paramNames}}%
, \hyperlink{ref_testNames}{\texttt{testNames}}%
\ (p.~\pageref{ref_testNames})%
\index[funcref]{@\fidxl{testNames}}%
%
\item[Author:]%
Cengiz Gunay <cgunay@emory.edu>, 2004/12/06%
\end{description}
\methodline%
\subsubsection[Method \texttt{getItemParams}]{Method \texttt{cip\_traces\_dataset/getItemParams}}%
\index[funcref]{cip_traces_dataset@\fidxlb{cip\_traces\_dataset}!getItemParams@\fidxl{getItemParams}}%
\label{ref_cip_traces_dataset__getItemParams}%
\hypertarget{ref_cip_traces_dataset__getItemParams}{}%
\begin{description}
%
\item[Usage:]~%
\begin{lyxcode}%
params\_row = getParams(dataset, index)
%
\end{lyxcode}%
%
%
\item[Parameters:]~
\begin{description}%
\item[\texttt{dataset}:]
 A params\_tests\_dataset.
\item[\texttt{index}:]
 Index of item in dataset.
\end{description}%
%
\item[Returns:]~

	params\_row: Parameter values in the same order of paramNames
%
%
\item[See also:]%
\hyperlink{ref_itemResultsRow}{\texttt{itemResultsRow}}%
\ (p.~\pageref{ref_itemResultsRow})%
\index[funcref]{@\fidxl{itemResultsRow}}%
, \hyperlink{ref_params_tests_dataset}{\texttt{params\_tests\_dataset}}%
\ (p.~\pageref{ref_params_tests_dataset})%
\index[funcref]{@\fidxl{params\_tests\_dataset}}%
, \hyperlink{ref_paramNames}{\texttt{paramNames}}%
\ (p.~\pageref{ref_paramNames})%
\index[funcref]{@\fidxl{paramNames}}%
, \hyperlink{ref_testNames}{\texttt{testNames}}%
\ (p.~\pageref{ref_testNames})%
\index[funcref]{@\fidxl{testNames}}%
%
\item[Author:]%
Cengiz Gunay <cgunay@emory.edu>, 2004/12/06%
\end{description}
\methodline%
\subsubsection[Method \texttt{loadItemProfile}]{Method \texttt{cip\_traces\_dataset/loadItemProfile}}%
\index[funcref]{cip_traces_dataset@\fidxlb{cip\_traces\_dataset}!loadItemProfile@\fidxl{loadItemProfile}}%
\label{ref_cip_traces_dataset__loadItemProfile}%
\hypertarget{ref_cip_traces_dataset__loadItemProfile}{}%
\begin{description}
\item[Summary:]Loads a profile object from a raw data item in the dataset.
%
\item[Usage:]~%
\begin{lyxcode}%
a\_profile = loadItemProfile(dataset, index)
%
\end{lyxcode}%
%
\item[Description:]%
Subclasses should overload this function to load the specific profile
 object they desire. The profile class should define a getResults method
 which is used in the itemResultsRow method.
%%
\item[Parameters:]~
\begin{description}%
\item[\texttt{dataset}:]
 A params\_tests\_dataset.
\item[\texttt{index}:]
 Index of item in dataset.
\end{description}%
%
\item[Returns:]~

	a\_profile: A profile object that implements the getResults method.
%
%
\item[See also:]%
\hyperlink{ref_itemResultsRow}{\texttt{itemResultsRow}}%
\ (p.~\pageref{ref_itemResultsRow})%
\index[funcref]{@\fidxl{itemResultsRow}}%
, \hyperlink{ref_params_tests_dataset}{\texttt{params\_tests\_dataset}}%
\ (p.~\pageref{ref_params_tests_dataset})%
\index[funcref]{@\fidxl{params\_tests\_dataset}}%
, \hyperlink{ref_paramNames}{\texttt{paramNames}}%
\ (p.~\pageref{ref_paramNames})%
\index[funcref]{@\fidxl{paramNames}}%
, \hyperlink{ref_testNames}{\texttt{testNames}}%
\ (p.~\pageref{ref_testNames})%
\index[funcref]{@\fidxl{testNames}}%
%
\item[Author:]%
Cengiz Gunay <cgunay@emory.edu>, 2004/09/14%
\end{description}
\methodline%
\subsection{Class \texttt{cip\_traceset}}%
\index[funcref]{cip_traceset@\fidxlb{cip\_traceset}}%
\label{ref_cip_traceset}%
\hypertarget{ref_cip_traceset}{}%
\subsubsection[Constructor \texttt{cip\_traceset}]{Constructor \texttt{cip\_traceset/cip\_traceset}}%
\index[funcref]{cip_traceset@\fidxlb{cip\_traceset}!cip_traceset@\fidxl{cip\_traceset}}%
\label{ref_cip_traceset__cip_traceset}%
\hypertarget{ref_cip_traceset__cip_traceset}{}%
\begin{description}
\item[Summary:]A traceset with varying cip magnitudes from a single cip\_traces object.
%
\item[Usage:]~%
\begin{lyxcode}%
obj = cip\_traceset(ct, cip\_mags, dy, props)
%
\end{lyxcode}%
%
\item[Description:]%
This is a subclass of params\_tests\_fileset. This traceset can create a 
 mini-database form a single cip\_traces object. The list contains cip\_mags.
 cip\_traceset\_dataset should be used to load multiple cip\_traceset objects.
%%
\item[Parameters:]~
\begin{description}%
\item[\texttt{ct}:]
 A cip\_traces object.
\item[\texttt{cip\_mags}:]
 An array of cip magnitudes to select from the object.
\item[\texttt{dy}:]
 y-axis resolution, [V] or [A] (default=1e-3).
\item[\texttt{props}:]
 A structure with any optional properties.
\begin{description}%
\item[\texttt{offsetPotential}:]
 Add this to physiology trace as compensation.
\end{description}%
\end{description}%
%
\item[Returns a structure object with the following fields:]~

	params\_tests\_dataset,
	ct, props (see above).
%
%
\item[See also:]%
\hyperlink{ref_cip_traces}{\texttt{cip\_traces}}%
\ (p.~\pageref{ref_cip_traces})%
\index[funcref]{@\fidxl{cip\_traces}}%
, \hyperlink{ref_params_tests_fileset}{\texttt{params\_tests\_fileset}}%
\ (p.~\pageref{ref_params_tests_fileset})%
\index[funcref]{@\fidxl{params\_tests\_fileset}}%
, \hyperlink{ref_params_tests_db}{\texttt{params\_tests\_db}}%
\ (p.~\pageref{ref_params_tests_db})%
\index[funcref]{@\fidxl{params\_tests\_db}}%
%
\item[Author:]%
Cengiz Gunay <cgunay@emory.edu>, 2004/11/30%
\end{description}
\methodline%
\subsubsection[Method \texttt{display}]{Method \texttt{cip\_traceset/display}}%
\index[funcref]{cip_traceset@\fidxlb{cip\_traceset}!display@\fidxl{display}}%
\label{ref_cip_traceset__display}%
\hypertarget{ref_cip_traceset__display}{}%
\begin{description}
%
%
%
%
%
%
%
\item[Author:]%
Cengiz Gunay <cgunay@emory.edu>, 2004/08/04%
\end{description}
\methodline%
\subsubsection[Method \texttt{get}]{Method \texttt{cip\_traceset/get}}%
\index[funcref]{cip_traceset@\fidxlb{cip\_traceset}!get@\fidxl{get}}%
\label{ref_cip_traceset__get}%
\hypertarget{ref_cip_traceset__get}{}%
\begin{description}
\item[Summary:]Defines generic attribute retrieval for objects.
%
%
%
%
%
%
%
\item[Author:]%
Cengiz Gunay <cgunay@emory.edu>, 2004/09/14%
\end{description}
\methodline%
\subsubsection[Method \texttt{cip\_trace\_profile}]{Method \texttt{cip\_traceset/cip\_trace\_profile}}%
\index[funcref]{cip_traceset@\fidxlb{cip\_traceset}!cip_trace_profile@\fidxl{cip\_trace\_profile}}%
\label{ref_cip_traceset__cip_trace_profile}%
\hypertarget{ref_cip_traceset__cip_trace_profile}{}%
\begin{description}
\item[Summary:]Loads a raw cip\_trace\_profile given an index in this traceset.
%
\item[Usage:]~%
\begin{lyxcode}%
a\_cip\_trace\_profile = cip\_trace\_profile(traceset, index)
%
\end{lyxcode}%
%
%
\item[Parameters:]~
\begin{description}%
\item[\texttt{traceset}:]
 A cip\_traceset.
\item[\texttt{index}:]
 Index of item in traceset.
\end{description}%
%
\item[Returns:]~

	a\_cip\_trace\_profile: A cip\_trace\_profile object.
%
%
\item[See also:]%
\hyperlink{ref_cip_trace_profile}{\texttt{cip\_trace\_profile}}%
\ (p.~\pageref{ref_cip_trace_profile})%
\index[funcref]{@\fidxl{cip\_trace\_profile}}%
, \hyperlink{ref_params_tests_dataset}{\texttt{params\_tests\_dataset}}%
\ (p.~\pageref{ref_params_tests_dataset})%
\index[funcref]{@\fidxl{params\_tests\_dataset}}%
%
\item[Author:]%
Cengiz Gunay <cgunay@emory.edu>, 2004/09/14%
\end{description}
\methodline%
\subsubsection[Method \texttt{paramNames}]{Method \texttt{cip\_traceset/paramNames}}%
\index[funcref]{cip_traceset@\fidxlb{cip\_traceset}!paramNames@\fidxl{paramNames}}%
\label{ref_cip_traceset__paramNames}%
\hypertarget{ref_cip_traceset__paramNames}{}%
\begin{description}
\item[Summary:]Returns the only parameter, 'pAcip,' for this traceset.
%
\item[Usage:]~%
\begin{lyxcode}%
param\_names = paramNames(traceset)
%
\end{lyxcode}%
%
\item[Description:]%
Looks at the filename of the first file to find the parameter names.
%%
\item[Parameters:]~
\begin{description}%
\item[\texttt{traceset}:]
 A cip\_traceset.
\end{description}%
%
\item[Returns:]~

	params\_names: Cell array with ordered parameter names.
%
%
\item[See also:]%
\hyperlink{ref_params_tests_dataset}{\texttt{params\_tests\_dataset}}%
\ (p.~\pageref{ref_params_tests_dataset})%
\index[funcref]{@\fidxl{params\_tests\_dataset}}%
, \hyperlink{ref_paramNames}{\texttt{paramNames}}%
\ (p.~\pageref{ref_paramNames})%
\index[funcref]{@\fidxl{paramNames}}%
, \hyperlink{ref_testNames}{\texttt{testNames}}%
\ (p.~\pageref{ref_testNames})%
\index[funcref]{@\fidxl{testNames}}%
%
\item[Author:]%
Cengiz Gunay <cgunay@emory.edu>, 2004/12/06%
\end{description}
\methodline%
\subsubsection[Method \texttt{getItemParams}]{Method \texttt{cip\_traceset/getItemParams}}%
\index[funcref]{cip_traceset@\fidxlb{cip\_traceset}!getItemParams@\fidxl{getItemParams}}%
\label{ref_cip_traceset__getItemParams}%
\hypertarget{ref_cip_traceset__getItemParams}{}%
\begin{description}
%
\item[Usage:]~%
\begin{lyxcode}%
params\_row = getParams(dataset, index)
%
\end{lyxcode}%
%
%
\item[Parameters:]~
\begin{description}%
\item[\texttt{dataset}:]
 A params\_tests\_dataset.
\item[\texttt{index}:]
 Index of item in dataset.
\item[\texttt{a\_profile}:]
 A profile object for the item (optional).
\end{description}%
%
\item[Returns:]~

	params\_row: Parameter values in the same order of paramNames
%
%
\item[See also:]%
\hyperlink{ref_itemResultsRow}{\texttt{itemResultsRow}}%
\ (p.~\pageref{ref_itemResultsRow})%
\index[funcref]{@\fidxl{itemResultsRow}}%
, \hyperlink{ref_params_tests_dataset}{\texttt{params\_tests\_dataset}}%
\ (p.~\pageref{ref_params_tests_dataset})%
\index[funcref]{@\fidxl{params\_tests\_dataset}}%
, \hyperlink{ref_paramNames}{\texttt{paramNames}}%
\ (p.~\pageref{ref_paramNames})%
\index[funcref]{@\fidxl{paramNames}}%
, \hyperlink{ref_testNames}{\texttt{testNames}}%
\ (p.~\pageref{ref_testNames})%
\index[funcref]{@\fidxl{testNames}}%
%
\item[Author:]%
Cengiz Gunay <cgunay@emory.edu>, 2004/12/06%
\end{description}
\methodline%
\subsubsection[Method \texttt{loadItemProfile}]{Method \texttt{cip\_traceset/loadItemProfile}}%
\index[funcref]{cip_traceset@\fidxlb{cip\_traceset}!loadItemProfile@\fidxl{loadItemProfile}}%
\label{ref_cip_traceset__loadItemProfile}%
\hypertarget{ref_cip_traceset__loadItemProfile}{}%
\begin{description}
\item[Summary:]Loads a profile object from a raw data item in the dataset.
%
\item[Usage:]~%
\begin{lyxcode}%
a\_profile = loadItemProfile(dataset, index)
%
\end{lyxcode}%
%
\item[Description:]%
Subclasses should overload this function to load the specific profile
 object they desire. The profile class should define a getResults method
 which is used in the itemResultsRow method.
%%
\item[Parameters:]~
\begin{description}%
\item[\texttt{dataset}:]
 A params\_tests\_dataset.
\item[\texttt{index}:]
 Index of item in dataset.
\end{description}%
%
\item[Returns:]~

	a\_profile: A profile object that implements the getResults method.
%
%
\item[See also:]%
\hyperlink{ref_itemResultsRow}{\texttt{itemResultsRow}}%
\ (p.~\pageref{ref_itemResultsRow})%
\index[funcref]{@\fidxl{itemResultsRow}}%
, \hyperlink{ref_params_tests_dataset}{\texttt{params\_tests\_dataset}}%
\ (p.~\pageref{ref_params_tests_dataset})%
\index[funcref]{@\fidxl{params\_tests\_dataset}}%
, \hyperlink{ref_paramNames}{\texttt{paramNames}}%
\ (p.~\pageref{ref_paramNames})%
\index[funcref]{@\fidxl{paramNames}}%
, \hyperlink{ref_testNames}{\texttt{testNames}}%
\ (p.~\pageref{ref_testNames})%
\index[funcref]{@\fidxl{testNames}}%
%
\item[Author:]%
Cengiz Gunay <cgunay@emory.edu>, 2004/09/14%
\end{description}
\methodline%
\subsection{Class \texttt{cip\_traceset\_dataset}}%
\index[funcref]{cip_traceset_dataset@\fidxlb{cip\_traceset\_dataset}}%
\label{ref_cip_traceset_dataset}%
\hypertarget{ref_cip_traceset_dataset}{}%
\subsubsection[Constructor \texttt{cip\_traceset\_dataset}]{Constructor \texttt{cip\_traceset\_dataset/cip\_traceset\_dataset}}%
\index[funcref]{cip_traceset_dataset@\fidxlb{cip\_traceset\_dataset}!cip_traceset_dataset@\fidxl{cip\_traceset\_dataset}}%
\label{ref_cip_traceset_dataset__cip_traceset_dataset}%
\hypertarget{ref_cip_traceset_dataset__cip_traceset_dataset}{}%
\begin{description}
\item[Summary:]Dataset of multiple cip magnitudes from cip\_traces objects .
%
\item[Usage:]~%
\begin{lyxcode}%
obj = cip\_traceset\_dataset(cts, cip\_mags, dy, id, props)
%
\end{lyxcode}%
%
\item[Description:]%
This is a subclass of params\_tests\_dataset. Designed to extract a trace
 for each cip magnitude from the cip\_traceset objects contained. Uses cip\_traceset
 objects to extract multiple traces from each cip\_traces object.
%%
\item[Parameters:]~
\begin{description}%
\item[\texttt{cts}:]
 Array or cell array of cip\_traces objects.
\item[\texttt{cip\_mags}:]
 An array of cip magnitudes to select from each cip\_traces object.
\item[\texttt{dy}:]
 y-axis resolution, [V] or [A] (default = 1e-3).
\item[\texttt{id}:]
 An identification string.
\item[\texttt{props}:]
 A structure with any optional properties passed to cip\_traceset.
\end{description}%
%
\item[Returns a structure object with the following fields:]~

	params\_tests\_dataset, cip\_mags
%
%
\item[See also:]%
\hyperlink{ref_physiol_cip_traceset}{\texttt{physiol\_cip\_traceset}}%
\ (p.~\pageref{ref_physiol_cip_traceset})%
\index[funcref]{@\fidxl{physiol\_cip\_traceset}}%
, \hyperlink{ref_params_tests_dataset}{\texttt{params\_tests\_dataset}}%
\ (p.~\pageref{ref_params_tests_dataset})%
\index[funcref]{@\fidxl{params\_tests\_dataset}}%
, \hyperlink{ref_params_tests_db}{\texttt{params\_tests\_db}}%
\ (p.~\pageref{ref_params_tests_db})%
\index[funcref]{@\fidxl{params\_tests\_db}}%
%
\item[Author:]%
Cengiz Gunay <cgunay@emory.edu>, 2005/01/28%
\end{description}
\methodline%
\subsubsection[Method \texttt{display}]{Method \texttt{cip\_traceset\_dataset/display}}%
\index[funcref]{cip_traceset_dataset@\fidxlb{cip\_traceset\_dataset}!display@\fidxl{display}}%
\label{ref_cip_traceset_dataset__display}%
\hypertarget{ref_cip_traceset_dataset__display}{}%
\begin{description}
%
%
%
%
%
%
%
\item[Author:]%
Cengiz Gunay <cgunay@emory.edu>, 2004/08/04%
\end{description}
\methodline%
\subsubsection[Method \texttt{get}]{Method \texttt{cip\_traceset\_dataset/get}}%
\index[funcref]{cip_traceset_dataset@\fidxlb{cip\_traceset\_dataset}!get@\fidxl{get}}%
\label{ref_cip_traceset_dataset__get}%
\hypertarget{ref_cip_traceset_dataset__get}{}%
\begin{description}
\item[Summary:]Defines generic attribute retrieval for objects.
%
%
%
%
%
%
%
\item[Author:]%
Cengiz Gunay <cgunay@emory.edu>, 2004/09/14%
\end{description}
\methodline%
\subsubsection[Method \texttt{loadItemProfile}]{Method \texttt{cip\_traceset\_dataset/loadItemProfile}}%
\index[funcref]{cip_traceset_dataset@\fidxlb{cip\_traceset\_dataset}!loadItemProfile@\fidxl{loadItemProfile}}%
\label{ref_cip_traceset_dataset__loadItemProfile}%
\hypertarget{ref_cip_traceset_dataset__loadItemProfile}{}%
\begin{description}
\item[Summary:]Loads a cip\_trace\_profile object from a raw data file in the fileset.
%
\item[Usage:]~%
\begin{lyxcode}%
a\_profile = loadItemProfile(fileset, neuron\_id, trace\_index)
%
\end{lyxcode}%
%
%
\item[Parameters:]~
\begin{description}%
\item[\texttt{fileset}:]
     A physiol\_cip\_traceset object.
\item[\texttt{neuron\_id }:]
  tells which item in fileset (corresponds to cells\_filename) to use grab the cell information 
\item[\texttt{trace\_index}:]
 Index of file in traceset.
\end{description}%
%
\item[Returns:]~

	a\_profile: A profile object that implements the getResults method.
%
%
\item[See also:]%
\hyperlink{ref_itemResultsRow}{\texttt{itemResultsRow}}%
\ (p.~\pageref{ref_itemResultsRow})%
\index[funcref]{@\fidxl{itemResultsRow}}%
, \hyperlink{ref_params_tests_fileset}{\texttt{params\_tests\_fileset}}%
\ (p.~\pageref{ref_params_tests_fileset})%
\index[funcref]{@\fidxl{params\_tests\_fileset}}%
, \hyperlink{ref_paramNames}{\texttt{paramNames}}%
\ (p.~\pageref{ref_paramNames})%
\index[funcref]{@\fidxl{paramNames}}%
, \hyperlink{ref_testNames}{\texttt{testNames}}%
\ (p.~\pageref{ref_testNames})%
\index[funcref]{@\fidxl{testNames}}%
%
\item[Author:]%
Cengiz Gunay <cgunay@emory.edu>, 2004/09/14 and Tom Sangrey%
\end{description}
\methodline%
\subsubsection[Method \texttt{readDBItems}]{Method \texttt{cip\_traceset\_dataset/readDBItems}}%
\index[funcref]{cip_traceset_dataset@\fidxlb{cip\_traceset\_dataset}!readDBItems@\fidxl{readDBItems}}%
\label{ref_cip_traceset_dataset__readDBItems}%
\hypertarget{ref_cip_traceset_dataset__readDBItems}{}%
\begin{description}
\item[Summary:]Reads all items to generate a params\_tests\_db object.
%
\item[Usage:]~%
\begin{lyxcode}%
[params, param\_names, tests, test\_names] = readDBItems(obj)
%
\end{lyxcode}%
%
\item[Description:]%
This is a specific method to convert from cip\_traceset\_dataset to
 a params\_tests\_db, or a subclass. Output of this function can be 
 directly fed to the constructor of a params\_tests\_db or a subclass.
%%
\item[Parameters:]~
\begin{description}%
\item[\texttt{obj}:]
 A physiol\_cip\_traceset\_fileset 
\end{description}%
%
\item[Returns:]~

	params, param\_names, tests, test\_names: See params\_tests\_db.
%
%
\item[See also:]%
\hyperlink{ref_params_tests_db}{\texttt{params\_tests\_db}}%
\ (p.~\pageref{ref_params_tests_db})%
\index[funcref]{@\fidxl{params\_tests\_db}}%
, \hyperlink{ref_params_tests_fileset}{\texttt{params\_tests\_fileset}}%
\ (p.~\pageref{ref_params_tests_fileset})%
\index[funcref]{@\fidxl{params\_tests\_fileset}}%
, \hyperlink{ref_itemResultsRow
	    testNames}{\texttt{itemResultsRow
	    testNames}}%
\ (p.~\pageref{ref_itemResultsRow
	    testNames})%
\index[funcref]{@\fidxl{itemResultsRow
	    testNames}}%
, \hyperlink{ref_paramNames}{\texttt{paramNames}}%
\ (p.~\pageref{ref_paramNames})%
\index[funcref]{@\fidxl{paramNames}}%
, \hyperlink{ref_physiol_cip_traceset_fileset}{\texttt{physiol\_cip\_traceset\_fileset}}%
\ (p.~\pageref{ref_physiol_cip_traceset_fileset})%
\index[funcref]{@\fidxl{physiol\_cip\_traceset\_fileset}}%
%
\item[Author:]%
Cengiz Gunay <cgunay@emory.edu>, 2005/01/28%
\end{description}
\methodline%
\subsection{Class \texttt{cluster\_db}}%
\index[funcref]{cluster_db@\fidxlb{cluster\_db}}%
\label{ref_cluster_db}%
\hypertarget{ref_cluster_db}{}%
\subsubsection[Constructor \texttt{cluster\_db}]{Constructor \texttt{cluster\_db/cluster\_db}}%
\index[funcref]{cluster_db@\fidxlb{cluster\_db}!cluster_db@\fidxl{cluster\_db}}%
\label{ref_cluster_db__cluster_db}%
\hypertarget{ref_cluster_db__cluster_db}{}%
\begin{description}
\item[Summary:]A database of cluster centroids generated by a clustering algorithm from a rows of orig\_db.
%
\item[Usage:]~%
\begin{lyxcode}%
a\_cluster\_db = cluster\_db(data, col\_names, orig\_db, cluster\_idx, id, props)
%
\end{lyxcode}%
%
\item[Description:]%
This is a subclass of tests\_db. Use one of the clustering methods in 
 tests\_db, such as kmeansCluster, to get an instance of this class.
%%
\item[Parameters:]~
\begin{description}%
\item[\texttt{data}:]
 Database contents.
\item[\texttt{col\_names}:]
 The column names.
\item[\texttt{orig\_db}:]
 DB whose rows are clustered.
\item[\texttt{cluster\_idx}:]
 Array of cluster numbers that correspond to each row in orig\_db.
\item[\texttt{id}:]
 An identifying string.
\item[\texttt{props}:]
 A structure with any optional properties.
\begin{description}%
\item[\texttt{sumDistances}:]
 Total distance of elements within each cluster.
\item[\texttt{distanceMeasure}:]
 Measure used to find clusters (Default='correlation')
\end{description}%
\end{description}%
%
\item[Returns a structure object with the following fields:]~

	tests\_db, 
	orig\_db: original DB from which clusters were obtained, 
	cluster\_idx: Array associating rows of orig\_db to each cluster here.
	props.
%
%
\item[See also:]%
\hyperlink{ref_tests_db}{\texttt{tests\_db}}%
\ (p.~\pageref{ref_tests_db})%
\index[funcref]{@\fidxl{tests\_db}}%
, \hyperlink{ref_tests_db__kmeansCluster}{\texttt{tests\_db/kmeansCluster}}%
\ (p.~\pageref{ref_tests_db__kmeansCluster})%
\index[funcref]{tests_db@\fidxlb{tests\_db}!kmeansCluster@\fidxl{kmeansCluster}}%
%
\item[Author:]%
Cengiz Gunay <cgunay@emory.edu>, 2005/04/08%
\end{description}
\methodline%
\subsubsection[Method \texttt{display}]{Method \texttt{cluster\_db/display}}%
\index[funcref]{cluster_db@\fidxlb{cluster\_db}!display@\fidxl{display}}%
\label{ref_cluster_db__display}%
\hypertarget{ref_cluster_db__display}{}%
\begin{description}
%
%
%
%
%
%
%
\item[Author:]%
Cengiz Gunay <cgunay@emory.edu>, 2004/08/04%
\end{description}
\methodline%
\subsubsection[Method \texttt{get}]{Method \texttt{cluster\_db/get}}%
\index[funcref]{cluster_db@\fidxlb{cluster\_db}!get@\fidxl{get}}%
\label{ref_cluster_db__get}%
\hypertarget{ref_cluster_db__get}{}%
\begin{description}
\item[Summary:]Defines generic attribute retrieval for objects.
%
%
%
%
%
%
%
\item[Author:]%
Cengiz Gunay <cgunay@emory.edu>, 2004/09/14%
\end{description}
\methodline%
\subsubsection[Method \texttt{plotHist}]{Method \texttt{cluster\_db/plotHist}}%
\index[funcref]{cluster_db@\fidxlb{cluster\_db}!plotHist@\fidxl{plotHist}}%
\label{ref_cluster_db__plotHist}%
\hypertarget{ref_cluster_db__plotHist}{}%
\begin{description}
\item[Summary:]Creates a histogram plot showing the clustering memberships.
%
\item[Usage:]~%
\begin{lyxcode}%
a\_plot = plotHist(a\_cluster\_db, title\_str)
%
\end{lyxcode}%
%
%
\item[Parameters:]~
\begin{description}%
\item[\texttt{a\_cluster\_db}:]
 A cluster\_db object.
\item[\texttt{title\_str}:]
 (Optional) String to append to plot title.
\end{description}%
%
\item[Returns:]~

	a\_plot: A plot\_abstract object that can be plotted.
%
%
\item[See also:]%
\hyperlink{ref_plot_abstract}{\texttt{plot\_abstract}}%
\ (p.~\pageref{ref_plot_abstract})%
\index[funcref]{@\fidxl{plot\_abstract}}%
, \hyperlink{ref_plotFigure}{\texttt{plotFigure}}%
\ (p.~\pageref{ref_plotFigure})%
\index[funcref]{@\fidxl{plotFigure}}%
, \hyperlink{ref_histogram_db}{\texttt{histogram\_db}}%
\ (p.~\pageref{ref_histogram_db})%
\index[funcref]{@\fidxl{histogram\_db}}%
, \hyperlink{ref_histogram_db__plot_abstract}{\texttt{histogram\_db/plot\_abstract}}%
\ (p.~\pageref{ref_histogram_db__plot_abstract})%
\index[funcref]{histogram_db@\fidxlb{histogram\_db}!plot_abstract@\fidxl{plot\_abstract}}%
%
\item[Author:]%
Cengiz Gunay <cgunay@emory.edu>, 2005/04/08%
\end{description}
\methodline%
\subsubsection[Method \texttt{plotQuality}]{Method \texttt{cluster\_db/plotQuality}}%
\index[funcref]{cluster_db@\fidxlb{cluster\_db}!plotQuality@\fidxl{plotQuality}}%
\label{ref_cluster_db__plotQuality}%
\hypertarget{ref_cluster_db__plotQuality}{}%
\begin{description}
\item[Summary:]Creates a plot\_abstract of the silhouette plot showing the clustering quality.
%
\item[Usage:]~%
\begin{lyxcode}%
a\_plot = plotQuality(a\_cluster\_db, title\_str)
%
\end{lyxcode}%
%
%
\item[Parameters:]~
\begin{description}%
\item[\texttt{a\_cluster\_db}:]
 A cluster\_db object.
\item[\texttt{title\_str}:]
 (Optional) String to append to plot title.
\end{description}%
%
\item[Returns:]~

	a\_plot: A plot\_abstract object that can be plotted.
%
%
\item[See also:]%
\hyperlink{ref_plot_abstract}{\texttt{plot\_abstract}}%
\ (p.~\pageref{ref_plot_abstract})%
\index[funcref]{@\fidxl{plot\_abstract}}%
, \hyperlink{ref_plotFigure}{\texttt{plotFigure}}%
\ (p.~\pageref{ref_plotFigure})%
\index[funcref]{@\fidxl{plotFigure}}%
, \hyperlink{ref_silhouette}{\texttt{silhouette}}%
\ (p.~\pageref{ref_silhouette})%
\index[funcref]{@\fidxl{silhouette}}%
%
\item[Author:]%
Cengiz Gunay <cgunay@emory.edu>, 2005/04/08%
\end{description}
\methodline%
\subsubsection[Method \texttt{plot\_abstract}]{Method \texttt{cluster\_db/plot\_abstract}}%
\index[funcref]{cluster_db@\fidxlb{cluster\_db}!plot_abstract@\fidxl{plot\_abstract}}%
\label{ref_cluster_db__plot_abstract}%
\hypertarget{ref_cluster_db__plot_abstract}{}%
\begin{description}
\item[Summary:]Creates a vertical plot\_stack of silhouette and membership histograms for the clusters.
%
\item[Usage:]~%
\begin{lyxcode}%
a\_plot = plot\_abstract(a\_cluster\_db, title\_str)
%
\end{lyxcode}%
%
%
\item[Parameters:]~
\begin{description}%
\item[\texttt{a\_cluster\_db}:]
 A cluster\_db object.
\item[\texttt{title\_str}:]
 (Optional) String to append to plot title.
\end{description}%
%
\item[Returns:]~

	a\_plot: A plot\_abstract object that can be plotted.
%
%
\item[See also:]%
\hyperlink{ref_cluster_db__plotQuality}{\texttt{cluster\_db/plotQuality}}%
\ (p.~\pageref{ref_cluster_db__plotQuality})%
\index[funcref]{cluster_db@\fidxlb{cluster\_db}!plotQuality@\fidxl{plotQuality}}%
, \hyperlink{ref_cluster_db__plotHist}{\texttt{cluster\_db/plotHist}}%
\ (p.~\pageref{ref_cluster_db__plotHist})%
\index[funcref]{cluster_db@\fidxlb{cluster\_db}!plotHist@\fidxl{plotHist}}%
%
\item[Author:]%
Cengiz Gunay <cgunay@emory.edu>, 2005/04/08%
\end{description}
\methodline%
\subsection{Class \texttt{corrcoefs\_db}}%
\index[funcref]{corrcoefs_db@\fidxlb{corrcoefs\_db}}%
\label{ref_corrcoefs_db}%
\hypertarget{ref_corrcoefs_db}{}%
\subsubsection[Constructor \texttt{corrcoefs\_db}]{Constructor \texttt{corrcoefs\_db/corrcoefs\_db}}%
\index[funcref]{corrcoefs_db@\fidxlb{corrcoefs\_db}!corrcoefs_db@\fidxl{corrcoefs\_db}}%
\label{ref_corrcoefs_db__corrcoefs_db}%
\hypertarget{ref_corrcoefs_db__corrcoefs_db}{}%
\begin{description}
\item[Summary:]A database of correlation coefficients generated from 
		a column of another database.
%
\item[Usage:]~%
\begin{lyxcode}%
a\_coef\_db = corrcoefs\_db(col\_name, coefs, coef\_names, pages, id, props)
%
\end{lyxcode}%
%
\item[Description:]%
This is a subclass of tests\_3d\_db. Allows generating a plot, etc.
%%
\item[Parameters:]~
\begin{description}%
\item[\texttt{col\_name}:]
 The column with which the others are correlated.
\item[\texttt{coefs}:]
 Matrix where each column has another coefficient.
\item[\texttt{coef\_names}:]
 Cell array of column names corresponding to coefficients.
\item[\texttt{pages}:]
 Column vector of page indices pointing to the tests\_3d\_db.
\item[\texttt{id}:]
 An identifying string.
\item[\texttt{props}:]
 A structure with any optional properties.
\end{description}%
%
\item[Returns a structure object with the following fields:]~

	tests\_db.
%
%
\item[See also:]%
\hyperlink{ref_tests_db}{\texttt{tests\_db}}%
\ (p.~\pageref{ref_tests_db})%
\index[funcref]{@\fidxl{tests\_db}}%
, \hyperlink{ref_plot_simple}{\texttt{plot\_simple}}%
\ (p.~\pageref{ref_plot_simple})%
\index[funcref]{@\fidxl{plot\_simple}}%
, \hyperlink{ref_tests_db__histogram}{\texttt{tests\_db/histogram}}%
\ (p.~\pageref{ref_tests_db__histogram})%
\index[funcref]{tests_db@\fidxlb{tests\_db}!histogram@\fidxl{histogram}}%
%
\item[Author:]%
Cengiz Gunay <cgunay@emory.edu>, 2004/10/06%
\end{description}
\methodline%
\subsection{Class \texttt{dataset\_db\_bundle}}%
\index[funcref]{dataset_db_bundle@\fidxlb{dataset\_db\_bundle}}%
\label{ref_dataset_db_bundle}%
\hypertarget{ref_dataset_db_bundle}{}%
\subsubsection[Constructor \texttt{dataset\_db\_bundle}]{Constructor \texttt{dataset\_db\_bundle/dataset\_db\_bundle}}%
\index[funcref]{dataset_db_bundle@\fidxlb{dataset\_db\_bundle}!dataset_db_bundle@\fidxl{dataset\_db\_bundle}}%
\label{ref_dataset_db_bundle__dataset_db_bundle}%
\hypertarget{ref_dataset_db_bundle__dataset_db_bundle}{}%
\begin{description}
\item[Summary:]The dataset and the DB created from it bundled together.
%
\item[Usage:]~%
\begin{lyxcode}%
a\_bundle = dataset\_db\_bundle(a\_dataset, a\_db, a\_joined\_db, props)
%
\end{lyxcode}%
%
\item[Description:]%
This class is made to enable operations that require seamless connection between 
 the high-level DB and the raw data. The raw DB is only required to bridge the gap 
 between the high-level DB and the dataset. Therefore it only needs to contain 
 columns necessary to make this connection. It is not required to include all 
 raw DB columns, which is inefficient.
%%
\item[Parameters:]~
\begin{description}%
\item[\texttt{a\_dataset}:]
 A params\_tests\_dataset object or a subclass.
\item[\texttt{a\_db}:]
 The raw tests\_db object (or a subclass) created from the dataset.
\item[\texttt{a\_joined\_db}:]
 The processed DB created from the raw DB.
\item[\texttt{props}:]
 A structure with any optional properties.
\end{description}%
%
\item[Returns a structure object with the following fields:]~

	dataset, db, joined\_db, props.
%
%
\item[See also:]%
\hyperlink{ref_tests_db}{\texttt{tests\_db}}%
\ (p.~\pageref{ref_tests_db})%
\index[funcref]{@\fidxl{tests\_db}}%
, \hyperlink{ref_params_tests_dataset}{\texttt{params\_tests\_dataset}}%
\ (p.~\pageref{ref_params_tests_dataset})%
\index[funcref]{@\fidxl{params\_tests\_dataset}}%
%
\item[Author:]%
Cengiz Gunay <cgunay@emory.edu>, 2005/12/13%
\end{description}
\methodline%
\subsubsection[Method \texttt{display}]{Method \texttt{dataset\_db\_bundle/display}}%
\index[funcref]{dataset_db_bundle@\fidxlb{dataset\_db\_bundle}!display@\fidxl{display}}%
\label{ref_dataset_db_bundle__display}%
\hypertarget{ref_dataset_db_bundle__display}{}%
\begin{description}
%
%
%
%
%
%
%
\item[Author:]%
Cengiz Gunay <cgunay@emory.edu>, 2004/08/04%
\end{description}
\methodline%
\subsubsection[Method \texttt{get}]{Method \texttt{dataset\_db\_bundle/get}}%
\index[funcref]{dataset_db_bundle@\fidxlb{dataset\_db\_bundle}!get@\fidxl{get}}%
\label{ref_dataset_db_bundle__get}%
\hypertarget{ref_dataset_db_bundle__get}{}%
\begin{description}
\item[Summary:]Defines generic attribute retrieval for objects.
%
%
%
%
%
%
%
\item[Author:]%
Cengiz Gunay <cgunay@emory.edu>, 2004/09/14%
\end{description}
\methodline%
\subsubsection[Method \texttt{set}]{Method \texttt{dataset\_db\_bundle/set}}%
\index[funcref]{dataset_db_bundle@\fidxlb{dataset\_db\_bundle}!set@\fidxl{set}}%
\label{ref_dataset_db_bundle__set}%
\hypertarget{ref_dataset_db_bundle__set}{}%
\begin{description}
\item[Summary:]Generic method for setting object attributes.
%
%
%
%
%
%
%
\item[Author:]%
Cengiz Gunay <cgunay@emory.edu>, 2004/10/08%
\end{description}
\methodline%
\subsubsection[Method \texttt{constrainedMeasuresPreset}]{Method \texttt{dataset\_db\_bundle/constrainedMeasuresPreset}}%
\index[funcref]{dataset_db_bundle@\fidxlb{dataset\_db\_bundle}!constrainedMeasuresPreset@\fidxl{constrainedMeasuresPreset}}%
\label{ref_dataset_db_bundle__constrainedMeasuresPreset}%
\hypertarget{ref_dataset_db_bundle__constrainedMeasuresPreset}{}%
\begin{description}
\item[Summary:]Returns a dataset\_db\_bundle with constrained measures according to chosen preset.
%
\item[Usage:]~%
\begin{lyxcode}%
[a\_bundle test\_names] = constrainedMeasuresPreset(a\_bundle, preset, props)
%
\end{lyxcode}%
%
%
\item[Parameters:]~
\begin{description}%
\item[\texttt{a\_bundle}:]
 A dataset\_db\_bundle object.
\item[\texttt{preset}:]
 Choose preset measure list (default=1).
\item[\texttt{props}:]
 A structure with any optional properties.
\end{description}%
%
\item[Returns:]~

	a\_bundle: Modified bundle.
%
%
\item[See also:]%
\hyperlink{ref_physiol_bundle__constrainedMeasuresPreset}{\texttt{physiol\_bundle/constrainedMeasuresPreset}}%
\ (p.~\pageref{ref_physiol_bundle__constrainedMeasuresPreset})%
\index[funcref]{physiol_bundle@\fidxlb{physiol\_bundle}!constrainedMeasuresPreset@\fidxl{constrainedMeasuresPreset}}%
%
\item[Author:]%
Cengiz Gunay <cgunay@emory.edu>, 2006/06/13%
\end{description}
\methodline%
\subsubsection[Method \texttt{rankingReportTeX}]{Method \texttt{dataset\_db\_bundle/rankingReportTeX}}%
\index[funcref]{dataset_db_bundle@\fidxlb{dataset\_db\_bundle}!rankingReportTeX@\fidxl{rankingReportTeX}}%
\label{ref_dataset_db_bundle__rankingReportTeX}%
\hypertarget{ref_dataset_db_bundle__rankingReportTeX}{}%
\begin{description}
\item[Summary:]Generates a report by comparing a\_bundle with the given match criteria, crit\_db from crit\_bundle.
%
\item[Usage:]~%
\begin{lyxcode}%
tex\_string = rankingReportTeX(a\_bundle, crit\_bundle, crit\_db, props)
%
\end{lyxcode}%
%
\item[Description:]%
Generates a LaTeX document with:
	- (optional) Raw traces compared with some best matches at different distances
	- Values of some top matching a\_db rows and match errors in a floating table.
	- colored-plot of measure errors for some top matches.
	- Parameter distributions of 50 best matches as a bar graph.
%%
\item[Parameters:]~
\begin{description}%
\item[\texttt{a\_bundle}:]
 A dataset\_db\_bundle object that contains the DB to compare rows from.
\item[\texttt{crit\_bundle}:]
 A dataset\_db\_bundle object that contains the criterion dataset.
\item[\texttt{crit\_db}:]
 A tests\_db object holding the match criterion tests and STDs

which can be created with matchingRow.\item[\texttt{props}:]
 A structure with any optional properties.
\begin{description}%
\item[\texttt{caption}:]
 Identification of the criterion db (not needed/used?).
\item[\texttt{num\_matches}:]
 Number of best matches to display (default=10).
\item[\texttt{rotate}:]
 Rotation angle for best matches table (default=90).
\end{description}%
\end{description}%
%
\item[Returns:]~

	tex\_string: LaTeX document string.
%
%
\item[See also:]%
\hyperlink{ref_displayRowsTeX}{\texttt{displayRowsTeX}}%
\ (p.~\pageref{ref_displayRowsTeX})%
\index[funcref]{@\fidxl{displayRowsTeX}}%
%
\item[Author:]%
Cengiz Gunay <cgunay@emory.edu>, 2005/12/13%
\end{description}
\methodline%
\subsubsection[Method \texttt{matchingRow}]{Method \texttt{dataset\_db\_bundle/matchingRow}}%
\index[funcref]{dataset_db_bundle@\fidxlb{dataset\_db\_bundle}!matchingRow@\fidxl{matchingRow}}%
\label{ref_dataset_db_bundle__matchingRow}%
\hypertarget{ref_dataset_db_bundle__matchingRow}{}%
\begin{description}
\item[Summary:]Creates a criterion database for matching the tests of a row.
%
\item[Usage:]~%
\begin{lyxcode}%
crit\_db = matchingRow(a\_bundle, row, props)
%
\end{lyxcode}%
%
\item[Description:]%
Copies selected test values from row as the first row into the 
 criterion db. Adds a second row for the STD of each column in the db.
%%
\item[Parameters:]~
\begin{description}%
\item[\texttt{a\_bundle}:]
 A tests\_db object.
\item[\texttt{row}:]
 A row index to match.
\item[\texttt{props}:]
 A structure with any optional properties.
\begin{description}%
\item[\texttt{std\_db}:]
 Take the standard deviation from this db instead.
\end{description}%
\end{description}%
%
\item[Returns:]~

	crit\_db: A tests\_db with two rows for values and STDs.
%
\item[Example:]~
\begin{lyxcode}        physiol\_bundle has an overloaded matchingRow method that\\%
        takes the TracesetIndex as argument:\\%
        >> crit\_db = matchingRow(pbundle, 61)\\%
\end{lyxcode}
%
\item[See also:]%
\hyperlink{ref_rankMatching}{\texttt{rankMatching}}%
\ (p.~\pageref{ref_rankMatching})%
\index[funcref]{@\fidxl{rankMatching}}%
, \hyperlink{ref_tests_db}{\texttt{tests\_db}}%
\ (p.~\pageref{ref_tests_db})%
\index[funcref]{@\fidxl{tests\_db}}%
, \hyperlink{ref_tests2cols}{\texttt{tests2cols}}%
\ (p.~\pageref{ref_tests2cols})%
\index[funcref]{@\fidxl{tests2cols}}%
%
\item[Author:]%
Cengiz Gunay <cgunay@emory.edu>, 2005/12/21%
\end{description}
\methodline%
\subsubsection[Method \texttt{reportNeuron}]{Method \texttt{dataset\_db\_bundle/reportNeuron}}%
\index[funcref]{dataset_db_bundle@\fidxlb{dataset\_db\_bundle}!reportNeuron@\fidxl{reportNeuron}}%
\label{ref_dataset_db_bundle__reportNeuron}%
\hypertarget{ref_dataset_db_bundle__reportNeuron}{}%
\begin{description}
\item[Summary:]Generates a report of neuron at given an\_index of a\_bundle.
%
\item[Usage:]~%
\begin{lyxcode}%
a\_doc\_multi = reportNeuron(a\_bundle, an\_index, props)
%
\end{lyxcode}%
%
\item[Description:]%
Generates a report document with preset layouts of annotated plots of
 the selected neuron. See reportLayout below for presets.
%%
\item[Parameters:]~
\begin{description}%
\item[\texttt{a\_bundle}:]
 a dataset\_db\_bundle object which contains the neuron
\item[\texttt{an\_index}:]
 The index to pass to ctFromRows method of a\_bundle.
\item[\texttt{props}:]
 A structure with any optional properties.
\begin{description}%
\item[\texttt{reportLayout}:]
 Allows choosing one of predefined report types (strings):
\begin{description}%
\item[\texttt{1}:]
 Only +/- 100 pA traces in one plot (default).

1a/b: Either one of the +/- 100 pA traces in one plot.\item[\texttt{2}:]
 Only +/- 100 pA traces and spike shapes in one horiz. plot.
\item[\texttt{3}:]
 +100 pA raw trace and rate profile stacked vertically.
\item[\texttt{3b}:]
 -100 pA raw trace and rate profile stacked vertically.
\item[\texttt{4}:]
 Horiz stack of +/- 100 pA raw trace with rate profiles underneath.
\item[\texttt{5}:]
 5-piece trace, spike shape, f-I curve, f-t curve quad-plot.
\end{description}%
\item[\texttt{numTraces}:]
 Limit number of traces to show in plot (>=1).
\item[\texttt{traces}:]
 List of acceptable traces to load.
\item[\texttt{traceAxisLimits}:]
 If given, use these limits for trace plots.
\item[\texttt{rateAxisLimits}:]
 If given, use these limits for rate plots.
\item[\texttt{fIAxisLimits}:]
 If given, use these limits for fIcurve plots.
\item[\texttt{fIstats}:]
 Add a fI-stats plot in addition to the curve.
\item[\texttt{sshapeAxisLimits}:]
 If given, use these limits for spike shape plots.
\item[\texttt{sshapeResults}:]
 If 1, plot measures on the spike shape (default=1).
\end{description}%
\end{description}%
%
\item[Returns:]~

	a\_doc\_multi: A doc\_multi object that can be printed as a PS or PDF file.
%
\item[Example:]~
\begin{lyxcode} >> printTeXFile(reportNeuron(mbundle, 2222), 'a.tex')\\%
 or:\\%
 >> plotFigure(get(reportNeuron(mbundle, 2222), 'plot'))\\%
\end{lyxcode}
%
\item[See also:]%
\hyperlink{ref_doc_multi}{\texttt{doc\_multi}}%
\ (p.~\pageref{ref_doc_multi})%
\index[funcref]{@\fidxl{doc\_multi}}%
, \hyperlink{ref_doc_generate}{\texttt{doc\_generate}}%
\ (p.~\pageref{ref_doc_generate})%
\index[funcref]{@\fidxl{doc\_generate}}%
, \hyperlink{ref_doc_generate__printTeXFile}{\texttt{doc\_generate/printTeXFile}}%
\ (p.~\pageref{ref_doc_generate__printTeXFile})%
\index[funcref]{doc_generate@\fidxlb{doc\_generate}!printTeXFile@\fidxl{printTeXFile}}%
%
\item[Author:]%
Cengiz Gunay <cgunay@emory.edu>, 2006/01/24%
\end{description}
\methodline%
\subsubsection[Method \texttt{plotfICurve}]{Method \texttt{dataset\_db\_bundle/plotfICurve}}%
\index[funcref]{dataset_db_bundle@\fidxlb{dataset\_db\_bundle}!plotfICurve@\fidxl{plotfICurve}}%
\label{ref_dataset_db_bundle__plotfICurve}%
\hypertarget{ref_dataset_db_bundle__plotfICurve}{}%
\begin{description}
\item[Summary:]Generates a f-I curve doc\_plot for neuron at given an\_index in a\_bundle.
%
\item[Usage:]~%
\begin{lyxcode}%
a\_plot = plotfICurve(a\_bundle, trial\_num, props)
%
\end{lyxcode}%
%
%
\item[Parameters:]~
\begin{description}%
\item[\texttt{a\_bundle}:]
 A dataset\_db\_bundle object.
\item[\texttt{an\_index}:]
 An index with which to address the a\_bundle.
\item[\texttt{props}:]
 A structure with any optional properties.
\begin{description}%
\item[\texttt{shortCaption}:]
 This appears in the figure caption.
\item[\texttt{plotMStats}:]
 If set, add the a\_bundle stats plot.
\item[\texttt{captionToStats}:]
 Use this as its legend label. 
\item[\texttt{quiet}:]
 if given, no title is produced

(passed to plot\_superpose)\end{description}%
\end{description}%
%
\item[Returns:]~

	a\_plot: A plot\_superpose that contains a f-I curve plot.
%
\item[Example:]~
\begin{lyxcode} >> a\_p = plotfICurve(r, 1);\\%
 >> plotFigure(a\_p, 'The f-I curve of best matching model');\\%
\end{lyxcode}
%
\item[See also:]%
\hyperlink{ref_plot_abstract}{\texttt{plot\_abstract}}%
\ (p.~\pageref{ref_plot_abstract})%
\index[funcref]{@\fidxl{plot\_abstract}}%
, \hyperlink{ref_plot_superpose}{\texttt{plot\_superpose}}%
\ (p.~\pageref{ref_plot_superpose})%
\index[funcref]{@\fidxl{plot\_superpose}}%
%
\item[Author:]%
Cengiz Gunay <cgunay@emory.edu>, 2006/01/16%
\end{description}
\methodline%
\subsubsection[Method \texttt{subsref}]{Method \texttt{dataset\_db\_bundle/subsref}}%
\index[funcref]{dataset_db_bundle@\fidxlb{dataset\_db\_bundle}!subsref@\fidxl{subsref}}%
\label{ref_dataset_db_bundle__subsref}%
\hypertarget{ref_dataset_db_bundle__subsref}{}%
\begin{description}
\item[Summary:]Defines indexing for tests\_db objects for () and . operations. 
%
\item[Usage:]~%
\begin{lyxcode}%
obj = obj(rows, tests)
 obj = obj.attribute
%
\end{lyxcode}%
%
\item[Description:]%
Returns attributes or selects the given test columns and rows
 and returns in a new tests\_db object.
%%
\item[Parameters:]~
\begin{description}%
\item[\texttt{obj}:]
 A tests\_db object.
\item[\texttt{rows}:]
 A logical or index vector of rows. If ':', all rows.
\item[\texttt{tests}:]
 Cell array of test names or column indices. If ':', all tests.
\item[\texttt{attribute}:]
 A tests\_db class attribute.
\end{description}%
%
\item[Returns:]~

	obj: The new tests\_db object.
%
%
\item[See also:]%
\hyperlink{ref_subsref}{\texttt{subsref}}%
\ (p.~\pageref{ref_subsref})%
\index[funcref]{@\fidxl{subsref}}%
, \hyperlink{ref_tests_db}{\texttt{tests\_db}}%
\ (p.~\pageref{ref_tests_db})%
\index[funcref]{@\fidxl{tests\_db}}%
%
\item[Author:]%
Cengiz Gunay <cgunay@emory.edu>, 2004/09/17%
\end{description}
\methodline%
\subsubsection[Method \texttt{getNeuronRowIndex}]{Method \texttt{dataset\_db\_bundle/getNeuronRowIndex}}%
\index[funcref]{dataset_db_bundle@\fidxlb{dataset\_db\_bundle}!getNeuronRowIndex@\fidxl{getNeuronRowIndex}}%
\label{ref_dataset_db_bundle__getNeuronRowIndex}%
\hypertarget{ref_dataset_db_bundle__getNeuronRowIndex}{}%
\begin{description}
\item[Summary:]Returns the neuron index from bundle.
%
\item[Usage:]~%
\begin{lyxcode}%
a\_row\_index = getNeuronRowIndex(a\_bundle, an\_index, props)
%
\end{lyxcode}%
%
\item[Description:]%
This is a polymorphic method. Therefor it is not defined for this class, 
 but see subclasses of dataset\_db\_bundle for its more meaningful implementations.
%%
\item[Parameters:]~
\begin{description}%
\item[\texttt{a\_bundle}:]
 A dataset\_db\_bundle subclass object.
\item[\texttt{an\_index}:]
 An index number of neuron, or a DB row containing this.
\item[\texttt{props}:]
 A structure with any optional properties.
\end{description}%
%
\item[Returns:]~

	a\_row\_index: A row index of neuron in a\_bundle.joined\_db.
%
\item[Example:]~
\begin{lyxcode} >> displayRows(mbundle.joined\_db(getNeuronRowIndex(mbundle, 98364), :))\\%
\end{lyxcode}
%
\item[See also:]%
\hyperlink{ref_dataset_db_bundle}{\texttt{dataset\_db\_bundle}}%
\ (p.~\pageref{ref_dataset_db_bundle})%
\index[funcref]{@\fidxl{dataset\_db\_bundle}}%
%
\item[Author:]%
Cengiz Gunay <cgunay@emory.edu>, 2006/06/09%
\end{description}
\methodline%
\subsection{Class \texttt{doc\_generate}}%
\index[funcref]{doc_generate@\fidxlb{doc\_generate}}%
\label{ref_doc_generate}%
\hypertarget{ref_doc_generate}{}%
\subsubsection[Constructor \texttt{doc\_generate}]{Constructor \texttt{doc\_generate/doc\_generate}}%
\index[funcref]{doc_generate@\fidxlb{doc\_generate}!doc_generate@\fidxl{doc\_generate}}%
\label{ref_doc_generate__doc_generate}%
\hypertarget{ref_doc_generate__doc_generate}{}%
\begin{description}
\item[Summary:]Generic class to help generate printed or annotated documents with results.
%
\item[Usage:]~%
\begin{lyxcode}%
a\_doc = doc\_generate(text\_string, id, props)
%
\end{lyxcode}%
%
\item[Description:]%
This constitutes the base class for other doc\_ classes. For convenience,
 this class holds a text\_string to be printed when the document is generated
 with the printTeXFile option.
%%
\item[Parameters:]~
\begin{description}%
\item[\texttt{text\_string}:]
 Contents of this document.
\item[\texttt{id}:]
 An identifying string.
\item[\texttt{props}:]
 A structure with any optional properties.
\end{description}%
%
\item[Returns a structure object with the following fields:]~

	text, id, props.
%
%
\item[See also:]%
\hyperlink{ref_doc_plot}{\texttt{doc\_plot}}%
\ (p.~\pageref{ref_doc_plot})%
\index[funcref]{@\fidxl{doc\_plot}}%
, \hyperlink{ref_doc_multi}{\texttt{doc\_multi}}%
\ (p.~\pageref{ref_doc_multi})%
\index[funcref]{@\fidxl{doc\_multi}}%
%
\item[Author:]%
Cengiz Gunay <cgunay@emory.edu>, 2006/01/17%
\end{description}
\methodline%
\subsubsection[Method \texttt{display}]{Method \texttt{doc\_generate/display}}%
\index[funcref]{doc_generate@\fidxlb{doc\_generate}!display@\fidxl{display}}%
\label{ref_doc_generate__display}%
\hypertarget{ref_doc_generate__display}{}%
\begin{description}
%
%
%
%
%
%
%
\item[Author:]%
Cengiz Gunay <cgunay@emory.edu>, 2004/08/04%
\end{description}
\methodline%
\subsubsection[Method \texttt{get}]{Method \texttt{doc\_generate/get}}%
\index[funcref]{doc_generate@\fidxlb{doc\_generate}!get@\fidxl{get}}%
\label{ref_doc_generate__get}%
\hypertarget{ref_doc_generate__get}{}%
\begin{description}
\item[Summary:]Defines generic attribute retrieval for objects.
%
%
%
%
%
%
%
\item[Author:]%
Cengiz Gunay <cgunay@emory.edu>, 2004/09/14%
\end{description}
\methodline%
\subsubsection[Method \texttt{set}]{Method \texttt{doc\_generate/set}}%
\index[funcref]{doc_generate@\fidxlb{doc\_generate}!set@\fidxl{set}}%
\label{ref_doc_generate__set}%
\hypertarget{ref_doc_generate__set}{}%
\begin{description}
\item[Summary:]Generic method for setting object attributes.
%
%
%
%
%
%
%
\item[Author:]%
Cengiz Gunay <cgunay@emory.edu>, 2004/10/08%
\end{description}
\methodline%
\subsubsection[Method \texttt{subsref}]{Method \texttt{doc\_generate/subsref}}%
\index[funcref]{doc_generate@\fidxlb{doc\_generate}!subsref@\fidxl{subsref}}%
\label{ref_doc_generate__subsref}%
\hypertarget{ref_doc_generate__subsref}{}%
\begin{description}
\item[Summary:]Defines generic indexing for objects.
%
%
%
%
%
%
%
%
\end{description}
\methodline%
\subsubsection[Method \texttt{printTeXFile}]{Method \texttt{doc\_generate/printTeXFile}}%
\index[funcref]{doc_generate@\fidxlb{doc\_generate}!printTeXFile@\fidxl{printTeXFile}}%
\label{ref_doc_generate__printTeXFile}%
\hypertarget{ref_doc_generate__printTeXFile}{}%
\begin{description}
\item[Summary:]Creates a TeX file with the contents of this document.
%
\item[Usage:]~%
\begin{lyxcode}%
printTeXFile(a\_doc, filename, props)
%
\end{lyxcode}%
%
\item[Description:]%
Calls getTeXString to generate the contents. The filename is adjusted with 
 a call to properFilename to generate an acceptable TeX filename. TeX-specific
 should only be added at this point or at getTeXString, because before we want
 the object to be a generic document container.
%%
\item[Parameters:]~
\begin{description}%
\item[\texttt{a\_doc}:]
 A tests\_db object.
\item[\texttt{filename}:]
 To write the TeX string.
\item[\texttt{props}:]
 A structure with any optional properties.
\end{description}%
%
\item[Returns:]~

	tex\_string: A string that contains TeX commands, which upon writing to a file,
	  can be interpreted by the TeX engine to produce a document.
%
\item[Example:]~
\begin{lyxcode}        >> a\_doc = doc\_plot(a\_plot, 'Results from cell.', 'Results.', struct, ''); \\%
        >> printTeXFile(a\_doc, 'my\_doc.tex')\\%
        then my\_doc.tex can be used by including from a valid LaTeX document.\\%
\end{lyxcode}
%
\item[See also:]%
\hyperlink{ref_doc_generate}{\texttt{doc\_generate}}%
\ (p.~\pageref{ref_doc_generate})%
\index[funcref]{@\fidxl{doc\_generate}}%
, \hyperlink{ref_doc_plot}{\texttt{doc\_plot}}%
\ (p.~\pageref{ref_doc_plot})%
\index[funcref]{@\fidxl{doc\_plot}}%
, \hyperlink{ref_string2File}{\texttt{string2File}}%
\ (p.~\pageref{ref_string2File})%
\index[funcref]{@\fidxl{string2File}}%
, \hyperlink{ref_properFilename}{\texttt{properFilename}}%
\ (p.~\pageref{ref_properFilename})%
\index[funcref]{@\fidxl{properFilename}}%
%
\item[Author:]%
Cengiz Gunay <cgunay@emory.edu>, 2006/01/17%
\end{description}
\methodline%
\subsubsection[Method \texttt{getTeXString}]{Method \texttt{doc\_generate/getTeXString}}%
\index[funcref]{doc_generate@\fidxlb{doc\_generate}!getTeXString@\fidxl{getTeXString}}%
\label{ref_doc_generate__getTeXString}%
\hypertarget{ref_doc_generate__getTeXString}{}%
\begin{description}
\item[Summary:]Returns the TeX representation for the document.
%
\item[Usage:]~%
\begin{lyxcode}%
tex\_string = getTeXString(a\_doc, props)
%
\end{lyxcode}%
%
\item[Description:]%
This is an abstract placeholder for this method. It specifies what this 
 method should do in the subclasses that implement it. This method should
 create all the auxiliary files needed by the document. The generated tex\_string
 should be ready to be visualized.
%%
\item[Parameters:]~
\begin{description}%
\item[\texttt{a\_doc}:]
 A tests\_db object.
\item[\texttt{props}:]
 A structure with any optional properties.
\end{description}%
%
\item[Returns:]~

	tex\_string: A string that contains TeX commands, which upon writing to a file,
	  can be interpreted by the TeX engine to produce a document.
%
\item[Example:]~
\begin{lyxcode}        doc\_plot has an overloaded getTeXString method:\\%
        >> tex\_string = getTeXString(a\_doc\_plot)\\%
        >> string2File(tex\_string, 'my\_doc.tex')\\%
        then my\_doc.tex can be used by including from a valid LaTeX document.\\%
\end{lyxcode}
%
\item[See also:]%
\hyperlink{ref_doc_generate}{\texttt{doc\_generate}}%
\ (p.~\pageref{ref_doc_generate})%
\index[funcref]{@\fidxl{doc\_generate}}%
, \hyperlink{ref_doc_plot}{\texttt{doc\_plot}}%
\ (p.~\pageref{ref_doc_plot})%
\index[funcref]{@\fidxl{doc\_plot}}%
%
\item[Author:]%
Cengiz Gunay <cgunay@emory.edu>, 2006/01/17%
\end{description}
\methodline%
\subsection{Class \texttt{doc\_multi}}%
\index[funcref]{doc_multi@\fidxlb{doc\_multi}}%
\label{ref_doc_multi}%
\hypertarget{ref_doc_multi}{}%
\subsubsection[Constructor \texttt{doc\_multi}]{Constructor \texttt{doc\_multi/doc\_multi}}%
\index[funcref]{doc_multi@\fidxlb{doc\_multi}!doc_multi@\fidxl{doc\_multi}}%
\label{ref_doc_multi__doc_multi}%
\hypertarget{ref_doc_multi__doc_multi}{}%
\begin{description}
\item[Summary:]A document that is composed of multiple other doc\_generate objects.
%
\item[Usage:]~%
\begin{lyxcode}%
a\_doc = doc\_multi(docs, id, props)
%
\end{lyxcode}%
%
%
\item[Parameters:]~
\begin{description}%
\item[\texttt{docs}:]
 A vector of doc\_generate objects.
\item[\texttt{id}:]
 An identifying string.
\item[\texttt{props}:]
 A structure with any optional properties.
\end{description}%
%
\item[Returns a structure object with the following fields:]~

	docs, doc\_generate.
%
\item[Example:]~
\begin{lyxcode} >> mydoc = doc\_multi([doc\_plot(a\_plot1), doc\_plot(a\_plot2)], 'Two plots')\\%
 >> printTeXFile(mydoc, 'two\_plots.tex')\\%
\end{lyxcode}
%
\item[See also:]%
\hyperlink{ref_doc_generate}{\texttt{doc\_generate}}%
\ (p.~\pageref{ref_doc_generate})%
\index[funcref]{@\fidxl{doc\_generate}}%
, \hyperlink{ref_getTeXString}{\texttt{getTeXString}}%
\ (p.~\pageref{ref_getTeXString})%
\index[funcref]{@\fidxl{getTeXString}}%
, \hyperlink{ref_doc_generate__printTeXFile}{\texttt{doc\_generate/printTeXFile}}%
\ (p.~\pageref{ref_doc_generate__printTeXFile})%
\index[funcref]{doc_generate@\fidxlb{doc\_generate}!printTeXFile@\fidxl{printTeXFile}}%
%
\item[Author:]%
Cengiz Gunay <cgunay@emory.edu>, 2006/01/17%
\end{description}
\methodline%
\subsubsection[Method \texttt{get}]{Method \texttt{doc\_multi/get}}%
\index[funcref]{doc_multi@\fidxlb{doc\_multi}!get@\fidxl{get}}%
\label{ref_doc_multi__get}%
\hypertarget{ref_doc_multi__get}{}%
\begin{description}
\item[Summary:]Defines generic attribute retrieval for objects.
%
%
%
%
%
%
%
\item[Author:]%
Cengiz Gunay <cgunay@emory.edu>, 2004/09/14%
\end{description}
\methodline%
\subsubsection[Method \texttt{set}]{Method \texttt{doc\_multi/set}}%
\index[funcref]{doc_multi@\fidxlb{doc\_multi}!set@\fidxl{set}}%
\label{ref_doc_multi__set}%
\hypertarget{ref_doc_multi__set}{}%
\begin{description}
\item[Summary:]Generic method for setting object attributes.
%
%
%
%
%
%
%
\item[Author:]%
Cengiz Gunay <cgunay@emory.edu>, 2004/10/08%
\end{description}
\methodline%
\subsubsection[Method \texttt{getTeXString}]{Method \texttt{doc\_multi/getTeXString}}%
\index[funcref]{doc_multi@\fidxlb{doc\_multi}!getTeXString@\fidxl{getTeXString}}%
\label{ref_doc_multi__getTeXString}%
\hypertarget{ref_doc_multi__getTeXString}{}%
\begin{description}
\item[Summary:]Returns the TeX representation for the document.
%
\item[Usage:]~%
\begin{lyxcode}%
tex\_string = getTeXString(a\_doc, props)
%
\end{lyxcode}%
%
\item[Description:]%
Concatenates TeX representations of doc\_generate, or subclass, objects it contains.
%%
\item[Parameters:]~
\begin{description}%
\item[\texttt{a\_doc}:]
 A tests\_db object.
\item[\texttt{props}:]
 A structure with any optional properties.
\end{description}%
%
\item[Returns:]~

	tex\_string: A string that contains TeX commands, which upon writing to a file,
	  can be interpreted by the TeX engine to produce a document.
%
\item[Example:]~
\begin{lyxcode}        doc\_plot has an overloaded getTeXString method:\\%
        >> tex\_string = getTeXString(a\_doc\_plot)\\%
        >> string2File(tex\_string, 'my\_doc.tex')\\%
        then my\_doc.tex can be used by including from a valid LaTeX document.\\%
\end{lyxcode}
%
\item[See also:]%
\hyperlink{ref_doc_generate}{\texttt{doc\_generate}}%
\ (p.~\pageref{ref_doc_generate})%
\index[funcref]{@\fidxl{doc\_generate}}%
, \hyperlink{ref_doc_plot}{\texttt{doc\_plot}}%
\ (p.~\pageref{ref_doc_plot})%
\index[funcref]{@\fidxl{doc\_plot}}%
%
\item[Author:]%
Cengiz Gunay <cgunay@emory.edu>, 2006/01/17%
\end{description}
\methodline%
\subsection{Class \texttt{doc\_plot}}%
\index[funcref]{doc_plot@\fidxlb{doc\_plot}}%
\label{ref_doc_plot}%
\hypertarget{ref_doc_plot}{}%
\subsubsection[Constructor \texttt{doc\_plot}]{Constructor \texttt{doc\_plot/doc\_plot}}%
\index[funcref]{doc_plot@\fidxlb{doc\_plot}!doc_plot@\fidxl{doc\_plot}}%
\label{ref_doc_plot__doc_plot}%
\hypertarget{ref_doc_plot__doc_plot}{}%
\begin{description}
\item[Summary:]Generates a formatted plot for printing, annotated with captions.
%
\item[Usage:]~%
\begin{lyxcode}%
a\_doc = doc\_plot(a\_plot, caption, plot\_filename, float\_props, id, props)
%
\end{lyxcode}%
%
\item[Description:]%
The generated file may take an extension according to chosen format.
%%
\item[Parameters:]~
\begin{description}%
\item[\texttt{a\_plot}:]
 A plot\_abstract ready to be visualized.
\item[\texttt{caption}:]
 Long caption to appear under the figure.
\item[\texttt{plot\_filename}:]
  Filename to be generated from the plot.
\item[\texttt{float\_props}:]
 Formatting instructions passed to TeXtable. 
\item[\texttt{id}:]
 An identifying string.
\item[\texttt{props}:]
 A structure with any optional properties.
\begin{description}%
\item[\texttt{orient}:]
 Passed to the orient command before printing to figure file.
\end{description}%
\end{description}%
%
\item[Returns a structure object with the following fields:]~

	plot, caption, plot\_filename, float\_props, doc\_generate.
%
\item[Example:]~
\begin{lyxcode}   >> a\_doc = doc\_plot(plotData(my\_cip\_trace), 'My CIP trace. Very interesting.', ...\\%
                       'trace1', struct, 'first doc');\\%
   >> printTeXFile(a\_doc, 'my\_doc.tex'); % it will pop-up the figure now\\%
\end{lyxcode}
%
\item[See also:]%
\hyperlink{ref_doc_generate}{\texttt{doc\_generate}}%
\ (p.~\pageref{ref_doc_generate})%
\index[funcref]{@\fidxl{doc\_generate}}%
, \hyperlink{ref_TeXtable}{\texttt{TeXtable}}%
\ (p.~\pageref{ref_TeXtable})%
\index[funcref]{@\fidxl{TeXtable}}%
%
\item[Author:]%
Cengiz Gunay <cgunay@emory.edu>, 2006/01/17%
\end{description}
\methodline%
\subsubsection[Method \texttt{get}]{Method \texttt{doc\_plot/get}}%
\index[funcref]{doc_plot@\fidxlb{doc\_plot}!get@\fidxl{get}}%
\label{ref_doc_plot__get}%
\hypertarget{ref_doc_plot__get}{}%
\begin{description}
\item[Summary:]Defines generic attribute retrieval for objects.
%
%
%
%
%
%
%
\item[Author:]%
Cengiz Gunay <cgunay@emory.edu>, 2004/09/14%
\end{description}
\methodline%
\subsubsection[Method \texttt{set}]{Method \texttt{doc\_plot/set}}%
\index[funcref]{doc_plot@\fidxlb{doc\_plot}!set@\fidxl{set}}%
\label{ref_doc_plot__set}%
\hypertarget{ref_doc_plot__set}{}%
\begin{description}
\item[Summary:]Generic method for setting object attributes.
%
%
%
%
%
%
%
\item[Author:]%
Cengiz Gunay <cgunay@emory.edu>, 2004/10/08%
\end{description}
\methodline%
\subsubsection[Method \texttt{plot}]{Method \texttt{doc\_plot/plot}}%
\index[funcref]{doc_plot@\fidxlb{doc\_plot}!plot@\fidxl{plot}}%
\label{ref_doc_plot__plot}%
\hypertarget{ref_doc_plot__plot}{}%
\begin{description}
\item[Summary:]Default plot method to preview the contained plot in a new figure.
%
\item[Usage:]~%
\begin{lyxcode}%
figure\_handle = plot(a\_doc, props)
%
\end{lyxcode}%
%
\item[Description:]%
Only generate the contained plot for previewing. Opens a new figure.
%%
\item[Parameters:]~
\begin{description}%
\item[\texttt{a\_doc}:]
 A doc\_plot object.
\item[\texttt{props}:]
 A structure with any optional properties.
\end{description}%
%
\item[Returns:]~

	figure\_handle: Handle of newly opened figure.
%
\item[Example:]~
\begin{lyxcode}        >> figure\_handle = plot(a\_doc\_plot)\\%
\end{lyxcode}
%
\item[See also:]%
\hyperlink{ref_plot_abstract__plotFigure}{\texttt{plot\_abstract/plotFigure}}%
\ (p.~\pageref{ref_plot_abstract__plotFigure})%
\index[funcref]{plot_abstract@\fidxlb{plot\_abstract}!plotFigure@\fidxl{plotFigure}}%
, \hyperlink{ref_doc_generate}{\texttt{doc\_generate}}%
\ (p.~\pageref{ref_doc_generate})%
\index[funcref]{@\fidxl{doc\_generate}}%
, \hyperlink{ref_doc_plot}{\texttt{doc\_plot}}%
\ (p.~\pageref{ref_doc_plot})%
\index[funcref]{@\fidxl{doc\_plot}}%
%
\item[Author:]%
Cengiz Gunay <cgunay@emory.edu>, 2006/01/17%
\end{description}
\methodline%
\subsubsection[Method \texttt{getTeXString}]{Method \texttt{doc\_plot/getTeXString}}%
\index[funcref]{doc_plot@\fidxlb{doc\_plot}!getTeXString@\fidxl{getTeXString}}%
\label{ref_doc_plot__getTeXString}%
\hypertarget{ref_doc_plot__getTeXString}{}%
\begin{description}
\item[Summary:]Returns the TeX representation for the plot document.
%
\item[Usage:]~%
\begin{lyxcode}%
tex\_string = getTeXString(a\_doc, props)
%
\end{lyxcode}%
%
\item[Description:]%
Plots, prints EPS files and generates the necessary LaTeX code.
%%
\item[Parameters:]~
\begin{description}%
\item[\texttt{a\_doc}:]
 A doc\_plot object.
\item[\texttt{props}:]
 A structure with any optional properties.
\end{description}%
%
\item[Returns:]~

	tex\_string: A string that contains TeX commands, which upon writing to a file,
	  can be interpreted by the TeX engine to produce a document.
%
\item[Example:]~
\begin{lyxcode}        doc\_plot has an overloaded getTeXString method:\\%
        >> tex\_string = getTeXString(a\_doc\_plot)\\%
        >> string2File(tex\_string, 'my\_doc.tex')\\%
        then my\_doc.tex can be used by including from a valid LaTeX document.\\%
\end{lyxcode}
%
\item[See also:]%
\hyperlink{ref_doc_generate}{\texttt{doc\_generate}}%
\ (p.~\pageref{ref_doc_generate})%
\index[funcref]{@\fidxl{doc\_generate}}%
, \hyperlink{ref_doc_plot}{\texttt{doc\_plot}}%
\ (p.~\pageref{ref_doc_plot})%
\index[funcref]{@\fidxl{doc\_plot}}%
%
\item[Author:]%
Cengiz Gunay <cgunay@emory.edu>, 2006/01/17%
\end{description}
\methodline%
\subsection{Class \texttt{histogram\_db}}%
\index[funcref]{histogram_db@\fidxlb{histogram\_db}}%
\label{ref_histogram_db}%
\hypertarget{ref_histogram_db}{}%
\subsubsection[Constructor \texttt{histogram\_db}]{Constructor \texttt{histogram\_db/histogram\_db}}%
\index[funcref]{histogram_db@\fidxlb{histogram\_db}!histogram_db@\fidxl{histogram\_db}}%
\label{ref_histogram_db__histogram_db}%
\hypertarget{ref_histogram_db__histogram_db}{}%
\begin{description}
\item[Summary:]A database of histogram values generated for 
		a column of another database.
%
\item[Usage:]~%
\begin{lyxcode}%
a\_hist\_db = histogram\_db(col\_name, bins, hist\_results, id, props)
%
\end{lyxcode}%
%
\item[Description:]%
This is a subclass of tests\_db. Allows generating a histogram plot, etc.
%%
\item[Parameters:]~
\begin{description}%
\item[\texttt{col\_name}:]
 The column name of the histogrammed value.
\item[\texttt{bins}:]
 The values for which the histogram values are calculated.
\item[\texttt{hist\_results}:]
 A column vector of histogram values.
\item[\texttt{id}:]
 An identifying string.
\item[\texttt{props}:]
 A structure with any optional properties.
\end{description}%
%
\item[Returns a structure object with the following fields:]~

	tests\_db, props.
%
%
\item[See also:]%
\hyperlink{ref_tests_db}{\texttt{tests\_db}}%
\ (p.~\pageref{ref_tests_db})%
\index[funcref]{@\fidxl{tests\_db}}%
, \hyperlink{ref_plot_simple}{\texttt{plot\_simple}}%
\ (p.~\pageref{ref_plot_simple})%
\index[funcref]{@\fidxl{plot\_simple}}%
, \hyperlink{ref_tests_db__histogram}{\texttt{tests\_db/histogram}}%
\ (p.~\pageref{ref_tests_db__histogram})%
\index[funcref]{tests_db@\fidxlb{tests\_db}!histogram@\fidxl{histogram}}%
%
\item[Author:]%
Cengiz Gunay <cgunay@emory.edu>, 2004/09/20%
\end{description}
\methodline%
\subsubsection[Method \texttt{get}]{Method \texttt{histogram\_db/get}}%
\index[funcref]{histogram_db@\fidxlb{histogram\_db}!get@\fidxl{get}}%
\label{ref_histogram_db__get}%
\hypertarget{ref_histogram_db__get}{}%
\begin{description}
\item[Summary:]Defines generic attribute retrieval for objects.
%
%
%
%
%
%
%
\item[Author:]%
Cengiz Gunay <cgunay@emory.edu>, 2004/09/14%
\end{description}
\methodline%
\subsubsection[Method \texttt{calcMode}]{Method \texttt{histogram\_db/calcMode}}%
\index[funcref]{histogram_db@\fidxlb{histogram\_db}!calcMode@\fidxl{calcMode}}%
\label{ref_histogram_db__calcMode}%
\hypertarget{ref_histogram_db__calcMode}{}%
\begin{description}
%
\item[Usage:]~%
\begin{lyxcode}%
[mode\_val, mode\_mag] = mode(a\_hist\_db)
%
\end{lyxcode}%
%
%
\item[Parameters:]~
\begin{description}%
\item[\texttt{a\_hist\_db}:]
 A histogram\_db object.
\end{description}%
%
\item[Returns:]~

	mode\_val: The center of the bin that has most members.
	mode\_mag: The value of the histogram bin.
%
%
\item[See also:]%
\hyperlink{ref_histogram_db}{\texttt{histogram\_db}}%
\ (p.~\pageref{ref_histogram_db})%
\index[funcref]{@\fidxl{histogram\_db}}%
%
\item[Author:]%
Cengiz Gunay <cgunay@emory.edu>, 2005/04/27%
\end{description}
\methodline%
\subsubsection[Method \texttt{plot\_abstract}]{Method \texttt{histogram\_db/plot\_abstract}}%
\index[funcref]{histogram_db@\fidxlb{histogram\_db}!plot_abstract@\fidxl{plot\_abstract}}%
\label{ref_histogram_db__plot_abstract}%
\hypertarget{ref_histogram_db__plot_abstract}{}%
\begin{description}
\item[Summary:]Generates a plottable description of this object.
%
\item[Usage:]~%
\begin{lyxcode}%
a\_plot = plot\_abstract(a\_hist\_db, title\_str, props)
%
\end{lyxcode}%
%
\item[Description:]%
Generates a plot\_simple object from this histogram.
%%
\item[Parameters:]~
\begin{description}%
\item[\texttt{a\_hist\_db}:]
 A histogram\_db object.
\item[\texttt{props}:]
 Optional properties passed to plot\_abstract.
\begin{description}%
\item[\texttt{command}:]
 Plot command (Optional, default='bar')
\item[\texttt{quiet}:]
 If 1, don't include database name on title.
\end{description}%
\end{description}%
%
\item[Returns:]~

	a\_plot: A object of plot\_abstract or one of its subclasses.
%
%
\item[See also:]%
\hyperlink{ref_plot_abstract}{\texttt{plot\_abstract}}%
\ (p.~\pageref{ref_plot_abstract})%
\index[funcref]{@\fidxl{plot\_abstract}}%
, \hyperlink{ref_plot_simple}{\texttt{plot\_simple}}%
\ (p.~\pageref{ref_plot_simple})%
\index[funcref]{@\fidxl{plot\_simple}}%
%
\item[Author:]%
Cengiz Gunay <cgunay@emory.edu>, 2004/09/22%
\end{description}
\methodline%
\subsubsection[Method \texttt{subsref}]{Method \texttt{histogram\_db/subsref}}%
\index[funcref]{histogram_db@\fidxlb{histogram\_db}!subsref@\fidxl{subsref}}%
\label{ref_histogram_db__subsref}%
\hypertarget{ref_histogram_db__subsref}{}%
\begin{description}
\item[Summary:]Defines generic indexing for objects.
%
%
%
%
%
%
%
%
\end{description}
\methodline%
\subsubsection[Method \texttt{plotPages}]{Method \texttt{histogram\_db/plotPages}}%
\index[funcref]{histogram_db@\fidxlb{histogram\_db}!plotPages@\fidxl{plotPages}}%
\label{ref_histogram_db__plotPages}%
\hypertarget{ref_histogram_db__plotPages}{}%
\begin{description}
\item[Summary:]Generates a plot containing subplots of histograms in each page.
%
\item[Usage:]~%
\begin{lyxcode}%
a\_plot = plotPages(a\_hist\_db, command, an\_orient)
%
\end{lyxcode}%
%
\item[Description:]%
For each page of the histogram, a histogram is placed in a subplot.
%%
\item[Parameters:]~
\begin{description}%
\item[\texttt{a\_hist\_db}:]
 A histogram\_db object.
\item[\texttt{command}:]
 Plot command (Optional, default='bar')
\item[\texttt{an\_orient}:]
 Stack orientation. One of 'x', 'y', or 'z'.
\end{description}%
%
\item[Returns:]~

	a\_plot: A object of plot\_abstract or one of its subclasses.
%
%
\item[See also:]%
\hyperlink{ref_plotPages}{\texttt{plotPages}}%
\ (p.~\pageref{ref_plotPages})%
\index[funcref]{@\fidxl{plotPages}}%
, \hyperlink{ref_plot_simple}{\texttt{plot\_simple}}%
\ (p.~\pageref{ref_plot_simple})%
\index[funcref]{@\fidxl{plot\_simple}}%
%
\item[Author:]%
Cengiz Gunay <cgunay@emory.edu>, 2004/10/04%
\end{description}
\methodline%
\subsubsection[Method \texttt{plotEqSpaced}]{Method \texttt{histogram\_db/plotEqSpaced}}%
\index[funcref]{histogram_db@\fidxlb{histogram\_db}!plotEqSpaced@\fidxl{plotEqSpaced}}%
\label{ref_histogram_db__plotEqSpaced}%
\hypertarget{ref_histogram_db__plotEqSpaced}{}%
\begin{description}
\item[Summary:]Generates a histogram plot where the values are equally spaced on the x-axis. For use with non-linear parameter values.
%
\item[Usage:]~%
\begin{lyxcode}%
a\_plot = plotEqSpaced(a\_hist\_db, command, props)
%
\end{lyxcode}%
%
\item[Description:]%
Generates a plot\_simple object from this histogram.
%%
\item[Parameters:]~
\begin{description}%
\item[\texttt{a\_hist\_db}:]
 A histogram\_db object.
\item[\texttt{command}:]
 Plot command (Optional, default='bar')
\item[\texttt{props}:]
 Optional properties passed to plot\_abstract.
\end{description}%
%
\item[Returns:]~

	a\_plot: A object of plot\_abstract or one of its subclasses.
%
%
\item[See also:]%
\hyperlink{ref_plot_abstract}{\texttt{plot\_abstract}}%
\ (p.~\pageref{ref_plot_abstract})%
\index[funcref]{@\fidxl{plot\_abstract}}%
, \hyperlink{ref_plot_simple}{\texttt{plot\_simple}}%
\ (p.~\pageref{ref_plot_simple})%
\index[funcref]{@\fidxl{plot\_simple}}%
%
\item[Author:]%
Cengiz Gunay <cgunay@emory.edu>, 2004/09/22%
\end{description}
\methodline%
\subsection{Class \texttt{model\_ct\_bundle}}%
\index[funcref]{model_ct_bundle@\fidxlb{model\_ct\_bundle}}%
\label{ref_model_ct_bundle}%
\hypertarget{ref_model_ct_bundle}{}%
\subsubsection[Constructor \texttt{model\_ct\_bundle}]{Constructor \texttt{model\_ct\_bundle/model\_ct\_bundle}}%
\index[funcref]{model_ct_bundle@\fidxlb{model\_ct\_bundle}!model_ct_bundle@\fidxl{model\_ct\_bundle}}%
\label{ref_model_ct_bundle__model_ct_bundle}%
\hypertarget{ref_model_ct_bundle__model_ct_bundle}{}%
\begin{description}
\item[Summary:]The model cip\_trace dataset and the DB created from it bundled together.
%
\item[Usage:]~%
\begin{lyxcode}%
a\_bundle = model\_ct\_bundle(a\_dataset, a\_db, a\_joined\_db, props)
%
\end{lyxcode}%
%
\item[Description:]%
This is a subclass of dataset\_db\_bundle, specialized for model datasets. 
%%
\item[Parameters:]~
\begin{description}%
\item[\texttt{a\_dataset}:]
 A params\_cip\_trace\_fileset object.
\item[\texttt{a\_db}:]
 The raw params\_tests\_db object created from the dataset. It only needs

to have the pAcip, trial, and ItemIndex columns.\item[\texttt{a\_joined\_db}:]
 The one-model-per-line DB created from the raw DB.
\item[\texttt{props}:]
 A structure with any optional properties.
\end{description}%
%
\item[Returns a structure object with the following fields:]~

	dataset\_db\_bundle.
%
%
\item[See also:]%
\hyperlink{ref_dataset_db_bundle}{\texttt{dataset\_db\_bundle}}%
\ (p.~\pageref{ref_dataset_db_bundle})%
\index[funcref]{@\fidxl{dataset\_db\_bundle}}%
, \hyperlink{ref_tests_db}{\texttt{tests\_db}}%
\ (p.~\pageref{ref_tests_db})%
\index[funcref]{@\fidxl{tests\_db}}%
, \hyperlink{ref_params_tests_dataset}{\texttt{params\_tests\_dataset}}%
\ (p.~\pageref{ref_params_tests_dataset})%
\index[funcref]{@\fidxl{params\_tests\_dataset}}%
%
\item[Author:]%
Cengiz Gunay <cgunay@emory.edu>, 2005/12/13%
\end{description}
\methodline%
\subsubsection[Method \texttt{getNeuronLabel}]{Method \texttt{model\_ct\_bundle/getNeuronLabel}}%
\index[funcref]{model_ct_bundle@\fidxlb{model\_ct\_bundle}!getNeuronLabel@\fidxl{getNeuronLabel}}%
\label{ref_model_ct_bundle__getNeuronLabel}%
\hypertarget{ref_model_ct_bundle__getNeuronLabel}{}%
\begin{description}
\item[Summary:]Constructs the neuron label from bundle.
%
\item[Usage:]~%
\begin{lyxcode}%
a\_label = getNeuronLabel(a\_bundle, trial\_num, props)
%
\end{lyxcode}%
%
%
\item[Parameters:]~
\begin{description}%
\item[\texttt{a\_bundle}:]
 A physiol\_cip\_traceset\_fileset object.
\item[\texttt{trial\_num}:]
 The trial number of model neuron.
\item[\texttt{props}:]
 A structure with any optional properties.
\end{description}%
%
\item[Returns:]~

	a\_label: A string label identifying selected neuron in bundle.
%
%
\item[See also:]%
\hyperlink{ref_dataset_db_bundle}{\texttt{dataset\_db\_bundle}}%
\ (p.~\pageref{ref_dataset_db_bundle})%
\index[funcref]{@\fidxl{dataset\_db\_bundle}}%
%
\item[Author:]%
Cengiz Gunay <cgunay@emory.edu>, 2006/05/26%
\end{description}
\methodline%
\subsubsection[Method \texttt{reportCompareModelToPhysiolNeuron}]{Method \texttt{model\_ct\_bundle/reportCompareModelToPhysiolNeuron}}%
\index[funcref]{model_ct_bundle@\fidxlb{model\_ct\_bundle}!reportCompareModelToPhysiolNeuron@\fidxl{reportCompareModelToPhysiolNeuron}}%
\label{ref_model_ct_bundle__reportCompareModelToPhysiolNeuron}%
\hypertarget{ref_model_ct_bundle__reportCompareModelToPhysiolNeuron}{}%
\begin{description}
\item[Summary:]Generates a report by comparing given model neuron to given physiol neuron.
%
\item[Usage:]~%
\begin{lyxcode}%
a\_doc\_multi = reportCompareModelToPhysiolNeuron(m\_bundle, trial\_num, p\_bundle, 
						  traceset\_index, props)
%
\end{lyxcode}%
%
\item[Description:]%
Generates a report document with:
	- Figure displaying raw traces of the physiol neuron compared with the model neuron
	- Figure comparing f-I curves of the two neurons.
	- Figure comparing spont and pulse spike shapes of the two neurons.
%%
\item[Parameters:]~
\begin{description}%
\item[\texttt{m\_bundle, p\_bundle}:]
 dataset\_db\_bundle objects of the model and physiology neurons.
\item[\texttt{trial\_num}:]
 Trial number of desired model neuron in m\_bundle.
\item[\texttt{traceset\_index}:]
 TracesetIndex of desired neuron in p\_bundle.
\item[\texttt{props}:]
 A structure with any optional properties.
\begin{description}%
\item[\texttt{horizRow}:]
 If defined, create a row-figure with all plots.
\item[\texttt{numPhysTraces}:]
 Number of physiology traces to show in plot (>=1).
\end{description}%
\end{description}%
%
\item[Returns:]~

	a\_doc\_multi: A doc\_multi object that can be printed as a PS or PDF file.
%
\item[Example:]~
\begin{lyxcode} >> printTeXFile(reportCompareModelToPhysiolNeuron(mbundle, 2222, pbundle, 34), 'a.tex')\\%
\end{lyxcode}
%
\item[See also:]%
\hyperlink{ref_doc_multi}{\texttt{doc\_multi}}%
\ (p.~\pageref{ref_doc_multi})%
\index[funcref]{@\fidxl{doc\_multi}}%
, \hyperlink{ref_doc_generate}{\texttt{doc\_generate}}%
\ (p.~\pageref{ref_doc_generate})%
\index[funcref]{@\fidxl{doc\_generate}}%
, \hyperlink{ref_doc_generate__printTeXFile}{\texttt{doc\_generate/printTeXFile}}%
\ (p.~\pageref{ref_doc_generate__printTeXFile})%
\index[funcref]{doc_generate@\fidxlb{doc\_generate}!printTeXFile@\fidxl{printTeXFile}}%
%
\item[Author:]%
Cengiz Gunay <cgunay@emory.edu>, 2006/01/24%
\end{description}
\methodline%
\subsubsection[Method \texttt{plotCompareRanks}]{Method \texttt{model\_ct\_bundle/plotCompareRanks}}%
\index[funcref]{model_ct_bundle@\fidxlb{model\_ct\_bundle}!plotCompareRanks@\fidxl{plotCompareRanks}}%
\label{ref_model_ct_bundle__plotCompareRanks}%
\hypertarget{ref_model_ct_bundle__plotCompareRanks}{}%
\begin{description}
\item[Summary:]Generates a plots of given ranks from the ranked\_bundle.
%
\item[Usage:]~%
\begin{lyxcode}%
plots = plotCompareRanks(m\_bundle, p\_bundle, a\_ranked\_db, ranks, props)
%
\end{lyxcode}%
%
%
\item[Parameters:]~
\begin{description}%
\item[\texttt{m\_bundle}:]
 A model\_ct\_bundle object.
\item[\texttt{p\_bundle}:]
 A dataset\_db\_bundle object that originated the criterion.
\item[\texttt{a\_ranked\_db}:]
 A ranked\_db generated from ranking m\_bundle.
\item[\texttt{ranks}:]
 Vector of rank indices for which to generate the plots.
\item[\texttt{props}:]
 A structure with any optional properties.
\end{description}%
%
\item[Returns:]~

	plots: A structure that contains the joined\_db, and the plot vectors 
	  trace\_d100\_plots and trace\_h100\_plots.
%
\item[Example:]~
\begin{lyxcode} >> plots = plotCompareRanks(r, 1:10);\\%
 >> plotFigure(plots.trace\_d100\_plots(1), 'The best matching +100 pA CIP trace');\\%
\end{lyxcode}
%
\item[See also:]%
%
\item[Author:]%
Cengiz Gunay <cgunay@emory.edu>, 2006/01/16%
\end{description}
\methodline%
\subsubsection[Method \texttt{get}]{Method \texttt{model\_ct\_bundle/get}}%
\index[funcref]{model_ct_bundle@\fidxlb{model\_ct\_bundle}!get@\fidxl{get}}%
\label{ref_model_ct_bundle__get}%
\hypertarget{ref_model_ct_bundle__get}{}%
\begin{description}
\item[Summary:]Defines generic attribute retrieval for objects.
%
%
%
%
%
%
%
\item[Author:]%
Cengiz Gunay <cgunay@emory.edu>, 2004/09/14%
\end{description}
\methodline%
\subsubsection[Method \texttt{set}]{Method \texttt{model\_ct\_bundle/set}}%
\index[funcref]{model_ct_bundle@\fidxlb{model\_ct\_bundle}!set@\fidxl{set}}%
\label{ref_model_ct_bundle__set}%
\hypertarget{ref_model_ct_bundle__set}{}%
\begin{description}
\item[Summary:]Generic method for setting object attributes.
%
%
%
%
%
%
%
\item[Author:]%
Cengiz Gunay <cgunay@emory.edu>, 2004/10/08%
\end{description}
\methodline%
\subsubsection[Method \texttt{collectPhysiolMatches}]{Method \texttt{model\_ct\_bundle/collectPhysiolMatches}}%
\index[funcref]{model_ct_bundle@\fidxlb{model\_ct\_bundle}!collectPhysiolMatches@\fidxl{collectPhysiolMatches}}%
\label{ref_model_ct_bundle__collectPhysiolMatches}%
\hypertarget{ref_model_ct_bundle__collectPhysiolMatches}{}%
\begin{description}
\item[Summary:]Compare model DB to given physiol criteria and return some top matches.
%
\item[Usage:]~%
\begin{lyxcode}%
row\_index = collectPhysiolMatches(a\_mbundle, a\_crit\_bundle, props)
%
\end{lyxcode}%
%
%
\item[Parameters:]~
\begin{description}%
\item[\texttt{a\_mbundle}:]
 A model\_ct\_bundle object.
\item[\texttt{a\_crit\_bundle}:]
 A physiol\_bundle object that holds the criterion neuron.
\item[\texttt{props}:]
 A structure with any optional properties.
\begin{description}%
\item[\texttt{showTopmost}:]
 Number of top matching models to return (default=50)
\end{description}%
\end{description}%
%
\item[Returns: ]~

	row\_index: Row indices of best matching models.
%
%
\item[See also:]%
%
\item[Author:]%
Cengiz Gunay <cgunay@emory.edu>, 2006/01/18%
\end{description}
\methodline%
\subsubsection[Method \texttt{plotComparefICurve}]{Method \texttt{model\_ct\_bundle/plotComparefICurve}}%
\index[funcref]{model_ct_bundle@\fidxlb{model\_ct\_bundle}!plotComparefICurve@\fidxl{plotComparefICurve}}%
\label{ref_model_ct_bundle__plotComparefICurve}%
\hypertarget{ref_model_ct_bundle__plotComparefICurve}{}%
\begin{description}
\item[Summary:]Generates a f-I curve doc\_plot comparing m\_trial and to\_index.
%
\item[Usage:]~%
\begin{lyxcode}%
a\_plot = plotComparefICurve(m\_bundle, m\_trial, to\_bundle, to\_index, props)
%
\end{lyxcode}%
%
\item[Description:]%
Note that this is not a general method. to\_bundle should have been able to accept
 any type of bundle. Most probably this method is redundant and deprecated.
%%
\item[Parameters:]~
\begin{description}%
\item[\texttt{m\_bundle}:]
 A model\_ct\_bundle object.
\item[\texttt{m\_trial}:]
 Trial number of model.
\item[\texttt{to\_bundle}:]
 A physiol\_bundle object.
\item[\texttt{to\_index}:]
 TracesetIndex of neuron.
\item[\texttt{props}:]
 A structure with any optional properties.
\begin{description}%
\item[\texttt{shortCaption}:]
 This appears in the figure caption.
\item[\texttt{plotMStats}:]
 If set, add the m\_bundle stats plot.
\item[\texttt{plotToStats}:]
 If set, add the to\_bundle stats plot.
\item[\texttt{captionToStats}:]
 Use this as its legend label. 
\item[\texttt{quiet}:]
 if given, no title is produced

(passed to plot\_superpose)\end{description}%
\end{description}%
%
\item[Returns:]~

	a\_plot: A plot\_superpose that contains a f-I curve plot.
%
\item[Example:]~
\begin{lyxcode} >> a\_p = plotComparefICurve(r, 1);\\%
 >> plotFigure(a\_p, 'The f-I curve of best matching model');\\%
\end{lyxcode}
%
\item[See also:]%
\hyperlink{ref_plot_abstract}{\texttt{plot\_abstract}}%
\ (p.~\pageref{ref_plot_abstract})%
\index[funcref]{@\fidxl{plot\_abstract}}%
, \hyperlink{ref_plot_superpose}{\texttt{plot\_superpose}}%
\ (p.~\pageref{ref_plot_superpose})%
\index[funcref]{@\fidxl{plot\_superpose}}%
%
\item[Author:]%
Cengiz Gunay <cgunay@emory.edu>, 2006/01/16%
\end{description}
\methodline%
\subsubsection[Method \texttt{getNeuronRowIndex}]{Method \texttt{model\_ct\_bundle/getNeuronRowIndex}}%
\index[funcref]{model_ct_bundle@\fidxlb{model\_ct\_bundle}!getNeuronRowIndex@\fidxl{getNeuronRowIndex}}%
\label{ref_model_ct_bundle__getNeuronRowIndex}%
\hypertarget{ref_model_ct_bundle__getNeuronRowIndex}{}%
\begin{description}
\item[Summary:]Returns the neuron index from bundle.
%
\item[Usage:]~%
\begin{lyxcode}%
a\_row\_index = getNeuronRowIndex(a\_bundle, trial\_num, props)
%
\end{lyxcode}%
%
%
\item[Parameters:]~
\begin{description}%
\item[\texttt{a\_bundle}:]
 A model\_ct\_bundle object.
\item[\texttt{trial\_num}:]
 The trial number of model neuron, or a DB row containing this.
\item[\texttt{props}:]
 A structure with any optional properties.
\end{description}%
%
\item[Returns:]~

	a\_row\_index: A row index of neuron in a\_bundle.joined\_db.
%
%
\item[See also:]%
\hyperlink{ref_dataset_db_bundle}{\texttt{dataset\_db\_bundle}}%
\ (p.~\pageref{ref_dataset_db_bundle})%
\index[funcref]{@\fidxl{dataset\_db\_bundle}}%
%
\item[Author:]%
Cengiz Gunay <cgunay@emory.edu>, 2006/06/09%
\end{description}
\methodline%
\subsubsection[Method \texttt{ctFromRows}]{Method \texttt{model\_ct\_bundle/ctFromRows}}%
\index[funcref]{model_ct_bundle@\fidxlb{model\_ct\_bundle}!ctFromRows@\fidxl{ctFromRows}}%
\label{ref_model_ct_bundle__ctFromRows}%
\hypertarget{ref_model_ct_bundle__ctFromRows}{}%
\begin{description}
\item[Summary:]Loads a cip\_trace object from a raw data file in the a\_mbundle.
%
\item[Usage:]~%
\begin{lyxcode}%
a\_cip\_trace = ctFromRows(a\_mbundle, a\_db/trials, cip\_levels, props)
%
\end{lyxcode}%
%
%
\item[Parameters:]~
\begin{description}%
\item[\texttt{a\_mbundle}:]
 A physiol\_cip\_traceset\_fileset object.
\item[\texttt{a\_db}:]
 A DB created by this fileset to read the trial numbers from.
\item[\texttt{trials}:]
 A column vector with trial numbers.
\item[\texttt{cip\_levels}:]
 A column vector of CIP-levels to be loaded.
\item[\texttt{props}:]
 A structure with any optional properties.

(passed to a\_mbundle.dataset/cip\_trace)\end{description}%
%
\item[Returns:]~

	a\_cip\_trace: One or more cip\_trace objects that hold the raw data.
%
%
\item[See also:]%
\hyperlink{ref_dataset_db_bundle__ctFromRows}{\texttt{dataset\_db\_bundle/ctFromRows}}%
\ (p.~\pageref{ref_dataset_db_bundle__ctFromRows})%
\index[funcref]{dataset_db_bundle@\fidxlb{dataset\_db\_bundle}!ctFromRows@\fidxl{ctFromRows}}%
%
\item[Author:]%
Cengiz Gunay <cgunay@emory.edu>, 2005/07/13%
\end{description}
\methodline%
\subsubsection[Method \texttt{addToDB}]{Method \texttt{model\_ct\_bundle/addToDB}}%
\index[funcref]{model_ct_bundle@\fidxlb{model\_ct\_bundle}!addToDB@\fidxl{addToDB}}%
\label{ref_model_ct_bundle__addToDB}%
\hypertarget{ref_model_ct_bundle__addToDB}{}%
\begin{description}
\item[Summary:]Caoncatenate to existing DB in the bundle.
%
\item[Usage:]~%
\begin{lyxcode}%
a\_mbundle = addToDB(a\_mbundle, a\_raw\_db, props)
%
\end{lyxcode}%
%
\item[Description:]%
If joinedDb is not given, calls treatSimDB to get the joined\_db from this raw DB. 
 Then concats to both db and joined\_db in bundle.
%%
\item[Parameters:]~
\begin{description}%
\item[\texttt{a\_mbundle}:]
 A model\_ct\_bundle object.
\item[\texttt{a\_crit\_bundle}:]
 A physiol\_bundle having a crit\_db as its joined\_db.
\item[\texttt{props}:]
 A structure with any optional properties.
\begin{description}%
\item[\texttt{joinedDb}:]
 The joined version of a\_raw\_db.
\item[\texttt{dataset}:]
 If given, this one is used to replace the fileset in the bundle.
\end{description}%
\end{description}%
%
\item[Returns:]~

	a\_mbundle: a model\_ct\_bundle object containing the added DB.
%
\item[Example:]~
\begin{lyxcode} >> mbundle = addToDB(mbundle, params\_tests\_db(mfileset, [19684:59956]))\\%
\end{lyxcode}
%
\item[See also:]%
\hyperlink{ref_params_tests_fileset__addFiles}{\texttt{params\_tests\_fileset/addFiles}}%
\ (p.~\pageref{ref_params_tests_fileset__addFiles})%
\index[funcref]{params_tests_fileset@\fidxlb{params\_tests\_fileset}!addFiles@\fidxl{addFiles}}%
, \hyperlink{ref_multi_fileset_gpsim_cns2005__addFileDir}{\texttt{multi\_fileset\_gpsim\_cns2005/addFileDir}}%
\ (p.~\pageref{ref_multi_fileset_gpsim_cns2005__addFileDir})%
\index[funcref]{multi_fileset_gpsim_cns2005@\fidxlb{multi\_fileset\_gpsim\_cns2005}!addFileDir@\fidxl{addFileDir}}%
%
\item[Author:]%
Cengiz Gunay <cgunay@emory.edu>, 2006/02/06%
\end{description}
\methodline%
\subsubsection[Method \texttt{reportRankingToPhysiolNeuronsTeXFile}]{Method \texttt{model\_ct\_bundle/reportRankingToPhysiolNeuronsTeXFile}}%
\index[funcref]{model_ct_bundle@\fidxlb{model\_ct\_bundle}!reportRankingToPhysiolNeuronsTeXFile@\fidxl{reportRankingToPhysiolNeuronsTeXFile}}%
\label{ref_model_ct_bundle__reportRankingToPhysiolNeuronsTeXFile}%
\hypertarget{ref_model_ct_bundle__reportRankingToPhysiolNeuronsTeXFile}{}%
\begin{description}
\item[Summary:]Compare model DB to given physiol criterion and create a report.
%
\item[Usage:]~%
\begin{lyxcode}%
tex\_filename = reportRankingToPhysiolNeuronsTeXFile(m\_bundle, p\_bundle, a\_crit\_db, props)
%
\end{lyxcode}%
%
\item[Description:]%
A LaTeX report is generated 
 following the example in physiol\_bundle/matchingRow. The filename contains the neuron
 name, followed by the traceset index as an identifier of pharmacological applications,
 as in gpd0421c\_s34. 
%%
\item[Parameters:]~
\begin{description}%
\item[\texttt{m\_bundle}:]
 A model\_ct\_bundle object.
\item[\texttt{p\_bundle}:]
 A physiol\_bundle object.
\item[\texttt{a\_crit\_db}:]
 The criterion neuron chosen with a matchingRow method.
\item[\texttt{props}:]
 A structure with any optional properties.
\begin{description}%
\item[\texttt{filenameSuffix}:]
 Append this identifier to the TeX filename.

(others passed to rankMatching)\end{description}%
\end{description}%
%
\item[Returns: ]~

	tex\_filename: Name of LaTeX file generated.
%
%
\item[See also:]%
\hyperlink{ref_loadItemProfile}{\texttt{loadItemProfile}}%
\ (p.~\pageref{ref_loadItemProfile})%
\index[funcref]{@\fidxl{loadItemProfile}}%
, \hyperlink{ref_physiol_cip_traceset__cip_trace}{\texttt{physiol\_cip\_traceset/cip\_trace}}%
\ (p.~\pageref{ref_physiol_cip_traceset__cip_trace})%
\index[funcref]{physiol_cip_traceset@\fidxlb{physiol\_cip\_traceset}!cip_trace@\fidxl{cip\_trace}}%
%
\item[Author:]%
Cengiz Gunay <cgunay@emory.edu>, 2006/01/18%
\end{description}
\methodline%
\subsubsection[Method \texttt{rankMatching}]{Method \texttt{model\_ct\_bundle/rankMatching}}%
\index[funcref]{model_ct_bundle@\fidxlb{model\_ct\_bundle}!rankMatching@\fidxl{rankMatching}}%
\label{ref_model_ct_bundle__rankMatching}%
\hypertarget{ref_model_ct_bundle__rankMatching}{}%
\begin{description}
\item[Summary:]Create a ranked\_db from given criterion db.
%
\item[Usage:]~%
\begin{lyxcode}%
a\_ranked\_db = rankMatching(a\_mbundle, a\_crit\_db, props)
%
\end{lyxcode}%
%
%
\item[Parameters:]~
\begin{description}%
\item[\texttt{a\_mbundle}:]
 A model\_ct\_bundle object.
\item[\texttt{a\_crit\_db}:]
 A crit\_db created by a matchingRow method.
\item[\texttt{props}:]
 A structure with any optional properties.

(passed to tests\_db/rankMatching)\end{description}%
%
\item[Returns:]~

	a\_ranked\_db: a ranked\_db object containing the rankings.
%
%
\item[See also:]%
\hyperlink{ref_tests_db__rankMatching}{\texttt{tests\_db/rankMatching}}%
\ (p.~\pageref{ref_tests_db__rankMatching})%
\index[funcref]{tests_db@\fidxlb{tests\_db}!rankMatching@\fidxl{rankMatching}}%
, \hyperlink{ref_ranked_db}{\texttt{ranked\_db}}%
\ (p.~\pageref{ref_ranked_db})%
\index[funcref]{@\fidxl{ranked\_db}}%
%
\item[Author:]%
Cengiz Gunay <cgunay@emory.edu>, 2006/01/18%
\end{description}
\methodline%
\subsection{Class \texttt{model\_ranked\_to\_physiol\_bundle}}%
\index[funcref]{model_ranked_to_physiol_bundle@\fidxlb{model\_ranked\_to\_physiol\_bundle}}%
\label{ref_model_ranked_to_physiol_bundle}%
\hypertarget{ref_model_ranked_to_physiol_bundle}{}%
\subsubsection[Constructor \texttt{model\_ranked\_to\_physiol\_bundle}]{Constructor \texttt{model\_ranked\_to\_physiol\_bundle/model\_ranked\_to\_physiol\_bundle}}%
\index[funcref]{model_ranked_to_physiol_bundle@\fidxlb{model\_ranked\_to\_physiol\_bundle}!model_ranked_to_physiol_bundle@\fidxl{model\_ranked\_to\_physiol\_bundle}}%
\label{ref_model_ranked_to_physiol_bundle__model_ranked_to_physiol_bundle}%
\hypertarget{ref_model_ranked_to_physiol_bundle__model_ranked_to_physiol_bundle}{}%
\begin{description}
\item[Summary:]A DB bundled with its dataset, ranked to a physiology DB bundle.
%
\item[Usage:]~%
\begin{lyxcode}%
r\_bundle = model\_ranked\_to\_physiol\_bundle(a\_dataset, a\_db, a\_ranked\_db, a\_crit\_bundle, props)
%
\end{lyxcode}%
%
\item[Description:]%
This is a subclass of model\_ct\_bundle, specialized for model datasets. 
%%
\item[Parameters:]~
\begin{description}%
\item[\texttt{a\_dataset}:]
 A params\_cip\_trace\_fileset object.
\item[\texttt{a\_db}:]
 The raw params\_tests\_db object created from the dataset. It only needs

to have the pAcip, trial, and ItemIndex columns.\item[\texttt{a\_ranked\_db}:]
 The one-model-per-line DB created from the raw DB.
\item[\texttt{a\_crit\_bundle}:]
 The bundle object associated with crit\_db that caused the ranking in a\_ranked\_db.
\item[\texttt{props}:]
 A structure with any optional properties.
\end{description}%
%
\item[Returns a structure object with the following fields:]~

	crit\_bundle, model\_ct\_bundle.
%
%
\item[See also:]%
\hyperlink{ref_model_ct_bundle}{\texttt{model\_ct\_bundle}}%
\ (p.~\pageref{ref_model_ct_bundle})%
\index[funcref]{@\fidxl{model\_ct\_bundle}}%
, \hyperlink{ref_ranked_db}{\texttt{ranked\_db}}%
\ (p.~\pageref{ref_ranked_db})%
\index[funcref]{@\fidxl{ranked\_db}}%
, \hyperlink{ref_params_tests_dataset}{\texttt{params\_tests\_dataset}}%
\ (p.~\pageref{ref_params_tests_dataset})%
\index[funcref]{@\fidxl{params\_tests\_dataset}}%
%
\item[Author:]%
Cengiz Gunay <cgunay@emory.edu>, 2005/12/13%
\end{description}
\methodline%
\subsubsection[Method \texttt{plotCompareRanks}]{Method \texttt{model\_ranked\_to\_physiol\_bundle/plotCompareRanks}}%
\index[funcref]{model_ranked_to_physiol_bundle@\fidxlb{model\_ranked\_to\_physiol\_bundle}!plotCompareRanks@\fidxl{plotCompareRanks}}%
\label{ref_model_ranked_to_physiol_bundle__plotCompareRanks}%
\hypertarget{ref_model_ranked_to_physiol_bundle__plotCompareRanks}{}%
\begin{description}
\item[Summary:]OBSOLETE - Generates a plots of given ranks from the ranked\_bundle.
%
\item[Usage:]~%
\begin{lyxcode}%
plots = plotCompareRanks(r\_bundle, crit\_bundle, crit\_db, props)
%
\end{lyxcode}%
%
%
\item[Parameters:]~
\begin{description}%
\item[\texttt{r\_bundle}:]
 A ranked\_bundle object.
\item[\texttt{ranks}:]
 Vector of rank indices for which to generate the plots.
\item[\texttt{props}:]
 A structure with any optional properties.
\end{description}%
%
\item[Returns:]~

	plots: A structure that contains the joined\_db, and the plot vectors 
	  trace\_d100\_plots and trace\_h100\_plots.
%
\item[Example:]~
\begin{lyxcode} >> plots = plotCompareRanks(r, 1:10);\\%
 >> plotFigure(plots.trace\_d100\_plots(1), 'The best matching +100 pA CIP trace');\\%
\end{lyxcode}
%
\item[See also:]%
%
\item[Author:]%
Cengiz Gunay <cgunay@emory.edu>, 2006/01/16%
\end{description}
\methodline%
\subsubsection[Method \texttt{plotfICurve}]{Method \texttt{model\_ranked\_to\_physiol\_bundle/plotfICurve}}%
\index[funcref]{model_ranked_to_physiol_bundle@\fidxlb{model\_ranked\_to\_physiol\_bundle}!plotfICurve@\fidxl{plotfICurve}}%
\label{ref_model_ranked_to_physiol_bundle__plotfICurve}%
\hypertarget{ref_model_ranked_to_physiol_bundle__plotfICurve}{}%
\begin{description}
%
\item[Usage:]~%
\begin{lyxcode}%
a\_doc = docfICurve(r\_bundle, crit\_bundle, crit\_db, props)
%
\end{lyxcode}%
%
%
\item[Parameters:]~
\begin{description}%
\item[\texttt{r\_bundle}:]
 A ranked\_bundle object.
\item[\texttt{rank\_num}:]
 Rank index for which to generate the a\_doc.
\item[\texttt{props}:]
 A structure with any optional properties.
\end{description}%
%
\item[Returns:]~

	a\_doc: A doc\_plot that contains a f-I curve plot and associated captions.
%
\item[Example:]~
\begin{lyxcode} >> a\_d = docfICurve(r, 1);\\%
 >> plot(a\_d, 'The f-I curve of best matching model');\\%
\end{lyxcode}
%
\item[See also:]%
\hyperlink{ref_doc_generate}{\texttt{doc\_generate}}%
\ (p.~\pageref{ref_doc_generate})%
\index[funcref]{@\fidxl{doc\_generate}}%
, \hyperlink{ref_doc_plot}{\texttt{doc\_plot}}%
\ (p.~\pageref{ref_doc_plot})%
\index[funcref]{@\fidxl{doc\_plot}}%
%
\item[Author:]%
Cengiz Gunay <cgunay@emory.edu>, 2006/01/16%
\end{description}
\methodline%
\subsubsection[Method \texttt{comparisonReport}]{Method \texttt{model\_ranked\_to\_physiol\_bundle/comparisonReport}}%
\index[funcref]{model_ranked_to_physiol_bundle@\fidxlb{model\_ranked\_to\_physiol\_bundle}!comparisonReport@\fidxl{comparisonReport}}%
\label{ref_model_ranked_to_physiol_bundle__comparisonReport}%
\hypertarget{ref_model_ranked_to_physiol_bundle__comparisonReport}{}%
\begin{description}
\item[Summary:]OBSOLETE - Generates a report by comparing r\_bundle with the given match criteria, crit\_db from crit\_bundle.
%
\item[Usage:]~%
\begin{lyxcode}%
a\_doc\_multi = comparisonReport(r\_bundle, crit\_bundle, crit\_db, props)
%
\end{lyxcode}%
%
\item[Description:]%
Generates a LaTeX document with:
	- (optional) Raw traces compared with some best matches at different distances
	- Values of some top matching a\_db rows and match errors in a floating table.
	- colored-plot of measure errors for some top matches.
	- Parameter distributions of 50 best matches as a bar graph.
%%
\item[Parameters:]~
\begin{description}%
\item[\texttt{r\_bundle}:]
 A dataset\_db\_bundle object that contains the DB to compare rows from.
\item[\texttt{crit\_bundle}:]
 A dataset\_db\_bundle object that contains the criterion dataset.
\item[\texttt{crit\_db}:]
 A tests\_db object holding the match criterion tests and STDs

which can be created with matchingRow.\item[\texttt{props}:]
 A structure with any optional properties.
\begin{description}%
\item[\texttt{caption}:]
 Identification of the criterion db (not needed/used?).
\item[\texttt{num\_matches}:]
 Number of best matches to display (default=10).
\item[\texttt{rotate}:]
 Rotation angle for best matches table (default=90).
\end{description}%
\end{description}%
%
\item[Returns:]~

	tex\_string: LaTeX document string.
%
%
\item[See also:]%
\hyperlink{ref_displayRowsTeX}{\texttt{displayRowsTeX}}%
\ (p.~\pageref{ref_displayRowsTeX})%
\index[funcref]{@\fidxl{displayRowsTeX}}%
%
\item[Author:]%
Cengiz Gunay <cgunay@emory.edu>, 2006/01/17%
\end{description}
\methodline%
\subsection{Class \texttt{params\_cip\_trace\_fileset}}%
\index[funcref]{params_cip_trace_fileset@\fidxlb{params\_cip\_trace\_fileset}}%
\label{ref_params_cip_trace_fileset}%
\hypertarget{ref_params_cip_trace_fileset}{}%
\subsubsection[Constructor \texttt{params\_cip\_trace\_fileset}]{Constructor \texttt{params\_cip\_trace\_fileset/params\_cip\_trace\_fileset}}%
\index[funcref]{params_cip_trace_fileset@\fidxlb{params\_cip\_trace\_fileset}!params_cip_trace_fileset@\fidxl{params\_cip\_trace\_fileset}}%
\label{ref_params_cip_trace_fileset__params_cip_trace_fileset}%
\hypertarget{ref_params_cip_trace_fileset__params_cip_trace_fileset}{}%
\begin{description}
\item[Summary:]Description of a raw dataset consisting of cip\_trace files varying 
	with parameter values.
%
\item[Usage:]~%
\begin{lyxcode}%
obj = params\_cip\_trace\_fileset(file\_pattern, dt, dy, 
				 pulse\_time\_start, pulse\_time\_width, id, props)
%
\end{lyxcode}%
%
\item[Description:]%
This is a subclass of params\_tests\_fileset.
%%
\item[Parameters:]~
\begin{description}%
\item[\texttt{file\_pattern}:]
 File pattern mathing all files to be loaded.
\item[\texttt{dt}:]
 Time resolution [s]
\item[\texttt{dy}:]
 y-axis resolution [ISI (V, A, etc.)]
\item[\texttt{pulse\_time\_start, pulse\_time\_width}:]


Start and width of the pulse [dt]\item[\texttt{id}:]
 An identification string
\item[\texttt{props}:]
 A structure with any optional properties.
\begin{description}%
\item[\texttt{profile\_class\_name}:]
 Use this profile class (Default: 'cip\_trace\_profile').

(All other props are passed to cip\_trace objects)\end{description}%
\end{description}%
%
\item[Returns a structure object with the following fields:]~

	params\_tests\_fileset,
	pulse\_time\_start, pulse\_time\_width.
%
\item[Example:]~
\begin{lyxcode}        >> fileset = params\_cip\_trace\_fileset('/home/abc/data/*.bin', 1e-4, 1e-3, 20001, 10000, 'sim dataset gpsc0501', struct('trace\_time\_start', 10001, 'type', 'sim', 'scale\_y', 1e3))\\%
\end{lyxcode}
%
\item[See also:]%
\hyperlink{ref_params_tests_fileset}{\texttt{params\_tests\_fileset}}%
\ (p.~\pageref{ref_params_tests_fileset})%
\index[funcref]{@\fidxl{params\_tests\_fileset}}%
, \hyperlink{ref_params_tests_db}{\texttt{params\_tests\_db}}%
\ (p.~\pageref{ref_params_tests_db})%
\index[funcref]{@\fidxl{params\_tests\_db}}%
%
\item[Author:]%
Cengiz Gunay <cgunay@emory.edu>, 2004/09/14%
\end{description}
\methodline%
\subsubsection[Method \texttt{display}]{Method \texttt{params\_cip\_trace\_fileset/display}}%
\index[funcref]{params_cip_trace_fileset@\fidxlb{params\_cip\_trace\_fileset}!display@\fidxl{display}}%
\label{ref_params_cip_trace_fileset__display}%
\hypertarget{ref_params_cip_trace_fileset__display}{}%
\begin{description}
%
%
%
%
%
%
%
\item[Author:]%
Cengiz Gunay <cgunay@emory.edu>, 2004/08/04%
\end{description}
\methodline%
\subsubsection[Method \texttt{get}]{Method \texttt{params\_cip\_trace\_fileset/get}}%
\index[funcref]{params_cip_trace_fileset@\fidxlb{params\_cip\_trace\_fileset}!get@\fidxl{get}}%
\label{ref_params_cip_trace_fileset__get}%
\hypertarget{ref_params_cip_trace_fileset__get}{}%
\begin{description}
\item[Summary:]Defines generic attribute retrieval for objects.
%
%
%
%
%
%
%
\item[Author:]%
Cengiz Gunay <cgunay@emory.edu>, 2004/09/14%
\end{description}
\methodline%
\subsubsection[Method \texttt{cip\_trace\_profile}]{Method \texttt{params\_cip\_trace\_fileset/cip\_trace\_profile}}%
\index[funcref]{params_cip_trace_fileset@\fidxlb{params\_cip\_trace\_fileset}!cip_trace_profile@\fidxl{cip\_trace\_profile}}%
\label{ref_params_cip_trace_fileset__cip_trace_profile}%
\hypertarget{ref_params_cip_trace_fileset__cip_trace_profile}{}%
\begin{description}
\item[Summary:]Loads a raw cip\_trace\_profile given a file\_index 
		      to this fileset.
%
\item[Usage:]~%
\begin{lyxcode}%
a\_cip\_trace\_profile = cip\_trace\_profile(fileset, file\_index)
%
\end{lyxcode}%
%
%
\item[Parameters:]~
\begin{description}%
\item[\texttt{fileset}:]
 A params\_tests\_fileset.
\item[\texttt{file\_index}:]
 Index of file in fileset.
\end{description}%
%
\item[Returns:]~

	a\_cip\_trace\_profile: A cip\_trace\_profile object.
%
%
\item[See also:]%
\hyperlink{ref_cip_trace_profile}{\texttt{cip\_trace\_profile}}%
\ (p.~\pageref{ref_cip_trace_profile})%
\index[funcref]{@\fidxl{cip\_trace\_profile}}%
, \hyperlink{ref_params_tests_fileset}{\texttt{params\_tests\_fileset}}%
\ (p.~\pageref{ref_params_tests_fileset})%
\index[funcref]{@\fidxl{params\_tests\_fileset}}%
%
\item[Author:]%
Cengiz Gunay <cgunay@emory.edu>, 2004/09/14%
\end{description}
\methodline%
\subsubsection[Method \texttt{loadItemProfile}]{Method \texttt{params\_cip\_trace\_fileset/loadItemProfile}}%
\index[funcref]{params_cip_trace_fileset@\fidxlb{params\_cip\_trace\_fileset}!loadItemProfile@\fidxl{loadItemProfile}}%
\label{ref_params_cip_trace_fileset__loadItemProfile}%
\hypertarget{ref_params_cip_trace_fileset__loadItemProfile}{}%
\begin{description}
\item[Summary:]Loads a cip\_trace\_profile object from a raw data file in the fileset.
%
\item[Usage:]~%
\begin{lyxcode}%
[params\_row, tests\_row] = loadItemProfile(fileset, file\_index)
%
\end{lyxcode}%
%
%
\item[Parameters:]~
\begin{description}%
\item[\texttt{fileset}:]
 A params\_tests\_fileset.
\item[\texttt{file\_index}:]
 Index of file in fileset.
\end{description}%
%
\item[Returns:]~

	a\_profile: A profile object that implements the getResults method.
%
%
\item[See also:]%
\hyperlink{ref_itemResultsRow}{\texttt{itemResultsRow}}%
\ (p.~\pageref{ref_itemResultsRow})%
\index[funcref]{@\fidxl{itemResultsRow}}%
, \hyperlink{ref_params_tests_fileset}{\texttt{params\_tests\_fileset}}%
\ (p.~\pageref{ref_params_tests_fileset})%
\index[funcref]{@\fidxl{params\_tests\_fileset}}%
, \hyperlink{ref_paramNames}{\texttt{paramNames}}%
\ (p.~\pageref{ref_paramNames})%
\index[funcref]{@\fidxl{paramNames}}%
, \hyperlink{ref_testNames}{\texttt{testNames}}%
\ (p.~\pageref{ref_testNames})%
\index[funcref]{@\fidxl{testNames}}%
%
\item[Author:]%
Cengiz Gunay <cgunay@emory.edu>, 2004/09/14%
\end{description}
\methodline%
\subsubsection[Method \texttt{cip\_trace}]{Method \texttt{params\_cip\_trace\_fileset/cip\_trace}}%
\index[funcref]{params_cip_trace_fileset@\fidxlb{params\_cip\_trace\_fileset}!cip_trace@\fidxl{cip\_trace}}%
\label{ref_params_cip_trace_fileset__cip_trace}%
\hypertarget{ref_params_cip_trace_fileset__cip_trace}{}%
\begin{description}
\item[Summary:]Loads a raw cip\_trace given a file\_index to this fileset.
%
%
%
\item[Parameters:]~
\begin{description}%
\item[\texttt{fileset}:]
 A params\_tests\_fileset.
\item[\texttt{file\_index}:]
 Index of file in fileset.
\item[\texttt{a\_db}:]
 A DB created by this fileset to read the item indices from.
\end{description}%
%
\item[Returns:]~

	a\_cip\_trace: A cip\_trace object.
%
%
\item[See also:]%
\hyperlink{ref_cip_trace}{\texttt{cip\_trace}}%
\ (p.~\pageref{ref_cip_trace})%
\index[funcref]{@\fidxl{cip\_trace}}%
, \hyperlink{ref_params_tests_fileset}{\texttt{params\_tests\_fileset}}%
\ (p.~\pageref{ref_params_tests_fileset})%
\index[funcref]{@\fidxl{params\_tests\_fileset}}%
%
\item[Author:]%
Cengiz Gunay <cgunay@emory.edu>, 2004/09/13%
\end{description}
\methodline%
\subsection{Class \texttt{params\_tests\_dataset}}%
\index[funcref]{params_tests_dataset@\fidxlb{params\_tests\_dataset}}%
\label{ref_params_tests_dataset}%
\hypertarget{ref_params_tests_dataset}{}%
\subsubsection[Constructor \texttt{params\_tests\_dataset}]{Constructor \texttt{params\_tests\_dataset/params\_tests\_dataset}}%
\index[funcref]{params_tests_dataset@\fidxlb{params\_tests\_dataset}!params_tests_dataset@\fidxl{params\_tests\_dataset}}%
\label{ref_params_tests_dataset__params_tests_dataset}%
\hypertarget{ref_params_tests_dataset__params_tests_dataset}{}%
\begin{description}
\item[Summary:]Contains a set of data objects or files of raw data varying with parameter values.
%
\item[Usage:]~%
\begin{lyxcode}%
obj = params\_tests\_dataset(list, dt, dy, id, props)
%
\end{lyxcode}%
%
\item[Description:]%
This is an abstract base class for keeping dataset information separate
 from the parameters-results database (params\_tests\_db). The list contents
 can be filenames or objects (such as cip\_traces) from which to get the raw data.
 The dataset should have all the necessary information to create a db when
 needed. This is an abstract class, thet it it cannot act on its own. Only 
 fully implemented subclasses can actually hold datasets. See methods below.
%%
\item[Parameters:]~
\begin{description}%
\item[\texttt{list}:]
 Array of dataset items (filenames, objects, etc.).
\item[\texttt{dt}:]
 Time resolution [s]
\item[\texttt{dy}:]
 y-axis resolution [integral V, A, etc.]
\item[\texttt{id}:]
 An identification string.
\item[\texttt{props}:]
 A structure with any optional properties.
\begin{description}%
\item[\texttt{type}:]
 type of file (default = '')
\end{description}%
\end{description}%
%
\item[Returns a structure object with the following fields:]~

	list, dt, dy, id, props (see above).
%
%
\item[See also:]%
\hyperlink{ref_params_tests_db}{\texttt{params\_tests\_db}}%
\ (p.~\pageref{ref_params_tests_db})%
\index[funcref]{@\fidxl{params\_tests\_db}}%
, \hyperlink{ref_params_tests_fileset}{\texttt{params\_tests\_fileset}}%
\ (p.~\pageref{ref_params_tests_fileset})%
\index[funcref]{@\fidxl{params\_tests\_fileset}}%
, \hyperlink{ref_cip_traces_dataset}{\texttt{cip\_traces\_dataset}}%
\ (p.~\pageref{ref_cip_traces_dataset})%
\index[funcref]{@\fidxl{cip\_traces\_dataset}}%
%
\item[Author:]%
Cengiz Gunay <cgunay@emory.edu>, 2004/12/02%
\end{description}
\methodline%
\subsubsection[Method \texttt{getItem}]{Method \texttt{params\_tests\_dataset/getItem}}%
\index[funcref]{params_tests_dataset@\fidxlb{params\_tests\_dataset}!getItem@\fidxl{getItem}}%
\label{ref_params_tests_dataset__getItem}%
\hypertarget{ref_params_tests_dataset__getItem}{}%
\begin{description}
\item[Summary:]Returns the dataset item at given index.
%
\item[Usage:]~%
\begin{lyxcode}%
item = getItem(dataset, index)
%
\end{lyxcode}%
%
%
\item[Parameters:]~
\begin{description}%
\item[\texttt{dataset}:]
 A params\_tests\_dataset.
\item[\texttt{index}:]
 Index of item in dataset.
\end{description}%
%
\item[Returns:]~

	item: Object, filename, etc.
%
%
\item[See also:]%
\hyperlink{ref_itemResultsRow}{\texttt{itemResultsRow}}%
\ (p.~\pageref{ref_itemResultsRow})%
\index[funcref]{@\fidxl{itemResultsRow}}%
, \hyperlink{ref_params_tests_dataset}{\texttt{params\_tests\_dataset}}%
\ (p.~\pageref{ref_params_tests_dataset})%
\index[funcref]{@\fidxl{params\_tests\_dataset}}%
, \hyperlink{ref_paramNames}{\texttt{paramNames}}%
\ (p.~\pageref{ref_paramNames})%
\index[funcref]{@\fidxl{paramNames}}%
, \hyperlink{ref_testNames}{\texttt{testNames}}%
\ (p.~\pageref{ref_testNames})%
\index[funcref]{@\fidxl{testNames}}%
%
\item[Author:]%
Cengiz Gunay <cgunay@emory.edu>, 2004/12/03%
\end{description}
\methodline%
\subsubsection[Method \texttt{display}]{Method \texttt{params\_tests\_dataset/display}}%
\index[funcref]{params_tests_dataset@\fidxlb{params\_tests\_dataset}!display@\fidxl{display}}%
\label{ref_params_tests_dataset__display}%
\hypertarget{ref_params_tests_dataset__display}{}%
\begin{description}
%
%
%
%
%
%
%
\item[Author:]%
Cengiz Gunay <cgunay@emory.edu>, 2004/08/04%
\end{description}
\methodline%
\subsubsection[Method \texttt{get}]{Method \texttt{params\_tests\_dataset/get}}%
\index[funcref]{params_tests_dataset@\fidxlb{params\_tests\_dataset}!get@\fidxl{get}}%
\label{ref_params_tests_dataset__get}%
\hypertarget{ref_params_tests_dataset__get}{}%
\begin{description}
\item[Summary:]Defines generic attribute retrieval for objects.
%
%
%
%
%
%
%
\item[Author:]%
Cengiz Gunay <cgunay@emory.edu>, 2004/09/14%
\end{description}
\methodline%
\subsubsection[Method \texttt{set}]{Method \texttt{params\_tests\_dataset/set}}%
\index[funcref]{params_tests_dataset@\fidxlb{params\_tests\_dataset}!set@\fidxl{set}}%
\label{ref_params_tests_dataset__set}%
\hypertarget{ref_params_tests_dataset__set}{}%
\begin{description}
\item[Summary:]Generic method for setting object attributes.
%
%
%
%
%
%
%
\item[Author:]%
Cengiz Gunay <cgunay@emory.edu>, 2004/10/08%
\end{description}
\methodline%
\subsubsection[Method \texttt{params\_tests\_db}]{Method \texttt{params\_tests\_dataset/params\_tests\_db}}%
\index[funcref]{params_tests_dataset@\fidxlb{params\_tests\_dataset}!params_tests_db@\fidxl{params\_tests\_db}}%
\label{ref_params_tests_dataset__params_tests_db}%
\hypertarget{ref_params_tests_dataset__params_tests_db}{}%
\begin{description}
\item[Summary:]Generates a params\_tests\_db object from the dataset.
%
\item[Usage:]~%
\begin{lyxcode}%
db\_obj = params\_tests\_db(obj, items, props)
%
\end{lyxcode}%
%
\item[Description:]%
This is a converter method to convert from params\_tests\_dataset to
 params\_tests\_db. Uses readDBItems to read the files.
 A customized subclass should provide the correct 
 paramNames, testNames, and itemResultsRow functions. Adds a ItemIndex
 column to the DB to keep track of raw data files after shuffling.
%%
\item[Parameters:]~
\begin{description}%
\item[\texttt{obj}:]
 A params\_tests\_dataset object.
\item[\texttt{items}:]
 (Optional) List of item indices to use to create the db.
\item[\texttt{props}:]
 Any optional params to pass to params\_tests\_db.
\end{description}%
%
\item[Returns:]~

	db\_obj: A params\_tests\_db object.
%
%
\item[See also:]%
\hyperlink{ref_readDBItems}{\texttt{readDBItems}}%
\ (p.~\pageref{ref_readDBItems})%
\index[funcref]{@\fidxl{readDBItems}}%
, \hyperlink{ref_params_tests_db}{\texttt{params\_tests\_db}}%
\ (p.~\pageref{ref_params_tests_db})%
\index[funcref]{@\fidxl{params\_tests\_db}}%
, \hyperlink{ref_params_tests_dataset}{\texttt{params\_tests\_dataset}}%
\ (p.~\pageref{ref_params_tests_dataset})%
\index[funcref]{@\fidxl{params\_tests\_dataset}}%
, \hyperlink{ref_itemResultsRow
	    testNames}{\texttt{itemResultsRow
	    testNames}}%
\ (p.~\pageref{ref_itemResultsRow
	    testNames})%
\index[funcref]{@\fidxl{itemResultsRow
	    testNames}}%
, \hyperlink{ref_paramNames}{\texttt{paramNames}}%
\ (p.~\pageref{ref_paramNames})%
\index[funcref]{@\fidxl{paramNames}}%
%
\item[Author:]%
Cengiz Gunay <cgunay@emory.edu>, 2004/09/09%
\end{description}
\methodline%
\subsubsection[Method \texttt{subsref}]{Method \texttt{params\_tests\_dataset/subsref}}%
\index[funcref]{params_tests_dataset@\fidxlb{params\_tests\_dataset}!subsref@\fidxl{subsref}}%
\label{ref_params_tests_dataset__subsref}%
\hypertarget{ref_params_tests_dataset__subsref}{}%
\begin{description}
\item[Summary:]Defines generic indexing for objects.
%
%
%
%
%
%
%
%
\end{description}
\methodline%
\subsubsection[Method \texttt{getItemParams}]{Method \texttt{params\_tests\_dataset/getItemParams}}%
\index[funcref]{params_tests_dataset@\fidxlb{params\_tests\_dataset}!getItemParams@\fidxl{getItemParams}}%
\label{ref_params_tests_dataset__getItemParams}%
\hypertarget{ref_params_tests_dataset__getItemParams}{}%
\begin{description}
\item[Summary:]Get the parameter values of a dataset item.
%
\item[Usage:]~%
\begin{lyxcode}%
params\_row = getItemParams(dataset, index, a\_profile)
%
\end{lyxcode}%
%
\item[Description:]%
This method can retrieve the item parameters by using either the 
 dataset and the index to find the item or simply by using
 the item profile, a\_profile.
%%
\item[Parameters:]~
\begin{description}%
\item[\texttt{dataset}:]
 A params\_tests\_dataset.
\item[\texttt{index}:]
 Index of item in dataset.
\item[\texttt{a\_profile}:]
 An item profile.
\end{description}%
%
\item[Returns:]~

	params\_row: Parameter values in the same order of paramNames
%
%
\item[See also:]%
\hyperlink{ref_itemResultsRow}{\texttt{itemResultsRow}}%
\ (p.~\pageref{ref_itemResultsRow})%
\index[funcref]{@\fidxl{itemResultsRow}}%
, \hyperlink{ref_params_tests_dataset}{\texttt{params\_tests\_dataset}}%
\ (p.~\pageref{ref_params_tests_dataset})%
\index[funcref]{@\fidxl{params\_tests\_dataset}}%
, \hyperlink{ref_paramNames}{\texttt{paramNames}}%
\ (p.~\pageref{ref_paramNames})%
\index[funcref]{@\fidxl{paramNames}}%
, \hyperlink{ref_testNames}{\texttt{testNames}}%
\ (p.~\pageref{ref_testNames})%
\index[funcref]{@\fidxl{testNames}}%
%
\item[Author:]%
Cengiz Gunay <cgunay@emory.edu>, 2004/09/10%
\end{description}
\methodline%
\subsubsection[Method \texttt{itemResultsRow}]{Method \texttt{params\_tests\_dataset/itemResultsRow}}%
\index[funcref]{params_tests_dataset@\fidxlb{params\_tests\_dataset}!itemResultsRow@\fidxl{itemResultsRow}}%
\label{ref_params_tests_dataset__itemResultsRow}%
\hypertarget{ref_params_tests_dataset__itemResultsRow}{}%
\begin{description}
\item[Summary:]Processes a raw data file from the dataset and return
		its parameter and test values.
%
\item[Usage:]~%
\begin{lyxcode}%
[params\_row, tests\_row] = itemResultsRow(dataset, index)
%
\end{lyxcode}%
%
\item[Description:]%
This method is designed to be reused from subclasses as long as the
 loadItemProfile method is properly overloaded. Adds an Index
 column to the DB to keep track of raw data items after shuffling.
%%
\item[Parameters:]~
\begin{description}%
\item[\texttt{dataset}:]
 A params\_tests\_dataset.
\item[\texttt{index}:]
 Index of file in dataset.
\end{description}%
%
\item[Returns:]~

	params\_row: Parameter values in the same order of paramNames
	tests\_row: Test values in the same order with testNames
%
%
\item[See also:]%
\hyperlink{ref_loadItemProfile}{\texttt{loadItemProfile}}%
\ (p.~\pageref{ref_loadItemProfile})%
\index[funcref]{@\fidxl{loadItemProfile}}%
, \hyperlink{ref_params_tests_dataset}{\texttt{params\_tests\_dataset}}%
\ (p.~\pageref{ref_params_tests_dataset})%
\index[funcref]{@\fidxl{params\_tests\_dataset}}%
, \hyperlink{ref_paramNames}{\texttt{paramNames}}%
\ (p.~\pageref{ref_paramNames})%
\index[funcref]{@\fidxl{paramNames}}%
, \hyperlink{ref_testNames}{\texttt{testNames}}%
\ (p.~\pageref{ref_testNames})%
\index[funcref]{@\fidxl{testNames}}%
%
\item[Author:]%
Cengiz Gunay <cgunay@emory.edu>, 2004/09/10%
\end{description}
\methodline%
\subsubsection[Method \texttt{addItem}]{Method \texttt{params\_tests\_dataset/addItem}}%
\index[funcref]{params_tests_dataset@\fidxlb{params\_tests\_dataset}!addItem@\fidxl{addItem}}%
\label{ref_params_tests_dataset__addItem}%
\hypertarget{ref_params_tests_dataset__addItem}{}%
\begin{description}
\item[Summary:]Returns the new dataset with the added item.
%
\item[Usage:]~%
\begin{lyxcode}%
dataset = addItem(dataset, item)
%
\end{lyxcode}%
%
\item[Description:]%
Note that, this is NOT the way to create a dataset. It is only intended for 
 small additions to an existing dataset. This method is too slow
 for creating large datasets. The normal method for creating datasets is
 providing the full list of items to the class constructor.
%%
\item[Parameters:]~
\begin{description}%
\item[\texttt{dataset}:]
 A params\_tests\_dataset.
\item[\texttt{item}:]
 New item to add in dataset.
\end{description}%
%
\item[Returns:]~

	dataset: With the added item.
%
%
\item[See also:]%
\hyperlink{ref_itemResultsRow}{\texttt{itemResultsRow}}%
\ (p.~\pageref{ref_itemResultsRow})%
\index[funcref]{@\fidxl{itemResultsRow}}%
, \hyperlink{ref_params_tests_dataset}{\texttt{params\_tests\_dataset}}%
\ (p.~\pageref{ref_params_tests_dataset})%
\index[funcref]{@\fidxl{params\_tests\_dataset}}%
, \hyperlink{ref_paramNames}{\texttt{paramNames}}%
\ (p.~\pageref{ref_paramNames})%
\index[funcref]{@\fidxl{paramNames}}%
, \hyperlink{ref_testNames}{\texttt{testNames}}%
\ (p.~\pageref{ref_testNames})%
\index[funcref]{@\fidxl{testNames}}%
%
\item[Author:]%
Cengiz Gunay <cgunay@emory.edu>, 2005/01/25%
\end{description}
\methodline%
\subsubsection[Method \texttt{testNames}]{Method \texttt{params\_tests\_dataset/testNames}}%
\index[funcref]{params_tests_dataset@\fidxlb{params\_tests\_dataset}!testNames@\fidxl{testNames}}%
\label{ref_params_tests_dataset__testNames}%
\hypertarget{ref_params_tests_dataset__testNames}{}%
\begin{description}
\item[Summary:]Returns the ordered names of tests for this dataset.
%
\item[Usage:]~%
\begin{lyxcode}%
test\_names = testNames(dataset, item)
%
\end{lyxcode}%
%
\item[Description:]%
Looks at the results of the first file to find the test names.
%%
\item[Parameters:]~
\begin{description}%
\item[\texttt{dataset}:]
 A params\_tests\_dataset.
\end{description}%
%
\item[Returns:]~

	params\_names: Cell array with ordered parameter names.
	item: (Optional) If given, read names by loading item at this index.
%
%
\item[See also:]%
\hyperlink{ref_params_tests_dataset}{\texttt{params\_tests\_dataset}}%
\ (p.~\pageref{ref_params_tests_dataset})%
\index[funcref]{@\fidxl{params\_tests\_dataset}}%
, \hyperlink{ref_paramNames}{\texttt{paramNames}}%
\ (p.~\pageref{ref_paramNames})%
\index[funcref]{@\fidxl{paramNames}}%
, \hyperlink{ref_testNames}{\texttt{testNames}}%
\ (p.~\pageref{ref_testNames})%
\index[funcref]{@\fidxl{testNames}}%
%
\item[Author:]%
Cengiz Gunay <cgunay@emory.edu>, 2004/09/10%
\end{description}
\methodline%
\subsubsection[Method \texttt{readDBItems}]{Method \texttt{params\_tests\_dataset/readDBItems}}%
\index[funcref]{params_tests_dataset@\fidxlb{params\_tests\_dataset}!readDBItems@\fidxl{readDBItems}}%
\label{ref_params_tests_dataset__readDBItems}%
\hypertarget{ref_params_tests_dataset__readDBItems}{}%
\begin{description}
\item[Summary:]Reads all items to generate a params\_tests\_db object.
%
\item[Usage:]~%
\begin{lyxcode}%
[params, param\_names, tests, test\_names] = readDBItems(obj, items)
%
\end{lyxcode}%
%
\item[Description:]%
This is a generic method to convert from params\_tests\_fileset to
 a params\_tests\_db, or a subclass. This method depends on the  
 paramNames, testNames, and itemResultsRow functions. 
 Outputs of this function can be directly fed to the constructor of
 a params\_tests\_db or a subclass.
%%
\item[Parameters:]~
\begin{description}%
\item[\texttt{obj}:]
 A params\_tests\_fileset object.
\item[\texttt{items}:]
 (Optional) List of item indices to use to create the db.
\end{description}%
%
\item[Returns:]~

	params, param\_names, tests, test\_names: See params\_tests\_db.
%
%
\item[See also:]%
\hyperlink{ref_params_tests_db}{\texttt{params\_tests\_db}}%
\ (p.~\pageref{ref_params_tests_db})%
\index[funcref]{@\fidxl{params\_tests\_db}}%
, \hyperlink{ref_params_tests_fileset}{\texttt{params\_tests\_fileset}}%
\ (p.~\pageref{ref_params_tests_fileset})%
\index[funcref]{@\fidxl{params\_tests\_fileset}}%
, \hyperlink{ref_itemResultsRow
	    testNames}{\texttt{itemResultsRow
	    testNames}}%
\ (p.~\pageref{ref_itemResultsRow
	    testNames})%
\index[funcref]{@\fidxl{itemResultsRow
	    testNames}}%
, \hyperlink{ref_paramNames}{\texttt{paramNames}}%
\ (p.~\pageref{ref_paramNames})%
\index[funcref]{@\fidxl{paramNames}}%
%
\item[Author:]%
Cengiz Gunay <cgunay@emory.edu>, 2004/11/24%
\end{description}
\methodline%
\subsection{Class \texttt{params\_tests\_db}}%
\index[funcref]{params_tests_db@\fidxlb{params\_tests\_db}}%
\label{ref_params_tests_db}%
\hypertarget{ref_params_tests_db}{}%
\subsubsection[Constructor \texttt{params\_tests\_db}]{Constructor \texttt{params\_tests\_db/params\_tests\_db}}%
\index[funcref]{params_tests_db@\fidxlb{params\_tests\_db}!params_tests_db@\fidxl{params\_tests\_db}}%
\label{ref_params_tests_db__params_tests_db}%
\hypertarget{ref_params_tests_db__params_tests_db}{}%
\begin{description}
\item[Summary:]A generic database of test results varying with 
		parameter values, organized in a matrix format.
%
%
\item[Description:]%
This is a subclass of tests\_db. Defines all operations on this
 structure so that subclasses can use them.
%%
\item[Parameters:]~
\begin{description}%
\item[\texttt{num\_params}:]
 Number of parameters.
\item[\texttt{a\_tests\_db}:]
 A tests\_db upon which to build the params\_tests\_db.
\item[\texttt{props}:]
 A structure with any optional properties.
\end{description}%
%
\item[Returns a structure object with the following fields:]~

	tests\_db
	num\_params: Number of variable parameters in simulations.
%
%
\item[See also:]%
\hyperlink{ref_tests_db}{\texttt{tests\_db}}%
\ (p.~\pageref{ref_tests_db})%
\index[funcref]{@\fidxl{tests\_db}}%
, \hyperlink{ref_test_variable_db (N__I)}{\texttt{test\_variable\_db (N/I)}}%
\ (p.~\pageref{ref_test_variable_db (N__I)})%
\index[funcref]{test_variable_db (N@\fidxlb{test\_variable\_db (N}!I)@\fidxl{I)}}%
%
\item[Author:]%
Cengiz Gunay <cgunay@emory.edu>, 2004/09/08%
\end{description}
\methodline%
\subsubsection[Method \texttt{paramsTestsCoefsHists}]{Method \texttt{params\_tests\_db/paramsTestsCoefsHists}}%
\index[funcref]{params_tests_db@\fidxlb{params\_tests\_db}!paramsTestsCoefsHists@\fidxl{paramsTestsCoefsHists}}%
\label{ref_params_tests_db__paramsTestsCoefsHists}%
\hypertarget{ref_params_tests_db__paramsTestsCoefsHists}{}%
\begin{description}
\item[Summary:]Calculates histograms for all pairs of params 
		  and tests coefficients and returns in a cell array.
%
\item[Usage:]~%
\begin{lyxcode}%
pt\_coefs\_hists = paramsTestsCoefsHists(a\_db, p\_coefs)
%
\end{lyxcode}%
%
\item[Description:]%
Skips the 'ItemIndex' test.
%%
\item[Parameters:]~
\begin{description}%
\item[\texttt{a\_db}:]
 A tests\_db object.
\item[\texttt{p\_coefs}:]
 Cell array of tests coefficients for each parameter.
\end{description}%
%
\item[Returns:]~

	pt\_coefs\_hists: A cell array of corrcoefs\_dbs for each param in a\_db.
%
%
\item[See also:]%
\hyperlink{ref_params_tests_profile}{\texttt{params\_tests\_profile}}%
\ (p.~\pageref{ref_params_tests_profile})%
\index[funcref]{@\fidxl{params\_tests\_profile}}%
%
\item[Author:]%
Cengiz Gunay <cgunay@emory.edu>, 2004/10/17%
\end{description}
\methodline%
\subsubsection[Method \texttt{onlyRowsTests}]{Method \texttt{params\_tests\_db/onlyRowsTests}}%
\index[funcref]{params_tests_db@\fidxlb{params\_tests\_db}!onlyRowsTests@\fidxl{onlyRowsTests}}%
\label{ref_params_tests_db__onlyRowsTests}%
\hypertarget{ref_params_tests_db__onlyRowsTests}{}%
\begin{description}
\item[Summary:]Returns a tests\_db that only contains the desired 
		tests and rows (and pages).
%
\item[Usage:]~%
\begin{lyxcode}%
obj = onlyRowsTests(obj, rows, tests, pages)
%
\end{lyxcode}%
%
\item[Description:]%
Selects the given dimensions and returns in a new tests\_db object.
%%
\item[Parameters:]~
\begin{description}%
\item[\texttt{obj}:]
 A tests\_db object.
\item[\texttt{rows}:]
 A logical or index vector of rows. If ':', all rows.
\item[\texttt{tests}:]
 Cell array of test names or column indices. If ':', all tests.
\item[\texttt{pages}:]
 (Optional) A logical or index vector of pages. ':' for all pages.
\end{description}%
%
\item[Returns:]~

	obj: The new tests\_db object.
%
%
\item[See also:]%
\hyperlink{ref_subsref}{\texttt{subsref}}%
\ (p.~\pageref{ref_subsref})%
\index[funcref]{@\fidxl{subsref}}%
, \hyperlink{ref_tests_db}{\texttt{tests\_db}}%
\ (p.~\pageref{ref_tests_db})%
\index[funcref]{@\fidxl{tests\_db}}%
%
\item[Author:]%
Cengiz Gunay <cgunay@emory.edu>, 2004/09/17%
\end{description}
\methodline%
\subsubsection[Method \texttt{joinRows}]{Method \texttt{params\_tests\_db/joinRows}}%
\index[funcref]{params_tests_db@\fidxlb{params\_tests\_db}!joinRows@\fidxl{joinRows}}%
\label{ref_params_tests_db__joinRows}%
\hypertarget{ref_params_tests_db__joinRows}{}%
\begin{description}
\item[Summary:]Joins the rows of the given db with rows of with\_db with matching
  	RowIndex values.
%
\item[Usage:]~%
\begin{lyxcode}%
a\_db = joinRows(db, tests, with\_db, w\_tests, index\_col\_name)
%
\end{lyxcode}%
%
\item[Description:]%
Takes the desired columns in with\_db with rows having a 
 row index and joins them next to dedired columns from the current db. 
 Assumes each row index only appears once in with\_db. The created
 db preserves the ordering of with\_db.
%%
\item[Parameters:]~
\begin{description}%
\item[\texttt{db}:]
 A param\_tests\_db object.
\item[\texttt{tests}:]
 Test columns to take from db.
\item[\texttt{with\_db}:]
 A tests\_db object with a RowIndex column.
\item[\texttt{w\_tests}:]
 Test columns to take from with\_db.
\item[\texttt{index\_col\_name}:]
 (Optional) Name of row index column (default='RowIndex').
\end{description}%
%
\item[Returns:]~

	a\_db: A params\_tests\_db object.
%
%
\item[See also:]%
\hyperlink{ref_tests_db}{\texttt{tests\_db}}%
\ (p.~\pageref{ref_tests_db})%
\index[funcref]{@\fidxl{tests\_db}}%
%
\item[Author:]%
Cengiz Gunay <cgunay@emory.edu>, 2004/10/16%
\end{description}
\methodline%
\subsubsection[Method \texttt{crossProd}]{Method \texttt{params\_tests\_db/crossProd}}%
\index[funcref]{params_tests_db@\fidxlb{params\_tests\_db}!crossProd@\fidxl{crossProd}}%
\label{ref_params_tests_db__crossProd}%
\hypertarget{ref_params_tests_db__crossProd}{}%
\begin{description}
\item[Summary:]Create a DB by taking the cross product of two database row sets.
%
\item[Usage:]~%
\begin{lyxcode}%
cross\_db = crossProd(a\_db, b\_db)
%
\end{lyxcode}%
%
\item[Description:]%
Overloaded function to maintain correct number of parameters after
 cross product operation. See original in tests\_db/crossProd.
%%
\item[Parameters:]~
\begin{description}%
\item[\texttt{a\_db, b\_db}:]
 A tests\_db object.
\end{description}%
%
\item[Returns:]~

	cross\_db: The tests\_db object with all combinations of rows.
%
%
\item[See also:]%
\hyperlink{ref_tests_db__crossProd}{\texttt{tests\_db/crossProd}}%
\ (p.~\pageref{ref_tests_db__crossProd})%
\index[funcref]{tests_db@\fidxlb{tests\_db}!crossProd@\fidxl{crossProd}}%
%
\item[Author:]%
Cengiz Gunay <cgunay@emory.edu>, 2005/10/11%
\end{description}
\methodline%
\subsubsection[Method \texttt{display}]{Method \texttt{params\_tests\_db/display}}%
\index[funcref]{params_tests_db@\fidxlb{params\_tests\_db}!display@\fidxl{display}}%
\label{ref_params_tests_db__display}%
\hypertarget{ref_params_tests_db__display}{}%
\begin{description}
%
%
%
%
%
%
%
\item[Author:]%
Cengiz Gunay <cgunay@emory.edu>, 2004/08/04%
\end{description}
\methodline%
\subsubsection[Method \texttt{testsHists}]{Method \texttt{params\_tests\_db/testsHists}}%
\index[funcref]{params_tests_db@\fidxlb{params\_tests\_db}!testsHists@\fidxl{testsHists}}%
\label{ref_params_tests_db__testsHists}%
\hypertarget{ref_params_tests_db__testsHists}{}%
\begin{description}
\item[Summary:]Calculates histograms for all tests and returns them in a cell array.
%
\item[Usage:]~%
\begin{lyxcode}%
t\_hists = testsHists(a\_db, num\_bins)
%
\end{lyxcode}%
%
\item[Description:]%
Skips the 'ItemIndex' test.
%%
\item[Parameters:]~
\begin{description}%
\item[\texttt{a\_db}:]
 One or more tests\_db objects in an array.
\item[\texttt{num\_bins}:]
 Number of histogram bins (Optional, default=100), or

vector of histogram bin centers.\end{description}%
%
\item[Returns:]~

	t\_hists: An array of histograms for each test in a\_db.
%
%
\item[See also:]%
\hyperlink{ref_params_tests_profile}{\texttt{params\_tests\_profile}}%
\ (p.~\pageref{ref_params_tests_profile})%
\index[funcref]{@\fidxl{params\_tests\_profile}}%
%
\item[Author:]%
Cengiz Gunay <cgunay@emory.edu>, 2004/10/17%
\end{description}
\methodline%
\subsubsection[Method \texttt{get}]{Method \texttt{params\_tests\_db/get}}%
\index[funcref]{params_tests_db@\fidxlb{params\_tests\_db}!get@\fidxl{get}}%
\label{ref_params_tests_db__get}%
\hypertarget{ref_params_tests_db__get}{}%
\begin{description}
\item[Summary:]Defines generic attribute retrieval for objects.
%
%
%
%
%
%
%
\item[Author:]%
Cengiz Gunay <cgunay@emory.edu>, 2004/09/14%
\end{description}
\methodline%
\subsubsection[Method \texttt{set}]{Method \texttt{params\_tests\_db/set}}%
\index[funcref]{params_tests_db@\fidxlb{params\_tests\_db}!set@\fidxl{set}}%
\label{ref_params_tests_db__set}%
\hypertarget{ref_params_tests_db__set}{}%
\begin{description}
\item[Summary:]Generic method for setting object attributes.
%
%
%
%
%
%
%
\item[Author:]%
Cengiz Gunay <cgunay@emory.edu>, 2004/10/08%
\end{description}
\methodline%
\subsubsection[Method \texttt{matchingRow}]{Method \texttt{params\_tests\_db/matchingRow}}%
\index[funcref]{params_tests_db@\fidxlb{params\_tests\_db}!matchingRow@\fidxl{matchingRow}}%
\label{ref_params_tests_db__matchingRow}%
\hypertarget{ref_params_tests_db__matchingRow}{}%
\begin{description}
\item[Summary:]Creates a criterion database for matching the tests of a row.
%
\item[Usage:]~%
\begin{lyxcode}%
crit\_db = matchingRow(a\_db, row, props)
%
\end{lyxcode}%
%
\item[Description:]%
Overloaded method for skipping parameter values. STD for param values will be NaNs.
%%
\item[Parameters:]~
\begin{description}%
\item[\texttt{a\_db}:]
 A tests\_db object.
\item[\texttt{row}:]
 A row index to match.
\item[\texttt{props}:]
 A structure with any optional properties.
\begin{description}%
\item[\texttt{std\_db}:]
 Take the standard deviation from this db instead.
\end{description}%
\end{description}%
%
\item[Returns:]~

	crit\_db: A tests\_db with two rows for values and STDs.
%
%
\item[See also:]%
\hyperlink{ref_tests_db__matchingRow}{\texttt{tests\_db/matchingRow}}%
\ (p.~\pageref{ref_tests_db__matchingRow})%
\index[funcref]{tests_db@\fidxlb{tests\_db}!matchingRow@\fidxl{matchingRow}}%
, \hyperlink{ref_rankMatching}{\texttt{rankMatching}}%
\ (p.~\pageref{ref_rankMatching})%
\index[funcref]{@\fidxl{rankMatching}}%
, \hyperlink{ref_tests_db}{\texttt{tests\_db}}%
\ (p.~\pageref{ref_tests_db})%
\index[funcref]{@\fidxl{tests\_db}}%
, \hyperlink{ref_tests2cols}{\texttt{tests2cols}}%
\ (p.~\pageref{ref_tests2cols})%
\index[funcref]{@\fidxl{tests2cols}}%
%
\item[Author:]%
Cengiz Gunay <cgunay@emory.edu>, 2006/06/13%
\end{description}
\methodline%
\subsubsection[Method \texttt{invarParam}]{Method \texttt{params\_tests\_db/invarParam}}%
\index[funcref]{params_tests_db@\fidxlb{params\_tests\_db}!invarParam@\fidxl{invarParam}}%
\label{ref_params_tests_db__invarParam}%
\hypertarget{ref_params_tests_db__invarParam}{}%
\begin{description}
\item[Summary:]Generates a 3D database of invariant values of a parameter and all test columns. 
%
\item[Usage:]~%
\begin{lyxcode}%
a\_3D\_db = invarParam(db, param)
%
\end{lyxcode}%
%
\item[Description:]%
Finds all combinations when the rest of the parameters are fixed,
 and saves the variation of the selected parameter and all tests in
 a new database.
%%
\item[Parameters:]~
\begin{description}%
\item[\texttt{db}:]
 A tests\_db object.
\item[\texttt{param}:]
 A parameter name/column number
\end{description}%
%
\item[Returns:]~

	a\_3D\_db: A tests\_3D\_db object of organized values.
%
%
\item[See also:]%
\hyperlink{ref_invarValues}{\texttt{invarValues}}%
\ (p.~\pageref{ref_invarValues})%
\index[funcref]{@\fidxl{invarValues}}%
, \hyperlink{ref_tests_3D_db}{\texttt{tests\_3D\_db}}%
\ (p.~\pageref{ref_tests_3D_db})%
\index[funcref]{@\fidxl{tests\_3D\_db}}%
, \hyperlink{ref_corrCoefs}{\texttt{corrCoefs}}%
\ (p.~\pageref{ref_corrCoefs})%
\index[funcref]{@\fidxl{corrCoefs}}%
, \hyperlink{ref_tests_3D_db__plotPair}{\texttt{tests\_3D\_db/plotPair}}%
\ (p.~\pageref{ref_tests_3D_db__plotPair})%
\index[funcref]{tests_3D_db@\fidxlb{tests\_3D\_db}!plotPair@\fidxl{plotPair}}%
%
\item[Author:]%
Cengiz Gunay <cgunay@emory.edu>, 2004/10/07%
\end{description}
\methodline%
\subsubsection[Method \texttt{paramsHists}]{Method \texttt{params\_tests\_db/paramsHists}}%
\index[funcref]{params_tests_db@\fidxlb{params\_tests\_db}!paramsHists@\fidxl{paramsHists}}%
\label{ref_params_tests_db__paramsHists}%
\hypertarget{ref_params_tests_db__paramsHists}{}%
\begin{description}
\item[Summary:]Calculates histograms for all parameters and returns in a 
		cell array.
%
\item[Usage:]~%
\begin{lyxcode}%
p\_hists = paramsHists(a\_db)
%
\end{lyxcode}%
%
\item[Description:]%
Skips the 'ItemIndex' test. Useful for looking at subset databases and
 find out what parameter values are used most.
%%
\item[Parameters:]~
\begin{description}%
\item[\texttt{a\_db}:]
 A tests\_db object.
\end{description}%
%
\item[Returns:]~

	p\_hists: An array of histograms for each parameter in a\_db.
%
%
\item[See also:]%
\hyperlink{ref_params_tests_profile}{\texttt{params\_tests\_profile}}%
\ (p.~\pageref{ref_params_tests_profile})%
\index[funcref]{@\fidxl{params\_tests\_profile}}%
%
\item[Author:]%
Cengiz Gunay <cgunay@emory.edu>, 2004/10/20%
\end{description}
\methodline%
\subsubsection[Method \texttt{makeGenesisParFile}]{Method \texttt{params\_tests\_db/makeGenesisParFile}}%
\index[funcref]{params_tests_db@\fidxlb{params\_tests\_db}!makeGenesisParFile@\fidxl{makeGenesisParFile}}%
\label{ref_params_tests_db__makeGenesisParFile}%
\hypertarget{ref_params_tests_db__makeGenesisParFile}{}%
\begin{description}
\item[Summary:]Creates a Genesis parameter file with all the parameter values in a\_db.
%
\item[Usage:]~%
\begin{lyxcode}%
makeGenesisParFile(a\_db, filename, props)
%
\end{lyxcode}%
%
\item[Description:]%
For each a\_db row, print the parameter names in a
 file formatted for Genesis.
%%
\item[Parameters:]~
\begin{description}%
\item[\texttt{a\_db}:]
 A params\_tests\_db object.
\item[\texttt{filename}:]
 Genesis parameter file to be created.
\item[\texttt{props}:]
 A structure with any optional properties.
\begin{description}%
\item[\texttt{trialStart}:]
 If given, adds/replaces the trial parameter and counts forward.
\end{description}%
\end{description}%
%
\item[Returns:]~

	nothing.
%
\item[Example:]~
\begin{lyxcode}>> blocked\_rows\_db = makeModifiedParamDB(ranked\_for\_gps0501a\_db, 1, [1, 2], 10, [-100 100]);\\%
>> makeGenesisParFile(blocked\_rows\_db, 'blocked\_gps0501-03.par')\\%
\end{lyxcode}
%
\item[See also:]%
\hyperlink{ref_makeModifiedParamDB}{\texttt{makeModifiedParamDB}}%
\ (p.~\pageref{ref_makeModifiedParamDB})%
\index[funcref]{@\fidxl{makeModifiedParamDB}}%
, \hyperlink{ref_scanParamAllRows}{\texttt{scanParamAllRows}}%
\ (p.~\pageref{ref_scanParamAllRows})%
\index[funcref]{@\fidxl{scanParamAllRows}}%
, \hyperlink{ref_scaleParamsOneRow}{\texttt{scaleParamsOneRow}}%
\ (p.~\pageref{ref_scaleParamsOneRow})%
\index[funcref]{@\fidxl{scaleParamsOneRow}}%
%
\item[Author:]%
Cengiz Gunay <cgunay@emory.edu>, 2005/03/13%
\end{description}
\methodline%
\subsubsection[Method \texttt{rankVsAllDB}]{Method \texttt{params\_tests\_db/rankVsAllDB}}%
\index[funcref]{params_tests_db@\fidxlb{params\_tests\_db}!rankVsAllDB@\fidxl{rankVsAllDB}}%
\label{ref_params_tests_db__rankVsAllDB}%
\hypertarget{ref_params_tests_db__rankVsAllDB}{}%
\begin{description}
\item[Summary:]Generates ranking DBs by comparing rows of a\_db with each row of to\_db.
%
\item[Usage:]~%
\begin{lyxcode}%
tex\_string = rankVsAllDB(a\_db, to\_db, a\_dataset, to\_dataset)
%
\end{lyxcode}%
%
\item[Description:]%
Distance is each measure difference divided by the STD in to\_db, squared and
 summed. Returned DB contains only the selected to\_tests and the parameters
 from initial DB.
%%
\item[Parameters:]~
\begin{description}%
\item[\texttt{a\_db}:]
 A params\_tests\_db object to compare rows from.
\item[\texttt{to\_db}:]
 A tests\_db object to compare it with.
\item[\texttt{a\_dataset}:]
 Dataset for a\_db.
\item[\texttt{to\_dataset}:]
 Dataset for crit\_db.
\end{description}%
%
\item[Returns:]~

	ranked\_dbs: Array of created DBs with original rows and a distance 
		   measure, in ascending order. 
	tex\_string: A LaTeX string for all tables created.
%
%
\item[See also:]%
\hyperlink{ref_rankVsDB}{\texttt{rankVsDB}}%
\ (p.~\pageref{ref_rankVsDB})%
\index[funcref]{@\fidxl{rankVsDB}}%
, \hyperlink{ref_matchingRow}{\texttt{matchingRow}}%
\ (p.~\pageref{ref_matchingRow})%
\index[funcref]{@\fidxl{matchingRow}}%
, \hyperlink{ref_rankMatching}{\texttt{rankMatching}}%
\ (p.~\pageref{ref_rankMatching})%
\index[funcref]{@\fidxl{rankMatching}}%
, \hyperlink{ref_joinRows}{\texttt{joinRows}}%
\ (p.~\pageref{ref_joinRows})%
\index[funcref]{@\fidxl{joinRows}}%
%
\item[Author:]%
Cengiz Gunay <cgunay@emory.edu>, 2004/12/10%
\end{description}
\methodline%
\subsubsection[Method \texttt{addParams}]{Method \texttt{params\_tests\_db/addParams}}%
\index[funcref]{params_tests_db@\fidxlb{params\_tests\_db}!addParams@\fidxl{addParams}}%
\label{ref_params_tests_db__addParams}%
\hypertarget{ref_params_tests_db__addParams}{}%
\begin{description}
\item[Summary:]Inserts new parameter columns to tests\_db.
%
\item[Usage:]~%
\begin{lyxcode}%
obj = addParams(obj, param\_names, param\_columns)
%
\end{lyxcode}%
%
\item[Description:]%
Adds new columns to the database and returns the new DB.
   This operation is expensive in the sense that the whole database matrix
   needs to be enlarged just to add a 
   single new column. The method of allocating a matrix, filling it up, and
   then providing it to the tests\_db constructor is the preferred method 
   of creating tests\_db objects. This method may be used for 
   measures obtained by operating on raw measures.
%%
\item[Parameters:]~
\begin{description}%
\item[\texttt{obj}:]
 A tests\_db object.
\item[\texttt{param\_names}:]
 A cell array of param names to be added.
\item[\texttt{param\_columns}:]
 Data matrix of columns to be added.
\end{description}%
%
\item[Returns:]~

	obj: The tests\_db object that includes the new columns.
%
%
\item[See also:]%
\hyperlink{ref_allocateRows}{\texttt{allocateRows}}%
\ (p.~\pageref{ref_allocateRows})%
\index[funcref]{@\fidxl{allocateRows}}%
, \hyperlink{ref_tests_db}{\texttt{tests\_db}}%
\ (p.~\pageref{ref_tests_db})%
\index[funcref]{@\fidxl{tests\_db}}%
%
\item[Author:]%
Cengiz Gunay <cgunay@emory.edu>, 2005/10/11%
\end{description}
\methodline%
\subsubsection[Method \texttt{mergeMultipleCIPsInOne}]{Method \texttt{params\_tests\_db/mergeMultipleCIPsInOne}}%
\index[funcref]{params_tests_db@\fidxlb{params\_tests\_db}!mergeMultipleCIPsInOne@\fidxl{mergeMultipleCIPsInOne}}%
\label{ref_params_tests_db__mergeMultipleCIPsInOne}%
\hypertarget{ref_params_tests_db__mergeMultipleCIPsInOne}{}%
\begin{description}
\item[Summary:]Merges multiple rows with different CIP data into one, generating a database of one row per neuron.
%
\item[Usage:]~%
\begin{lyxcode}%
a\_db = mergeMultipleCIPsInOne(db, names\_tests\_cell, index\_col\_name)
%
\end{lyxcode}%
%
\item[Description:]%
It calls invarParam to separate db into pages with different CIP level data.
 Then uses the names\_tests\_cell to choose tests from each page to be merged into the 
 final database row. The tests will be suffixed with the field name so that 
 they can be distinguished. RowIndex columns
 will be automatically included, and one of them can be chosen with index\_col\_name
 that has values for all cells. The suffixed for needs to be used to 
 choose index\_col\_name, such as 'RowIndex\_H100pA', assuming 'H100pA' was the field
 name in names\_tests\_cell that corresponds to page -100 pA.
%%
\item[Parameters:]~
\begin{description}%
\item[\texttt{db}:]
 A params\_tests\_db object.
\item[\texttt{names\_tests\_cell}:]
 A cell array alternating suffix names and test column vectors.

The order of names correspond to each unique CIP level in db, 
with increasing order.\item[\texttt{index\_col\_name}:]
 (Optional) Name of row index column 

(default is 'RowIndex' suffixed with the first field name).\end{description}%
%
\item[Returns:]~

	a\_db: A params\_tests\_db object of organized values.
%
\item[Example:]~
\begin{lyxcode}        >> control\_phys\_sdb = \\%
             mergeMultipleCIPsInOne(control\_phys\_db, \\%
                                     struct('\_H100pA', [1:10], '\_D100pA', [1:10 16:18]), \\%
                                     'RowIndex\_H100pA')\\%
\end{lyxcode}
%
\item[See also:]%
\hyperlink{ref_invarValues}{\texttt{invarValues}}%
\ (p.~\pageref{ref_invarValues})%
\index[funcref]{@\fidxl{invarValues}}%
, \hyperlink{ref_tests_3D_db}{\texttt{tests\_3D\_db}}%
\ (p.~\pageref{ref_tests_3D_db})%
\index[funcref]{@\fidxl{tests\_3D\_db}}%
, \hyperlink{ref_corrCoefs}{\texttt{corrCoefs}}%
\ (p.~\pageref{ref_corrCoefs})%
\index[funcref]{@\fidxl{corrCoefs}}%
, \hyperlink{ref_tests_3D_db__plotVarBox}{\texttt{tests\_3D\_db/plotVarBox}}%
\ (p.~\pageref{ref_tests_3D_db__plotVarBox})%
\index[funcref]{tests_3D_db@\fidxlb{tests\_3D\_db}!plotVarBox@\fidxl{plotVarBox}}%
%
\item[Author:]%
Cengiz Gunay <cgunay@emory.edu>, 2005/01/13%
\end{description}
\methodline%
\subsubsection[Method \texttt{subsref}]{Method \texttt{params\_tests\_db/subsref}}%
\index[funcref]{params_tests_db@\fidxlb{params\_tests\_db}!subsref@\fidxl{subsref}}%
\label{ref_params_tests_db__subsref}%
\hypertarget{ref_params_tests_db__subsref}{}%
\begin{description}
\item[Summary:]Defines generic indexing for objects.
%
%
%
%
%
%
%
%
\end{description}
\methodline%
\subsubsection[Method \texttt{paramsParamsCoefs}]{Method \texttt{params\_tests\_db/paramsParamsCoefs}}%
\index[funcref]{params_tests_db@\fidxlb{params\_tests\_db}!paramsParamsCoefs@\fidxl{paramsParamsCoefs}}%
\label{ref_params_tests_db__paramsParamsCoefs}%
\hypertarget{ref_params_tests_db__paramsParamsCoefs}{}%
\begin{description}
\item[Summary:]Calculates a corrcoefs\_db for each param from correlations of variant params and invariant param coefs and collects them in a cell array.
%
\item[Usage:]~%
\begin{lyxcode}%
pp\_coefs = paramsParamsCoefs(a\_db, p\_t3ds, p\_coefs)
%
\end{lyxcode}%
%
\item[Description:]%
Skips the 'ItemIndex' test.
%%
\item[Parameters:]~
\begin{description}%
\item[\texttt{a\_db}:]
 A tests\_db object.
\item[\texttt{p\_t3ds}:]
 Cell array of invariant parameter databases.
\item[\texttt{p\_coefs}:]
 Cell array of tests coefficients for each parameter.
\end{description}%
%
\item[Returns:]~

	pp\_coefs: A cell array of corrcoefs\_dbs for each param 
		  combination in a\_db.
%
%
\item[See also:]%
\hyperlink{ref_params_tests_profile}{\texttt{params\_tests\_profile}}%
\ (p.~\pageref{ref_params_tests_profile})%
\index[funcref]{@\fidxl{params\_tests\_profile}}%
%
\item[Author:]%
Cengiz Gunay <cgunay@emory.edu>, 2004/10/17%
\end{description}
\methodline%
\subsubsection[Method \texttt{displayRankingsTeX}]{Method \texttt{params\_tests\_db/displayRankingsTeX}}%
\index[funcref]{params_tests_db@\fidxlb{params\_tests\_db}!displayRankingsTeX@\fidxl{displayRankingsTeX}}%
\label{ref_params_tests_db__displayRankingsTeX}%
\hypertarget{ref_params_tests_db__displayRankingsTeX}{}%
\begin{description}
\item[Summary:]Generates and displays a ranking DB by comparing rows of a\_db with the given match criteria.
%
\item[Usage:]~%
\begin{lyxcode}%
tex\_string = displayRankingsTeX(a\_db, crit\_db, props)
%
\end{lyxcode}%
%
\item[Description:]%
Generates a LaTeX document with:
	- Values of 10 best matching a\_db rows in a floating table.
	- (optional) Raw traces compared with some best matches at different distances
	- Parameter distributions of 50 best matches as a bar graph.
%%
\item[Parameters:]~
\begin{description}%
\item[\texttt{a\_db}:]
 A params\_tests\_db object to compare rows from.
\item[\texttt{crit\_db}:]
 A tests\_db object holding the match criterion tests and STDs

which can be created with matchingRow.\item[\texttt{props}:]
 A structure with any optional properties.
\begin{description}%
\item[\texttt{caption}:]
 Identification of the criterion db (not needed/used?).
\item[\texttt{a\_dataset}:]
 Dataset for a\_db.
\item[\texttt{a\_dball}:]
 The non-joined DB for for a\_db.
\item[\texttt{crit\_dataset}:]
 Dataset for crit\_db.
\item[\texttt{crit\_dball}:]
 Dataset for crit\_db.
\item[\texttt{num\_matches}:]
 Number of best matches to display (default=10).
\item[\texttt{rotate}:]
 Rotation angle for best matches table (default=90).
\end{description}%
\end{description}%
%
\item[Returns:]~

	tex\_string: LaTeX document string.
%
%
\item[See also:]%
\hyperlink{ref_rankVsDB}{\texttt{rankVsDB}}%
\ (p.~\pageref{ref_rankVsDB})%
\index[funcref]{@\fidxl{rankVsDB}}%
, \hyperlink{ref_displayRowsTeX}{\texttt{displayRowsTeX}}%
\ (p.~\pageref{ref_displayRowsTeX})%
\index[funcref]{@\fidxl{displayRowsTeX}}%
%
\item[Author:]%
Cengiz Gunay <cgunay@emory.edu>, 2004/10/20%
\end{description}
\methodline%
\subsubsection[Method \texttt{getParamRowIndices}]{Method \texttt{params\_tests\_db/getParamRowIndices}}%
\index[funcref]{params_tests_db@\fidxlb{params\_tests\_db}!getParamRowIndices@\fidxl{getParamRowIndices}}%
\label{ref_params_tests_db__getParamRowIndices}%
\hypertarget{ref_params_tests_db__getParamRowIndices}{}%
\begin{description}
\item[Summary:]Returns indices of rows with matching parameter values from rows of this db.
%
\item[Usage:]~%
\begin{lyxcode}%
row\_indices = getParamRowIndices(a\_db, rows, to\_db)
%
\end{lyxcode}%
%
%
\item[Parameters:]~
\begin{description}%
\item[\texttt{a\_db}:]
 A params\_tests\_db object.
\item[\texttt{rows}:]
 rows to find indices for.
\item[\texttt{to\_db}:]
 Where to find the matching rows.
\end{description}%
%
\item[Returns:]~

	row\_indices: Array of row indices.
%
%
\item[See also:]%
\hyperlink{ref_makeModifiedParamDB}{\texttt{makeModifiedParamDB}}%
\ (p.~\pageref{ref_makeModifiedParamDB})%
\index[funcref]{@\fidxl{makeModifiedParamDB}}%
, \hyperlink{ref_scanParamAllRows}{\texttt{scanParamAllRows}}%
\ (p.~\pageref{ref_scanParamAllRows})%
\index[funcref]{@\fidxl{scanParamAllRows}}%
, \hyperlink{ref_scaleParamsOneRow}{\texttt{scaleParamsOneRow}}%
\ (p.~\pageref{ref_scaleParamsOneRow})%
\index[funcref]{@\fidxl{scaleParamsOneRow}}%
, \hyperlink{ref_makeGenesisParFile}{\texttt{makeGenesisParFile}}%
\ (p.~\pageref{ref_makeGenesisParFile})%
\index[funcref]{@\fidxl{makeGenesisParFile}}%
%
\item[Author:]%
Cengiz Gunay <cgunay@emory.edu>, 2005/01/14%
\end{description}
\methodline%
\subsubsection[Method \texttt{plotParamsHists}]{Method \texttt{params\_tests\_db/plotParamsHists}}%
\index[funcref]{params_tests_db@\fidxlb{params\_tests\_db}!plotParamsHists@\fidxl{plotParamsHists}}%
\label{ref_params_tests_db__plotParamsHists}%
\hypertarget{ref_params_tests_db__plotParamsHists}{}%
\begin{description}
\item[Summary:]Create a horizontal plot\_stack of parameter histograms.
%
\item[Usage:]~%
\begin{lyxcode}%
a\_ps = plotParamsHists(a\_db, title\_str, props)
%
\end{lyxcode}%
%
\item[Description:]%
Skips the 'ItemIndex' test.
%%
\item[Parameters:]~
\begin{description}%
\item[\texttt{a\_db}:]
 A params\_tests\_db object.
\item[\texttt{title\_str}:]
 (Optional) A string to be concatanated to the title.
\item[\texttt{props}:]
 A structure with any optional properties.
\begin{description}%
\item[\texttt{quiet}:]
 Do not display the DB id on the plot title.
\end{description}%
\end{description}%
%
\item[Returns:]~

	a\_ps: A horizontal plot\_stack of plots
%
%
\item[See also:]%
\hyperlink{ref_plot_stack}{\texttt{plot\_stack}}%
\ (p.~\pageref{ref_plot_stack})%
\index[funcref]{@\fidxl{plot\_stack}}%
, \hyperlink{ref_paramsHists}{\texttt{paramsHists}}%
\ (p.~\pageref{ref_paramsHists})%
\index[funcref]{@\fidxl{paramsHists}}%
, \hyperlink{ref_plotEqSpaced}{\texttt{plotEqSpaced}}%
\ (p.~\pageref{ref_plotEqSpaced})%
\index[funcref]{@\fidxl{plotEqSpaced}}%
%
\item[Author:]%
Cengiz Gunay <cgunay@emory.edu>, 2005/04/07%
\end{description}
\methodline%
\subsubsection[Method \texttt{rankVsDB}]{Method \texttt{params\_tests\_db/rankVsDB}}%
\index[funcref]{params_tests_db@\fidxlb{params\_tests\_db}!rankVsDB@\fidxl{rankVsDB}}%
\label{ref_params_tests_db__rankVsDB}%
\hypertarget{ref_params_tests_db__rankVsDB}{}%
\begin{description}
\item[Summary:]Generates a ranking DB by comparing rows of this db with the given test criteria.
%
\item[Usage:]~%
\begin{lyxcode}%
a\_ranked\_db = rankVsDB(a\_db, crit\_db)
%
\end{lyxcode}%
%
\item[Description:]%
Distance is each measure difference divided by the STD in to\_db, squared and
 summed. Returned DB contains only the selected tests from crit\_db and the parameters
 from initial a\_db.
%%
\item[Parameters:]~
\begin{description}%
\item[\texttt{a\_db}:]
 A params\_tests\_db object to compare rows from.
\item[\texttt{crit\_db}:]
 A tests\_db object holding the match criterion tests and STDs

which can be created with matchingRow.\end{description}%
%
\item[Returns:]~

	a\_ranked\_db: The created DB with original rows and a distance measure, 
		   in ascending order. 
%
%
\item[See also:]%
\hyperlink{ref_matchingRow}{\texttt{matchingRow}}%
\ (p.~\pageref{ref_matchingRow})%
\index[funcref]{@\fidxl{matchingRow}}%
, \hyperlink{ref_rankMatching}{\texttt{rankMatching}}%
\ (p.~\pageref{ref_rankMatching})%
\index[funcref]{@\fidxl{rankMatching}}%
, \hyperlink{ref_joinRows}{\texttt{joinRows}}%
\ (p.~\pageref{ref_joinRows})%
\index[funcref]{@\fidxl{joinRows}}%
%
\item[Author:]%
Cengiz Gunay <cgunay@emory.edu>, 2004/10/20%
\end{description}
\methodline%
\subsubsection[Method \texttt{delColumns}]{Method \texttt{params\_tests\_db/delColumns}}%
\index[funcref]{params_tests_db@\fidxlb{params\_tests\_db}!delColumns@\fidxl{delColumns}}%
\label{ref_params_tests_db__delColumns}%
\hypertarget{ref_params_tests_db__delColumns}{}%
\begin{description}
\item[Summary:]Deletes columns from tests\_db.
%
\item[Usage:]~%
\begin{lyxcode}%
index = delColumns(obj, tests)
%
\end{lyxcode}%
%
\item[Description:]%
Overloaded function that maintains correct number of parameters. See
 original tests\_db/delColumns.
%%
\item[Parameters:]~
\begin{description}%
\item[\texttt{obj}:]
 A tests\_db object.
\item[\texttt{tests}:]
 Numbers or names of tests (see tests2cols)
\end{description}%
%
\item[Returns:]~

	obj: The tests\_db object that is missing the columns.
%
%
\item[See also:]%
\hyperlink{ref_tests_db__delColumns}{\texttt{tests\_db/delColumns}}%
\ (p.~\pageref{ref_tests_db__delColumns})%
\index[funcref]{tests_db@\fidxlb{tests\_db}!delColumns@\fidxl{delColumns}}%
%
\item[Author:]%
Cengiz Gunay <cgunay@emory.edu>, 2005/10/11%
\end{description}
\methodline%
\subsubsection[Method \texttt{paramsCoefs}]{Method \texttt{params\_tests\_db/paramsCoefs}}%
\index[funcref]{params_tests_db@\fidxlb{params\_tests\_db}!paramsCoefs@\fidxl{paramsCoefs}}%
\label{ref_params_tests_db__paramsCoefs}%
\hypertarget{ref_params_tests_db__paramsCoefs}{}%
\begin{description}
\item[Summary:]Calculates a corrcoefs\_db for each param and collects them in a cell array.
%
\item[Usage:]~%
\begin{lyxcode}%
p\_coefs = paramsCoefs(a\_db, p\_t3ds)
%
\end{lyxcode}%
%
\item[Description:]%
Skips the 'ItemIndex' test.
%%
\item[Parameters:]~
\begin{description}%
\item[\texttt{a\_db}:]
 A tests\_db object.
\item[\texttt{p\_t3ds}:]
 Cell array of invariant parameter databases.
\end{description}%
%
\item[Returns:]~

	p\_coefs: A cell array of corrcoefs\_dbs for each param in a\_db.
%
%
\item[See also:]%
\hyperlink{ref_params_tests_profile}{\texttt{params\_tests\_profile}}%
\ (p.~\pageref{ref_params_tests_profile})%
\index[funcref]{@\fidxl{params\_tests\_profile}}%
%
\item[Author:]%
Cengiz Gunay <cgunay@emory.edu>, 2004/10/17%
\end{description}
\methodline%
\subsubsection[Method \texttt{getProfile}]{Method \texttt{params\_tests\_db/getProfile}}%
\index[funcref]{params_tests_db@\fidxlb{params\_tests\_db}!getProfile@\fidxl{getProfile}}%
\label{ref_params_tests_db__getProfile}%
\hypertarget{ref_params_tests_db__getProfile}{}%
\begin{description}
\item[Summary:]Create a profile object from a params\_tests\_db by collecting
			 statistics.
%
\item[Usage:]~%
\begin{lyxcode}%
a\_pt\_profile = getProfile(a\_db, props)
%
\end{lyxcode}%
%
\item[Description:]%
Calculates the following results items:
	idx: Name-index pairs for accessing results arrays.
	t\_hists: Cell array of histograms of each test.
	p\_t3ds: Cell array of invariant relations of each parameter with all tests.
	pt\_hists: Cell array of separate test value histograms 
		for uniques value of each parameter.
	p\_stats: Cell array of test stats for each param.
	p\_coefs: Cell array of correlation coefficients 
		for each parameter with all tests.
	pt\_coefs\_hists: Cell matrix of histograms of coefficients from 
		correlations of each parameter with each test.
	pp\_coefs: Cell 3D matrix of mean coefficients from 
		correlations of each parameter with correlation 
		coefficients of each parameter with each test.		
%%
\item[Parameters:]~
\begin{description}%
\item[\texttt{a\_db}:]
 A params\_tests\_db object.
\item[\texttt{props}:]
 A structure with any optional properties.
\end{description}%
%
\item[Returns a params\_tests\_profile object.]~

%
%
\item[See also:]%
\hyperlink{ref_params_tests_profile}{\texttt{params\_tests\_profile}}%
\ (p.~\pageref{ref_params_tests_profile})%
\index[funcref]{@\fidxl{params\_tests\_profile}}%
, \hyperlink{ref_results_profile}{\texttt{results\_profile}}%
\ (p.~\pageref{ref_results_profile})%
\index[funcref]{@\fidxl{results\_profile}}%
, \hyperlink{ref_params_tests_db}{\texttt{params\_tests\_db}}%
\ (p.~\pageref{ref_params_tests_db})%
\index[funcref]{@\fidxl{params\_tests\_db}}%
, \hyperlink{ref_params_tests_fileset}{\texttt{params\_tests\_fileset}}%
\ (p.~\pageref{ref_params_tests_fileset})%
\index[funcref]{@\fidxl{params\_tests\_fileset}}%
, \hyperlink{ref_tests_db}{\texttt{tests\_db}}%
\ (p.~\pageref{ref_tests_db})%
\index[funcref]{@\fidxl{tests\_db}}%
, \hyperlink{ref_tests_3D_db}{\texttt{tests\_3D\_db}}%
\ (p.~\pageref{ref_tests_3D_db})%
\index[funcref]{@\fidxl{tests\_3D\_db}}%
, \hyperlink{ref_histogram_db}{\texttt{histogram\_db}}%
\ (p.~\pageref{ref_histogram_db})%
\index[funcref]{@\fidxl{histogram\_db}}%
, \hyperlink{ref_stats_db}{\texttt{stats\_db}}%
\ (p.~\pageref{ref_stats_db})%
\index[funcref]{@\fidxl{stats\_db}}%
, \hyperlink{ref_corrcoefs_db}{\texttt{corrcoefs\_db}}%
\ (p.~\pageref{ref_corrcoefs_db})%
\index[funcref]{@\fidxl{corrcoefs\_db}}%
%
\item[Author:]%
Cengiz Gunay <cgunay@emory.edu>, 2004/10/13%
\end{description}
\methodline%
\subsubsection[Method \texttt{plotVarBoxMatrix}]{Method \texttt{params\_tests\_db/plotVarBoxMatrix}}%
\index[funcref]{params_tests_db@\fidxlb{params\_tests\_db}!plotVarBoxMatrix@\fidxl{plotVarBoxMatrix}}%
\label{ref_params_tests_db__plotVarBoxMatrix}%
\hypertarget{ref_params_tests_db__plotVarBoxMatrix}{}%
\begin{description}
\item[Summary:]Create a stack of parameter-test variation plots 
		organized in a matrix.
%
\item[Usage:]~%
\begin{lyxcode}%
a\_plot\_stack = plotVarBoxMatrix(a\_db, p\_t3ds)
%
\end{lyxcode}%
%
\item[Description:]%
Skips the 'ItemIndex' test.
%%
\item[Parameters:]~
\begin{description}%
\item[\texttt{a\_db}:]
 A tests\_db object.
\item[\texttt{p\_t3ds}:]
 Cell array of invariant parameter databases.
\end{description}%
%
\item[Returns:]~

	a\_plot\_stack: A plot\_stack with the plots organized in matrix form
%
%
\item[See also:]%
\hyperlink{ref_params_tests_profile}{\texttt{params\_tests\_profile}}%
\ (p.~\pageref{ref_params_tests_profile})%
\index[funcref]{@\fidxl{params\_tests\_profile}}%
, \hyperlink{ref_plotVar}{\texttt{plotVar}}%
\ (p.~\pageref{ref_plotVar})%
\index[funcref]{@\fidxl{plotVar}}%
%
\item[Author:]%
Cengiz Gunay <cgunay@emory.edu>, 2004/10/17%
\end{description}
\methodline%
\subsubsection[Method \texttt{invarParams}]{Method \texttt{params\_tests\_db/invarParams}}%
\index[funcref]{params_tests_db@\fidxlb{params\_tests\_db}!invarParams@\fidxl{invarParams}}%
\label{ref_params_tests_db__invarParams}%
\hypertarget{ref_params_tests_db__invarParams}{}%
\begin{description}
\item[Summary:]Calculates invariant param dbs for all parameters and returns in an array.
%
\item[Usage:]~%
\begin{lyxcode}%
p\_t3ds = invarParams(a\_db)
%
\end{lyxcode}%
%
\item[Description:]%
Skips the 'ItemIndex' test.
%%
\item[Parameters:]~
\begin{description}%
\item[\texttt{a\_db}:]
 A tests\_db object.
\end{description}%
%
\item[Returns:]~

	p\_t3ds: An array of tests\_3D\_dbs for each param in a\_db.
%
%
\item[See also:]%
\hyperlink{ref_params_tests_profile}{\texttt{params\_tests\_profile}}%
\ (p.~\pageref{ref_params_tests_profile})%
\index[funcref]{@\fidxl{params\_tests\_profile}}%
%
\item[Author:]%
Cengiz Gunay <cgunay@emory.edu>, 2004/10/17%
\end{description}
\methodline%
\subsubsection[Method \texttt{getDualCIPdb}]{Method \texttt{params\_tests\_db/getDualCIPdb}}%
\index[funcref]{params_tests_db@\fidxlb{params\_tests\_db}!getDualCIPdb@\fidxl{getDualCIPdb}}%
\label{ref_params_tests_db__getDualCIPdb}%
\hypertarget{ref_params_tests_db__getDualCIPdb}{}%
\begin{description}
\item[Summary:]Generates a database by merging selected tests of depolarizing and hyperpolarizing cip results.
%
\item[Usage:]~%
\begin{lyxcode}%
a\_db = getDualCIPdb(db, depol\_tests, hyper\_tests, depol\_suffix, hyper\_suffix)
%
\end{lyxcode}%
%
\item[Description:]%
depol\_tests need to have the RowIndex column in it.
%%
\item[Parameters:]~
\begin{description}%
\item[\texttt{db}:]
 A params\_tests\_db object.
\end{description}%
%
\item[Returns:]~

	a\_db: A params\_tests\_db object of organized values.
%
\item[Example:]~
\begin{lyxcode}        >> control\_phys\_sdb = getDualCIPdb(control\_phys\_db, depol\_tests, hyper\_tests, '', 'Hyp100pA')\\%
        where depol\_tests and hyper\_tests are cell arrays of selected tests.\\%
\end{lyxcode}
%
\item[See also:]%
\hyperlink{ref_invarValues}{\texttt{invarValues}}%
\ (p.~\pageref{ref_invarValues})%
\index[funcref]{@\fidxl{invarValues}}%
, \hyperlink{ref_tests_3D_db}{\texttt{tests\_3D\_db}}%
\ (p.~\pageref{ref_tests_3D_db})%
\index[funcref]{@\fidxl{tests\_3D\_db}}%
, \hyperlink{ref_corrCoefs}{\texttt{corrCoefs}}%
\ (p.~\pageref{ref_corrCoefs})%
\index[funcref]{@\fidxl{corrCoefs}}%
, \hyperlink{ref_tests_3D_db__plotPair}{\texttt{tests\_3D\_db/plotPair}}%
\ (p.~\pageref{ref_tests_3D_db__plotPair})%
\index[funcref]{tests_3D_db@\fidxlb{tests\_3D\_db}!plotPair@\fidxl{plotPair}}%
%
\item[Author:]%
Cengiz Gunay <cgunay@emory.edu>, 2005/01/13%
\end{description}
\methodline%
\subsubsection[Method \texttt{scanParamAllRows}]{Method \texttt{params\_tests\_db/scanParamAllRows}}%
\index[funcref]{params_tests_db@\fidxlb{params\_tests\_db}!scanParamAllRows@\fidxl{scanParamAllRows}}%
\label{ref_params_tests_db__scanParamAllRows}%
\hypertarget{ref_params_tests_db__scanParamAllRows}{}%
\begin{description}
\item[Summary:]Scans given parameter range for each row in DB.
%
\item[Usage:]~%
\begin{lyxcode}%
a\_params\_db = scanParamAllRows(a\_db, param, min\_val, max\_val, num\_levels, props)
%
\end{lyxcode}%
%
\item[Description:]%
Produces rows by replacing the desired parameter value, in all rows of DB, 
 with num\_levels values between the given boundaries, min\_val and max\_val. 
 This results in a DB with num\_levels times more rows than the original DB. 
 Then, makeGenesisParFile can be used to generate a parameter file from 
 this DB to drive new simulations.
%%
\item[Parameters:]~
\begin{description}%
\item[\texttt{a\_db}:]
 A params\_tests\_db object whose first row is subject to modifications.
\item[\texttt{param}:]
 The parameter to be varied (see tests2cols for param description).
\item[\texttt{min\_val, max\_val}:]
 The low and high boundaries for the parameter value.
\item[\texttt{num\_levels}:]
 Number of levels to produce, including the boundaries.
\item[\texttt{props}:]
 A structure with any optional properties.
\begin{description}%
\item[\texttt{renameTrial}:]
 If given, the 'trial' column is renamed to this name.
\item[\texttt{levelFunc}:]
 Use this function to get the parameter range with 

feval(levelFunc, min\_val, max\_val, num\_levels). Example: 'logLevels'\end{description}%
\end{description}%
%
\item[Returns:]~

	a\_params\_db: A db only with params.
%
\item[Example:]~
\begin{lyxcode} Sets NaF to given range with 100 levels:\\%
 >> naf\_rows\_db = scanParamAllRows(a\_db(desired\_rows, :), 'NaF', 0, 1000, 100);\\%
\end{lyxcode}
%
\item[See also:]%
\hyperlink{ref_makeGenesisParFile}{\texttt{makeGenesisParFile}}%
\ (p.~\pageref{ref_makeGenesisParFile})%
\index[funcref]{@\fidxl{makeGenesisParFile}}%
, \hyperlink{ref_scaleParamsOneRow}{\texttt{scaleParamsOneRow}}%
\ (p.~\pageref{ref_scaleParamsOneRow})%
\index[funcref]{@\fidxl{scaleParamsOneRow}}%
, \hyperlink{ref_ranked_db__blockedDistances}{\texttt{ranked\_db/blockedDistances}}%
\ (p.~\pageref{ref_ranked_db__blockedDistances})%
\index[funcref]{ranked_db@\fidxlb{ranked\_db}!blockedDistances@\fidxl{blockedDistances}}%
, \hyperlink{ref_getParamRowIndices}{\texttt{getParamRowIndices}}%
\ (p.~\pageref{ref_getParamRowIndices})%
\index[funcref]{@\fidxl{getParamRowIndices}}%
, \hyperlink{ref_logLevels}{\texttt{logLevels}}%
\ (p.~\pageref{ref_logLevels})%
\index[funcref]{@\fidxl{logLevels}}%
%
\item[Author:]%
Cengiz Gunay <cgunay@emory.edu>, 2006/02/16%
\end{description}
\methodline%
\subsubsection[Method \texttt{scaleParamsOneRow}]{Method \texttt{params\_tests\_db/scaleParamsOneRow}}%
\index[funcref]{params_tests_db@\fidxlb{params\_tests\_db}!scaleParamsOneRow@\fidxl{scaleParamsOneRow}}%
\label{ref_params_tests_db__scaleParamsOneRow}%
\hypertarget{ref_params_tests_db__scaleParamsOneRow}{}%
\begin{description}
\item[Summary:]Scales chosen parameters in a row by multiplying with levels to create a new parameter db with as many rows as values in levels.
%
\item[Usage:]~%
\begin{lyxcode}%
a\_params\_db = scaleParamsOneRow(a\_db, params, levels)
%
\end{lyxcode}%
%
\item[Description:]%
Produces rows by multiplying desired params, in the first row of DB, 
 with each value in levels. Then, makeGenesisParFile can be used to generate
 a parameter file from this DB to drive new simulations.
%%
\item[Parameters:]~
\begin{description}%
\item[\texttt{a\_db}:]
 A params\_tests\_db object whose first row is subject to modifications.
\item[\texttt{params}:]
 Parameters to be varied (see tests2cols for param description).
\item[\texttt{levels}:]
 Column vector of parameter value multipliers (1=unity).
\end{description}%
%
\item[Returns:]~

	a\_params\_db: A db only with params.
%
\item[Example:]~
\begin{lyxcode} Blocks NaF from 0%-100% with 10% increments.\\%
 >> naf\_rows\_db = scanOneParam(a\_db(desired\_row, :), 'NaF', 0:0.1:1);\\%
\end{lyxcode}
%
\item[See also:]%
\hyperlink{ref_ranked_db__blockedDistances}{\texttt{ranked\_db/blockedDistances}}%
\ (p.~\pageref{ref_ranked_db__blockedDistances})%
\index[funcref]{ranked_db@\fidxlb{ranked\_db}!blockedDistances@\fidxl{blockedDistances}}%
, \hyperlink{ref_getParamRowIndices}{\texttt{getParamRowIndices}}%
\ (p.~\pageref{ref_getParamRowIndices})%
\index[funcref]{@\fidxl{getParamRowIndices}}%
, \hyperlink{ref_makeGenesisParFile}{\texttt{makeGenesisParFile}}%
\ (p.~\pageref{ref_makeGenesisParFile})%
\index[funcref]{@\fidxl{makeGenesisParFile}}%
%
\item[Author:]%
Cengiz Gunay <cgunay@emory.edu>, 2006/02/16%
\end{description}
\methodline%
\subsubsection[Method \texttt{makeModifiedParamDB}]{Method \texttt{params\_tests\_db/makeModifiedParamDB}}%
\index[funcref]{params_tests_db@\fidxlb{params\_tests\_db}!makeModifiedParamDB@\fidxl{makeModifiedParamDB}}%
\label{ref_params_tests_db__makeModifiedParamDB}%
\hypertarget{ref_params_tests_db__makeModifiedParamDB}{}%
\begin{description}
\item[Summary:]Modifies chosen parameters to create a new parameter db.
%
\item[Usage:]~%
\begin{lyxcode}%
a\_params\_db = makeModifiedParamDB(a\_db, params, levels, props)
%
\end{lyxcode}%
%
\item[Description:]%
For a given a\_db row, produces rows by multiplying desired params for each value in levels.
 If specified, adds each value of cipLevels element in separate rows. If trialStart
 is given, a counter is also set. Then, makeGenesisParFile can be used to generate
 a parameter file from this DB to drive new simulations. Trial numbers must start
 from 1 for a new parameter file, or offset from the end of an existing parameter file. 
%%
\item[Parameters:]~
\begin{description}%
\item[\texttt{a\_db}:]
 A params\_tests\_db object with rows parameter modifications to be applied.
\item[\texttt{params}:]
 Parameters to be varied (see tests2cols for param description).
\item[\texttt{levels}:]
 Vector of multipliers for generating different parameter levels (1=unity).
\item[\texttt{props}:]
 A structure with any optional properties.
\begin{description}%
\item[\texttt{cipLevels}:]
 If given, replicates each row with these pAcip parameters.
\item[\texttt{trialStart}:]
 If given, adds/replaces the trial parameter and counts forward.
\end{description}%
\end{description}%
%
\item[Returns:]~

	a\_params\_db: A db only with params.
%
\item[Example:]~
\begin{lyxcode} First get ranked\_db:\\%
 >> ranked\_for\_gps0501a\_db = ...\\%
      joinOriginal(rankMatching(sdball, matchingRow(mini\_ttx\_control\_rsrdb, 1)))\\%
 Then, get 10%-increment blocking for parameters in the first (best-matching) row\\%
 >> blocked\_rows\_db = ...\\%
      makeModifiedParamDB(ranked\_for\_gps0501a\_db(1, :), {'NaF', 'NaP'}, 0:.1:1, ...\\%
                           struct('cipLevels', [-100 100]));\\%
\end{lyxcode}
%
\item[See also:]%
\hyperlink{ref_ranked_db__blockedDistances}{\texttt{ranked\_db/blockedDistances}}%
\ (p.~\pageref{ref_ranked_db__blockedDistances})%
\index[funcref]{ranked_db@\fidxlb{ranked\_db}!blockedDistances@\fidxl{blockedDistances}}%
, \hyperlink{ref_getParamRowIndices}{\texttt{getParamRowIndices}}%
\ (p.~\pageref{ref_getParamRowIndices})%
\index[funcref]{@\fidxl{getParamRowIndices}}%
, \hyperlink{ref_makeGenesisParFile}{\texttt{makeGenesisParFile}}%
\ (p.~\pageref{ref_makeGenesisParFile})%
\index[funcref]{@\fidxl{makeGenesisParFile}}%
%
\item[Author:]%
Cengiz Gunay <cgunay@emory.edu>, 2005/01/14 (originally called getBlockedParamRows)%
\end{description}
\methodline%
\subsection{Class \texttt{params\_tests\_fileset}}%
\index[funcref]{params_tests_fileset@\fidxlb{params\_tests\_fileset}}%
\label{ref_params_tests_fileset}%
\hypertarget{ref_params_tests_fileset}{}%
\subsubsection[Constructor \texttt{params\_tests\_fileset}]{Constructor \texttt{params\_tests\_fileset/params\_tests\_fileset}}%
\index[funcref]{params_tests_fileset@\fidxlb{params\_tests\_fileset}!params_tests_fileset@\fidxl{params\_tests\_fileset}}%
\label{ref_params_tests_fileset__params_tests_fileset}%
\hypertarget{ref_params_tests_fileset__params_tests_fileset}{}%
\begin{description}
\item[Summary:]Description of a set of data files of raw data varying with parameter values.
%
\item[Usage:]~%
\begin{lyxcode}%
obj = params\_tests\_fileset(file\_pattern, dt, dy, id, props)
%
\end{lyxcode}%
%
\item[Description:]%
This is a subclass of params\_tests\_dataset. This class is used to generate 
 params\_tests\_db objects and keep 
 a connection to the raw data files. This class only keeps names of
 files and loads raw data files whenever it's requested. A database
 object can easily be generated using the convertion methods.
 Most methods defined here can 
 be used as-is, however some should be overloaded in subclasses. 
 The specific methods are loadItemProfile.
%%
\item[Parameters:]~
\begin{description}%
\item[\texttt{file\_pattern}:]
 File pattern, or cell array of patterns, matching all 

files to be loaded.\item[\texttt{dt}:]
 Time resolution [s]
\item[\texttt{dy}:]
 y-axis resolution [ISI (V, A, etc.)]
\item[\texttt{id}:]
 An identification string
\item[\texttt{props}:]
 A structure with any optional properties.
\begin{description}%
\item[\texttt{num\_params}:]
 Number of parameters that appear in filenames.
\item[\texttt{param\_row\_filename}:]
 If given, the 'trial' parameter will be used

to address rows from this file and acquire parameters.\item[\texttt{param\_desc\_filename}:]
 Contains the parameter range descriptions one per 

each row. The parameter names are acquired from this file.\item[\texttt{param\_names}:]
 Cell array of parameter names corresponding to the 

param\_row\_filename columns can be specified as an alternative to
specifying param\_desc\_filename. These names are not for the 
parameters present in the data filename.\item[\texttt{profile\_method\_name}:]
 It can be one of the profile-creating methods in this

class. E.g., 'trace\_profile', 'srp\_trace\_profile', etc.
(See parent classes and cip\_trace object for more props)\end{description}%
\end{description}%
%
\item[Returns a structure object with the following fields:]~

	params\_tests\_dataset,
	path: The pathname to files.
%
%
\item[See also:]%
\hyperlink{ref_params_tests_db}{\texttt{params\_tests\_db}}%
\ (p.~\pageref{ref_params_tests_db})%
\index[funcref]{@\fidxl{params\_tests\_db}}%
, \hyperlink{ref_tests_db}{\texttt{tests\_db}}%
\ (p.~\pageref{ref_tests_db})%
\index[funcref]{@\fidxl{tests\_db}}%
, \hyperlink{ref_test_variable_db (N__I)}{\texttt{test\_variable\_db (N/I)}}%
\ (p.~\pageref{ref_test_variable_db (N__I)})%
\index[funcref]{test_variable_db (N@\fidxlb{test\_variable\_db (N}!I)@\fidxl{I)}}%
%
\item[Author:]%
Cengiz Gunay <cgunay@emory.edu>, 2004/09/09%
\end{description}
\methodline%
\subsubsection[Method \texttt{addFiles}]{Method \texttt{params\_tests\_fileset/addFiles}}%
\index[funcref]{params_tests_fileset@\fidxlb{params\_tests\_fileset}!addFiles@\fidxl{addFiles}}%
\label{ref_params_tests_fileset__addFiles}%
\hypertarget{ref_params_tests_fileset__addFiles}{}%
\begin{description}
\item[Summary:]Adds to existing list of files in set.
%
\item[Usage:]~%
\begin{lyxcode}%
[a\_fileset, index\_list] = addFiles(a\_fileset, file\_pattern, props)
%
\end{lyxcode}%
%
%
\item[Parameters:]~
\begin{description}%
\item[\texttt{a\_fileset}:]
 A params\_tests\_fileset object.
\item[\texttt{file\_pattern}:]
 File pattern, or cell array of patterns, matching additional files.
\item[\texttt{props}:]
 A structure with any optional properties.
\begin{description}%
\item[\texttt{param\_row\_filename}:]
 Update parameters from here. The 'trial' parameter is used

to address rows from this file and acquire parameters.\end{description}%
\end{description}%
%
\item[Returns:]~

	a\_fileset: The augmented fileset object.
	index\_list: The vector of index numbers of the new files added. Can be used
		to selectively load the new files into a DB using params\_test\_db.
%
%
\item[See also:]%
\hyperlink{ref_params_tests_fileset}{\texttt{params\_tests\_fileset}}%
\ (p.~\pageref{ref_params_tests_fileset})%
\index[funcref]{@\fidxl{params\_tests\_fileset}}%
, \hyperlink{ref_params_tests_dataset__params_test_db.}{\texttt{params\_tests\_dataset/params\_test\_db.}}%
\ (p.~\pageref{ref_params_tests_dataset__params_test_db.})%
\index[funcref]{params_tests_dataset@\fidxlb{params\_tests\_dataset}!params_test_db.@\fidxl{params\_test\_db.}}%
%
\item[Author:]%
Cengiz Gunay <cgunay@emory.edu>, 2006/02/01%
\end{description}
\methodline%
\subsubsection[Method \texttt{display}]{Method \texttt{params\_tests\_fileset/display}}%
\index[funcref]{params_tests_fileset@\fidxlb{params\_tests\_fileset}!display@\fidxl{display}}%
\label{ref_params_tests_fileset__display}%
\hypertarget{ref_params_tests_fileset__display}{}%
\begin{description}
%
%
%
%
%
%
%
\item[Author:]%
Cengiz Gunay <cgunay@emory.edu>, 2004/08/04%
\end{description}
\methodline%
\subsubsection[Method \texttt{get}]{Method \texttt{params\_tests\_fileset/get}}%
\index[funcref]{params_tests_fileset@\fidxlb{params\_tests\_fileset}!get@\fidxl{get}}%
\label{ref_params_tests_fileset__get}%
\hypertarget{ref_params_tests_fileset__get}{}%
\begin{description}
\item[Summary:]Defines generic attribute retrieval for objects.
%
%
%
%
%
%
%
\item[Author:]%
Cengiz Gunay <cgunay@emory.edu>, 2004/09/14%
\end{description}
\methodline%
\subsubsection[Method \texttt{trace}]{Method \texttt{params\_tests\_fileset/trace}}%
\index[funcref]{params_tests_fileset@\fidxlb{params\_tests\_fileset}!trace@\fidxl{trace}}%
\label{ref_params_tests_fileset__trace}%
\hypertarget{ref_params_tests_fileset__trace}{}%
\begin{description}
\item[Summary:]Loads a raw trace given a file\_index to this fileset.
%
\item[Usage:]~%
\begin{lyxcode}%
a\_trace = trace(fileset, file\_index)
%
\end{lyxcode}%
%
%
\item[Parameters:]~
\begin{description}%
\item[\texttt{fileset}:]
 A params\_tests\_fileset.
\item[\texttt{file\_index}:]
 Index of file in fileset.
\end{description}%
%
\item[Returns:]~

	a\_trace: A trace object.
%
%
\item[See also:]%
\hyperlink{ref_trace}{\texttt{trace}}%
\ (p.~\pageref{ref_trace})%
\index[funcref]{@\fidxl{trace}}%
, \hyperlink{ref_params_tests_fileset}{\texttt{params\_tests\_fileset}}%
\ (p.~\pageref{ref_params_tests_fileset})%
\index[funcref]{@\fidxl{params\_tests\_fileset}}%
%
\item[Author:]%
Cengiz Gunay <cgunay@emory.edu>, 2004/09/13%
\end{description}
\methodline%
\subsubsection[Method \texttt{paramNames}]{Method \texttt{params\_tests\_fileset/paramNames}}%
\index[funcref]{params_tests_fileset@\fidxlb{params\_tests\_fileset}!paramNames@\fidxl{paramNames}}%
\label{ref_params_tests_fileset__paramNames}%
\hypertarget{ref_params_tests_fileset__paramNames}{}%
\begin{description}
\item[Summary:]Returns the ordered names of parameters for this fileset.
%
\item[Usage:]~%
\begin{lyxcode}%
param\_names = paramNames(fileset, item)
%
\end{lyxcode}%
%
\item[Description:]%
Looks at the filename of the first file to find the parameter names.
%%
\item[Parameters:]~
\begin{description}%
\item[\texttt{fileset}:]
 A params\_tests\_fileset.
\item[\texttt{item}:]
 (Optional) If given, read param names by loading item at this index.
\end{description}%
%
\item[Returns:]~

	params\_names: Cell array with ordered parameter names.
%
%
\item[See also:]%
\hyperlink{ref_params_tests_fileset}{\texttt{params\_tests\_fileset}}%
\ (p.~\pageref{ref_params_tests_fileset})%
\index[funcref]{@\fidxl{params\_tests\_fileset}}%
, \hyperlink{ref_paramNames}{\texttt{paramNames}}%
\ (p.~\pageref{ref_paramNames})%
\index[funcref]{@\fidxl{paramNames}}%
, \hyperlink{ref_testNames}{\texttt{testNames}}%
\ (p.~\pageref{ref_testNames})%
\index[funcref]{@\fidxl{testNames}}%
%
\item[Author:]%
Cengiz Gunay <cgunay@emory.edu>, 2004/09/10%
\end{description}
\methodline%
\subsubsection[Method \texttt{getItemParams}]{Method \texttt{params\_tests\_fileset/getItemParams}}%
\index[funcref]{params_tests_fileset@\fidxlb{params\_tests\_fileset}!getItemParams@\fidxl{getItemParams}}%
\label{ref_params_tests_fileset__getItemParams}%
\hypertarget{ref_params_tests_fileset__getItemParams}{}%
\begin{description}
\item[Summary:]Get the parameter values of a dataset item.
%
\item[Usage:]~%
\begin{lyxcode}%
params\_row = getItemParams(dataset, index)
%
\end{lyxcode}%
%
%
\item[Parameters:]~
\begin{description}%
\item[\texttt{dataset}:]
 A params\_tests\_dataset.
\item[\texttt{index}:]
 Index of item in dataset.
\end{description}%
%
\item[Returns:]~

	params\_row: Parameter values in the same order of paramNames
%
%
\item[See also:]%
\hyperlink{ref_itemResultsRow}{\texttt{itemResultsRow}}%
\ (p.~\pageref{ref_itemResultsRow})%
\index[funcref]{@\fidxl{itemResultsRow}}%
, \hyperlink{ref_params_tests_dataset}{\texttt{params\_tests\_dataset}}%
\ (p.~\pageref{ref_params_tests_dataset})%
\index[funcref]{@\fidxl{params\_tests\_dataset}}%
, \hyperlink{ref_paramNames}{\texttt{paramNames}}%
\ (p.~\pageref{ref_paramNames})%
\index[funcref]{@\fidxl{paramNames}}%
, \hyperlink{ref_testNames}{\texttt{testNames}}%
\ (p.~\pageref{ref_testNames})%
\index[funcref]{@\fidxl{testNames}}%
%
\item[Author:]%
Cengiz Gunay <cgunay@emory.edu>, 2004/12/03%
\end{description}
\methodline%
\subsubsection[Method \texttt{trace\_profile}]{Method \texttt{params\_tests\_fileset/trace\_profile}}%
\index[funcref]{params_tests_fileset@\fidxlb{params\_tests\_fileset}!trace_profile@\fidxl{trace\_profile}}%
\label{ref_params_tests_fileset__trace_profile}%
\hypertarget{ref_params_tests_fileset__trace_profile}{}%
\begin{description}
\item[Summary:]Loads a raw trace\_profile given a file\_index to this fileset.
%
\item[Usage:]~%
\begin{lyxcode}%
a\_trace\_profile = trace\_profile(fileset, file\_index)
%
\end{lyxcode}%
%
%
\item[Parameters:]~
\begin{description}%
\item[\texttt{fileset}:]
 A params\_tests\_fileset.
\item[\texttt{file\_index}:]
 Index of file in fileset.
\end{description}%
%
\item[Returns:]~

	a\_trace\_profile: A trace\_profile object.
%
%
\item[See also:]%
\hyperlink{ref_trace_profile}{\texttt{trace\_profile}}%
\ (p.~\pageref{ref_trace_profile})%
\index[funcref]{@\fidxl{trace\_profile}}%
, \hyperlink{ref_params_tests_fileset}{\texttt{params\_tests\_fileset}}%
\ (p.~\pageref{ref_params_tests_fileset})%
\index[funcref]{@\fidxl{params\_tests\_fileset}}%
%
\item[Author:]%
Cengiz Gunay <cgunay@emory.edu>, 2004/09/13%
\end{description}
\methodline%
\subsubsection[Method \texttt{loadItemProfile}]{Method \texttt{params\_tests\_fileset/loadItemProfile}}%
\index[funcref]{params_tests_fileset@\fidxlb{params\_tests\_fileset}!loadItemProfile@\fidxl{loadItemProfile}}%
\label{ref_params_tests_fileset__loadItemProfile}%
\hypertarget{ref_params_tests_fileset__loadItemProfile}{}%
\begin{description}
\item[Summary:]Loads a profile object from a raw data file in the fileset.
%
\item[Usage:]~%
\begin{lyxcode}%
a\_profile = loadItemProfile(fileset, file\_index)
%
\end{lyxcode}%
%
\item[Description:]%
Subclasses should overload this function to load the specific profile
 object they desire. The profile class should define a getResults method
 which is used in the itemResultsRow method.
%%
\item[Parameters:]~
\begin{description}%
\item[\texttt{fileset}:]
 A params\_tests\_fileset.
\item[\texttt{file\_index}:]
 Index of file in fileset.
\end{description}%
%
\item[Returns:]~

	a\_profile: A profile object that implements the getResults method.
%
%
\item[See also:]%
\hyperlink{ref_itemResultsRow}{\texttt{itemResultsRow}}%
\ (p.~\pageref{ref_itemResultsRow})%
\index[funcref]{@\fidxl{itemResultsRow}}%
, \hyperlink{ref_params_tests_fileset}{\texttt{params\_tests\_fileset}}%
\ (p.~\pageref{ref_params_tests_fileset})%
\index[funcref]{@\fidxl{params\_tests\_fileset}}%
, \hyperlink{ref_paramNames}{\texttt{paramNames}}%
\ (p.~\pageref{ref_paramNames})%
\index[funcref]{@\fidxl{paramNames}}%
, \hyperlink{ref_testNames}{\texttt{testNames}}%
\ (p.~\pageref{ref_testNames})%
\index[funcref]{@\fidxl{testNames}}%
%
\item[Author:]%
Cengiz Gunay <cgunay@emory.edu>, 2004/09/14%
\end{description}
\methodline%
\subsection{Class \texttt{params\_tests\_profile}}%
\index[funcref]{params_tests_profile@\fidxlb{params\_tests\_profile}}%
\label{ref_params_tests_profile}%
\hypertarget{ref_params_tests_profile}{}%
\subsubsection[Constructor \texttt{params\_tests\_profile}]{Constructor \texttt{params\_tests\_profile/params\_tests\_profile}}%
\index[funcref]{params_tests_profile@\fidxlb{params\_tests\_profile}!params_tests_profile@\fidxl{params\_tests\_profile}}%
\label{ref_params_tests_profile__params_tests_profile}%
\hypertarget{ref_params_tests_profile__params_tests_profile}{}%
\begin{description}
\item[Summary:]Holds the results profile from a params\_tests\_db.
%
\item[Usage:]~%
\begin{lyxcode}%
a\_pt\_profile = params\_tests\_profile(results, a\_db, props)
%
\end{lyxcode}%
%
%
\item[Parameters:]~
\begin{description}%
\item[\texttt{a\_db}:]
 A params\_tests\_db object.
\item[\texttt{results}:]
 A structure containing test results.
\item[\texttt{props}:]
 A structure with any optional properties.
\end{description}%
%
\item[Returns a structure object with the following fields:]~

	results\_profile: Contains results of tests.
	db: The params\_tests\_db.
	props.
%
%
\item[See also:]%
\hyperlink{ref_results_profile}{\texttt{results\_profile}}%
\ (p.~\pageref{ref_results_profile})%
\index[funcref]{@\fidxl{results\_profile}}%
, \hyperlink{ref_params_tests_db__params_tests_profile}{\texttt{params\_tests\_db/params\_tests\_profile}}%
\ (p.~\pageref{ref_params_tests_db__params_tests_profile})%
\index[funcref]{params_tests_db@\fidxlb{params\_tests\_db}!params_tests_profile@\fidxl{params\_tests\_profile}}%
%
\item[Author:]%
Cengiz Gunay <cgunay@emory.edu>, 2004/10/13%
\end{description}
\methodline%
\subsubsection[Method \texttt{get}]{Method \texttt{params\_tests\_profile/get}}%
\index[funcref]{params_tests_profile@\fidxlb{params\_tests\_profile}!get@\fidxl{get}}%
\label{ref_params_tests_profile__get}%
\hypertarget{ref_params_tests_profile__get}{}%
\begin{description}
\item[Summary:]Defines generic attribute retrieval for objects.
%
%
%
%
%
%
%
\item[Author:]%
Cengiz Gunay <cgunay@emory.edu>, 2004/09/14%
\end{description}
\methodline%
\subsection{Class \texttt{period}}%
\index[funcref]{period@\fidxlb{period}}%
\label{ref_period}%
\hypertarget{ref_period}{}%
\subsubsection[Constructor \texttt{period}]{Constructor \texttt{period/period}}%
\index[funcref]{period@\fidxlb{period}!period@\fidxl{period}}%
\label{ref_period__period}%
\hypertarget{ref_period__period}{}%
\begin{description}
\item[Summary:]Start and end times of a period in terms of the dt of the trace
	to which belongs.
%
\item[Usage:]~%
\begin{lyxcode}%
obj = period(start\_time, end\_time)
%
\end{lyxcode}%
%
%
\item[Parameters:]~

(see below for the rest)%
\item[Returns a structure object with the following fields:]~

	start\_time, end\_time: Inclusive period [dt].
%
%
\item[See also:]%
\hyperlink{ref_trace}{\texttt{trace}}%
\ (p.~\pageref{ref_trace})%
\index[funcref]{@\fidxl{trace}}%
, \hyperlink{ref_spikes}{\texttt{spikes}}%
\ (p.~\pageref{ref_spikes})%
\index[funcref]{@\fidxl{spikes}}%
, \hyperlink{ref_spike_shape}{\texttt{spike\_shape}}%
\ (p.~\pageref{ref_spike_shape})%
\index[funcref]{@\fidxl{spike\_shape}}%
%
\item[Author:]%
Cengiz Gunay <cgunay@emory.edu>, 2004/07/30%
\end{description}
\methodline%
\subsubsection[Method \texttt{display}]{Method \texttt{period/display}}%
\index[funcref]{period@\fidxlb{period}!display@\fidxl{display}}%
\label{ref_period__display}%
\hypertarget{ref_period__display}{}%
\begin{description}
%
%
%
%
%
%
%
\item[Author:]%
Cengiz Gunay <cgunay@emory.edu>, 2004/08/04%
\end{description}
\methodline%
\subsubsection[Method \texttt{get}]{Method \texttt{period/get}}%
\index[funcref]{period@\fidxlb{period}!get@\fidxl{get}}%
\label{ref_period__get}%
\hypertarget{ref_period__get}{}%
\begin{description}
\item[Summary:]Defines generic attribute retrieval for objects.
%
%
%
%
%
%
%
\item[Author:]%
Cengiz Gunay <cgunay@emory.edu>, 2004/09/14%
\end{description}
\methodline%
\subsubsection[Method \texttt{set}]{Method \texttt{period/set}}%
\index[funcref]{period@\fidxlb{period}!set@\fidxl{set}}%
\label{ref_period__set}%
\hypertarget{ref_period__set}{}%
\begin{description}
\item[Summary:]Generic method for setting object attributes.
%
%
%
%
%
%
%
\item[Author:]%
Cengiz Gunay <cgunay@emory.edu>, 2004/10/08%
\end{description}
\methodline%
\subsubsection[Method \texttt{subsref}]{Method \texttt{period/subsref}}%
\index[funcref]{period@\fidxlb{period}!subsref@\fidxl{subsref}}%
\label{ref_period__subsref}%
\hypertarget{ref_period__subsref}{}%
\begin{description}
\item[Summary:]Defines generic indexing for objects.
%
%
%
%
%
%
%
%
\end{description}
\methodline%
\subsubsection[Method \texttt{SpikeTimesinPeriod}]{Method \texttt{period/SpikeTimesinPeriod}}%
\index[funcref]{period@\fidxlb{period}!SpikeTimesinPeriod@\fidxl{SpikeTimesinPeriod}}%
\label{ref_period__SpikeTimesinPeriod}%
\hypertarget{ref_period__SpikeTimesinPeriod}{}%
\begin{description}
%
\item[Usage:]~%
\begin{lyxcode}%
SpkTimes=Interval(times, period)
%
\end{lyxcode}%
%
%
\item[Parameters:]~
\begin{description}%
\item[\texttt{times}:]
 an array of spike times.
\item[\texttt{period}:]
 A period object
\end{description}%
%
\item[Returns:]~

	the\_period: The cropped set of spike times that fall within a period.
%
%
\item[See also:]%
\hyperlink{ref_period}{\texttt{period}}%
\ (p.~\pageref{ref_period})%
\index[funcref]{@\fidxl{period}}%
, \hyperlink{ref_cip_trace}{\texttt{cip\_trace}}%
\ (p.~\pageref{ref_cip_trace})%
\index[funcref]{@\fidxl{cip\_trace}}%
, \hyperlink{ref_trace}{\texttt{trace}}%
\ (p.~\pageref{ref_trace})%
\index[funcref]{@\fidxl{trace}}%
, \hyperlink{ref_spikes}{\texttt{spikes}}%
\ (p.~\pageref{ref_spikes})%
\index[funcref]{@\fidxl{spikes}}%
%
\item[Author:]%
Tom Sangrey, 2006/01/26%
\end{description}
\methodline%
\subsection{Class \texttt{physiol\_bundle}}%
\index[funcref]{physiol_bundle@\fidxlb{physiol\_bundle}}%
\label{ref_physiol_bundle}%
\hypertarget{ref_physiol_bundle}{}%
\subsubsection[Constructor \texttt{physiol\_bundle}]{Constructor \texttt{physiol\_bundle/physiol\_bundle}}%
\index[funcref]{physiol_bundle@\fidxlb{physiol\_bundle}!physiol_bundle@\fidxl{physiol\_bundle}}%
\label{ref_physiol_bundle__physiol_bundle}%
\hypertarget{ref_physiol_bundle__physiol_bundle}{}%
\begin{description}
\item[Summary:]The physiology dataset and the DB created from it bundled together.
%
\item[Usage:]~%
\begin{lyxcode}%
a\_bundle = physiol\_bundle(a\_dataset, a\_db, a\_joined\_db, props)
%
\end{lyxcode}%
%
\item[Description:]%
This is a subclass of dataset\_db\_bundle, specialized for physiology datasets. 
%%
\item[Parameters:]~
\begin{description}%
\item[\texttt{a\_dataset}:]
 A physiol\_cip\_traceset\_fileset object.
\item[\texttt{a\_db}:]
 The raw params\_tests\_db object created from the dataset. 

It only needs to have the pAcip, pAbias, TracesetIndex, and ItemIndex columns.\item[\texttt{a\_joined\_db}:]
 The one-treatment-per-line DB created from the raw DB.
\item[\texttt{props}:]
 A structure with any optional properties.
\end{description}%
%
\item[Returns a structure object with the following fields:]~

	dataset\_db\_bundle, 
	joined\_control\_db: DB of control neurons (no pharmacological applications).
%
%
\item[See also:]%
\hyperlink{ref_dataset_db_bundle}{\texttt{dataset\_db\_bundle}}%
\ (p.~\pageref{ref_dataset_db_bundle})%
\index[funcref]{@\fidxl{dataset\_db\_bundle}}%
, \hyperlink{ref_tests_db}{\texttt{tests\_db}}%
\ (p.~\pageref{ref_tests_db})%
\index[funcref]{@\fidxl{tests\_db}}%
, \hyperlink{ref_params_tests_dataset}{\texttt{params\_tests\_dataset}}%
\ (p.~\pageref{ref_params_tests_dataset})%
\index[funcref]{@\fidxl{params\_tests\_dataset}}%
%
\item[Author:]%
Cengiz Gunay <cgunay@emory.edu>, 2005/12/13%
\end{description}
\methodline%
\subsubsection[Method \texttt{getNeuronLabel}]{Method \texttt{physiol\_bundle/getNeuronLabel}}%
\index[funcref]{physiol_bundle@\fidxlb{physiol\_bundle}!getNeuronLabel@\fidxl{getNeuronLabel}}%
\label{ref_physiol_bundle__getNeuronLabel}%
\hypertarget{ref_physiol_bundle__getNeuronLabel}{}%
\begin{description}
\item[Summary:]Constructs the neuron label from dataset.
%
\item[Usage:]~%
\begin{lyxcode}%
a\_label = getNeuronLabel(a\_bundle, traceset\_index, props)
%
\end{lyxcode}%
%
%
\item[Parameters:]~
\begin{description}%
\item[\texttt{a\_bundle}:]
 A physiol\_cip\_traceset\_fileset object.
\item[\texttt{traceset\_index}:]
 The traceset index of neuron.
\item[\texttt{props}:]
 A structure with any optional properties.
\end{description}%
%
\item[Returns:]~

	a\_label: A string label identifying selected neuron in bundle.
%
%
\item[See also:]%
\hyperlink{ref_dataset_db_bundle}{\texttt{dataset\_db\_bundle}}%
\ (p.~\pageref{ref_dataset_db_bundle})%
\index[funcref]{@\fidxl{dataset\_db\_bundle}}%
%
\item[Author:]%
Cengiz Gunay <cgunay@emory.edu>, 2006/05/05%
\end{description}
\methodline%
\subsubsection[Method \texttt{get}]{Method \texttt{physiol\_bundle/get}}%
\index[funcref]{physiol_bundle@\fidxlb{physiol\_bundle}!get@\fidxl{get}}%
\label{ref_physiol_bundle__get}%
\hypertarget{ref_physiol_bundle__get}{}%
\begin{description}
\item[Summary:]Defines generic attribute retrieval for objects.
%
%
%
%
%
%
%
\item[Author:]%
Cengiz Gunay <cgunay@emory.edu>, 2004/09/14%
\end{description}
\methodline%
\subsubsection[Method \texttt{set}]{Method \texttt{physiol\_bundle/set}}%
\index[funcref]{physiol_bundle@\fidxlb{physiol\_bundle}!set@\fidxl{set}}%
\label{ref_physiol_bundle__set}%
\hypertarget{ref_physiol_bundle__set}{}%
\begin{description}
\item[Summary:]Generic method for setting object attributes.
%
%
%
%
%
%
%
\item[Author:]%
Cengiz Gunay <cgunay@emory.edu>, 2004/10/08%
\end{description}
\methodline%
\subsubsection[Method \texttt{constrainedMeasuresPreset}]{Method \texttt{physiol\_bundle/constrainedMeasuresPreset}}%
\index[funcref]{physiol_bundle@\fidxlb{physiol\_bundle}!constrainedMeasuresPreset@\fidxl{constrainedMeasuresPreset}}%
\label{ref_physiol_bundle__constrainedMeasuresPreset}%
\hypertarget{ref_physiol_bundle__constrainedMeasuresPreset}{}%
\begin{description}
\item[Summary:]Returns a physiol\_bundle with constrained measures according to chosen preset.
%
\item[Usage:]~%
\begin{lyxcode}%
[a\_pbundle test\_names] = constrainedMeasuresPreset(a\_pbundle, preset, props)
%
\end{lyxcode}%
%
%
\item[Parameters:]~
\begin{description}%
\item[\texttt{a\_pbundle}:]
 A physiol\_cip\_traceset\_fileset object.
\item[\texttt{preset}:]
 Choose preset measure list (default=1).
\item[\texttt{props}:]
 A structure with any optional properties.
\end{description}%
%
\item[Returns:]~

	a\_pbundle: One or more cip\_trace object that holds the raw data.
%
%
\item[See also:]%
\hyperlink{ref_loadItemProfile}{\texttt{loadItemProfile}}%
\ (p.~\pageref{ref_loadItemProfile})%
\index[funcref]{@\fidxl{loadItemProfile}}%
, \hyperlink{ref_physiol_cip_traceset__cip_trace}{\texttt{physiol\_cip\_traceset/cip\_trace}}%
\ (p.~\pageref{ref_physiol_cip_traceset__cip_trace})%
\index[funcref]{physiol_cip_traceset@\fidxlb{physiol\_cip\_traceset}!cip_trace@\fidxl{cip\_trace}}%
%
\item[Author:]%
Cengiz Gunay <cgunay@emory.edu>, 2006/01/19%
\end{description}
\methodline%
\subsubsection[Method \texttt{matchingRow}]{Method \texttt{physiol\_bundle/matchingRow}}%
\index[funcref]{physiol_bundle@\fidxlb{physiol\_bundle}!matchingRow@\fidxl{matchingRow}}%
\label{ref_physiol_bundle__matchingRow}%
\hypertarget{ref_physiol_bundle__matchingRow}{}%
\begin{description}
\item[Summary:]Creates a criterion database for matching the neuron at traceset\_index.
%
\item[Usage:]~%
\begin{lyxcode}%
a\_crit\_db = matchingRow(p\_bundle, traceset\_index, props)
%
\end{lyxcode}%
%
\item[Description:]%
Copies selected test values from row as the first row into the 
 criterion db. Adds a second row for the STD of each column in the db.
%%
\item[Parameters:]~
\begin{description}%
\item[\texttt{p\_bundle}:]
 A physiol\_bundle object.
\item[\texttt{traceset\_index}:]
 A TracesetIndex of the neuron and treatments to match.
\item[\texttt{props}:]
 A structure with any optional properties.
\end{description}%
%
\item[Returns:]~

	a\_crit\_db: A tests\_db with two rows for values and STDs.
%
\item[Example:]~
\begin{lyxcode}        physiol\_bundle has an overloaded matchingRow method that\\%
        takes the TracesetIndex as argument:\\%
        >> a\_crit\_bundle = matchingRow(pbundle, 61)\\%
        >> a\_ranked\_bundle = rankMatching(mbundle, a\_crit\_bundle);\\%
        >> printTeXFile(comparisonReport(a\_ranked\_bundle), 'my\_report.tex')\\%
\end{lyxcode}
%
\item[See also:]%
\hyperlink{ref_rankMatching}{\texttt{rankMatching}}%
\ (p.~\pageref{ref_rankMatching})%
\index[funcref]{@\fidxl{rankMatching}}%
, \hyperlink{ref_tests_db__matchingRow}{\texttt{tests\_db/matchingRow}}%
\ (p.~\pageref{ref_tests_db__matchingRow})%
\index[funcref]{tests_db@\fidxlb{tests\_db}!matchingRow@\fidxl{matchingRow}}%
%
\item[Author:]%
Cengiz Gunay <cgunay@emory.edu>, 2005/12/21%
\end{description}
\methodline%
\subsubsection[Method \texttt{plotfICurveStats}]{Method \texttt{physiol\_bundle/plotfICurveStats}}%
\index[funcref]{physiol_bundle@\fidxlb{physiol\_bundle}!plotfICurveStats@\fidxl{plotfICurveStats}}%
\label{ref_physiol_bundle__plotfICurveStats}%
\hypertarget{ref_physiol_bundle__plotfICurveStats}{}%
\begin{description}
\item[Summary:]Generates a f-I curve mean-std plot of physiology DB.
%
\item[Usage:]~%
\begin{lyxcode}%
a\_plot = plotfICurveStats(p\_bundle, title\_str, props)
%
\end{lyxcode}%
%
%
\item[Parameters:]~
\begin{description}%
\item[\texttt{p\_bundle}:]
 A physiol\_bundle object.
\item[\texttt{title\_str}:]
 (Optional) String to append to plot title.
\item[\texttt{props}:]
 A structure with any optional properties.
\begin{description}%
\item[\texttt{quiet}:]
 if given, no title is produced

(passed to plot\_superpose)\end{description}%
\end{description}%
%
\item[Returns:]~

	a\_plot: An f-I curve plot.
%
\item[Example:]~
\begin{lyxcode} >> plotFigure(plotfICurveStats(pbundle));\\%
\end{lyxcode}
%
\item[See also:]%
\hyperlink{ref_dataset_db_bundle__plotfICurve}{\texttt{dataset\_db\_bundle/plotfICurve}}%
\ (p.~\pageref{ref_dataset_db_bundle__plotfICurve})%
\index[funcref]{dataset_db_bundle@\fidxlb{dataset\_db\_bundle}!plotfICurve@\fidxl{plotfICurve}}%
, \hyperlink{ref_plot_abstract}{\texttt{plot\_abstract}}%
\ (p.~\pageref{ref_plot_abstract})%
\index[funcref]{@\fidxl{plot\_abstract}}%
, \hyperlink{ref_plot_superpose}{\texttt{plot\_superpose}}%
\ (p.~\pageref{ref_plot_superpose})%
\index[funcref]{@\fidxl{plot\_superpose}}%
%
\item[Author:]%
Cengiz Gunay <cgunay@emory.edu>, 2006/06/16%
\end{description}
\methodline%
\subsubsection[Method \texttt{getNeuronRowIndex}]{Method \texttt{physiol\_bundle/getNeuronRowIndex}}%
\index[funcref]{physiol_bundle@\fidxlb{physiol\_bundle}!getNeuronRowIndex@\fidxl{getNeuronRowIndex}}%
\label{ref_physiol_bundle__getNeuronRowIndex}%
\hypertarget{ref_physiol_bundle__getNeuronRowIndex}{}%
\begin{description}
\item[Summary:]Returns the neuron index from bundle.
%
\item[Usage:]~%
\begin{lyxcode}%
a\_row\_index = getNeuronRowIndex(a\_bundle, traceset\_index, props)
%
\end{lyxcode}%
%
%
\item[Parameters:]~
\begin{description}%
\item[\texttt{a\_bundle}:]
 A physiol\_bundle object.
\item[\texttt{traceset\_index}:]
 The TracesetIndex number of neuron, or a DB row containing this.
\item[\texttt{props}:]
 A structure with any optional properties.
\end{description}%
%
\item[Returns:]~

	a\_row\_index: A row index of neuron in a\_bundle.joined\_db.
%
%
\item[See also:]%
\hyperlink{ref_dataset_db_bundle__getNeuronRowIndex}{\texttt{dataset\_db\_bundle/getNeuronRowIndex}}%
\ (p.~\pageref{ref_dataset_db_bundle__getNeuronRowIndex})%
\index[funcref]{dataset_db_bundle@\fidxlb{dataset\_db\_bundle}!getNeuronRowIndex@\fidxl{getNeuronRowIndex}}%
%
\item[Author:]%
Cengiz Gunay <cgunay@emory.edu>, 2006/06/09%
\end{description}
\methodline%
\subsubsection[Method \texttt{ctFromRows}]{Method \texttt{physiol\_bundle/ctFromRows}}%
\index[funcref]{physiol_bundle@\fidxlb{physiol\_bundle}!ctFromRows@\fidxl{ctFromRows}}%
\label{ref_physiol_bundle__ctFromRows}%
\hypertarget{ref_physiol_bundle__ctFromRows}{}%
\begin{description}
\item[Summary:]Loads a cip\_trace object from a raw data file in the a\_pbundle.
%
\item[Usage:]~%
\begin{lyxcode}%
a\_cip\_trace = ctFromRows(a\_pbundle, a\_db/traceset\_idx, cip\_levels, props)
%
\end{lyxcode}%
%
%
\item[Parameters:]~
\begin{description}%
\item[\texttt{a\_pbundle}:]
 A physiol\_cip\_traceset\_fileset object.
\item[\texttt{a\_db}:]
 A DB created by this fileset to read the traceset indices from.
\item[\texttt{traceset\_idx}:]
 A column vector with traceset indices.
\item[\texttt{cip\_levels}:]
 A column vector of CIP-levels to be loaded.
\item[\texttt{props}:]
 A structure with any optional properties.
\begin{description}%
\item[\texttt{traces}:]
 column vector of trace indices to load.
\item[\texttt{showParamsList}:]
 Cell array of params or treatments to include in the id field.
\end{description}%
\end{description}%
%
\item[Returns:]~

	a\_cip\_trace: One or more cip\_trace object that holds the raw data.
%
%
\item[See also:]%
\hyperlink{ref_loadItemProfile}{\texttt{loadItemProfile}}%
\ (p.~\pageref{ref_loadItemProfile})%
\index[funcref]{@\fidxl{loadItemProfile}}%
, \hyperlink{ref_physiol_cip_traceset__cip_trace}{\texttt{physiol\_cip\_traceset/cip\_trace}}%
\ (p.~\pageref{ref_physiol_cip_traceset__cip_trace})%
\index[funcref]{physiol_cip_traceset@\fidxlb{physiol\_cip\_traceset}!cip_trace@\fidxl{cip\_trace}}%
%
\item[Author:]%
Cengiz Gunay <cgunay@emory.edu>, 2005/07/13%
\end{description}
\methodline%
\subsubsection[Method \texttt{matchingControlNeuron}]{Method \texttt{physiol\_bundle/matchingControlNeuron}}%
\index[funcref]{physiol_bundle@\fidxlb{physiol\_bundle}!matchingControlNeuron@\fidxl{matchingControlNeuron}}%
\label{ref_physiol_bundle__matchingControlNeuron}%
\hypertarget{ref_physiol_bundle__matchingControlNeuron}{}%
\begin{description}
\item[Summary:]Creates a criterion database for matching the neuron at traceset\_index.
%
\item[Usage:]~%
\begin{lyxcode}%
a\_crit\_bundle = matchingControlNeuron(a\_bundle, neuron\_id, props)
%
\end{lyxcode}%
%
\item[Description:]%
Copies selected test values from row as the first row into the 
 criterion db. Adds a second row for the STD of each column in the db.
%%
\item[Parameters:]~
\begin{description}%
\item[\texttt{a\_bundle}:]
 A physiol\_bundle object.
\item[\texttt{neuron\_id}:]
 A NeuronId of the neuron to match.
\item[\texttt{props}:]
 A structure with any optional properties.
\end{description}%
%
\item[Returns:]~

	a\_crit\_bundle: A tests\_db with two rows for values and STDs.
%
\item[Example:]~
\begin{lyxcode}        Matches gpd0421c from cip\_traces\_all\_axoclamp.txt:\\%
        >> a\_crit\_bundle = matchingControlNeuron(pbundle, 33)\\%
        (see example in matchingRow)\\%
\end{lyxcode}
%
\item[See also:]%
\hyperlink{ref_rankMatching}{\texttt{rankMatching}}%
\ (p.~\pageref{ref_rankMatching})%
\index[funcref]{@\fidxl{rankMatching}}%
, \hyperlink{ref_tests_db}{\texttt{tests\_db}}%
\ (p.~\pageref{ref_tests_db})%
\index[funcref]{@\fidxl{tests\_db}}%
, \hyperlink{ref_tests2cols}{\texttt{tests2cols}}%
\ (p.~\pageref{ref_tests2cols})%
\index[funcref]{@\fidxl{tests2cols}}%
%
\item[Author:]%
Cengiz Gunay <cgunay@emory.edu>, 2005/12/21%
\end{description}
\methodline%
\subsection{Class \texttt{physiol\_cip\_traceset}}%
\index[funcref]{physiol_cip_traceset@\fidxlb{physiol\_cip\_traceset}}%
\label{ref_physiol_cip_traceset}%
\hypertarget{ref_physiol_cip_traceset}{}%
\subsubsection[Constructor \texttt{physiol\_cip\_traceset}]{Constructor \texttt{physiol\_cip\_traceset/physiol\_cip\_traceset}}%
\index[funcref]{physiol_cip_traceset@\fidxlb{physiol\_cip\_traceset}!physiol_cip_traceset@\fidxl{physiol\_cip\_traceset}}%
\label{ref_physiol_cip_traceset__physiol_cip_traceset}%
\hypertarget{ref_physiol_cip_traceset__physiol_cip_traceset}{}%
\begin{description}
\item[Summary:]Dataset of cip traces from same PCDX file.
%
\item[Usage:]~%
\begin{lyxcode}%
obj = physiol\_cip\_traceset(trace\_str, data\_src, chaninfo, dt, dy, treatments, id, props);
%
\end{lyxcode}%
%
\item[Description:]%
This is a subclass of params\_tests\_dataset. Each trace varies in bias, 
 pulse times and cip magnitude.
%%
\item[Parameters:]~
\begin{description}%
\item[\texttt{trace\_str}:]
 Trace list in the format for loadtraces.
\item[\texttt{data\_src}:]
 Absolute path of PCDX data source.
\item[\texttt{chaninfo}:]
 4-element array containing vchan, ichan, vgain, igain
\begin{description}%
\item[\texttt{vchan, ichan}:]
 Current and voltage channels.
\item[\texttt{vgain, igain}:]
 External gain factors for voltage channel and current 

channel
(vgain does NOT include the 10X amplification from the Axoclamp,
so vgain = 1 would mean no additional amplification beyond the 10X.)\end{description}%
\item[\texttt{dt}:]
 Time resolution [s].
\item[\texttt{dy}:]
 Y-axis resolution [V] or [A].
\item[\texttt{treatments}:]
 Structure containing the names and concentrations

of compounds.\item[\texttt{id}:]
 Neuron name.
\item[\texttt{props}:]
 A structure with any optional properties.
\begin{description}%
\item[\texttt{profile\_class\_name}:]
 Use this profile class (Default: 'cip\_trace\_profile').
\item[\texttt{cip\_list}:]
 Vector of cip levels to which the current trace will be matched.

(All other props are passed to cip\_trace objects)\end{description}%
\end{description}%
%
\item[Returns a structure object with the following fields:]~

	params\_tests\_dataset,
	data\_src, ichan, vchan, vgain, igain, treatments, id.
%
%
\item[See also:]%
\hyperlink{ref_cip_traces}{\texttt{cip\_traces}}%
\ (p.~\pageref{ref_cip_traces})%
\index[funcref]{@\fidxl{cip\_traces}}%
, \hyperlink{ref_params_tests_dataset}{\texttt{params\_tests\_dataset}}%
\ (p.~\pageref{ref_params_tests_dataset})%
\index[funcref]{@\fidxl{params\_tests\_dataset}}%
, \hyperlink{ref_params_tests_db}{\texttt{params\_tests\_db}}%
\ (p.~\pageref{ref_params_tests_db})%
\index[funcref]{@\fidxl{params\_tests\_db}}%
%
\item[Author:]%
Cengiz Gunay <cgunay@emory.edu> and Thomas Sangrey, 2005/01/17%
\end{description}
\methodline%
\subsubsection[Method \texttt{setProp}]{Method \texttt{physiol\_cip\_traceset/setProp}}%
\index[funcref]{physiol_cip_traceset@\fidxlb{physiol\_cip\_traceset}!setProp@\fidxl{setProp}}%
\label{ref_physiol_cip_traceset__setProp}%
\hypertarget{ref_physiol_cip_traceset__setProp}{}%
\begin{description}
\item[Summary:]Generic method for setting optional object properties.
%
\item[Usage:]~%
\begin{lyxcode}%
obj = setProp(obj, prop1, val1, prop2, val2, ...)
%
\end{lyxcode}%
%
\item[Description:]%
Modifies or adds property values. As many property name-value 
 pairs can be specified.
%%
\item[Parameters:]~
\begin{description}%
\item[\texttt{obj}:]
 Any object that has a props field.
\item[\texttt{attr}:]
 Property name
\item[\texttt{val}:]
 Property value.
\end{description}%
%
\item[Returns:]~

	obj: The new object with the updated properties.
%
%
\item[See also:]%
%
\item[Author:]%
Cengiz Gunay <cgunay@emory.edu>, 2004/11/22%
\end{description}
\methodline%
\subsubsection[Method \texttt{get}]{Method \texttt{physiol\_cip\_traceset/get}}%
\index[funcref]{physiol_cip_traceset@\fidxlb{physiol\_cip\_traceset}!get@\fidxl{get}}%
\label{ref_physiol_cip_traceset__get}%
\hypertarget{ref_physiol_cip_traceset__get}{}%
\begin{description}
\item[Summary:]Defines generic attribute retrieval for objects.
%
%
%
%
%
%
%
\item[Author:]%
Cengiz Gunay <cgunay@emory.edu>, 2004/09/14%
\end{description}
\methodline%
\subsubsection[Method \texttt{set}]{Method \texttt{physiol\_cip\_traceset/set}}%
\index[funcref]{physiol_cip_traceset@\fidxlb{physiol\_cip\_traceset}!set@\fidxl{set}}%
\label{ref_physiol_cip_traceset__set}%
\hypertarget{ref_physiol_cip_traceset__set}{}%
\begin{description}
\item[Summary:]Generic method for setting object attributes.
%
%
%
%
%
%
%
\item[Author:]%
Cengiz Gunay <cgunay@emory.edu>, 2004/10/08%
\end{description}
\methodline%
\subsubsection[Method \texttt{cip\_trace\_profile}]{Method \texttt{physiol\_cip\_traceset/cip\_trace\_profile}}%
\index[funcref]{physiol_cip_traceset@\fidxlb{physiol\_cip\_traceset}!cip_trace_profile@\fidxl{cip\_trace\_profile}}%
\label{ref_physiol_cip_traceset__cip_trace_profile}%
\hypertarget{ref_physiol_cip_traceset__cip_trace_profile}{}%
\begin{description}
\item[Summary:]Loads a cip\_trace\_profile object from a raw data file in the traceset.
%
\item[Usage:]~%
\begin{lyxcode}%
a\_profile = cip\_trace\_profile(traceset, trace\_index)
%
\end{lyxcode}%
%
%
\item[Parameters:]~
\begin{description}%
\item[\texttt{traceset}:]
 A physiol\_cip\_traceset object.
\item[\texttt{trace\_index}:]
 Index of file in traceset.
\end{description}%
%
\item[Returns:]~

	a\_profile: A profile object that implements the getResults method.
%
%
\item[See also:]%
\hyperlink{ref_itemResultsRow}{\texttt{itemResultsRow}}%
\ (p.~\pageref{ref_itemResultsRow})%
\index[funcref]{@\fidxl{itemResultsRow}}%
, \hyperlink{ref_params_tests_fileset}{\texttt{params\_tests\_fileset}}%
\ (p.~\pageref{ref_params_tests_fileset})%
\index[funcref]{@\fidxl{params\_tests\_fileset}}%
, \hyperlink{ref_paramNames}{\texttt{paramNames}}%
\ (p.~\pageref{ref_paramNames})%
\index[funcref]{@\fidxl{paramNames}}%
, \hyperlink{ref_testNames}{\texttt{testNames}}%
\ (p.~\pageref{ref_testNames})%
\index[funcref]{@\fidxl{testNames}}%
%
\item[Author:]%
Cengiz Gunay <cgunay@emory.edu> and Thomas Sangrey, 2005/01/18%
\end{description}
\methodline%
\subsubsection[Method \texttt{subsref}]{Method \texttt{physiol\_cip\_traceset/subsref}}%
\index[funcref]{physiol_cip_traceset@\fidxlb{physiol\_cip\_traceset}!subsref@\fidxl{subsref}}%
\label{ref_physiol_cip_traceset__subsref}%
\hypertarget{ref_physiol_cip_traceset__subsref}{}%
\begin{description}
\item[Summary:]Defines generic indexing for objects.
%
%
%
%
%
%
%
%
\end{description}
\methodline%
\subsubsection[Method \texttt{CIPform}]{Method \texttt{physiol\_cip\_traceset/CIPform}}%
\index[funcref]{physiol_cip_traceset@\fidxlb{physiol\_cip\_traceset}!CIPform@\fidxl{CIPform}}%
\label{ref_physiol_cip_traceset__CIPform}%
\hypertarget{ref_physiol_cip_traceset__CIPform}{}%
\begin{description}
\item[Summary:]Extracts current bias and pulse information from the current channel.
%
\item[Usage:]~%
\begin{lyxcode}%
[ciptype, on, off, finish, bias, pulse] = CIPform(traceset,trace\_index)
%
\end{lyxcode}%
%
%
\item[Parameters:]~
\begin{description}%
\item[\texttt{traceset}:]
 A physiol\_cip\_traceset object.
\item[\texttt{trace\_index}:]
 Index of item in traceset
\end{description}%
%
%
%
\item[See also:]%
\hyperlink{ref_cip_traces}{\texttt{cip\_traces}}%
\ (p.~\pageref{ref_cip_traces})%
\index[funcref]{@\fidxl{cip\_traces}}%
, \hyperlink{ref_params_tests_dataset}{\texttt{params\_tests\_dataset}}%
\ (p.~\pageref{ref_params_tests_dataset})%
\index[funcref]{@\fidxl{params\_tests\_dataset}}%
, \hyperlink{ref_params_tests_db}{\texttt{params\_tests\_db}}%
\ (p.~\pageref{ref_params_tests_db})%
\index[funcref]{@\fidxl{params\_tests\_db}}%
%
\item[Author:]%
Thomas Sangrey, 2005%
\end{description}
\methodline%
\subsubsection[Method \texttt{paramNames}]{Method \texttt{physiol\_cip\_traceset/paramNames}}%
\index[funcref]{physiol_cip_traceset@\fidxlb{physiol\_cip\_traceset}!paramNames@\fidxl{paramNames}}%
\label{ref_physiol_cip_traceset__paramNames}%
\hypertarget{ref_physiol_cip_traceset__paramNames}{}%
\begin{description}
\item[Summary:]Returns the parameter names for this fileset.
%
\item[Usage:]~%
\begin{lyxcode}%
param\_names = paramNames(fileset)
%
\end{lyxcode}%
%
\item[Description:]%
Looks at the filename of the first file to find the parameter names.
%%
\item[Parameters:]~
\begin{description}%
\item[\texttt{fileset}:]
 A params\_tests\_fileset.
\end{description}%
%
\item[Returns:]~

	param\_names: Cell array with ordered parameter names.
%
%
\item[See also:]%
\hyperlink{ref_params_tests_fileset}{\texttt{params\_tests\_fileset}}%
\ (p.~\pageref{ref_params_tests_fileset})%
\index[funcref]{@\fidxl{params\_tests\_fileset}}%
, \hyperlink{ref_paramNames}{\texttt{paramNames}}%
\ (p.~\pageref{ref_paramNames})%
\index[funcref]{@\fidxl{paramNames}}%
, \hyperlink{ref_testNames}{\texttt{testNames}}%
\ (p.~\pageref{ref_testNames})%
\index[funcref]{@\fidxl{testNames}}%
%
\item[Author:]%
Cengiz Gunay <cgunay@emory.edu>, 2004/12/06%
\end{description}
\methodline%
\subsubsection[Method \texttt{getItemParams}]{Method \texttt{physiol\_cip\_traceset/getItemParams}}%
\index[funcref]{physiol_cip_traceset@\fidxlb{physiol\_cip\_traceset}!getItemParams@\fidxl{getItemParams}}%
\label{ref_physiol_cip_traceset__getItemParams}%
\hypertarget{ref_physiol_cip_traceset__getItemParams}{}%
\begin{description}
%
\item[Usage:]~%
\begin{lyxcode}%
params\_row = getParams(dataset, index)
%
\end{lyxcode}%
%
%
\item[Parameters:]~
\begin{description}%
\item[\texttt{dataset}:]
 A params\_tests\_dataset.
\item[\texttt{index}:]
 Index of item in dataset.
\end{description}%
%
\item[Returns:]~

	params\_row: Parameter values in the same order of paramNames
%
%
\item[See also:]%
\hyperlink{ref_itemResultsRow}{\texttt{itemResultsRow}}%
\ (p.~\pageref{ref_itemResultsRow})%
\index[funcref]{@\fidxl{itemResultsRow}}%
, \hyperlink{ref_params_tests_dataset}{\texttt{params\_tests\_dataset}}%
\ (p.~\pageref{ref_params_tests_dataset})%
\index[funcref]{@\fidxl{params\_tests\_dataset}}%
, \hyperlink{ref_paramNames}{\texttt{paramNames}}%
\ (p.~\pageref{ref_paramNames})%
\index[funcref]{@\fidxl{paramNames}}%
, \hyperlink{ref_testNames}{\texttt{testNames}}%
\ (p.~\pageref{ref_testNames})%
\index[funcref]{@\fidxl{testNames}}%
%
\item[Author:]%
Cengiz Gunay <cgunay@emory.edu>, 2004/12/06%
\end{description}
\methodline%
\subsubsection[Method \texttt{itemResultsRow}]{Method \texttt{physiol\_cip\_traceset/itemResultsRow}}%
\index[funcref]{physiol_cip_traceset@\fidxlb{physiol\_cip\_traceset}!itemResultsRow@\fidxl{itemResultsRow}}%
\label{ref_physiol_cip_traceset__itemResultsRow}%
\hypertarget{ref_physiol_cip_traceset__itemResultsRow}{}%
\begin{description}
\item[Summary:]Processes a raw data file from the dataset and return
		its parameter and test values.
%
\item[Usage:]~%
\begin{lyxcode}%
[params\_row, tests\_row] = itemResultsRow(dataset, index)
%
\end{lyxcode}%
%
\item[Description:]%
This method is designed to be reused from subclasses as long as the
 loadItemProfile method is properly overloaded. Adds an Index
 column to the DB to keep track of raw data items after shuffling.
%%
\item[Parameters:]~
\begin{description}%
\item[\texttt{dataset}:]
 A params\_tests\_dataset.
\item[\texttt{index}:]
 Index of file in dataset.
\end{description}%
%
\item[Returns:]~

	params\_row: Parameter values in the same order of paramNames
	tests\_row: Test values in the same order with testNames
%
%
\item[See also:]%
\hyperlink{ref_loadItemProfile}{\texttt{loadItemProfile}}%
\ (p.~\pageref{ref_loadItemProfile})%
\index[funcref]{@\fidxl{loadItemProfile}}%
, \hyperlink{ref_params_tests_dataset}{\texttt{params\_tests\_dataset}}%
\ (p.~\pageref{ref_params_tests_dataset})%
\index[funcref]{@\fidxl{params\_tests\_dataset}}%
, \hyperlink{ref_paramNames}{\texttt{paramNames}}%
\ (p.~\pageref{ref_paramNames})%
\index[funcref]{@\fidxl{paramNames}}%
, \hyperlink{ref_testNames}{\texttt{testNames}}%
\ (p.~\pageref{ref_testNames})%
\index[funcref]{@\fidxl{testNames}}%
%
\item[Author:]%
Cengiz Gunay <cgunay@emory.edu>, 2004/09/10%
\end{description}
\methodline%
\subsubsection[Method \texttt{loadItemProfile}]{Method \texttt{physiol\_cip\_traceset/loadItemProfile}}%
\index[funcref]{physiol_cip_traceset@\fidxlb{physiol\_cip\_traceset}!loadItemProfile@\fidxl{loadItemProfile}}%
\label{ref_physiol_cip_traceset__loadItemProfile}%
\hypertarget{ref_physiol_cip_traceset__loadItemProfile}{}%
\begin{description}
\item[Summary:]Loads a cip\_trace\_profile object from a raw data file in the traceset.
%
\item[Usage:]~%
\begin{lyxcode}%
a\_profile = loadItemProfile(traceset, trace\_index)
%
\end{lyxcode}%
%
%
\item[Parameters:]~
\begin{description}%
\item[\texttt{traceset}:]
 A physiol\_cip\_traceset object.
\item[\texttt{trace\_index}:]
 Index of file in traceset.
\end{description}%
%
\item[Returns:]~

	a\_profile: A profile object that implements the getResults method.
%
%
\item[See also:]%
\hyperlink{ref_itemResultsRow}{\texttt{itemResultsRow}}%
\ (p.~\pageref{ref_itemResultsRow})%
\index[funcref]{@\fidxl{itemResultsRow}}%
, \hyperlink{ref_params_tests_fileset}{\texttt{params\_tests\_fileset}}%
\ (p.~\pageref{ref_params_tests_fileset})%
\index[funcref]{@\fidxl{params\_tests\_fileset}}%
, \hyperlink{ref_paramNames}{\texttt{paramNames}}%
\ (p.~\pageref{ref_paramNames})%
\index[funcref]{@\fidxl{paramNames}}%
, \hyperlink{ref_testNames}{\texttt{testNames}}%
\ (p.~\pageref{ref_testNames})%
\index[funcref]{@\fidxl{testNames}}%
%
\item[Author:]%
Cengiz Gunay <cgunay@emory.edu>, 2004/09/14%
\end{description}
\methodline%
\subsubsection[Method \texttt{cip\_trace}]{Method \texttt{physiol\_cip\_traceset/cip\_trace}}%
\index[funcref]{physiol_cip_traceset@\fidxlb{physiol\_cip\_traceset}!cip_trace@\fidxl{cip\_trace}}%
\label{ref_physiol_cip_traceset__cip_trace}%
\hypertarget{ref_physiol_cip_traceset__cip_trace}{}%
\begin{description}
\item[Summary:]Loads a cip\_trace object from a raw data file in the traceset.
%
\item[Usage:]~%
\begin{lyxcode}%
a\_cip\_trace = cip\_trace(traceset, trace\_index, props)
%
\end{lyxcode}%
%
%
\item[Parameters:]~
\begin{description}%
\item[\texttt{traceset}:]
 A physiol\_cip\_traceset object.
\item[\texttt{trace\_index}:]
 Index of file in traceset.
\item[\texttt{props}:]
 A structure with any optional properties.
\begin{description}%
\item[\texttt{TracesetIndex}:]
 Indicates in the id field.
\item[\texttt{showParamsList}:]
 Cell array of params to add to id field.
\item[\texttt{showName}:]
 Show the name of the cell in the id field (default=1).
\end{description}%
\end{description}%
%
\item[Returns:]~

	a\_cip\_trace: A cip\_trace object that holds the raw data.
%
%
\item[See also:]%
\hyperlink{ref_itemResultsRow}{\texttt{itemResultsRow}}%
\ (p.~\pageref{ref_itemResultsRow})%
\index[funcref]{@\fidxl{itemResultsRow}}%
, \hyperlink{ref_params_tests_fileset}{\texttt{params\_tests\_fileset}}%
\ (p.~\pageref{ref_params_tests_fileset})%
\index[funcref]{@\fidxl{params\_tests\_fileset}}%
, \hyperlink{ref_paramNames}{\texttt{paramNames}}%
\ (p.~\pageref{ref_paramNames})%
\index[funcref]{@\fidxl{paramNames}}%
, \hyperlink{ref_testNames}{\texttt{testNames}}%
\ (p.~\pageref{ref_testNames})%
\index[funcref]{@\fidxl{testNames}}%
%
\item[Author:]%
Cengiz Gunay <cgunay@emory.edu>, 2005/07/13%
\end{description}
\methodline%
\subsection{Class \texttt{physiol\_cip\_traceset\_fileset}}%
\index[funcref]{physiol_cip_traceset_fileset@\fidxlb{physiol\_cip\_traceset\_fileset}}%
\label{ref_physiol_cip_traceset_fileset}%
\hypertarget{ref_physiol_cip_traceset_fileset}{}%
\subsubsection[Constructor \texttt{physiol\_cip\_traceset\_fileset}]{Constructor \texttt{physiol\_cip\_traceset\_fileset/physiol\_cip\_traceset\_fileset}}%
\index[funcref]{physiol_cip_traceset_fileset@\fidxlb{physiol\_cip\_traceset\_fileset}!physiol_cip_traceset_fileset@\fidxl{physiol\_cip\_traceset\_fileset}}%
\label{ref_physiol_cip_traceset_fileset__physiol_cip_traceset_fileset}%
\hypertarget{ref_physiol_cip_traceset_fileset__physiol_cip_traceset_fileset}{}%
\begin{description}
\item[Summary:]Physiological fileset of traceset objects (concatenated).
%
\item[Usage:]~%
\begin{lyxcode}%
obj = physiol\_cip\_traceset\_fileset(cells\_filename, dt, dy, props)
%
\end{lyxcode}%
%
\item[Description:]%
This is a subclass of params\_tests\_dataset. Each trace varies in bias, 
 pulse times and cip magnitude.
%%
\item[Parameters:]~
\begin{description}%
\item[\texttt{cells\_filename}:]
 Ascii file containing the following tab-delimited items:

1. Neuron ID (name to associate with the neuron). If left blank, use
the filename with the '.all' extension removed.
2. The absolute path of the data file
3. The trace numbers to load, space-delimited (e.g. 1-21 24 26 27)
4. Vchan: voltage channel number
5. Ichan: current channel number
6. Vgain: external gain on voltage channel IN ADDITION to the 10X that
automatically comes from the Axoclamp 2B.
7. Igain: external gain on current channel.
8. Pairs of condition names and molar concentrations in any order
e.g.: TTX       1e-8    apamin  2e-7    picrotoxin      1e-4\end{description}%
%
\item[Returns a structure object with the following fields:]~

	neuron\_idx: A structure that points from neuron names to NeuronId numbers.
	params\_tests\_dataset
%
%
\item[See also:]%
\hyperlink{ref_physiol_cip_traceset}{\texttt{physiol\_cip\_traceset}}%
\ (p.~\pageref{ref_physiol_cip_traceset})%
\index[funcref]{@\fidxl{physiol\_cip\_traceset}}%
, \hyperlink{ref_params_tests_dataset}{\texttt{params\_tests\_dataset}}%
\ (p.~\pageref{ref_params_tests_dataset})%
\index[funcref]{@\fidxl{params\_tests\_dataset}}%
, \hyperlink{ref_params_tests_db}{\texttt{params\_tests\_db}}%
\ (p.~\pageref{ref_params_tests_db})%
\index[funcref]{@\fidxl{params\_tests\_db}}%
%
\item[Author:]%
Cengiz Gunay <cgunay@emory.edu> and Thomas Sangrey, 2005/01/17%
\end{description}
\methodline%
\subsubsection[Method \texttt{setProp}]{Method \texttt{physiol\_cip\_traceset\_fileset/setProp}}%
\index[funcref]{physiol_cip_traceset_fileset@\fidxlb{physiol\_cip\_traceset\_fileset}!setProp@\fidxl{setProp}}%
\label{ref_physiol_cip_traceset_fileset__setProp}%
\hypertarget{ref_physiol_cip_traceset_fileset__setProp}{}%
\begin{description}
\item[Summary:]Generic method for setting optional object properties.
%
\item[Usage:]~%
\begin{lyxcode}%
obj = setProp(obj, prop1, val1, prop2, val2, ...)
%
\end{lyxcode}%
%
\item[Description:]%
Modifies or adds property values. As many property name-value 
 pairs can be specified.
%%
\item[Parameters:]~
\begin{description}%
\item[\texttt{obj}:]
 Any object that has a props field.
\item[\texttt{attr}:]
 Property name
\item[\texttt{val}:]
 Property value.
\end{description}%
%
\item[Returns:]~

	obj: The new object with the updated properties.
%
%
\item[See also:]%
%
\item[Author:]%
Cengiz Gunay <cgunay@emory.edu>, 2004/11/22%
\end{description}
\methodline%
\subsubsection[Method \texttt{display}]{Method \texttt{physiol\_cip\_traceset\_fileset/display}}%
\index[funcref]{physiol_cip_traceset_fileset@\fidxlb{physiol\_cip\_traceset\_fileset}!display@\fidxl{display}}%
\label{ref_physiol_cip_traceset_fileset__display}%
\hypertarget{ref_physiol_cip_traceset_fileset__display}{}%
\begin{description}
%
%
%
%
%
%
%
\item[Author:]%
Cengiz Gunay <cgunay@emory.edu>, 2004/08/04%
\end{description}
\methodline%
\subsubsection[Method \texttt{get}]{Method \texttt{physiol\_cip\_traceset\_fileset/get}}%
\index[funcref]{physiol_cip_traceset_fileset@\fidxlb{physiol\_cip\_traceset\_fileset}!get@\fidxl{get}}%
\label{ref_physiol_cip_traceset_fileset__get}%
\hypertarget{ref_physiol_cip_traceset_fileset__get}{}%
\begin{description}
\item[Summary:]Defines generic attribute retrieval for objects.
%
%
%
%
%
%
%
\item[Author:]%
Cengiz Gunay <cgunay@emory.edu>, 2004/09/14%
\end{description}
\methodline%
\subsubsection[Method \texttt{set}]{Method \texttt{physiol\_cip\_traceset\_fileset/set}}%
\index[funcref]{physiol_cip_traceset_fileset@\fidxlb{physiol\_cip\_traceset\_fileset}!set@\fidxl{set}}%
\label{ref_physiol_cip_traceset_fileset__set}%
\hypertarget{ref_physiol_cip_traceset_fileset__set}{}%
\begin{description}
\item[Summary:]Generic method for setting object attributes.
%
%
%
%
%
%
%
\item[Author:]%
Cengiz Gunay <cgunay@emory.edu>, 2004/10/08%
\end{description}
\methodline%
\subsubsection[Method \texttt{loadItemProfile}]{Method \texttt{physiol\_cip\_traceset\_fileset/loadItemProfile}}%
\index[funcref]{physiol_cip_traceset_fileset@\fidxlb{physiol\_cip\_traceset\_fileset}!loadItemProfile@\fidxl{loadItemProfile}}%
\label{ref_physiol_cip_traceset_fileset__loadItemProfile}%
\hypertarget{ref_physiol_cip_traceset_fileset__loadItemProfile}{}%
\begin{description}
\item[Summary:]Loads a cip\_trace\_profile object from a raw data file in the fileset.
%
\item[Usage:]~%
\begin{lyxcode}%
a\_profile = loadItemProfile(fileset, traceset\_index, trace\_index)
%
\end{lyxcode}%
%
%
\item[Parameters:]~
\begin{description}%
\item[\texttt{fileset}:]
     A physiol\_cip\_traceset object.
\item[\texttt{traceset\_index }:]
  Index of traceset item in this fileset (corresponds 

to row in cells\_filename) to use grab the cell information.\item[\texttt{trace\_index}:]
 Index of item in the traceset.
\end{description}%
%
\item[Returns:]~

	a\_profile: A profile object that implements the getResults method.
%
%
\item[See also:]%
\hyperlink{ref_itemResultsRow}{\texttt{itemResultsRow}}%
\ (p.~\pageref{ref_itemResultsRow})%
\index[funcref]{@\fidxl{itemResultsRow}}%
, \hyperlink{ref_params_tests_fileset}{\texttt{params\_tests\_fileset}}%
\ (p.~\pageref{ref_params_tests_fileset})%
\index[funcref]{@\fidxl{params\_tests\_fileset}}%
, \hyperlink{ref_paramNames}{\texttt{paramNames}}%
\ (p.~\pageref{ref_paramNames})%
\index[funcref]{@\fidxl{paramNames}}%
, \hyperlink{ref_testNames}{\texttt{testNames}}%
\ (p.~\pageref{ref_testNames})%
\index[funcref]{@\fidxl{testNames}}%
%
\item[Author:]%
Cengiz Gunay <cgunay@emory.edu>, 2004/09/14 and Tom Sangrey%
\end{description}
\methodline%
\subsubsection[Method \texttt{cip\_trace}]{Method \texttt{physiol\_cip\_traceset\_fileset/cip\_trace}}%
\index[funcref]{physiol_cip_traceset_fileset@\fidxlb{physiol\_cip\_traceset\_fileset}!cip_trace@\fidxl{cip\_trace}}%
\label{ref_physiol_cip_traceset_fileset__cip_trace}%
\hypertarget{ref_physiol_cip_traceset_fileset__cip_trace}{}%
\begin{description}
\item[Summary:]Loads a cip\_trace object from a raw data file in the fileset.
%
%
%
\item[Parameters:]~
\begin{description}%
\item[\texttt{fileset}:]
 A physiol\_cip\_traceset\_fileset object.
\item[\texttt{traceset\_index}:]
 Index of traceset item in this fileset (corresponds 

to row in cells\_filename) to find the cell information.\item[\texttt{trace\_index}:]
 Index of item in the traceset.
\item[\texttt{a\_db}:]
 A DB created by this fileset to read the traceset and item indices from.
\item[\texttt{props}:]
 A structure with any optional properties, passed to physiol\_cip\_traceset/cip\_trace.
\end{description}%
%
\item[Returns:]~

	a\_cip\_trace: One or more cip\_trace object that holds the raw data.
%
%
\item[See also:]%
\hyperlink{ref_loadItemProfile}{\texttt{loadItemProfile}}%
\ (p.~\pageref{ref_loadItemProfile})%
\index[funcref]{@\fidxl{loadItemProfile}}%
, \hyperlink{ref_physiol_cip_traceset__cip_trace}{\texttt{physiol\_cip\_traceset/cip\_trace}}%
\ (p.~\pageref{ref_physiol_cip_traceset__cip_trace})%
\index[funcref]{physiol_cip_traceset@\fidxlb{physiol\_cip\_traceset}!cip_trace@\fidxl{cip\_trace}}%
%
\item[Author:]%
Cengiz Gunay <cgunay@emory.edu>, 2005/07/13%
\end{description}
\methodline%
\subsubsection[Method \texttt{readDBItems}]{Method \texttt{physiol\_cip\_traceset\_fileset/readDBItems}}%
\index[funcref]{physiol_cip_traceset_fileset@\fidxlb{physiol\_cip\_traceset\_fileset}!readDBItems@\fidxl{readDBItems}}%
\label{ref_physiol_cip_traceset_fileset__readDBItems}%
\hypertarget{ref_physiol_cip_traceset_fileset__readDBItems}{}%
\begin{description}
\item[Summary:]Reads all items to generate a params\_tests\_db object.
%
\item[Usage:]~%
\begin{lyxcode}%
[params, param\_names, tests, test\_names] = readDBItems(obj, items)
%
\end{lyxcode}%
%
\item[Description:]%
This is a specific method to convert from physiol\_cip\_traceset\_fileset to
 a params\_tests\_db, or a subclass. 
 Outputs of this function can be directly fed to the constructor of
 a params\_tests\_db or a subclass.
%%
\item[Parameters:]~
\begin{description}%
\item[\texttt{obj}:]
 A physiol\_cip\_traceset\_fileset 
\item[\texttt{items}:]
 (Optional) List of item indices to use to create the db.
\end{description}%
%
\item[Returns:]~

	params, param\_names, tests, test\_names: See params\_tests\_db.
%
%
\item[See also:]%
\hyperlink{ref_params_tests_db}{\texttt{params\_tests\_db}}%
\ (p.~\pageref{ref_params_tests_db})%
\index[funcref]{@\fidxl{params\_tests\_db}}%
, \hyperlink{ref_params_tests_fileset}{\texttt{params\_tests\_fileset}}%
\ (p.~\pageref{ref_params_tests_fileset})%
\index[funcref]{@\fidxl{params\_tests\_fileset}}%
, \hyperlink{ref_itemResultsRow}{\texttt{itemResultsRow}}%
\ (p.~\pageref{ref_itemResultsRow})%
\index[funcref]{@\fidxl{itemResultsRow}}%
%
%
\end{description}
\methodline%
\subsection{Class \texttt{plot\_abstract}}%
\index[funcref]{plot_abstract@\fidxlb{plot\_abstract}}%
\label{ref_plot_abstract}%
\hypertarget{ref_plot_abstract}{}%
\subsubsection[Constructor \texttt{plot\_abstract}]{Constructor \texttt{plot\_abstract/plot\_abstract}}%
\index[funcref]{plot_abstract@\fidxlb{plot\_abstract}!plot_abstract@\fidxl{plot\_abstract}}%
\label{ref_plot_abstract__plot_abstract}%
\hypertarget{ref_plot_abstract__plot_abstract}{}%
\begin{description}
\item[Summary:]Abstract description of a single plot.
%
\item[Usage:]~%
\begin{lyxcode}%
obj = plot\_abstract(data, axis\_labels, title, legend, command, props)
%
\end{lyxcode}%
%
\item[Description:]%
Base class that holds the necessary data to draw a plot. This data
 can then be used to generate different plots. Subclasses define specific
 plots with additional data. Subclasses should conform to the standard 
 that the series of commands found in plotFigure should produce a valid
 figure.
%%
\item[Parameters:]~
\begin{description}%
\item[\texttt{data}:]
 A cell array of data arrays (x, y, z, etc.) that can be 

fed to plot commands.\item[\texttt{axis\_labels}:]
 Cell array of axis label strings.
\item[\texttt{title}:]
 Plot description string.
\item[\texttt{legend}:]
 Cell array of descriptions for each item plotted.
\item[\texttt{command}:]
 Plotting command to use (Optional, default='plot')
\item[\texttt{props}:]
 A structure with any optional properties.
\begin{description}%
\item[\texttt{axisLimits}:]
 Sets axis limits of non-NaN values in vector.
\item[\texttt{tightLimits}:]
 If 1, issues an "axis tight" command (default=0)
\item[\texttt{border}:]
 Size of border spacing around axis, between 0 - 1. (default=0)
\item[\texttt{fontSize}:]
 Set the fontsize.
\item[\texttt{grid}:]
 Display dashed grid in background.
\item[\texttt{noXLabel}:]
 No X-axis label.
\item[\texttt{noYLabel}:]
 No Y-axis label.
\item[\texttt{noTitle}:]
 No title.
\item[\texttt{rotateXLabel}:]
 Rotates the X-axis label for smaller width.
\item[\texttt{rotateYLabel}:]
 Rotates the Y-axis label for smaller width.
\item[\texttt{numXTicks}:]
 Number of ticks on X-axis.
\item[\texttt{formatXTickLabels}:]
 The sprintf format string for tick labels.
\item[\texttt{XTick, YTick}:]
 Point locations for axis ticks.
\item[\texttt{XTickLabel, YTickLabel}:]
 Axis tick labels.
\item[\texttt{ColorOrder}:]
 Set the ColorOrder of the axis.
\item[\texttt{LineStyleOrder}:]
 Set the LineStyleOrder of the axis.
\item[\texttt{legendLocation}:]
 Passed to legend(..., 'location', legendLocation).
\item[\texttt{legendOrientation}:]
 Passed to legend(..., 'orientation', legendLocation).
\item[\texttt{noLegends}:]
 If exists, no legends are displayed.
\item[\texttt{PaperPosition}:]
 Sets the figure property for printing at this size.
\item[\texttt{resizeControl}:]
 If 0, drawing after resize is disabled and prints at screen 

size; if 1 (default), redraws figure after each resize event and 
prints at PaperPosition size.\end{description}%
\end{description}%
%
\item[Returns a structure object with the following fields:]~

	data, axis\_labels, title, legend, command, props
%
%
\item[See also:]%
\hyperlink{ref_plot_abstract__plot}{\texttt{plot\_abstract/plot}}%
\ (p.~\pageref{ref_plot_abstract__plot})%
\index[funcref]{plot_abstract@\fidxlb{plot\_abstract}!plot@\fidxl{plot}}%
, \hyperlink{ref_plot_abstract__plotFigure}{\texttt{plot\_abstract/plotFigure}}%
\ (p.~\pageref{ref_plot_abstract__plotFigure})%
\index[funcref]{plot_abstract@\fidxlb{plot\_abstract}!plotFigure@\fidxl{plotFigure}}%
%
\item[Author:]%
Cengiz Gunay <cgunay@emory.edu>, 2004/09/22%
\end{description}
\methodline%
\subsubsection[Method \texttt{setProp}]{Method \texttt{plot\_abstract/setProp}}%
\index[funcref]{plot_abstract@\fidxlb{plot\_abstract}!setProp@\fidxl{setProp}}%
\label{ref_plot_abstract__setProp}%
\hypertarget{ref_plot_abstract__setProp}{}%
\begin{description}
\item[Summary:]Generic method for setting optional object properties.
%
\item[Usage:]~%
\begin{lyxcode}%
obj = setProp(obj, prop1, val1, prop2, val2, ...)
%
\end{lyxcode}%
%
\item[Description:]%
Modifies or adds property values. As many property name-value 
 pairs can be specified.
%%
\item[Parameters:]~
\begin{description}%
\item[\texttt{obj}:]
 Any object that has a props field.
\item[\texttt{attr}:]
 Property name
\item[\texttt{val}:]
 Property value.
\end{description}%
%
\item[Returns:]~

	obj: The new object with the updated properties.
%
%
\item[See also:]%
%
\item[Author:]%
Cengiz Gunay <cgunay@emory.edu>, 2004/11/22%
\end{description}
\methodline%
\subsubsection[Method \texttt{display}]{Method \texttt{plot\_abstract/display}}%
\index[funcref]{plot_abstract@\fidxlb{plot\_abstract}!display@\fidxl{display}}%
\label{ref_plot_abstract__display}%
\hypertarget{ref_plot_abstract__display}{}%
\begin{description}
%
%
%
%
%
%
%
\item[Author:]%
Cengiz Gunay <cgunay@emory.edu>, 2004/08/04%
\end{description}
\methodline%
\subsubsection[Method \texttt{get}]{Method \texttt{plot\_abstract/get}}%
\index[funcref]{plot_abstract@\fidxlb{plot\_abstract}!get@\fidxl{get}}%
\label{ref_plot_abstract__get}%
\hypertarget{ref_plot_abstract__get}{}%
\begin{description}
\item[Summary:]Defines generic attribute retrieval for objects.
%
%
%
%
%
%
%
\item[Author:]%
Cengiz Gunay <cgunay@emory.edu>, 2004/09/14%
\end{description}
\methodline%
\subsubsection[Method \texttt{set}]{Method \texttt{plot\_abstract/set}}%
\index[funcref]{plot_abstract@\fidxlb{plot\_abstract}!set@\fidxl{set}}%
\label{ref_plot_abstract__set}%
\hypertarget{ref_plot_abstract__set}{}%
\begin{description}
\item[Summary:]Generic method for setting object attributes.
%
%
%
%
%
%
%
\item[Author:]%
Cengiz Gunay <cgunay@emory.edu>, 2004/10/08%
\end{description}
\methodline%
\subsubsection[Method \texttt{plotFigure}]{Method \texttt{plot\_abstract/plotFigure}}%
\index[funcref]{plot_abstract@\fidxlb{plot\_abstract}!plotFigure@\fidxl{plotFigure}}%
\label{ref_plot_abstract__plotFigure}%
\hypertarget{ref_plot_abstract__plotFigure}{}%
\begin{description}
\item[Summary:]Draws this plot alone in a new figure window.
%
\item[Usage:]~%
\begin{lyxcode}%
handle = plotFigure(a\_plot)
%
\end{lyxcode}%
%
%
\item[Parameters:]~
\begin{description}%
\item[\texttt{a\_plot}:]
 A plot\_abstract object, or a subclass object.
\item[\texttt{title\_str}:]
 (Optional) String to append to plot title.
\end{description}%
%
\item[Returns:]~

	handle: Handle of new figure.
%
%
\item[See also:]%
\hyperlink{ref_plot_abstract}{\texttt{plot\_abstract}}%
\ (p.~\pageref{ref_plot_abstract})%
\index[funcref]{@\fidxl{plot\_abstract}}%
, \hyperlink{ref_plot_abstract__plot}{\texttt{plot\_abstract/plot}}%
\ (p.~\pageref{ref_plot_abstract__plot})%
\index[funcref]{plot_abstract@\fidxlb{plot\_abstract}!plot@\fidxl{plot}}%
, \hyperlink{ref_plot_abstract__decorate}{\texttt{plot\_abstract/decorate}}%
\ (p.~\pageref{ref_plot_abstract__decorate})%
\index[funcref]{plot_abstract@\fidxlb{plot\_abstract}!decorate@\fidxl{decorate}}%
%
\item[Author:]%
Cengiz Gunay <cgunay@emory.edu>, 2004/09/22%
\end{description}
\methodline%
\subsubsection[Method \texttt{openAxis}]{Method \texttt{plot\_abstract/openAxis}}%
\index[funcref]{plot_abstract@\fidxlb{plot\_abstract}!openAxis@\fidxl{openAxis}}%
\label{ref_plot_abstract__openAxis}%
\hypertarget{ref_plot_abstract__openAxis}{}%
\begin{description}
\item[Summary:]Calculates the extents for the axis of this plot and opens it.
%
\item[Usage:]~%
\begin{lyxcode}%
[axis\_handle, layout\_axis] = openAxis(a\_plot, layout\_axis)
%
\end{lyxcode}%
%
%
\item[Parameters:]~
\begin{description}%
\item[\texttt{a\_plot}:]
 A plot\_abstract object, or a subclass object.
\item[\texttt{layout\_axis}:]
 The axis position to layout this plot (Optional). 

If NaN, doesn't open a new axis.\end{description}%
%
\item[Returns:]~

	handles: Handles of graphical objects drawn.
%
%
\item[See also:]%
\hyperlink{ref_plot_abstract}{\texttt{plot\_abstract}}%
\ (p.~\pageref{ref_plot_abstract})%
\index[funcref]{@\fidxl{plot\_abstract}}%
%
\item[Author:]%
Cengiz Gunay <cgunay@emory.edu>, 2004/09/22%
\end{description}
\methodline%
\subsubsection[Method \texttt{axis}]{Method \texttt{plot\_abstract/axis}}%
\index[funcref]{plot_abstract@\fidxlb{plot\_abstract}!axis@\fidxl{axis}}%
\label{ref_plot_abstract__axis}%
\hypertarget{ref_plot_abstract__axis}{}%
\begin{description}
\item[Summary:]Returns the estimated axis ranges of this plot according to its data.
%
\item[Usage:]~%
\begin{lyxcode}%
ranges = axis(a\_plot)
%
\end{lyxcode}%
%
%
\item[Parameters:]~
\begin{description}%
\item[\texttt{a\_plot}:]
 A plot\_abstract object, or a subclass object.
\end{description}%
%
\item[Returns:]~

	ranges: The ranges as a vector in the same way 'axis' would return.
%
%
\item[See also:]%
\hyperlink{ref_plot_abstract}{\texttt{plot\_abstract}}%
\ (p.~\pageref{ref_plot_abstract})%
\index[funcref]{@\fidxl{plot\_abstract}}%
, \hyperlink{ref_plot_abstract__plot}{\texttt{plot\_abstract/plot}}%
\ (p.~\pageref{ref_plot_abstract__plot})%
\index[funcref]{plot_abstract@\fidxlb{plot\_abstract}!plot@\fidxl{plot}}%
%
\item[Author:]%
Cengiz Gunay <cgunay@emory.edu>, 2004/10/13%
\end{description}
\methodline%
\subsubsection[Method \texttt{superposePlots}]{Method \texttt{plot\_abstract/superposePlots}}%
\index[funcref]{plot_abstract@\fidxlb{plot\_abstract}!superposePlots@\fidxl{superposePlots}}%
\label{ref_plot_abstract__superposePlots}%
\hypertarget{ref_plot_abstract__superposePlots}{}%
\begin{description}
\item[Summary:]Superpose multiple plots with common command onto a single axis.
%
\item[Usage:]~%
\begin{lyxcode}%
a\_plot = superposePlots(plots, axis\_labels, title\_str, command, props)
%
\end{lyxcode}%
%
\item[Description:]%
The plot decoration will be taken from the last plot in the list, 
 with the exception of legend labels.
%%
\item[Parameters:]~
\begin{description}%
\item[\texttt{plots}:]
 Array of plot\_abstract or subclass objects.
\item[\texttt{axis\_labels}:]
 Cell array of axis label strings (optional, taken from plots).
\item[\texttt{title\_str}:]
 Plot description string (optional, taken from plots).
\item[\texttt{command}:]
 Plotting command to use (optional, taken from plots)
\item[\texttt{props}:]
 A structure with any optional properties.
\begin{description}%
\item[\texttt{noLegends}:]
 If exists, no legends are created.
\end{description}%
\end{description}%
%
\item[Returns:]~

	a\_plot: A plot\_abstract object.
%
%
\item[See also:]%
\hyperlink{ref_plot_abstract}{\texttt{plot\_abstract}}%
\ (p.~\pageref{ref_plot_abstract})%
\index[funcref]{@\fidxl{plot\_abstract}}%
, \hyperlink{ref_plot_abstract__plot}{\texttt{plot\_abstract/plot}}%
\ (p.~\pageref{ref_plot_abstract__plot})%
\index[funcref]{plot_abstract@\fidxlb{plot\_abstract}!plot@\fidxl{plot}}%
, \hyperlink{ref_plot_abstract__plotFigure}{\texttt{plot\_abstract/plotFigure}}%
\ (p.~\pageref{ref_plot_abstract__plotFigure})%
\index[funcref]{plot_abstract@\fidxlb{plot\_abstract}!plotFigure@\fidxl{plotFigure}}%
%
\item[Author:]%
Cengiz Gunay <cgunay@emory.edu>, 2004/09/23%
\end{description}
\methodline%
\subsubsection[Method \texttt{matrixPlots}]{Method \texttt{plot\_abstract/matrixPlots}}%
\index[funcref]{plot_abstract@\fidxlb{plot\_abstract}!matrixPlots@\fidxl{matrixPlots}}%
\label{ref_plot_abstract__matrixPlots}%
\hypertarget{ref_plot_abstract__matrixPlots}{}%
\begin{description}
\item[Summary:]Superpose multiple plots with common command onto a single axis.
%
\item[Usage:]~%
\begin{lyxcode}%
a\_plot = matrixPlots(plots, axis\_labels, title\_str, props)
%
\end{lyxcode}%
%
%
\item[Parameters:]~
\begin{description}%
\item[\texttt{plots}:]
 Array of plot\_abstract or subclass objects.
\item[\texttt{axis\_labels}:]
 Cell array of axis label strings (optional, taken from plots).
\item[\texttt{title\_str}:]
 Plot description string (optional, taken from plots).
\item[\texttt{props}:]
 A structure with any optional properties passed to the Y stack\_plot.
\begin{description}%
\item[\texttt{titlesPos}:]
 if specified, passed to the X stack\_plots.
\item[\texttt{rotateYLabel}:]
 if specified, passed to the X stack\_plots.
\item[\texttt{axisLimits}:]
 if specified, passed to the X stack\_plots.
\item[\texttt{goldratio}:]
 try to make the figure in this aspect ratio.
\item[\texttt{width, height}:]
 if specified, make the figure have this many plots in 

corresponding dimension.\end{description}%
\end{description}%
%
\item[Returns:]~

	a\_plot: A plot\_abstract object.
%
%
\item[See also:]%
\hyperlink{ref_plot_abstract}{\texttt{plot\_abstract}}%
\ (p.~\pageref{ref_plot_abstract})%
\index[funcref]{@\fidxl{plot\_abstract}}%
, \hyperlink{ref_plot_abstract__plot}{\texttt{plot\_abstract/plot}}%
\ (p.~\pageref{ref_plot_abstract__plot})%
\index[funcref]{plot_abstract@\fidxlb{plot\_abstract}!plot@\fidxl{plot}}%
, \hyperlink{ref_plot_abstract__plotFigure}{\texttt{plot\_abstract/plotFigure}}%
\ (p.~\pageref{ref_plot_abstract__plotFigure})%
\index[funcref]{plot_abstract@\fidxlb{plot\_abstract}!plotFigure@\fidxl{plotFigure}}%
%
\item[Author:]%
Cengiz Gunay <cgunay@emory.edu>, 2004/12/07%
\end{description}
\methodline%
\subsubsection[Method \texttt{subsref}]{Method \texttt{plot\_abstract/subsref}}%
\index[funcref]{plot_abstract@\fidxlb{plot\_abstract}!subsref@\fidxl{subsref}}%
\label{ref_plot_abstract__subsref}%
\hypertarget{ref_plot_abstract__subsref}{}%
\begin{description}
\item[Summary:]Defines generic indexing for objects.
%
%
%
%
%
%
%
%
\end{description}
\methodline%
\subsubsection[Method \texttt{plot}]{Method \texttt{plot\_abstract/plot}}%
\index[funcref]{plot_abstract@\fidxlb{plot\_abstract}!plot@\fidxl{plot}}%
\label{ref_plot_abstract__plot}%
\hypertarget{ref_plot_abstract__plot}{}%
\begin{description}
\item[Summary:]Draws this plot in the current axis.
%
\item[Usage:]~%
\begin{lyxcode}%
handles = plot(a\_plot, layout\_axis)
%
\end{lyxcode}%
%
%
\item[Parameters:]~
\begin{description}%
\item[\texttt{a\_plot}:]
 A plot\_abstract object, or a subclass object.
\item[\texttt{layout\_axis}:]
 The axis position to layout this plot (Optional). 

If NaN, doesn't open a new axis.\end{description}%
%
\item[Returns:]~

	handles: Handles of graphical objects drawn.
%
%
\item[See also:]%
\hyperlink{ref_plot_abstract}{\texttt{plot\_abstract}}%
\ (p.~\pageref{ref_plot_abstract})%
\index[funcref]{@\fidxl{plot\_abstract}}%
%
\item[Author:]%
Cengiz Gunay <cgunay@emory.edu>, 2004/09/22%
\end{description}
\methodline%
\subsubsection[Method \texttt{subsasgn}]{Method \texttt{plot\_abstract/subsasgn}}%
\index[funcref]{plot_abstract@\fidxlb{plot\_abstract}!subsasgn@\fidxl{subsasgn}}%
\label{ref_plot_abstract__subsasgn}%
\hypertarget{ref_plot_abstract__subsasgn}{}%
\begin{description}
\item[Summary:]Defines generic index-based assignment for objects.
%
%
%
%
%
%
%
\item[Author:]%
Cengiz Gunay <cgunay@emory.edu>, 2006/02/06%
\end{description}
\methodline%
\subsubsection[Method \texttt{decorate}]{Method \texttt{plot\_abstract/decorate}}%
\index[funcref]{plot_abstract@\fidxlb{plot\_abstract}!decorate@\fidxl{decorate}}%
\label{ref_plot_abstract__decorate}%
\hypertarget{ref_plot_abstract__decorate}{}%
\begin{description}
\item[Summary:]Places decorations (titles, labels, ticks, etc.) on the plot.
%
\item[Usage:]~%
\begin{lyxcode}%
handles = decorate(a\_plot)
%
\end{lyxcode}%
%
%
\item[Parameters:]~
\begin{description}%
\item[\texttt{a\_plot}:]
 A plot\_abstract object, or a subclass object.
\end{description}%
%
\item[Returns:]~

	handles: Handles of graphical objects drawn.
%
%
\item[See also:]%
\hyperlink{ref_plot_abstract}{\texttt{plot\_abstract}}%
\ (p.~\pageref{ref_plot_abstract})%
\index[funcref]{@\fidxl{plot\_abstract}}%
, \hyperlink{ref_plot_abstract__plot}{\texttt{plot\_abstract/plot}}%
\ (p.~\pageref{ref_plot_abstract__plot})%
\index[funcref]{plot_abstract@\fidxlb{plot\_abstract}!plot@\fidxl{plot}}%
%
\item[Author:]%
Cengiz Gunay <cgunay@emory.edu>, 2004/09/22%
\end{description}
\methodline%
\subsection{Class \texttt{plot\_bars}}%
\index[funcref]{plot_bars@\fidxlb{plot\_bars}}%
\label{ref_plot_bars}%
\hypertarget{ref_plot_bars}{}%
\subsubsection[Constructor \texttt{plot\_bars}]{Constructor \texttt{plot\_bars/plot\_bars}}%
\index[funcref]{plot_bars@\fidxlb{plot\_bars}!plot_bars@\fidxl{plot\_bars}}%
\label{ref_plot_bars__plot_bars}%
\hypertarget{ref_plot_bars__plot_bars}{}%
\begin{description}
\item[Summary:]Bar plot with error lines in individual axes for each variable.
%
\item[Usage:]~%
\begin{lyxcode}%
a\_plot = plot\_bars(mid\_vals, lo\_vals, hi\_vals, n\_vals, x\_labels, y\_labels, ...
		     title, axis\_limits, props)
%
\end{lyxcode}%
%
\item[Description:]%
Subclass of plot\_stack. The plot\_abstract/plot command can be used to
 plot this data. Rows of *\_vals will create grouped bars, columns will
 create new axes.
%%
\item[Parameters:]~
\begin{description}%
\item[\texttt{mid\_vals}:]
 Middle points of error bars.
\item[\texttt{lo\_vals}:]
 Low points of error bars.
\item[\texttt{hi\_vals}:]
 High points of error bars.
\item[\texttt{n\_vals}:]
 Number of samples used for the statistic (Optional).
\item[\texttt{x\_labels, y\_labels}:]
 Axis labels for each bar group. Must match with data columns.
\item[\texttt{title}:]
 Plot description.
\item[\texttt{axis\_limits}:]
 If given, all plots contained will have these axis limits.
\item[\texttt{props}:]
 A structure with any optional properties.
\begin{description}%
\item[\texttt{dispErrorbars}:]
 If 1, display errorbars for lo\_vals and hi\_vals deviation from mid\_vals 

(default=1).\item[\texttt{dispNvals}:]
 If 1, display n\_vals on top of each bar.
\item[\texttt{groupValues}:]
 Array of within-group numeric labels, instead of just a sequence of numbers.
\item[\texttt{truncateDecDigits}:]
 Truncate labels to this many decimal digits.
\end{description}%
\end{description}%
%
\item[Returns a structure object with the following fields:]~

	plot\_abstract
%
%
\item[See also:]%
\hyperlink{ref_plot_abstract}{\texttt{plot\_abstract}}%
\ (p.~\pageref{ref_plot_abstract})%
\index[funcref]{@\fidxl{plot\_abstract}}%
, \hyperlink{ref_plot_abstract__plot}{\texttt{plot\_abstract/plot}}%
\ (p.~\pageref{ref_plot_abstract__plot})%
\index[funcref]{plot_abstract@\fidxlb{plot\_abstract}!plot@\fidxl{plot}}%
%
\item[Author:]%
Cengiz Gunay <cgunay@emory.edu>, 2004/10/07%
\end{description}
\methodline%
\subsubsection[Method \texttt{concatBars}]{Method \texttt{plot\_bars/concatBars}}%
\index[funcref]{plot_bars@\fidxlb{plot\_bars}!concatBars@\fidxl{concatBars}}%
\label{ref_plot_bars__concatBars}%
\hypertarget{ref_plot_bars__concatBars}{}%
\begin{description}
\item[Summary:]FLAWED DESIGN! Concatenates bars from multiple plot\_bars plots into one.
%
\item[Usage:]~%
\begin{lyxcode}%
a\_plot = concatBars(plots, props)
%
\end{lyxcode}%
%
%
\item[Parameters:]~
\begin{description}%
\item[\texttt{plots}:]
 Array of plot\_bars objects.
\item[\texttt{props}:]
 A structure with any optional properties.
\end{description}%
%
%
%
\item[See also:]%
\hyperlink{ref_plot_bars}{\texttt{plot\_bars}}%
\ (p.~\pageref{ref_plot_bars})%
\index[funcref]{@\fidxl{plot\_bars}}%
%
\item[Author:]%
Cengiz Gunay <cgunay@emory.edu>, 2006/06/07%
\end{description}
\methodline%
\subsubsection[Method \texttt{set}]{Method \texttt{plot\_bars/set}}%
\index[funcref]{plot_bars@\fidxlb{plot\_bars}!set@\fidxl{set}}%
\label{ref_plot_bars__set}%
\hypertarget{ref_plot_bars__set}{}%
\begin{description}
\item[Summary:]Generic method for setting object attributes.
%
%
%
%
%
%
%
\item[Author:]%
Cengiz Gunay <cgunay@emory.edu>, 2004/10/08%
\end{description}
\methodline%
\subsection{Class \texttt{plot\_errorbar}}%
\index[funcref]{plot_errorbar@\fidxlb{plot\_errorbar}}%
\label{ref_plot_errorbar}%
\hypertarget{ref_plot_errorbar}{}%
\subsubsection[Constructor \texttt{plot\_errorbar}]{Constructor \texttt{plot\_errorbar/plot\_errorbar}}%
\index[funcref]{plot_errorbar@\fidxlb{plot\_errorbar}!plot_errorbar@\fidxl{plot\_errorbar}}%
\label{ref_plot_errorbar__plot_errorbar}%
\hypertarget{ref_plot_errorbar__plot_errorbar}{}%
\begin{description}
\item[Summary:]Generic errorbar plot.
%
\item[Usage:]~%
\begin{lyxcode}%
a\_plot = plot\_errorbar(x\_vals, mid\_vals, lo\_vals, hi\_vals, line\_spec, 
			 axis\_labels, title, legend, props)
%
\end{lyxcode}%
%
\item[Description:]%
Subclass of plot\_abstract. The plot\_abstract/plot command can be used to
 plot this data. Needed to create this as a separate class to have the
 axis ranges method to measure the errorbars.
%%
\item[Parameters:]~
\begin{description}%
\item[\texttt{x\_vals}:]
 X coordinates of errorbars.
\item[\texttt{mid\_vals}:]
 Middle points of error bars.
\item[\texttt{lo\_vals}:]
 Low points of error bars.
\item[\texttt{hi\_vals}:]
 High points of error bars.
\item[\texttt{line\_spec}:]
 Plot line spec to be passed to errorbar
\item[\texttt{axis\_labels}:]
 Cell array for X, Y axis labels.
\item[\texttt{title}:]
 Plot description.
\item[\texttt{legend}:]
 For multiple errorbar plots (matrix form), description of each plot.
\item[\texttt{props}:]
 A structure with any optional properties to be passed to plot\_abstract.
\end{description}%
%
\item[Returns a structure object with the following fields:]~

	plot\_abstract.
%
%
\item[See also:]%
\hyperlink{ref_plot_abstract}{\texttt{plot\_abstract}}%
\ (p.~\pageref{ref_plot_abstract})%
\index[funcref]{@\fidxl{plot\_abstract}}%
, \hyperlink{ref_plot_abstract__plot}{\texttt{plot\_abstract/plot}}%
\ (p.~\pageref{ref_plot_abstract__plot})%
\index[funcref]{plot_abstract@\fidxlb{plot\_abstract}!plot@\fidxl{plot}}%
%
\item[Author:]%
Cengiz Gunay <cgunay@emory.edu>, 2004/10/07%
\end{description}
\methodline%
\subsubsection[Method \texttt{get}]{Method \texttt{plot\_errorbar/get}}%
\index[funcref]{plot_errorbar@\fidxlb{plot\_errorbar}!get@\fidxl{get}}%
\label{ref_plot_errorbar__get}%
\hypertarget{ref_plot_errorbar__get}{}%
\begin{description}
\item[Summary:]Defines generic attribute retrieval for objects.
%
%
%
%
%
%
%
\item[Author:]%
Cengiz Gunay <cgunay@emory.edu>, 2004/09/14%
\end{description}
\methodline%
\subsubsection[Method \texttt{axis}]{Method \texttt{plot\_errorbar/axis}}%
\index[funcref]{plot_errorbar@\fidxlb{plot\_errorbar}!axis@\fidxl{axis}}%
\label{ref_plot_errorbar__axis}%
\hypertarget{ref_plot_errorbar__axis}{}%
\begin{description}
\item[Summary:]Returns the estimated axis ranges of this plot according to its data.
%
\item[Usage:]~%
\begin{lyxcode}%
ranges = axis(a\_plot)
%
\end{lyxcode}%
%
%
\item[Parameters:]~
\begin{description}%
\item[\texttt{a\_plot}:]
 A plot\_abstract object, or a subclass object.
\end{description}%
%
\item[Returns:]~

	ranges: The ranges as a vector in the same way 'axis' would return.
%
%
\item[See also:]%
\hyperlink{ref_plot_abstract}{\texttt{plot\_abstract}}%
\ (p.~\pageref{ref_plot_abstract})%
\index[funcref]{@\fidxl{plot\_abstract}}%
, \hyperlink{ref_plot_abstract__plot}{\texttt{plot\_abstract/plot}}%
\ (p.~\pageref{ref_plot_abstract__plot})%
\index[funcref]{plot_abstract@\fidxlb{plot\_abstract}!plot@\fidxl{plot}}%
%
\item[Author:]%
Cengiz Gunay <cgunay@emory.edu>, 2004/10/13%
\end{description}
\methodline%
\subsection{Class \texttt{plot\_errorbars}}%
\index[funcref]{plot_errorbars@\fidxlb{plot\_errorbars}}%
\label{ref_plot_errorbars}%
\hypertarget{ref_plot_errorbars}{}%
\subsubsection[Constructor \texttt{plot\_errorbars}]{Constructor \texttt{plot\_errorbars/plot\_errorbars}}%
\index[funcref]{plot_errorbars@\fidxlb{plot\_errorbars}!plot_errorbars@\fidxl{plot\_errorbars}}%
\label{ref_plot_errorbars__plot_errorbars}%
\hypertarget{ref_plot_errorbars__plot_errorbars}{}%
\begin{description}
\item[Summary:]Special plot for plotting distributions of variables in separate axes.
%
\item[Usage:]~%
\begin{lyxcode}%
a\_plot = plot\_errorbars(labels, mid\_vals, lo\_vals, hi\_vals, labels, 
			 title, axis\_limits, props)
%
\end{lyxcode}%
%
\item[Description:]%
Subclass of plot\_stack. The plot\_abstract/plot command can be used to
 plot this data.
%%
\item[Parameters:]~
\begin{description}%
\item[\texttt{labels}:]
 Labels of parameters to appear at bottom of each errorbar.
\item[\texttt{mid\_vals}:]
 Middle points of error bars.
\item[\texttt{lo\_vals}:]
 Low points of error bars.
\item[\texttt{hi\_vals}:]
 High points of error bars.
\item[\texttt{title}:]
 Plot description.
\item[\texttt{axis\_limits}:]
 If given, all plots contained will have these axis limits.
\item[\texttt{props}:]
 A structure with any optional properties.
\end{description}%
%
\item[Returns a structure object with the following fields:]~

	plot\_abstract, labels.
%
%
\item[See also:]%
\hyperlink{ref_plot_abstract}{\texttt{plot\_abstract}}%
\ (p.~\pageref{ref_plot_abstract})%
\index[funcref]{@\fidxl{plot\_abstract}}%
, \hyperlink{ref_plot_abstract__plot}{\texttt{plot\_abstract/plot}}%
\ (p.~\pageref{ref_plot_abstract__plot})%
\index[funcref]{plot_abstract@\fidxlb{plot\_abstract}!plot@\fidxl{plot}}%
%
\item[Author:]%
Cengiz Gunay <cgunay@emory.edu>, 2004/10/07%
\end{description}
\methodline%
\subsection{Class \texttt{plot\_simple}}%
\index[funcref]{plot_simple@\fidxlb{plot\_simple}}%
\label{ref_plot_simple}%
\hypertarget{ref_plot_simple}{}%
\subsubsection[Constructor \texttt{plot\_simple}]{Constructor \texttt{plot\_simple/plot\_simple}}%
\index[funcref]{plot_simple@\fidxlb{plot\_simple}!plot_simple@\fidxl{plot\_simple}}%
\label{ref_plot_simple__plot_simple}%
\hypertarget{ref_plot_simple__plot_simple}{}%
\begin{description}
\item[Summary:]Abstract description of a single plot.
%
\item[Usage:]~%
\begin{lyxcode}%
a\_plot = plot\_simple(data\_x, data\_y, title, 
		       label\_x, label\_y, legend, command, props)
%
\end{lyxcode}%
%
\item[Description:]%
Subclass of plot\_abstract. The plot\_abstract/plot command can be used to
 plot this data.
%%
\item[Parameters:]~
\begin{description}%
\item[\texttt{data\_x}:]
 X-axis values for the plot.
\item[\texttt{data\_y}:]
 Y-axis values for the plot.
\item[\texttt{title}:]
 Plot description.
\item[\texttt{label\_x}:]
 X-axis label string.
\item[\texttt{label\_y}:]
 Y-axis label string.
\item[\texttt{legend}:]
 Short description of data points.
\item[\texttt{command}:]
 Plotting command to use (Optional, default='plot')
\item[\texttt{props}:]
 A structure with any optional properties.
\end{description}%
%
\item[Returns a structure object with the following fields:]~

	plot\_abstract.
%
%
\item[See also:]%
\hyperlink{ref_plot_abstract}{\texttt{plot\_abstract}}%
\ (p.~\pageref{ref_plot_abstract})%
\index[funcref]{@\fidxl{plot\_abstract}}%
, \hyperlink{ref_plot_abstract__plot}{\texttt{plot\_abstract/plot}}%
\ (p.~\pageref{ref_plot_abstract__plot})%
\index[funcref]{plot_abstract@\fidxlb{plot\_abstract}!plot@\fidxl{plot}}%
%
\item[Author:]%
Cengiz Gunay <cgunay@emory.edu>, 2004/09/22%
\end{description}
\methodline%
\subsubsection[Method \texttt{get}]{Method \texttt{plot\_simple/get}}%
\index[funcref]{plot_simple@\fidxlb{plot\_simple}!get@\fidxl{get}}%
\label{ref_plot_simple__get}%
\hypertarget{ref_plot_simple__get}{}%
\begin{description}
\item[Summary:]Defines generic attribute retrieval for objects.
%
%
%
%
%
%
%
\item[Author:]%
Cengiz Gunay <cgunay@emory.edu>, 2004/09/14%
\end{description}
\methodline%
\subsection{Class \texttt{plot\_stack}}%
\index[funcref]{plot_stack@\fidxlb{plot\_stack}}%
\label{ref_plot_stack}%
\hypertarget{ref_plot_stack}{}%
\subsubsection[Constructor \texttt{plot\_stack}]{Constructor \texttt{plot\_stack/plot\_stack}}%
\index[funcref]{plot_stack@\fidxlb{plot\_stack}!plot_stack@\fidxl{plot\_stack}}%
\label{ref_plot_stack__plot_stack}%
\hypertarget{ref_plot_stack__plot_stack}{}%
\begin{description}
\item[Summary:]A horizontal or vertical stack of plots.
%
\item[Usage:]~%
\begin{lyxcode}%
a\_plot = plot\_stack(plots, axis\_limits, orientation, title\_str, props)
%
\end{lyxcode}%
%
\item[Description:]%
Subclass of plot\_abstract. Contains other plot\_abstract objects or
 subclasses thereof to be layout in stack format. 
%%
\item[Parameters:]~
\begin{description}%
\item[\texttt{plots}:]
 Cell array of plot\_abstract or subclass objects.
\item[\texttt{axis\_limits}:]
 If given, all plots contained will have these axis limits.
\item[\texttt{orientation}:]
 Stack orientation 'x' for horizontal, 'y' for vertical, etc.
\item[\texttt{title\_str}:]
 Title to go on top of the stack
\item[\texttt{props}:]
 A structure with any optional properties.
\begin{description}%
\item[\texttt{yLabelsPos}:]
 'left' means only put y-axis label to leftmost plot.
\item[\texttt{yTicksPos}:]
 'left' means only put y-axis ticks to leftmost plot.
\item[\texttt{xLabelsPos}:]
 'bottom' means only put x-axis label to lowest plot.
\item[\texttt{xTicksPos}:]
 'bottom' means only put x-axis ticks to lowest plot.
\item[\texttt{titlesPos}:]
 'top' means only put title to top plot.
\item[\texttt{relaxedLimits}:]
 Add 10% to all axis limits, overriding Matlab's layout
\item[\texttt{relativeSizes}:]
 An array specifying relative size of each plot with one value.

(Example: relativeSizes=[1 2] makes second plot twice wider than first.)\end{description}%
\end{description}%
%
\item[Returns a structure object with the following fields:]~

	plot\_abstract, plots, axis\_limits, orient.
%
%
\item[See also:]%
\hyperlink{ref_plot_abstract}{\texttt{plot\_abstract}}%
\ (p.~\pageref{ref_plot_abstract})%
\index[funcref]{@\fidxl{plot\_abstract}}%
, \hyperlink{ref_plot_abstract__plotFigure}{\texttt{plot\_abstract/plotFigure}}%
\ (p.~\pageref{ref_plot_abstract__plotFigure})%
\index[funcref]{plot_abstract@\fidxlb{plot\_abstract}!plotFigure@\fidxl{plotFigure}}%
%
\item[Author:]%
Cengiz Gunay <cgunay@emory.edu>, 2004/10/04%
\end{description}
\methodline%
\subsubsection[Method \texttt{display}]{Method \texttt{plot\_stack/display}}%
\index[funcref]{plot_stack@\fidxlb{plot\_stack}!display@\fidxl{display}}%
\label{ref_plot_stack__display}%
\hypertarget{ref_plot_stack__display}{}%
\begin{description}
%
%
%
%
%
%
%
\item[Author:]%
Cengiz Gunay <cgunay@emory.edu>, 2004/08/04%
\end{description}
\methodline%
\subsubsection[Method \texttt{get}]{Method \texttt{plot\_stack/get}}%
\index[funcref]{plot_stack@\fidxlb{plot\_stack}!get@\fidxl{get}}%
\label{ref_plot_stack__get}%
\hypertarget{ref_plot_stack__get}{}%
\begin{description}
\item[Summary:]Defines generic attribute retrieval for objects.
%
%
%
%
%
%
%
\item[Author:]%
Cengiz Gunay <cgunay@emory.edu>, 2004/09/14%
\end{description}
\methodline%
\subsubsection[Method \texttt{set}]{Method \texttt{plot\_stack/set}}%
\index[funcref]{plot_stack@\fidxlb{plot\_stack}!set@\fidxl{set}}%
\label{ref_plot_stack__set}%
\hypertarget{ref_plot_stack__set}{}%
\begin{description}
\item[Summary:]Generic method for setting object attributes.
%
%
%
%
%
%
%
\item[Author:]%
Cengiz Gunay <cgunay@emory.edu>, 2004/10/08%
\end{description}
\methodline%
\subsubsection[Method \texttt{superposePlots}]{Method \texttt{plot\_stack/superposePlots}}%
\index[funcref]{plot_stack@\fidxlb{plot\_stack}!superposePlots@\fidxl{superposePlots}}%
\label{ref_plot_stack__superposePlots}%
\hypertarget{ref_plot_stack__superposePlots}{}%
\begin{description}
\item[Summary:]Superpose multiple plot\_stack objects that contain exact same contents.
%
\item[Usage:]~%
\begin{lyxcode}%
a\_plot = superposePlots(plots, axis\_labels, title\_str, command, props)
%
\end{lyxcode}%
%
\item[Description:]%
The plot decoration will be taken from the last plot in the list, 
 with the exception of legend labels.
%%
\item[Parameters:]~
\begin{description}%
\item[\texttt{plots}:]
 Array of plot\_stack objects.
\item[\texttt{axis\_labels}:]
 Cell array of axis label strings (optional, taken from plots).
\item[\texttt{title\_str}:]
 Plot description string (optional, taken from plots).
\item[\texttt{command}:]
 Plotting command to use (optional, taken from plots)
\item[\texttt{props}:]
 A structure with any optional properties.
\begin{description}%
\item[\texttt{noLegends}:]
 If exists, no legends are created.
\end{description}%
\end{description}%
%
\item[Returns:]~

	a\_plot: A plot\_stack object.
%
%
\item[See also:]%
\hyperlink{ref_plot_abstract}{\texttt{plot\_abstract}}%
\ (p.~\pageref{ref_plot_abstract})%
\index[funcref]{@\fidxl{plot\_abstract}}%
, \hyperlink{ref_plot_abstract__plot}{\texttt{plot\_abstract/plot}}%
\ (p.~\pageref{ref_plot_abstract__plot})%
\index[funcref]{plot_abstract@\fidxlb{plot\_abstract}!plot@\fidxl{plot}}%
, \hyperlink{ref_plot_abstract__plotFigure}{\texttt{plot\_abstract/plotFigure}}%
\ (p.~\pageref{ref_plot_abstract__plotFigure})%
\index[funcref]{plot_abstract@\fidxlb{plot\_abstract}!plotFigure@\fidxl{plotFigure}}%
%
\item[Author:]%
Cengiz Gunay <cgunay@emory.edu>, 2006/06/14%
\end{description}
\methodline%
\subsubsection[Method \texttt{plot}]{Method \texttt{plot\_stack/plot}}%
\index[funcref]{plot_stack@\fidxlb{plot\_stack}!plot@\fidxl{plot}}%
\label{ref_plot_stack__plot}%
\hypertarget{ref_plot_stack__plot}{}%
\begin{description}
\item[Summary:]Draws this plot in the current axis or at the position in
	layout\_axis.
%
\item[Usage:]~%
\begin{lyxcode}%
handles = plot(a\_plot, layout\_axis)
%
\end{lyxcode}%
%
%
\item[Parameters:]~
\begin{description}%
\item[\texttt{a\_plot}:]
 A plot\_abstract object, or a subclass object.
\item[\texttt{layout\_axis}:]
 The axis position to layout this plot (Optional). 
\end{description}%
%
\item[Returns:]~

	handles: Handles of graphical objects drawn.
%
%
\item[See also:]%
\hyperlink{ref_plot_stack}{\texttt{plot\_stack}}%
\ (p.~\pageref{ref_plot_stack})%
\index[funcref]{@\fidxl{plot\_stack}}%
, \hyperlink{ref_plot_abstract}{\texttt{plot\_abstract}}%
\ (p.~\pageref{ref_plot_abstract})%
\index[funcref]{@\fidxl{plot\_abstract}}%
%
\item[Author:]%
Cengiz Gunay <cgunay@emory.edu>, 2004/10/04%
\end{description}
\methodline%
\subsubsection[Method \texttt{decorate}]{Method \texttt{plot\_stack/decorate}}%
\index[funcref]{plot_stack@\fidxlb{plot\_stack}!decorate@\fidxl{decorate}}%
\label{ref_plot_stack__decorate}%
\hypertarget{ref_plot_stack__decorate}{}%
\begin{description}
\item[Summary:]No additional decorations for stacked plots.
%
\item[Usage:]~%
\begin{lyxcode}%
a\_histogram\_db = decorate(a\_plot)
%
\end{lyxcode}%
%
%
\item[Parameters:]~
\begin{description}%
\item[\texttt{a\_plot}:]
 A plot\_abstract object, or a subclass object.
\end{description}%
%
\item[Returns:]~

	handles: Handles of graphical objects drawn.
%
%
\item[See also:]%
\hyperlink{ref_plot_abstract}{\texttt{plot\_abstract}}%
\ (p.~\pageref{ref_plot_abstract})%
\index[funcref]{@\fidxl{plot\_abstract}}%
, \hyperlink{ref_plot_abstract__plot}{\texttt{plot\_abstract/plot}}%
\ (p.~\pageref{ref_plot_abstract__plot})%
\index[funcref]{plot_abstract@\fidxlb{plot\_abstract}!plot@\fidxl{plot}}%
%
\item[Author:]%
Cengiz Gunay <cgunay@emory.edu>, 2004/10/04%
\end{description}
\methodline%
\subsection{Class \texttt{plot\_superpose}}%
\index[funcref]{plot_superpose@\fidxlb{plot\_superpose}}%
\label{ref_plot_superpose}%
\hypertarget{ref_plot_superpose}{}%
\subsubsection[Constructor \texttt{plot\_superpose}]{Constructor \texttt{plot\_superpose/plot\_superpose}}%
\index[funcref]{plot_superpose@\fidxlb{plot\_superpose}!plot_superpose@\fidxl{plot\_superpose}}%
\label{ref_plot_superpose__plot_superpose}%
\hypertarget{ref_plot_superpose__plot_superpose}{}%
\begin{description}
\item[Summary:]Multiple plot\_abstract objects superposed on the same axis.
%
\item[Usage:]~%
\begin{lyxcode}%
obj = plot\_superpose(plots, axis\_labels, title\_str, props)
%
\end{lyxcode}%
%
\item[Description:]%
Subclass of plot\_abstract. Contains multiple plot\_abstract objects
 to be plotted on the same axis. This is different than the 
 plot\_abstract/superpose, where only using the same plot command is allowed.
 Here, each plot\_abstract can have its own special plotting command. Subclasses
 of plot\_abstract is also allowed here. The decorations comes from this object
 and not children plots. This behavior is different than plot\_stack, where
 each plot has its own decorations.
%%
\item[Parameters:]~
\begin{description}%
\item[\texttt{plots}:]
 Cell array of plot\_abstract or subclass objects.
\item[\texttt{axis\_labels}:]
 Cell array of axis label strings.
\item[\texttt{title\_str}:]
 Plot description string.
\item[\texttt{props}:]
 A structure with any optional properties (passed to plot\_abstract).
\end{description}%
%
\item[Returns a structure object with the following fields:]~

	plot\_abstract, plots
%
%
\item[See also:]%
\hyperlink{ref_plot_abstract__superpose}{\texttt{plot\_abstract/superpose}}%
\ (p.~\pageref{ref_plot_abstract__superpose})%
\index[funcref]{plot_abstract@\fidxlb{plot\_abstract}!superpose@\fidxl{superpose}}%
, \hyperlink{ref_plot_superpose__plot}{\texttt{plot\_superpose/plot}}%
\ (p.~\pageref{ref_plot_superpose__plot})%
\index[funcref]{plot_superpose@\fidxlb{plot\_superpose}!plot@\fidxl{plot}}%
%
\item[Author:]%
Cengiz Gunay <cgunay@emory.edu>, 2004/09/22%
\end{description}
\methodline%
\subsubsection[Method \texttt{display}]{Method \texttt{plot\_superpose/display}}%
\index[funcref]{plot_superpose@\fidxlb{plot\_superpose}!display@\fidxl{display}}%
\label{ref_plot_superpose__display}%
\hypertarget{ref_plot_superpose__display}{}%
\begin{description}
%
%
%
%
%
%
%
\item[Author:]%
Cengiz Gunay <cgunay@emory.edu>, 2004/08/04%
\end{description}
\methodline%
\subsubsection[Method \texttt{get}]{Method \texttt{plot\_superpose/get}}%
\index[funcref]{plot_superpose@\fidxlb{plot\_superpose}!get@\fidxl{get}}%
\label{ref_plot_superpose__get}%
\hypertarget{ref_plot_superpose__get}{}%
\begin{description}
\item[Summary:]Defines generic attribute retrieval for objects.
%
%
%
%
%
%
%
\item[Author:]%
Cengiz Gunay <cgunay@emory.edu>, 2004/09/14%
\end{description}
\methodline%
\subsubsection[Method \texttt{set}]{Method \texttt{plot\_superpose/set}}%
\index[funcref]{plot_superpose@\fidxlb{plot\_superpose}!set@\fidxl{set}}%
\label{ref_plot_superpose__set}%
\hypertarget{ref_plot_superpose__set}{}%
\begin{description}
\item[Summary:]Generic method for setting object attributes.
%
%
%
%
%
%
%
\item[Author:]%
Cengiz Gunay <cgunay@emory.edu>, 2004/10/08%
\end{description}
\methodline%
\subsubsection[Method \texttt{axis}]{Method \texttt{plot\_superpose/axis}}%
\index[funcref]{plot_superpose@\fidxlb{plot\_superpose}!axis@\fidxl{axis}}%
\label{ref_plot_superpose__axis}%
\hypertarget{ref_plot_superpose__axis}{}%
\begin{description}
\item[Summary:]Returns the maximal axis ranges according to superposed subplots.
%
\item[Usage:]~%
\begin{lyxcode}%
ranges = axis(a\_plot)
%
\end{lyxcode}%
%
%
\item[Parameters:]~
\begin{description}%
\item[\texttt{a\_plot}:]
 A plot\_abstract object, or a subclass object.
\end{description}%
%
\item[Returns:]~

	ranges: The ranges as a vector in the same way 'axis' would return.
%
%
\item[See also:]%
\hyperlink{ref_plot_abstract}{\texttt{plot\_abstract}}%
\ (p.~\pageref{ref_plot_abstract})%
\index[funcref]{@\fidxl{plot\_abstract}}%
, \hyperlink{ref_plot_abstract__plot}{\texttt{plot\_abstract/plot}}%
\ (p.~\pageref{ref_plot_abstract__plot})%
\index[funcref]{plot_abstract@\fidxlb{plot\_abstract}!plot@\fidxl{plot}}%
%
\item[Author:]%
Cengiz Gunay <cgunay@emory.edu>, 2006/05/22%
\end{description}
\methodline%
\subsubsection[Method \texttt{superposePlots}]{Method \texttt{plot\_superpose/superposePlots}}%
\index[funcref]{plot_superpose@\fidxlb{plot\_superpose}!superposePlots@\fidxl{superposePlots}}%
\label{ref_plot_superpose__superposePlots}%
\hypertarget{ref_plot_superpose__superposePlots}{}%
\begin{description}
\item[Summary:]Superpose multiple plot\_superpose objects by merging them into one.
%
\item[Usage:]~%
\begin{lyxcode}%
a\_plot = superposePlots(plots, axis\_labels, title\_str, command, props)
%
\end{lyxcode}%
%
%
\item[Parameters:]~
\begin{description}%
\item[\texttt{plots}:]
 Array of plot\_superpose objects.
\item[\texttt{axis\_labels}:]
 Cell array of axis label strings (optional, taken from plots).
\item[\texttt{title\_str}:]
 Plot description string (optional, taken from plots).
\item[\texttt{command}:]
 Plotting command to use (optional, taken from plots)
\item[\texttt{props}:]
 A structure with any optional properties.
\begin{description}%
\item[\texttt{noLegends}:]
 If exists, no legends are created.
\end{description}%
\end{description}%
%
\item[Returns:]~

	a\_plot: A plot\_superpose object.
%
%
\item[See also:]%
\hyperlink{ref_plot_abstract__superposePlots}{\texttt{plot\_abstract/superposePlots}}%
\ (p.~\pageref{ref_plot_abstract__superposePlots})%
\index[funcref]{plot_abstract@\fidxlb{plot\_abstract}!superposePlots@\fidxl{superposePlots}}%
, \hyperlink{ref_plot_stack__superposePlots}{\texttt{plot\_stack/superposePlots}}%
\ (p.~\pageref{ref_plot_stack__superposePlots})%
\index[funcref]{plot_stack@\fidxlb{plot\_stack}!superposePlots@\fidxl{superposePlots}}%
%
\item[Author:]%
Cengiz Gunay <cgunay@emory.edu>, 2006/06/14%
\end{description}
\methodline%
\subsubsection[Method \texttt{plot}]{Method \texttt{plot\_superpose/plot}}%
\index[funcref]{plot_superpose@\fidxlb{plot\_superpose}!plot@\fidxl{plot}}%
\label{ref_plot_superpose__plot}%
\hypertarget{ref_plot_superpose__plot}{}%
\begin{description}
\item[Summary:]Draws this plot in the current axis.
%
\item[Usage:]~%
\begin{lyxcode}%
handles = plot(a\_plot, layout\_axis)
%
\end{lyxcode}%
%
%
\item[Parameters:]~
\begin{description}%
\item[\texttt{a\_plot}:]
 A plot\_superpose object.
\item[\texttt{layout\_axis}:]
 The axis position to layout this plot (Optional). 
\end{description}%
%
\item[Returns:]~

	handles: Handles of graphical objects drawn.
%
%
\item[See also:]%
\hyperlink{ref_plot_abstract}{\texttt{plot\_abstract}}%
\ (p.~\pageref{ref_plot_abstract})%
\index[funcref]{@\fidxl{plot\_abstract}}%
%
\item[Author:]%
Cengiz Gunay <cgunay@emory.edu>, 2005/04/08%
\end{description}
\methodline%
\subsubsection[Method \texttt{decorate}]{Method \texttt{plot\_superpose/decorate}}%
\index[funcref]{plot_superpose@\fidxlb{plot\_superpose}!decorate@\fidxl{decorate}}%
\label{ref_plot_superpose__decorate}%
\hypertarget{ref_plot_superpose__decorate}{}%
\begin{description}
\item[Summary:]Places decorations using the first plot of the superposed plots.
%
\item[Usage:]~%
\begin{lyxcode}%
handles = decorate(a\_plot)
%
\end{lyxcode}%
%
%
\item[Parameters:]~
\begin{description}%
\item[\texttt{a\_plot}:]
 A plot\_abstract object, or a subclass object.
\end{description}%
%
\item[Returns:]~

	handles: Handles of graphical objects drawn.
%
%
\item[See also:]%
\hyperlink{ref_plot_abstract}{\texttt{plot\_abstract}}%
\ (p.~\pageref{ref_plot_abstract})%
\index[funcref]{@\fidxl{plot\_abstract}}%
, \hyperlink{ref_plot_abstract__plot}{\texttt{plot\_abstract/plot}}%
\ (p.~\pageref{ref_plot_abstract__plot})%
\index[funcref]{plot_abstract@\fidxlb{plot\_abstract}!plot@\fidxl{plot}}%
%
\item[Author:]%
Cengiz Gunay <cgunay@emory.edu>, 2005/04/11%
\end{description}
\methodline%
\subsection{Class \texttt{ranked\_db}}%
\index[funcref]{ranked_db@\fidxlb{ranked\_db}}%
\label{ref_ranked_db}%
\hypertarget{ref_ranked_db}{}%
\subsubsection[Constructor \texttt{ranked\_db}]{Constructor \texttt{ranked\_db/ranked\_db}}%
\index[funcref]{ranked_db@\fidxlb{ranked\_db}!ranked_db@\fidxl{ranked\_db}}%
\label{ref_ranked_db__ranked_db}%
\hypertarget{ref_ranked_db__ranked_db}{}%
\begin{description}
\item[Summary:]A database of distance values generated by ranking rows of orig\_db with the criterion in crit\_db.
%
\item[Usage:]~%
\begin{lyxcode}%
a\_ranked\_db = ranked\_db(data, col\_names, orig\_db, crit\_db, id, props)
%
\end{lyxcode}%
%
\item[Description:]%
This is a subclass of tests\_db. It should contain a Distance column. A
 more general ranked db class may be needed later. Use the rankMatching method
 to get an instance of this class.
%%
\item[Parameters:]~
\begin{description}%
\item[\texttt{data}:]
 Database contents.
\item[\texttt{col\_names}:]
 The column names.
\item[\texttt{orig\_db}:]
 DB whose rows are ranked.
\item[\texttt{crit\_db}:]
 The criterion DB used for generating the ranking scores.
\item[\texttt{id}:]
 An identifying string.
\item[\texttt{props}:]
 A structure with any optional properties.
\begin{description}%
\item[\texttt{tolerateNaNs}:]
 If 0, rows with any NaN values are skipped (default=1).
\end{description}%
\end{description}%
%
\item[Returns a structure object with the following fields:]~

	tests\_db, orig\_db, crit\_db, props.
%
%
\item[See also:]%
\hyperlink{ref_tests_db}{\texttt{tests\_db}}%
\ (p.~\pageref{ref_tests_db})%
\index[funcref]{@\fidxl{tests\_db}}%
, \hyperlink{ref_tests_db__rankMatching}{\texttt{tests\_db/rankMatching}}%
\ (p.~\pageref{ref_tests_db__rankMatching})%
\index[funcref]{tests_db@\fidxlb{tests\_db}!rankMatching@\fidxl{rankMatching}}%
, \hyperlink{ref_tests_db__matchingRow}{\texttt{tests\_db/matchingRow}}%
\ (p.~\pageref{ref_tests_db__matchingRow})%
\index[funcref]{tests_db@\fidxlb{tests\_db}!matchingRow@\fidxl{matchingRow}}%
%
\item[Author:]%
Cengiz Gunay <cgunay@emory.edu>, 2004/12/21%
\end{description}
\methodline%
\subsubsection[Method \texttt{blockedDistances}]{Method \texttt{ranked\_db/blockedDistances}}%
\index[funcref]{ranked_db@\fidxlb{ranked\_db}!blockedDistances@\fidxl{blockedDistances}}%
\label{ref_ranked_db__blockedDistances}%
\hypertarget{ref_ranked_db__blockedDistances}{}%
\begin{description}
\item[Summary:]Creates a db of distances to blocked versions of top ranks.
%
\item[Usage:]~%
\begin{lyxcode}%
[a\_db, ranked\_dbs] = 
   blockedDistances(a\_ranked\_db, rows, blocked\_db, blocked\_param\_indices, 
	  	     block\_levels, crit\_db)
%
\end{lyxcode}%
%
%
\item[Parameters:]~
\begin{description}%
\item[\texttt{a\_ranked\_db}:]
 A ranked\_db object.
\item[\texttt{rows}:]
 Use the given row rankings.
\item[\texttt{blocked\_db}:]
 db with blocked versions of original ranks.
\item[\texttt{blocked\_param\_indices}:]
 Indices of parameters to be blocked.
\item[\texttt{block\_levels}:]
 Number of parameter levels for blocking.
\item[\texttt{crit\_db}:]
 Calculate distance from this criterion.
\end{description}%
%
\item[Returns:]~

	a\_db: A tests\_db object with the matrix of distances.
	ranked\_dbs: A cell array of ranked\_dbs for each row.
%
\item[Example:]~
\begin{lyxcode}        >> dist\_matx\_db = blockedDistances(rankMatching(super\_db, matchingRow(rsuper\_phys\_db, 20)), 1:5, super\_blocker\_db, [1 2], 10, matchingRow(rsuper\_phys\_db, 21))\\%
\end{lyxcode}
%
\item[See also:]%
\hyperlink{ref_makeModifiedParamDB}{\texttt{makeModifiedParamDB}}%
\ (p.~\pageref{ref_makeModifiedParamDB})%
\index[funcref]{@\fidxl{makeModifiedParamDB}}%
, \hyperlink{ref_getParamRowIndices}{\texttt{getParamRowIndices}}%
\ (p.~\pageref{ref_getParamRowIndices})%
\index[funcref]{@\fidxl{getParamRowIndices}}%
%
\item[Author:]%
Cengiz Gunay <cgunay@emory.edu>, 2005/01/14%
\end{description}
\methodline%
\subsubsection[Method \texttt{get}]{Method \texttt{ranked\_db/get}}%
\index[funcref]{ranked_db@\fidxlb{ranked\_db}!get@\fidxl{get}}%
\label{ref_ranked_db__get}%
\hypertarget{ref_ranked_db__get}{}%
\begin{description}
\item[Summary:]Defines generic attribute retrieval for objects.
%
%
%
%
%
%
%
\item[Author:]%
Cengiz Gunay <cgunay@emory.edu>, 2004/09/14%
\end{description}
\methodline%
\subsubsection[Method \texttt{set}]{Method \texttt{ranked\_db/set}}%
\index[funcref]{ranked_db@\fidxlb{ranked\_db}!set@\fidxl{set}}%
\label{ref_ranked_db__set}%
\hypertarget{ref_ranked_db__set}{}%
\begin{description}
\item[Summary:]Generic method for setting object attributes.
%
%
%
%
%
%
%
\item[Author:]%
Cengiz Gunay <cgunay@emory.edu>, 2004/10/08%
\end{description}
\methodline%
\subsubsection[Method \texttt{plotDistMatrix}]{Method \texttt{ranked\_db/plotDistMatrix}}%
\index[funcref]{ranked_db@\fidxlb{ranked\_db}!plotDistMatrix@\fidxl{plotDistMatrix}}%
\label{ref_ranked_db__plotDistMatrix}%
\hypertarget{ref_ranked_db__plotDistMatrix}{}%
\begin{description}
\item[Summary:]Create a color-coded matrix plot of with total errors from the ranked DB.
%
\item[Usage:]~%
\begin{lyxcode}%
a\_plot = plotDistMatrix(db, rows, col\_size, col\_name, num\_col\_labels, 
			  row\_name, num\_row\_labels, title\_str, props)
%
\end{lyxcode}%
%
\item[Description:]%
The col\_size parameter is used to find the number of rows that make up the 
 x-dimension of the color matrix plot.
%%
\item[Parameters:]~
\begin{description}%
\item[\texttt{db}:]
 A ranked\_db object.
\item[\texttt{rows}:]
 Indices of rows in db after joining (and sorting).
\item[\texttt{col\_size}:]
 Number of rows to take from DB to form the columns of matrix plot.
\item[\texttt{col\_name, row\_name}:]
 DB column to use for the figure column and row, respectively.
\item[\texttt{num\_col\_labels, num\_row\_labels}:]
 Number of labels to put on each axis.
\item[\texttt{title\_str}:]
 If non-empty, replaces generic title with db name. 
\item[\texttt{props}:]
 A structure with any optional properties.
\begin{description}%
\item[\texttt{sortBy}:]
 If specified, db is sorted after being joined with original using this column.
\item[\texttt{colorbar}:]
 Put a colorbar on the figure.

(also passed to plot\_abstract)\end{description}%
\end{description}%
%
\item[Returns:]~

	a\_plot: A plot\_abstract object.
%
\item[Example:]~
\begin{lyxcode} >> plotFigure(plotDistMatrix(scored\_blocked\_sk\_gps0503b\_control\_db, ':', 10, 'SK', 10, 'trial', 10, 'gps0503b (control), preset 6 - top 50 matches', struct('sortBy', 'trial', 'colorbar', 1, 'PaperPosition', [0 0 5 3])));\\%
\end{lyxcode}
%
\item[See also:]%
\hyperlink{ref_ranked_db}{\texttt{ranked\_db}}%
\ (p.~\pageref{ref_ranked_db})%
\index[funcref]{@\fidxl{ranked\_db}}%
, \hyperlink{ref_plot_abstract}{\texttt{plot\_abstract}}%
\ (p.~\pageref{ref_plot_abstract})%
\index[funcref]{@\fidxl{plot\_abstract}}%
, \hyperlink{ref_getDistMatrix}{\texttt{getDistMatrix}}%
\ (p.~\pageref{ref_getDistMatrix})%
\index[funcref]{@\fidxl{getDistMatrix}}%
, \hyperlink{ref_plotCompareDistMatx}{\texttt{plotCompareDistMatx}}%
\ (p.~\pageref{ref_plotCompareDistMatx})%
\index[funcref]{@\fidxl{plotCompareDistMatx}}%
%
\item[Author:]%
Cengiz Gunay <cgunay@emory.edu>, 2005/12/12%
\end{description}
\methodline%
\subsubsection[Method \texttt{plotCompareDistMatx}]{Method \texttt{ranked\_db/plotCompareDistMatx}}%
\index[funcref]{ranked_db@\fidxlb{ranked\_db}!plotCompareDistMatx@\fidxl{plotCompareDistMatx}}%
\label{ref_ranked_db__plotCompareDistMatx}%
\hypertarget{ref_ranked_db__plotCompareDistMatx}{}%
\begin{description}
\item[Summary:]Compare differences and correlations of distance matrices from two ranked DBs.
%
\item[Usage:]~%
\begin{lyxcode}%
a\_plot = plotCompareDistMatx(db, rows, col\_size, col\_name, num\_col\_labels, 
			  row\_name, num\_row\_labels, title\_str, props)
%
\end{lyxcode}%
%
\item[Description:]%
Produces three plots: (1) distance difference matrix, (2) 2D cross-correlogram, 
 and (3) repeated 1D cross-correlogram for each row.
%%
\item[Parameters:]~
\begin{description}%
\item[\texttt{db, w\_db}:]
 The ranked\_db objects to be compared.
\item[\texttt{rows}:]
 Indices of rows in db after joining (and sorting) for both DBs.
\item[\texttt{col\_size}:]
 Number of rows to take from DB to form the columns of matrix plot.
\item[\texttt{col\_name, row\_name}:]
 DB column to use fot the figure column and row, respectively.
\item[\texttt{num\_col\_labels, num\_row\_labels}:]
 Number of labels to put on each axis.
\item[\texttt{title\_str}:]
 If non-empty, replaces generic title with db name. 
\item[\texttt{props}:]
 A structure with any optional properties.
\begin{description}%
\item[\texttt{sortBy}:]
 If specified, db is sorted after being joined with original using this column.
\item[\texttt{colorbar}:]
 Put a colorbar on the figure.

(also passed to plot\_abstract)\end{description}%
\end{description}%
%
\item[Returns:]~

	a\_plot: A plot\_abstract object.
%
%
\item[See also:]%
\hyperlink{ref_tests_db}{\texttt{tests\_db}}%
\ (p.~\pageref{ref_tests_db})%
\index[funcref]{@\fidxl{tests\_db}}%
, \hyperlink{ref_plot_abstract}{\texttt{plot\_abstract}}%
\ (p.~\pageref{ref_plot_abstract})%
\index[funcref]{@\fidxl{plot\_abstract}}%
%
\item[Author:]%
Cengiz Gunay <cgunay@emory.edu>, 2005/12/12%
\end{description}
\methodline%
\subsubsection[Method \texttt{plotRowErrors}]{Method \texttt{ranked\_db/plotRowErrors}}%
\index[funcref]{ranked_db@\fidxlb{ranked\_db}!plotRowErrors@\fidxl{plotRowErrors}}%
\label{ref_ranked_db__plotRowErrors}%
\hypertarget{ref_ranked_db__plotRowErrors}{}%
\begin{description}
\item[Summary:]Create plot of rankings with errors associated with each measure color-coded.
%
\item[Usage:]~%
\begin{lyxcode}%
a\_plot = plotRowErrors(db, rows, props)
%
\end{lyxcode}%
%
%
\item[Parameters:]~
\begin{description}%
\item[\texttt{db}:]
 A tests\_db object.
\item[\texttt{rows}:]
 Indices of rows in db.
\item[\texttt{title\_str}:]
 (Optional) String to append to plot title.
\item[\texttt{props}:]
 A structure with any optional properties.
\begin{description}%
\item[\texttt{sortMeasures}:]
 If specified, measure order is determined with increasing 

overall distance.\item[\texttt{rowSteps}:]
 Steps to jump in labeling rows on the x-axis.

(rest passed to plot\_abstract)\end{description}%
\end{description}%
%
\item[Returns:]~

	a\_plot: A plot\_abstract object.
%
%
\item[See also:]%
\hyperlink{ref_tests_db}{\texttt{tests\_db}}%
\ (p.~\pageref{ref_tests_db})%
\index[funcref]{@\fidxl{tests\_db}}%
, \hyperlink{ref_plot_abstract}{\texttt{plot\_abstract}}%
\ (p.~\pageref{ref_plot_abstract})%
\index[funcref]{@\fidxl{plot\_abstract}}%
%
\item[Author:]%
Cengiz Gunay <cgunay@emory.edu>, 2005/12/12%
\end{description}
\methodline%
\subsubsection[Method \texttt{displayRows}]{Method \texttt{ranked\_db/displayRows}}%
\index[funcref]{ranked_db@\fidxlb{ranked\_db}!displayRows@\fidxl{displayRows}}%
\label{ref_ranked_db__displayRows}%
\hypertarget{ref_ranked_db__displayRows}{}%
\begin{description}
\item[Summary:]Displays rows of rankings together with errors associated with each measure.
%
\item[Usage:]~%
\begin{lyxcode}%
s = displayRows(db, rows)
%
\end{lyxcode}%
%
%
\item[Parameters:]~
\begin{description}%
\item[\texttt{db}:]
 A tests\_db object.
\item[\texttt{rows}:]
 Indices of rows in db.
\end{description}%
%
\item[Returns:]~

	s: A structure of column name and value pairs.
%
%
\item[See also:]%
\hyperlink{ref_tests_db}{\texttt{tests\_db}}%
\ (p.~\pageref{ref_tests_db})%
\index[funcref]{@\fidxl{tests\_db}}%
%
\item[Author:]%
Cengiz Gunay <cgunay@emory.edu>, 2004/09/15%
\end{description}
\methodline%
\subsubsection[Method \texttt{subsref}]{Method \texttt{ranked\_db/subsref}}%
\index[funcref]{ranked_db@\fidxlb{ranked\_db}!subsref@\fidxl{subsref}}%
\label{ref_ranked_db__subsref}%
\hypertarget{ref_ranked_db__subsref}{}%
\begin{description}
\item[Summary:]Defines generic indexing for objects.
%
%
%
%
%
%
%
%
\end{description}
\methodline%
\subsubsection[Method \texttt{joinOriginal}]{Method \texttt{ranked\_db/joinOriginal}}%
\index[funcref]{ranked_db@\fidxlb{ranked\_db}!joinOriginal@\fidxl{joinOriginal}}%
\label{ref_ranked_db__joinOriginal}%
\hypertarget{ref_ranked_db__joinOriginal}{}%
\begin{description}
\item[Summary:]Joins the distance values to the original db rows with matching row indices.
%
\item[Usage:]~%
\begin{lyxcode}%
a\_db = joinOriginal(a\_ranked\_db, rows)
%
\end{lyxcode}%
%
\item[Description:]%
Takes the parameter columns from orig\_db and all tests from crit\_db.
%%
\item[Parameters:]~
\begin{description}%
\item[\texttt{a\_ranked\_db}:]
 A ranked\_db object.
\item[\texttt{rows}:]
 Join only the given rows.
\end{description}%
%
\item[Returns:]~

	a\_db: A params\_tests\_db object (same type as a\_ranked\_db.orig\_db) containing 
		the desired rows in ascending order of distance.
%
%
\item[See also:]%
\hyperlink{ref_tests_db}{\texttt{tests\_db}}%
\ (p.~\pageref{ref_tests_db})%
\index[funcref]{@\fidxl{tests\_db}}%
%
\item[Author:]%
Cengiz Gunay <cgunay@emory.edu>, 2004/12/21%
\end{description}
\methodline%
\subsubsection[Method \texttt{renameColumns}]{Method \texttt{ranked\_db/renameColumns}}%
\index[funcref]{ranked_db@\fidxlb{ranked\_db}!renameColumns@\fidxl{renameColumns}}%
\label{ref_ranked_db__renameColumns}%
\hypertarget{ref_ranked_db__renameColumns}{}%
\begin{description}
\item[Summary:]Rename an existing column or columns.
%
\item[Usage:]~%
\begin{lyxcode}%
a\_db = renameColumns(a\_db, test\_names, new\_names)
%
\end{lyxcode}%
%
\item[Description:]%
This method is an overloaded method for ranked\_db that keeps the column names
 of the ranked, criterion and original DBs consistent.
%%
\item[Parameters:]~
\begin{description}%
\item[\texttt{a\_db}:]
 A ranked\_db object.
\item[\texttt{test\_names}:]
 A cell array of existing test names.
\item[\texttt{new\_names}:]
 New names to replace existing ones.
\end{description}%
%
\item[Returns:]~

	a\_db: The ranked\_db object that includes the new columns.
%
%
\item[See also:]%
\hyperlink{ref_tests_db__renameColumns}{\texttt{tests\_db/renameColumns}}%
\ (p.~\pageref{ref_tests_db__renameColumns})%
\index[funcref]{tests_db@\fidxlb{tests\_db}!renameColumns@\fidxl{renameColumns}}%
%
\item[Author:]%
Cengiz Gunay <cgunay@emory.edu>, 2006/06/07%
\end{description}
\methodline%
\subsubsection[Method \texttt{getDistMatrix}]{Method \texttt{ranked\_db/getDistMatrix}}%
\index[funcref]{ranked_db@\fidxlb{ranked\_db}!getDistMatrix@\fidxl{getDistMatrix}}%
\label{ref_ranked_db__getDistMatrix}%
\hypertarget{ref_ranked_db__getDistMatrix}{}%
\begin{description}
\item[Summary:]Create a matrix of total errors from the ranked DB.
%
\item[Usage:]~%
\begin{lyxcode}%
distmatx = getDistMatrix(db, rows, col\_size, props)
%
\end{lyxcode}%
%
\item[Description:]%
The col\_size parameter is used to find the number of rows that make up the 
 x-dimension of the matrix.
%%
\item[Parameters:]~
\begin{description}%
\item[\texttt{db}:]
 A tests\_db object.
\item[\texttt{rows}:]
 Indices of rows in db after joining (and sorting).
\item[\texttt{col\_size}:]
 Number of rows to take from DB to form the columns of matrix plot.
\item[\texttt{props}:]
 A structure with any optional properties.
\begin{description}%
\item[\texttt{sortBy}:]
 If specified, db is sorted after being joined with original using this column.
\end{description}%
\end{description}%
%
\item[Returns:]~

	a\_plot: A plot\_abstract object.
%
%
\item[See also:]%
\hyperlink{ref_tests_db}{\texttt{tests\_db}}%
\ (p.~\pageref{ref_tests_db})%
\index[funcref]{@\fidxl{tests\_db}}%
, \hyperlink{ref_plot_abstract}{\texttt{plot\_abstract}}%
\ (p.~\pageref{ref_plot_abstract})%
\index[funcref]{@\fidxl{plot\_abstract}}%
%
\item[Author:]%
Cengiz Gunay <cgunay@emory.edu>, 2005/12/12%
\end{description}
\methodline%
\subsection{Class \texttt{results\_profile}}%
\index[funcref]{results_profile@\fidxlb{results\_profile}}%
\label{ref_results_profile}%
\hypertarget{ref_results_profile}{}%
\subsubsection[Constructor \texttt{results\_profile}]{Constructor \texttt{results\_profile/results\_profile}}%
\index[funcref]{results_profile@\fidxlb{results\_profile}!results_profile@\fidxl{results\_profile}}%
\label{ref_results_profile__results_profile}%
\hypertarget{ref_results_profile__results_profile}{}%
\begin{description}
\item[Summary:]Creates and collects result profiles for data objects.
%
\item[Usage:]~%
\begin{lyxcode}%
obj = results\_profile(results, id, props)
%
\end{lyxcode}%
%
\item[Description:]%
This is the base class for all profile classes.
%%
\item[Parameters:]~
\begin{description}%
\item[\texttt{results}:]
 A structure containing test results.
\item[\texttt{id}:]
 Identification string.
\item[\texttt{props}:]
 A structure with any optional properties.
\end{description}%
%
\item[Returns a structure object with the following fields:]~

	results, id, props.
%
%
\item[See also:]%
\hyperlink{ref_trace_profile}{\texttt{trace\_profile}}%
\ (p.~\pageref{ref_trace_profile})%
\index[funcref]{@\fidxl{trace\_profile}}%
, \hyperlink{ref_cip_trace_profile}{\texttt{cip\_trace\_profile}}%
\ (p.~\pageref{ref_cip_trace_profile})%
\index[funcref]{@\fidxl{cip\_trace\_profile}}%
%
\item[Author:]%
Cengiz Gunay <cgunay@emory.edu>, 2004/09/14%
\end{description}
\methodline%
\subsubsection[Method \texttt{display}]{Method \texttt{results\_profile/display}}%
\index[funcref]{results_profile@\fidxlb{results\_profile}!display@\fidxl{display}}%
\label{ref_results_profile__display}%
\hypertarget{ref_results_profile__display}{}%
\begin{description}
%
%
%
%
%
%
%
\item[Author:]%
Cengiz Gunay <cgunay@emory.edu>, 2004/08/04%
\end{description}
\methodline%
\subsubsection[Method \texttt{get}]{Method \texttt{results\_profile/get}}%
\index[funcref]{results_profile@\fidxlb{results\_profile}!get@\fidxl{get}}%
\label{ref_results_profile__get}%
\hypertarget{ref_results_profile__get}{}%
\begin{description}
\item[Summary:]Defines generic attribute retrieval for objects.
%
%
%
%
%
%
%
\item[Author:]%
Cengiz Gunay <cgunay@emory.edu>, 2004/09/14%
\end{description}
\methodline%
\subsubsection[Method \texttt{subsref}]{Method \texttt{results\_profile/subsref}}%
\index[funcref]{results_profile@\fidxlb{results\_profile}!subsref@\fidxl{subsref}}%
\label{ref_results_profile__subsref}%
\hypertarget{ref_results_profile__subsref}{}%
\begin{description}
\item[Summary:]Defines generic indexing for objects.
%
%
%
%
%
%
%
%
\end{description}
\methodline%
\subsubsection[Method \texttt{plot}]{Method \texttt{results\_profile/plot}}%
\index[funcref]{results_profile@\fidxlb{results\_profile}!plot@\fidxl{plot}}%
\label{ref_results_profile__plot}%
\hypertarget{ref_results_profile__plot}{}%
\begin{description}
\item[Summary:]Generic method to plot a tests\_db or a subclass. Requires a 
	plot\_abstract method to be defined for this object.
%
\item[Usage:]~%
\begin{lyxcode}%
h = plot(a\_tests\_db, title\_str)
%
\end{lyxcode}%
%
%
\item[Parameters:]~
\begin{description}%
\item[\texttt{a\_tests\_db}:]
 A histogram\_db object.
\item[\texttt{title\_str}:]
 (Optional) String to append to plot title.
\end{description}%
%
\item[Returns:]~

	h: The figure handle created.
%
%
\item[See also:]%
\hyperlink{ref_plot_abstract}{\texttt{plot\_abstract}}%
\ (p.~\pageref{ref_plot_abstract})%
\index[funcref]{@\fidxl{plot\_abstract}}%
, \hyperlink{ref_plotFigure}{\texttt{plotFigure}}%
\ (p.~\pageref{ref_plotFigure})%
\index[funcref]{@\fidxl{plotFigure}}%
%
\item[Author:]%
Cengiz Gunay <cgunay@emory.edu>, 2004/10/06%
\end{description}
\methodline%
\subsubsection[Method \texttt{getResults}]{Method \texttt{results\_profile/getResults}}%
\index[funcref]{results_profile@\fidxlb{results\_profile}!getResults@\fidxl{getResults}}%
\label{ref_results_profile__getResults}%
\hypertarget{ref_results_profile__getResults}{}%
\begin{description}
\item[Summary:]Return the results profile structure.
%
\item[Usage:]~%
\begin{lyxcode}%
results = getResults(p)
%
\end{lyxcode}%
%
%
\item[Parameters:]~
\begin{description}%
\item[\texttt{p}:]
 A result\_profile object.
\end{description}%
%
\item[Returns:]~

	results: A structure associating test names to values.
%
%
\item[See also:]%
\hyperlink{ref_results_profile}{\texttt{results\_profile}}%
\ (p.~\pageref{ref_results_profile})%
\index[funcref]{@\fidxl{results\_profile}}%
%
\item[Author:]%
Cengiz Gunay <cgunay@emory.edu>, 2004/09/14%
\end{description}
\methodline%
\subsection{Class \texttt{script\_array}}%
\index[funcref]{script_array@\fidxlb{script\_array}}%
\label{ref_script_array}%
\hypertarget{ref_script_array}{}%
\subsubsection[Constructor \texttt{script\_array}]{Constructor \texttt{script\_array/script\_array}}%
\index[funcref]{script_array@\fidxlb{script\_array}!script_array@\fidxl{script\_array}}%
\label{ref_script_array__script_array}%
\hypertarget{ref_script_array__script_array}{}%
\begin{description}
\item[Summary:]Generic class that provides the scripts for a repetitive array job.
%
\item[Usage:]~%
\begin{lyxcode}%
obj = script\_array(num\_runs, id, props)
%
\end{lyxcode}%
%
\item[Description:]%
This is the base class for all script\_array classes. Runs the runJob method as 
 num\_runs many times.
%%
\item[Parameters:]~
\begin{description}%
\item[\texttt{num\_runs}:]
 The number of times the runJob script should be evoked.
\item[\texttt{id}:]
 Identification string.
\item[\texttt{props}:]
 A structure with any optional properties.
\begin{description}%
\item[\texttt{runJobFunc}:]
 A function name or handle to be used instead of default runJob.
\end{description}%
\end{description}%
%
\item[Returns a structure object with the following fields:]~

	num\_runs, id, props.
%
\item[Example:]~
\begin{lyxcode} >> func1 = inline('x\textasciicircum{}2')\\%
 >> runFirst(script\_array(10, 'squares numbers up to 10'), struct('runJobFunc', func1))\\%
 ans = [  1]    [  4]    [  9]    [ 16]    [ 25]    [ 36]    [ 49]    [ 64]    [ 81]    [100]\\%
\end{lyxcode}
%
\item[See also:]%
\hyperlink{ref_runFirst}{\texttt{runFirst}}%
\ (p.~\pageref{ref_runFirst})%
\index[funcref]{@\fidxl{runFirst}}%
, \hyperlink{ref_runLast}{\texttt{runLast}}%
\ (p.~\pageref{ref_runLast})%
\index[funcref]{@\fidxl{runLast}}%
, \hyperlink{ref_runJob}{\texttt{runJob}}%
\ (p.~\pageref{ref_runJob})%
\index[funcref]{@\fidxl{runJob}}%
%
\item[Author:]%
Cengiz Gunay <cgunay@emory.edu>, 2006/02/01%
\end{description}
\methodline%
\subsubsection[Method \texttt{runFirst}]{Method \texttt{script\_array/runFirst}}%
\index[funcref]{script_array@\fidxlb{script\_array}!runFirst@\fidxl{runFirst}}%
\label{ref_script_array__runFirst}%
\hypertarget{ref_script_array__runFirst}{}%
\begin{description}
\item[Summary:]Method to be called at beginning of script\_array jobs.
%
\item[Usage:]~%
\begin{lyxcode}%
job\_results = runFirst(a\_script\_array)
%
\end{lyxcode}%
%
\item[Description:]%
This method initiates the script\_array jobs. It loops and calls runJob and 
 finally calls runLast.
%%
\item[Parameters:]~
\begin{description}%
\item[\texttt{a\_script\_array}:]
 A script\_array object.
\end{description}%
%
\item[Returns:]~

	job\_results: A cell array of results collected from each item of the vector jobs.
%
\item[Example:]~
\begin{lyxcode} >> runFirst(script\_array(10, 'this one does nothing for 10 times'));\\%
\end{lyxcode}
%
\item[See also:]%
\hyperlink{ref_runLast}{\texttt{runLast}}%
\ (p.~\pageref{ref_runLast})%
\index[funcref]{@\fidxl{runLast}}%
, \hyperlink{ref_runJob}{\texttt{runJob}}%
\ (p.~\pageref{ref_runJob})%
\index[funcref]{@\fidxl{runJob}}%
%
\item[Author:]%
Cengiz Gunay <cgunay@emory.edu>, 2006/02/01%
\end{description}
\methodline%
\subsubsection[Method \texttt{get}]{Method \texttt{script\_array/get}}%
\index[funcref]{script_array@\fidxlb{script\_array}!get@\fidxl{get}}%
\label{ref_script_array__get}%
\hypertarget{ref_script_array__get}{}%
\begin{description}
\item[Summary:]Defines generic attribute retrieval for objects.
%
%
%
%
%
%
%
\item[Author:]%
Cengiz Gunay <cgunay@emory.edu>, 2004/09/14%
\end{description}
\methodline%
\subsubsection[Method \texttt{set}]{Method \texttt{script\_array/set}}%
\index[funcref]{script_array@\fidxlb{script\_array}!set@\fidxl{set}}%
\label{ref_script_array__set}%
\hypertarget{ref_script_array__set}{}%
\begin{description}
\item[Summary:]Generic method for setting object attributes.
%
%
%
%
%
%
%
\item[Author:]%
Cengiz Gunay <cgunay@emory.edu>, 2006/02/06%
\end{description}
\methodline%
\subsubsection[Method \texttt{runLast}]{Method \texttt{script\_array/runLast}}%
\index[funcref]{script_array@\fidxlb{script\_array}!runLast@\fidxl{runLast}}%
\label{ref_script_array__runLast}%
\hypertarget{ref_script_array__runLast}{}%
\begin{description}
\item[Summary:]Method to be called last after the script\_array jobs.
%
\item[Usage:]~%
\begin{lyxcode}%
job\_results = runLast(a\_script\_array, job\_results)
%
\end{lyxcode}%
%
\item[Description:]%
This method is provided as a placeholder and does nothing. It can filter-out the
 results returned from the jobs run. Normally it is invoked internally by the runFirst
 method, after running and collecting results from the vector jobs with the runJob method.
%%
\item[Parameters:]~
\begin{description}%
\item[\texttt{a\_script\_array}:]
 A script\_array object.
\item[\texttt{job\_results}:]
 The index within the vector job.
\end{description}%
%
\item[Returns:]~

   job\_results: Any output produced by the job.
%
\item[Example:]~
\begin{lyxcode} Call it directly:\\%
 >> runLast(script\_array(10, 'this one does nothing for 10 times'), {});\\%
\end{lyxcode}
%
\item[See also:]%
\hyperlink{ref_runJob}{\texttt{runJob}}%
\ (p.~\pageref{ref_runJob})%
\index[funcref]{@\fidxl{runJob}}%
, \hyperlink{ref_runFirst}{\texttt{runFirst}}%
\ (p.~\pageref{ref_runFirst})%
\index[funcref]{@\fidxl{runFirst}}%
%
\item[Author:]%
Cengiz Gunay <cgunay@emory.edu>, 2006/02/01%
\end{description}
\methodline%
\subsubsection[Method \texttt{runJob}]{Method \texttt{script\_array/runJob}}%
\index[funcref]{script_array@\fidxlb{script\_array}!runJob@\fidxl{runJob}}%
\label{ref_script_array__runJob}%
\hypertarget{ref_script_array__runJob}{}%
\begin{description}
\item[Summary:]Method to be called for each of the script\_array jobs.
%
\item[Usage:]~%
\begin{lyxcode}%
job\_result = runJob(a\_script\_array, vector\_index)
%
\end{lyxcode}%
%
\item[Description:]%
This method is provided as a placeholder and does nothing. If the run\_job\_func
 property is defined, it will call that function.
%%
\item[Parameters:]~
\begin{description}%
\item[\texttt{a\_script\_array}:]
 A script\_array object.
\item[\texttt{vector\_index}:]
 The index within the vector job.
\end{description}%
%
\item[Returns:]~

   job\_result: Any output produced by the job.
%
\item[Example:]~
\begin{lyxcode} See real example in script\_array. Call the 5th job:\\%
 >> runJob(script\_array(10, 'this one does nothing for 10 times'), 5);\\%
\end{lyxcode}
%
\item[See also:]%
\hyperlink{ref_runLast}{\texttt{runLast}}%
\ (p.~\pageref{ref_runLast})%
\index[funcref]{@\fidxl{runLast}}%
, \hyperlink{ref_runFirst}{\texttt{runFirst}}%
\ (p.~\pageref{ref_runFirst})%
\index[funcref]{@\fidxl{runFirst}}%
%
\item[Author:]%
Cengiz Gunay <cgunay@emory.edu>, 2006/02/01%
\end{description}
\methodline%
\subsubsection[Method \texttt{subsref}]{Method \texttt{script\_array/subsref}}%
\index[funcref]{script_array@\fidxlb{script\_array}!subsref@\fidxl{subsref}}%
\label{ref_script_array__subsref}%
\hypertarget{ref_script_array__subsref}{}%
\begin{description}
\item[Summary:]Defines generic indexing for objects.
%
%
%
%
%
%
%
\item[Author:]%
Cengiz Gunay <cgunay@emory.edu>, 2004/08/04%
\end{description}
\methodline%
\subsubsection[Method \texttt{subsasgn}]{Method \texttt{script\_array/subsasgn}}%
\index[funcref]{script_array@\fidxlb{script\_array}!subsasgn@\fidxl{subsasgn}}%
\label{ref_script_array__subsasgn}%
\hypertarget{ref_script_array__subsasgn}{}%
\begin{description}
\item[Summary:]Defines generic index-based assignment for objects.
%
%
%
%
%
%
%
\item[Author:]%
Cengiz Gunay <cgunay@emory.edu>, 2006/02/06%
\end{description}
\methodline%
\subsection{Class \texttt{script\_array\_for\_cluster}}%
\index[funcref]{script_array_for_cluster@\fidxlb{script\_array\_for\_cluster}}%
\label{ref_script_array_for_cluster}%
\hypertarget{ref_script_array_for_cluster}{}%
\subsubsection[Constructor \texttt{script\_array\_for\_cluster}]{Constructor \texttt{script\_array\_for\_cluster/script\_array\_for\_cluster}}%
\index[funcref]{script_array_for_cluster@\fidxlb{script\_array\_for\_cluster}!script_array_for_cluster@\fidxl{script\_array\_for\_cluster}}%
\label{ref_script_array_for_cluster__script_array_for_cluster}%
\hypertarget{ref_script_array_for_cluster__script_array_for_cluster}{}%
\begin{description}
\item[Summary:]Generic class defining a repetitive vector job to be run on a Sun Grid Engine (SGE) computing cluster.
%
\item[Usage:]~%
\begin{lyxcode}%
a\_script\_cluster = script\_array\_for\_cluster(num\_runs, sge\_wrapper\_script, id, props)
%
\end{lyxcode}%
%
\item[Description:]%
This is a subclass of the script\_array class. The runFirst method spawns num\_runs
 copies of the runJob method in parallel on the cluster, followed by the invocation 
 of the runLast method.
%%
\item[Parameters:]~
\begin{description}%
\item[\texttt{num\_runs}:]
 The number of times the runJob script should be evoked.
\item[\texttt{sge\_wrapper\_script}:]
 A script that can be submitted with qsub and can execute arbitrary

Matlab commands on the cluster nodes. It can have qsub options prepended to it
such as '-p -100 -q all.q <abs\_path\_to>/sge\_matlab.sh'.\item[\texttt{id}:]
 Identification string.
\item[\texttt{props}:]
 A structure with any optional properties.
\begin{description}%
\item[\texttt{notifyByMail}:]
 An SGE notification email is sent to this address after lastJob.

(others passed to script\_array)\end{description}%
\end{description}%
%
\item[Returns a structure object with the following fields:]~

	num\_runs, id, props.
%
%
\item[See also:]%
\hyperlink{ref_runFirst}{\texttt{runFirst}}%
\ (p.~\pageref{ref_runFirst})%
\index[funcref]{@\fidxl{runFirst}}%
, \hyperlink{ref_runLast}{\texttt{runLast}}%
\ (p.~\pageref{ref_runLast})%
\index[funcref]{@\fidxl{runLast}}%
, \hyperlink{ref_runJob}{\texttt{runJob}}%
\ (p.~\pageref{ref_runJob})%
\index[funcref]{@\fidxl{runJob}}%
, \hyperlink{ref_script_array}{\texttt{script\_array}}%
\ (p.~\pageref{ref_script_array})%
\index[funcref]{@\fidxl{script\_array}}%
%
\item[Author:]%
Cengiz Gunay <cgunay@emory.edu>, 2006/02/02%
\end{description}
\methodline%
\subsubsection[Method \texttt{runFirst}]{Method \texttt{script\_array\_for\_cluster/runFirst}}%
\index[funcref]{script_array_for_cluster@\fidxlb{script\_array\_for\_cluster}!runFirst@\fidxl{runFirst}}%
\label{ref_script_array_for_cluster__runFirst}%
\hypertarget{ref_script_array_for_cluster__runFirst}{}%
\begin{description}
\item[Summary:]Method to be called at beginning of script\_array\_for\_cluster jobs.
%
\item[Usage:]~%
\begin{lyxcode}%
job\_results = runFirst(a\_script\_cluster)
%
\end{lyxcode}%
%
\item[Description:]%
This method initiates the script\_array\_for\_cluster jobs. It submits an SGE vector job for running
 each runJob and finally runLast. There is no way of collecting outputs from 
 individual runJob calls.
%%
\item[Parameters:]~
\begin{description}%
\item[\texttt{a\_script\_cluster}:]
 A script\_array\_for\_cluster object.
\end{description}%
%
\item[Returns:]~

	job\_results: A cell array of results collected from each item of the vector jobs.
%
\item[Example:]~
\begin{lyxcode} >> runFirst(script\_array\_for\_cluster(10, 'this one does nothing for 10 times'));\\%
\end{lyxcode}
%
\item[See also:]%
\hyperlink{ref_script_array_for_cluster}{\texttt{script\_array\_for\_cluster}}%
\ (p.~\pageref{ref_script_array_for_cluster})%
\index[funcref]{@\fidxl{script\_array\_for\_cluster}}%
%
\item[Author:]%
Cengiz Gunay <cgunay@emory.edu>, 2006/02/01%
\end{description}
\methodline%
\subsubsection[Method \texttt{get}]{Method \texttt{script\_array\_for\_cluster/get}}%
\index[funcref]{script_array_for_cluster@\fidxlb{script\_array\_for\_cluster}!get@\fidxl{get}}%
\label{ref_script_array_for_cluster__get}%
\hypertarget{ref_script_array_for_cluster__get}{}%
\begin{description}
\item[Summary:]Defines generic attribute retrieval for objects.
%
%
%
%
%
%
%
\item[Author:]%
Cengiz Gunay <cgunay@emory.edu>, 2004/09/14%
\end{description}
\methodline%
\subsubsection[Method \texttt{set}]{Method \texttt{script\_array\_for\_cluster/set}}%
\index[funcref]{script_array_for_cluster@\fidxlb{script\_array\_for\_cluster}!set@\fidxl{set}}%
\label{ref_script_array_for_cluster__set}%
\hypertarget{ref_script_array_for_cluster__set}{}%
\begin{description}
\item[Summary:]Generic method for setting object attributes.
%
%
%
%
%
%
%
\item[Author:]%
Cengiz Gunay <cgunay@emory.edu>, 2006/02/06%
\end{description}
\methodline%
\subsection{Class \texttt{script\_factory}}%
\index[funcref]{script_factory@\fidxlb{script\_factory}}%
\label{ref_script_factory}%
\hypertarget{ref_script_factory}{}%
\subsubsection[Constructor \texttt{script\_factory}]{Constructor \texttt{script\_factory/script\_factory}}%
\index[funcref]{script_factory@\fidxlb{script\_factory}!script_factory@\fidxl{script\_factory}}%
\label{ref_script_factory__script_factory}%
\hypertarget{ref_script_factory__script_factory}{}%
\begin{description}
\item[Summary:]Generic class to automatically create a set of scripts.
%
\item[Usage:]~%
\begin{lyxcode}%
obj = script\_factory(num\_scripts, out\_name, id, props)
%
\end{lyxcode}%
%
\item[Description:]%
This is the base class for all script\_factory classes.
%%
\item[Parameters:]~
\begin{description}%
\item[\texttt{num\_scripts}:]
 Number of scripts to create.
\item[\texttt{out\_name}:]
 The file name for the output scripts. A '%d' in the

filename corresponds to the script number.\item[\texttt{id}:]
 Identification string.
\item[\texttt{props}:]
 A structure with any optional properties.
\end{description}%
%
\item[Returns a structure object with the following fields:]~

	num\_scripts, out\_name, id, props.
%
%
\item[See also:]%
\hyperlink{ref_script_factory__writeScripts}{\texttt{script\_factory/writeScripts}}%
\ (p.~\pageref{ref_script_factory__writeScripts})%
\index[funcref]{script_factory@\fidxlb{script\_factory}!writeScripts@\fidxl{writeScripts}}%
%
\item[Author:]%
Cengiz Gunay <cgunay@emory.edu>, 2005/11/28%
\end{description}
\methodline%
\subsubsection[Method \texttt{get}]{Method \texttt{script\_factory/get}}%
\index[funcref]{script_factory@\fidxlb{script\_factory}!get@\fidxl{get}}%
\label{ref_script_factory__get}%
\hypertarget{ref_script_factory__get}{}%
\begin{description}
\item[Summary:]Defines generic attribute retrieval for objects.
%
%
%
%
%
%
%
\item[Author:]%
Cengiz Gunay <cgunay@emory.edu>, 2004/09/14%
\end{description}
\methodline%
\subsection{Class \texttt{spike\_shape}}%
\index[funcref]{spike_shape@\fidxlb{spike\_shape}}%
\label{ref_spike_shape}%
\hypertarget{ref_spike_shape}{}%
\subsubsection[Constructor \texttt{spike\_shape}]{Constructor \texttt{spike\_shape/spike\_shape}}%
\index[funcref]{spike_shape@\fidxlb{spike\_shape}!spike_shape@\fidxl{spike\_shape}}%
\label{ref_spike_shape__spike_shape}%
\hypertarget{ref_spike_shape__spike_shape}{}%
\begin{description}
\item[Summary:]An action potential shape trace.
%
\item[Usage:]~%
\begin{lyxcode}%
obj = spike\_shape(data, dt, dy, id)
%
\end{lyxcode}%
%
%
\item[Parameters:]~
\begin{description}%
\item[\texttt{data}:]
 A vector of data points containing the spike shape.
\item[\texttt{dt}:]
 Time resolution [s].
\item[\texttt{dy}:]
 y-axis resolution [ISI (V, A, etc.)]
\item[\texttt{id}:]
 Identification string.
\item[\texttt{props}:]
 A structure with any optional properties.
\begin{description}%
\item[\texttt{baseline}:]
 Resting potential.
\item[\texttt{threshold}:]
 Spike threshold.
\item[\texttt{init\_Vm\_method}:]
 Method to obtain spike initiation voltage.

1- maximum acceleration point
2- threshold crossing of acceleration (needs threshold)
3- threshold crossing of slope (needs threshold)
4- maximum acceleration in phase space
(optionally specify maximal threshold as init\_threshold)
5- point of maximum curvature, when slope is between 
init\_lo\_thr and init\_hi\_thr
6- local maximum of second derivative in the phase space
nearest slope crossing init\_threshold
7- threshold crossing of interpolated slope (needs threshold)
8- maximum curvature in phase-plane
9- Combined curvature and inflection method in time-domain.\item[\texttt{init\_threshold}:]
 Spike initiation threshold (deriv or accel).

(see above methods and implementation in calcInitVm)\item[\texttt{init\_lo\_thr, init\_hi\_thr}:]
 Low and high thresholds for slope.
\end{description}%
\end{description}%
%
\item[Returns a structure object with the following fields:]~

	trace, props.
%
%
\item[See also:]%
\hyperlink{ref_trace__spike_shape}{\texttt{trace/spike\_shape}}%
\ (p.~\pageref{ref_trace__spike_shape})%
\index[funcref]{trace@\fidxlb{trace}!spike_shape@\fidxl{spike\_shape}}%
, \hyperlink{ref_trace__analyzeSpikesInPeriod}{\texttt{trace/analyzeSpikesInPeriod}}%
\ (p.~\pageref{ref_trace__analyzeSpikesInPeriod})%
\index[funcref]{trace@\fidxlb{trace}!analyzeSpikesInPeriod@\fidxl{analyzeSpikesInPeriod}}%
, \hyperlink{ref_trace}{\texttt{trace}}%
\ (p.~\pageref{ref_trace})%
\index[funcref]{@\fidxl{trace}}%
, \hyperlink{ref_spikes}{\texttt{spikes}}%
\ (p.~\pageref{ref_spikes})%
\index[funcref]{@\fidxl{spikes}}%
, \hyperlink{ref_period}{\texttt{period}}%
\ (p.~\pageref{ref_period})%
\index[funcref]{@\fidxl{period}}%
%
\item[Author:]%
Cengiz Gunay <cgunay@emory.edu>, 2004/07/30%
\end{description}
\methodline%
\subsubsection[Method \texttt{calcInitVmSlopeThresholdSupsample}]{Method \texttt{spike\_shape/calcInitVmSlopeThresholdSupsample}}%
\index[funcref]{spike_shape@\fidxlb{spike\_shape}!calcInitVmSlopeThresholdSupsample@\fidxl{calcInitVmSlopeThresholdSupsample}}%
\label{ref_spike_shape__calcInitVmSlopeThresholdSupsample}%
\hypertarget{ref_spike_shape__calcInitVmSlopeThresholdSupsample}{}%
\begin{description}
\item[Summary:]Estimates the AP threshold as the first slope threshold crossing by first supersampling the data using cubic spline interpolation.
%
\item[Usage:]~%
\begin{lyxcode}%
[init\_idx, a\_plot] = calcInitVmSlopeThresholdSupsample(s, max\_idx, min\_idx, thr, plotit)
%
\end{lyxcode}%
%
%
\item[Parameters:]~
\begin{description}%
\item[\texttt{s}:]
 A spike\_shape object.
\item[\texttt{max\_idx}:]
 The index of the maximal point of the spike\_shape [dt].
\item[\texttt{min\_idx}:]
 The index of the minimal point of the spike\_shape [dt].
\item[\texttt{thr}:]
 Threshold for time derivative of voltage.
\item[\texttt{plotit}:]
 If non-zero, plot a graph annotating the test results 

(optional).\end{description}%
%
\item[Returns:]~

	init\_idx: AP threshold index in the spike\_shape [dt].
	a\_plot: plot\_abstract, if requested.
%
%
\item[See also:]%
\hyperlink{ref_calcInitVm}{\texttt{calcInitVm}}%
\ (p.~\pageref{ref_calcInitVm})%
\index[funcref]{@\fidxl{calcInitVm}}%
%
\item[Author:]%
Cengiz Gunay <cgunay@emory.edu>, 2005/03/23%
\end{description}
\methodline%
\subsubsection[Method \texttt{plotCompareMethods}]{Method \texttt{spike\_shape/plotCompareMethods}}%
\index[funcref]{spike_shape@\fidxlb{spike\_shape}!plotCompareMethods@\fidxl{plotCompareMethods}}%
\label{ref_spike_shape__plotCompareMethods}%
\hypertarget{ref_spike_shape__plotCompareMethods}{}%
\begin{description}
\item[Summary:]Creates a multi-plot comparing different action potential
			threshold finding methods.
%
\item[Usage:]~%
\begin{lyxcode}%
a\_plot = plotCompareMethods(s, title\_str)
%
\end{lyxcode}%
%
%
\item[Parameters:]~
\begin{description}%
\item[\texttt{s}:]
 A spike\_shape object.
\item[\texttt{title\_str}:]
 Title suffix (optional).
\end{description}%
%
\item[Returns:]~

	a\_plot: A plot\_abstract object that can be visualized.
%
%
\item[See also:]%
\hyperlink{ref_spike_shape}{\texttt{spike\_shape}}%
\ (p.~\pageref{ref_spike_shape})%
\index[funcref]{@\fidxl{spike\_shape}}%
, \hyperlink{ref_plot_abstract}{\texttt{plot\_abstract}}%
\ (p.~\pageref{ref_plot_abstract})%
\index[funcref]{@\fidxl{plot\_abstract}}%
%
\item[Author:]%
Cengiz Gunay <cgunay@emory.edu>, 2004/11/19%
\end{description}
\methodline%
\subsubsection[Method \texttt{display}]{Method \texttt{spike\_shape/display}}%
\index[funcref]{spike_shape@\fidxlb{spike\_shape}!display@\fidxl{display}}%
\label{ref_spike_shape__display}%
\hypertarget{ref_spike_shape__display}{}%
\begin{description}
%
%
%
%
%
%
%
\item[Author:]%
Cengiz Gunay <cgunay@emory.edu>, 2004/08/04%
\end{description}
\methodline%
\subsubsection[Method \texttt{calcMaxVm}]{Method \texttt{spike\_shape/calcMaxVm}}%
\index[funcref]{spike_shape@\fidxlb{spike\_shape}!calcMaxVm@\fidxl{calcMaxVm}}%
\label{ref_spike_shape__calcMaxVm}%
\hypertarget{ref_spike_shape__calcMaxVm}{}%
\begin{description}
\item[Summary:]Calculates the maximal value of the spike\_shape, s. 
%
\item[Usage:]~%
\begin{lyxcode}%
[max\_val, max\_idx] = calcMaxVm(s)
%
\end{lyxcode}%
%
%
\item[Parameters:]~
\begin{description}%
\item[\texttt{s}:]
 A spike\_shape object.
\end{description}%
%
\item[Returns:]~

	max\_val: The max value.
	max\_idx: Its index in the spike\_shape [dt].
%
%
\item[See also:]%
\hyperlink{ref_period}{\texttt{period}}%
\ (p.~\pageref{ref_period})%
\index[funcref]{@\fidxl{period}}%
, \hyperlink{ref_spike_shape}{\texttt{spike\_shape}}%
\ (p.~\pageref{ref_spike_shape})%
\index[funcref]{@\fidxl{spike\_shape}}%
, \hyperlink{ref_trace__calcMax}{\texttt{trace/calcMax}}%
\ (p.~\pageref{ref_trace__calcMax})%
\index[funcref]{trace@\fidxlb{trace}!calcMax@\fidxl{calcMax}}%
%
\item[Author:]%
Cengiz Gunay <cgunay@emory.edu>, 2004/08/02%
\end{description}
\methodline%
\subsubsection[Method \texttt{get}]{Method \texttt{spike\_shape/get}}%
\index[funcref]{spike_shape@\fidxlb{spike\_shape}!get@\fidxl{get}}%
\label{ref_spike_shape__get}%
\hypertarget{ref_spike_shape__get}{}%
\begin{description}
\item[Summary:]Defines generic attribute retrieval for objects.
%
%
%
%
%
%
%
\item[Author:]%
Cengiz Gunay <cgunay@emory.edu>, 2004/09/14%
\end{description}
\methodline%
\subsubsection[Method \texttt{calcInitVmV3hKpTinterp}]{Method \texttt{spike\_shape/calcInitVmV3hKpTinterp}}%
\index[funcref]{spike_shape@\fidxlb{spike\_shape}!calcInitVmV3hKpTinterp@\fidxl{calcInitVmV3hKpTinterp}}%
\label{ref_spike_shape__calcInitVmV3hKpTinterp}%
\hypertarget{ref_spike_shape__calcInitVmV3hKpTinterp}{}%
\begin{description}
\item[Summary:]Calculates candidates for action potential threshold using the first three time-domain derivatives.
%
\item[Usage:]~%
\begin{lyxcode}%
[init\_idx, a\_plot] = 
   calcInitVmV3hKpTinterp(s, max\_idx, min\_idx, lo\_thr, hi\_thr, plotit)
%
\end{lyxcode}%
%
\item[Description:]%
First uses interpolation to increase time points. Calculates h,
 the second derivative of phase-plane (d\textasciicircum{}2 v'/dv\textasciicircum{}2), in terms of 
 time-domain derivatives. Also calculates Kp = V''[1 + (V')\textasciicircum{}2]\textasciicircum{}(-3/2), 
 the curvature. The maxima of these functions are used as candidates 
 for AP thresholds.
%%
\item[Parameters:]~
\begin{description}%
\item[\texttt{s}:]
 A spike\_shape object.
\item[\texttt{max\_idx}:]
 The index of the maximal point of the spike\_shape [dt].
\item[\texttt{min\_idx}:]
 The index of the minimal point of the spike\_shape [dt].
\item[\texttt{lo\_thr, hi\_thr}:]
 Lower and higher thresholds for time derivative of voltage.
\item[\texttt{plotit}:]
 If non-zero, plot a graph annotating the test results 

(optional).\end{description}%
%
\item[Returns:]~

	init\_idx: Indices of threshold candidates in the spike\_shape [dt].
	a\_plot: plot\_abstract, if requested.
%
%
\item[See also:]%
\hyperlink{ref_calcInitVm}{\texttt{calcInitVm}}%
\ (p.~\pageref{ref_calcInitVm})%
\index[funcref]{@\fidxl{calcInitVm}}%
%
\item[Author:]%
Cengiz Gunay <cgunay@emory.edu>, 2004/11/18%
\end{description}
\methodline%
\subsubsection[Method \texttt{set}]{Method \texttt{spike\_shape/set}}%
\index[funcref]{spike_shape@\fidxlb{spike\_shape}!set@\fidxl{set}}%
\label{ref_spike_shape__set}%
\hypertarget{ref_spike_shape__set}{}%
\begin{description}
\item[Summary:]Generic method for setting object attributes.
%
%
%
%
%
%
%
\item[Author:]%
Cengiz Gunay <cgunay@emory.edu>, 2004/10/08%
\end{description}
\methodline%
\subsubsection[Method \texttt{calcInitVmSekerliV2}]{Method \texttt{spike\_shape/calcInitVmSekerliV2}}%
\index[funcref]{spike_shape@\fidxlb{spike\_shape}!calcInitVmSekerliV2@\fidxl{calcInitVmSekerliV2}}%
\label{ref_spike_shape__calcInitVmSekerliV2}%
\hypertarget{ref_spike_shape__calcInitVmSekerliV2}{}%
\begin{description}
\item[Summary:]Calculates the action potential threshold using the maximum second derivative of the phase space of voltage-time slope versus voltage.
%
\item[Usage:]~%
\begin{lyxcode}%
[init\_idx, a\_plot] = calcInitVmSekerliV2(s, max\_idx, min\_idx, plotit)
%
\end{lyxcode}%
%
%
\item[Parameters:]~
\begin{description}%
\item[\texttt{s}:]
 A spike\_shape object.
\item[\texttt{max\_idx}:]
 The index of the maximal point of the spike\_shape [dt].
\item[\texttt{min\_idx}:]
 The index of the minimal point of the spike\_shape [dt].
\item[\texttt{plotit}:]
 If non-zero, plot a graph annotating the test results 

(optional).\end{description}%
%
\item[Returns:]~

	init\_idx: Its index in the spike\_shape [dt].
	a\_plot: plot\_abstract, if requested.
%
%
\item[See also:]%
\hyperlink{ref_calcInitVm}{\texttt{calcInitVm}}%
\ (p.~\pageref{ref_calcInitVm})%
\index[funcref]{@\fidxl{calcInitVm}}%
%
\item[Author:]%
Cengiz Gunay <cgunay@emory.edu>, 2004/11/18%
\end{description}
\methodline%
\subsubsection[Method \texttt{calcMinVm}]{Method \texttt{spike\_shape/calcMinVm}}%
\index[funcref]{spike_shape@\fidxlb{spike\_shape}!calcMinVm@\fidxl{calcMinVm}}%
\label{ref_spike_shape__calcMinVm}%
\hypertarget{ref_spike_shape__calcMinVm}{}%
\begin{description}
\item[Summary:]Calculates the minimal value of the spike\_shape, s. 
%
\item[Usage:]~%
\begin{lyxcode}%
[min\_val, min\_idx, max\_min\_time] = calcMinVm(s, max\_idx)
%
\end{lyxcode}%
%
%
\item[Parameters:]~
\begin{description}%
\item[\texttt{s}:]
 A spike\_shape object.
\item[\texttt{max\_idx}:]
 The index of the maximal point of the spike\_shape [dt].
\end{description}%
%
\item[Returns:]~

	min\_val: The min value [dy].
	min\_idx: Its index in the spike\_shape [dt].
	max\_min\_time: Time from max to min [dt].
%
%
\item[See also:]%
\hyperlink{ref_period}{\texttt{period}}%
\ (p.~\pageref{ref_period})%
\index[funcref]{@\fidxl{period}}%
, \hyperlink{ref_spike_shape}{\texttt{spike\_shape}}%
\ (p.~\pageref{ref_spike_shape})%
\index[funcref]{@\fidxl{spike\_shape}}%
, \hyperlink{ref_trace__calcMin}{\texttt{trace/calcMin}}%
\ (p.~\pageref{ref_trace__calcMin})%
\index[funcref]{trace@\fidxlb{trace}!calcMin@\fidxl{calcMin}}%
%
\item[Author:]%
Cengiz Gunay <cgunay@emory.edu>, 2004/08/02%
\end{description}
\methodline%
\subsubsection[Method \texttt{plotTPP}]{Method \texttt{spike\_shape/plotTPP}}%
\index[funcref]{spike_shape@\fidxlb{spike\_shape}!plotTPP@\fidxl{plotTPP}}%
\label{ref_spike_shape__plotTPP}%
\hypertarget{ref_spike_shape__plotTPP}{}%
\begin{description}
\item[Summary:]Plots the dV/dt vs. V phase-plane representation of the spike shape.
%
\item[Usage:]~%
\begin{lyxcode}%
a\_plot = plotTPP(s)
%
\end{lyxcode}%
%
\item[Description:]%
Uses the Taylor series estimation for finding the derivative dV/dt.
%%
\item[Parameters:]~
\begin{description}%
\item[\texttt{s}:]
 A spike\_shape object.
\end{description}%
%
\item[Returns:]~

	a\_plot: A plot\_abstract object that can be visualized.
%
%
\item[See also:]%
\hyperlink{ref_spike_shape}{\texttt{spike\_shape}}%
\ (p.~\pageref{ref_spike_shape})%
\index[funcref]{@\fidxl{spike\_shape}}%
, \hyperlink{ref_plot_abstract}{\texttt{plot\_abstract}}%
\ (p.~\pageref{ref_plot_abstract})%
\index[funcref]{@\fidxl{plot\_abstract}}%
, \hyperlink{ref_diffT}{\texttt{diffT}}%
\ (p.~\pageref{ref_diffT})%
\index[funcref]{@\fidxl{diffT}}%
%
\item[Author:]%
Cengiz Gunay <cgunay@emory.edu>, 2004/11/16%
\end{description}
\methodline%
\subsubsection[Method \texttt{calcInitVm}]{Method \texttt{spike\_shape/calcInitVm}}%
\index[funcref]{spike_shape@\fidxlb{spike\_shape}!calcInitVm@\fidxl{calcInitVm}}%
\label{ref_spike_shape__calcInitVm}%
\hypertarget{ref_spike_shape__calcInitVm}{}%
\begin{description}
\item[Summary:]Calculates spike threshold related measures of the spike\_shape, s. 
%
\item[Usage:]~%
\begin{lyxcode}%
[init\_val, init\_idx, rise\_time, amplitude,
  peak\_mag, peak\_idx, max\_d1o, a\_plot] = 
	calcInitVm(s, max\_idx, min\_idx)
%
\end{lyxcode}%
%
%
\item[Parameters:]~
\begin{description}%
\item[\texttt{s}:]
 A spike\_shape object.
\item[\texttt{max\_idx}:]
 The index of the maximal point of the spike\_shape [dt].
\item[\texttt{min\_idx}:]
 The index of the minimal point of the spike\_shape [dt].
\item[\texttt{plotit}:]
 If non-zero, plot a graph annotating the test results 

(optional).\end{description}%
%
\item[Returns:]~

	init\_val: The potential value [dy].
	init\_idx: Its index in the spike\_shape [dt].
	rise\_time: Time from initiation to maximum [dt].
	amplitude: Magnitude from initiation to max [dy].
	peak\_mag: Peak value [dy].
	peak\_idx: Extrapolated spike peak index [dt].
	max\_d1o: Maximal value of first voltage derivative [dy].
	a\_plot: plot\_abstract, if requested.
%
%
\item[See also:]%
\hyperlink{ref_spike_shape}{\texttt{spike\_shape}}%
\ (p.~\pageref{ref_spike_shape})%
\index[funcref]{@\fidxl{spike\_shape}}%
%
\item[Author:]%
Cengiz Gunay <cgunay@emory.edu>, 2004/08/02%
\end{description}
\methodline%
\subsubsection[Method \texttt{calcInitVmSlopeThreshold}]{Method \texttt{spike\_shape/calcInitVmSlopeThreshold}}%
\index[funcref]{spike_shape@\fidxlb{spike\_shape}!calcInitVmSlopeThreshold@\fidxl{calcInitVmSlopeThreshold}}%
\label{ref_spike_shape__calcInitVmSlopeThreshold}%
\hypertarget{ref_spike_shape__calcInitVmSlopeThreshold}{}%
\begin{description}
\item[Summary:]Calculates the AP threshold using the slope threhold crossing.
%
\item[Usage:]~%
\begin{lyxcode}%
[init\_idx, a\_plot] = calcInitVmSlopeThreshold(s, max\_idx, min\_idx, thr, plotit)
%
\end{lyxcode}%
%
%
\item[Parameters:]~
\begin{description}%
\item[\texttt{s}:]
 A spike\_shape object.
\item[\texttt{max\_idx}:]
 The index of the maximal point of the spike\_shape [dt].
\item[\texttt{min\_idx}:]
 The index of the minimal point of the spike\_shape [dt].
\item[\texttt{thr}:]
 Threshold for time derivative of voltage.
\item[\texttt{plotit}:]
 If non-zero, plot a graph annotating the test results 

(optional).\end{description}%
%
\item[Returns:]~

	init\_idx: AP threshold index in the spike\_shape [dt].
	a\_plot: plot\_abstract, if requested.
%
%
\item[See also:]%
\hyperlink{ref_calcInitVm}{\texttt{calcInitVm}}%
\ (p.~\pageref{ref_calcInitVm})%
\index[funcref]{@\fidxl{calcInitVm}}%
%
\item[Author:]%
Cengiz Gunay <cgunay@emory.edu>, 2004/11/24%
\end{description}
\methodline%
\subsubsection[Method \texttt{plotCompareMethodsSimple}]{Method \texttt{spike\_shape/plotCompareMethodsSimple}}%
\index[funcref]{spike_shape@\fidxlb{spike\_shape}!plotCompareMethodsSimple@\fidxl{plotCompareMethodsSimple}}%
\label{ref_spike_shape__plotCompareMethodsSimple}%
\hypertarget{ref_spike_shape__plotCompareMethodsSimple}{}%
\begin{description}
\item[Summary:]Creates a multi-plot comparing different action potential
			threshold finding methods.
%
\item[Usage:]~%
\begin{lyxcode}%
a\_plot = plotCompareMethodsSimple(s, title\_str)
%
\end{lyxcode}%
%
%
\item[Parameters:]~
\begin{description}%
\item[\texttt{s}:]
 A spike\_shape object.
\item[\texttt{title\_str}:]
 Title suffix (optional).
\end{description}%
%
\item[Returns:]~

	a\_plot: A plot\_abstract object that can be visualized.
%
%
\item[See also:]%
\hyperlink{ref_spike_shape}{\texttt{spike\_shape}}%
\ (p.~\pageref{ref_spike_shape})%
\index[funcref]{@\fidxl{spike\_shape}}%
, \hyperlink{ref_plot_abstract}{\texttt{plot\_abstract}}%
\ (p.~\pageref{ref_plot_abstract})%
\index[funcref]{@\fidxl{plot\_abstract}}%
%
\item[Author:]%
Cengiz Gunay <cgunay@emory.edu>, 2004/11/19%
\end{description}
\methodline%
\subsubsection[Method \texttt{calcInitVmMaxCurvature}]{Method \texttt{spike\_shape/calcInitVmMaxCurvature}}%
\index[funcref]{spike_shape@\fidxlb{spike\_shape}!calcInitVmMaxCurvature@\fidxl{calcInitVmMaxCurvature}}%
\label{ref_spike_shape__calcInitVmMaxCurvature}%
\hypertarget{ref_spike_shape__calcInitVmMaxCurvature}{}%
\begin{description}
\item[Summary:]Calculates the action potential threshold using the
			maximum of the curvature equation.
%
\item[Usage:]~%
\begin{lyxcode}%
[init\_idx, a\_plot] = calcInitVmMaxCurvature(s, max\_idx, min\_idx, plotit)
%
\end{lyxcode}%
%
\item[Description:]%
Point of maximum curvature: Kp = V''[1 + (V')\textasciicircum{}2]\textasciicircum{}(-3/2)
 Taken from Sekerli, Del Negro, Lee and Butera. 
 IEEE Trans. Biomed. Eng., 51(9): 1665-71, 2004.
%%
\item[Parameters:]~
\begin{description}%
\item[\texttt{s}:]
 A spike\_shape object.
\item[\texttt{max\_idx}:]
 The index of the maximal point of the spike\_shape [dt].
\item[\texttt{min\_idx}:]
 The index of the minimal point of the spike\_shape [dt].
\item[\texttt{plotit}:]
 If non-zero, plot a graph annotating the test results 

(optional).\end{description}%
%
\item[Returns:]~

	init\_idx: AP threshold index in the spike\_shape [dt].
	a\_plot: plot\_abstract, if requested.
%
%
\item[See also:]%
\hyperlink{ref_calcInitVm}{\texttt{calcInitVm}}%
\ (p.~\pageref{ref_calcInitVm})%
\index[funcref]{@\fidxl{calcInitVm}}%
%
\item[Author:]%
Cengiz Gunay <cgunay@emory.edu>, 2004/11/19%
\end{description}
\methodline%
\subsubsection[Method \texttt{calcInitVmV2PPLocal}]{Method \texttt{spike\_shape/calcInitVmV2PPLocal}}%
\index[funcref]{spike_shape@\fidxlb{spike\_shape}!calcInitVmV2PPLocal@\fidxl{calcInitVmV2PPLocal}}%
\label{ref_spike_shape__calcInitVmV2PPLocal}%
\hypertarget{ref_spike_shape__calcInitVmV2PPLocal}{}%
\begin{description}
\item[Summary:]Calculates the action potential threshold by finding the local second derivative maximum in voltage-time slope versus voltage phase plane, nearest a slope threshold crossing.
%
\item[Usage:]~%
\begin{lyxcode}%
[init\_idx, a\_plot] = calcInitVmV2PPLocal(s, max\_idx, min\_idx, lo\_thr, plotit)
%
\end{lyxcode}%
%
%
\item[Parameters:]~
\begin{description}%
\item[\texttt{s}:]
 A spike\_shape object.
\item[\texttt{max\_idx}:]
 The index of the maximal point of the spike\_shape [dt].
\item[\texttt{min\_idx}:]
 The index of the minimal point of the spike\_shape [dt].
\item[\texttt{lo\_thr}:]
 Lower threshold for time voltage slope.
\item[\texttt{plotit}:]
 If non-zero, plot a graph annotating the test results 

(optional).\end{description}%
%
\item[Returns:]~

	init\_idx: Its index in the spike\_shape [dt].
	a\_plot: plot\_abstract, if requested.
%
%
\item[See also:]%
\hyperlink{ref_calcInitVm}{\texttt{calcInitVm}}%
\ (p.~\pageref{ref_calcInitVm})%
\index[funcref]{@\fidxl{calcInitVm}}%
%
\item[Author:]%
Cengiz Gunay <cgunay@emory.edu>, 2004/11/18%
\end{description}
\methodline%
\subsubsection[Method \texttt{getResults}]{Method \texttt{spike\_shape/getResults}}%
\index[funcref]{spike_shape@\fidxlb{spike\_shape}!getResults@\fidxl{getResults}}%
\label{ref_spike_shape__getResults}%
\hypertarget{ref_spike_shape__getResults}{}%
\begin{description}
\item[Summary:]Runs all tests defined by this class and return them in a 
		structure.
%
\item[Usage:]~%
\begin{lyxcode}%
[results, a\_plot] = getResults(s, plotit)
%
\end{lyxcode}%
%
%
\item[Parameters:]~
\begin{description}%
\item[\texttt{s}:]
 A spike\_shape object.
\item[\texttt{plotit}:]
 If non-zero, plot a graph annotating the test results 

(optional).\end{description}%
%
\item[Returns:]~

	results: A structure associating test names to values in ms and mV.
	a\_plot: plot\_abstract, if requested.
%
%
\item[See also:]%
\hyperlink{ref_spike_shape}{\texttt{spike\_shape}}%
\ (p.~\pageref{ref_spike_shape})%
\index[funcref]{@\fidxl{spike\_shape}}%
%
\item[Author:]%
Cengiz Gunay <cgunay@emory.edu>, 2004/08/02%
\end{description}
\methodline%
\subsubsection[Method \texttt{calcWidthFall}]{Method \texttt{spike\_shape/calcWidthFall}}%
\index[funcref]{spike_shape@\fidxlb{spike\_shape}!calcWidthFall@\fidxl{calcWidthFall}}%
\label{ref_spike_shape__calcWidthFall}%
\hypertarget{ref_spike_shape__calcWidthFall}{}%
\begin{description}
\item[Summary:]Calculates the spike width and fall information of the 
		spike\_shape, s. 
%
\item[Usage:]~%
\begin{lyxcode}%
[base\_width, half\_width, half\_Vm, fall\_time, min\_idx, min\_val, 
  max\_ahp, ahp\_decay\_constant, dahp\_mag, dahp\_idx] = ...
      calcWidthFall(s, max\_idx, max\_val, init\_idx, init\_val)
%
\end{lyxcode}%
%
\item[Description:]%
max\_* can be the peak\_* from calcInitVm.
%%
\item[Parameters:]~
\begin{description}%
\item[\texttt{s}:]
 A spike\_shape object.
\item[\texttt{max\_idx}:]
 The index of the maximal point [dt].
\item[\texttt{max\_val}:]
 The value of the maximal point [dy].
\item[\texttt{init\_idx}:]
 The index of spike initiation point [dt].
\item[\texttt{init\_val}:]
 The value of spike initiation point [dy].
\end{description}%
%
\item[Returns:]~

	base\_width: Width of spike at base [dt]
	half\_width: Width of spike at half\_Vm [dt]
	half\_Vm: Half height of spike [dy]
	fall\_time: Time from peak to initialization level [dt].
	min\_idx: The index of the minimal point of the spike\_shape [dt].
	max\_ahp: Magnitude from initiation to minimum [dy].
	ahp\_decay\_constant: Approximation to refractory decay after maxAHP [dt].
	dahp\_mag: Magnitude of the double AHP peak
	dahp\_mag: Index of the double AHP peak
%
%
\item[See also:]%
\hyperlink{ref_spike_shape}{\texttt{spike\_shape}}%
\ (p.~\pageref{ref_spike_shape})%
\index[funcref]{@\fidxl{spike\_shape}}%
%
\item[Author:]%
Cengiz Gunay <cgunay@emory.edu>, 2004/08/02%
\end{description}
\methodline%
\subsubsection[Method \texttt{calcInitVmLtdMaxCurv}]{Method \texttt{spike\_shape/calcInitVmLtdMaxCurv}}%
\index[funcref]{spike_shape@\fidxlb{spike\_shape}!calcInitVmLtdMaxCurv@\fidxl{calcInitVmLtdMaxCurv}}%
\label{ref_spike_shape__calcInitVmLtdMaxCurv}%
\hypertarget{ref_spike_shape__calcInitVmLtdMaxCurv}{}%
\begin{description}
\item[Summary:]Calculates the action potential threshold using the maximum of the curvature equation only in the limited range given with two voltage slope thresholds.
%
\item[Usage:]~%
\begin{lyxcode}%
[init\_idx, a\_plot] = calcInitVmLtdMaxCurv(s, max\_idx, min\_idx, lo\_thr, hi\_thr, plotit)
%
\end{lyxcode}%
%
\item[Description:]%
Point of maximum curvature: Kp = V''[1 + (V')\textasciicircum{}2]\textasciicircum{}(-3/2)
 Taken from Sekerli, Del Negro, Lee and Butera. 
 IEEE Trans. Biomed. Eng., 51(9): 1665-71, 2004.
%%
\item[Parameters:]~
\begin{description}%
\item[\texttt{s}:]
 A spike\_shape object.
\item[\texttt{max\_idx}:]
 The index of the maximal point of the spike\_shape [dt].
\item[\texttt{min\_idx}:]
 The index of the minimal point of the spike\_shape [dt].
\item[\texttt{lo\_thr, hi\_thr}:]
 Lower and higher thresholds for time derivative of voltage.
\item[\texttt{plotit}:]
 If non-zero, plot a graph annotating the test results 

(optional).\end{description}%
%
\item[Returns:]~

	init\_idx: AP threshold index in the spike\_shape [dt].
	a\_plot: plot\_abstract, if requested.
%
%
\item[See also:]%
\hyperlink{ref_calcInitVm}{\texttt{calcInitVm}}%
\ (p.~\pageref{ref_calcInitVm})%
\index[funcref]{@\fidxl{calcInitVm}}%
%
\item[Author:]%
Cengiz Gunay <cgunay@emory.edu>, 2004/11/19%
\end{description}
\methodline%
\subsubsection[Method \texttt{plotPP}]{Method \texttt{spike\_shape/plotPP}}%
\index[funcref]{spike_shape@\fidxlb{spike\_shape}!plotPP@\fidxl{plotPP}}%
\label{ref_spike_shape__plotPP}%
\hypertarget{ref_spike_shape__plotPP}{}%
\begin{description}
\item[Summary:]Plots the dV/dt vs. V phase-plane representation of the spike shape.
%
\item[Usage:]~%
\begin{lyxcode}%
a\_plot = plotPP(s)
%
\end{lyxcode}%
%
%
\item[Parameters:]~
\begin{description}%
\item[\texttt{s}:]
 A spike\_shape object.
\end{description}%
%
\item[Returns:]~

	a\_plot: A plot\_abstract object that can be visualized.
%
%
\item[See also:]%
\hyperlink{ref_spike_shape}{\texttt{spike\_shape}}%
\ (p.~\pageref{ref_spike_shape})%
\index[funcref]{@\fidxl{spike\_shape}}%
, \hyperlink{ref_plot_abstract}{\texttt{plot\_abstract}}%
\ (p.~\pageref{ref_plot_abstract})%
\index[funcref]{@\fidxl{plot\_abstract}}%
%
\item[Author:]%
Cengiz Gunay <cgunay@emory.edu>, 2004/11/16%
\end{description}
\methodline%
\subsubsection[Method \texttt{plotResults}]{Method \texttt{spike\_shape/plotResults}}%
\index[funcref]{spike_shape@\fidxlb{spike\_shape}!plotResults@\fidxl{plotResults}}%
\label{ref_spike_shape__plotResults}%
\hypertarget{ref_spike_shape__plotResults}{}%
\begin{description}
\item[Summary:]Plots the spike shape annotated with result characteristics.
%
\item[Usage:]~%
\begin{lyxcode}%
a\_plot = plotResults(s, title\_str, props)
%
\end{lyxcode}%
%
%
\item[Parameters:]~
\begin{description}%
\item[\texttt{s}:]
 A spike\_shape object.
\end{description}%
%
\item[Returns:]~

	a\_plot: A plot\_abstract object that can be visualized.
	title\_str: (Optional) String to append to plot title.
	props: A structure with any optional properties, passed to trace/plotData.
%
%
\item[See also:]%
\hyperlink{ref_spike_shape}{\texttt{spike\_shape}}%
\ (p.~\pageref{ref_spike_shape})%
\index[funcref]{@\fidxl{spike\_shape}}%
, \hyperlink{ref_plot_abstract}{\texttt{plot\_abstract}}%
\ (p.~\pageref{ref_plot_abstract})%
\index[funcref]{@\fidxl{plot\_abstract}}%
%
\item[Author:]%
Cengiz Gunay <cgunay@emory.edu>, 2004/11/17%
\end{description}
\methodline%
\subsubsection[Method \texttt{calcInitVmMaxCurvPhasePlane}]{Method \texttt{spike\_shape/calcInitVmMaxCurvPhasePlane}}%
\index[funcref]{spike_shape@\fidxlb{spike\_shape}!calcInitVmMaxCurvPhasePlane@\fidxl{calcInitVmMaxCurvPhasePlane}}%
\label{ref_spike_shape__calcInitVmMaxCurvPhasePlane}%
\hypertarget{ref_spike_shape__calcInitVmMaxCurvPhasePlane}{}%
\begin{description}
\item[Summary:]Calculates the voltage at the maximum curvature in the phase plane as action potential threshold.
%
\item[Usage:]~%
\begin{lyxcode}%
[init\_idx, max\_d1o, a\_plot, fail\_cond] = 
	calcInitVmMaxCurvPhasePlane(s, max\_idx, min\_idx, plotit)
%
\end{lyxcode}%
%
\item[Description:]%
First take the phase-plane v'-v from the beginning to max(v'). Then regulate 
 intervals by interpolation. Point of maximum curvature: Kp = V''[1 + (V')\textasciicircum{}2]\textasciicircum{}(-3/2)
 Taken from Sekerli, Del Negro, Lee and Butera. 
 IEEE Trans. Biomed. Eng., 51(9): 1665-71, 2004.
%%
\item[Parameters:]~
\begin{description}%
\item[\texttt{s}:]
 A spike\_shape object.
\item[\texttt{max\_idx}:]
 The index of the maximal point of the spike\_shape [dt].
\item[\texttt{min\_idx}:]
 The index of the minimal point of the spike\_shape [dt].
\item[\texttt{plotit}:]
 If non-zero, plot a graph annotating the test results 

(optional).\end{description}%
%
\item[Returns:]~

	init\_idx: AP threshold index in the spike\_shape [dt].
	max\_d1o: Maximal value of first voltage derivative [dy].
	a\_plot: plot\_abstract, if requested.
	fail\_cond: True if this algorithm fails to be trustable.
%
%
\item[See also:]%
\hyperlink{ref_calcInitVm}{\texttt{calcInitVm}}%
\ (p.~\pageref{ref_calcInitVm})%
\index[funcref]{@\fidxl{calcInitVm}}%
%
\item[Author:]%
Cengiz Gunay <cgunay@emory.edu>, 2005/04/12%
\end{description}
\methodline%
\subsection{Class \texttt{spike\_shape\_profile}}%
\index[funcref]{spike_shape_profile@\fidxlb{spike\_shape\_profile}}%
\label{ref_spike_shape_profile}%
\hypertarget{ref_spike_shape_profile}{}%
\subsubsection[Constructor \texttt{spike\_shape\_profile}]{Constructor \texttt{spike\_shape\_profile/spike\_shape\_profile}}%
\index[funcref]{spike_shape_profile@\fidxlb{spike\_shape\_profile}!spike_shape_profile@\fidxl{spike\_shape\_profile}}%
\label{ref_spike_shape_profile__spike_shape_profile}%
\hypertarget{ref_spike_shape_profile__spike_shape_profile}{}%
\begin{description}
\item[Summary:]Holds the results profile from a spike\_shape object.
%
\item[Usage:]~%
\begin{lyxcode}%
a\_ss\_profile = spike\_shape\_profile(results, a\_spike\_shape, props)
%
\end{lyxcode}%
%
%
\item[Parameters:]~
\begin{description}%
\item[\texttt{results}:]
 A structure containing test results.
\item[\texttt{a\_spike\_shape}:]
 A spike\_shape object.
\item[\texttt{props}:]
 A structure with any optional properties.
\end{description}%
%
\item[Returns a structure object with the following fields:]~

	results\_profile: Contains results of tests.
	spike\_shape: The spike\_shape object from which results were obtained.
	props.
%
%
\item[See also:]%
\hyperlink{ref_results_profile}{\texttt{results\_profile}}%
\ (p.~\pageref{ref_results_profile})%
\index[funcref]{@\fidxl{results\_profile}}%
%
\item[Author:]%
Cengiz Gunay <cgunay@emory.edu>, 2005/08/17%
\end{description}
\methodline%
\subsubsection[Method \texttt{get}]{Method \texttt{spike\_shape\_profile/get}}%
\index[funcref]{spike_shape_profile@\fidxlb{spike\_shape\_profile}!get@\fidxl{get}}%
\label{ref_spike_shape_profile__get}%
\hypertarget{ref_spike_shape_profile__get}{}%
\begin{description}
\item[Summary:]Defines generic attribute retrieval for objects.
%
%
%
%
%
%
%
\item[Author:]%
Cengiz Gunay <cgunay@emory.edu>, 2004/09/14%
\end{description}
\methodline%
\subsubsection[Method \texttt{plot\_abstract}]{Method \texttt{spike\_shape\_profile/plot\_abstract}}%
\index[funcref]{spike_shape_profile@\fidxlb{spike\_shape\_profile}!plot_abstract@\fidxl{plot\_abstract}}%
\label{ref_spike_shape_profile__plot_abstract}%
\hypertarget{ref_spike_shape_profile__plot_abstract}{}%
\begin{description}
\item[Summary:]Plots the spike shape with measurements marked in red.
%
\item[Usage:]~%
\begin{lyxcode}%
a\_plot = plot\_abstract(s, props)
%
\end{lyxcode}%
%
%
\item[Parameters:]~
\begin{description}%
\item[\texttt{s}:]
 A spike\_shape object.
\item[\texttt{props}:]
 A structure with any optional properties.
\begin{description}%
\item[\texttt{absolute\_peak\_time}:]
 Shift the peak to this point on the plot.
\item[\texttt{no\_plot\_spike}:]
 Do not plot the spike shape first.
\end{description}%
\end{description}%
%
\item[Returns:]~

	a\_plot: A plot\_abstract object that can be visualized.
%
%
\item[See also:]%
\hyperlink{ref_spike_shape}{\texttt{spike\_shape}}%
\ (p.~\pageref{ref_spike_shape})%
\index[funcref]{@\fidxl{spike\_shape}}%
, \hyperlink{ref_plot_abstract}{\texttt{plot\_abstract}}%
\ (p.~\pageref{ref_plot_abstract})%
\index[funcref]{@\fidxl{plot\_abstract}}%
%
\item[Author:]%
Cengiz Gunay <cgunay@emory.edu>, 2005/08/17%
\end{description}
\methodline%
\subsection{Class \texttt{spikes}}%
\index[funcref]{spikes@\fidxlb{spikes}}%
\label{ref_spikes}%
\hypertarget{ref_spikes}{}%
\subsubsection[Constructor \texttt{spikes}]{Constructor \texttt{spikes/spikes}}%
\index[funcref]{spikes@\fidxlb{spikes}!spikes@\fidxl{spikes}}%
\label{ref_spikes__spikes}%
\hypertarget{ref_spikes__spikes}{}%
\begin{description}
\item[Summary:]Spike times from a trace.
%
\item[Usage:]~%
\begin{lyxcode}%
obj = spikes(times, num\_samples, dt, id)
%
\end{lyxcode}%
%
%
\item[Parameters:]~
\begin{description}%
\item[\texttt{times}:]
 The spike times [dt].
\item[\texttt{num\_samples}:]
 Number of samples in the original trace.
\item[\texttt{dt}:]
 Time resolution [s].
\item[\texttt{id}:]
 Identification string.
\end{description}%
%
\item[Returns a structure object with the following fields:]~

	times, num\_samples, dt, id.
%
%
\item[See also:]%
\hyperlink{ref_trace__spikes}{\texttt{trace/spikes}}%
\ (p.~\pageref{ref_trace__spikes})%
\index[funcref]{trace@\fidxlb{trace}!spikes@\fidxl{spikes}}%
, \hyperlink{ref_trace}{\texttt{trace}}%
\ (p.~\pageref{ref_trace})%
\index[funcref]{@\fidxl{trace}}%
, \hyperlink{ref_spike_shape}{\texttt{spike\_shape}}%
\ (p.~\pageref{ref_spike_shape})%
\index[funcref]{@\fidxl{spike\_shape}}%
, \hyperlink{ref_period}{\texttt{period}}%
\ (p.~\pageref{ref_period})%
\index[funcref]{@\fidxl{period}}%
%
\item[Author:]%
Cengiz Gunay <cgunay@emory.edu>, 2004/07/30%
\end{description}
\methodline%
\subsubsection[Method \texttt{display}]{Method \texttt{spikes/display}}%
\index[funcref]{spikes@\fidxlb{spikes}!display@\fidxl{display}}%
\label{ref_spikes__display}%
\hypertarget{ref_spikes__display}{}%
\begin{description}
%
%
%
%
%
%
%
\item[Author:]%
Cengiz Gunay <cgunay@emory.edu>, 2004/08/04%
\end{description}
\methodline%
\subsubsection[Method \texttt{SFA}]{Method \texttt{spikes/SFA}}%
\index[funcref]{spikes@\fidxlb{spikes}!SFA@\fidxl{SFA}}%
\label{ref_spikes__SFA}%
\hypertarget{ref_spikes__SFA}{}%
\begin{description}
\item[Summary:]Calculates the spike frequency accommodation (SFA) of the 
	inter-spike-intervals (ISI).
%
\item[Usage:]~%
\begin{lyxcode}%
sfa = SFA(s, a\_period)
%
\end{lyxcode}%
%
\item[Description:]%
SFA is the ration of the last ISI to the first ISI in the period.
%%
\item[Parameters:]~
\begin{description}%
\item[\texttt{s}:]
 A spikes object.
\item[\texttt{a\_period}:]
 The period where spikes were found (optional)
\end{description}%
%
\item[Returns:]~

	sfa: Spike frequency accommodation.
%
%
\item[See also:]%
\hyperlink{ref_spikes}{\texttt{spikes}}%
\ (p.~\pageref{ref_spikes})%
\index[funcref]{@\fidxl{spikes}}%
, \hyperlink{ref_period}{\texttt{period}}%
\ (p.~\pageref{ref_period})%
\index[funcref]{@\fidxl{period}}%
%
\item[Author:]%
Cengiz Gunay <cgunay@emory.edu>, 2004/09/13%
\end{description}
\methodline%
\subsubsection[Method \texttt{get}]{Method \texttt{spikes/get}}%
\index[funcref]{spikes@\fidxlb{spikes}!get@\fidxl{get}}%
\label{ref_spikes__get}%
\hypertarget{ref_spikes__get}{}%
\begin{description}
\item[Summary:]Defines generic attribute retrieval for objects.
%
%
%
%
%
%
%
\item[Author:]%
Cengiz Gunay <cgunay@emory.edu>, 2004/09/14%
\end{description}
\methodline%
\subsubsection[Method \texttt{periodWhole}]{Method \texttt{spikes/periodWhole}}%
\index[funcref]{spikes@\fidxlb{spikes}!periodWhole@\fidxl{periodWhole}}%
\label{ref_spikes__periodWhole}%
\hypertarget{ref_spikes__periodWhole}{}%
\begin{description}
\item[Summary:]Returns the boundaries of the whole period of spikes, s. 
%
\item[Usage:]~%
\begin{lyxcode}%
whole\_period = periodWhole(s)
%
\end{lyxcode}%
%
%
\item[Parameters:]~
\begin{description}%
\item[\texttt{s}:]
 A spikes object.
\end{description}%
%
%
%
\item[See also:]%
\hyperlink{ref_period}{\texttt{period}}%
\ (p.~\pageref{ref_period})%
\index[funcref]{@\fidxl{period}}%
, \hyperlink{ref_spikes}{\texttt{spikes}}%
\ (p.~\pageref{ref_spikes})%
\index[funcref]{@\fidxl{spikes}}%
%
\item[Author:]%
Cengiz Gunay <cgunay@emory.edu>, 2004/07/30%
\end{description}
\methodline%
\subsubsection[Method \texttt{set}]{Method \texttt{spikes/set}}%
\index[funcref]{spikes@\fidxlb{spikes}!set@\fidxl{set}}%
\label{ref_spikes__set}%
\hypertarget{ref_spikes__set}{}%
\begin{description}
\item[Summary:]Generic method for setting object attributes.
%
%
%
%
%
%
%
\item[Author:]%
Cengiz Gunay <cgunay@emory.edu>, 2004/10/08%
\end{description}
\methodline%
\subsubsection[Method \texttt{plotData}]{Method \texttt{spikes/plotData}}%
\index[funcref]{spikes@\fidxlb{spikes}!plotData@\fidxl{plotData}}%
\label{ref_spikes__plotData}%
\hypertarget{ref_spikes__plotData}{}%
\begin{description}
\item[Summary:]Plots a spikes object.
%
\item[Usage:]~%
\begin{lyxcode}%
a\_plot = plotData(s, title\_str)
%
\end{lyxcode}%
%
\item[Description:]%
If s is a vector of spikes objects, returns a vector of plot objects.
%%
\item[Parameters:]~
\begin{description}%
\item[\texttt{s}:]
 A spikes object.
\end{description}%
%
\item[Returns:]~

	a\_plot: A plot\_abstract object that can be visualized.
	title\_str: (Optional) String to append to plot title.
%
%
\item[See also:]%
\hyperlink{ref_trace}{\texttt{trace}}%
\ (p.~\pageref{ref_trace})%
\index[funcref]{@\fidxl{trace}}%
, \hyperlink{ref_plot_abstract}{\texttt{plot\_abstract}}%
\ (p.~\pageref{ref_plot_abstract})%
\index[funcref]{@\fidxl{plot\_abstract}}%
%
\item[Author:]%
Cengiz Gunay <cgunay@emory.edu>, 2005/10/21%
\end{description}
\methodline%
\subsubsection[Method \texttt{plotISIs}]{Method \texttt{spikes/plotISIs}}%
\index[funcref]{spikes@\fidxlb{spikes}!plotISIs@\fidxl{plotISIs}}%
\label{ref_spikes__plotISIs}%
\hypertarget{ref_spikes__plotISIs}{}%
\begin{description}
\item[Summary:]Plots a spikes object.
%
\item[Usage:]~%
\begin{lyxcode}%
a\_plot = plotISIs(s, title\_str)
%
\end{lyxcode}%
%
\item[Description:]%
If s is a vector of spikes objects, returns a vector of plot objects.
%%
\item[Parameters:]~
\begin{description}%
\item[\texttt{s}:]
 A spikes object.
\end{description}%
%
\item[Returns:]~

	a\_plot: A plot\_abstract object that can be visualized.
	title\_str: (Optional) String to append to plot title.
%
%
\item[See also:]%
\hyperlink{ref_trace}{\texttt{trace}}%
\ (p.~\pageref{ref_trace})%
\index[funcref]{@\fidxl{trace}}%
, \hyperlink{ref_plot_abstract}{\texttt{plot\_abstract}}%
\ (p.~\pageref{ref_plot_abstract})%
\index[funcref]{@\fidxl{plot\_abstract}}%
%
\item[Author:]%
Cengiz Gunay <cgunay@emory.edu>, 2005/10/21%
\end{description}
\methodline%
\subsubsection[Method \texttt{spikeRateISI}]{Method \texttt{spikes/spikeRateISI}}%
\index[funcref]{spikes@\fidxlb{spikes}!spikeRateISI@\fidxl{spikeRateISI}}%
\label{ref_spikes__spikeRateISI}%
\hypertarget{ref_spikes__spikeRateISI}{}%
\begin{description}
\item[Summary:]Calculates the firing rate of the spikes found in the given 
		period with an averaged inter-spike-interval approach.
%
\item[Usage:]~%
\begin{lyxcode}%
freq = spikeRateISI(s, trace\_index, times, period)
%
\end{lyxcode}%
%
%
\item[Parameters:]~
\begin{description}%
\item[\texttt{s}:]
 A spikes object.
\item[\texttt{period}:]
 The period where spikes were found (optional)
\end{description}%
%
\item[Returns:]~

	freq: Firing rate [Hz]
%
%
\item[See also:]%
\hyperlink{ref_trace}{\texttt{trace}}%
\ (p.~\pageref{ref_trace})%
\index[funcref]{@\fidxl{trace}}%
, \hyperlink{ref_spikes}{\texttt{spikes}}%
\ (p.~\pageref{ref_spikes})%
\index[funcref]{@\fidxl{spikes}}%
, \hyperlink{ref_period}{\texttt{period}}%
\ (p.~\pageref{ref_period})%
\index[funcref]{@\fidxl{period}}%
%
\item[Author:]%
Cengiz Gunay <cgunay@emory.edu>, 2004/03/09%
\end{description}
\methodline%
\subsubsection[Method \texttt{plotFreqVsTime}]{Method \texttt{spikes/plotFreqVsTime}}%
\index[funcref]{spikes@\fidxlb{spikes}!plotFreqVsTime@\fidxl{plotFreqVsTime}}%
\label{ref_spikes__plotFreqVsTime}%
\hypertarget{ref_spikes__plotFreqVsTime}{}%
\begin{description}
\item[Summary:]Plots a frequency-time graph from the spikes object.
%
\item[Usage:]~%
\begin{lyxcode}%
a\_plot = plotFreqVsTime(s, title\_str, props)
%
\end{lyxcode}%
%
\item[Description:]%
If s is a vector of spikes objects, returns a vector of plot objects.
%%
\item[Parameters:]~
\begin{description}%
\item[\texttt{s}:]
 A spikes object.
\item[\texttt{title\_str}:]
 (Optional) String to append to plot title.
\item[\texttt{props}:]
 A structure with any optional properties.
\begin{description}%
\item[\texttt{type}:]
 If 'simple' plots 1/is for each spike time, 

'manhattan' uses flat lines of 1/isi height between spike times (default).
(others passed to plot\_abstract)\end{description}%
\end{description}%
%
\item[Returns:]~

	a\_plot: A plot\_abstract object that can be visualized.
	title\_str: (Optional) String to append to plot title.
%
%
\item[See also:]%
\hyperlink{ref_trace}{\texttt{trace}}%
\ (p.~\pageref{ref_trace})%
\index[funcref]{@\fidxl{trace}}%
, \hyperlink{ref_plot_abstract}{\texttt{plot\_abstract}}%
\ (p.~\pageref{ref_plot_abstract})%
\index[funcref]{@\fidxl{plot\_abstract}}%
%
\item[Author:]%
Cengiz Gunay <cgunay@emory.edu>, 2006/05/05%
\end{description}
\methodline%
\subsubsection[Method \texttt{addSpikes}]{Method \texttt{spikes/addSpikes}}%
\index[funcref]{spikes@\fidxlb{spikes}!addSpikes@\fidxl{addSpikes}}%
\label{ref_spikes__addSpikes}%
\hypertarget{ref_spikes__addSpikes}{}%
\begin{description}
%
\item[Usage:]~%
\begin{lyxcode}%
s = addSpike(s, times)
%
\end{lyxcode}%
%
%
\item[Parameters:]~
\begin{description}%
\item[\texttt{s}:]
 A spikes object.
\item[\texttt{times}:]
 Times of spikes to add
\end{description}%
%
\item[Returns:]~

	s: The updated object.
%
%
\item[See also:]%
\hyperlink{ref_spikes}{\texttt{spikes}}%
\ (p.~\pageref{ref_spikes})%
\index[funcref]{@\fidxl{spikes}}%
%
\item[Author:]%
Cengiz Gunay <cgunay@emory.edu>, 2005/08/16%
\end{description}
\methodline%
\subsubsection[Method \texttt{subsref}]{Method \texttt{spikes/subsref}}%
\index[funcref]{spikes@\fidxlb{spikes}!subsref@\fidxl{subsref}}%
\label{ref_spikes__subsref}%
\hypertarget{ref_spikes__subsref}{}%
\begin{description}
\item[Summary:]Defines generic indexing for objects.
%
%
%
%
%
%
%
\item[Author:]%
Cengiz Gunay <cgunay@emory.edu>, 2004/08/04%
\end{description}
\methodline%
\subsubsection[Method \texttt{spikeAmpSlope}]{Method \texttt{spikes/spikeAmpSlope}}%
\index[funcref]{spikes@\fidxlb{spikes}!spikeAmpSlope@\fidxl{spikeAmpSlope}}%
\label{ref_spikes__spikeAmpSlope}%
\hypertarget{ref_spikes__spikeAmpSlope}{}%
\begin{description}
\item[Summary:]Calculates the time constant and steady-state value
		      of the spike amplitude for slow inactivating decays.
%
\item[Usage:]~%
\begin{lyxcode}%
[a\_tau, da\_inf] = spikeAmpSlope(a\_spikes, a\_trace, a\_period)
%
\end{lyxcode}%
%
%
\item[Parameters:]~
\begin{description}%
\item[\texttt{a\_spikes}:]
 A spikes object.
\item[\texttt{a\_trace}:]
 A trace object.
\item[\texttt{a\_period}:]
 The desired period (optional)
\end{description}%
%
\item[Returns:]~

	a\_tau: Approximate amplitude decay constant.
	da\_inf: Delta change in final spike peak value from initial.
%
%
\item[See also:]%
\hyperlink{ref_period}{\texttt{period}}%
\ (p.~\pageref{ref_period})%
\index[funcref]{@\fidxl{period}}%
, \hyperlink{ref_spikes}{\texttt{spikes}}%
\ (p.~\pageref{ref_spikes})%
\index[funcref]{@\fidxl{spikes}}%
, \hyperlink{ref_trace}{\texttt{trace}}%
\ (p.~\pageref{ref_trace})%
\index[funcref]{@\fidxl{trace}}%
%
\item[Author:]%
Cengiz Gunay <cgunay@emory.edu>, 2004/09/15%
\end{description}
\methodline%
\subsubsection[Method \texttt{intoPeriod}]{Method \texttt{spikes/intoPeriod}}%
\index[funcref]{spikes@\fidxlb{spikes}!intoPeriod@\fidxl{intoPeriod}}%
\label{ref_spikes__intoPeriod}%
\hypertarget{ref_spikes__intoPeriod}{}%
\begin{description}
\item[Summary:]Shifts the spikes times to be within the given period.
%
\item[Usage:]~%
\begin{lyxcode}%
obj = intoPeriod(s, a\_period)
%
\end{lyxcode}%
%
\item[Description:]%
Assuming this spikes object's length fits into the given period, it shifts
 all times to start from the beginning of the given period. This may be used
 to reconstruct the original spikes object from subperiods that were cut out
 previously, using the withinPeriod method.
%%
\item[Parameters:]~
\begin{description}%
\item[\texttt{s}:]
 A spikes object.
\item[\texttt{a\_period}:]
 The desired period 
\end{description}%
%
\item[Returns:]~

	obj: A spikes object
%
%
\item[See also:]%
\hyperlink{ref_spikes}{\texttt{spikes}}%
\ (p.~\pageref{ref_spikes})%
\index[funcref]{@\fidxl{spikes}}%
, \hyperlink{ref_period}{\texttt{period}}%
\ (p.~\pageref{ref_period})%
\index[funcref]{@\fidxl{period}}%
%
\item[Author:]%
Cengiz Gunay <cgunay@emory.edu>, 2004/07/31%
\end{description}
\methodline%
\subsubsection[Method \texttt{plot}]{Method \texttt{spikes/plot}}%
\index[funcref]{spikes@\fidxlb{spikes}!plot@\fidxl{plot}}%
\label{ref_spikes__plot}%
\hypertarget{ref_spikes__plot}{}%
\begin{description}
\item[Summary:]Plots spikes.
%
\item[Usage:]~%
\begin{lyxcode}%
h = plot(t)
%
\end{lyxcode}%
%
%
\item[Parameters:]~
\begin{description}%
\item[\texttt{t}:]
 A spikes object.
\item[\texttt{title\_str}:]
 (Optional) String to append to plot title.
\end{description}%
%
\item[Returns:]~

	h: Handle to figure object.
%
%
\item[See also:]%
\hyperlink{ref_spikes}{\texttt{spikes}}%
\ (p.~\pageref{ref_spikes})%
\index[funcref]{@\fidxl{spikes}}%
, \hyperlink{ref_plot_abstract}{\texttt{plot\_abstract}}%
\ (p.~\pageref{ref_plot_abstract})%
\index[funcref]{@\fidxl{plot\_abstract}}%
%
\item[Author:]%
Cengiz Gunay <cgunay@emory.edu>, 2004/08/04%
\end{description}
\methodline%
\subsubsection[Method \texttt{withinPeriod}]{Method \texttt{spikes/withinPeriod}}%
\index[funcref]{spikes@\fidxlb{spikes}!withinPeriod@\fidxl{withinPeriod}}%
\label{ref_spikes__withinPeriod}%
\hypertarget{ref_spikes__withinPeriod}{}%
\begin{description}
\item[Summary:]Returns a spikes object valid only within the given period, subtracts the offset.
%
\item[Usage:]~%
\begin{lyxcode}%
obj = withinPeriod(s, a\_period)
%
\end{lyxcode}%
%
%
\item[Parameters:]~
\begin{description}%
\item[\texttt{s}:]
 A spikes object.
\item[\texttt{a\_period}:]
 The desired period 
\end{description}%
%
\item[Returns:]~

	obj: A spikes object
%
%
\item[See also:]%
\hyperlink{ref_spikes}{\texttt{spikes}}%
\ (p.~\pageref{ref_spikes})%
\index[funcref]{@\fidxl{spikes}}%
, \hyperlink{ref_period}{\texttt{period}}%
\ (p.~\pageref{ref_period})%
\index[funcref]{@\fidxl{period}}%
%
\item[Author:]%
Cengiz Gunay <cgunay@emory.edu>, 2004/07/31%
\end{description}
\methodline%
\subsubsection[Method \texttt{ISICV}]{Method \texttt{spikes/ISICV}}%
\index[funcref]{spikes@\fidxlb{spikes}!ISICV@\fidxl{ISICV}}%
\label{ref_spikes__ISICV}%
\hypertarget{ref_spikes__ISICV}{}%
\begin{description}
\item[Summary:]Calculates the coefficient of variation (CV) of the 
	inter-spike-intervals (ISI).
%
\item[Usage:]~%
\begin{lyxcode}%
cv = ISICV(s, a\_period)
%
\end{lyxcode}%
%
%
\item[Parameters:]~
\begin{description}%
\item[\texttt{s}:]
 A spikes object.
\item[\texttt{a\_period}:]
 The period where spikes were found (optional)
\end{description}%
%
\item[Returns:]~

	cv: Coefficient of variation.
%
%
\item[See also:]%
\hyperlink{ref_spikes}{\texttt{spikes}}%
\ (p.~\pageref{ref_spikes})%
\index[funcref]{@\fidxl{spikes}}%
, \hyperlink{ref_period}{\texttt{period}}%
\ (p.~\pageref{ref_period})%
\index[funcref]{@\fidxl{period}}%
%
\item[Author:]%
Cengiz Gunay <cgunay@emory.edu>, 2004/09/13%
\end{description}
\methodline%
\subsubsection[Method \texttt{getResults}]{Method \texttt{spikes/getResults}}%
\index[funcref]{spikes@\fidxlb{spikes}!getResults@\fidxl{getResults}}%
\label{ref_spikes__getResults}%
\hypertarget{ref_spikes__getResults}{}%
\begin{description}
\item[Summary:]Runs all tests defined by this class and return them in a 
		structure.
%
\item[Usage:]~%
\begin{lyxcode}%
results = getResults(s)
%
\end{lyxcode}%
%
%
\item[Parameters:]~
\begin{description}%
\item[\texttt{s}:]
 A spikes object.
\end{description}%
%
\item[Returns:]~

	results: A structure associating test names to values 
		in ms and mV (or mA).
%
%
\item[See also:]%
\hyperlink{ref_spikes}{\texttt{spikes}}%
\ (p.~\pageref{ref_spikes})%
\index[funcref]{@\fidxl{spikes}}%
%
\item[Author:]%
Cengiz Gunay <cgunay@emory.edu>, 2004/09/13%
\end{description}
\methodline%
\subsubsection[Method \texttt{vertcat}]{Method \texttt{spikes/vertcat}}%
\index[funcref]{spikes@\fidxlb{spikes}!vertcat@\fidxl{vertcat}}%
\label{ref_spikes__vertcat}%
\hypertarget{ref_spikes__vertcat}{}%
\begin{description}
\item[Summary:]Vertical concatanation [a\_spikes;with\_spikes;...] operator.
%
\item[Usage:]~%
\begin{lyxcode}%
a\_spikes = vertcat(a\_spikes, with\_spikes, ...)
%
\end{lyxcode}%
%
\item[Description:]%
Concatanates spike times of with\_spikes with that of a\_spikes. Overrides the built-in
 vertcat function that is called when [a\_spikes;with\_spikes] is executed.
%%
\item[Parameters:]~

a\_spikes, with\_spikes, ...: Spikes objects.%
\item[Returns:]~

	a\_spikes: A tests\_spikes that contains times of all given spikes objects.
%
%
\item[See also:]%
\hyperlink{ref_vertcat}{\texttt{vertcat}}%
\ (p.~\pageref{ref_vertcat})%
\index[funcref]{@\fidxl{vertcat}}%
, \hyperlink{ref_spikes}{\texttt{spikes}}%
\ (p.~\pageref{ref_spikes})%
\index[funcref]{@\fidxl{spikes}}%
%
\item[Author:]%
Cengiz Gunay <cgunay@emory.edu>, 2005/08/16%
\end{description}
\methodline%
\subsubsection[Method \texttt{spikeRate}]{Method \texttt{spikes/spikeRate}}%
\index[funcref]{spikes@\fidxlb{spikes}!spikeRate@\fidxl{spikeRate}}%
\label{ref_spikes__spikeRate}%
\hypertarget{ref_spikes__spikeRate}{}%
\begin{description}
\item[Summary:]Calculates the average firing rate [Hz] of the given spike train.
%
\item[Usage:]~%
\begin{lyxcode}%
freq = spikeRate(s, a\_period)
%
\end{lyxcode}%
%
%
\item[Parameters:]~
\begin{description}%
\item[\texttt{s}:]
 A spikes object.
\item[\texttt{a\_period}:]
 The period where spikes were found (optional)
\end{description}%
%
\item[Returns:]~

	freq: Firing rate [Hz]
%
%
\item[See also:]%
\hyperlink{ref_spikes}{\texttt{spikes}}%
\ (p.~\pageref{ref_spikes})%
\index[funcref]{@\fidxl{spikes}}%
, \hyperlink{ref_period}{\texttt{period}}%
\ (p.~\pageref{ref_period})%
\index[funcref]{@\fidxl{period}}%
%
\item[Author:]%
Cengiz Gunay <cgunay@emory.edu>, 2004/03/09%
\end{description}
\methodline%
\subsubsection[Method \texttt{withinPeriodWOffset}]{Method \texttt{spikes/withinPeriodWOffset}}%
\index[funcref]{spikes@\fidxlb{spikes}!withinPeriodWOffset@\fidxl{withinPeriodWOffset}}%
\label{ref_spikes__withinPeriodWOffset}%
\hypertarget{ref_spikes__withinPeriodWOffset}{}%
\begin{description}
\item[Summary:]Returns a spikes object valid only within the given period, keeps the offset.
%
\item[Usage:]~%
\begin{lyxcode}%
obj = withinPeriodWOffset(s, a\_period)
%
\end{lyxcode}%
%
%
\item[Parameters:]~
\begin{description}%
\item[\texttt{s}:]
 A spikes object.
\item[\texttt{a\_period}:]
 The desired period 
\end{description}%
%
\item[Returns:]~

	obj: A spikes object
%
%
\item[See also:]%
\hyperlink{ref_spikes}{\texttt{spikes}}%
\ (p.~\pageref{ref_spikes})%
\index[funcref]{@\fidxl{spikes}}%
, \hyperlink{ref_period}{\texttt{period}}%
\ (p.~\pageref{ref_period})%
\index[funcref]{@\fidxl{period}}%
%
\item[Author:]%
Cengiz Gunay <cgunay@emory.edu>, 2005/05/09%
\end{description}
\methodline%
\subsubsection[Method \texttt{getISIs}]{Method \texttt{spikes/getISIs}}%
\index[funcref]{spikes@\fidxlb{spikes}!getISIs@\fidxl{getISIs}}%
\label{ref_spikes__getISIs}%
\hypertarget{ref_spikes__getISIs}{}%
\begin{description}
\item[Summary:]Calculates the firing rate of the spikes found in the given 
		period with an averaged inter-spike-interval approach.
%
\item[Usage:]~%
\begin{lyxcode}%
isi = getISIs(s, period)
%
\end{lyxcode}%
%
%
\item[Parameters:]~
\begin{description}%
\item[\texttt{s}:]
 A spikes object.
\item[\texttt{period}:]
 The period where spikes were found (optional)
\end{description}%
%
\item[Returns:]~

	isi: Inter-spike-interval vector [dt]
%
%
\item[See also:]%
\hyperlink{ref_trace}{\texttt{trace}}%
\ (p.~\pageref{ref_trace})%
\index[funcref]{@\fidxl{trace}}%
, \hyperlink{ref_spikes}{\texttt{spikes}}%
\ (p.~\pageref{ref_spikes})%
\index[funcref]{@\fidxl{spikes}}%
, \hyperlink{ref_period}{\texttt{period}}%
\ (p.~\pageref{ref_period})%
\index[funcref]{@\fidxl{period}}%
%
\item[Author:]%
Cengiz Gunay <cgunay@emory.edu>, 2004/03/09%
\end{description}
\methodline%
\subsection{Class \texttt{spikes\_db}}%
\index[funcref]{spikes_db@\fidxlb{spikes\_db}}%
\label{ref_spikes_db}%
\hypertarget{ref_spikes_db}{}%
\subsubsection[Constructor \texttt{spikes\_db}]{Constructor \texttt{spikes\_db/spikes\_db}}%
\index[funcref]{spikes_db@\fidxlb{spikes\_db}!spikes_db@\fidxl{spikes\_db}}%
\label{ref_spikes_db__spikes_db}%
\hypertarget{ref_spikes_db__spikes_db}{}%
\begin{description}
\item[Summary:]A database of spike shape results obtained from a period in a trace.
%
\item[Usage:]~%
\begin{lyxcode}%
a\_spikes\_db = spikes\_db(data, col\_names, a\_trace, a\_period, id, props)
%
\end{lyxcode}%
%
\item[Description:]%
This is a subclass of tests\_db. Use trace/analyzeSpikesInPeriod to 
 get an instance of this class.
%%
\item[Parameters:]~
\begin{description}%
\item[\texttt{data}:]
 Database contents.
\item[\texttt{col\_names}:]
 The column names.
\item[\texttt{a\_trace}:]
 The trace where the spikes were found.
\item[\texttt{a\_period}:]
 The period inside a\_trace where spikes were found.
\item[\texttt{id}:]
 An identifying string.
\item[\texttt{props}:]
 A structure with any optional properties.
\end{description}%
%
\item[Returns a structure object with the following fields:]~

	tests\_db, trace, period, props.
%
%
\item[See also:]%
\hyperlink{ref_tests_db}{\texttt{tests\_db}}%
\ (p.~\pageref{ref_tests_db})%
\index[funcref]{@\fidxl{tests\_db}}%
, \hyperlink{ref_trace}{\texttt{trace}}%
\ (p.~\pageref{ref_trace})%
\index[funcref]{@\fidxl{trace}}%
, \hyperlink{ref_period}{\texttt{period}}%
\ (p.~\pageref{ref_period})%
\index[funcref]{@\fidxl{period}}%
, \hyperlink{ref_trace__analyzeSpikesInPeriod}{\texttt{trace/analyzeSpikesInPeriod}}%
\ (p.~\pageref{ref_trace__analyzeSpikesInPeriod})%
\index[funcref]{trace@\fidxlb{trace}!analyzeSpikesInPeriod@\fidxl{analyzeSpikesInPeriod}}%
%
\item[Author:]%
Cengiz Gunay <cgunay@emory.edu>, 2005/08/17%
\end{description}
\methodline%
\subsubsection[Method \texttt{plot\_abstract}]{Method \texttt{spikes\_db/plot\_abstract}}%
\index[funcref]{spikes_db@\fidxlb{spikes\_db}!plot_abstract@\fidxl{plot\_abstract}}%
\label{ref_spikes_db__plot_abstract}%
\hypertarget{ref_spikes_db__plot_abstract}{}%
\begin{description}
\item[Summary:]Visualizes the spikes\_db by marking spike shapes measurements on the trace plot.
%
\item[Usage:]~%
\begin{lyxcode}%
a\_pm = plot\_abstract(a\_db, title\_str)
%
\end{lyxcode}%
%
%
\item[Parameters:]~
\begin{description}%
\item[\texttt{a\_db}:]
 A spikes\_db object.
\item[\texttt{title\_str}:]
 (Optional) A string to be concatanated to the title.
\end{description}%
%
\item[Returns:]~

	a\_pm: A trace plot.
%
%
\item[See also:]%
\hyperlink{ref_plot_abstract__plot_abstract}{\texttt{plot\_abstract/plot\_abstract}}%
\ (p.~\pageref{ref_plot_abstract__plot_abstract})%
\index[funcref]{plot_abstract@\fidxlb{plot\_abstract}!plot_abstract@\fidxl{plot\_abstract}}%
, \hyperlink{ref_tests_db__plot_abstract}{\texttt{tests\_db/plot\_abstract}}%
\ (p.~\pageref{ref_tests_db__plot_abstract})%
\index[funcref]{tests_db@\fidxlb{tests\_db}!plot_abstract@\fidxl{plot\_abstract}}%
, \hyperlink{ref_plotFigure}{\texttt{plotFigure}}%
\ (p.~\pageref{ref_plotFigure})%
\index[funcref]{@\fidxl{plotFigure}}%
%
\item[Author:]%
Cengiz Gunay <cgunay@emory.edu>, 2005/08/17%
\end{description}
\methodline%
\subsection{Class \texttt{stats\_db}}%
\index[funcref]{stats_db@\fidxlb{stats\_db}}%
\label{ref_stats_db}%
\hypertarget{ref_stats_db}{}%
\subsubsection[Constructor \texttt{stats\_db}]{Constructor \texttt{stats\_db/stats\_db}}%
\index[funcref]{stats_db@\fidxlb{stats\_db}!stats_db@\fidxl{stats\_db}}%
\label{ref_stats_db__stats_db}%
\hypertarget{ref_stats_db__stats_db}{}%
\begin{description}
\item[Summary:]A database of rows corresponding to statistical distribution
		properties of tests. Multiple pages can be used to
		indicate another dimension.
%
\item[Usage:]~%
\begin{lyxcode}%
a\_stats\_db = stats\_db(test\_results, col\_names, row\_names, page\_names, 
			id, props)
%
\end{lyxcode}%
%
\item[Description:]%
This is a subclass of tests\_3D\_db. Allows generating a plot, etc.
%%
\item[Parameters:]~
\begin{description}%
\item[\texttt{test\_results}:]
 The 3-d array of rows, columns, and pages.
\item[\texttt{col\_names}:]
 Test names in this db.
\item[\texttt{row\_names}:]
 Statistical test names for each row.
\item[\texttt{page\_names}:]
 Meaning of each separate page of data 

(e.g., a different invariant parameter).\item[\texttt{id}:]
 An identifying string.
\item[\texttt{props}:]
 A structure with any optional properties.
\begin{description}%
\item[\texttt{axis\_limits}:]
 Limits in the form of [xmin xmax ymin ymax]

for errorbar axes.\item[\texttt{yTicksPos}:]
 'left' means only put y-axis ticks to leftmost plot.
\item[\texttt{xTicksPos}:]
 'bottom' means only put x-axis ticks to lowest plot.
\end{description}%
\end{description}%
%
\item[Returns a structure object with the following fields:]~

	tests\_3D\_db.
%
%
\item[See also:]%
\hyperlink{ref_tests_3D_db}{\texttt{tests\_3D\_db}}%
\ (p.~\pageref{ref_tests_3D_db})%
\index[funcref]{@\fidxl{tests\_3D\_db}}%
, \hyperlink{ref_plot_abstract}{\texttt{plot\_abstract}}%
\ (p.~\pageref{ref_plot_abstract})%
\index[funcref]{@\fidxl{plot\_abstract}}%
%
\item[Author:]%
Cengiz Gunay <cgunay@emory.edu>, 2004/10/07%
\end{description}
\methodline%
\subsubsection[Method \texttt{onlyRowsTests}]{Method \texttt{stats\_db/onlyRowsTests}}%
\index[funcref]{stats_db@\fidxlb{stats\_db}!onlyRowsTests@\fidxl{onlyRowsTests}}%
\label{ref_stats_db__onlyRowsTests}%
\hypertarget{ref_stats_db__onlyRowsTests}{}%
\begin{description}
\item[Summary:]Returns a tests\_db that only contains the desired 
		tests and rows (and pages).
%
\item[Usage:]~%
\begin{lyxcode}%
obj = onlyRowsTests(obj, rows, tests, pages)
%
\end{lyxcode}%
%
\item[Description:]%
Selects the given dimensions and returns in a new tests\_db object.
%%
\item[Parameters:]~
\begin{description}%
\item[\texttt{obj}:]
 A tests\_db object.
\item[\texttt{rows}:]
 A logical or index vector of rows. If ':', all rows.
\item[\texttt{tests}:]
 Cell array of test names or column indices. If ':', all tests.
\item[\texttt{pages}:]
 (Optional) A logical or index vector of pages. ':' for all pages.
\end{description}%
%
\item[Returns:]~

	obj: The new tests\_db object.
%
%
\item[See also:]%
\hyperlink{ref_subsref}{\texttt{subsref}}%
\ (p.~\pageref{ref_subsref})%
\index[funcref]{@\fidxl{subsref}}%
, \hyperlink{ref_tests_db}{\texttt{tests\_db}}%
\ (p.~\pageref{ref_tests_db})%
\index[funcref]{@\fidxl{tests\_db}}%
%
\item[Author:]%
Cengiz Gunay <cgunay@emory.edu>, 2004/09/17%
\end{description}
\methodline%
\subsubsection[Method \texttt{get}]{Method \texttt{stats\_db/get}}%
\index[funcref]{stats_db@\fidxlb{stats\_db}!get@\fidxl{get}}%
\label{ref_stats_db__get}%
\hypertarget{ref_stats_db__get}{}%
\begin{description}
\item[Summary:]Defines generic attribute retrieval for objects.
%
%
%
%
%
%
%
\item[Author:]%
Cengiz Gunay <cgunay@emory.edu>, 2004/09/14%
\end{description}
\methodline%
\subsubsection[Method \texttt{set}]{Method \texttt{stats\_db/set}}%
\index[funcref]{stats_db@\fidxlb{stats\_db}!set@\fidxl{set}}%
\label{ref_stats_db__set}%
\hypertarget{ref_stats_db__set}{}%
\begin{description}
\item[Summary:]Generic method for setting object attributes.
%
%
%
%
%
%
%
\item[Author:]%
Cengiz Gunay <cgunay@emory.edu>, 2004/10/08%
\end{description}
\methodline%
\subsubsection[Method \texttt{plotVar}]{Method \texttt{stats\_db/plotVar}}%
\index[funcref]{stats_db@\fidxlb{stats\_db}!plotVar@\fidxl{plotVar}}%
\label{ref_stats_db__plotVar}%
\hypertarget{ref_stats_db__plotVar}{}%
\begin{description}
\item[Summary:]Generates a plot of the variation between two tests.
%
\item[Usage:]~%
\begin{lyxcode}%
a\_plot = plotVar(a\_stats\_db, test1, test2, props)
%
\end{lyxcode}%
%
\item[Description:]%
Creates a plot description where the mean values are used for solid lines
 and the std values of test2 is indicated with errorbars. It is assumed that 
 each page of the stats\_db contains a value to be matched.
%%
\item[Parameters:]~
\begin{description}%
\item[\texttt{a\_stats\_db}:]
 A stats\_db object.
\item[\texttt{test1}:]
 Test column for the x-axis, only mean values are used.
\item[\texttt{test2}:]
 Test column for the y-axis, std values are indicated with errorbars.
\item[\texttt{props}:]
 Optional properties.
\begin{description}%
\item[\texttt{plotType}:]
 1, only errorbars (default); 2, errorbars extending from bars.

(rest passed to plot\_abstract)\end{description}%
\end{description}%
%
\item[Returns:]~

	a\_plot: A plot\_abstract object or one of its subclasses.
%
%
\item[See also:]%
\hyperlink{ref_plotVar}{\texttt{plotVar}}%
\ (p.~\pageref{ref_plotVar})%
\index[funcref]{@\fidxl{plotVar}}%
, \hyperlink{ref_plot_simple}{\texttt{plot\_simple}}%
\ (p.~\pageref{ref_plot_simple})%
\index[funcref]{@\fidxl{plot\_simple}}%
%
\item[Author:]%
Cengiz Gunay <cgunay@emory.edu>, 2004/10/13%
\end{description}
\methodline%
\subsubsection[Method \texttt{plotColorVar}]{Method \texttt{stats\_db/plotColorVar}}%
\index[funcref]{stats_db@\fidxlb{stats\_db}!plotColorVar@\fidxl{plotColorVar}}%
\label{ref_stats_db__plotColorVar}%
\hypertarget{ref_stats_db__plotColorVar}{}%
\begin{description}
\item[Summary:]Create a color-plot of parameter-test variations in a matrix.
%
\item[Usage:]~%
\begin{lyxcode}%
a\_plot = plotColorVar(p\_stats, props)
%
\end{lyxcode}%
%
\item[Description:]%
Skips the 'ItemIndex' test.
%%
\item[Parameters:]~
\begin{description}%
\item[\texttt{p\_stats}:]
 Array of invariant parameter databases obtained from

calling tests\_3D\_db/paramsTestsHistsStats.\item[\texttt{title\_str}:]
 (Optional) String to append to plot title.
\item[\texttt{props}:]
 A structure with any optional properties, passed to plot\_stack.
\begin{description}%
\item[\texttt{plotMethod}:]
 'plotVar' uses stats\_db/plotVar (default)

'plot\_bars' uses stats\_db/plot\_bars\end{description}%
\end{description}%
%
\item[Returns:]~

	a\_plot: A plot\_abstract with the color plot
%
%
\item[See also:]%
\hyperlink{ref_paramsTestsHistsStats}{\texttt{paramsTestsHistsStats}}%
\ (p.~\pageref{ref_paramsTestsHistsStats})%
\index[funcref]{@\fidxl{paramsTestsHistsStats}}%
, \hyperlink{ref_params_tests_profile}{\texttt{params\_tests\_profile}}%
\ (p.~\pageref{ref_params_tests_profile})%
\index[funcref]{@\fidxl{params\_tests\_profile}}%
, \hyperlink{ref_plotVar.}{\texttt{plotVar.}}%
\ (p.~\pageref{ref_plotVar.})%
\index[funcref]{@\fidxl{plotVar.}}%
%
\item[Author:]%
Cengiz Gunay <cgunay@emory.edu>, 2004/10/17%
\end{description}
\methodline%
\subsubsection[Method \texttt{plotVarMatrix}]{Method \texttt{stats\_db/plotVarMatrix}}%
\index[funcref]{stats_db@\fidxlb{stats\_db}!plotVarMatrix@\fidxl{plotVarMatrix}}%
\label{ref_stats_db__plotVarMatrix}%
\hypertarget{ref_stats_db__plotVarMatrix}{}%
\begin{description}
\item[Summary:]Create a stack of parameter-test variation plots organized in a matrix.
%
\item[Usage:]~%
\begin{lyxcode}%
a\_plot\_stack = plotVarMatrix(p\_stats, props)
%
\end{lyxcode}%
%
\item[Description:]%
Skips the 'ItemIndex' test.
%%
\item[Parameters:]~
\begin{description}%
\item[\texttt{p\_stats}:]
 Array of invariant parameter databases obtained from

calling tests\_3D\_db/paramsTestsHistsStats.\item[\texttt{props}:]
 A structure with any optional properties, passed to plot\_stack.
\begin{description}%
\item[\texttt{plotMethod}:]
 'plotVar' uses stats\_db/plotVar (default)

'plot\_bars' uses stats\_db/plot\_bars\item[\texttt{rotateYLabel}:]
 Rotate row labels this much (default=60).
\end{description}%
\end{description}%
%
\item[Returns:]~

	a\_plot\_stack: A plot\_stack with the plots organized in matrix form
%
%
\item[See also:]%
\hyperlink{ref_paramsTestsHistsStats}{\texttt{paramsTestsHistsStats}}%
\ (p.~\pageref{ref_paramsTestsHistsStats})%
\index[funcref]{@\fidxl{paramsTestsHistsStats}}%
, \hyperlink{ref_params_tests_profile}{\texttt{params\_tests\_profile}}%
\ (p.~\pageref{ref_params_tests_profile})%
\index[funcref]{@\fidxl{params\_tests\_profile}}%
, \hyperlink{ref_plotVar.}{\texttt{plotVar.}}%
\ (p.~\pageref{ref_plotVar.})%
\index[funcref]{@\fidxl{plotVar.}}%
%
\item[Author:]%
Cengiz Gunay <cgunay@emory.edu>, 2004/10/17%
\end{description}
\methodline%
\subsubsection[Method \texttt{plot\_abstract}]{Method \texttt{stats\_db/plot\_abstract}}%
\index[funcref]{stats_db@\fidxlb{stats\_db}!plot_abstract@\fidxl{plot\_abstract}}%
\label{ref_stats_db__plot_abstract}%
\hypertarget{ref_stats_db__plot_abstract}{}%
\begin{description}
\item[Summary:]Generates an error bar graph for each db columns. Looks for 'min', 'max', and 'STD' labels in the row\_idx for drawing the errorbars.
%
\item[Usage:]~%
\begin{lyxcode}%
a\_plot = plot\_abstract(a\_stats\_db, title\_str, props)
%
\end{lyxcode}%
%
\item[Description:]%
Generates a plot\_simple object from this histogram.
%%
\item[Parameters:]~
\begin{description}%
\item[\texttt{a\_stats\_db}:]
 A histogram\_db object.
\item[\texttt{title\_str}:]
 A title string on the plot
\item[\texttt{props}:]
 A structure with any optional properties.
\end{description}%
%
\item[Returns:]~

	a\_plot: A object of plot\_abstract or one of its subclasses.
%
%
\item[See also:]%
\hyperlink{ref_plot_abstract}{\texttt{plot\_abstract}}%
\ (p.~\pageref{ref_plot_abstract})%
\index[funcref]{@\fidxl{plot\_abstract}}%
, \hyperlink{ref_plot_simple}{\texttt{plot\_simple}}%
\ (p.~\pageref{ref_plot_simple})%
\index[funcref]{@\fidxl{plot\_simple}}%
%
\item[Author:]%
Cengiz Gunay <cgunay@emory.edu>, 2004/10/08%
\end{description}
\methodline%
\subsubsection[Method \texttt{subsref}]{Method \texttt{stats\_db/subsref}}%
\index[funcref]{stats_db@\fidxlb{stats\_db}!subsref@\fidxl{subsref}}%
\label{ref_stats_db__subsref}%
\hypertarget{ref_stats_db__subsref}{}%
\begin{description}
\item[Summary:]Defines generic indexing for objects.
%
%
%
%
%
%
%
%
\end{description}
\methodline%
\subsubsection[Method \texttt{compareStats}]{Method \texttt{stats\_db/compareStats}}%
\index[funcref]{stats_db@\fidxlb{stats\_db}!compareStats@\fidxl{compareStats}}%
\label{ref_stats_db__compareStats}%
\hypertarget{ref_stats_db__compareStats}{}%
\begin{description}
\item[Summary:]Merges multiple stats\_dbs into pages of a single stats\_db for comparison.
%
\item[Usage:]~%
\begin{lyxcode}%
a\_mult\_stats\_db = compareStats(a\_stats\_db)
%
\end{lyxcode}%
%
\item[Description:]%
Generates a plot\_simple object from this histogram.
%%
\item[Parameters:]~
\begin{description}%
\item[\texttt{a\_stats\_db}:]
 A histogram\_db object.
\item[\texttt{command}:]
 Plot command (Optional, default='bar')
\end{description}%
%
\item[Returns:]~

	a\_mult\_stats\_db: A object of compareStats or one of its subclasses.
%
%
\item[See also:]%
\hyperlink{ref_plot_abstract}{\texttt{plot\_abstract}}%
\ (p.~\pageref{ref_plot_abstract})%
\index[funcref]{@\fidxl{plot\_abstract}}%
, \hyperlink{ref_plot_simple}{\texttt{plot\_simple}}%
\ (p.~\pageref{ref_plot_simple})%
\index[funcref]{@\fidxl{plot\_simple}}%
%
\item[Author:]%
Cengiz Gunay <cgunay@emory.edu>, 2004/10/08%
\end{description}
\methodline%
\subsubsection[Method \texttt{plotYTests}]{Method \texttt{stats\_db/plotYTests}}%
\index[funcref]{stats_db@\fidxlb{stats\_db}!plotYTests@\fidxl{plotYTests}}%
\label{ref_stats_db__plotYTests}%
\hypertarget{ref_stats_db__plotYTests}{}%
\begin{description}
\item[Summary:]Create an errorbar plot of database stats measures against given X-axis values.
%
\item[Usage:]~%
\begin{lyxcode}%
a\_p = plotYTests(a\_stats\_db, x\_vals, tests, axis\_labels, title\_str, short\_title, command, props)
%
\end{lyxcode}%
%
%
\item[Parameters:]~
\begin{description}%
\item[\texttt{a\_stats\_db}:]
 A params\_tests\_db object.
\item[\texttt{x\_vals}:]
 A vector of X-axis values.
\item[\texttt{tests}:]
 A vector or cell array of columns to correspond to each value from x\_vals.
\item[\texttt{title\_str}:]
 (Optional) A string to be concatanated to the title.
\item[\texttt{short\_title}:]
 (Optional) Few words that may appear in legends of multiplot.
\item[\texttt{command}:]
 (Optional) Command to do the plotting with (default: 'plot')
\item[\texttt{props}:]
 A structure with any optional properties.
\begin{description}%
\item[\texttt{LineStyle}:]
 Plot line style to use. (default: 'd-')
\item[\texttt{quiet}:]
 If 1, don't include database name on title.
\end{description}%
\end{description}%
%
\item[Returns:]~

	a\_p: A plot\_abstract.
%
\item[Example:]~
\begin{lyxcode} >> a\_p = plotYTests(a\_stats\_db, [0 40 100 200], ...\\%
                      {'IniSpontSpikeRateISI\_0pA', 'PulseIni100msSpikeRateISI\_D40pA', ...\\%
                       'PulseIni100msSpikeRateISI\_D100pA', 'PulseIni100msSpikeRateISI\_D200pA'}, ...\\%
                      {'current pulse [pA]', 'firing rate [Hz]'}, ', f-I curves', 'neuron 1');\\%
 >> plotFigure(a\_p);\\%
\end{lyxcode}
%
\item[See also:]%
\hyperlink{ref_plotFigure}{\texttt{plotFigure}}%
\ (p.~\pageref{ref_plotFigure})%
\index[funcref]{@\fidxl{plotFigure}}%
%
\item[Author:]%
Cengiz Gunay <cgunay@emory.edu>, 2006/01/23%
\end{description}
\methodline%
\subsubsection[Method \texttt{plot\_bars}]{Method \texttt{stats\_db/plot\_bars}}%
\index[funcref]{stats_db@\fidxlb{stats\_db}!plot_bars@\fidxl{plot\_bars}}%
\label{ref_stats_db__plot_bars}%
\hypertarget{ref_stats_db__plot_bars}{}%
\begin{description}
\item[Summary:]Creates a bar graph with errorbars for each db column. 
%
\item[Usage:]~%
\begin{lyxcode}%
a\_plot = plot\_bars(a\_stats\_db, title\_str, props)
%
\end{lyxcode}%
%
\item[Description:]%
Looks for 'min', 'max', and 'STD' labels in the row\_idx for drawing the errorbars. 
 Each page of the DB will produce grouped bars.
%%
\item[Parameters:]~
\begin{description}%
\item[\texttt{a\_stats\_db}:]
 A stats\_db object.
\item[\texttt{title\_str}:]
 The plot title.
\item[\texttt{props}:]
 A structure with any optional properties, passed to plot\_bars/plot\_bars.
\begin{description}%
\item[\texttt{pageVariable}:]
 The column used for denoting page values.
\end{description}%
\end{description}%
%
\item[Returns:]~

	a\_plot: A object of plot\_bars or one of its subclasses.
%
%
\item[See also:]%
\hyperlink{ref_plot_abstract}{\texttt{plot\_abstract}}%
\ (p.~\pageref{ref_plot_abstract})%
\index[funcref]{@\fidxl{plot\_abstract}}%
, \hyperlink{ref_plot_bars__plot_bars}{\texttt{plot\_bars/plot\_bars}}%
\ (p.~\pageref{ref_plot_bars__plot_bars})%
\index[funcref]{plot_bars@\fidxlb{plot\_bars}!plot_bars@\fidxl{plot\_bars}}%
%
\item[Author:]%
Cengiz Gunay <cgunay@emory.edu>, 2004/10/08%
\end{description}
\methodline%
\subsection{Class \texttt{tests\_3D\_db}}%
\index[funcref]{tests_3D_db@\fidxlb{tests\_3D\_db}}%
\label{ref_tests_3D_db}%
\hypertarget{ref_tests_3D_db}{}%
\subsubsection[Constructor \texttt{tests\_3D\_db}]{Constructor \texttt{tests\_3D\_db/tests\_3D\_db}}%
\index[funcref]{tests_3D_db@\fidxlb{tests\_3D\_db}!tests_3D_db@\fidxl{tests\_3D\_db}}%
\label{ref_tests_3D_db__tests_3D_db}%
\hypertarget{ref_tests_3D_db__tests_3D_db}{}%
\begin{description}
\item[Summary:]A database multiple pages with rows of test columns. 
		Each page may represent aspects of the data that are
		different, but not defined in this object.
%
\item[Usage:]~%
\begin{lyxcode}%
a\_3D\_db = tests\_3D\_db(data, col\_names, row\_names, page\_names, id, props)
%
\end{lyxcode}%
%
\item[Description:]%
This is a subclass of tests\_db. Usually it contains a RowIndex
 column that points to an original db from which this data originated. 
 The row indices can be used to reach the values associated with different
 pages of information contained in this object.
%%
\item[Parameters:]~
\begin{description}%
\item[\texttt{data}:]
 The 3-d vector of rows, columns, and pages.
\item[\texttt{col\_names}:]
 Colun names of the database.
\item[\texttt{id}:]
 An identifying string.
\item[\texttt{props}:]
 A structure with any optional properties.
\begin{description}%
\item[\texttt{invarName}:]
 Name of the invariant parameter for this db.
\end{description}%
\end{description}%
%
\item[Returns a structure object with the following fields:]~

	tests\_db, page\_idx.
%
%
\item[See also:]%
\hyperlink{ref_tests_db}{\texttt{tests\_db}}%
\ (p.~\pageref{ref_tests_db})%
\index[funcref]{@\fidxl{tests\_db}}%
, \hyperlink{ref_tests_db__invarValues}{\texttt{tests\_db/invarValues}}%
\ (p.~\pageref{ref_tests_db__invarValues})%
\index[funcref]{tests_db@\fidxlb{tests\_db}!invarValues@\fidxl{invarValues}}%
%
\item[Author:]%
Cengiz Gunay <cgunay@emory.edu>, 2004/09/30%
\end{description}
\methodline%
\subsubsection[Method \texttt{joinPages}]{Method \texttt{tests\_3D\_db/joinPages}}%
\index[funcref]{tests_3D_db@\fidxlb{tests\_3D\_db}!joinPages@\fidxl{joinPages}}%
\label{ref_tests_3D_db__joinPages}%
\hypertarget{ref_tests_3D_db__joinPages}{}%
\begin{description}
\item[Summary:]Joins the rows of the given db to the with\_db rows matching with the PageIndex
 	column.
%
\item[Usage:]~%
\begin{lyxcode}%
a\_db = joinPages(db, tests, with\_db, w\_tests)
%
\end{lyxcode}%
%
\item[Description:]%
Replicates the desired columns in the with\_db with rows having a 
 page index and joins them next to desired columns from the current 3D\_db. Flattens 
 the resulting 3D\_db to become a 2D db. Assumes each page index only 
 appears once in with\_db.
%%
\item[Parameters:]~
\begin{description}%
\item[\texttt{db}:]
 A tests\_3D\_db object.
\item[\texttt{with\_db}:]
 A tests\_db object with a PageIndex column.
\end{description}%
%
\item[Returns:]~

	a\_db: A tests\_db object.
%
%
\item[See also:]%
\hyperlink{ref_tests_db}{\texttt{tests\_db}}%
\ (p.~\pageref{ref_tests_db})%
\index[funcref]{@\fidxl{tests\_db}}%
%
\item[Author:]%
Cengiz Gunay <cgunay@emory.edu>, 2004/10/15%
\end{description}
\methodline%
\subsubsection[Method \texttt{display}]{Method \texttt{tests\_3D\_db/display}}%
\index[funcref]{tests_3D_db@\fidxlb{tests\_3D\_db}!display@\fidxl{display}}%
\label{ref_tests_3D_db__display}%
\hypertarget{ref_tests_3D_db__display}{}%
\begin{description}
%
%
%
%
%
%
%
\item[Author:]%
Cengiz Gunay <cgunay@emory.edu>, 2004/08/04%
\end{description}
\methodline%
\subsubsection[Method \texttt{diff2D}]{Method \texttt{tests\_3D\_db/diff2D}}%
\index[funcref]{tests_3D_db@\fidxlb{tests\_3D\_db}!diff2D@\fidxl{diff2D}}%
\label{ref_tests_3D_db__diff2D}%
\hypertarget{ref_tests_3D_db__diff2D}{}%
\begin{description}
\item[Summary:]Creates a tests\_db by taking the derivative of the given test.
%
\item[Usage:]~%
\begin{lyxcode}%
a\_tests\_db = diff2D(a\_db, test, props)
%
\end{lyxcode}%
%
\item[Description:]%
Applies the diff function to the chosen test, and collapses the middle
 dimension of the 3D DB to create a 2D DB.
%%
\item[Parameters:]~
\begin{description}%
\item[\texttt{a\_db}:]
 A tests\_3D\_db object.
\item[\texttt{test}:]
 Test column.
\item[\texttt{props}:]
 Optional properties.
\end{description}%
%
\item[Returns:]~

	a\_tests\_db: A tests\_db that holds the requested differences of parameter values.
%
%
\item[See also:]%
\hyperlink{ref_boxplot}{\texttt{boxplot}}%
\ (p.~\pageref{ref_boxplot})%
\index[funcref]{@\fidxl{boxplot}}%
, \hyperlink{ref_plot_abstract}{\texttt{plot\_abstract}}%
\ (p.~\pageref{ref_plot_abstract})%
\index[funcref]{@\fidxl{plot\_abstract}}%
%
\item[Author:]%
Cengiz Gunay <cgunay@emory.edu>, 2005/05/22%
\end{description}
\methodline%
\subsubsection[Method \texttt{get}]{Method \texttt{tests\_3D\_db/get}}%
\index[funcref]{tests_3D_db@\fidxlb{tests\_3D\_db}!get@\fidxl{get}}%
\label{ref_tests_3D_db__get}%
\hypertarget{ref_tests_3D_db__get}{}%
\begin{description}
\item[Summary:]Defines generic attribute retrieval for objects.
%
%
%
%
%
%
%
\item[Author:]%
Cengiz Gunay <cgunay@emory.edu>, 2004/09/14%
\end{description}
\methodline%
\subsubsection[Method \texttt{set}]{Method \texttt{tests\_3D\_db/set}}%
\index[funcref]{tests_3D_db@\fidxlb{tests\_3D\_db}!set@\fidxl{set}}%
\label{ref_tests_3D_db__set}%
\hypertarget{ref_tests_3D_db__set}{}%
\begin{description}
\item[Summary:]Generic method for setting object attributes.
%
%
%
%
%
%
%
\item[Author:]%
Cengiz Gunay <cgunay@emory.edu>, 2004/10/08%
\end{description}
\methodline%
\subsubsection[Method \texttt{plotParamPairImage}]{Method \texttt{tests\_3D\_db/plotParamPairImage}}%
\index[funcref]{tests_3D_db@\fidxlb{tests\_3D\_db}!plotParamPairImage@\fidxl{plotParamPairImage}}%
\label{ref_tests_3D_db__plotParamPairImage}%
\hypertarget{ref_tests_3D_db__plotParamPairImage}{}%
\begin{description}
\item[Summary:]Generates an image plot of variation of a test with two parameters in the first page.
%
\item[Usage:]~%
\begin{lyxcode}%
a\_plot = plotParamPairImage(a\_db, test, title\_str, props)
%
\end{lyxcode}%
%
\item[Description:]%
It is assumed that the 3D DB is created by invariant combinations of two parameters,
 which are the first two columns. Each page of the db must contain a same parameter 
 values. This is the default character of tests\_3D\_db created by 
 params\_tests\_db/invarParam. Parameter values will be enumerated and then an 
 image plot is created.
%%
\item[Parameters:]~
\begin{description}%
\item[\texttt{a\_db}:]
 A tests\_3D\_db object.
\item[\texttt{test}:]
 Test column to take the measure value.
\item[\texttt{title\_str}:]
 (Optional) String to append to plot title.
\item[\texttt{props}:]
 Optional properties to be passed to plot\_abstract.
\begin{description}%
\item[\texttt{truncateDecDigits}:]
 Truncate labels to this many decimal digits.
\item[\texttt{labelSteps}:]
 Skip this many labels between ticks to reduce to total number.
\end{description}%
\end{description}%
%
\item[Returns:]~

	a\_plot: A plot\_abstract object or one of its subclasses.
%
\item[Example:]~
\begin{lyxcode} Find relationship of two parameters against a measure:\\%
 >> plotFigure(plotParamPairImage(invarParam(a\_db, {'NaF', 'KCNQ'}), 'PulseIni100msRest2SpikeRateISI\_D100pA'));\\%
\end{lyxcode}
%
\item[See also:]%
\hyperlink{ref_params_tests_db__invarParam}{\texttt{params\_tests\_db/invarParam}}%
\ (p.~\pageref{ref_params_tests_db__invarParam})%
\index[funcref]{params_tests_db@\fidxlb{params\_tests\_db}!invarParam@\fidxl{invarParam}}%
, \hyperlink{ref_plotImage}{\texttt{plotImage}}%
\ (p.~\pageref{ref_plotImage})%
\index[funcref]{@\fidxl{plotImage}}%
, \hyperlink{ref_plot_abstract.}{\texttt{plot\_abstract.}}%
\ (p.~\pageref{ref_plot_abstract.})%
\index[funcref]{@\fidxl{plot\_abstract.}}%
%
\item[Author:]%
Cengiz Gunay <cgunay@emory.edu>, 2004/11/10%
\end{description}
\methodline%
\subsubsection[Method \texttt{histograms}]{Method \texttt{tests\_3D\_db/histograms}}%
\index[funcref]{tests_3D_db@\fidxlb{tests\_3D\_db}!histograms@\fidxl{histograms}}%
\label{ref_tests_3D_db__histograms}%
\hypertarget{ref_tests_3D_db__histograms}{}%
\begin{description}
%
\item[Usage:]~%
\begin{lyxcode}%
a\_histogram\_db = histogram(db, col, num\_bins)
%
\end{lyxcode}%
%
\item[Description:]%
If one wants to get histograms of test values for each single value of
 the selected invariant parameter, then swapRowsPages should be done
 first on db.
%%
\item[Parameters:]~
\begin{description}%
\item[\texttt{db}:]
 A tests\_3D\_db object.
\item[\texttt{col}:]
 Column to find the histogram.
\item[\texttt{num\_bins}:]
 Number of histogram bins (Optional, default=100)
\end{description}%
%
\item[Returns:]~

	a\_histogram\_db: A histogram\_db object containing the histogram.
%
%
\item[See also:]%
\hyperlink{ref_histogram_db}{\texttt{histogram\_db}}%
\ (p.~\pageref{ref_histogram_db})%
\index[funcref]{@\fidxl{histogram\_db}}%
, \hyperlink{ref_tests_db}{\texttt{tests\_db}}%
\ (p.~\pageref{ref_tests_db})%
\index[funcref]{@\fidxl{tests\_db}}%
%
\item[Author:]%
Cengiz Gunay <cgunay@emory.edu>, 2004/10/04%
\end{description}
\methodline%
\subsubsection[Method \texttt{swapRowsPages}]{Method \texttt{tests\_3D\_db/swapRowsPages}}%
\index[funcref]{tests_3D_db@\fidxlb{tests\_3D\_db}!swapRowsPages@\fidxl{swapRowsPages}}%
\label{ref_tests_3D_db__swapRowsPages}%
\hypertarget{ref_tests_3D_db__swapRowsPages}{}%
\begin{description}
\item[Summary:]Swaps the row dimension with the page dimension of the
		  tests\_3D\_db.
%
\item[Usage:]~%
\begin{lyxcode}%
a\_3D\_db = swapRowsPages(db)
%
\end{lyxcode}%
%
\item[Description:]%
Assuming that this is a invariant parameter and tests relations db, 
 this function swaps the pages with rows. Each resulting page correspond
 to a single value of the chosen parameter, with each row contianing a 
 test result with different combinations of the rest of the parameters.
%%
\item[Parameters:]~
\begin{description}%
\item[\texttt{db}:]
 A tests\_db object.
\end{description}%
%
\item[Returns:]~

	a\_3D\_db: A tests\_3D\_db object.
%
%
\item[See also:]%
\hyperlink{ref_tests_db}{\texttt{tests\_db}}%
\ (p.~\pageref{ref_tests_db})%
\index[funcref]{@\fidxl{tests\_db}}%
%
\item[Author:]%
Cengiz Gunay <cgunay@emory.edu>, 2004/10/04%
\end{description}
\methodline%
\subsubsection[Method \texttt{paramsTestsHistsStats}]{Method \texttt{tests\_3D\_db/paramsTestsHistsStats}}%
\index[funcref]{tests_3D_db@\fidxlb{tests\_3D\_db}!paramsTestsHistsStats@\fidxl{paramsTestsHistsStats}}%
\label{ref_tests_3D_db__paramsTestsHistsStats}%
\hypertarget{ref_tests_3D_db__paramsTestsHistsStats}{}%
\begin{description}
\item[Summary:]Calculates histograms and statistics for DB.
%
\item[Usage:]~%
\begin{lyxcode}%
[pt\_hists, p\_stats] = paramsTestsHistsStats(p\_t3ds, props)
%
\end{lyxcode}%
%
\item[Description:]%
Calculates histograms and statistics for all combinations of tests 
 and params and returns them in a cell array. Skips the 'ItemIndex' test.
%%
\item[Parameters:]~
\begin{description}%
\item[\texttt{p\_t3ds}:]
 Array of invariant parameter databases obtained by

calling the params\_tests\_db/invarParams method.\item[\texttt{props}:]
 Optional properties.
\begin{description}%
\item[\texttt{statsMethod}:]
 method to call to get a stats\_db (default='statsMeanSE')
\item[\texttt{useDiff}:]
 If 1, takes the derivative with diff on the 3D DBs (default=0).
\end{description}%
\end{description}%
%
\item[Returns:]~

	pt\_hists: An array of 3D histograms for each pair of param 
		  and test.
	p\_stats: An array of stats\_dbs for each param.
%
%
\item[See also:]%
\hyperlink{ref_invarParams}{\texttt{invarParams}}%
\ (p.~\pageref{ref_invarParams})%
\index[funcref]{@\fidxl{invarParams}}%
, \hyperlink{ref_params_tests_profile}{\texttt{params\_tests\_profile}}%
\ (p.~\pageref{ref_params_tests_profile})%
\index[funcref]{@\fidxl{params\_tests\_profile}}%
%
\item[Author:]%
Cengiz Gunay <cgunay@emory.edu>, 2004/10/17%
\end{description}
\methodline%
\subsubsection[Method \texttt{mergePages}]{Method \texttt{tests\_3D\_db/mergePages}}%
\index[funcref]{tests_3D_db@\fidxlb{tests\_3D\_db}!mergePages@\fidxl{mergePages}}%
\label{ref_tests_3D_db__mergePages}%
\hypertarget{ref_tests_3D_db__mergePages}{}%
\begin{description}
\item[Summary:]Merges tests from separate pages into a 2D params\_tests\_db.
%
\item[Usage:]~%
\begin{lyxcode}%
a\_db = mergePages(db, page\_tests, page\_suffixes)
%
\end{lyxcode}%
%
\item[Description:]%
Keeps uniqueness by adding suffixes to test names.
 If you're using invarParams, do swapRowsPages, then join with original db to get
 the parameter values.
%%
\item[Parameters:]~
\begin{description}%
\item[\texttt{db}:]
 A tests\_3D\_db object.
\item[\texttt{page\_tests}:]
 Cell array of list of tests to take from each page.
\item[\texttt{page\_suffixes}:]
 Cell array of suffixes to append to tests from each page.
\end{description}%
%
\item[Returns:]~

	a\_db: A tests\_db object.
%
%
\item[See also:]%
\hyperlink{ref_tests_db}{\texttt{tests\_db}}%
\ (p.~\pageref{ref_tests_db})%
\index[funcref]{@\fidxl{tests\_db}}%
, \hyperlink{ref_tests_3D_db}{\texttt{tests\_3D\_db}}%
\ (p.~\pageref{ref_tests_3D_db})%
\index[funcref]{@\fidxl{tests\_3D\_db}}%
%
\item[Author:]%
Cengiz Gunay <cgunay@emory.edu>, 2005/01/13%
\end{description}
\methodline%
\subsubsection[Method \texttt{plotVarBox}]{Method \texttt{tests\_3D\_db/plotVarBox}}%
\index[funcref]{tests_3D_db@\fidxlb{tests\_3D\_db}!plotVarBox@\fidxl{plotVarBox}}%
\label{ref_tests_3D_db__plotVarBox}%
\hypertarget{ref_tests_3D_db__plotVarBox}{}%
\begin{description}
\item[Summary:]Generates a boxplot of the variation between two tests.
%
\item[Usage:]~%
\begin{lyxcode}%
a\_plot = plotVarBox(a\_db, test1, test2, notch, sym, vert, whis, props)
%
\end{lyxcode}%
%
\item[Description:]%
It is assumed that each page of the db contains a different parameter value.
%%
\item[Parameters:]~
\begin{description}%
\item[\texttt{a\_db}:]
 A tests\_3D\_db object.
\item[\texttt{test1}:]
 Test column for the x-axis, only mean values are used.
\item[\texttt{test2}:]
 Test column for the y-axis, used for boxplot.
\item[\texttt{notch, sym, vert, whis}:]
 See boxplot, defaults = (1, '+', 1, 1.5).
\item[\texttt{props}:]
 Optional properties to be passed to plot\_abstract.
\end{description}%
%
\item[Returns:]~

	a\_plot: A plot\_abstract object or one of its subclasses.
%
%
\item[See also:]%
\hyperlink{ref_boxplot}{\texttt{boxplot}}%
\ (p.~\pageref{ref_boxplot})%
\index[funcref]{@\fidxl{boxplot}}%
, \hyperlink{ref_plot_abstract}{\texttt{plot\_abstract}}%
\ (p.~\pageref{ref_plot_abstract})%
\index[funcref]{@\fidxl{plot\_abstract}}%
%
\item[Author:]%
Cengiz Gunay <cgunay@emory.edu>, 2004/11/10%
\end{description}
\methodline%
\subsection{Class \texttt{tests\_db}}%
\index[funcref]{tests_db@\fidxlb{tests\_db}}%
\label{ref_tests_db}%
\hypertarget{ref_tests_db}{}%
\subsubsection[Constructor \texttt{tests\_db}]{Constructor \texttt{tests\_db/tests\_db}}%
\index[funcref]{tests_db@\fidxlb{tests\_db}!tests_db@\fidxl{tests\_db}}%
\label{ref_tests_db__tests_db}%
\hypertarget{ref_tests_db__tests_db}{}%
\begin{description}
\item[Summary:]A generic database of test results organized in a matrix format.
%
\item[Usage:]~%
\begin{lyxcode}%
obj = tests\_db(test\_results, col\_names, row\_names, id, props)
%
\end{lyxcode}%
%
\item[Description:]%
Defines all operations on this structure so that subclasses can use them.
%%
\item[Parameters:]~
\begin{description}%
\item[\texttt{test\_results}:]
 A matrix that contains columns associated with

tests and rows for separate observations.\item[\texttt{col\_names}:]
 Cell array of column names of test\_results.
\item[\texttt{row\_names}:]
 Cell array of row names of test\_results.
\item[\texttt{id}:]
 An identifying string.
\item[\texttt{props}:]
 A structure with any optional properties.
\end{description}%
%
\item[Returns a structure object with the following fields:]~

	data: The data matrix.
	row\_idx, col\_idx: Structure associating row/column names to indices.
	id, props.
%
%
\item[See also:]%
\hyperlink{ref_params_tests_db}{\texttt{params\_tests\_db}}%
\ (p.~\pageref{ref_params_tests_db})%
\index[funcref]{@\fidxl{params\_tests\_db}}%
, \hyperlink{ref_params_db}{\texttt{params\_db}}%
\ (p.~\pageref{ref_params_db})%
\index[funcref]{@\fidxl{params\_db}}%
, \hyperlink{ref_test_variable_db (N__I)}{\texttt{test\_variable\_db (N/I)}}%
\ (p.~\pageref{ref_test_variable_db (N__I)})%
\index[funcref]{test_variable_db (N@\fidxlb{test\_variable\_db (N}!I)@\fidxl{I)}}%
%
\item[Author:]%
Cengiz Gunay <cgunay@emory.edu>, 2004/09/01%
\end{description}
\methodline%
\subsubsection[Method \texttt{eq}]{Method \texttt{tests\_db/eq}}%
\index[funcref]{tests_db@\fidxlb{tests\_db}!eq@\fidxl{eq}}%
\label{ref_tests_db__eq}%
\hypertarget{ref_tests_db__eq}{}%
\begin{description}
\item[Summary:]Equality (==) operator. Returns logical indices of db rows 
	that match with given row.
%
\item[Usage:]~%
\begin{lyxcode}%
rows = eq(db, row)
%
\end{lyxcode}%
%
%
\item[Parameters:]~
\begin{description}%
\item[\texttt{db}:]
 A tests\_db object.
\item[\texttt{row}:]
 Row array to be compared with db rows.
\end{description}%
%
\item[Returns:]~

	rows: A logical or index vector to be used in indexing db objects. 
%
%
\item[See also:]%
\hyperlink{ref_eq}{\texttt{eq}}%
\ (p.~\pageref{ref_eq})%
\index[funcref]{@\fidxl{eq}}%
, \hyperlink{ref_tests_db}{\texttt{tests\_db}}%
\ (p.~\pageref{ref_tests_db})%
\index[funcref]{@\fidxl{tests\_db}}%
%
\item[Author:]%
Cengiz Gunay <cgunay@emory.edu>, 2004/09/17%
\end{description}
\methodline%
\subsubsection[Method \texttt{ge}]{Method \texttt{tests\_db/ge}}%
\index[funcref]{tests_db@\fidxlb{tests\_db}!ge@\fidxl{ge}}%
\label{ref_tests_db__ge}%
\hypertarget{ref_tests_db__ge}{}%
\begin{description}
\item[Summary:]Greater or equal to (>=) operator. Returns logical indices of db rows 
	that are greater than or equal to given row.
%
\item[Usage:]~%
\begin{lyxcode}%
rows = ge(db, row)
%
\end{lyxcode}%
%
%
\item[Parameters:]~
\begin{description}%
\item[\texttt{db}:]
 A tests\_db object.
\item[\texttt{row}:]
 Row array to be compared with db rows.
\end{description}%
%
\item[Returns:]~

	rows: A logical or index vector to be used in indexing db objects. 
%
%
\item[See also:]%
\hyperlink{ref_ge}{\texttt{ge}}%
\ (p.~\pageref{ref_ge})%
\index[funcref]{@\fidxl{ge}}%
, \hyperlink{ref_tests_db}{\texttt{tests\_db}}%
\ (p.~\pageref{ref_tests_db})%
\index[funcref]{@\fidxl{tests\_db}}%
%
\item[Author:]%
Cengiz Gunay <cgunay@emory.edu>, 2004/09/17%
\end{description}
\methodline%
\subsubsection[Method \texttt{gt}]{Method \texttt{tests\_db/gt}}%
\index[funcref]{tests_db@\fidxlb{tests\_db}!gt@\fidxl{gt}}%
\label{ref_tests_db__gt}%
\hypertarget{ref_tests_db__gt}{}%
\begin{description}
\item[Summary:]Greater than (>) operator. Returns logical indices of db rows 
	that are greater than given row.
%
\item[Usage:]~%
\begin{lyxcode}%
rows = gt(db, row)
%
\end{lyxcode}%
%
%
\item[Parameters:]~
\begin{description}%
\item[\texttt{db}:]
 A tests\_db object.
\item[\texttt{row}:]
 Row array to be compared with db rows.
\end{description}%
%
\item[Returns:]~

	rows: A logical or index vector to be used in indexing db objects. 
%
%
\item[See also:]%
\hyperlink{ref_gt}{\texttt{gt}}%
\ (p.~\pageref{ref_gt})%
\index[funcref]{@\fidxl{gt}}%
, \hyperlink{ref_tests_db}{\texttt{tests\_db}}%
\ (p.~\pageref{ref_tests_db})%
\index[funcref]{@\fidxl{tests\_db}}%
%
\item[Author:]%
Cengiz Gunay <cgunay@emory.edu>, 2004/09/17%
\end{description}
\methodline%
\subsubsection[Method \texttt{le}]{Method \texttt{tests\_db/le}}%
\index[funcref]{tests_db@\fidxlb{tests\_db}!le@\fidxl{le}}%
\label{ref_tests_db__le}%
\hypertarget{ref_tests_db__le}{}%
\begin{description}
\item[Summary:]Less or equal (<=) operator. Returns logical indices of db rows 
	that are less than or equal to given row.
%
\item[Usage:]~%
\begin{lyxcode}%
rows = le(db, row)
%
\end{lyxcode}%
%
%
\item[Parameters:]~
\begin{description}%
\item[\texttt{db}:]
 A tests\_db object.
\item[\texttt{row}:]
 Row array to be compared with db rows.
\end{description}%
%
\item[Returns:]~

	rows: A logical or index vector to be used in indexing db objects. 
%
%
\item[See also:]%
\hyperlink{ref_le}{\texttt{le}}%
\ (p.~\pageref{ref_le})%
\index[funcref]{@\fidxl{le}}%
, \hyperlink{ref_tests_db}{\texttt{tests\_db}}%
\ (p.~\pageref{ref_tests_db})%
\index[funcref]{@\fidxl{tests\_db}}%
%
\item[Author:]%
Cengiz Gunay <cgunay@emory.edu>, 2004/09/17%
\end{description}
\methodline%
\subsubsection[Method \texttt{lt}]{Method \texttt{tests\_db/lt}}%
\index[funcref]{tests_db@\fidxlb{tests\_db}!lt@\fidxl{lt}}%
\label{ref_tests_db__lt}%
\hypertarget{ref_tests_db__lt}{}%
\begin{description}
\item[Summary:]Less than (<) operator. Returns logical indices of db rows 
	that are less than given row.
%
\item[Usage:]~%
\begin{lyxcode}%
rows = lt(db, row)
%
\end{lyxcode}%
%
%
\item[Parameters:]~
\begin{description}%
\item[\texttt{db}:]
 A tests\_db object.
\item[\texttt{row}:]
 Row array to be compared with db rows.
\end{description}%
%
\item[Returns:]~

	rows: A logical or index vector to be used in indexing db objects. 
%
%
\item[See also:]%
\hyperlink{ref_lt}{\texttt{lt}}%
\ (p.~\pageref{ref_lt})%
\index[funcref]{@\fidxl{lt}}%
, \hyperlink{ref_tests_db}{\texttt{tests\_db}}%
\ (p.~\pageref{ref_tests_db})%
\index[funcref]{@\fidxl{tests\_db}}%
%
\item[Author:]%
Cengiz Gunay <cgunay@emory.edu>, 2004/09/17%
\end{description}
\methodline%
\subsubsection[Method \texttt{ne}]{Method \texttt{tests\_db/ne}}%
\index[funcref]{tests_db@\fidxlb{tests\_db}!ne@\fidxl{ne}}%
\label{ref_tests_db__ne}%
\hypertarget{ref_tests_db__ne}{}%
\begin{description}
\item[Summary:]Returns logical indices of db rows that doesn't match with given row.
%
\item[Usage:]~%
\begin{lyxcode}%
rows = ne(db, row)
%
\end{lyxcode}%
%
%
\item[Parameters:]~
\begin{description}%
\item[\texttt{db}:]
 A tests\_db object.
\item[\texttt{row}:]
 Row array to be compared with db rows.
\end{description}%
%
\item[Returns:]~

	rows: A logical or index vector to be used in indexing db objects. 
%
%
\item[See also:]%
\hyperlink{ref_ne}{\texttt{ne}}%
\ (p.~\pageref{ref_ne})%
\index[funcref]{@\fidxl{ne}}%
, \hyperlink{ref_tests_db}{\texttt{tests\_db}}%
\ (p.~\pageref{ref_tests_db})%
\index[funcref]{@\fidxl{tests\_db}}%
%
\item[Author:]%
Cengiz Gunay <cgunay@emory.edu>, 2004/09/17%
\end{description}
\methodline%
\subsubsection[Method \texttt{onlyRowsTests}]{Method \texttt{tests\_db/onlyRowsTests}}%
\index[funcref]{tests_db@\fidxlb{tests\_db}!onlyRowsTests@\fidxl{onlyRowsTests}}%
\label{ref_tests_db__onlyRowsTests}%
\hypertarget{ref_tests_db__onlyRowsTests}{}%
\begin{description}
\item[Summary:]Returns a tests\_db that only contains the desired 
		tests and rows (and pages).
%
\item[Usage:]~%
\begin{lyxcode}%
obj = onlyRowsTests(obj, rows, tests, pages)
%
\end{lyxcode}%
%
\item[Description:]%
Selects the given dimensions and returns in a new tests\_db object.
%%
\item[Parameters:]~
\begin{description}%
\item[\texttt{obj}:]
 A tests\_db object.
\item[\texttt{rows}:]
 A logical or index vector of rows. If ':', all rows.
\item[\texttt{tests}:]
 Cell array of test names or column indices. If ':', all tests.
\item[\texttt{pages}:]
 (Optional) A logical or index vector of pages. ':' for all pages.
\end{description}%
%
\item[Returns:]~

	obj: The new tests\_db object.
%
%
\item[See also:]%
\hyperlink{ref_subsref}{\texttt{subsref}}%
\ (p.~\pageref{ref_subsref})%
\index[funcref]{@\fidxl{subsref}}%
, \hyperlink{ref_tests_db}{\texttt{tests\_db}}%
\ (p.~\pageref{ref_tests_db})%
\index[funcref]{@\fidxl{tests\_db}}%
%
\item[Author:]%
Cengiz Gunay <cgunay@emory.edu>, 2004/09/17%
\end{description}
\methodline%
\subsubsection[Method \texttt{setProp}]{Method \texttt{tests\_db/setProp}}%
\index[funcref]{tests_db@\fidxlb{tests\_db}!setProp@\fidxl{setProp}}%
\label{ref_tests_db__setProp}%
\hypertarget{ref_tests_db__setProp}{}%
\begin{description}
\item[Summary:]Generic method for setting optional object properties.
%
\item[Usage:]~%
\begin{lyxcode}%
obj = setProp(obj, prop1, val1, prop2, val2, ...)
%
\end{lyxcode}%
%
\item[Description:]%
Modifies or adds property values. As many property name-value 
 pairs can be specified.
%%
\item[Parameters:]~
\begin{description}%
\item[\texttt{obj}:]
 Any object that has a props field.
\item[\texttt{attr}:]
 Property name
\item[\texttt{val}:]
 Property value.
\end{description}%
%
\item[Returns:]~

	obj: The new object with the updated properties.
%
%
\item[See also:]%
%
\item[Author:]%
Cengiz Gunay <cgunay@emory.edu>, 2004/11/22%
\end{description}
\methodline%
\subsubsection[Method \texttt{isnanrows}]{Method \texttt{tests\_db/isnanrows}}%
\index[funcref]{tests_db@\fidxlb{tests\_db}!isnanrows@\fidxl{isnanrows}}%
\label{ref_tests_db__isnanrows}%
\hypertarget{ref_tests_db__isnanrows}{}%
\begin{description}
\item[Summary:]Finds rows with any NaN values. Returns logical indices of db rows.
%
\item[Usage:]~%
\begin{lyxcode}%
rows = isnanrows(db)
%
\end{lyxcode}%
%
\item[Description:]%
Some operations need that no NaN values exist in the matrix. This method
 can be used to find and then remove NaN-contaminated rows from DB. Note
 that sometimes no rows can  be found, and some columns should be discarded
 before this operation.
%%
\item[Parameters:]~
\begin{description}%
\item[\texttt{db}:]
 A tests\_db object.
\end{description}%
%
\item[Returns:]~

	rows: A logical vector to be used in indexing db objects or passed
		through other logical operators. 
%
%
\item[See also:]%
\hyperlink{ref_isnan}{\texttt{isnan}}%
\ (p.~\pageref{ref_isnan})%
\index[funcref]{@\fidxl{isnan}}%
, \hyperlink{ref_tests_db}{\texttt{tests\_db}}%
\ (p.~\pageref{ref_tests_db})%
\index[funcref]{@\fidxl{tests\_db}}%
%
\item[Author:]%
Cengiz Gunay <cgunay@emory.edu>, 2004/11/08%
\end{description}
\methodline%
\subsubsection[Method \texttt{joinRows}]{Method \texttt{tests\_db/joinRows}}%
\index[funcref]{tests_db@\fidxlb{tests\_db}!joinRows@\fidxl{joinRows}}%
\label{ref_tests_db__joinRows}%
\hypertarget{ref_tests_db__joinRows}{}%
\begin{description}
\item[Summary:]Joins the rows of the given db with rows of with\_db with matching
  	RowIndex values.
%
\item[Usage:]~%
\begin{lyxcode}%
a\_db = joinRows(db, tests, with\_db, w\_tests, index\_col\_name)
%
\end{lyxcode}%
%
\item[Description:]%
Takes the desired columns in with\_db with rows having a 
 row index and joins them next to desired columns from the current db. 
 Assumes each row index only appears once in with\_db. The created
 db preserves the ordering of with\_db.
%%
\item[Parameters:]~
\begin{description}%
\item[\texttt{db}:]
 A param\_tests\_db object.
\item[\texttt{tests}:]
 Test columns to take from db.
\item[\texttt{with\_db}:]
 A tests\_db object with a RowIndex column.
\item[\texttt{w\_tests}:]
 Test columns to take from with\_db.
\item[\texttt{index\_col\_name}:]
 (Optional) Name of row index column (default='RowIndex').
\end{description}%
%
\item[Returns:]~

	a\_db: A tests\_db object.
%
%
\item[See also:]%
\hyperlink{ref_tests_db}{\texttt{tests\_db}}%
\ (p.~\pageref{ref_tests_db})%
\index[funcref]{@\fidxl{tests\_db}}%
%
\item[Author:]%
Cengiz Gunay <cgunay@emory.edu>, 2004/10/16%
\end{description}
\methodline%
\subsubsection[Method \texttt{setRows}]{Method \texttt{tests\_db/setRows}}%
\index[funcref]{tests_db@\fidxlb{tests\_db}!setRows@\fidxl{setRows}}%
\label{ref_tests_db__setRows}%
\hypertarget{ref_tests_db__setRows}{}%
\begin{description}
\item[Summary:]Sets the rows of observations in tests\_db.
%
\item[Usage:]~%
\begin{lyxcode}%
index = setRows(obj, rows)
%
\end{lyxcode}%
%
\item[Description:]%
Sets a new set of observations to the database and returns the new DB.
%%
\item[Parameters:]~
\begin{description}%
\item[\texttt{obj}:]
 A tests\_db object.
\item[\texttt{rows}:]
 A matrix that contains rows for the DB.
\end{description}%
%
\item[Returns:]~

	obj: The tests\_db object with the new rows.
%
%
\item[See also:]%
\hyperlink{ref_allocateRows}{\texttt{allocateRows}}%
\ (p.~\pageref{ref_allocateRows})%
\index[funcref]{@\fidxl{allocateRows}}%
, \hyperlink{ref_addRow}{\texttt{addRow}}%
\ (p.~\pageref{ref_addRow})%
\index[funcref]{@\fidxl{addRow}}%
, \hyperlink{ref_tests_db}{\texttt{tests\_db}}%
\ (p.~\pageref{ref_tests_db})%
\index[funcref]{@\fidxl{tests\_db}}%
%
\item[Author:]%
Cengiz Gunay <cgunay@emory.edu>, 2004/09/08%
\end{description}
\methodline%
\subsubsection[Method \texttt{plotTestsHistsMatrix}]{Method \texttt{tests\_db/plotTestsHistsMatrix}}%
\index[funcref]{tests_db@\fidxlb{tests\_db}!plotTestsHistsMatrix@\fidxl{plotTestsHistsMatrix}}%
\label{ref_tests_db__plotTestsHistsMatrix}%
\hypertarget{ref_tests_db__plotTestsHistsMatrix}{}%
\begin{description}
\item[Summary:]Create a matrix plot of test histograms.
%
\item[Usage:]~%
\begin{lyxcode}%
a\_pm = plotTestsHistsMatrix(a\_db, title\_str, props)
%
\end{lyxcode}%
%
\item[Description:]%
Skips the 'ItemIndex' test.
%%
\item[Parameters:]~
\begin{description}%
\item[\texttt{a\_db}:]
 A params\_tests\_db object.
\item[\texttt{title\_str}:]
 (Optional) A string to be concatanated to the title.
\item[\texttt{props}:]
 A structure with any optional properties, passed to plot\_abstract.
\begin{description}%
\item[\texttt{orient}:]
 Orientation of the plot\_stack. 'x', 'y', or 'matrix' (default).
\item[\texttt{histBins}:]
 Number of histogram bins.
\item[\texttt{quiet}:]
 Don't put the DB id on the title.
\item[\texttt{axisLimits}:]
 Only x-ranges are used from this expression.
\end{description}%
\end{description}%
%
\item[Returns:]~

	a\_pm: A plot\_stack with the plots organized in matrix form
%
%
\item[See also:]%
\hyperlink{ref_params_tests_profile}{\texttt{params\_tests\_profile}}%
\ (p.~\pageref{ref_params_tests_profile})%
\index[funcref]{@\fidxl{params\_tests\_profile}}%
, \hyperlink{ref_plotVar}{\texttt{plotVar}}%
\ (p.~\pageref{ref_plotVar})%
\index[funcref]{@\fidxl{plotVar}}%
%
\item[Author:]%
Cengiz Gunay <cgunay@emory.edu>, 2004/10/17%
\end{description}
\methodline%
\subsubsection[Method \texttt{crossProd}]{Method \texttt{tests\_db/crossProd}}%
\index[funcref]{tests_db@\fidxlb{tests\_db}!crossProd@\fidxl{crossProd}}%
\label{ref_tests_db__crossProd}%
\hypertarget{ref_tests_db__crossProd}{}%
\begin{description}
\item[Summary:]Create a DB by taking the cross product of two database row sets.
%
\item[Usage:]~%
\begin{lyxcode}%
cross\_db = crossProd(a\_db, b\_db)
%
\end{lyxcode}%
%
\item[Description:]%
This is not a vector cross product operation. Each row of the two DBs are matched 
 and added as a new row to a DB. The end is a DB with all combinations 
 of rows from both DBs. The final DB contains columns of both DBs.
%%
\item[Parameters:]~
\begin{description}%
\item[\texttt{a\_db, b\_db}:]
 A tests\_db object.
\end{description}%
%
\item[Returns:]~

	cross\_db: The tests\_db object with all combinations of rows.
%
%
\item[See also:]%
\hyperlink{ref_allocateRows}{\texttt{allocateRows}}%
\ (p.~\pageref{ref_allocateRows})%
\index[funcref]{@\fidxl{allocateRows}}%
, \hyperlink{ref_tests_db}{\texttt{tests\_db}}%
\ (p.~\pageref{ref_tests_db})%
\index[funcref]{@\fidxl{tests\_db}}%
%
\item[Author:]%
Cengiz Gunay <cgunay@emory.edu>, 2005/10/11%
\end{description}
\methodline%
\subsubsection[Method \texttt{display}]{Method \texttt{tests\_db/display}}%
\index[funcref]{tests_db@\fidxlb{tests\_db}!display@\fidxl{display}}%
\label{ref_tests_db__display}%
\hypertarget{ref_tests_db__display}{}%
\begin{description}
%
%
%
%
%
%
%
\item[Author:]%
Cengiz Gunay <cgunay@emory.edu>, 2004/08/04%
\end{description}
\methodline%
\subsubsection[Method \texttt{testsHists}]{Method \texttt{tests\_db/testsHists}}%
\index[funcref]{tests_db@\fidxlb{tests\_db}!testsHists@\fidxl{testsHists}}%
\label{ref_tests_db__testsHists}%
\hypertarget{ref_tests_db__testsHists}{}%
\begin{description}
\item[Summary:]Calculates histograms for all tests and returns them in a cell array.
%
\item[Usage:]~%
\begin{lyxcode}%
t\_hists = testsHists(a\_db, num\_bins)
%
\end{lyxcode}%
%
%
\item[Parameters:]~
\begin{description}%
\item[\texttt{a\_db}:]
 A tests\_db object.
\item[\texttt{num\_bins}:]
 Number of histogram bins (Optional, default=100), or

vector of histogram bin centers.\end{description}%
%
\item[Returns:]~

	t\_hists: An array of histograms for each test in a\_db.
%
%
\item[See also:]%
\hyperlink{ref_params_tests_profile}{\texttt{params\_tests\_profile}}%
\ (p.~\pageref{ref_params_tests_profile})%
\index[funcref]{@\fidxl{params\_tests\_profile}}%
%
\item[Author:]%
Cengiz Gunay <cgunay@emory.edu>, 2005/04/27%
\end{description}
\methodline%
\subsubsection[Method \texttt{mtimes}]{Method \texttt{tests\_db/mtimes}}%
\index[funcref]{tests_db@\fidxlb{tests\_db}!mtimes@\fidxl{mtimes}}%
\label{ref_tests_db__mtimes}%
\hypertarget{ref_tests_db__mtimes}{}%
\begin{description}
\item[Summary:]Multiplies the DB with a scalar.
%
\item[Usage:]~%
\begin{lyxcode}%
a\_db = mtimes(left\_obj, right\_obj)
%
\end{lyxcode}%
%
%
\item[Parameters:]~
\begin{description}%
\item[\texttt{left\_obj, right\_obj}:]
 Operands of the multiplication. One must be of type tests\_db.
\end{description}%
%
\item[Returns:]~

	a\_db: The resulting tests\_db.
%
%
\item[See also:]%
\hyperlink{ref_tests_db__times}{\texttt{tests\_db/times}}%
\ (p.~\pageref{ref_tests_db__times})%
\index[funcref]{tests_db@\fidxlb{tests\_db}!times@\fidxl{times}}%
, \hyperlink{ref_mtimes}{\texttt{mtimes}}%
\ (p.~\pageref{ref_mtimes})%
\index[funcref]{@\fidxl{mtimes}}%
%
\item[Author:]%
Cengiz Gunay <cgunay@emory.edu>, 2006/05/24%
\end{description}
\methodline%
\subsubsection[Method \texttt{histogram}]{Method \texttt{tests\_db/histogram}}%
\index[funcref]{tests_db@\fidxlb{tests\_db}!histogram@\fidxl{histogram}}%
\label{ref_tests_db__histogram}%
\hypertarget{ref_tests_db__histogram}{}%
\begin{description}
\item[Summary:]Generates a histogram\_db object with rows corresponding to 
		histogram entries.
%
\item[Usage:]~%
\begin{lyxcode}%
a\_histogram\_db = histogram(db, col, num\_bins)
%
\end{lyxcode}%
%
%
\item[Parameters:]~
\begin{description}%
\item[\texttt{db}:]
 A tests\_db object.
\item[\texttt{col}:]
 Column to find the histogram.
\item[\texttt{num\_bins}:]
 Number of histogram bins (Optional, default=100), or

vector of histogram bin centers.\end{description}%
%
\item[Returns:]~

	a\_histogram\_db: A histogram\_db object containing the histogram.
%
%
\item[See also:]%
\hyperlink{ref_histogram_db}{\texttt{histogram\_db}}%
\ (p.~\pageref{ref_histogram_db})%
\index[funcref]{@\fidxl{histogram\_db}}%
, \hyperlink{ref_tests_db}{\texttt{tests\_db}}%
\ (p.~\pageref{ref_tests_db})%
\index[funcref]{@\fidxl{tests\_db}}%
%
\item[Author:]%
Cengiz Gunay <cgunay@emory.edu>, 2004/09/17%
\end{description}
\methodline%
\subsubsection[Method \texttt{end}]{Method \texttt{tests\_db/end}}%
\index[funcref]{tests_db@\fidxlb{tests\_db}!end@\fidxl{end}}%
\label{ref_tests_db__end}%
\hypertarget{ref_tests_db__end}{}%
\begin{description}
\item[Summary:]Overloaded primitive matlab function, returns maximal dimension size.
%
\item[Usage:]~%
\begin{lyxcode}%
s = end(db, index, total)
%
\end{lyxcode}%
%
%
\item[Parameters:]~
\begin{description}%
\item[\texttt{db}:]
 A tests\_db object.
\end{description}%
%
\item[Returns:]~

	s: The size.
%
%
\item[See also:]%
\hyperlink{ref_size}{\texttt{size}}%
\ (p.~\pageref{ref_size})%
\index[funcref]{@\fidxl{size}}%
, \hyperlink{ref_tests_db}{\texttt{tests\_db}}%
\ (p.~\pageref{ref_tests_db})%
\index[funcref]{@\fidxl{tests\_db}}%
%
\item[Author:]%
Cengiz Gunay <cgunay@emory.edu>, 2004/10/06%
\end{description}
\methodline%
\subsubsection[Method \texttt{get}]{Method \texttt{tests\_db/get}}%
\index[funcref]{tests_db@\fidxlb{tests\_db}!get@\fidxl{get}}%
\label{ref_tests_db__get}%
\hypertarget{ref_tests_db__get}{}%
\begin{description}
\item[Summary:]Defines generic attribute retrieval for objects.
%
%
%
%
%
%
%
\item[Author:]%
Cengiz Gunay <cgunay@emory.edu>, 2004/09/14%
\end{description}
\methodline%
\subsubsection[Method \texttt{sortrows}]{Method \texttt{tests\_db/sortrows}}%
\index[funcref]{tests_db@\fidxlb{tests\_db}!sortrows@\fidxl{sortrows}}%
\label{ref_tests_db__sortrows}%
\hypertarget{ref_tests_db__sortrows}{}%
\begin{description}
\item[Summary:]Returns a sorted db according to given columns. 
%
\item[Usage:]~%
\begin{lyxcode}%
[sorted, idx] = sortrows(db, cols)
%
\end{lyxcode}%
%
\item[Description:]%
For multi-page dbs, sorts only the first page and applies the ordering 
 to all other pages.
%%
\item[Parameters:]~
\begin{description}%
\item[\texttt{db}:]
 A tests\_db object.
\item[\texttt{cols}:]
 Columns to use for sorting.
\end{description}%
%
\item[Returns:]~

	sorted: The sorted tests\_db.
	idx: The row index permutation vector. 
%
%
\item[See also:]%
\hyperlink{ref_sortrows}{\texttt{sortrows}}%
\ (p.~\pageref{ref_sortrows})%
\index[funcref]{@\fidxl{sortrows}}%
, \hyperlink{ref_tests_db}{\texttt{tests\_db}}%
\ (p.~\pageref{ref_tests_db})%
\index[funcref]{@\fidxl{tests\_db}}%
%
\item[Author:]%
Cengiz Gunay <cgunay@emory.edu>, 2004/10/11%
\end{description}
\methodline%
\subsubsection[Method \texttt{set}]{Method \texttt{tests\_db/set}}%
\index[funcref]{tests_db@\fidxlb{tests\_db}!set@\fidxl{set}}%
\label{ref_tests_db__set}%
\hypertarget{ref_tests_db__set}{}%
\begin{description}
\item[Summary:]Generic method for setting object attributes.
%
%
%
%
%
%
%
\item[Author:]%
Cengiz Gunay <cgunay@emory.edu>, 2004/10/08%
\end{description}
\methodline%
\subsubsection[Method \texttt{std}]{Method \texttt{tests\_db/std}}%
\index[funcref]{tests_db@\fidxlb{tests\_db}!std@\fidxl{std}}%
\label{ref_tests_db__std}%
\hypertarget{ref_tests_db__std}{}%
\begin{description}
\item[Summary:]Returns the std of the data matrix of a\_db. Ignores NaN values.
%
\item[Usage:]~%
\begin{lyxcode}%
[a\_db, n] = std(a\_db, sflag, dim)
%
\end{lyxcode}%
%
\item[Description:]%
Does a recursive operation over dimensions in order to remove NaN values.
 This takes considerable amount of time compared with a straightforward std
 operation. 
%%
\item[Parameters:]~
\begin{description}%
\item[\texttt{a\_db}:]
 A tests\_db object.
\item[\texttt{dim}:]
 Work down dimension.
\end{description}%
%
\item[Returns:]~

	a\_db: The DB with std values.
	n: (Optional) Numbers of non-NaN rows included in calculating each column.
%
%
\item[See also:]%
\hyperlink{ref_std}{\texttt{std}}%
\ (p.~\pageref{ref_std})%
\index[funcref]{@\fidxl{std}}%
, \hyperlink{ref_tests_db}{\texttt{tests\_db}}%
\ (p.~\pageref{ref_tests_db})%
\index[funcref]{@\fidxl{tests\_db}}%
%
\item[Author:]%
Cengiz Gunay <cgunay@emory.edu>, 2004/10/06%
\end{description}
\methodline%
\subsubsection[Method \texttt{sum}]{Method \texttt{tests\_db/sum}}%
\index[funcref]{tests_db@\fidxlb{tests\_db}!sum@\fidxl{sum}}%
\label{ref_tests_db__sum}%
\hypertarget{ref_tests_db__sum}{}%
\begin{description}
\item[Summary:]Creates a tests\_db by summing all rows.
%
\item[Usage:]~%
\begin{lyxcode}%
a\_db = sum(a\_db, props)
%
\end{lyxcode}%
%
\item[Description:]%
Applies the sum function to whole DB. The resulting DB will have one row.
%%
\item[Parameters:]~
\begin{description}%
\item[\texttt{a\_db}:]
 A tests\_db object.
\item[\texttt{props}:]
 Optional properties.
\end{description}%
%
\item[Returns:]~

	a\_db: The resulting tests\_db.
%
%
\item[See also:]%
\hyperlink{ref_sum}{\texttt{sum}}%
\ (p.~\pageref{ref_sum})%
\index[funcref]{@\fidxl{sum}}%
%
\item[Author:]%
Cengiz Gunay <cgunay@emory.edu>, 2006/05/24%
\end{description}
\methodline%
\subsubsection[Method \texttt{kmeansCluster}]{Method \texttt{tests\_db/kmeansCluster}}%
\index[funcref]{tests_db@\fidxlb{tests\_db}!kmeansCluster@\fidxl{kmeansCluster}}%
\label{ref_tests_db__kmeansCluster}%
\hypertarget{ref_tests_db__kmeansCluster}{}%
\begin{description}
\item[Summary:]Generates a database of cluster centers obtained from a k-means cluster analysis with the command kmeans.
%
\item[Usage:]~%
\begin{lyxcode}%
a\_cluster\_db = kmeansCluster(db, num\_clusters, props)
%
\end{lyxcode}%
%
%
\item[Parameters:]~
\begin{description}%
\item[\texttt{db}:]
 A tests\_db object.
\item[\texttt{num\_clusters}:]
 Number of clusters to form.
\item[\texttt{props}:]
 A structure with any optional properties.
\begin{description}%
\item[\texttt{DistanceMeasure}:]
 Choose one appropriate for kmeans.
\end{description}%
\end{description}%
%
\item[Returns:]~

	a\_cluster\_db: A tests\_db where each row is a cluster center.
	a\_hist\_db: histogram\_db showing cluster membership from original db.
	idx: Cluster indices of each row or original db.
	sum\_distances: Quality of clustering indicated by total distance from
			centroid for each cluster.
%
%
\item[See also:]%
\hyperlink{ref_tests_db}{\texttt{tests\_db}}%
\ (p.~\pageref{ref_tests_db})%
\index[funcref]{@\fidxl{tests\_db}}%
, \hyperlink{ref_histogram_db}{\texttt{histogram\_db}}%
\ (p.~\pageref{ref_histogram_db})%
\index[funcref]{@\fidxl{histogram\_db}}%
%
\item[Author:]%
Cengiz Gunay <cgunay@emory.edu>, 2005/04/06%
\end{description}
\methodline%
\subsubsection[Method \texttt{noNaNRows}]{Method \texttt{tests\_db/noNaNRows}}%
\index[funcref]{tests_db@\fidxlb{tests\_db}!noNaNRows@\fidxl{noNaNRows}}%
\label{ref_tests_db__noNaNRows}%
\hypertarget{ref_tests_db__noNaNRows}{}%
\begin{description}
\item[Summary:]Returns a DB by removing any NaN or Inf containing rows.
%
\item[Usage:]~%
\begin{lyxcode}%
a\_db = noNaNRows(a\_db)
%
\end{lyxcode}%
%
%
\item[Parameters:]~
\begin{description}%
\item[\texttt{a\_db}:]
 A tests\_db object.
\end{description}%
%
\item[Returns:]~

	a\_db: DB with rows with NaN values removed.
%
%
\item[See also:]%
\hyperlink{ref_tests_db__isnanrows}{\texttt{tests\_db/isnanrows}}%
\ (p.~\pageref{ref_tests_db__isnanrows})%
\index[funcref]{tests_db@\fidxlb{tests\_db}!isnanrows@\fidxl{isnanrows}}%
%
\item[Author:]%
Cengiz Gunay <cgunay@emory.edu>, 2005/09/21%
\end{description}
\methodline%
\subsubsection[Method \texttt{statsMeanStd}]{Method \texttt{tests\_db/statsMeanStd}}%
\index[funcref]{tests_db@\fidxlb{tests\_db}!statsMeanStd@\fidxl{statsMeanStd}}%
\label{ref_tests_db__statsMeanStd}%
\hypertarget{ref_tests_db__statsMeanStd}{}%
\begin{description}
\item[Summary:]Generates a stats\_db object with two rows corresponding to 
		the mean and std of the tests' distributions.
%
\item[Usage:]~%
\begin{lyxcode}%
a\_stats\_db = statsMeanStd(db, tests, props)
%
\end{lyxcode}%
%
%
\item[Parameters:]~
\begin{description}%
\item[\texttt{db}:]
 A tests\_db object.
\item[\texttt{tests}:]
 A selection of tests (see onlyRowsTests).
\item[\texttt{props}:]
 A structure with any optional properties for stats\_db.
\end{description}%
%
\item[Returns:]~

	a\_stats\_db: A stats\_db object.
%
%
\item[See also:]%
\hyperlink{ref_tests_db}{\texttt{tests\_db}}%
\ (p.~\pageref{ref_tests_db})%
\index[funcref]{@\fidxl{tests\_db}}%
%
\item[Author:]%
Cengiz Gunay <cgunay@emory.edu>, 2004/10/07%
\end{description}
\methodline%
\subsubsection[Method \texttt{rows2Struct}]{Method \texttt{tests\_db/rows2Struct}}%
\index[funcref]{tests_db@\fidxlb{tests\_db}!rows2Struct@\fidxl{rows2Struct}}%
\label{ref_tests_db__rows2Struct}%
\hypertarget{ref_tests_db__rows2Struct}{}%
\begin{description}
\item[Summary:]Convert given rows of database to a structure array.
%
\item[Usage:]~%
\begin{lyxcode}%
s = rows2Struct(db, rows, pages)
%
\end{lyxcode}%
%
%
\item[Parameters:]~
\begin{description}%
\item[\texttt{db}:]
 A tests\_db object.
\item[\texttt{rows}:]
 Indices of rows in db.
\item[\texttt{pages}:]
 Pages of db.
\end{description}%
%
\item[Returns:]~

	s: A structure of column name and value pairs.
%
%
\item[See also:]%
\hyperlink{ref_tests_db}{\texttt{tests\_db}}%
\ (p.~\pageref{ref_tests_db})%
\index[funcref]{@\fidxl{tests\_db}}%
%
\item[Author:]%
Cengiz Gunay <cgunay@emory.edu>, 2005/08/17%
\end{description}
\methodline%
\subsubsection[Method \texttt{getColNames}]{Method \texttt{tests\_db/getColNames}}%
\index[funcref]{tests_db@\fidxlb{tests\_db}!getColNames@\fidxl{getColNames}}%
\label{ref_tests_db__getColNames}%
\hypertarget{ref_tests_db__getColNames}{}%
\begin{description}
\item[Summary:]Gets column names.
%
\item[Usage:]~%
\begin{lyxcode}%
col\_names = getColNames(db, tests)
%
\end{lyxcode}%
%
\item[Description:]%
Performs a light operation without touching the data matrix.
%%
\item[Parameters:]~
\begin{description}%
\item[\texttt{db}:]
 A tests\_db object.
\item[\texttt{tests}:]
 Columns for which to get names (Optional, default = ':')
\end{description}%
%
\item[Returns:]~

	col\_names: A cell array of strings.
%
%
\item[See also:]%
\hyperlink{ref_getColNames}{\texttt{getColNames}}%
\ (p.~\pageref{ref_getColNames})%
\index[funcref]{@\fidxl{getColNames}}%
, \hyperlink{ref_tests_db}{\texttt{tests\_db}}%
\ (p.~\pageref{ref_tests_db})%
\index[funcref]{@\fidxl{tests\_db}}%
%
\item[Author:]%
Cengiz Gunay <cgunay@emory.edu>, 2006/05/24%
\end{description}
\methodline%
\subsubsection[Method \texttt{plotrow}]{Method \texttt{tests\_db/plotrow}}%
\index[funcref]{tests_db@\fidxlb{tests\_db}!plotrow@\fidxl{plotrow}}%
\label{ref_tests_db__plotrow}%
\hypertarget{ref_tests_db__plotrow}{}%
\begin{description}
\item[Summary:]Creates a plot\_abstract describing the desired db row.
%
\item[Usage:]~%
\begin{lyxcode}%
a\_plot = plotrow(a\_tests\_db, row)
%
\end{lyxcode}%
%
%
\item[Parameters:]~
\begin{description}%
\item[\texttt{a\_tests\_db}:]
 A tests\_db object.
\item[\texttt{row}:]
 Row number to visualize.
\item[\texttt{props}:]
 A structure with any optional properties.
\begin{description}%
\item[\texttt{putLabels}:]
 Put special column name labels.
\end{description}%
\end{description}%
%
\item[Returns:]~

	a\_plot: A plot\_abstract object that can be plotted.
%
%
\item[See also:]%
\hyperlink{ref_plot_abstract}{\texttt{plot\_abstract}}%
\ (p.~\pageref{ref_plot_abstract})%
\index[funcref]{@\fidxl{plot\_abstract}}%
, \hyperlink{ref_plotFigure}{\texttt{plotFigure}}%
\ (p.~\pageref{ref_plotFigure})%
\index[funcref]{@\fidxl{plotFigure}}%
%
\item[Author:]%
Cengiz Gunay <cgunay@emory.edu>, 2004/11/08%
\end{description}
\methodline%
\subsubsection[Method \texttt{dbsize}]{Method \texttt{tests\_db/dbsize}}%
\index[funcref]{tests_db@\fidxlb{tests\_db}!dbsize@\fidxl{dbsize}}%
\label{ref_tests_db__dbsize}%
\hypertarget{ref_tests_db__dbsize}{}%
\begin{description}
\item[Summary:]Returns the size of the data matrix of db.
%
\item[Usage:]~%
\begin{lyxcode}%
s = dbsize(db)
%
\end{lyxcode}%
%
%
\item[Parameters:]~
\begin{description}%
\item[\texttt{db}:]
 A tests\_db object.
\end{description}%
%
\item[Returns:]~

	s: The size values.
%
%
\item[See also:]%
\hyperlink{ref_size}{\texttt{size}}%
\ (p.~\pageref{ref_size})%
\index[funcref]{@\fidxl{size}}%
, \hyperlink{ref_tests_db}{\texttt{tests\_db}}%
\ (p.~\pageref{ref_tests_db})%
\index[funcref]{@\fidxl{tests\_db}}%
%
\item[Author:]%
Cengiz Gunay <cgunay@emory.edu>, 2004/10/06%
\end{description}
\methodline%
\subsubsection[Method \texttt{displayRowsTeX}]{Method \texttt{tests\_db/displayRowsTeX}}%
\index[funcref]{tests_db@\fidxlb{tests\_db}!displayRowsTeX@\fidxl{displayRowsTeX}}%
\label{ref_tests_db__displayRowsTeX}%
\hypertarget{ref_tests_db__displayRowsTeX}{}%
\begin{description}
\item[Summary:]Generates a LaTeX table that lists rows of this DB.
%
\item[Usage:]~%
\begin{lyxcode}%
tex\_string = displayRowsTeX(a\_db, caption, props)
%
\end{lyxcode}%
%
\item[Description:]%
By default table is rotated 90 degrees and scaled to 90% of page height.
%%
\item[Parameters:]~
\begin{description}%
\item[\texttt{a\_db}:]
 A tests\_db object.
\item[\texttt{caption}:]
 Table caption.
\item[\texttt{props}:]
 A structure with any optional properties, passed to TeXtable.
\end{description}%
%
\item[Returns:]~

	tex\_string: LaTeX string for table float.
%
%
\item[See also:]%
\hyperlink{ref_displayRows}{\texttt{displayRows}}%
\ (p.~\pageref{ref_displayRows})%
\index[funcref]{@\fidxl{displayRows}}%
, \hyperlink{ref_TeXtable}{\texttt{TeXtable}}%
\ (p.~\pageref{ref_TeXtable})%
\index[funcref]{@\fidxl{TeXtable}}%
, \hyperlink{ref_cell2TeX}{\texttt{cell2TeX}}%
\ (p.~\pageref{ref_cell2TeX})%
\index[funcref]{@\fidxl{cell2TeX}}%
%
\item[Author:]%
Cengiz Gunay <cgunay@emory.edu>, 2004/12/16%
\end{description}
\methodline%
\subsubsection[Method \texttt{matchingRow}]{Method \texttt{tests\_db/matchingRow}}%
\index[funcref]{tests_db@\fidxlb{tests\_db}!matchingRow@\fidxl{matchingRow}}%
\label{ref_tests_db__matchingRow}%
\hypertarget{ref_tests_db__matchingRow}{}%
\begin{description}
\item[Summary:]Creates a criterion database for matching the tests of a row.
%
\item[Usage:]~%
\begin{lyxcode}%
crit\_db = matchingRow(db, row, props)
%
\end{lyxcode}%
%
\item[Description:]%
Copies selected test values from row as the first row into the 
 criterion db. Adds a second row for the STD of each column in the db.
%%
\item[Parameters:]~
\begin{description}%
\item[\texttt{db}:]
 A tests\_db object.
\item[\texttt{row}:]
 A row index to match.
\item[\texttt{props}:]
 A structure with any optional properties.
\begin{description}%
\item[\texttt{std\_db}:]
 Take the standard deviation from this db instead.
\end{description}%
\end{description}%
%
\item[Returns:]~

	crit\_db: A tests\_db with two rows for values and STDs.
%
\item[Example:]~
\begin{lyxcode}        >> crit\_db = matchingRow(phys\_control\_compare\_db, \\%
                find(phys\_control\_compare\_db(:, 'TracesetIndex') == 61))\\%
\end{lyxcode}
%
\item[See also:]%
\hyperlink{ref_rankMatching}{\texttt{rankMatching}}%
\ (p.~\pageref{ref_rankMatching})%
\index[funcref]{@\fidxl{rankMatching}}%
, \hyperlink{ref_tests_db}{\texttt{tests\_db}}%
\ (p.~\pageref{ref_tests_db})%
\index[funcref]{@\fidxl{tests\_db}}%
, \hyperlink{ref_tests2cols}{\texttt{tests2cols}}%
\ (p.~\pageref{ref_tests2cols})%
\index[funcref]{@\fidxl{tests2cols}}%
%
\item[Author:]%
Cengiz Gunay <cgunay@emory.edu>, 2004/12/08%
\end{description}
\methodline%
\subsubsection[Method \texttt{invarValues}]{Method \texttt{tests\_db/invarValues}}%
\index[funcref]{tests_db@\fidxlb{tests\_db}!invarValues@\fidxl{invarValues}}%
\label{ref_tests_db__invarValues}%
\hypertarget{ref_tests_db__invarValues}{}%
\begin{description}
\item[Summary:]Generates a 3D database of invariant values of given columns.
%
\item[Usage:]~%
\begin{lyxcode}%
a\_tests\_3D\_db = invarValues(db, cols, main\_cols)
%
\end{lyxcode}%
%
\item[Description:]%
The invariant values of a column are its values when all other 
 column values are fixed. The invariant values of desired columns
 forms a matrix of rows. This function finds all combinations of the
 rest of the columns and returns the invariant value matrices 
 for each such combination in a page of a three-dimensional vector; 
 i.e. a tests\_3D\_db. Each matrix page will contain an additional 
 column for the original row index for the invariant values. This
 index can be used to find the test columns that were omitted.
 Note: the trial column will be ignored for finding invariant values.
%%
\item[Parameters:]~
\begin{description}%
\item[\texttt{db}:]
 A tests\_db object.
\item[\texttt{cols}:]
 Vector of column numbers to find invariant values.
\item[\texttt{main\_cols}:]
 Vector of columns that need to be unique in each page 

(Optional; used only if database is not symmetric, to ignore 
missing values of main\_cols)\end{description}%
%
\item[Returns:]~

	a\_tests\_3D\_db: A tests\_3D\_db object of organized values.
%
%
\item[See also:]%
\hyperlink{ref_tests_3D_db}{\texttt{tests\_3D\_db}}%
\ (p.~\pageref{ref_tests_3D_db})%
\index[funcref]{@\fidxl{tests\_3D\_db}}%
, \hyperlink{ref_tests_3D_db__corrCoefs}{\texttt{tests\_3D\_db/corrCoefs}}%
\ (p.~\pageref{ref_tests_3D_db__corrCoefs})%
\index[funcref]{tests_3D_db@\fidxlb{tests\_3D\_db}!corrCoefs@\fidxl{corrCoefs}}%
, \hyperlink{ref_tests_3D_db__plotPair}{\texttt{tests\_3D\_db/plotPair}}%
\ (p.~\pageref{ref_tests_3D_db__plotPair})%
\index[funcref]{tests_3D_db@\fidxlb{tests\_3D\_db}!plotPair@\fidxl{plotPair}}%
%
\item[Author:]%
Cengiz Gunay <cgunay@emory.edu>, 2004/09/30%
\end{description}
\methodline%
\subsubsection[Method \texttt{meanDuplicateRows}]{Method \texttt{tests\_db/meanDuplicateRows}}%
\index[funcref]{tests_db@\fidxlb{tests\_db}!meanDuplicateRows@\fidxl{meanDuplicateRows}}%
\label{ref_tests_db__meanDuplicateRows}%
\hypertarget{ref_tests_db__meanDuplicateRows}{}%
\begin{description}
\item[Summary:]Row-reduces a db by finding sets of rows with same main\_cols values, and replacing each set with a single row containing main\_cols and the mean of rest\_cols.
%
\item[Usage:]~%
\begin{lyxcode}%
a\_tests\_db = meanDuplicateRows(db, main\_cols, rest\_cols)
%
\end{lyxcode}%
%
\item[Description:]%
The database is sorted for the values of the columns of 
 interest (main\_cols) and all rows with duplicate values of 
 these columns are identified. The rest of the columns (rest\_cols) 
 are averaged and reduced to a single row, and attached to the
 nominal values of main\_cols. Two additional parameter columns will be added to the
 database created. The NumDuplicates column is the the number of duplicates 
 used in the mean operation, and RowIndex is the row number points 
 to the first of a set of duplicate values.
%%
\item[Parameters:]~
\begin{description}%
\item[\texttt{db}:]
 A tests\_db object.
\item[\texttt{main\_cols}:]
 Vector of columns in which to find duplicates.
\item[\texttt{rest\_cols}:]
 Vector of columns to be averaged for duplicate main\_cols.
\end{description}%
%
\item[Returns:]~

	a\_tests\_db: The db object of with the means on page 1 
		    and standard deviations on page 2.
%
%
\item[See also:]%
\hyperlink{ref_tests_db__mean}{\texttt{tests\_db/mean}}%
\ (p.~\pageref{ref_tests_db__mean})%
\index[funcref]{tests_db@\fidxlb{tests\_db}!mean@\fidxl{mean}}%
, \hyperlink{ref_tests_db__std}{\texttt{tests\_db/std}}%
\ (p.~\pageref{ref_tests_db__std})%
\index[funcref]{tests_db@\fidxlb{tests\_db}!std@\fidxl{std}}%
, \hyperlink{ref_sortedUniqueValues}{\texttt{sortedUniqueValues}}%
\ (p.~\pageref{ref_sortedUniqueValues})%
\index[funcref]{@\fidxl{sortedUniqueValues}}%
%
\item[Author:]%
Cengiz Gunay <cgunay@emory.edu>, 2004/09/30%
\end{description}
\methodline%
\subsubsection[Method \texttt{displayRows}]{Method \texttt{tests\_db/displayRows}}%
\index[funcref]{tests_db@\fidxlb{tests\_db}!displayRows@\fidxl{displayRows}}%
\label{ref_tests_db__displayRows}%
\hypertarget{ref_tests_db__displayRows}{}%
\begin{description}
\item[Summary:]Displays rows of data with associated column labels.
%
\item[Usage:]~%
\begin{lyxcode}%
s = displayRows(db, rows, pages)
%
\end{lyxcode}%
%
%
\item[Parameters:]~
\begin{description}%
\item[\texttt{db}:]
 A tests\_db object.
\item[\texttt{rows}:]
 Indices of rows in db.
\item[\texttt{pages}:]
 Pages of db.
\end{description}%
%
\item[Returns:]~

	s: A cell array of trasposed database contents, prefixed with 
	   column names on each row. Meant to be displayed on the screen.
%
%
\item[See also:]%
\hyperlink{ref_tests_db}{\texttt{tests\_db}}%
\ (p.~\pageref{ref_tests_db})%
\index[funcref]{@\fidxl{tests\_db}}%
%
\item[Author:]%
Cengiz Gunay <cgunay@emory.edu>, 2004/09/15%
\end{description}
\methodline%
\subsubsection[Method \texttt{times}]{Method \texttt{tests\_db/times}}%
\index[funcref]{tests_db@\fidxlb{tests\_db}!times@\fidxl{times}}%
\label{ref_tests_db__times}%
\hypertarget{ref_tests_db__times}{}%
\begin{description}
\item[Summary:]Multiplies the DB with a scalar.
%
\item[Usage:]~%
\begin{lyxcode}%
a\_db = times(left\_obj, right\_obj)
%
\end{lyxcode}%
%
%
\item[Parameters:]~
\begin{description}%
\item[\texttt{left\_obj, right\_obj}:]
 Operands of the multiplication. One must be of type tests\_db.
\end{description}%
%
\item[Returns:]~

	a\_db: The resulting tests\_db.
%
%
\item[See also:]%
\hyperlink{ref_times}{\texttt{times}}%
\ (p.~\pageref{ref_times})%
\index[funcref]{@\fidxl{times}}%
%
\item[Author:]%
Cengiz Gunay <cgunay@emory.edu>, 2006/05/24%
\end{description}
\methodline%
\subsubsection[Method \texttt{addLastRow}]{Method \texttt{tests\_db/addLastRow}}%
\index[funcref]{tests_db@\fidxlb{tests\_db}!addLastRow@\fidxl{addLastRow}}%
\label{ref_tests_db__addLastRow}%
\hypertarget{ref_tests_db__addLastRow}{}%
\begin{description}
\item[Summary:]Inserts a row of observations at the end of tests\_db.
%
\item[Usage:]~%
\begin{lyxcode}%
index = addLastRow(obj, row)
%
\end{lyxcode}%
%
\item[Description:]%
Adds a new set of observations to the database and returns its row index.
   This operation is expensive because the whole 
   database matrix needs to be duplicated and resized in order to add a 
   single new row. The method of allocating a matrix, filling it up, and
   then providing it to the tests\_db constructor is the preferred method 
   of creating tests\_db objects.
%%
\item[Parameters:]~
\begin{description}%
\item[\texttt{obj}:]
 A tests\_db object.
\item[\texttt{row}:]
 A row vector that contains values for each DB column.
\end{description}%
%
\item[Returns:]~

	obj: The tests\_db object that includes the new row.
%
%
\item[See also:]%
\hyperlink{ref_allocateRows}{\texttt{allocateRows}}%
\ (p.~\pageref{ref_allocateRows})%
\index[funcref]{@\fidxl{allocateRows}}%
, \hyperlink{ref_addRow}{\texttt{addRow}}%
\ (p.~\pageref{ref_addRow})%
\index[funcref]{@\fidxl{addRow}}%
, \hyperlink{ref_tests_db}{\texttt{tests\_db}}%
\ (p.~\pageref{ref_tests_db})%
\index[funcref]{@\fidxl{tests\_db}}%
%
\item[Author:]%
Cengiz Gunay <cgunay@emory.edu>, 2004/09/08%
\end{description}
\methodline%
\subsubsection[Method \texttt{diff}]{Method \texttt{tests\_db/diff}}%
\index[funcref]{tests_db@\fidxlb{tests\_db}!diff@\fidxl{diff}}%
\label{ref_tests_db__diff}%
\hypertarget{ref_tests_db__diff}{}%
\begin{description}
\item[Summary:]Creates a tests\_db by taking the derivative of all tests.
%
\item[Usage:]~%
\begin{lyxcode}%
a\_db = diff(a\_db, props)
%
\end{lyxcode}%
%
\item[Description:]%
Applies the diff function to whole DB. The resulting DB will have one less row.
%%
\item[Parameters:]~
\begin{description}%
\item[\texttt{a\_db}:]
 A tests\_db object.
\item[\texttt{props}:]
 Optional properties.
\end{description}%
%
\item[Returns:]~

	a\_db: The resulting tests\_db.
%
%
\item[See also:]%
\hyperlink{ref_diff}{\texttt{diff}}%
\ (p.~\pageref{ref_diff})%
\index[funcref]{@\fidxl{diff}}%
, \hyperlink{ref_tests_3D_db__getDiff2DDB}{\texttt{tests\_3D\_db/getDiff2DDB}}%
\ (p.~\pageref{ref_tests_3D_db__getDiff2DDB})%
\index[funcref]{tests_3D_db@\fidxlb{tests\_3D\_db}!getDiff2DDB@\fidxl{getDiff2DDB}}%
%
\item[Author:]%
Cengiz Gunay <cgunay@emory.edu>, 2006/05/24%
\end{description}
\methodline%
\subsubsection[Method \texttt{plot\_abstract}]{Method \texttt{tests\_db/plot\_abstract}}%
\index[funcref]{tests_db@\fidxlb{tests\_db}!plot_abstract@\fidxl{plot\_abstract}}%
\label{ref_tests_db__plot_abstract}%
\hypertarget{ref_tests_db__plot_abstract}{}%
\begin{description}
\item[Summary:]Default visualization for a database.
%
\item[Usage:]~%
\begin{lyxcode}%
a\_pm = plot\_abstract(a\_db, title\_str)
%
\end{lyxcode}%
%
\item[Description:]%
Calls plotTestsHistsMatrix. Subclasses should override this method
 to provide their own visualization.
%%
\item[Parameters:]~
\begin{description}%
\item[\texttt{a\_db}:]
 A params\_tests\_db object.
\item[\texttt{title\_str}:]
 (Optional) A string to be concatanated to the title.
\end{description}%
%
\item[Returns:]~

	a\_pm: A plot\_stack with the plots organized in matrix form
%
\item[Example:]~
\begin{lyxcode}   >> plot(my\_db, ': first impression')\\%
 will call this function and send the generated plot to the plotFigure function.\\%
\end{lyxcode}
%
\item[See also:]%
\hyperlink{ref_plot_abstract__plot_abstract}{\texttt{plot\_abstract/plot\_abstract}}%
\ (p.~\pageref{ref_plot_abstract__plot_abstract})%
\index[funcref]{plot_abstract@\fidxlb{plot\_abstract}!plot_abstract@\fidxl{plot\_abstract}}%
, \hyperlink{ref_plotTestsHistsMatrix}{\texttt{plotTestsHistsMatrix}}%
\ (p.~\pageref{ref_plotTestsHistsMatrix})%
\index[funcref]{@\fidxl{plotTestsHistsMatrix}}%
, \hyperlink{ref_plotFigure}{\texttt{plotFigure}}%
\ (p.~\pageref{ref_plotFigure})%
\index[funcref]{@\fidxl{plotFigure}}%
%
\item[Author:]%
Cengiz Gunay <cgunay@emory.edu>, 2005/08/17%
\end{description}
\methodline%
\subsubsection[Method \texttt{addRow}]{Method \texttt{tests\_db/addRow}}%
\index[funcref]{tests_db@\fidxlb{tests\_db}!addRow@\fidxl{addRow}}%
\label{ref_tests_db__addRow}%
\hypertarget{ref_tests_db__addRow}{}%
\begin{description}
\item[Summary:]Inserts a row of observations to tests\_db at the given row index.
%
\item[Usage:]~%
\begin{lyxcode}%
index = addRow(obj, row, index)
%
\end{lyxcode}%
%
\item[Description:]%
Adds a new set of observations to the database and returns the new DB.
   This operation is expensive in the sense that the whole database matrix
   needs to be copied to be passed to this function just to add a 
   single new row. The method of allocating a matrix, filling it up, and
   then providing it to the tests\_db constructor is the preferred method 
   of creating tests\_db objects.
%%
\item[Parameters:]~
\begin{description}%
\item[\texttt{obj}:]
 A tests\_db object.
\item[\texttt{row}:]
 A row vector that contains values for each DB column.
\item[\texttt{index}:]
 The row index.
\end{description}%
%
\item[Returns:]~

	obj: The tests\_db object that includes the new row.
%
%
\item[See also:]%
\hyperlink{ref_addLastRow}{\texttt{addLastRow}}%
\ (p.~\pageref{ref_addLastRow})%
\index[funcref]{@\fidxl{addLastRow}}%
, \hyperlink{ref_allocateRows}{\texttt{allocateRows}}%
\ (p.~\pageref{ref_allocateRows})%
\index[funcref]{@\fidxl{allocateRows}}%
, \hyperlink{ref_tests_db}{\texttt{tests\_db}}%
\ (p.~\pageref{ref_tests_db})%
\index[funcref]{@\fidxl{tests\_db}}%
%
\item[Author:]%
Cengiz Gunay <cgunay@emory.edu>, 2004/09/08%
\end{description}
\methodline%
\subsubsection[Method \texttt{subsref}]{Method \texttt{tests\_db/subsref}}%
\index[funcref]{tests_db@\fidxlb{tests\_db}!subsref@\fidxl{subsref}}%
\label{ref_tests_db__subsref}%
\hypertarget{ref_tests_db__subsref}{}%
\begin{description}
\item[Summary:]Defines indexing for tests\_db objects for () and . operations. 
%
\item[Usage:]~%
\begin{lyxcode}%
obj = obj(rows, tests)
 obj = obj.attribute
%
\end{lyxcode}%
%
\item[Description:]%
Returns attributes or selects the given test columns and rows
 and returns in a new tests\_db object.
%%
\item[Parameters:]~
\begin{description}%
\item[\texttt{obj}:]
 A tests\_db object.
\item[\texttt{rows}:]
 A logical or index vector of rows. If ':', all rows.
\item[\texttt{tests}:]
 Cell array of test names or column indices. If ':', all tests.
\item[\texttt{attribute}:]
 A tests\_db class attribute.
\end{description}%
%
\item[Returns:]~

	obj: The new tests\_db object.
%
%
\item[See also:]%
\hyperlink{ref_subsref}{\texttt{subsref}}%
\ (p.~\pageref{ref_subsref})%
\index[funcref]{@\fidxl{subsref}}%
, \hyperlink{ref_tests_db}{\texttt{tests\_db}}%
\ (p.~\pageref{ref_tests_db})%
\index[funcref]{@\fidxl{tests\_db}}%
%
\item[Author:]%
Cengiz Gunay <cgunay@emory.edu>, 2004/09/17%
\end{description}
\methodline%
\subsubsection[Method \texttt{shufflerows}]{Method \texttt{tests\_db/shufflerows}}%
\index[funcref]{tests_db@\fidxlb{tests\_db}!shufflerows@\fidxl{shufflerows}}%
\label{ref_tests_db__shufflerows}%
\hypertarget{ref_tests_db__shufflerows}{}%
\begin{description}
\item[Summary:]Returns a db with rows of given test columns are shuffled. 
%
\item[Usage:]~%
\begin{lyxcode}%
s = shufflerows(db, dim)
%
\end{lyxcode}%
%
\item[Description:]%
Can be used for shuffle prediction. Basically, shuffle rows of tests and rerun
 high order analyses. 
%%
\item[Parameters:]~
\begin{description}%
\item[\texttt{db}:]
 A tests\_db object.
\item[\texttt{tests}:]
 Tests to shuffle.
\item[\texttt{grouped}:]
 If 1 then shuffle tests all together, 

if 0 shuffle each test separately.\end{description}%
%
\item[Returns:]~

	a\_db: The shuffled db.
%
%
\item[See also:]%
\hyperlink{ref_tests_db}{\texttt{tests\_db}}%
\ (p.~\pageref{ref_tests_db})%
\index[funcref]{@\fidxl{tests\_db}}%
%
\item[Author:]%
Cengiz Gunay <cgunay@emory.edu>, 2004/11/10%
\end{description}
\methodline%
\subsubsection[Method \texttt{renameColumns}]{Method \texttt{tests\_db/renameColumns}}%
\index[funcref]{tests_db@\fidxlb{tests\_db}!renameColumns@\fidxl{renameColumns}}%
\label{ref_tests_db__renameColumns}%
\hypertarget{ref_tests_db__renameColumns}{}%
\begin{description}
\item[Summary:]Rename an existing column or columns.
%
\item[Usage:]~%
\begin{lyxcode}%
a\_db = renameColumns(a\_db, test\_names, new\_names)
%
\end{lyxcode}%
%
\item[Description:]%
This is a cheap operation than modifies meta-data kept in object.
%%
\item[Parameters:]~
\begin{description}%
\item[\texttt{a\_db}:]
 A tests\_db object.
\item[\texttt{test\_names}:]
 A cell array of existing test names.
\item[\texttt{new\_names}:]
 New names to replace existing ones.
\end{description}%
%
\item[Returns:]~

	a\_db: The tests\_db object that includes the new columns.
%
\item[Example:]~
\begin{lyxcode} % Renaming a single column:\\%
 >> new\_db = renameColumns(a\_db, 'PulseIni100msSpikeRateISI\_D40pA', 'Firing\_rate');\\%
 % Renaming multiple columns:\\%
 >> new\_db = renameColumns(a\_db, {'a', 'b'}, {'c', 'd'});\\%
\end{lyxcode}
%
\item[See also:]%
\hyperlink{ref_allocateRows}{\texttt{allocateRows}}%
\ (p.~\pageref{ref_allocateRows})%
\index[funcref]{@\fidxl{allocateRows}}%
, \hyperlink{ref_tests_db}{\texttt{tests\_db}}%
\ (p.~\pageref{ref_tests_db})%
\index[funcref]{@\fidxl{tests\_db}}%
%
\item[Author:]%
Cengiz Gunay <cgunay@emory.edu>, 2005/09/30%
\end{description}
\methodline%
\subsubsection[Method \texttt{mean}]{Method \texttt{tests\_db/mean}}%
\index[funcref]{tests_db@\fidxlb{tests\_db}!mean@\fidxl{mean}}%
\label{ref_tests_db__mean}%
\hypertarget{ref_tests_db__mean}{}%
\begin{description}
\item[Summary:]Returns the mean of the data matrix of a\_db. Ignores NaN values.
%
\item[Usage:]~%
\begin{lyxcode}%
[a\_db, n] = mean(a\_db, dim)
%
\end{lyxcode}%
%
\item[Description:]%
Does a recursive operation over dimensions in order to remove NaN values.
 This takes more time, compared with a straightforward mean operation. 
%%
\item[Parameters:]~
\begin{description}%
\item[\texttt{a\_db}:]
 A tests\_db object.
\item[\texttt{dim}:]
 Work down dimension.
\end{description}%
%
\item[Returns:]~

	a\_db: The DB with one row of mean values.
	n: (Optional) Numbers of non-NaN rows included in calculating each column.
%
%
\item[See also:]%
\hyperlink{ref_mean}{\texttt{mean}}%
\ (p.~\pageref{ref_mean})%
\index[funcref]{@\fidxl{mean}}%
, \hyperlink{ref_tests_db}{\texttt{tests\_db}}%
\ (p.~\pageref{ref_tests_db})%
\index[funcref]{@\fidxl{tests\_db}}%
%
\item[Author:]%
Cengiz Gunay <cgunay@emory.edu>, 2004/10/06%
\end{description}
\methodline%
\subsubsection[Method \texttt{plot}]{Method \texttt{tests\_db/plot}}%
\index[funcref]{tests_db@\fidxlb{tests\_db}!plot@\fidxl{plot}}%
\label{ref_tests_db__plot}%
\hypertarget{ref_tests_db__plot}{}%
\begin{description}
\item[Summary:]Generic method to plot a tests\_db or a subclass. Requires a 
	plot\_abstract method to be defined for this object.
%
\item[Usage:]~%
\begin{lyxcode}%
h = plot(a\_tests\_db, title\_str)
%
\end{lyxcode}%
%
%
\item[Parameters:]~
\begin{description}%
\item[\texttt{a\_tests\_db}:]
 A histogram\_db object.
\item[\texttt{title\_str}:]
 (Optional) String to append to plot title.
\end{description}%
%
\item[Returns:]~

	h: The figure handle created.
%
%
\item[See also:]%
\hyperlink{ref_plot_abstract}{\texttt{plot\_abstract}}%
\ (p.~\pageref{ref_plot_abstract})%
\index[funcref]{@\fidxl{plot\_abstract}}%
, \hyperlink{ref_plotFigure}{\texttt{plotFigure}}%
\ (p.~\pageref{ref_plotFigure})%
\index[funcref]{@\fidxl{plotFigure}}%
%
\item[Author:]%
Cengiz Gunay <cgunay@emory.edu>, 2004/10/06%
\end{description}
\methodline%
\subsubsection[Method \texttt{subsasgn}]{Method \texttt{tests\_db/subsasgn}}%
\index[funcref]{tests_db@\fidxlb{tests\_db}!subsasgn@\fidxl{subsasgn}}%
\label{ref_tests_db__subsasgn}%
\hypertarget{ref_tests_db__subsasgn}{}%
\begin{description}
\item[Summary:]Defines generic index-based assignment for objects.
%
%
%
%
%
%
%
\item[Author:]%
Cengiz Gunay <cgunay@emory.edu>, 2006/02/06%
\end{description}
\methodline%
\subsubsection[Method \texttt{princomp}]{Method \texttt{tests\_db/princomp}}%
\index[funcref]{tests_db@\fidxlb{tests\_db}!princomp@\fidxl{princomp}}%
\label{ref_tests_db__princomp}%
\hypertarget{ref_tests_db__princomp}{}%
\begin{description}
\item[Summary:]Generates a database of the principal components of given DB.
%
\item[Usage:]~%
\begin{lyxcode}%
a\_pca\_db = princomp(db, props)
%
\end{lyxcode}%
%
%
\item[Parameters:]~
\begin{description}%
\item[\texttt{db}:]
 A tests\_db object.
\item[\texttt{props}:]
 A structure with any optional properties.
\begin{description}%
\item[\texttt{normalized}:]
 If specified zscore is used before princomp.
\end{description}%
\end{description}%
%
\item[Returns:]~

	a\_pca\_db: A tests\_db where each row is a principal component.
%
%
\item[See also:]%
\hyperlink{ref_princomp}{\texttt{princomp}}%
\ (p.~\pageref{ref_princomp})%
\index[funcref]{@\fidxl{princomp}}%
, \hyperlink{ref_zscore}{\texttt{zscore}}%
\ (p.~\pageref{ref_zscore})%
\index[funcref]{@\fidxl{zscore}}%
%
\item[Author:]%
Cengiz Gunay <cgunay@emory.edu>, 2005/09/21%
\end{description}
\methodline%
\subsubsection[Method \texttt{allocateRows}]{Method \texttt{tests\_db/allocateRows}}%
\index[funcref]{tests_db@\fidxlb{tests\_db}!allocateRows@\fidxl{allocateRows}}%
\label{ref_tests_db__allocateRows}%
\hypertarget{ref_tests_db__allocateRows}{}%
\begin{description}
\item[Summary:]Preallocates a NaN-filled num\_rows rows in tests\_db.
%
\item[Usage:]~%
\begin{lyxcode}%
obj = allocateRows(obj, num\_rows)
%
\end{lyxcode}%
%
\item[Description:]%
Allocates the desired number of rows to speed up filling up the data matrix 
   using assignRowsTests. Using addRow after this operation is still expensive.
   The method of allocating a matrix, filling it up, and then providing it to 
   the tests\_db constructor is the preferred method of creating tests\_db objects.
%%
\item[Parameters:]~
\begin{description}%
\item[\texttt{obj}:]
 A tests\_db object.
\item[\texttt{num\_rows}:]
 The predicted number of observations for this tests\_db.
\end{description}%
%
\item[Returns:]~

	obj: The new tests\_db object.
%
%
\item[See also:]%
\hyperlink{ref_assignRowsTests}{\texttt{assignRowsTests}}%
\ (p.~\pageref{ref_assignRowsTests})%
\index[funcref]{@\fidxl{assignRowsTests}}%
, \hyperlink{ref_addRow}{\texttt{addRow}}%
\ (p.~\pageref{ref_addRow})%
\index[funcref]{@\fidxl{addRow}}%
, \hyperlink{ref_setRows}{\texttt{setRows}}%
\ (p.~\pageref{ref_setRows})%
\index[funcref]{@\fidxl{setRows}}%
, \hyperlink{ref_tests_db}{\texttt{tests\_db}}%
\ (p.~\pageref{ref_tests_db})%
\index[funcref]{@\fidxl{tests\_db}}%
%
\item[Author:]%
Cengiz Gunay <cgunay@emory.edu>, 2004/09/08%
\end{description}
\methodline%
\subsubsection[Method \texttt{compareRows}]{Method \texttt{tests\_db/compareRows}}%
\index[funcref]{tests_db@\fidxlb{tests\_db}!compareRows@\fidxl{compareRows}}%
\label{ref_tests_db__compareRows}%
\hypertarget{ref_tests_db__compareRows}{}%
\begin{description}
\item[Summary:]Returns comparison results of db and the given row in a column vector.
%
\item[Usage:]~%
\begin{lyxcode}%
rows = compareRows(db, row)
%
\end{lyxcode}%
%
\item[Description:]%
If the row argument has multiple rows, the comparison is done separately
 for each of its rows and the results are the logical OR of those.
 Note that, it uses summation of distance for magnitude comparison. 
 That is, all columns have the same weight.
%%
\item[Parameters:]~
\begin{description}%
\item[\texttt{db}:]
 A tests\_db object.
\item[\texttt{row}:]
 Row array, matrix or database to be compared with db rows.
\end{description}%
%
\item[Returns:]~

	rows: A column vector of comparison results. 
		(<0: db < row, 0: db == row, >0: db > row)
%
%
\item[See also:]%
\hyperlink{ref_eq}{\texttt{eq}}%
\ (p.~\pageref{ref_eq})%
\index[funcref]{@\fidxl{eq}}%
, \hyperlink{ref_tests_db}{\texttt{tests\_db}}%
\ (p.~\pageref{ref_tests_db})%
\index[funcref]{@\fidxl{tests\_db}}%
%
\item[Author:]%
Cengiz Gunay <cgunay@emory.edu>, 2004/09/17%
\end{description}
\methodline%
\subsubsection[Method \texttt{plotYTests}]{Method \texttt{tests\_db/plotYTests}}%
\index[funcref]{tests_db@\fidxlb{tests\_db}!plotYTests@\fidxl{plotYTests}}%
\label{ref_tests_db__plotYTests}%
\hypertarget{ref_tests_db__plotYTests}{}%
\begin{description}
\item[Summary:]Create a plot given database measures against given X-axis values, for each row.
%
\item[Usage:]~%
\begin{lyxcode}%
a\_p = plotYTests(a\_db, x\_vals, tests, axis\_labels, title\_str, short\_title, command, props)
%
\end{lyxcode}%
%
%
\item[Parameters:]~
\begin{description}%
\item[\texttt{a\_db}:]
 A params\_tests\_db object.
\item[\texttt{x\_vals}:]
 A vector of X-axis values.
\item[\texttt{tests}:]
 A vector or cell array of columns to correspond to each value from x\_vals.
\item[\texttt{title\_str}:]
 (Optional) A string to be concatanated to the title.
\item[\texttt{short\_title}:]
 (Optional) Few words that may appear in legends of multiplot.
\item[\texttt{command}:]
 (Optional) Command to do the plotting with (default: 'plot')
\item[\texttt{props}:]
 A structure with any optional properties.
\begin{description}%
\item[\texttt{LineStyle}:]
 Plot line style to use. (default: 'd-')
\item[\texttt{ShowErrorbars}:]
 If 1, errorbars are added to each point.
\item[\texttt{StatsDB}:]
 If given, use this stats\_db for the errorbar (default=statsMeanStd(a\_db)).
\item[\texttt{quiet}:]
 If 1, don't include database name on title.
\end{description}%
\end{description}%
%
\item[Returns:]~

	a\_p: A plot\_abstract.
%
\item[Example:]~
\begin{lyxcode} >> a\_p = plotYTests(a\_db\_row, [0 40 100 200], ...\\%
                      {'IniSpontSpikeRateISI\_0pA', 'PulseIni100msSpikeRateISI\_D40pA', ...\\%
                       'PulseIni100msSpikeRateISI\_D100pA', 'PulseIni100msSpikeRateISI\_D200pA'}, ...\\%
                      {'current pulse [pA]', 'firing rate [Hz]'}, ', f-I curves', 'neuron 1');\\%
 >> plotFigure(a\_p);\\%
\end{lyxcode}
%
\item[See also:]%
\hyperlink{ref_plotFigure}{\texttt{plotFigure}}%
\ (p.~\pageref{ref_plotFigure})%
\index[funcref]{@\fidxl{plotFigure}}%
%
\item[Author:]%
Cengiz Gunay <cgunay@emory.edu>, 2006/01/23%
\end{description}
\methodline%
\subsubsection[Method \texttt{plotScatter}]{Method \texttt{tests\_db/plotScatter}}%
\index[funcref]{tests_db@\fidxlb{tests\_db}!plotScatter@\fidxl{plotScatter}}%
\label{ref_tests_db__plotScatter}%
\hypertarget{ref_tests_db__plotScatter}{}%
\begin{description}
\item[Summary:]Create a scatter plot of the given two tests.
%
\item[Usage:]~%
\begin{lyxcode}%
a\_p = plotScatter(a\_db, test1, test2, title\_str, short\_title, props)
%
\end{lyxcode}%
%
%
\item[Parameters:]~
\begin{description}%
\item[\texttt{a\_db}:]
 A params\_tests\_db object.
\item[\texttt{test1, test2}:]
 X \& Y variables.
\item[\texttt{title\_str}:]
 (Optional) A string to be concatanated to the title.
\item[\texttt{short\_title}:]
 (Optional) Few words that may appear in legends of multiplot.
\item[\texttt{props}:]
 A structure with any optional properties.
\begin{description}%
\item[\texttt{LineStyle}:]
 Plot line style to use. (default: 'x')
\item[\texttt{Regress}:]
 Calculate and plot a linear regression.
\item[\texttt{quiet}:]
 If 1, don't include database name on title.
\end{description}%
\end{description}%
%
\item[Returns:]~

	a\_p: A plot\_abstract.
%
%
\item[See also:]%
%
\item[Author:]%
Cengiz Gunay <cgunay@emory.edu>, 2005/09/29%
\end{description}
\methodline%
\subsubsection[Method \texttt{addColumns}]{Method \texttt{tests\_db/addColumns}}%
\index[funcref]{tests_db@\fidxlb{tests\_db}!addColumns@\fidxl{addColumns}}%
\label{ref_tests_db__addColumns}%
\hypertarget{ref_tests_db__addColumns}{}%
\begin{description}
\item[Summary:]Inserts new columns to tests\_db.
%
%
\item[Description:]%
Adds new test columns to the database and returns the new DB.
 Usage 2 concatanates two DBs columnwise. This operation is 
 expensive in the sense that the whole database matrix needs to be 
 enlarged just to add a single new column. The method of allocating
 a matrix, filling it up, and then providing it to the tests\_db 
 constructor is the preferred method of creating tests\_db objects. 
 This method may be used for measures obtained by operating on raw measures.
%%
\item[Parameters:]~
\begin{description}%
\item[\texttt{obj, b\_obj}:]
 A tests\_db object.
\item[\texttt{test\_names}:]
 A cell array of test names to be added.
\item[\texttt{test\_columns}:]
 Data matrix of columns to be added.
\end{description}%
%
\item[Returns:]~

	obj: The tests\_db object that includes the new columns.
%
%
\item[See also:]%
\hyperlink{ref_allocateRows}{\texttt{allocateRows}}%
\ (p.~\pageref{ref_allocateRows})%
\index[funcref]{@\fidxl{allocateRows}}%
, \hyperlink{ref_tests_db}{\texttt{tests\_db}}%
\ (p.~\pageref{ref_tests_db})%
\index[funcref]{@\fidxl{tests\_db}}%
%
\item[Author:]%
Cengiz Gunay <cgunay@emory.edu>, 2005/09/30%
\end{description}
\methodline%
\subsubsection[Method \texttt{delColumns}]{Method \texttt{tests\_db/delColumns}}%
\index[funcref]{tests_db@\fidxlb{tests\_db}!delColumns@\fidxl{delColumns}}%
\label{ref_tests_db__delColumns}%
\hypertarget{ref_tests_db__delColumns}{}%
\begin{description}
\item[Summary:]Deletes columns from tests\_db.
%
\item[Usage:]~%
\begin{lyxcode}%
obj = delColumns(obj, tests)
%
\end{lyxcode}%
%
\item[Description:]%
Deletes test columns from the database and returns the new DB.
   This operation is expensive in the sense that the whole database matrix
   needs to be copied just to delete a 
   single column. The method of allocating a matrix, filling it up, and
   then providing it to the tests\_db constructor is the preferred method 
   of creating tests\_db objects. This method may be used for 
   measures obtained by operating on raw measures.
%%
\item[Parameters:]~
\begin{description}%
\item[\texttt{obj}:]
 A tests\_db object.
\item[\texttt{tests}:]
 Numbers or names of tests (see tests2cols)
\end{description}%
%
\item[Returns:]~

	obj: The tests\_db object that is missing the columns.
%
%
\item[See also:]%
\hyperlink{ref_allocateRows}{\texttt{allocateRows}}%
\ (p.~\pageref{ref_allocateRows})%
\index[funcref]{@\fidxl{allocateRows}}%
, \hyperlink{ref_tests_db}{\texttt{tests\_db}}%
\ (p.~\pageref{ref_tests_db})%
\index[funcref]{@\fidxl{tests\_db}}%
%
\item[Author:]%
Cengiz Gunay <cgunay@emory.edu>, 2005/10/06%
\end{description}
\methodline%
\subsubsection[Method \texttt{factoran}]{Method \texttt{tests\_db/factoran}}%
\index[funcref]{tests_db@\fidxlb{tests\_db}!factoran@\fidxl{factoran}}%
\label{ref_tests_db__factoran}%
\hypertarget{ref_tests_db__factoran}{}%
\begin{description}
\item[Summary:]Generates a database of factor loadings obtained from the 
		factor analysis of db with factoran. Each row corresponds
		to a rotated factor and columns represent observed variables.
%
\item[Usage:]~%
\begin{lyxcode}%
a\_factors\_db = factoran(db, num\_factors, props)
%
\end{lyxcode}%
%
\item[Description:]%
Uses the promax method to rotate common factors.
%%
\item[Parameters:]~
\begin{description}%
\item[\texttt{db}:]
 A tests\_db object.
\item[\texttt{num\_factors}:]
 Number of common factors to look for.
\item[\texttt{props}:]
 A structure with any optional properties.
\end{description}%
%
\item[Returns:]~

	a\_factors\_db: A corrcoefs\_db of the coefficients and page indices.
%
%
\item[See also:]%
\hyperlink{ref_tests_db}{\texttt{tests\_db}}%
\ (p.~\pageref{ref_tests_db})%
\index[funcref]{@\fidxl{tests\_db}}%
, \hyperlink{ref_corrcoefs_db}{\texttt{corrcoefs\_db}}%
\ (p.~\pageref{ref_corrcoefs_db})%
\index[funcref]{@\fidxl{corrcoefs\_db}}%
%
\item[Author:]%
Cengiz Gunay <cgunay@emory.edu>, 2004/11/08%
\end{description}
\methodline%
\subsubsection[Method \texttt{statsAll}]{Method \texttt{tests\_db/statsAll}}%
\index[funcref]{tests_db@\fidxlb{tests\_db}!statsAll@\fidxl{statsAll}}%
\label{ref_tests_db__statsAll}%
\hypertarget{ref_tests_db__statsAll}{}%
\begin{description}
\item[Summary:]Makes a stats\_db with rows of mean, STD, SE, and CV of the tests' distributions in db.
%
\item[Usage:]~%
\begin{lyxcode}%
a\_stats\_db = statsAll(db, tests, props)
%
\end{lyxcode}%
%
%
\item[Parameters:]~
\begin{description}%
\item[\texttt{db}:]
 A tests\_db object.
\item[\texttt{tests}:]
 A selection of tests (see onlyRowsTests).
\item[\texttt{props}:]
 A structure with any optional properties for stats\_db.
\end{description}%
%
\item[Returns:]~

	a\_stats\_db: A stats\_db object.
%
%
\item[See also:]%
\hyperlink{ref_tests_db}{\texttt{tests\_db}}%
\ (p.~\pageref{ref_tests_db})%
\index[funcref]{@\fidxl{tests\_db}}%
%
\item[Author:]%
Cengiz Gunay <cgunay@emory.edu>, 2005/08/24%
\end{description}
\methodline%
\subsubsection[Method \texttt{enumerateColumns}]{Method \texttt{tests\_db/enumerateColumns}}%
\index[funcref]{tests_db@\fidxlb{tests\_db}!enumerateColumns@\fidxl{enumerateColumns}}%
\label{ref_tests_db__enumerateColumns}%
\hypertarget{ref_tests_db__enumerateColumns}{}%
\begin{description}
\item[Summary:]Replaces each value with an integer pointing to the index of enumerated unique values in a column.
%
\item[Usage:]~%
\begin{lyxcode}%
a\_db = enumerateColumns(a\_db, tests, props)
%
\end{lyxcode}%
%
\item[Description:]%
Finds unique values of each column, and replaces the original values
 with the enumerated indices of these unique values. Useful for normalizing all 
 parameter values in a hypercube.
%%
\item[Parameters:]~
\begin{description}%
\item[\texttt{a\_db}:]
 A tests\_db object.
\item[\texttt{tests}:]
 Array of tests to be enumerated.
\item[\texttt{props}:]
 Optional properties.
\begin{description}%
\item[\texttt{truncateDecDigits}:]
 Use only up to this many decimal digits after the point 

when checking for uniqueness.\end{description}%
\end{description}%
%
\item[Returns:]~

	a\_db: The modified DB.
%
\item[Example:]~
\begin{lyxcode} >> enumerated\_db = enumerateColumns(a\_db(:, 1:9));\\%
\end{lyxcode}
%
\item[See also:]%
\hyperlink{ref_uniqueValues}{\texttt{uniqueValues}}%
\ (p.~\pageref{ref_uniqueValues})%
\index[funcref]{@\fidxl{uniqueValues}}%
%
\item[Author:]%
Cengiz Gunay <cgunay@emory.edu>, 2006/06/14%
\end{description}
\methodline%
\subsubsection[Method \texttt{tests2cols}]{Method \texttt{tests\_db/tests2cols}}%
\index[funcref]{tests_db@\fidxlb{tests\_db}!tests2cols@\fidxl{tests2cols}}%
\label{ref_tests_db__tests2cols}%
\hypertarget{ref_tests_db__tests2cols}{}%
\begin{description}
\item[Summary:]Find column numbers from a test names/numbers specification.
%
\item[Usage:]~%
\begin{lyxcode}%
cols = tests2cols(db, tests)
%
\end{lyxcode}%
%
%
\item[Parameters:]~
\begin{description}%
\item[\texttt{db}:]
 A tests\_db object.
\item[\texttt{tests}:]
 Either a single or array of column numbers, or a single

test name or a cell array of test names. If ':', all tests.\end{description}%
%
\item[Returns:]~

	cols: Array of column indices.
%
%
\item[See also:]%
\hyperlink{ref_tests_db}{\texttt{tests\_db}}%
\ (p.~\pageref{ref_tests_db})%
\index[funcref]{@\fidxl{tests\_db}}%
%
\item[Author:]%
Cengiz Gunay <cgunay@emory.edu>, 2004/10/07%
\end{description}
\methodline%
\subsubsection[Method \texttt{plotrows}]{Method \texttt{tests\_db/plotrows}}%
\index[funcref]{tests_db@\fidxlb{tests\_db}!plotrows@\fidxl{plotrows}}%
\label{ref_tests_db__plotrows}%
\hypertarget{ref_tests_db__plotrows}{}%
\begin{description}
\item[Summary:]Creates a plot\_stack describing the db rows.
%
\item[Usage:]~%
\begin{lyxcode}%
a\_plot = plotrows(a\_tests\_db, axis\_limits, orientation, props)
%
\end{lyxcode}%
%
%
\item[Parameters:]~
\begin{description}%
\item[\texttt{a\_tests\_db}:]
 A tests\_db object.
\item[\texttt{axis\_limits}:]
 If given, all plots contained will have these axis limits.
\item[\texttt{orientation}:]
 Stack orientation 'x' for horizontal, 'y' for vertical, etc.
\item[\texttt{props}:]
 A structure with any optional properties passed to plot\_stack.
\end{description}%
%
\item[Returns:]~

	a\_plot: A plot\_stack object that can be plotted.
%
%
\item[See also:]%
\hyperlink{ref_plot_abstract}{\texttt{plot\_abstract}}%
\ (p.~\pageref{ref_plot_abstract})%
\index[funcref]{@\fidxl{plot\_abstract}}%
, \hyperlink{ref_plotFigure}{\texttt{plotFigure}}%
\ (p.~\pageref{ref_plotFigure})%
\index[funcref]{@\fidxl{plotFigure}}%
%
\item[Author:]%
Cengiz Gunay <cgunay@emory.edu>, 2004/11/09%
\end{description}
\methodline%
\subsubsection[Method \texttt{statsMeanSE}]{Method \texttt{tests\_db/statsMeanSE}}%
\index[funcref]{tests_db@\fidxlb{tests\_db}!statsMeanSE@\fidxl{statsMeanSE}}%
\label{ref_tests_db__statsMeanSE}%
\hypertarget{ref_tests_db__statsMeanSE}{}%
\begin{description}
\item[Summary:]Generates a stats\_db object with two rows corresponding to 
		the mean and standard error (SE) of the tests' distributions.
%
\item[Usage:]~%
\begin{lyxcode}%
a\_stats\_db = statsMeanSE(db, tests, props)
%
\end{lyxcode}%
%
%
\item[Parameters:]~
\begin{description}%
\item[\texttt{db}:]
 A tests\_db object.
\item[\texttt{tests}:]
 A selection of tests (see onlyRowsTests).
\item[\texttt{props}:]
 A structure with any optional properties for stats\_db.
\end{description}%
%
\item[Returns:]~

	a\_stats\_db: A stats\_db object.
%
%
\item[See also:]%
\hyperlink{ref_tests_db}{\texttt{tests\_db}}%
\ (p.~\pageref{ref_tests_db})%
\index[funcref]{@\fidxl{tests\_db}}%
%
\item[Author:]%
Cengiz Gunay <cgunay@emory.edu>, 2004/10/07%
\end{description}
\methodline%
\subsubsection[Method \texttt{plot\_bars}]{Method \texttt{tests\_db/plot\_bars}}%
\index[funcref]{tests_db@\fidxlb{tests\_db}!plot_bars@\fidxl{plot\_bars}}%
\label{ref_tests_db__plot_bars}%
\hypertarget{ref_tests_db__plot_bars}{}%
\begin{description}
\item[Summary:]Creates a bar graph comparing all DB rows in groups, with a separate axis for each column.
%
\item[Usage:]~%
\begin{lyxcode}%
a\_plot = plot\_bars(a\_tests\_db, title\_str, props)
%
\end{lyxcode}%
%
%
\item[Parameters:]~
\begin{description}%
\item[\texttt{a\_tests\_db}:]
 A tests\_db object.
\item[\texttt{title\_str}:]
 (Optional) The plot title.
\item[\texttt{props}:]
 A structure with any optional properties.
\end{description}%
%
\item[Returns:]~

	a\_plot: A object of plot\_bars or one of its subclasses.
%
%
\item[See also:]%
\hyperlink{ref_plot_abstract}{\texttt{plot\_abstract}}%
\ (p.~\pageref{ref_plot_abstract})%
\index[funcref]{@\fidxl{plot\_abstract}}%
, \hyperlink{ref_plot_simple}{\texttt{plot\_simple}}%
\ (p.~\pageref{ref_plot_simple})%
\index[funcref]{@\fidxl{plot\_simple}}%
%
\item[Author:]%
Cengiz Gunay <cgunay@emory.edu>, 2006/03/13%
\end{description}
\methodline%
\subsubsection[Method \texttt{corrCoefs}]{Method \texttt{tests\_db/corrCoefs}}%
\index[funcref]{tests_db@\fidxlb{tests\_db}!corrCoefs@\fidxl{corrCoefs}}%
\label{ref_tests_db__corrCoefs}%
\hypertarget{ref_tests_db__corrCoefs}{}%
\begin{description}
\item[Summary:]Generates a database of correlation coefficients 
		by comparing col1 with other cols in the database. 
		If db has multiple pages, then each page in db 
		produces a row of coefficients and matching PageIndex.
%
\item[Usage:]~%
\begin{lyxcode}%
a\_coefs\_db = corrCoefs(db, col1, cols, props)
%
\end{lyxcode}%
%
\item[Description:]%
Assuming the db was created with invarValues, this function finds the
 invariant correlation coefficients between its columns. 
 The invariant correlation coefficients are the correlation of one column
 value with another column value when some other column values are fixed.
 Since there are many occurences of the invariant coefficients, a histogram
 can then be created and returned from the created db. The other
 columns that are fixed are not in this db object, but can be reached 
 using the row indices in the original db. The page number is saved in the 
 created db, so that it can be used to find the page from which the 
 coefficient came. Then row indices of the page points to original 
 constant column values.
%%
\item[Parameters:]~
\begin{description}%
\item[\texttt{db}:]
 A tests\_db object.
\item[\texttt{col1}:]
 Column to compare.
\item[\texttt{cols}:]
 Columns to be compared with col1.
\item[\texttt{props}:]
 A structure with any optional properties.
\begin{description}%
\item[\texttt{skipCoefs}:]
 If coefficients of less confidence than %95 

should be skipped.\end{description}%
\end{description}%
%
\item[Returns:]~

	a\_coefs\_db: A corrcoefs\_db of the coefficients and page indices.
%
%
\item[See also:]%
\hyperlink{ref_tests_db}{\texttt{tests\_db}}%
\ (p.~\pageref{ref_tests_db})%
\index[funcref]{@\fidxl{tests\_db}}%
, \hyperlink{ref_corrcoefs_db}{\texttt{corrcoefs\_db}}%
\ (p.~\pageref{ref_corrcoefs_db})%
\index[funcref]{@\fidxl{corrcoefs\_db}}%
%
\item[Author:]%
Cengiz Gunay <cgunay@emory.edu>, 2004/09/30%
\end{description}
\methodline%
\subsubsection[Method \texttt{vertcat}]{Method \texttt{tests\_db/vertcat}}%
\index[funcref]{tests_db@\fidxlb{tests\_db}!vertcat@\fidxl{vertcat}}%
\label{ref_tests_db__vertcat}%
\hypertarget{ref_tests_db__vertcat}{}%
\begin{description}
\item[Summary:]Vertical concatanation [db;with\_db;...] operator.
%
\item[Usage:]~%
\begin{lyxcode}%
a\_db = vertcat(db, with\_db, ...)
%
\end{lyxcode}%
%
\item[Description:]%
Concatanates rows of with\_db to rows of db. Overrides the built-in
 vertcat function that is called when [db;with\_db] is executed.
%%
\item[Parameters:]~
\begin{description}%
\item[\texttt{db}:]
 A tests\_db object.
\item[\texttt{with\_db}:]
 A tests\_db object whose rows are concatanated to db.
\end{description}%
%
\item[Returns:]~

	a\_db: A tests\_db that contains rows of db and with\_db.
%
%
\item[See also:]%
\hyperlink{ref_vertcat}{\texttt{vertcat}}%
\ (p.~\pageref{ref_vertcat})%
\index[funcref]{@\fidxl{vertcat}}%
, \hyperlink{ref_tests_db}{\texttt{tests\_db}}%
\ (p.~\pageref{ref_tests_db})%
\index[funcref]{@\fidxl{tests\_db}}%
%
\item[Author:]%
Cengiz Gunay <cgunay@emory.edu>, 2005/01/25%
\end{description}
\methodline%
\subsubsection[Method \texttt{assignRowsTests}]{Method \texttt{tests\_db/assignRowsTests}}%
\index[funcref]{tests_db@\fidxlb{tests\_db}!assignRowsTests@\fidxl{assignRowsTests}}%
\label{ref_tests_db__assignRowsTests}%
\hypertarget{ref_tests_db__assignRowsTests}{}%
\begin{description}
\item[Summary:]Assign the values to the tests and rows (and pages) of the tests\_db.
%
\item[Usage:]~%
\begin{lyxcode}%
obj = assignRowsTests(obj, val, rows, tests, pages)
%
\end{lyxcode}%
%
\item[Description:]%
Selects the given dimensions and returns in a new tests\_db object.
%%
\item[Parameters:]~
\begin{description}%
\item[\texttt{obj}:]
 A tests\_db object.
\item[\texttt{val}:]
 DB object or data matrix to be assigned to the addressed indices.
\item[\texttt{rows}:]
 A logical or index vector of rows. If ':', all rows.
\item[\texttt{tests}:]
 Cell array of test names or column indices. If ':', all tests.
\item[\texttt{pages}:]
 (Optional) A logical or index vector of pages. ':' for all pages.
\end{description}%
%
\item[Returns:]~

	obj: The new tests\_db object.
%
%
\item[See also:]%
\hyperlink{ref_subsref}{\texttt{subsref}}%
\ (p.~\pageref{ref_subsref})%
\index[funcref]{@\fidxl{subsref}}%
, \hyperlink{ref_tests_db}{\texttt{tests\_db}}%
\ (p.~\pageref{ref_tests_db})%
\index[funcref]{@\fidxl{tests\_db}}%
%
\item[Author:]%
Cengiz Gunay <cgunay@emory.edu>, 2006/02/08%
\end{description}
\methodline%
\subsubsection[Method \texttt{isinf}]{Method \texttt{tests\_db/isinf}}%
\index[funcref]{tests_db@\fidxlb{tests\_db}!isinf@\fidxl{isinf}}%
\label{ref_tests_db__isinf}%
\hypertarget{ref_tests_db__isinf}{}%
\begin{description}
\item[Summary:]Returns logical row indices of Inf-valued columns.
%
\item[Usage:]~%
\begin{lyxcode}%
rows = isinf(db, col)
%
\end{lyxcode}%
%
%
\item[Parameters:]~
\begin{description}%
\item[\texttt{db}:]
 A tests\_db object.
\item[\texttt{col}:]
 Column to check (Optional, default = 1)
\end{description}%
%
\item[Returns:]~

	rows: A logical column vector of rows.
%
%
\item[See also:]%
\hyperlink{ref_isinf}{\texttt{isinf}}%
\ (p.~\pageref{ref_isinf})%
\index[funcref]{@\fidxl{isinf}}%
, \hyperlink{ref_tests_db}{\texttt{tests\_db}}%
\ (p.~\pageref{ref_tests_db})%
\index[funcref]{@\fidxl{tests\_db}}%
%
\item[Author:]%
Cengiz Gunay <cgunay@emory.edu>, 2005/08/16%
\end{description}
\methodline%
\subsubsection[Method \texttt{isnan}]{Method \texttt{tests\_db/isnan}}%
\index[funcref]{tests_db@\fidxlb{tests\_db}!isnan@\fidxl{isnan}}%
\label{ref_tests_db__isnan}%
\hypertarget{ref_tests_db__isnan}{}%
\begin{description}
\item[Summary:]Returns logical row indices of NaN-valued columns.
%
\item[Usage:]~%
\begin{lyxcode}%
rows = isnan(db, col)
%
\end{lyxcode}%
%
%
\item[Parameters:]~
\begin{description}%
\item[\texttt{db}:]
 A tests\_db object.
\item[\texttt{col}:]
 Column to check (Optional, default = 1)
\end{description}%
%
\item[Returns:]~

	rows: A logical column vector of rows.
%
%
\item[See also:]%
\hyperlink{ref_isnan}{\texttt{isnan}}%
\ (p.~\pageref{ref_isnan})%
\index[funcref]{@\fidxl{isnan}}%
, \hyperlink{ref_tests_db}{\texttt{tests\_db}}%
\ (p.~\pageref{ref_tests_db})%
\index[funcref]{@\fidxl{tests\_db}}%
%
\item[Author:]%
Cengiz Gunay <cgunay@emory.edu>, 2004/10/06%
\end{description}
\methodline%
\subsubsection[Method \texttt{rankMatching}]{Method \texttt{tests\_db/rankMatching}}%
\index[funcref]{tests_db@\fidxlb{tests\_db}!rankMatching@\fidxl{rankMatching}}%
\label{ref_tests_db__rankMatching}%
\hypertarget{ref_tests_db__rankMatching}{}%
\begin{description}
\item[Summary:]Create a ranking db of row distances of db to given criterion db.
%
\item[Usage:]~%
\begin{lyxcode}%
a\_ranked\_db = rankMatching(db, crit\_db, props)
%
\end{lyxcode}%
%
\item[Description:]%
crit\_db can be created with the matchingRow method. TestWeights modify the importance 
 of each measure.
%%
\item[Parameters:]~
\begin{description}%
\item[\texttt{db}:]
 A tests\_db to rank.
\item[\texttt{crit\_db}:]
 A tests\_db object holding the match criterion tests and stds.
\item[\texttt{props}:]
 A structure with any optional properties.
\begin{description}%
\item[\texttt{tolerateNaNs}:]
 If 0, rows with any NaN values are skipped (default=1).
\item[\texttt{testWeights}:]
 Structure array associating tests and multiplicative weights.
\item[\texttt{topRows}:]
 If given, only return this many of the top rows.
\end{description}%
\end{description}%
%
\item[Returns:]~

	a\_ranked\_db: A ranked\_db object.
%
%
\item[See also:]%
\hyperlink{ref_matchingRow}{\texttt{matchingRow}}%
\ (p.~\pageref{ref_matchingRow})%
\index[funcref]{@\fidxl{matchingRow}}%
, \hyperlink{ref_tests_db}{\texttt{tests\_db}}%
\ (p.~\pageref{ref_tests_db})%
\index[funcref]{@\fidxl{tests\_db}}%
%
\item[Author:]%
Cengiz Gunay <cgunay@emory.edu>, 2004/12/08%
\end{description}
\methodline%
\subsubsection[Method \texttt{statsBounds}]{Method \texttt{tests\_db/statsBounds}}%
\index[funcref]{tests_db@\fidxlb{tests\_db}!statsBounds@\fidxl{statsBounds}}%
\label{ref_tests_db__statsBounds}%
\hypertarget{ref_tests_db__statsBounds}{}%
\begin{description}
\item[Summary:]Generates a stats\_db object with three rows corresponding to the mean, min, and max of the tests' distributions. 
%
\item[Usage:]~%
\begin{lyxcode}%
a\_stats\_db = statsBounds(a\_db, tests, props)
%
\end{lyxcode}%
%
\item[Description:]%
A page is generated for each page of data in db.
%%
\item[Parameters:]~
\begin{description}%
\item[\texttt{a\_db}:]
 A tests\_db object.
\item[\texttt{tests}:]
 A selection of tests (see onlyRowsTests).
\item[\texttt{props}:]
 A structure with any optional properties for stats\_db.
\end{description}%
%
\item[Returns:]~

	a\_stats\_db: A stats\_db object.
%
%
\item[See also:]%
\hyperlink{ref_tests_db}{\texttt{tests\_db}}%
\ (p.~\pageref{ref_tests_db})%
\index[funcref]{@\fidxl{tests\_db}}%
%
\item[Author:]%
Cengiz Gunay <cgunay@emory.edu>, 2004/10/07%
\end{description}
\methodline%
\subsection{Class \texttt{trace}}%
\index[funcref]{trace@\fidxlb{trace}}%
\label{ref_trace}%
\hypertarget{ref_trace}{}%
\subsubsection[Constructor \texttt{trace}]{Constructor \texttt{trace/trace}}%
\index[funcref]{trace@\fidxlb{trace}!trace@\fidxl{trace}}%
\label{ref_trace__trace}%
\hypertarget{ref_trace__trace}{}%
\begin{description}
\item[Summary:]A generic trace from a cell. It can be voltage, current, etc.
%
\item[Usage:]~%
\begin{lyxcode}%
obj = trace(data\_src, dt, dy, id, props)
%
\end{lyxcode}%
%
\item[Description:]%
Traces for specific experimental or simulation protocols can extend 
 this class for adding new parameters. This object is designed to recognize
 most data file formats. See the data\_src parameter below.
%%
\item[Parameters:]~
\begin{description}%
\item[\texttt{data\_src}:]
 Trace data as a column vector OR name of a data file generated by either 

Genesis (.bin, .gbin, .genflac), PCDX (.all), or Matlab (.mat).\item[\texttt{dt}:]
 Time resolution [s]
\item[\texttt{dy}:]
 y-axis resolution [ISI (V, A, etc.)]
\item[\texttt{id}:]
 Identification string
\item[\texttt{props}:]
 A structure with any optional properties.
\begin{description}%
\item[\texttt{scale\_y}:]
 Y-axis scale to be applied to loaded data.
\item[\texttt{offset\_y}:]
 Y-axis offset to be added to loaded and scaled data.
\item[\texttt{trace\_time\_start}:]
 Samples in the beginning to discard [dt]
\item[\texttt{baseline}:]
 Resting potential.
\item[\texttt{channel}:]
 Channel to read from file Genesis or PCDX file.
\item[\texttt{traces}:]
 Traces to read from PCDX file.
\item[\texttt{spike\_finder}:]
 Method of finding spikes 

(1 for findFilteredSpikes, 2 for findspikes).\item[\texttt{init\_Vm\_method}:]
 Method of finding spike thresholds 

(see spike\_shape/spike\_shape).\item[\texttt{init\_threshold}:]
 Spike initiation threshold (deriv or accel).

(see above methods and implementation in calcInitVm)\item[\texttt{init\_lo\_thr, init\_hi\_thr}:]
 Low and high thresholds for slope.
\item[\texttt{threshold}:]
 Spike threshold.
\item[\texttt{quiet}:]
 If 1, reduces the amount of textual description in plots, etc.
\end{description}%
\end{description}%
%
\item[Returns a structure object with the following fields:]~

	data: The trace column matrix.
	dt, dy, id, props (see above)
%
%
\item[See also:]%
\hyperlink{ref_spikes}{\texttt{spikes}}%
\ (p.~\pageref{ref_spikes})%
\index[funcref]{@\fidxl{spikes}}%
, \hyperlink{ref_spike_shape}{\texttt{spike\_shape}}%
\ (p.~\pageref{ref_spike_shape})%
\index[funcref]{@\fidxl{spike\_shape}}%
, \hyperlink{ref_cip_trace}{\texttt{cip\_trace}}%
\ (p.~\pageref{ref_cip_trace})%
\index[funcref]{@\fidxl{cip\_trace}}%
, \hyperlink{ref_period}{\texttt{period}}%
\ (p.~\pageref{ref_period})%
\index[funcref]{@\fidxl{period}}%
%
\item[Author:]%
Cengiz Gunay <cgunay@emory.edu>, 2004/07/30%
\end{description}
\methodline%
\subsubsection[Method \texttt{setProp}]{Method \texttt{trace/setProp}}%
\index[funcref]{trace@\fidxlb{trace}!setProp@\fidxl{setProp}}%
\label{ref_trace__setProp}%
\hypertarget{ref_trace__setProp}{}%
\begin{description}
\item[Summary:]Generic method for setting optional object properties.
%
\item[Usage:]~%
\begin{lyxcode}%
obj = setProp(obj, prop1, val1, prop2, val2, ...)
%
\end{lyxcode}%
%
\item[Description:]%
Modifies or adds property values. As many property name-value 
 pairs can be specified.
%%
\item[Parameters:]~
\begin{description}%
\item[\texttt{obj}:]
 Any object that has a props field.
\item[\texttt{attr}:]
 Property name
\item[\texttt{val}:]
 Property value.
\end{description}%
%
\item[Returns:]~

	obj: The new object with the updated properties.
%
%
\item[See also:]%
%
\item[Author:]%
Cengiz Gunay <cgunay@emory.edu>, 2004/11/22%
\end{description}
\methodline%
\subsubsection[Method \texttt{display}]{Method \texttt{trace/display}}%
\index[funcref]{trace@\fidxlb{trace}!display@\fidxl{display}}%
\label{ref_trace__display}%
\hypertarget{ref_trace__display}{}%
\begin{description}
%
%
%
%
%
%
%
\item[Author:]%
Cengiz Gunay <cgunay@emory.edu>, 2004/08/04%
\end{description}
\methodline%
\subsubsection[Method \texttt{get}]{Method \texttt{trace/get}}%
\index[funcref]{trace@\fidxlb{trace}!get@\fidxl{get}}%
\label{ref_trace__get}%
\hypertarget{ref_trace__get}{}%
\begin{description}
\item[Summary:]Defines generic attribute retrieval for objects.
%
%
%
%
%
%
%
\item[Author:]%
Cengiz Gunay <cgunay@emory.edu>, 2004/09/14%
\end{description}
\methodline%
\subsubsection[Method \texttt{periodWhole}]{Method \texttt{trace/periodWhole}}%
\index[funcref]{trace@\fidxlb{trace}!periodWhole@\fidxl{periodWhole}}%
\label{ref_trace__periodWhole}%
\hypertarget{ref_trace__periodWhole}{}%
\begin{description}
\item[Summary:]Returns the boundaries of the whole period of trace, t. 
%
\item[Usage:]~%
\begin{lyxcode}%
whole\_period = periodWhole(t)
%
\end{lyxcode}%
%
%
\item[Parameters:]~
\begin{description}%
\item[\texttt{t}:]
 A trace object.
\end{description}%
%
%
%
\item[See also:]%
\hyperlink{ref_period}{\texttt{period}}%
\ (p.~\pageref{ref_period})%
\index[funcref]{@\fidxl{period}}%
, \hyperlink{ref_trace}{\texttt{trace}}%
\ (p.~\pageref{ref_trace})%
\index[funcref]{@\fidxl{trace}}%
%
\item[Author:]%
Cengiz Gunay <cgunay@emory.edu>, 2004/07/30%
\end{description}
\methodline%
\subsubsection[Method \texttt{set}]{Method \texttt{trace/set}}%
\index[funcref]{trace@\fidxlb{trace}!set@\fidxl{set}}%
\label{ref_trace__set}%
\hypertarget{ref_trace__set}{}%
\begin{description}
\item[Summary:]Generic method for setting object attributes.
%
%
%
%
%
%
%
\item[Author:]%
Cengiz Gunay <cgunay@emory.edu>, 2004/10/08%
\end{description}
\methodline%
\subsubsection[Method \texttt{plotData}]{Method \texttt{trace/plotData}}%
\index[funcref]{trace@\fidxlb{trace}!plotData@\fidxl{plotData}}%
\label{ref_trace__plotData}%
\hypertarget{ref_trace__plotData}{}%
\begin{description}
\item[Summary:]Plots a trace.
%
\item[Usage:]~%
\begin{lyxcode}%
a\_plot = plotData(t, title\_str, props)
%
\end{lyxcode}%
%
\item[Description:]%
If t is a vector of traces, returns a vector of plot objects.
%%
\item[Parameters:]~
\begin{description}%
\item[\texttt{t}:]
 A trace object.
\end{description}%
%
\item[Returns:]~

	a\_plot: A plot\_abstract object that can be visualized.
	title\_str: (Optional) String to append to plot title.
	props: A structure with any optional properties.
	  timeScale: 's' for seconds, or 'ms' for milliseconds.
	  (rest passed to plot\_abstract.)
%
%
\item[See also:]%
\hyperlink{ref_trace}{\texttt{trace}}%
\ (p.~\pageref{ref_trace})%
\index[funcref]{@\fidxl{trace}}%
, \hyperlink{ref_trace__plot}{\texttt{trace/plot}}%
\ (p.~\pageref{ref_trace__plot})%
\index[funcref]{trace@\fidxlb{trace}!plot@\fidxl{plot}}%
, \hyperlink{ref_plot_abstract}{\texttt{plot\_abstract}}%
\ (p.~\pageref{ref_plot_abstract})%
\index[funcref]{@\fidxl{plot\_abstract}}%
%
\item[Author:]%
Cengiz Gunay <cgunay@emory.edu>, 2004/11/17%
\end{description}
\methodline%
\subsubsection[Method \texttt{spikes}]{Method \texttt{trace/spikes}}%
\index[funcref]{trace@\fidxlb{trace}!spikes@\fidxl{spikes}}%
\label{ref_trace__spikes}%
\hypertarget{ref_trace__spikes}{}%
\begin{description}
\item[Summary:]Convert trace to spikes object for spike timing calculations.
%
\item[Usage:]~%
\begin{lyxcode}%
obj = spikes(trace, a\_period, plotit, minamp)
%
\end{lyxcode}%
%
\item[Description:]%
Creates a spikes object.
%%
\item[Parameters:]~
\begin{description}%
\item[\texttt{trace}:]
 A trace object.
\item[\texttt{a\_period}:]
 A period object denoting the part of trace of interest 

(optional, if empty vector, taken as wholePeriod).\item[\texttt{plotit}:]
 If non-zero, a plot is generated for showing spikes found

(optional).\item[\texttt{minamp}:]
 minimum amplitude that must be reached if using findFilteredSpikes.

--> adjust as needed to discriminate spikes from EPSPs.
(optional)\end{description}%
%
%
%
\item[See also:]%
\hyperlink{ref_spikes}{\texttt{spikes}}%
\ (p.~\pageref{ref_spikes})%
\index[funcref]{@\fidxl{spikes}}%
%
\item[Author:]%
Cengiz Gunay <cgunay@emory.edu>, 2004/07/30%
\end{description}
\methodline%
\subsubsection[Method \texttt{findFilteredSpikes}]{Method \texttt{trace/findFilteredSpikes}}%
\index[funcref]{trace@\fidxlb{trace}!findFilteredSpikes@\fidxl{findFilteredSpikes}}%
\label{ref_trace__findFilteredSpikes}%
\hypertarget{ref_trace__findFilteredSpikes}{}%
\begin{description}
\item[Summary:]Runs a frequency filter over the data and then 
			finds all peaks using findspikes.
%
\item[Usage:]~%
\begin{lyxcode}%
[times, peaks, n] = 
	findFilteredSpikes(t, a\_period, plotit, minamp)
%
\end{lyxcode}%
%
\item[Description:]%
Runs a 50-300 Hz band-pass filter over the data and then calls findspikes.
   The filter both removes low-frequency offset changes, such as 
   cip period effects, and high-frequency noise that is detected 
   as local peaks by findspikes. The spikes found are 
   post-processed to make sure the rise and fall times are consistent.
   Note: The filter employed only works with data sampled at 10kHz.
%%
\item[Parameters:]~
\begin{description}%
\item[\texttt{t}:]
 Trace object
\item[\texttt{a\_period}:]
 Period of interest.
\item[\texttt{plotit}:]
 Plots the spikes found if 1.

minamp (optional): minimum amplitude above baseline that must be reached.
--> adjust as necessary to discriminate spikes from EPSPs.\item[\texttt{Returns}:]

\item[\texttt{times}:]
 The times of spikes [dt].
\item[\texttt{peaks}:]
 The peaks corresponding to the times of spikes.
\item[\texttt{n}:]
 The number of spikes.
\end{description}%
%
%
%
\item[See also:]%
\hyperlink{ref_findspikes}{\texttt{findspikes}}%
\ (p.~\pageref{ref_findspikes})%
\index[funcref]{@\fidxl{findspikes}}%
, \hyperlink{ref_period}{\texttt{period}}%
\ (p.~\pageref{ref_period})%
\index[funcref]{@\fidxl{period}}%
%
\item[Author:]%
Cengiz Gunay <cgunay@emory.edu>, 2004/03/08%
\end{description}
\methodline%
\subsubsection[Method \texttt{calcAvg}]{Method \texttt{trace/calcAvg}}%
\index[funcref]{trace@\fidxlb{trace}!calcAvg@\fidxl{calcAvg}}%
\label{ref_trace__calcAvg}%
\hypertarget{ref_trace__calcAvg}{}%
\begin{description}
\item[Summary:]Calculates the average value of the given period 
 		of the trace, t. 
%
\item[Usage:]~%
\begin{lyxcode}%
avg\_val = calcAvg(t, period)
%
\end{lyxcode}%
%
%
\item[Parameters:]~
\begin{description}%
\item[\texttt{t}:]
 A trace object.
\item[\texttt{period}:]
 A period object (optional).
\end{description}%
%
%
%
\item[See also:]%
\hyperlink{ref_period}{\texttt{period}}%
\ (p.~\pageref{ref_period})%
\index[funcref]{@\fidxl{period}}%
, \hyperlink{ref_trace}{\texttt{trace}}%
\ (p.~\pageref{ref_trace})%
\index[funcref]{@\fidxl{trace}}%
%
\item[Author:]%
Cengiz Gunay <cgunay@emory.edu>, 2004/07/30%
\end{description}
\methodline%
\subsubsection[Method \texttt{getDy}]{Method \texttt{trace/getDy}}%
\index[funcref]{trace@\fidxlb{trace}!getDy@\fidxl{getDy}}%
\label{ref_trace__getDy}%
\hypertarget{ref_trace__getDy}{}%
\begin{description}
\item[Summary:]Returns dy.
%
\item[Usage:]~%
\begin{lyxcode}%
dy = getDy(t)
%
\end{lyxcode}%
%
%
\item[Parameters:]~
\begin{description}%
\item[\texttt{t}:]
 A trace object.
\end{description}%
%
\item[Returns:]~

	dy: The dy value.
%
%
\item[See also:]%
\hyperlink{ref_trace}{\texttt{trace}}%
\ (p.~\pageref{ref_trace})%
\index[funcref]{@\fidxl{trace}}%
%
\item[Author:]%
Cengiz Gunay <cgunay@emory.edu>, 2004/08/31%
\end{description}
\methodline%
\subsubsection[Method \texttt{calcMax}]{Method \texttt{trace/calcMax}}%
\index[funcref]{trace@\fidxlb{trace}!calcMax@\fidxl{calcMax}}%
\label{ref_trace__calcMax}%
\hypertarget{ref_trace__calcMax}{}%
\begin{description}
\item[Summary:]Calculates the maximal value of the given period 
 		of the trace, t. 
%
\item[Usage:]~%
\begin{lyxcode}%
[max\_val, max\_idx] = calcMax(t, period)
%
\end{lyxcode}%
%
%
\item[Parameters:]~
\begin{description}%
\item[\texttt{t}:]
 A trace object.
\item[\texttt{period}:]
 A period object (optional).
\end{description}%
%
\item[Returns:]~

	max\_val: The max value.
	max\_idx: Its index in the trace.
%
%
\item[See also:]%
\hyperlink{ref_period}{\texttt{period}}%
\ (p.~\pageref{ref_period})%
\index[funcref]{@\fidxl{period}}%
, \hyperlink{ref_trace}{\texttt{trace}}%
\ (p.~\pageref{ref_trace})%
\index[funcref]{@\fidxl{trace}}%
%
\item[Author:]%
Cengiz Gunay <cgunay@emory.edu>, 2004/07/30%
\end{description}
\methodline%
\subsubsection[Method \texttt{calcMin}]{Method \texttt{trace/calcMin}}%
\index[funcref]{trace@\fidxlb{trace}!calcMin@\fidxl{calcMin}}%
\label{ref_trace__calcMin}%
\hypertarget{ref_trace__calcMin}{}%
\begin{description}
\item[Summary:]Calculates the minimal value of the given period 
 		of the trace, t. 
%
\item[Usage:]~%
\begin{lyxcode}%
[min\_val, min\_idx] = calcMin(t, a\_period)
%
\end{lyxcode}%
%
%
\item[Parameters:]~
\begin{description}%
\item[\texttt{t}:]
 A trace object.
\item[\texttt{a\_period}:]
 A period object (optional).
\end{description}%
%
\item[Returns:]~

	min\_val: The min value.
	min\_idx: Its index in the trace.
%
%
\item[See also:]%
\hyperlink{ref_period}{\texttt{period}}%
\ (p.~\pageref{ref_period})%
\index[funcref]{@\fidxl{period}}%
, \hyperlink{ref_trace}{\texttt{trace}}%
\ (p.~\pageref{ref_trace})%
\index[funcref]{@\fidxl{trace}}%
%
\item[Author:]%
Cengiz Gunay <cgunay@emory.edu>, 2004/07/30%
\end{description}
\methodline%
\subsubsection[Method \texttt{subsref}]{Method \texttt{trace/subsref}}%
\index[funcref]{trace@\fidxlb{trace}!subsref@\fidxl{subsref}}%
\label{ref_trace__subsref}%
\hypertarget{ref_trace__subsref}{}%
\begin{description}
\item[Summary:]Defines generic indexing for objects.
%
%
%
%
%
%
%
\item[Author:]%
Cengiz Gunay <cgunay@emory.edu>, 2004/08/04%
\end{description}
\methodline%
\subsubsection[Method \texttt{plot}]{Method \texttt{trace/plot}}%
\index[funcref]{trace@\fidxlb{trace}!plot@\fidxl{plot}}%
\label{ref_trace__plot}%
\hypertarget{ref_trace__plot}{}%
\begin{description}
\item[Summary:]Plots a trace.
%
\item[Usage:]~%
\begin{lyxcode}%
h = plot(t)
%
\end{lyxcode}%
%
%
\item[Parameters:]~
\begin{description}%
\item[\texttt{t}:]
 A trace object.
\item[\texttt{title\_str}:]
 (Optional) String to append to plot title.
\end{description}%
%
\item[Returns:]~

	h: Handle to figure object.
%
%
\item[See also:]%
\hyperlink{ref_trace}{\texttt{trace}}%
\ (p.~\pageref{ref_trace})%
\index[funcref]{@\fidxl{trace}}%
, \hyperlink{ref_plot_abstract}{\texttt{plot\_abstract}}%
\ (p.~\pageref{ref_plot_abstract})%
\index[funcref]{@\fidxl{plot\_abstract}}%
%
\item[Author:]%
Cengiz Gunay <cgunay@emory.edu>, 2004/08/04%
\end{description}
\methodline%
\subsubsection[Method \texttt{getSpike}]{Method \texttt{trace/getSpike}}%
\index[funcref]{trace@\fidxlb{trace}!getSpike@\fidxl{getSpike}}%
\label{ref_trace__getSpike}%
\hypertarget{ref_trace__getSpike}{}%
\begin{description}
\item[Summary:]Convert a spike in the trace to a spike\_shape object.
%
\item[Usage:]~%
\begin{lyxcode}%
obj = getSpike(trace, spikes, spike\_num, props)
%
\end{lyxcode}%
%
\item[Description:]%
Creates a spike\_shape object from desired spike. It is more efficient if
 you already have the spikes object.
%%
\item[Parameters:]~
\begin{description}%
\item[\texttt{trace}:]
 A trace object.
\item[\texttt{spikes}:]
 (Optional) A spikes object obtained from trace, 

calculated automatically if given as [].\item[\texttt{spike\_num}:]
 The index of spike to extract.
\item[\texttt{props}:]
 A structure with any optional properties.
\begin{description}%
\item[\texttt{spike\_id}:]
 A prefix string added to the spike\_shape object's id.
\end{description}%
\end{description}%
%
%
\item[Example:]~
\begin{lyxcode} This will create an annotated plot of the third spike in my\_trace:\\%
 >> plotFigure(plotResults(getSpike(my\_trace, [], 3)))\\%
\end{lyxcode}
%
\item[See also:]%
\hyperlink{ref_spike_shape}{\texttt{spike\_shape}}%
\ (p.~\pageref{ref_spike_shape})%
\index[funcref]{@\fidxl{spike\_shape}}%
%
\item[Author:]%
Cengiz Gunay <cgunay@emory.edu>, 2005/04/19%
\end{description}
\methodline%
\subsubsection[Method \texttt{withinPeriod}]{Method \texttt{trace/withinPeriod}}%
\index[funcref]{trace@\fidxlb{trace}!withinPeriod@\fidxl{withinPeriod}}%
\label{ref_trace__withinPeriod}%
\hypertarget{ref_trace__withinPeriod}{}%
\begin{description}
\item[Summary:]Returns a trace object valid only within the given period.
%
\item[Usage:]~%
\begin{lyxcode}%
obj = withinPeriod(t, a\_period)
%
\end{lyxcode}%
%
%
\item[Parameters:]~
\begin{description}%
\item[\texttt{t}:]
 A trace object.
\item[\texttt{a\_period}:]
 The desired period
\end{description}%
%
\item[Returns:]~

	obj: A trace object
%
%
\item[See also:]%
\hyperlink{ref_trace}{\texttt{trace}}%
\ (p.~\pageref{ref_trace})%
\index[funcref]{@\fidxl{trace}}%
, \hyperlink{ref_period}{\texttt{period}}%
\ (p.~\pageref{ref_period})%
\index[funcref]{@\fidxl{period}}%
%
\item[Author:]%
Cengiz Gunay <cgunay@emory.edu>, 2004/08/25%
\end{description}
\methodline%
\subsubsection[Method \texttt{getResults}]{Method \texttt{trace/getResults}}%
\index[funcref]{trace@\fidxlb{trace}!getResults@\fidxl{getResults}}%
\label{ref_trace__getResults}%
\hypertarget{ref_trace__getResults}{}%
\begin{description}
\item[Summary:]Runs all tests defined by this class and return them in a 
		structure.
%
\item[Usage:]~%
\begin{lyxcode}%
results = getResults(t)
%
\end{lyxcode}%
%
%
\item[Parameters:]~
\begin{description}%
\item[\texttt{t}:]
 A trace object.
\end{description}%
%
\item[Returns:]~

	results: A structure associating test names to values 
		in ms and mV (or mA).
%
%
\item[See also:]%
\hyperlink{ref_spike_shape}{\texttt{spike\_shape}}%
\ (p.~\pageref{ref_spike_shape})%
\index[funcref]{@\fidxl{spike\_shape}}%
%
\item[Author:]%
Cengiz Gunay <cgunay@emory.edu>, 2004/09/13%
\end{description}
\methodline%
\subsubsection[Method \texttt{spike\_shape}]{Method \texttt{trace/spike\_shape}}%
\index[funcref]{trace@\fidxlb{trace}!spike_shape@\fidxl{spike\_shape}}%
\label{ref_trace__spike_shape}%
\hypertarget{ref_trace__spike_shape}{}%
\begin{description}
\item[Summary:]Convert averaged spikes in the trace to a spike\_shape object.
%
\item[Usage:]~%
\begin{lyxcode}%
obj = spike\_shape(trace, spikes, props)
%
\end{lyxcode}%
%
\item[Description:]%
Creates a spike\_shape object.
%%
\item[Parameters:]~
\begin{description}%
\item[\texttt{trace}:]
 A trace object.
\item[\texttt{spikes}:]
 A spikes object on trace.
\end{description}%
%
%
%
\item[See also:]%
\hyperlink{ref_spike_shape}{\texttt{spike\_shape}}%
\ (p.~\pageref{ref_spike_shape})%
\index[funcref]{@\fidxl{spike\_shape}}%
%
\item[Author:]%
Cengiz Gunay <cgunay@emory.edu>, 2004/08/04%
\end{description}
\methodline%
\subsubsection[Method \texttt{analyzeSpikesInPeriod}]{Method \texttt{trace/analyzeSpikesInPeriod}}%
\index[funcref]{trace@\fidxlb{trace}!analyzeSpikesInPeriod@\fidxl{analyzeSpikesInPeriod}}%
\label{ref_trace__analyzeSpikesInPeriod}%
\hypertarget{ref_trace__analyzeSpikesInPeriod}{}%
\begin{description}
\item[Summary:]Returns results and a db of spikes by collecting test results of a cip\_trace, analyzing each individual spike.
%
\item[Usage:]~%
\begin{lyxcode}%
[results period\_spikes a\_spikes\_db spikes\_stats\_db spikes\_hists\_dbs] =
      analyzeSpikesInPeriod(a\_cip\_trace, a\_spikes, period, prefix\_str)
%
\end{lyxcode}%
%
%
\item[Parameters:]~
\begin{description}%
\item[\texttt{a\_cip\_trace}:]
 A cip\_trace object.
\item[\texttt{a\_spikes}:]
 A spikes object from the a\_cip\_trace object.
\item[\texttt{period}:]
 A period of object of a\_cip\_trace object of interest.
\item[\texttt{prefix\_str}:]
 Prefix string to be added to spike shape results.
\end{description}%
%
\item[Returns:]~

	results: Results structure names prefixed with prefix\_str.
	period\_spikes: Corrected spikes object for this period.
	a\_spikes\_db: A mini spikes database of results from each spike in period. 
	spikes\_stats\_db: Statistics from the mini spikes database.
	spikes\_hists\_dbs: Cell array of histograms from the mini spikes database.
%
%
\item[See also:]%
\hyperlink{ref_cip_trace}{\texttt{cip\_trace}}%
\ (p.~\pageref{ref_cip_trace})%
\index[funcref]{@\fidxl{cip\_trace}}%
, \hyperlink{ref_spikes}{\texttt{spikes}}%
\ (p.~\pageref{ref_spikes})%
\index[funcref]{@\fidxl{spikes}}%
, \hyperlink{ref_period}{\texttt{period}}%
\ (p.~\pageref{ref_period})%
\index[funcref]{@\fidxl{period}}%
, \hyperlink{ref_spike_shape}{\texttt{spike\_shape}}%
\ (p.~\pageref{ref_spike_shape})%
\index[funcref]{@\fidxl{spike\_shape}}%
, \hyperlink{ref_getProfileAllSpikes}{\texttt{getProfileAllSpikes}}%
\ (p.~\pageref{ref_getProfileAllSpikes})%
\index[funcref]{@\fidxl{getProfileAllSpikes}}%
%
\item[Author:]%
Cengiz Gunay <cgunay@emory.edu>, 2005/05/04%
\end{description}
\methodline%
\subsection{Class \texttt{trace\_profile}}%
\index[funcref]{trace_profile@\fidxlb{trace\_profile}}%
\label{ref_trace_profile}%
\hypertarget{ref_trace_profile}{}%
\subsubsection[Constructor \texttt{trace\_profile}]{Constructor \texttt{trace\_profile/trace\_profile}}%
\index[funcref]{trace_profile@\fidxlb{trace\_profile}!trace_profile@\fidxl{trace\_profile}}%
\label{ref_trace_profile__trace_profile}%
\hypertarget{ref_trace_profile__trace_profile}{}%
\begin{description}
\item[Summary:]Creates and collects test results of a trace.
%
%
\item[Description:]%
The first usage is fully customizable to be used from subclass constructors.
 The second usage generates the spikes and spike\_shape objects, and
 collects some generic test results from them. This usage is only provided
 as an example and is not used practically.
%%
\item[Parameters:]~
\begin{description}%
\item[\texttt{data\_src}:]
 The trace column OR the .MAT filename.
\item[\texttt{dt}:]
 Time resolution [s]
\item[\texttt{dy}:]
 y-axis resolution [ISI (V, A, etc.)]
\item[\texttt{props}:]
 See trace object.
\end{description}%
%
\item[Returns a structure object with the following fields:]~

	trace, spikes, spike\_shape, results, id, props.
%
%
\item[See also:]%
\hyperlink{ref_trace}{\texttt{trace}}%
\ (p.~\pageref{ref_trace})%
\index[funcref]{@\fidxl{trace}}%
, \hyperlink{ref_spikes}{\texttt{spikes}}%
\ (p.~\pageref{ref_spikes})%
\index[funcref]{@\fidxl{spikes}}%
, \hyperlink{ref_spike_shape}{\texttt{spike\_shape}}%
\ (p.~\pageref{ref_spike_shape})%
\index[funcref]{@\fidxl{spike\_shape}}%
%
\item[Author:]%
Cengiz Gunay <cgunay@emory.edu>, 2004/09/13%
\end{description}
\methodline%
\subsubsection[Method \texttt{get}]{Method \texttt{trace\_profile/get}}%
\index[funcref]{trace_profile@\fidxlb{trace\_profile}!get@\fidxl{get}}%
\label{ref_trace_profile__get}%
\hypertarget{ref_trace_profile__get}{}%
\begin{description}
\item[Summary:]Defines generic attribute retrieval for objects.
%
%
%
%
%
%
%
\item[Author:]%
Cengiz Gunay <cgunay@emory.edu>, 2004/09/14%
\end{description}
\methodline%
\subsection{Utility functions}%
\label{ref_utils}%
\hypertarget{ref_utils}{}%
\subsubsection[Function \texttt{readNeuronVecBin}]{Function \texttt{utils/readNeuronVecBin}}%
\index[funcref]{utils@\fidxlb{utils}!readNeuronVecBin@\fidxl{readNeuronVecBin}}%
\label{ref_utils__readNeuronVecBin}%
\hypertarget{ref_utils__readNeuronVecBin}{}%
\begin{description}
%
%
%
%
%
%
%
\item[Author:]%
Konstantin Miller <miller@cs.tu-berlin.de>, Aug 09, 2005.%
\end{description}
\methodline%
\subsubsection[Function \texttt{plotImage}]{Function \texttt{utils/plotImage}}%
\index[funcref]{utils@\fidxlb{utils}!plotImage@\fidxl{plotImage}}%
\label{ref_utils__plotImage}%
\hypertarget{ref_utils__plotImage}{}%
\begin{description}
\item[Summary:]Function that plots a color matrix on current figure.
%
\item[Usage:]~%
\begin{lyxcode}%
h = plotImage(image\_data, colormap\_func, num\_colors, props)
%
\end{lyxcode}%
%
%
\item[Parameters:]~
\begin{description}%
\item[\texttt{image\_data}:]
 2D matrix with image data.
\item[\texttt{colormap\_func}:]
 Function name or handle to colormap (e.g., 'jet').
\item[\texttt{num\_colors}:]
 Parameter to be passed to the colormap\_func.
\item[\texttt{props}:]
 A structure with any optional properties.
\begin{description}%
\item[\texttt{colorbar}:]
 If defined, show colorbar on plot.
\end{description}%
\end{description}%
%
\item[Returns:]~

	colors: RGB color matrix to be passed to colormap function.
%
%
\item[See also:]%
\hyperlink{ref_colormap}{\texttt{colormap}}%
\ (p.~\pageref{ref_colormap})%
\index[funcref]{@\fidxl{colormap}}%
%
\item[Author:]%
Cengiz Gunay <cgunay@emory.edu>, 2006/06/05%
\end{description}
\methodline%
\subsubsection[Function \texttt{colormapBlueCrossRed}]{Function \texttt{utils/colormapBlueCrossRed}}%
\index[funcref]{utils@\fidxlb{utils}!colormapBlueCrossRed@\fidxl{colormapBlueCrossRed}}%
\label{ref_utils__colormapBlueCrossRed}%
\hypertarget{ref_utils__colormapBlueCrossRed}{}%
\begin{description}
\item[Summary:]Blue to red crossing colormap, with a black-colored zero-crossing.
%
\item[Usage:]~%
\begin{lyxcode}%
colors = colormapBlueCrossRed(num\_half\_colors)
%
\end{lyxcode}%
%
\item[Description:]%
Colormap contains (2 * num\_half\_colors + 1) colors, where (num\_half\_colors + 1) is the 
 zero crossing.
%%
\item[Parameters:]~
\begin{description}%
\item[\texttt{num\_half\_colors}:]
 Number of colors to generate on one of the red or blue scales.
\item[\texttt{props}:]
 A structure with any optional properties.
\end{description}%
%
\item[Returns:]~

	colors: RGB color matrix to be passed to colormap function.
%
%
\item[See also:]%
\hyperlink{ref_colormap}{\texttt{colormap}}%
\ (p.~\pageref{ref_colormap})%
\index[funcref]{@\fidxl{colormap}}%
%
\item[Author:]%
Cengiz Gunay <cgunay@emory.edu>, 2006/06/05%
\end{description}
\methodline%

