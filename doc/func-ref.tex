\newcommand{\fidxl}[1]{{\small \texttt{#1}}}
\newcommand{\fidxlb}[1]{{\small \bf \texttt{#1}}}
\newcommand{\boldhyperpage}[1]{\textbf{\hyperpage{#1}}}
\newcommand{\methodline}{%
  {\normalsize \vspace{1ex} \hrule width \columnwidth \vspace{1ex}}%
}%
\subsection{Class \texttt{chans\_db}}%
\index[funcref]{chans_db@\fidxl{chans\_db}|boldhyperpage}%
\label{ref_chans_db}%
\hypertarget{ref_chans_db}{}%
\subsubsection[Constructor \texttt{chans\_db}]{Constructor \texttt{chans\_db/chans\_db}}%
\index[funcref]{chans_db@\fidxl{chans\_db}!chans_db@\fidxl{chans\_db}}%
\label{ref_chans_db__chans_db}%
\hypertarget{ref_chans_db__chans_db}{}%
\begin{description}
\item[Summary:]A database of channel activation and kinetics.
%
\item[Usage:]~%
\begin{lyxcode}%
a\_chans\_db = chans\_db(data, col\_names, channel\_info, id, props)
%
\end{lyxcode}%
%
\item[Description:]%
This is a subclass of tests\_db. Channel tables can be imported from
 Genesis using the utils/chanTables2DB script.
%%
\item[Parameters:]~
\begin{description}%
\item[\texttt{data}:]
 Database contents.
\item[\texttt{col\_names}:]
 The channel variable names.
\item[\texttt{channel\_info}:]
 Structure that holds scalar data elements such as Gbar.
\item[\texttt{id}:]
 An identifying string.
\item[\texttt{props}:]
 A structure with any optional properties.
\end{description}%
%
\item[Returns a structure object with the following fields:
]~

	tests\_db, channel\_info, props.
%
%
\item[See also:]%
\hyperlink{ref_tests_db}{\texttt{tests\_db}}%
\ (p.~\pageref{ref_tests_db})%
\index[funcref]{tests_db@\fidxl{tests\_db}}%
, \hyperlink{ref_chanTables2DB}{\texttt{chanTables2DB}}%
\ (p.~\pageref{ref_chanTables2DB})%
\index[funcref]{chanTables2DB@\fidxl{chanTables2DB}}%
%
\item[Author:]%
Cengiz Gunay <cgunay@emory.edu>, 2007/06/26
%
\end{description}
\methodline%
\subsubsection[Method \texttt{display}]{Method \texttt{chans\_db/display}}%
\index[funcref]{chans_db@\fidxl{chans\_db}!display@\fidxl{display}}%
\label{ref_chans_db__display}%
\hypertarget{ref_chans_db__display}{}%
\begin{description}
%
%
%
%
%
%
%
\item[Author:]%
Cengiz Gunay <cgunay@emory.edu>, 2004/08/04
%
\end{description}
\methodline%
\subsubsection[Method \texttt{get}]{Method \texttt{chans\_db/get}}%
\index[funcref]{chans_db@\fidxl{chans\_db}!get@\fidxl{get}}%
\label{ref_chans_db__get}%
\hypertarget{ref_chans_db__get}{}%
\begin{description}
\item[Summary:]Defines generic attribute retrieval for objects.
%
%
%
%
%
%
%
\item[Author:]%
Cengiz Gunay <cgunay@emory.edu>, 2004/09/14
%
\end{description}
\methodline%
\subsubsection[Method \texttt{plotAllInf}]{Method \texttt{chans\_db/plotAllInf}}%
\index[funcref]{chans_db@\fidxl{chans\_db}!plotAllInf@\fidxl{plotAllInf}}%
\label{ref_chans_db__plotAllInf}%
\hypertarget{ref_chans_db__plotAllInf}{}%
\begin{description}
\item[Summary:]Plots the steady-state (infinity) response of all channels.
%
\item[Usage:]~%
\begin{lyxcode}%
a\_plot = plotAllInf(a\_chans\_db, title\_str, props)
%
\end{lyxcode}%
%
%
\item[Parameters:]~
\begin{description}%
\item[\texttt{a\_chans\_db}:]
 a chans\_db
\item[\texttt{title\_str}:]
 Plot title.
\item[\texttt{props}:]
 A structure with any optional properties.

(rest passed to matrixPlots.)
\end{description}%
%
\item[Returns:
]~

	a\_plot: A plot\_abstract object that can be visualized.
%
%
\item[See also:]%
\hyperlink{ref_trace}{\texttt{trace}}%
\ (p.~\pageref{ref_trace})%
\index[funcref]{trace@\fidxl{trace}}%
, \hyperlink{ref_trace__plot}{\texttt{trace/plot}}%
\ (p.~\pageref{ref_trace__plot})%
\index[funcref]{trace@\fidxl{trace}!plot@\fidxl{plot}}%
, \hyperlink{ref_plot_abstract}{\texttt{plot\_abstract}}%
\ (p.~\pageref{ref_plot_abstract})%
\index[funcref]{plot_abstract@\fidxl{plot\_abstract}}%
%
\item[Author:]%
Cengiz Gunay <cgunay@emory.edu>, 2007/03/05
%
\end{description}
\methodline%
\subsubsection[Method \texttt{plotAllVars}]{Method \texttt{chans\_db/plotAllVars}}%
\index[funcref]{chans_db@\fidxl{chans\_db}!plotAllVars@\fidxl{plotAllVars}}%
\label{ref_chans_db__plotAllVars}%
\hypertarget{ref_chans_db__plotAllVars}{}%
\begin{description}
\item[Summary:]Plot all channel variables by grouping activation and time constant curves per channel.
%
\item[Usage:]~%
\begin{lyxcode}%
a\_plot = plotAllVars(a\_chans\_db, title\_str, props)
%
\end{lyxcode}%
%
%
\item[Parameters:]~
\begin{description}%
\item[\texttt{a\_chans\_db}:]
 A chans\_db describing channel variables.
\item[\texttt{id}:]
 String that identify the source of the tables structure.
\item[\texttt{props}:]
 A structure with any optional properties.

(rest passed to plot\_abstract.)
\end{description}%
%
\item[Returns:
]~

	a\_plot: A plot\_abstract object that can be visualized.
%
%
\item[See also:]%
\hyperlink{ref_trace}{\texttt{trace}}%
\ (p.~\pageref{ref_trace})%
\index[funcref]{trace@\fidxl{trace}}%
, \hyperlink{ref_trace__plot}{\texttt{trace/plot}}%
\ (p.~\pageref{ref_trace__plot})%
\index[funcref]{trace@\fidxl{trace}!plot@\fidxl{plot}}%
, \hyperlink{ref_plot_abstract}{\texttt{plot\_abstract}}%
\ (p.~\pageref{ref_plot_abstract})%
\index[funcref]{plot_abstract@\fidxl{plot\_abstract}}%
%
\item[Author:]%
Cengiz Gunay <cgunay@emory.edu>, 2007/03/05
%
\end{description}
\methodline%
\subsubsection[Method \texttt{plotGateVars}]{Method \texttt{chans\_db/plotGateVars}}%
\index[funcref]{chans_db@\fidxl{chans\_db}!plotGateVars@\fidxl{plotGateVars}}%
\label{ref_chans_db__plotGateVars}%
\hypertarget{ref_chans_db__plotGateVars}{}%
\begin{description}
\item[Summary:]Plot given channel gate variables of the same channel superposed.
%
\item[Usage:]~%
\begin{lyxcode}%
a\_plot = plotGateVars(a\_chans\_db, chan\_name, gate\_subnames, title\_str, props)
%
\end{lyxcode}%
%
%
\item[Parameters:]~
\begin{description}%
\item[\texttt{a\_chans\_db}:]
 A chans\_db describing channel variables.
\item[\texttt{chan\_name}:]
 Name of channel that make up the stem of variable

names.
\item[\texttt{gate\_subnames}:]
 Gate names of the channel.
\item[\texttt{title\_str}:]
 (Optional) A string to be concatanated to the title.
\item[\texttt{props}:]
 A structure with any optional properties.
\begin{description}%
\item[\texttt{usePowers}:]
 Use the gate powers, Luke.

(rest passed to plot\_abstract.)
\end{description}%
\end{description}%
%
\item[Returns:
]~

	a\_plot: A plot\_abstract object that can be visualized.
%
%
\item[See also:]%
\hyperlink{ref_trace}{\texttt{trace}}%
\ (p.~\pageref{ref_trace})%
\index[funcref]{trace@\fidxl{trace}}%
, \hyperlink{ref_trace__plot}{\texttt{trace/plot}}%
\ (p.~\pageref{ref_trace__plot})%
\index[funcref]{trace@\fidxl{trace}!plot@\fidxl{plot}}%
, \hyperlink{ref_plot_abstract}{\texttt{plot\_abstract}}%
\ (p.~\pageref{ref_plot_abstract})%
\index[funcref]{plot_abstract@\fidxl{plot\_abstract}}%
%
\item[Author:]%
Cengiz Gunay <cgunay@emory.edu>, 2007/07/01
%
\end{description}
\methodline%
\subsubsection[Method \texttt{plotInf}]{Method \texttt{chans\_db/plotInf}}%
\index[funcref]{chans_db@\fidxl{chans\_db}!plotInf@\fidxl{plotInf}}%
\label{ref_chans_db__plotInf}%
\hypertarget{ref_chans_db__plotInf}{}%
\begin{description}
\item[Summary:]Plot the product of minf variables and the gmax of the given channel.
%
\item[Usage:]~%
\begin{lyxcode}%
a\_plot = plotInf(a\_chans\_db, chan\_name, gate\_subnames, title\_str, props)
%
\end{lyxcode}%
%
%
\item[Parameters:]~
\begin{description}%
\item[\texttt{a\_chans\_db}:]
 A chans\_db describing channel variables.
\item[\texttt{chan\_name}:]
 Name of channel that make up the stem of variable

names.
\item[\texttt{gate\_subnames}:]
 Gate names of the channel.
\item[\texttt{title\_str}:]
 (Optional) A string to be concatanated to the title.
\item[\texttt{props}:]
 A structure with any optional properties.

(rest passed to plot\_abstract.)
\end{description}%
%
\item[Returns:
]~

	a\_plot: A plot\_abstract object that can be visualized.
%
%
\item[See also:]%
\hyperlink{ref_trace}{\texttt{trace}}%
\ (p.~\pageref{ref_trace})%
\index[funcref]{trace@\fidxl{trace}}%
, \hyperlink{ref_trace__plot}{\texttt{trace/plot}}%
\ (p.~\pageref{ref_trace__plot})%
\index[funcref]{trace@\fidxl{trace}!plot@\fidxl{plot}}%
, \hyperlink{ref_plot_abstract}{\texttt{plot\_abstract}}%
\ (p.~\pageref{ref_plot_abstract})%
\index[funcref]{plot_abstract@\fidxl{plot\_abstract}}%
%
\item[Author:]%
Cengiz Gunay <cgunay@emory.edu>, 2007/07/01
%
\end{description}
\methodline%
\subsubsection[Method \texttt{set}]{Method \texttt{chans\_db/set}}%
\index[funcref]{chans_db@\fidxl{chans\_db}!set@\fidxl{set}}%
\label{ref_chans_db__set}%
\hypertarget{ref_chans_db__set}{}%
\begin{description}
\item[Summary:]Generic method for setting object attributes.
%
%
%
%
%
%
%
\item[Author:]%
Cengiz Gunay <cgunay@emory.edu>, 2004/10/08
%
\end{description}
\methodline%
\subsection{Class \texttt{cip\_trace}}%
\index[funcref]{cip_trace@\fidxl{cip\_trace}|boldhyperpage}%
\label{ref_cip_trace}%
\hypertarget{ref_cip_trace}{}%
\subsubsection[Constructor \texttt{cip\_trace}]{Constructor \texttt{cip\_trace/cip\_trace}}%
\index[funcref]{cip_trace@\fidxl{cip\_trace}!cip_trace@\fidxl{cip\_trace}}%
\label{ref_cip_trace__cip_trace}%
\hypertarget{ref_cip_trace__cip_trace}{}%
\begin{description}
\item[Summary:]A trace with a current injection pulse (CIP).
%
\item[Usage:]~%
\begin{lyxcode}%
obj = cip\_trace(datasrc, dt, dy,
		  pulse\_time\_start, pulse\_time\_width, id, props)
%
\end{lyxcode}%
%
%
\item[Parameters:]~
\begin{description}%
\item[\texttt{datasrc}:]
 A vector of data points containing the spike shape.
\item[\texttt{dt}:]
 Time resolution [s].
\item[\texttt{dy}:]
 y-axis resolution [ISI (V, A, etc.)]
\item[\texttt{pulse\_time\_start, pulse\_time\_width}:]


Start and width of the pulse [dt]
\item[\texttt{id}:]
 Identification string.
\item[\texttt{props}:]
 A structure with any optional properties, such as:
\begin{description}%
\item[\texttt{trace\_time\_start}:]
 Samples in the beginning to discard [dt]

(see trace for more)
\end{description}%
\end{description}%
%
\item[Returns a structure object with the following fields:
]~

	trace, pulse\_time\_start, pulse\_time\_width, props.
%
%
\item[See also:]%
\hyperlink{ref_trace}{\texttt{trace}}%
\ (p.~\pageref{ref_trace})%
\index[funcref]{trace@\fidxl{trace}}%
, \hyperlink{ref_spikes}{\texttt{spikes}}%
\ (p.~\pageref{ref_spikes})%
\index[funcref]{spikes@\fidxl{spikes}}%
, \hyperlink{ref_spike_shape}{\texttt{spike\_shape}}%
\ (p.~\pageref{ref_spike_shape})%
\index[funcref]{spike_shape@\fidxl{spike\_shape}}%
, \hyperlink{ref_period}{\texttt{period}}%
\ (p.~\pageref{ref_period})%
\index[funcref]{period@\fidxl{period}}%
%
\item[Author:]%
Cengiz Gunay <cgunay@emory.edu>, 2004/07/30
%
\end{description}
\methodline%
\subsubsection[Method \texttt{calcPulsePotAvg}]{Method \texttt{cip\_trace/calcPulsePotAvg}}%
\index[funcref]{cip_trace@\fidxl{cip\_trace}!calcPulsePotAvg@\fidxl{calcPulsePotAvg}}%
\label{ref_cip_trace__calcPulsePotAvg}%
\hypertarget{ref_cip_trace__calcPulsePotAvg}{}%
\begin{description}
\item[Summary:]Calculates the average potential value of the 
		CIP period of the cip\_trace, t. 
%
\item[Usage:]~%
\begin{lyxcode}%
avg\_val = calcPulsePotAvg(t)
%
\end{lyxcode}%
%
%
\item[Parameters:]~
\begin{description}%
\item[\texttt{t}:]
 A cip\_trace object.
\end{description}%
%
\item[Returns:
]~

	avg\_val: The avg value [dy].
%
%
\item[See also:]%
\hyperlink{ref_period}{\texttt{period}}%
\ (p.~\pageref{ref_period})%
\index[funcref]{period@\fidxl{period}}%
, \hyperlink{ref_trace}{\texttt{trace}}%
\ (p.~\pageref{ref_trace})%
\index[funcref]{trace@\fidxl{trace}}%
, \hyperlink{ref_trace__calcAvg}{\texttt{trace/calcAvg}}%
\ (p.~\pageref{ref_trace__calcAvg})%
\index[funcref]{trace@\fidxl{trace}!calcAvg@\fidxl{calcAvg}}%
%
\item[Author:]%
Cengiz Gunay <cgunay@emory.edu>, 2004/08/25
%
\end{description}
\methodline%
\subsubsection[Method \texttt{calcPulsePotSag}]{Method \texttt{cip\_trace/calcPulsePotSag}}%
\index[funcref]{cip_trace@\fidxl{cip\_trace}!calcPulsePotSag@\fidxl{calcPulsePotSag}}%
\label{ref_cip_trace__calcPulsePotSag}%
\hypertarget{ref_cip_trace__calcPulsePotSag}{}%
\begin{description}
\item[Summary:]Calculates the minimal sag and sag amount of the CIP period of the cip\_trace, t. 
%
\item[Usage:]~%
\begin{lyxcode}%
[min\_val, min\_idx, sag\_val] = calcPulsePotSag(t)
%
\end{lyxcode}%
%
\item[Description:]%
The minimal sag is the minimal potential value of the 
 first half of the CIP period. The sag amount is calculated by 
 comparing this to the steady-state value at the end of the CIP period.
%%
\item[Parameters:]~
\begin{description}%
\item[\texttt{t}:]
 A cip\_trace object.
\end{description}%
%
\item[Returns:
]~

	min\_val: The min value [dy].
	min\_idx: The index of the min value [dt].
	sag\_val: The sag amount [dy].
%
%
\item[See also:]%
\hyperlink{ref_period}{\texttt{period}}%
\ (p.~\pageref{ref_period})%
\index[funcref]{period@\fidxl{period}}%
, \hyperlink{ref_trace}{\texttt{trace}}%
\ (p.~\pageref{ref_trace})%
\index[funcref]{trace@\fidxl{trace}}%
, \hyperlink{ref_trace__calcMin}{\texttt{trace/calcMin}}%
\ (p.~\pageref{ref_trace__calcMin})%
\index[funcref]{trace@\fidxl{trace}!calcMin@\fidxl{calcMin}}%
%
\item[Author:]%
Cengiz Gunay <cgunay@emory.edu>, 2004/08/25
%
\end{description}
\methodline%
\subsubsection[Method \texttt{calcRecSpontPotAvg}]{Method \texttt{cip\_trace/calcRecSpontPotAvg}}%
\index[funcref]{cip_trace@\fidxl{cip\_trace}!calcRecSpontPotAvg@\fidxl{calcRecSpontPotAvg}}%
\label{ref_cip_trace__calcRecSpontPotAvg}%
\hypertarget{ref_cip_trace__calcRecSpontPotAvg}{}%
\begin{description}
\item[Summary:]Calculates the average potential value of the 
			recovery period of the cip\_trace, t. 
%
\item[Usage:]~%
\begin{lyxcode}%
avg\_val = calcRecSpontPotAvg(t)
%
\end{lyxcode}%
%
%
\item[Parameters:]~
\begin{description}%
\item[\texttt{t}:]
 A cip\_trace object.
\end{description}%
%
\item[Returns:
]~

	avg\_val: The avg value [dy].
%
%
\item[See also:]%
\hyperlink{ref_period}{\texttt{period}}%
\ (p.~\pageref{ref_period})%
\index[funcref]{period@\fidxl{period}}%
, \hyperlink{ref_trace}{\texttt{trace}}%
\ (p.~\pageref{ref_trace})%
\index[funcref]{trace@\fidxl{trace}}%
, \hyperlink{ref_trace__calcAvg}{\texttt{trace/calcAvg}}%
\ (p.~\pageref{ref_trace__calcAvg})%
\index[funcref]{trace@\fidxl{trace}!calcAvg@\fidxl{calcAvg}}%
%
\item[Author:]%
Cengiz Gunay <cgunay@emory.edu>, 2004/08/25
%
\end{description}
\methodline%
\subsubsection[Method \texttt{display}]{Method \texttt{cip\_trace/display}}%
\index[funcref]{cip_trace@\fidxl{cip\_trace}!display@\fidxl{display}}%
\label{ref_cip_trace__display}%
\hypertarget{ref_cip_trace__display}{}%
\begin{description}
%
%
%
%
%
%
%
\item[Author:]%
Cengiz Gunay <cgunay@emory.edu>, 2004/08/04
%
\end{description}
\methodline%
\subsubsection[Method \texttt{get}]{Method \texttt{cip\_trace/get}}%
\index[funcref]{cip_trace@\fidxl{cip\_trace}!get@\fidxl{get}}%
\label{ref_cip_trace__get}%
\hypertarget{ref_cip_trace__get}{}%
\begin{description}
\item[Summary:]Defines generic attribute retrieval for objects.
%
%
%
%
%
%
%
\item[Author:]%
Cengiz Gunay <cgunay@emory.edu>, 2004/09/14
%
\end{description}
\methodline%
\subsubsection[Method \texttt{getBurstResults}]{Method \texttt{cip\_trace/getBurstResults}}%
\index[funcref]{cip_trace@\fidxl{cip\_trace}!getBurstResults@\fidxl{getBurstResults}}%
\label{ref_cip_trace__getBurstResults}%
\hypertarget{ref_cip_trace__getBurstResults}{}%
\begin{description}
\item[Summary:]Calculate test results related to Burst behavior.
%
\item[Usage:]~%
\begin{lyxcode}%
results = getRateResults(a\_cip\_trace, a\_spikes)
%
\end{lyxcode}%
%
%
\item[Parameters:]~
\begin{description}%
\item[\texttt{a\_cip\_trace}:]
 A cip\_trace object.
\item[\texttt{a\_spikes}:]
 A spikes object.
\end{description}%
%
\item[Returns:
]~

	results: A structure associating test names with result values.
%
%
\item[See also:]%
\hyperlink{ref_cip_trace}{\texttt{cip\_trace}}%
\ (p.~\pageref{ref_cip_trace})%
\index[funcref]{cip_trace@\fidxl{cip\_trace}}%
, \hyperlink{ref_spikes}{\texttt{spikes}}%
\ (p.~\pageref{ref_spikes})%
\index[funcref]{spikes@\fidxl{spikes}}%
, \hyperlink{ref_spike_shape}{\texttt{spike\_shape}}%
\ (p.~\pageref{ref_spike_shape})%
\index[funcref]{spike_shape@\fidxl{spike\_shape}}%
%
\item[Author:]%
Cengiz Gunay <cgunay@emory.edu>, 2004/08/30, Tom Sangrey
%
\end{description}
\methodline%
\subsubsection[Method \texttt{getCIPResults}]{Method \texttt{cip\_trace/getCIPResults}}%
\index[funcref]{cip_trace@\fidxl{cip\_trace}!getCIPResults@\fidxl{getCIPResults}}%
\label{ref_cip_trace__getCIPResults}%
\hypertarget{ref_cip_trace__getCIPResults}{}%
\begin{description}
\item[Summary:]Calculate test results about CIP protocol.
%
\item[Usage:]~%
\begin{lyxcode}%
results = getCIPResults(a\_cip\_trace, a\_spikes)
%
\end{lyxcode}%
%
%
\item[Parameters:]~
\begin{description}%
\item[\texttt{a\_cip\_trace}:]
 A cip\_trace object.
\item[\texttt{a\_spikes}:]
 A spikes object.
\end{description}%
%
\item[Returns:
]~

	results: A structure associating test names with result values.
%
%
\item[See also:]%
\hyperlink{ref_cip_trace}{\texttt{cip\_trace}}%
\ (p.~\pageref{ref_cip_trace})%
\index[funcref]{cip_trace@\fidxl{cip\_trace}}%
, \hyperlink{ref_spikes}{\texttt{spikes}}%
\ (p.~\pageref{ref_spikes})%
\index[funcref]{spikes@\fidxl{spikes}}%
, \hyperlink{ref_spike_shape}{\texttt{spike\_shape}}%
\ (p.~\pageref{ref_spike_shape})%
\index[funcref]{spike_shape@\fidxl{spike\_shape}}%
%
\item[Author:]%
Cengiz Gunay <cgunay@emory.edu>, 2004/08/30
%
\end{description}
\methodline%
\subsubsection[Method \texttt{getProfileAllSpikes}]{Method \texttt{cip\_trace/getProfileAllSpikes}}%
\index[funcref]{cip_trace@\fidxl{cip\_trace}!getProfileAllSpikes@\fidxl{getProfileAllSpikes}}%
\label{ref_cip_trace__getProfileAllSpikes}%
\hypertarget{ref_cip_trace__getProfileAllSpikes}{}%
\begin{description}
\item[Summary:]Creates a cip\_trace\_allspikes\_profile object by collecting test results of a cip\_trace, analyzing each individual spike.
%
\item[Usage:]~%
\begin{lyxcode}%
profile\_obj = getProfileAllSpikes(a\_cip\_trace)
%
\end{lyxcode}%
%
\item[Description:]%
Analyzes the spontaneous (periodIniSpont), pulse (periodPulse) and the
 recovery (periodRecSpont) periods separately and produces spike shape
 distribution results. Rate and CIP measurements are added to these.
%%
\item[Parameters:]~
\begin{description}%
\item[\texttt{a\_cip\_trace}:]
 A cip\_trace object.
\end{description}%
%
\item[Returns:
]~

	profile\_obj: A cip\_trace\_allspikes\_profile object.
%
%
\item[See also:]%
\hyperlink{ref_cip_trace}{\texttt{cip\_trace}}%
\ (p.~\pageref{ref_cip_trace})%
\index[funcref]{cip_trace@\fidxl{cip\_trace}}%
, \hyperlink{ref_cip_trace_allspikes_profile}{\texttt{cip\_trace\_allspikes\_profile}}%
\ (p.~\pageref{ref_cip_trace_allspikes_profile})%
\index[funcref]{cip_trace_allspikes_profile@\fidxl{cip\_trace\_allspikes\_profile}}%
%
\item[Author:]%
Cengiz Gunay <cgunay@emory.edu>, 2005/04/26
%
\end{description}
\methodline%
\subsubsection[Method \texttt{getPulseSpike}]{Method \texttt{cip\_trace/getPulseSpike}}%
\index[funcref]{cip_trace@\fidxl{cip\_trace}!getPulseSpike@\fidxl{getPulseSpike}}%
\label{ref_cip_trace__getPulseSpike}%
\hypertarget{ref_cip_trace__getPulseSpike}{}%
\begin{description}
\item[Summary:]Convert a spike in the CIP period to a spike\_shape object.
%
\item[Usage:]~%
\begin{lyxcode}%
obj = getPulseSpike(trace, spikes, spike\_num, props)
%
\end{lyxcode}%
%
\item[Description:]%
Creates a spike\_shape object from desired spike. Calls trace/getSpike method.
%%
\item[Parameters:]~
\begin{description}%
\item[\texttt{trace}:]
 A trace object.
\item[\texttt{spikes}:]
 (Optional) A spikes object obtained from trace, 

calculated automatically if given as [].
\item[\texttt{spike\_num}:]
 The index of spike to extract.
\item[\texttt{props}:]
 A structure with any optional properties passed to getSpike.
\end{description}%
%
%
\item[Example:]~
\begin{lyxcode} Get 2nd pulse spike and plot it:
\\%
 >> plotFigure(plotResults(getPulseSpike(t, [], 2)))
\\%
\end{lyxcode}
%
\item[See also:]%
\hyperlink{ref_spike_shape}{\texttt{spike\_shape}}%
\ (p.~\pageref{ref_spike_shape})%
\index[funcref]{spike_shape@\fidxl{spike\_shape}}%
, \hyperlink{ref_trace__getSpike}{\texttt{trace/getSpike}}%
\ (p.~\pageref{ref_trace__getSpike})%
\index[funcref]{trace@\fidxl{trace}!getSpike@\fidxl{getSpike}}%
%
\item[Author:]%
Cengiz Gunay <cgunay@emory.edu>, 2005/04/19
%
\end{description}
\methodline%
\subsubsection[Method \texttt{getRateResults}]{Method \texttt{cip\_trace/getRateResults}}%
\index[funcref]{cip_trace@\fidxl{cip\_trace}!getRateResults@\fidxl{getRateResults}}%
\label{ref_cip_trace__getRateResults}%
\hypertarget{ref_cip_trace__getRateResults}{}%
\begin{description}
\item[Summary:]Calculate test results related to spike rate.
%
\item[Usage:]~%
\begin{lyxcode}%
results = getRateResults(a\_cip\_trace, a\_spikes)
%
\end{lyxcode}%
%
%
\item[Parameters:]~
\begin{description}%
\item[\texttt{a\_cip\_trace}:]
 A cip\_trace object.
\item[\texttt{a\_spikes}:]
 A spikes object.
\end{description}%
%
\item[Returns:
]~

	results: A structure associating test names with result values.
%
%
\item[See also:]%
\hyperlink{ref_cip_trace}{\texttt{cip\_trace}}%
\ (p.~\pageref{ref_cip_trace})%
\index[funcref]{cip_trace@\fidxl{cip\_trace}}%
, \hyperlink{ref_spikes}{\texttt{spikes}}%
\ (p.~\pageref{ref_spikes})%
\index[funcref]{spikes@\fidxl{spikes}}%
, \hyperlink{ref_spike_shape}{\texttt{spike\_shape}}%
\ (p.~\pageref{ref_spike_shape})%
\index[funcref]{spike_shape@\fidxl{spike\_shape}}%
%
\item[Author:]%
Cengiz Gunay <cgunay@emory.edu>, 2004/08/30
%
\end{description}
\methodline%
\subsubsection[Method \texttt{getRecSpontSpike}]{Method \texttt{cip\_trace/getRecSpontSpike}}%
\index[funcref]{cip_trace@\fidxl{cip\_trace}!getRecSpontSpike@\fidxl{getRecSpontSpike}}%
\label{ref_cip_trace__getRecSpontSpike}%
\hypertarget{ref_cip_trace__getRecSpontSpike}{}%
\begin{description}
\item[Summary:]Convert a spike in the CIP period to a spike\_shape object.
%
\item[Usage:]~%
\begin{lyxcode}%
obj = getRecSpontSpike(trace, spikes, spike\_num, props)
%
\end{lyxcode}%
%
\item[Description:]%
Creates a spike\_shape object from desired spike.
%%
\item[Parameters:]~
\begin{description}%
\item[\texttt{trace}:]
 A trace object.
\item[\texttt{spikes}:]
 A spikes object on trace.
\item[\texttt{spike\_num}:]
 The index of spike to extract.
\end{description}%
%
%
%
\item[See also:]%
\hyperlink{ref_spike_shape}{\texttt{spike\_shape}}%
\ (p.~\pageref{ref_spike_shape})%
\index[funcref]{spike_shape@\fidxl{spike\_shape}}%
%
\item[Author:]%
Cengiz Gunay <cgunay@emory.edu>, 2005/05/08
%
\end{description}
\methodline%
\subsubsection[Method \texttt{getResults}]{Method \texttt{cip\_trace/getResults}}%
\index[funcref]{cip_trace@\fidxl{cip\_trace}!getResults@\fidxl{getResults}}%
\label{ref_cip_trace__getResults}%
\hypertarget{ref_cip_trace__getResults}{}%
\begin{description}
\item[Summary:]Calculate test results given a\_spikes object.
%
\item[Usage:]~%
\begin{lyxcode}%
results = getResults(a\_cip\_trace, a\_spikes)
%
\end{lyxcode}%
%
%
\item[Parameters:]~
\begin{description}%
\item[\texttt{a\_cip\_trace}:]
 A cip\_trace object.
\item[\texttt{a\_spikes}:]
 A spikes object.
\end{description}%
%
\item[Returns:
]~

	results: A structure associating test names with result values.
%
%
\item[See also:]%
\hyperlink{ref_cip_trace}{\texttt{cip\_trace}}%
\ (p.~\pageref{ref_cip_trace})%
\index[funcref]{cip_trace@\fidxl{cip\_trace}}%
, \hyperlink{ref_spikes}{\texttt{spikes}}%
\ (p.~\pageref{ref_spikes})%
\index[funcref]{spikes@\fidxl{spikes}}%
, \hyperlink{ref_spike_shape}{\texttt{spike\_shape}}%
\ (p.~\pageref{ref_spike_shape})%
\index[funcref]{spike_shape@\fidxl{spike\_shape}}%
%
\item[Author:]%
Cengiz Gunay <cgunay@emory.edu>, 2004/09/14
%
\end{description}
\methodline%
\subsubsection[Method \texttt{measureNames}]{Method \texttt{cip\_trace/measureNames}}%
\index[funcref]{cip_trace@\fidxl{cip\_trace}!measureNames@\fidxl{measureNames}}%
\label{ref_cip_trace__measureNames}%
\hypertarget{ref_cip_trace__measureNames}{}%
\begin{description}
\item[Summary:]Returns taxonomy of measurements collected by cip\_trace.
%
\item[Usage:]~%
\begin{lyxcode}%
measures = measureNames(a\_cip\_trace)
%
\end{lyxcode}%
%
\item[Description:]%
This is a static method, in the sense that it does not need the object passed as argument.
 Therefore it can be called directly by using the default constructor; e.g., measureNames(cip\_trace).
 The measure names are required for merging columns of a database generated by profiling these objects.
%%
\item[Parameters:]~
\begin{description}%
\item[\texttt{a\_cip\_trace}:]
 A cip\_trace object. It can be created by the the default constructor 'cip\_trace'.
\end{description}%
%
\item[Returns:
]~

	measures: A structure with cell arrays of types of measures, and measure names inside.
%
%
\item[See also:]%
\hyperlink{ref_getResults}{\texttt{getResults}}%
\ (p.~\pageref{ref_getResults})%
\index[funcref]{getResults@\fidxl{getResults}}%
, \hyperlink{ref_getProfileAllSpikes}{\texttt{getProfileAllSpikes}}%
\ (p.~\pageref{ref_getProfileAllSpikes})%
\index[funcref]{getProfileAllSpikes@\fidxl{getProfileAllSpikes}}%
, \hyperlink{ref_mergeMultipleCIPsInOne}{\texttt{mergeMultipleCIPsInOne}}%
\ (p.~\pageref{ref_mergeMultipleCIPsInOne})%
\index[funcref]{mergeMultipleCIPsInOne@\fidxl{mergeMultipleCIPsInOne}}%
%
\item[Author:]%
Cengiz Gunay <cgunay@emory.edu>, 2004/07/30
%
\end{description}
\methodline%
\subsubsection[Method \texttt{periodIniSpont}]{Method \texttt{cip\_trace/periodIniSpont}}%
\index[funcref]{cip_trace@\fidxl{cip\_trace}!periodIniSpont@\fidxl{periodIniSpont}}%
\label{ref_cip_trace__periodIniSpont}%
\hypertarget{ref_cip_trace__periodIniSpont}{}%
\begin{description}
\item[Summary:]Returns the initial spontaneous activity period of 
		cip\_trace, t. 
%
\item[Usage:]~%
\begin{lyxcode}%
the\_period = periodIniSpont(t)
%
\end{lyxcode}%
%
%
\item[Parameters:]~
\begin{description}%
\item[\texttt{t}:]
 A trace object.
\end{description}%
%
\item[Returns:
]~

	the\_period: A period object.
%
%
\item[See also:]%
\hyperlink{ref_period}{\texttt{period}}%
\ (p.~\pageref{ref_period})%
\index[funcref]{period@\fidxl{period}}%
, \hyperlink{ref_cip_trace}{\texttt{cip\_trace}}%
\ (p.~\pageref{ref_cip_trace})%
\index[funcref]{cip_trace@\fidxl{cip\_trace}}%
, \hyperlink{ref_trace}{\texttt{trace}}%
\ (p.~\pageref{ref_trace})%
\index[funcref]{trace@\fidxl{trace}}%
%
\item[Author:]%
Cengiz Gunay <cgunay@emory.edu>, 2004/08/25
%
\end{description}
\methodline%
\subsubsection[Method \texttt{periodPulse}]{Method \texttt{cip\_trace/periodPulse}}%
\index[funcref]{cip_trace@\fidxl{cip\_trace}!periodPulse@\fidxl{periodPulse}}%
\label{ref_cip_trace__periodPulse}%
\hypertarget{ref_cip_trace__periodPulse}{}%
\begin{description}
\item[Summary:]Returns the CIP period of cip\_trace, t. 
%
\item[Usage:]~%
\begin{lyxcode}%
the\_period = periodPulse(t)
%
\end{lyxcode}%
%
%
\item[Parameters:]~
\begin{description}%
\item[\texttt{t}:]
 A trace object.
\end{description}%
%
\item[Returns:
]~

	the\_period: A period object.
%
%
\item[See also:]%
\hyperlink{ref_period}{\texttt{period}}%
\ (p.~\pageref{ref_period})%
\index[funcref]{period@\fidxl{period}}%
, \hyperlink{ref_cip_trace}{\texttt{cip\_trace}}%
\ (p.~\pageref{ref_cip_trace})%
\index[funcref]{cip_trace@\fidxl{cip\_trace}}%
, \hyperlink{ref_trace}{\texttt{trace}}%
\ (p.~\pageref{ref_trace})%
\index[funcref]{trace@\fidxl{trace}}%
%
\item[Author:]%
Cengiz Gunay <cgunay@emory.edu>, 2004/08/25
%
\end{description}
\methodline%
\subsubsection[Method \texttt{periodPulseHalf1}]{Method \texttt{cip\_trace/periodPulseHalf1}}%
\index[funcref]{cip_trace@\fidxl{cip\_trace}!periodPulseHalf1@\fidxl{periodPulseHalf1}}%
\label{ref_cip_trace__periodPulseHalf1}%
\hypertarget{ref_cip_trace__periodPulseHalf1}{}%
\begin{description}
\item[Summary:]Returns the first half of the CIP period of cip\_trace, t. 
%
\item[Usage:]~%
\begin{lyxcode}%
the\_period = periodPulseHalf1(t)
%
\end{lyxcode}%
%
%
\item[Parameters:]~
\begin{description}%
\item[\texttt{t}:]
 A trace object.
\end{description}%
%
\item[Returns:
]~

	the\_period: A period object.
%
%
\item[See also:]%
\hyperlink{ref_period}{\texttt{period}}%
\ (p.~\pageref{ref_period})%
\index[funcref]{period@\fidxl{period}}%
, \hyperlink{ref_cip_trace}{\texttt{cip\_trace}}%
\ (p.~\pageref{ref_cip_trace})%
\index[funcref]{cip_trace@\fidxl{cip\_trace}}%
, \hyperlink{ref_trace}{\texttt{trace}}%
\ (p.~\pageref{ref_trace})%
\index[funcref]{trace@\fidxl{trace}}%
%
\item[Author:]%
Cengiz Gunay <cgunay@emory.edu>, 2004/08/25
%
\end{description}
\methodline%
\subsubsection[Method \texttt{periodPulseIni100ms}]{Method \texttt{cip\_trace/periodPulseIni100ms}}%
\index[funcref]{cip_trace@\fidxl{cip\_trace}!periodPulseIni100ms@\fidxl{periodPulseIni100ms}}%
\label{ref_cip_trace__periodPulseIni100ms}%
\hypertarget{ref_cip_trace__periodPulseIni100ms}{}%
\begin{description}
\item[Summary:]Returns the first 100ms of the CIP period of 
			cip\_trace, t. 
%
\item[Usage:]~%
\begin{lyxcode}%
the\_period = periodPulseIni100ms(t)
%
\end{lyxcode}%
%
%
\item[Parameters:]~
\begin{description}%
\item[\texttt{t}:]
 A trace object.
\end{description}%
%
\item[Returns:
]~

	the\_period: A period object.
%
%
\item[See also:]%
\hyperlink{ref_period}{\texttt{period}}%
\ (p.~\pageref{ref_period})%
\index[funcref]{period@\fidxl{period}}%
, \hyperlink{ref_cip_trace}{\texttt{cip\_trace}}%
\ (p.~\pageref{ref_cip_trace})%
\index[funcref]{cip_trace@\fidxl{cip\_trace}}%
, \hyperlink{ref_trace}{\texttt{trace}}%
\ (p.~\pageref{ref_trace})%
\index[funcref]{trace@\fidxl{trace}}%
%
\item[Author:]%
Cengiz Gunay <cgunay@emory.edu>, 2004/08/25
%
\end{description}
\methodline%
\subsubsection[Method \texttt{periodPulseIni100msRest1}]{Method \texttt{cip\_trace/periodPulseIni100msRest1}}%
\index[funcref]{cip_trace@\fidxl{cip\_trace}!periodPulseIni100msRest1@\fidxl{periodPulseIni100msRest1}}%
\label{ref_cip_trace__periodPulseIni100msRest1}%
\hypertarget{ref_cip_trace__periodPulseIni100msRest1}{}%
\begin{description}
%
\item[Usage:]~%
\begin{lyxcode}%
the\_period = periodPulseIni50msRest1(t)
%
\end{lyxcode}%
%
%
\item[Parameters:]~
\begin{description}%
\item[\texttt{t}:]
 A trace object.
\end{description}%
%
\item[Returns:
]~

	the\_period: A period object.
%
%
\item[See also:]%
\hyperlink{ref_period}{\texttt{period}}%
\ (p.~\pageref{ref_period})%
\index[funcref]{period@\fidxl{period}}%
, \hyperlink{ref_cip_trace}{\texttt{cip\_trace}}%
\ (p.~\pageref{ref_cip_trace})%
\index[funcref]{cip_trace@\fidxl{cip\_trace}}%
, \hyperlink{ref_trace}{\texttt{trace}}%
\ (p.~\pageref{ref_trace})%
\index[funcref]{trace@\fidxl{trace}}%
%
\item[Author:]%
Cengiz Gunay <cgunay@emory.edu>, 2004/08/25
%
\end{description}
\methodline%
\subsubsection[Method \texttt{periodPulseIni100msRest2}]{Method \texttt{cip\_trace/periodPulseIni100msRest2}}%
\index[funcref]{cip_trace@\fidxl{cip\_trace}!periodPulseIni100msRest2@\fidxl{periodPulseIni100msRest2}}%
\label{ref_cip_trace__periodPulseIni100msRest2}%
\hypertarget{ref_cip_trace__periodPulseIni100msRest2}{}%
\begin{description}
%
\item[Usage:]~%
\begin{lyxcode}%
the\_period = periodPulseIni50msRest2(t)
%
\end{lyxcode}%
%
%
\item[Parameters:]~
\begin{description}%
\item[\texttt{t}:]
 A trace object.
\end{description}%
%
\item[Returns:
]~

	the\_period: A period object.
%
%
\item[See also:]%
\hyperlink{ref_period}{\texttt{period}}%
\ (p.~\pageref{ref_period})%
\index[funcref]{period@\fidxl{period}}%
, \hyperlink{ref_cip_trace}{\texttt{cip\_trace}}%
\ (p.~\pageref{ref_cip_trace})%
\index[funcref]{cip_trace@\fidxl{cip\_trace}}%
, \hyperlink{ref_trace}{\texttt{trace}}%
\ (p.~\pageref{ref_trace})%
\index[funcref]{trace@\fidxl{trace}}%
%
\item[Author:]%
Cengiz Gunay <cgunay@emory.edu>, 2004/08/25
%
\end{description}
\methodline%
\subsubsection[Method \texttt{periodPulseIni50ms}]{Method \texttt{cip\_trace/periodPulseIni50ms}}%
\index[funcref]{cip_trace@\fidxl{cip\_trace}!periodPulseIni50ms@\fidxl{periodPulseIni50ms}}%
\label{ref_cip_trace__periodPulseIni50ms}%
\hypertarget{ref_cip_trace__periodPulseIni50ms}{}%
\begin{description}
\item[Summary:]Returns the first 50ms of the CIP period of 
			cip\_trace, t. 
%
\item[Usage:]~%
\begin{lyxcode}%
the\_period = periodPulseIni50ms(t)
%
\end{lyxcode}%
%
%
\item[Parameters:]~
\begin{description}%
\item[\texttt{t}:]
 A trace object.
\end{description}%
%
\item[Returns:
]~

	the\_period: A period object.
%
%
\item[See also:]%
\hyperlink{ref_period}{\texttt{period}}%
\ (p.~\pageref{ref_period})%
\index[funcref]{period@\fidxl{period}}%
, \hyperlink{ref_cip_trace}{\texttt{cip\_trace}}%
\ (p.~\pageref{ref_cip_trace})%
\index[funcref]{cip_trace@\fidxl{cip\_trace}}%
, \hyperlink{ref_trace}{\texttt{trace}}%
\ (p.~\pageref{ref_trace})%
\index[funcref]{trace@\fidxl{trace}}%
%
\item[Author:]%
Cengiz Gunay <cgunay@emory.edu>, 2004/08/25
%
\end{description}
\methodline%
\subsubsection[Method \texttt{periodPulseIni50msRest1}]{Method \texttt{cip\_trace/periodPulseIni50msRest1}}%
\index[funcref]{cip_trace@\fidxl{cip\_trace}!periodPulseIni50msRest1@\fidxl{periodPulseIni50msRest1}}%
\label{ref_cip_trace__periodPulseIni50msRest1}%
\hypertarget{ref_cip_trace__periodPulseIni50msRest1}{}%
\begin{description}
\item[Summary:]Returns the first half of the rest after the 
			first 50ms of the CIP period of cip\_trace, t. 
%
\item[Usage:]~%
\begin{lyxcode}%
the\_period = periodPulseIni50msRest1(t)
%
\end{lyxcode}%
%
%
\item[Parameters:]~
\begin{description}%
\item[\texttt{t}:]
 A trace object.
\end{description}%
%
\item[Returns:
]~

	the\_period: A period object.
%
%
\item[See also:]%
\hyperlink{ref_period}{\texttt{period}}%
\ (p.~\pageref{ref_period})%
\index[funcref]{period@\fidxl{period}}%
, \hyperlink{ref_cip_trace}{\texttt{cip\_trace}}%
\ (p.~\pageref{ref_cip_trace})%
\index[funcref]{cip_trace@\fidxl{cip\_trace}}%
, \hyperlink{ref_trace}{\texttt{trace}}%
\ (p.~\pageref{ref_trace})%
\index[funcref]{trace@\fidxl{trace}}%
%
\item[Author:]%
Cengiz Gunay <cgunay@emory.edu>, 2004/08/25
%
\end{description}
\methodline%
\subsubsection[Method \texttt{periodPulseIni50msRest2}]{Method \texttt{cip\_trace/periodPulseIni50msRest2}}%
\index[funcref]{cip_trace@\fidxl{cip\_trace}!periodPulseIni50msRest2@\fidxl{periodPulseIni50msRest2}}%
\label{ref_cip_trace__periodPulseIni50msRest2}%
\hypertarget{ref_cip_trace__periodPulseIni50msRest2}{}%
\begin{description}
\item[Summary:]Returns the second half of the rest after the 
			first 50ms of the CIP period of cip\_trace, t. 
%
\item[Usage:]~%
\begin{lyxcode}%
the\_period = periodPulseIni50msRest2(t)
%
\end{lyxcode}%
%
%
\item[Parameters:]~
\begin{description}%
\item[\texttt{t}:]
 A trace object.
\end{description}%
%
\item[Returns:
]~

	the\_period: A period object.
%
%
\item[See also:]%
\hyperlink{ref_period}{\texttt{period}}%
\ (p.~\pageref{ref_period})%
\index[funcref]{period@\fidxl{period}}%
, \hyperlink{ref_cip_trace}{\texttt{cip\_trace}}%
\ (p.~\pageref{ref_cip_trace})%
\index[funcref]{cip_trace@\fidxl{cip\_trace}}%
, \hyperlink{ref_trace}{\texttt{trace}}%
\ (p.~\pageref{ref_trace})%
\index[funcref]{trace@\fidxl{trace}}%
%
\item[Author:]%
Cengiz Gunay <cgunay@emory.edu>, 2004/08/25
%
\end{description}
\methodline%
\subsubsection[Method \texttt{periodRecSpont}]{Method \texttt{cip\_trace/periodRecSpont}}%
\index[funcref]{cip_trace@\fidxl{cip\_trace}!periodRecSpont@\fidxl{periodRecSpont}}%
\label{ref_cip_trace__periodRecSpont}%
\hypertarget{ref_cip_trace__periodRecSpont}{}%
\begin{description}
\item[Summary:]Returns the recovery spontaneous activity period 
		of cip\_trace, t. 
%
\item[Usage:]~%
\begin{lyxcode}%
the\_period = periodRecSpont(t)
%
\end{lyxcode}%
%
%
\item[Parameters:]~
\begin{description}%
\item[\texttt{t}:]
 A trace object.
\end{description}%
%
\item[Returns:
]~

	the\_period: A period object.
%
%
\item[See also:]%
\hyperlink{ref_period}{\texttt{period}}%
\ (p.~\pageref{ref_period})%
\index[funcref]{period@\fidxl{period}}%
, \hyperlink{ref_cip_trace}{\texttt{cip\_trace}}%
\ (p.~\pageref{ref_cip_trace})%
\index[funcref]{cip_trace@\fidxl{cip\_trace}}%
, \hyperlink{ref_trace}{\texttt{trace}}%
\ (p.~\pageref{ref_trace})%
\index[funcref]{trace@\fidxl{trace}}%
%
\item[Author:]%
Cengiz Gunay <cgunay@emory.edu>, 2004/08/25
%
\end{description}
\methodline%
\subsubsection[Method \texttt{periodRecSpont1}]{Method \texttt{cip\_trace/periodRecSpont1}}%
\index[funcref]{cip_trace@\fidxl{cip\_trace}!periodRecSpont1@\fidxl{periodRecSpont1}}%
\label{ref_cip_trace__periodRecSpont1}%
\hypertarget{ref_cip_trace__periodRecSpont1}{}%
\begin{description}
\item[Summary:]Returns the first half of the recovery spontaneous
		 activity period of cip\_trace, t. 
%
\item[Usage:]~%
\begin{lyxcode}%
the\_period = periodRecSpont1(t)
%
\end{lyxcode}%
%
%
\item[Parameters:]~
\begin{description}%
\item[\texttt{t}:]
 A trace object.
\end{description}%
%
\item[Returns:
]~

	the\_period: A period object.
%
%
\item[See also:]%
\hyperlink{ref_period}{\texttt{period}}%
\ (p.~\pageref{ref_period})%
\index[funcref]{period@\fidxl{period}}%
, \hyperlink{ref_cip_trace}{\texttt{cip\_trace}}%
\ (p.~\pageref{ref_cip_trace})%
\index[funcref]{cip_trace@\fidxl{cip\_trace}}%
, \hyperlink{ref_trace}{\texttt{trace}}%
\ (p.~\pageref{ref_trace})%
\index[funcref]{trace@\fidxl{trace}}%
%
\item[Author:]%
Cengiz Gunay <cgunay@emory.edu>, 2004/08/25
%
\end{description}
\methodline%
\subsubsection[Method \texttt{periodRecSpont2}]{Method \texttt{cip\_trace/periodRecSpont2}}%
\index[funcref]{cip_trace@\fidxl{cip\_trace}!periodRecSpont2@\fidxl{periodRecSpont2}}%
\label{ref_cip_trace__periodRecSpont2}%
\hypertarget{ref_cip_trace__periodRecSpont2}{}%
\begin{description}
\item[Summary:]Returns the second half of the recovery spontaneous
		 activity period of cip\_trace, t. 
%
\item[Usage:]~%
\begin{lyxcode}%
the\_period = periodRecSpont2(t)
%
\end{lyxcode}%
%
%
\item[Parameters:]~
\begin{description}%
\item[\texttt{t}:]
 A trace object.
\end{description}%
%
\item[Returns:
]~

	the\_period: A period object.
%
%
\item[See also:]%
\hyperlink{ref_period}{\texttt{period}}%
\ (p.~\pageref{ref_period})%
\index[funcref]{period@\fidxl{period}}%
, \hyperlink{ref_cip_trace}{\texttt{cip\_trace}}%
\ (p.~\pageref{ref_cip_trace})%
\index[funcref]{cip_trace@\fidxl{cip\_trace}}%
, \hyperlink{ref_trace}{\texttt{trace}}%
\ (p.~\pageref{ref_trace})%
\index[funcref]{trace@\fidxl{trace}}%
%
\item[Author:]%
Cengiz Gunay <cgunay@emory.edu>, 2004/08/25
%
\end{description}
\methodline%
\subsubsection[Method \texttt{periodRecSpontIniPeriod}]{Method \texttt{cip\_trace/periodRecSpontIniPeriod}}%
\index[funcref]{cip_trace@\fidxl{cip\_trace}!periodRecSpontIniPeriod@\fidxl{periodRecSpontIniPeriod}}%
\label{ref_cip_trace__periodRecSpontIniPeriod}%
\hypertarget{ref_cip_trace__periodRecSpontIniPeriod}{}%
\begin{description}
%
\item[Usage:]~%
\begin{lyxcode}%
the\_period = periodRecSpont(t)
%
\end{lyxcode}%
%
%
\item[Parameters:]~
\begin{description}%
\item[\texttt{t}:]
 A trace object.
\item[\texttt{iniPeriod}:]
 the time following pulse offset that is kept, the rest of

the time is ignored.
\end{description}%
%
\item[Returns:
]~

	the\_period: A period object.
%
%
\item[See also:]%
\hyperlink{ref_period}{\texttt{period}}%
\ (p.~\pageref{ref_period})%
\index[funcref]{period@\fidxl{period}}%
, \hyperlink{ref_cip_trace}{\texttt{cip\_trace}}%
\ (p.~\pageref{ref_cip_trace})%
\index[funcref]{cip_trace@\fidxl{cip\_trace}}%
, \hyperlink{ref_trace}{\texttt{trace}}%
\ (p.~\pageref{ref_trace})%
\index[funcref]{trace@\fidxl{trace}}%
%
\item[Author:]%
Cengiz Gunay <cgunay@emory.edu>,Tom Sangrey 2006/01/26
%
\end{description}
\methodline%
\subsubsection[Method \texttt{periodRecSpontRestPeriod}]{Method \texttt{cip\_trace/periodRecSpontRestPeriod}}%
\index[funcref]{cip_trace@\fidxl{cip\_trace}!periodRecSpontRestPeriod@\fidxl{periodRecSpontRestPeriod}}%
\label{ref_cip_trace__periodRecSpontRestPeriod}%
\hypertarget{ref_cip_trace__periodRecSpontRestPeriod}{}%
\begin{description}
%
\item[Usage:]~%
\begin{lyxcode}%
the\_period = periodRecSpont(t)
%
\end{lyxcode}%
%
%
\item[Parameters:]~
\begin{description}%
\item[\texttt{t}:]
 A trace object.
\item[\texttt{iniPeriod}:]
 the time following pulse offset that is ignored. The rest of

the time is kept
\end{description}%
%
\item[Returns:
]~

	the\_period: A period object.
%
%
\item[See also:]%
\hyperlink{ref_period}{\texttt{period}}%
\ (p.~\pageref{ref_period})%
\index[funcref]{period@\fidxl{period}}%
, \hyperlink{ref_cip_trace}{\texttt{cip\_trace}}%
\ (p.~\pageref{ref_cip_trace})%
\index[funcref]{cip_trace@\fidxl{cip\_trace}}%
, \hyperlink{ref_trace}{\texttt{trace}}%
\ (p.~\pageref{ref_trace})%
\index[funcref]{trace@\fidxl{trace}}%
%
\item[Author:]%
Cengiz Gunay <cgunay@emory.edu>,Tom Sangrey 2006/01/26
%
\end{description}
\methodline%
\subsubsection[Method \texttt{plotData}]{Method \texttt{cip\_trace/plotData}}%
\index[funcref]{cip_trace@\fidxl{cip\_trace}!plotData@\fidxl{plotData}}%
\label{ref_cip_trace__plotData}%
\hypertarget{ref_cip_trace__plotData}{}%
\begin{description}
\item[Summary:]Plots a trace by calling trace/plotData but also adds optionaldecorations.
%
\item[Usage:]~%
\begin{lyxcode}%
a\_plot = plotData(t, title\_str, props)
%
\end{lyxcode}%
%
\item[Description:]%
If t is a vector of traces, returns a vector of plot objects.
%%
\item[Parameters:]~
\begin{description}%
\item[\texttt{t}:]
 A trace object.
\item[\texttt{title\_str}:]
 (Optional) String to append to plot title.
\item[\texttt{props}:]
 A structure with any optional properties.
\begin{description}%
\item[\texttt{stimBar}:]
 If true, put a bar indicating the CIP duration.

(rest passed to trace/plotData)
\end{description}%
\end{description}%
%
\item[Returns:
]~

	a\_plot: A plot\_abstract object that can be visualized.
%
%
\item[See also:]%
\hyperlink{ref_trace}{\texttt{trace}}%
\ (p.~\pageref{ref_trace})%
\index[funcref]{trace@\fidxl{trace}}%
, \hyperlink{ref_trace__plot}{\texttt{trace/plot}}%
\ (p.~\pageref{ref_trace__plot})%
\index[funcref]{trace@\fidxl{trace}!plot@\fidxl{plot}}%
, \hyperlink{ref_plot_abstract}{\texttt{plot\_abstract}}%
\ (p.~\pageref{ref_plot_abstract})%
\index[funcref]{plot_abstract@\fidxl{plot\_abstract}}%
%
\item[Author:]%
Cengiz Gunay <cgunay@emory.edu>, 2004/11/17
%
\end{description}
\methodline%
\subsubsection[Method \texttt{plot\_abstract}]{Method \texttt{cip\_trace/plot\_abstract}}%
\index[funcref]{cip_trace@\fidxl{cip\_trace}!plot_abstract@\fidxl{plot\_abstract}}%
\label{ref_cip_trace__plot_abstract}%
\hypertarget{ref_cip_trace__plot_abstract}{}%
\begin{description}
\item[Summary:]Plots a trace by calling plotData.
%
\item[Usage:]~%
\begin{lyxcode}%
a\_plot = plot\_abstract(t, title\_str, props)
%
\end{lyxcode}%
%
\item[Description:]%
If t is a vector of traces, returns a vector of plot objects.
%%
\item[Parameters:]~
\begin{description}%
\item[\texttt{t}:]
 A trace object.
\item[\texttt{title\_str}:]
 (Optional) String to append to plot title.
\item[\texttt{props}:]
 A structure with any optional properties.
\begin{description}%
\item[\texttt{timeScale}:]
 's' for seconds, or 'ms' for milliseconds.

(rest passed to plot\_abstract.)
\end{description}%
\end{description}%
%
\item[Returns:
]~

	a\_plot: A plot\_abstract object that can be visualized.
%
%
\item[See also:]%
\hyperlink{ref_trace}{\texttt{trace}}%
\ (p.~\pageref{ref_trace})%
\index[funcref]{trace@\fidxl{trace}}%
, \hyperlink{ref_trace__plot}{\texttt{trace/plot}}%
\ (p.~\pageref{ref_trace__plot})%
\index[funcref]{trace@\fidxl{trace}!plot@\fidxl{plot}}%
, \hyperlink{ref_plot_abstract}{\texttt{plot\_abstract}}%
\ (p.~\pageref{ref_plot_abstract})%
\index[funcref]{plot_abstract@\fidxl{plot\_abstract}}%
%
\item[Author:]%
Cengiz Gunay <cgunay@emory.edu>, 2004/11/17
%
\end{description}
\methodline%
\subsubsection[Method \texttt{set}]{Method \texttt{cip\_trace/set}}%
\index[funcref]{cip_trace@\fidxl{cip\_trace}!set@\fidxl{set}}%
\label{ref_cip_trace__set}%
\hypertarget{ref_cip_trace__set}{}%
\begin{description}
\item[Summary:]Generic method for setting object attributes.
%
%
%
%
%
%
%
\item[Author:]%
Cengiz Gunay <cgunay@emory.edu>, 2004/10/08
%
\end{description}
\methodline%
\subsubsection[Method \texttt{spikes}]{Method \texttt{cip\_trace/spikes}}%
\index[funcref]{cip_trace@\fidxl{cip\_trace}!spikes@\fidxl{spikes}}%
\label{ref_cip_trace__spikes}%
\hypertarget{ref_cip_trace__spikes}{}%
\begin{description}
\item[Summary:]Convert cip\_trace to spikes object for spike timing calculations.
%
\item[Usage:]~%
\begin{lyxcode}%
obj = spikes(trace, plotit)
%
\end{lyxcode}%
%
\item[Description:]%
Creates a spikes object by finding the spikes in the three 
 separate periods, initial spontaneous activity period, CIP period, and
 final recovery period.
%%
\item[Parameters:]~
\begin{description}%
\item[\texttt{trace}:]
 A trace object.
\item[\texttt{plotit}:]
 If non-zero, a plot is generated for showing spikes found

(optional).
\end{description}%
%
%
%
\item[See also:]%
\hyperlink{ref_spikes}{\texttt{spikes}}%
\ (p.~\pageref{ref_spikes})%
\index[funcref]{spikes@\fidxl{spikes}}%
, \hyperlink{ref_period}{\texttt{period}}%
\ (p.~\pageref{ref_period})%
\index[funcref]{period@\fidxl{period}}%
%
\item[Author:]%
Cengiz Gunay <cgunay@emory.edu>, 2004/08/25
%
\end{description}
\methodline%
\subsubsection[Method \texttt{subsref}]{Method \texttt{cip\_trace/subsref}}%
\index[funcref]{cip_trace@\fidxl{cip\_trace}!subsref@\fidxl{subsref}}%
\label{ref_cip_trace__subsref}%
\hypertarget{ref_cip_trace__subsref}{}%
\begin{description}
\item[Summary:]Defines generic indexing for objects.
%
%
%
%
%
%
%
\item[Author:]%
Cengiz Gunay <cgunay@emory.edu>, 2004/08/04
%
\end{description}
\methodline%
\subsection{Class \texttt{cip\_trace\_allspikes\_profile}}%
\index[funcref]{cip_trace_allspikes_profile@\fidxl{cip\_trace\_allspikes\_profile}|boldhyperpage}%
\label{ref_cip_trace_allspikes_profile}%
\hypertarget{ref_cip_trace_allspikes_profile}{}%
\subsubsection[Constructor \texttt{cip\_trace\_allspikes\_profile}]{Constructor \texttt{cip\_trace\_allspikes\_profile/cip\_trace\_allspikes\_profile}}%
\index[funcref]{cip_trace_allspikes_profile@\fidxl{cip\_trace\_allspikes\_profile}!cip_trace_allspikes_profile@\fidxl{cip\_trace\_allspikes\_profile}}%
\label{ref_cip_trace_allspikes_profile__cip_trace_allspikes_profile}%
\hypertarget{ref_cip_trace_allspikes_profile__cip_trace_allspikes_profile}{}%
\begin{description}
\item[Summary:]Creates and collects test results of a cip\_trace.
%
\item[Usage:]~%
\begin{lyxcode}%
obj = 
   cip\_trace\_allspikes\_profile(a\_cip\_trace, a\_spikes, a\_spont\_spike\_shape, 
				results, id, props)
%
\end{lyxcode}%
%
\item[Description:]%
This is a subclass of results\_profile. It is made to be used from 
 subclass constructors.
%%
\item[Parameters:]~
\begin{description}%
\item[\texttt{a\_cip\_trace}:]
 A cip\_trace object.
\item[\texttt{a\_spikes}:]
 A spikes object.
\item[\texttt{spont\_spikes\_db, pulse\_spikes\_db, recov\_spikes\_db}:]
 

tests\_dbs with spontaneous, pulse and recovery period spike info.
\item[\texttt{results\_obj}:]
 A results\_profile object with test results.
\item[\texttt{id}:]
 Identification string.
\item[\texttt{props}:]
 A structure with any optional properties.
\end{description}%
%
\item[Returns a structure object with the following fields:
]~

	trace, spikes, spont\_spikes\_db, 
	pulse\_spikes\_db, recov\_spikes\_db, props
%
%
\item[See also:]%
\hyperlink{ref_cip_trace}{\texttt{cip\_trace}}%
\ (p.~\pageref{ref_cip_trace})%
\index[funcref]{cip_trace@\fidxl{cip\_trace}}%
, \hyperlink{ref_spikes}{\texttt{spikes}}%
\ (p.~\pageref{ref_spikes})%
\index[funcref]{spikes@\fidxl{spikes}}%
, \hyperlink{ref_tests_db}{\texttt{tests\_db}}%
\ (p.~\pageref{ref_tests_db})%
\index[funcref]{tests_db@\fidxl{tests\_db}}%
%
\item[Author:]%
Cengiz Gunay <cgunay@emory.edu>, 2005/05/04
%
\end{description}
\methodline%
\subsubsection[Method \texttt{display}]{Method \texttt{cip\_trace\_allspikes\_profile/display}}%
\index[funcref]{cip_trace_allspikes_profile@\fidxl{cip\_trace\_allspikes\_profile}!display@\fidxl{display}}%
\label{ref_cip_trace_allspikes_profile__display}%
\hypertarget{ref_cip_trace_allspikes_profile__display}{}%
\begin{description}
%
%
%
%
%
%
%
\item[Author:]%
Cengiz Gunay <cgunay@emory.edu>, 2004/08/04
%
\end{description}
\methodline%
\subsubsection[Method \texttt{get}]{Method \texttt{cip\_trace\_allspikes\_profile/get}}%
\index[funcref]{cip_trace_allspikes_profile@\fidxl{cip\_trace\_allspikes\_profile}!get@\fidxl{get}}%
\label{ref_cip_trace_allspikes_profile__get}%
\hypertarget{ref_cip_trace_allspikes_profile__get}{}%
\begin{description}
\item[Summary:]Defines generic attribute retrieval for objects.
%
%
%
%
%
%
%
\item[Author:]%
Cengiz Gunay <cgunay@emory.edu>, 2004/09/14
%
\end{description}
\methodline%
\subsubsection[Method \texttt{plotRowSpontSpikeAnal}]{Method \texttt{cip\_trace\_allspikes\_profile/plotRowSpontSpikeAnal}}%
\index[funcref]{cip_trace_allspikes_profile@\fidxl{cip\_trace\_allspikes\_profile}!plotRowSpontSpikeAnal@\fidxl{plotRowSpontSpikeAnal}}%
\label{ref_cip_trace_allspikes_profile__plotRowSpontSpikeAnal}%
\hypertarget{ref_cip_trace_allspikes_profile__plotRowSpontSpikeAnal}{}%
\begin{description}
\item[Summary:]Creates a row of plots that show spontaneous spikes, starting from the whole trace, zooming into the individual spike.
%
\item[Usage:]~%
\begin{lyxcode}%
a\_plot = plotRowSpontSpikeAnal(prof, title\_str)
%
\end{lyxcode}%
%
%
\item[Parameters:]~
\begin{description}%
\item[\texttt{prof}:]
 A cip\_trace\_allspikes\_profile object.
\item[\texttt{title\_str}:]
 (Optional) String to append to plot title.
\end{description}%
%
\item[Returns:
]~

	a\_plot: A plot\_abstract object that can be visualized.
%
%
\item[See also:]%
\hyperlink{ref_trace}{\texttt{trace}}%
\ (p.~\pageref{ref_trace})%
\index[funcref]{trace@\fidxl{trace}}%
, \hyperlink{ref_cip_trace}{\texttt{cip\_trace}}%
\ (p.~\pageref{ref_cip_trace})%
\index[funcref]{cip_trace@\fidxl{cip\_trace}}%
, \hyperlink{ref_spike_shape__plotCompareMethodsSimple}{\texttt{spike\_shape/plotCompareMethodsSimple}}%
\ (p.~\pageref{ref_spike_shape__plotCompareMethodsSimple})%
\index[funcref]{spike_shape@\fidxl{spike\_shape}!plotCompareMethodsSimple@\fidxl{plotCompareMethodsSimple}}%
, \hyperlink{ref_plot_abstract}{\texttt{plot\_abstract}}%
\ (p.~\pageref{ref_plot_abstract})%
\index[funcref]{plot_abstract@\fidxl{plot\_abstract}}%
%
\item[Author:]%
Cengiz Gunay <cgunay@emory.edu>, 2005/05/23
%
\end{description}
\methodline%
\subsubsection[Method \texttt{set}]{Method \texttt{cip\_trace\_allspikes\_profile/set}}%
\index[funcref]{cip_trace_allspikes_profile@\fidxl{cip\_trace\_allspikes\_profile}!set@\fidxl{set}}%
\label{ref_cip_trace_allspikes_profile__set}%
\hypertarget{ref_cip_trace_allspikes_profile__set}{}%
\begin{description}
\item[Summary:]Generic method for setting object attributes.
%
%
%
%
%
%
%
\item[Author:]%
Cengiz Gunay <cgunay@emory.edu>, 2004/10/08
%
\end{description}
\methodline%
\subsection{Class \texttt{cip\_trace\_profile}}%
\index[funcref]{cip_trace_profile@\fidxl{cip\_trace\_profile}|boldhyperpage}%
\label{ref_cip_trace_profile}%
\hypertarget{ref_cip_trace_profile}{}%
\subsubsection[Constructor \texttt{cip\_trace\_profile}]{Constructor \texttt{cip\_trace\_profile/cip\_trace\_profile}}%
\index[funcref]{cip_trace_profile@\fidxl{cip\_trace\_profile}!cip_trace_profile@\fidxl{cip\_trace\_profile}}%
\label{ref_cip_trace_profile__cip_trace_profile}%
\hypertarget{ref_cip_trace_profile__cip_trace_profile}{}%
\begin{description}
\item[Summary:]Creates and collects test results of a cip\_trace.
%
%
\item[Description:]%
The first usage is fully customizable to be used from subclass constructors.
 The second usage generates the spikes and spont\_spike\_shape objects, and
 collects some generic test results from them. 
%%
\item[Parameters:]~
\begin{description}%
\item[\texttt{data\_src}:]
 The trace column OR the filename.
\item[\texttt{dt}:]
 Time resolution [s]
\item[\texttt{dy}:]
 y-axis resolution [ISI (V, A, etc.)]
\item[\texttt{pulse\_time\_start, pulse\_time\_width}:]


Start and width of the pulse [dt]
\item[\texttt{id}:]
 Identification string.
\item[\texttt{props}:]
 See trace object.
\end{description}%
%
\item[Returns a structure object with the following fields:
]~

	trace, spikes, spont\_spike\_shape, results, id, props.
%
%
\item[See also:]%
\hyperlink{ref_cip_trace}{\texttt{cip\_trace}}%
\ (p.~\pageref{ref_cip_trace})%
\index[funcref]{cip_trace@\fidxl{cip\_trace}}%
, \hyperlink{ref_spikes}{\texttt{spikes}}%
\ (p.~\pageref{ref_spikes})%
\index[funcref]{spikes@\fidxl{spikes}}%
, \hyperlink{ref_spike_shape}{\texttt{spike\_shape}}%
\ (p.~\pageref{ref_spike_shape})%
\index[funcref]{spike_shape@\fidxl{spike\_shape}}%
%
\item[Author:]%
Cengiz Gunay <cgunay@emory.edu>, 2004/08/25
%
\end{description}
\methodline%
\subsubsection[Method \texttt{display}]{Method \texttt{cip\_trace\_profile/display}}%
\index[funcref]{cip_trace_profile@\fidxl{cip\_trace\_profile}!display@\fidxl{display}}%
\label{ref_cip_trace_profile__display}%
\hypertarget{ref_cip_trace_profile__display}{}%
\begin{description}
%
%
%
%
%
%
%
\item[Author:]%
Cengiz Gunay <cgunay@emory.edu>, 2004/08/04
%
\end{description}
\methodline%
\subsubsection[Method \texttt{get}]{Method \texttt{cip\_trace\_profile/get}}%
\index[funcref]{cip_trace_profile@\fidxl{cip\_trace\_profile}!get@\fidxl{get}}%
\label{ref_cip_trace_profile__get}%
\hypertarget{ref_cip_trace_profile__get}{}%
\begin{description}
\item[Summary:]Defines generic attribute retrieval for objects.
%
%
%
%
%
%
%
\item[Author:]%
Cengiz Gunay <cgunay@emory.edu>, 2004/09/14
%
\end{description}
\methodline%
\subsubsection[Method \texttt{plot}]{Method \texttt{cip\_trace\_profile/plot}}%
\index[funcref]{cip_trace_profile@\fidxl{cip\_trace\_profile}!plot@\fidxl{plot}}%
\label{ref_cip_trace_profile__plot}%
\hypertarget{ref_cip_trace_profile__plot}{}%
\begin{description}
\item[Summary:]Plots a cip\_trace\_profile object.
%
\item[Usage:]~%
\begin{lyxcode}%
h = plot(t)
%
\end{lyxcode}%
%
\item[Description:]%
Plots contents of this object.
%%
\item[Parameters:]~
\begin{description}%
\item[\texttt{t}:]
 A cip\_trace\_profile object.
\end{description}%
%
\item[Returns:
]~

	h: Plot handle(s).
%
%
%
\item[Author:]%
Cengiz Gunay <cgunay@emory.edu>, 2004/09/15
%
\end{description}
\methodline%
\subsubsection[Method \texttt{set}]{Method \texttt{cip\_trace\_profile/set}}%
\index[funcref]{cip_trace_profile@\fidxl{cip\_trace\_profile}!set@\fidxl{set}}%
\label{ref_cip_trace_profile__set}%
\hypertarget{ref_cip_trace_profile__set}{}%
\begin{description}
\item[Summary:]Generic method for setting object attributes.
%
%
%
%
%
%
%
\item[Author:]%
Cengiz Gunay <cgunay@emory.edu>, 2004/10/08
%
\end{description}
\methodline%
\subsubsection[Method \texttt{subsref}]{Method \texttt{cip\_trace\_profile/subsref}}%
\index[funcref]{cip_trace_profile@\fidxl{cip\_trace\_profile}!subsref@\fidxl{subsref}}%
\label{ref_cip_trace_profile__subsref}%
\hypertarget{ref_cip_trace_profile__subsref}{}%
\begin{description}
\item[Summary:]Defines generic indexing for objects.
%
%
%
%
%
%
%
\item[Author:]%
Cengiz Gunay <cgunay@emory.edu>, 2004/08/04
%
\end{description}
\methodline%
\subsection{Class \texttt{cip\_traces\_dataset}}%
\index[funcref]{cip_traces_dataset@\fidxl{cip\_traces\_dataset}|boldhyperpage}%
\label{ref_cip_traces_dataset}%
\hypertarget{ref_cip_traces_dataset}{}%
\subsubsection[Constructor \texttt{cip\_traces\_dataset}]{Constructor \texttt{cip\_traces\_dataset/cip\_traces\_dataset}}%
\index[funcref]{cip_traces_dataset@\fidxl{cip\_traces\_dataset}!cip_traces_dataset@\fidxl{cip\_traces\_dataset}}%
\label{ref_cip_traces_dataset__cip_traces_dataset}%
\hypertarget{ref_cip_traces_dataset__cip_traces_dataset}{}%
\begin{description}
\item[Summary:]Dataset of cip\_traces objects, each with varying cip magnitudes.
%
\item[Usage:]~%
\begin{lyxcode}%
obj = cip\_traces\_dataset(ts, cipmag, id, props)
%
\end{lyxcode}%
%
\item[Description:]%
This is a subclass of params\_tests\_fileset.
%%
\item[Parameters:]~
\begin{description}%
\item[\texttt{ts}:]
 A cell array of cip\_traces objects.
\item[\texttt{cipmag}:]
 A single cip magnitude to trace take from objects.
\item[\texttt{id}:]
 An identification string for the whole dataset.
\item[\texttt{props}:]
 A structure with any optional properties passed to cip\_trace\_profile.
\end{description}%
%
\item[Returns a structure object with the following fields:
]~

	params\_tests\_dataset,
	cipmag, props (see above).
%
%
\item[See also:]%
\hyperlink{ref_cip_traces}{\texttt{cip\_traces}}%
\ (p.~\pageref{ref_cip_traces})%
\index[funcref]{cip_traces@\fidxl{cip\_traces}}%
, \hyperlink{ref_params_tests_fileset}{\texttt{params\_tests\_fileset}}%
\ (p.~\pageref{ref_params_tests_fileset})%
\index[funcref]{params_tests_fileset@\fidxl{params\_tests\_fileset}}%
, \hyperlink{ref_params_tests_db}{\texttt{params\_tests\_db}}%
\ (p.~\pageref{ref_params_tests_db})%
\index[funcref]{params_tests_db@\fidxl{params\_tests\_db}}%
%
\item[Author:]%
Cengiz Gunay <cgunay@emory.edu>, 2004/11/30
%
\end{description}
\methodline%
\subsubsection[Method \texttt{cip\_trace\_profile}]{Method \texttt{cip\_traces\_dataset/cip\_trace\_profile}}%
\index[funcref]{cip_traces_dataset@\fidxl{cip\_traces\_dataset}!cip_trace_profile@\fidxl{cip\_trace\_profile}}%
\label{ref_cip_traces_dataset__cip_trace_profile}%
\hypertarget{ref_cip_traces_dataset__cip_trace_profile}{}%
\begin{description}
\item[Summary:]Loads a raw cip\_trace\_profile given a index 
		      to this dataset.
%
\item[Usage:]~%
\begin{lyxcode}%
a\_cip\_trace\_profile = cip\_trace\_profile(dataset, index)
%
\end{lyxcode}%
%
%
\item[Parameters:]~
\begin{description}%
\item[\texttt{dataset}:]
 A params\_tests\_dataset.
\item[\texttt{index}:]
 Index of file in dataset.
\end{description}%
%
\item[Returns:
]~

	a\_cip\_trace\_profile: A cip\_trace\_profile object.
%
%
\item[See also:]%
\hyperlink{ref_cip_trace_profile}{\texttt{cip\_trace\_profile}}%
\ (p.~\pageref{ref_cip_trace_profile})%
\index[funcref]{cip_trace_profile@\fidxl{cip\_trace\_profile}}%
, \hyperlink{ref_params_tests_dataset}{\texttt{params\_tests\_dataset}}%
\ (p.~\pageref{ref_params_tests_dataset})%
\index[funcref]{params_tests_dataset@\fidxl{params\_tests\_dataset}}%
%
\item[Author:]%
Cengiz Gunay <cgunay@emory.edu>, 2004/09/14
%
\end{description}
\methodline%
\subsubsection[Method \texttt{display}]{Method \texttt{cip\_traces\_dataset/display}}%
\index[funcref]{cip_traces_dataset@\fidxl{cip\_traces\_dataset}!display@\fidxl{display}}%
\label{ref_cip_traces_dataset__display}%
\hypertarget{ref_cip_traces_dataset__display}{}%
\begin{description}
%
%
%
%
%
%
%
\item[Author:]%
Cengiz Gunay <cgunay@emory.edu>, 2004/08/04
%
\end{description}
\methodline%
\subsubsection[Method \texttt{get}]{Method \texttt{cip\_traces\_dataset/get}}%
\index[funcref]{cip_traces_dataset@\fidxl{cip\_traces\_dataset}!get@\fidxl{get}}%
\label{ref_cip_traces_dataset__get}%
\hypertarget{ref_cip_traces_dataset__get}{}%
\begin{description}
\item[Summary:]Defines generic attribute retrieval for objects.
%
%
%
%
%
%
%
\item[Author:]%
Cengiz Gunay <cgunay@emory.edu>, 2004/09/14
%
\end{description}
\methodline%
\subsubsection[Method \texttt{getItemParams}]{Method \texttt{cip\_traces\_dataset/getItemParams}}%
\index[funcref]{cip_traces_dataset@\fidxl{cip\_traces\_dataset}!getItemParams@\fidxl{getItemParams}}%
\label{ref_cip_traces_dataset__getItemParams}%
\hypertarget{ref_cip_traces_dataset__getItemParams}{}%
\begin{description}
%
\item[Usage:]~%
\begin{lyxcode}%
params\_row = getParams(dataset, index)
%
\end{lyxcode}%
%
%
\item[Parameters:]~
\begin{description}%
\item[\texttt{dataset}:]
 A params\_tests\_dataset.
\item[\texttt{index}:]
 Index of item in dataset.
\end{description}%
%
\item[Returns:
]~

	params\_row: Parameter values in the same order of paramNames
%
%
\item[See also:]%
\hyperlink{ref_itemResultsRow}{\texttt{itemResultsRow}}%
\ (p.~\pageref{ref_itemResultsRow})%
\index[funcref]{itemResultsRow@\fidxl{itemResultsRow}}%
, \hyperlink{ref_params_tests_dataset}{\texttt{params\_tests\_dataset}}%
\ (p.~\pageref{ref_params_tests_dataset})%
\index[funcref]{params_tests_dataset@\fidxl{params\_tests\_dataset}}%
, \hyperlink{ref_paramNames}{\texttt{paramNames}}%
\ (p.~\pageref{ref_paramNames})%
\index[funcref]{paramNames@\fidxl{paramNames}}%
, \hyperlink{ref_testNames}{\texttt{testNames}}%
\ (p.~\pageref{ref_testNames})%
\index[funcref]{testNames@\fidxl{testNames}}%
%
\item[Author:]%
Cengiz Gunay <cgunay@emory.edu>, 2004/12/06
%
\end{description}
\methodline%
\subsubsection[Method \texttt{loadItemProfile}]{Method \texttt{cip\_traces\_dataset/loadItemProfile}}%
\index[funcref]{cip_traces_dataset@\fidxl{cip\_traces\_dataset}!loadItemProfile@\fidxl{loadItemProfile}}%
\label{ref_cip_traces_dataset__loadItemProfile}%
\hypertarget{ref_cip_traces_dataset__loadItemProfile}{}%
\begin{description}
\item[Summary:]Loads a profile object from a raw data item in the dataset.
%
\item[Usage:]~%
\begin{lyxcode}%
a\_profile = loadItemProfile(dataset, index)
%
\end{lyxcode}%
%
\item[Description:]%
Subclasses should overload this function to load the specific profile
 object they desire. The profile class should define a getResults method
 which is used in the itemResultsRow method.
%%
\item[Parameters:]~
\begin{description}%
\item[\texttt{dataset}:]
 A params\_tests\_dataset.
\item[\texttt{index}:]
 Index of item in dataset.
\end{description}%
%
\item[Returns:
]~

	a\_profile: A profile object that implements the getResults method.
%
%
\item[See also:]%
\hyperlink{ref_itemResultsRow}{\texttt{itemResultsRow}}%
\ (p.~\pageref{ref_itemResultsRow})%
\index[funcref]{itemResultsRow@\fidxl{itemResultsRow}}%
, \hyperlink{ref_params_tests_dataset}{\texttt{params\_tests\_dataset}}%
\ (p.~\pageref{ref_params_tests_dataset})%
\index[funcref]{params_tests_dataset@\fidxl{params\_tests\_dataset}}%
, \hyperlink{ref_paramNames}{\texttt{paramNames}}%
\ (p.~\pageref{ref_paramNames})%
\index[funcref]{paramNames@\fidxl{paramNames}}%
, \hyperlink{ref_testNames}{\texttt{testNames}}%
\ (p.~\pageref{ref_testNames})%
\index[funcref]{testNames@\fidxl{testNames}}%
%
\item[Author:]%
Cengiz Gunay <cgunay@emory.edu>, 2004/09/14
%
\end{description}
\methodline%
\subsubsection[Method \texttt{paramNames}]{Method \texttt{cip\_traces\_dataset/paramNames}}%
\index[funcref]{cip_traces_dataset@\fidxl{cip\_traces\_dataset}!paramNames@\fidxl{paramNames}}%
\label{ref_cip_traces_dataset__paramNames}%
\hypertarget{ref_cip_traces_dataset__paramNames}{}%
\begin{description}
\item[Summary:]Returns the only parameter, 'pAcip,' for this fileset.
%
\item[Usage:]~%
\begin{lyxcode}%
param\_names = paramNames(fileset)
%
\end{lyxcode}%
%
\item[Description:]%
Looks at the filename of the first file to find the parameter names.
%%
\item[Parameters:]~
\begin{description}%
\item[\texttt{fileset}:]
 A params\_tests\_fileset.
\end{description}%
%
\item[Returns:
]~

	params\_names: Cell array with ordered parameter names.
%
%
\item[See also:]%
\hyperlink{ref_params_tests_fileset}{\texttt{params\_tests\_fileset}}%
\ (p.~\pageref{ref_params_tests_fileset})%
\index[funcref]{params_tests_fileset@\fidxl{params\_tests\_fileset}}%
, \hyperlink{ref_paramNames}{\texttt{paramNames}}%
\ (p.~\pageref{ref_paramNames})%
\index[funcref]{paramNames@\fidxl{paramNames}}%
, \hyperlink{ref_testNames}{\texttt{testNames}}%
\ (p.~\pageref{ref_testNames})%
\index[funcref]{testNames@\fidxl{testNames}}%
%
\item[Author:]%
Cengiz Gunay <cgunay@emory.edu>, 2004/12/06
%
\end{description}
\methodline%
\subsubsection[Method \texttt{set}]{Method \texttt{cip\_traces\_dataset/set}}%
\index[funcref]{cip_traces_dataset@\fidxl{cip\_traces\_dataset}!set@\fidxl{set}}%
\label{ref_cip_traces_dataset__set}%
\hypertarget{ref_cip_traces_dataset__set}{}%
\begin{description}
\item[Summary:]Generic method for setting object attributes.
%
%
%
%
%
%
%
\item[Author:]%
Cengiz Gunay <cgunay@emory.edu>, 2004/10/08
%
\end{description}
\methodline%
\subsubsection[Method \texttt{subsref}]{Method \texttt{cip\_traces\_dataset/subsref}}%
\index[funcref]{cip_traces_dataset@\fidxl{cip\_traces\_dataset}!subsref@\fidxl{subsref}}%
\label{ref_cip_traces_dataset__subsref}%
\hypertarget{ref_cip_traces_dataset__subsref}{}%
\begin{description}
\item[Summary:]Defines generic indexing for objects.
%
%
%
%
%
%
%
\item[Author:]%
Cengiz Gunay <cgunay@emory.edu>, 2004/08/04
%
\end{description}
\methodline%
\subsection{Class \texttt{cip\_traceset}}%
\index[funcref]{cip_traceset@\fidxl{cip\_traceset}|boldhyperpage}%
\label{ref_cip_traceset}%
\hypertarget{ref_cip_traceset}{}%
\subsubsection[Constructor \texttt{cip\_traceset}]{Constructor \texttt{cip\_traceset/cip\_traceset}}%
\index[funcref]{cip_traceset@\fidxl{cip\_traceset}!cip_traceset@\fidxl{cip\_traceset}}%
\label{ref_cip_traceset__cip_traceset}%
\hypertarget{ref_cip_traceset__cip_traceset}{}%
\begin{description}
\item[Summary:]A traceset with varying cip magnitudes from a single cip\_traces object.
%
\item[Usage:]~%
\begin{lyxcode}%
obj = cip\_traceset(ct, cip\_mags, dy, props)
%
\end{lyxcode}%
%
\item[Description:]%
This is a subclass of params\_tests\_fileset. This traceset can create a 
 mini-database form a single cip\_traces object. The list contains cip\_mags.
 cip\_traceset\_dataset should be used to load multiple cip\_traceset objects.
%%
\item[Parameters:]~
\begin{description}%
\item[\texttt{ct}:]
 A cip\_traces object.
\item[\texttt{cip\_mags}:]
 An array of cip magnitudes to select from the object.
\item[\texttt{dy}:]
 y-axis resolution, [V] or [A] (default=1e-3).
\item[\texttt{props}:]
 A structure with any optional properties.
\begin{description}%
\item[\texttt{offsetPotential}:]
 Add this to physiology trace as compensation.
\end{description}%
\end{description}%
%
\item[Returns a structure object with the following fields:
]~

	params\_tests\_dataset,
	ct, props (see above).
%
%
\item[See also:]%
\hyperlink{ref_cip_traces}{\texttt{cip\_traces}}%
\ (p.~\pageref{ref_cip_traces})%
\index[funcref]{cip_traces@\fidxl{cip\_traces}}%
, \hyperlink{ref_params_tests_fileset}{\texttt{params\_tests\_fileset}}%
\ (p.~\pageref{ref_params_tests_fileset})%
\index[funcref]{params_tests_fileset@\fidxl{params\_tests\_fileset}}%
, \hyperlink{ref_params_tests_db}{\texttt{params\_tests\_db}}%
\ (p.~\pageref{ref_params_tests_db})%
\index[funcref]{params_tests_db@\fidxl{params\_tests\_db}}%
%
\item[Author:]%
Cengiz Gunay <cgunay@emory.edu>, 2004/11/30
%
\end{description}
\methodline%
\subsubsection[Method \texttt{cip\_trace\_profile}]{Method \texttt{cip\_traceset/cip\_trace\_profile}}%
\index[funcref]{cip_traceset@\fidxl{cip\_traceset}!cip_trace_profile@\fidxl{cip\_trace\_profile}}%
\label{ref_cip_traceset__cip_trace_profile}%
\hypertarget{ref_cip_traceset__cip_trace_profile}{}%
\begin{description}
\item[Summary:]Loads a raw cip\_trace\_profile given an index in this traceset.
%
\item[Usage:]~%
\begin{lyxcode}%
a\_cip\_trace\_profile = cip\_trace\_profile(traceset, index)
%
\end{lyxcode}%
%
%
\item[Parameters:]~
\begin{description}%
\item[\texttt{traceset}:]
 A cip\_traceset.
\item[\texttt{index}:]
 Index of item in traceset.
\end{description}%
%
\item[Returns:
]~

	a\_cip\_trace\_profile: A cip\_trace\_profile object.
%
%
\item[See also:]%
\hyperlink{ref_cip_trace_profile}{\texttt{cip\_trace\_profile}}%
\ (p.~\pageref{ref_cip_trace_profile})%
\index[funcref]{cip_trace_profile@\fidxl{cip\_trace\_profile}}%
, \hyperlink{ref_params_tests_dataset}{\texttt{params\_tests\_dataset}}%
\ (p.~\pageref{ref_params_tests_dataset})%
\index[funcref]{params_tests_dataset@\fidxl{params\_tests\_dataset}}%
%
\item[Author:]%
Cengiz Gunay <cgunay@emory.edu>, 2004/09/14
%
\end{description}
\methodline%
\subsubsection[Method \texttt{display}]{Method \texttt{cip\_traceset/display}}%
\index[funcref]{cip_traceset@\fidxl{cip\_traceset}!display@\fidxl{display}}%
\label{ref_cip_traceset__display}%
\hypertarget{ref_cip_traceset__display}{}%
\begin{description}
%
%
%
%
%
%
%
\item[Author:]%
Cengiz Gunay <cgunay@emory.edu>, 2004/08/04
%
\end{description}
\methodline%
\subsubsection[Method \texttt{get}]{Method \texttt{cip\_traceset/get}}%
\index[funcref]{cip_traceset@\fidxl{cip\_traceset}!get@\fidxl{get}}%
\label{ref_cip_traceset__get}%
\hypertarget{ref_cip_traceset__get}{}%
\begin{description}
\item[Summary:]Defines generic attribute retrieval for objects.
%
%
%
%
%
%
%
\item[Author:]%
Cengiz Gunay <cgunay@emory.edu>, 2004/09/14
%
\end{description}
\methodline%
\subsubsection[Method \texttt{getItemParams}]{Method \texttt{cip\_traceset/getItemParams}}%
\index[funcref]{cip_traceset@\fidxl{cip\_traceset}!getItemParams@\fidxl{getItemParams}}%
\label{ref_cip_traceset__getItemParams}%
\hypertarget{ref_cip_traceset__getItemParams}{}%
\begin{description}
%
\item[Usage:]~%
\begin{lyxcode}%
params\_row = getParams(dataset, index)
%
\end{lyxcode}%
%
%
\item[Parameters:]~
\begin{description}%
\item[\texttt{dataset}:]
 A params\_tests\_dataset.
\item[\texttt{index}:]
 Index of item in dataset.
\item[\texttt{a\_profile}:]
 A profile object for the item (optional).
\end{description}%
%
\item[Returns:
]~

	params\_row: Parameter values in the same order of paramNames
%
%
\item[See also:]%
\hyperlink{ref_itemResultsRow}{\texttt{itemResultsRow}}%
\ (p.~\pageref{ref_itemResultsRow})%
\index[funcref]{itemResultsRow@\fidxl{itemResultsRow}}%
, \hyperlink{ref_params_tests_dataset}{\texttt{params\_tests\_dataset}}%
\ (p.~\pageref{ref_params_tests_dataset})%
\index[funcref]{params_tests_dataset@\fidxl{params\_tests\_dataset}}%
, \hyperlink{ref_paramNames}{\texttt{paramNames}}%
\ (p.~\pageref{ref_paramNames})%
\index[funcref]{paramNames@\fidxl{paramNames}}%
, \hyperlink{ref_testNames}{\texttt{testNames}}%
\ (p.~\pageref{ref_testNames})%
\index[funcref]{testNames@\fidxl{testNames}}%
%
\item[Author:]%
Cengiz Gunay <cgunay@emory.edu>, 2004/12/06
%
\end{description}
\methodline%
\subsubsection[Method \texttt{loadItemProfile}]{Method \texttt{cip\_traceset/loadItemProfile}}%
\index[funcref]{cip_traceset@\fidxl{cip\_traceset}!loadItemProfile@\fidxl{loadItemProfile}}%
\label{ref_cip_traceset__loadItemProfile}%
\hypertarget{ref_cip_traceset__loadItemProfile}{}%
\begin{description}
\item[Summary:]Loads a profile object from a raw data item in the dataset.
%
\item[Usage:]~%
\begin{lyxcode}%
a\_profile = loadItemProfile(dataset, index)
%
\end{lyxcode}%
%
\item[Description:]%
Subclasses should overload this function to load the specific profile
 object they desire. The profile class should define a getResults method
 which is used in the itemResultsRow method.
%%
\item[Parameters:]~
\begin{description}%
\item[\texttt{dataset}:]
 A params\_tests\_dataset.
\item[\texttt{index}:]
 Index of item in dataset.
\end{description}%
%
\item[Returns:
]~

	a\_profile: A profile object that implements the getResults method.
%
%
\item[See also:]%
\hyperlink{ref_itemResultsRow}{\texttt{itemResultsRow}}%
\ (p.~\pageref{ref_itemResultsRow})%
\index[funcref]{itemResultsRow@\fidxl{itemResultsRow}}%
, \hyperlink{ref_params_tests_dataset}{\texttt{params\_tests\_dataset}}%
\ (p.~\pageref{ref_params_tests_dataset})%
\index[funcref]{params_tests_dataset@\fidxl{params\_tests\_dataset}}%
, \hyperlink{ref_paramNames}{\texttt{paramNames}}%
\ (p.~\pageref{ref_paramNames})%
\index[funcref]{paramNames@\fidxl{paramNames}}%
, \hyperlink{ref_testNames}{\texttt{testNames}}%
\ (p.~\pageref{ref_testNames})%
\index[funcref]{testNames@\fidxl{testNames}}%
%
\item[Author:]%
Cengiz Gunay <cgunay@emory.edu>, 2004/09/14
%
\end{description}
\methodline%
\subsubsection[Method \texttt{paramNames}]{Method \texttt{cip\_traceset/paramNames}}%
\index[funcref]{cip_traceset@\fidxl{cip\_traceset}!paramNames@\fidxl{paramNames}}%
\label{ref_cip_traceset__paramNames}%
\hypertarget{ref_cip_traceset__paramNames}{}%
\begin{description}
\item[Summary:]Returns the only parameter, 'pAcip,' for this traceset.
%
\item[Usage:]~%
\begin{lyxcode}%
param\_names = paramNames(traceset)
%
\end{lyxcode}%
%
\item[Description:]%
Looks at the filename of the first file to find the parameter names.
%%
\item[Parameters:]~
\begin{description}%
\item[\texttt{traceset}:]
 A cip\_traceset.
\end{description}%
%
\item[Returns:
]~

	params\_names: Cell array with ordered parameter names.
%
%
\item[See also:]%
\hyperlink{ref_params_tests_dataset}{\texttt{params\_tests\_dataset}}%
\ (p.~\pageref{ref_params_tests_dataset})%
\index[funcref]{params_tests_dataset@\fidxl{params\_tests\_dataset}}%
, \hyperlink{ref_paramNames}{\texttt{paramNames}}%
\ (p.~\pageref{ref_paramNames})%
\index[funcref]{paramNames@\fidxl{paramNames}}%
, \hyperlink{ref_testNames}{\texttt{testNames}}%
\ (p.~\pageref{ref_testNames})%
\index[funcref]{testNames@\fidxl{testNames}}%
%
\item[Author:]%
Cengiz Gunay <cgunay@emory.edu>, 2004/12/06
%
\end{description}
\methodline%
\subsection{Class \texttt{cip\_traceset\_dataset}}%
\index[funcref]{cip_traceset_dataset@\fidxl{cip\_traceset\_dataset}|boldhyperpage}%
\label{ref_cip_traceset_dataset}%
\hypertarget{ref_cip_traceset_dataset}{}%
\subsubsection[Constructor \texttt{cip\_traceset\_dataset}]{Constructor \texttt{cip\_traceset\_dataset/cip\_traceset\_dataset}}%
\index[funcref]{cip_traceset_dataset@\fidxl{cip\_traceset\_dataset}!cip_traceset_dataset@\fidxl{cip\_traceset\_dataset}}%
\label{ref_cip_traceset_dataset__cip_traceset_dataset}%
\hypertarget{ref_cip_traceset_dataset__cip_traceset_dataset}{}%
\begin{description}
\item[Summary:]Dataset of multiple cip magnitudes from cip\_traces objects .
%
\item[Usage:]~%
\begin{lyxcode}%
obj = cip\_traceset\_dataset(cts, cip\_mags, dy, id, props)
%
\end{lyxcode}%
%
\item[Description:]%
This is a subclass of params\_tests\_dataset. Designed to extract a trace
 for each cip magnitude from the cip\_traceset objects contained. Uses cip\_traceset
 objects to extract multiple traces from each cip\_traces object.
%%
\item[Parameters:]~
\begin{description}%
\item[\texttt{cts}:]
 Array or cell array of cip\_traces objects.
\item[\texttt{cip\_mags}:]
 An array of cip magnitudes to select from each cip\_traces object.
\item[\texttt{dy}:]
 y-axis resolution, [V] or [A] (default = 1e-3).
\item[\texttt{id}:]
 An identification string.
\item[\texttt{props}:]
 A structure with any optional properties passed to cip\_traceset.
\end{description}%
%
\item[Returns a structure object with the following fields:
]~

	params\_tests\_dataset, cip\_mags
%
%
\item[See also:]%
\hyperlink{ref_physiol_cip_traceset}{\texttt{physiol\_cip\_traceset}}%
\ (p.~\pageref{ref_physiol_cip_traceset})%
\index[funcref]{physiol_cip_traceset@\fidxl{physiol\_cip\_traceset}}%
, \hyperlink{ref_params_tests_dataset}{\texttt{params\_tests\_dataset}}%
\ (p.~\pageref{ref_params_tests_dataset})%
\index[funcref]{params_tests_dataset@\fidxl{params\_tests\_dataset}}%
, \hyperlink{ref_params_tests_db}{\texttt{params\_tests\_db}}%
\ (p.~\pageref{ref_params_tests_db})%
\index[funcref]{params_tests_db@\fidxl{params\_tests\_db}}%
%
\item[Author:]%
Cengiz Gunay <cgunay@emory.edu>, 2005/01/28
%
\end{description}
\methodline%
\subsubsection[Method \texttt{display}]{Method \texttt{cip\_traceset\_dataset/display}}%
\index[funcref]{cip_traceset_dataset@\fidxl{cip\_traceset\_dataset}!display@\fidxl{display}}%
\label{ref_cip_traceset_dataset__display}%
\hypertarget{ref_cip_traceset_dataset__display}{}%
\begin{description}
%
%
%
%
%
%
%
\item[Author:]%
Cengiz Gunay <cgunay@emory.edu>, 2004/08/04
%
\end{description}
\methodline%
\subsubsection[Method \texttt{get}]{Method \texttt{cip\_traceset\_dataset/get}}%
\index[funcref]{cip_traceset_dataset@\fidxl{cip\_traceset\_dataset}!get@\fidxl{get}}%
\label{ref_cip_traceset_dataset__get}%
\hypertarget{ref_cip_traceset_dataset__get}{}%
\begin{description}
\item[Summary:]Defines generic attribute retrieval for objects.
%
%
%
%
%
%
%
\item[Author:]%
Cengiz Gunay <cgunay@emory.edu>, 2004/09/14
%
\end{description}
\methodline%
\subsubsection[Method \texttt{loadItemProfile}]{Method \texttt{cip\_traceset\_dataset/loadItemProfile}}%
\index[funcref]{cip_traceset_dataset@\fidxl{cip\_traceset\_dataset}!loadItemProfile@\fidxl{loadItemProfile}}%
\label{ref_cip_traceset_dataset__loadItemProfile}%
\hypertarget{ref_cip_traceset_dataset__loadItemProfile}{}%
\begin{description}
\item[Summary:]Loads a cip\_trace\_profile object from a raw data file in the fileset.
%
\item[Usage:]~%
\begin{lyxcode}%
a\_profile = loadItemProfile(fileset, neuron\_id, trace\_index)
%
\end{lyxcode}%
%
%
\item[Parameters:]~
\begin{description}%
\item[\texttt{fileset}:]
     A physiol\_cip\_traceset object.
\item[\texttt{neuron\_id }:]
  tells which item in fileset (corresponds to cells\_filename) to use grab the cell information 
\item[\texttt{trace\_index}:]
 Index of file in traceset.
\end{description}%
%
\item[Returns:
]~

	a\_profile: A profile object that implements the getResults method.
%
%
\item[See also:]%
\hyperlink{ref_itemResultsRow}{\texttt{itemResultsRow}}%
\ (p.~\pageref{ref_itemResultsRow})%
\index[funcref]{itemResultsRow@\fidxl{itemResultsRow}}%
, \hyperlink{ref_params_tests_fileset}{\texttt{params\_tests\_fileset}}%
\ (p.~\pageref{ref_params_tests_fileset})%
\index[funcref]{params_tests_fileset@\fidxl{params\_tests\_fileset}}%
, \hyperlink{ref_paramNames}{\texttt{paramNames}}%
\ (p.~\pageref{ref_paramNames})%
\index[funcref]{paramNames@\fidxl{paramNames}}%
, \hyperlink{ref_testNames}{\texttt{testNames}}%
\ (p.~\pageref{ref_testNames})%
\index[funcref]{testNames@\fidxl{testNames}}%
%
\item[Author:]%
Cengiz Gunay <cgunay@emory.edu>, 2004/09/14 and Tom Sangrey
%
\end{description}
\methodline%
\subsubsection[Method \texttt{readDBItems}]{Method \texttt{cip\_traceset\_dataset/readDBItems}}%
\index[funcref]{cip_traceset_dataset@\fidxl{cip\_traceset\_dataset}!readDBItems@\fidxl{readDBItems}}%
\label{ref_cip_traceset_dataset__readDBItems}%
\hypertarget{ref_cip_traceset_dataset__readDBItems}{}%
\begin{description}
\item[Summary:]Reads all items to generate a params\_tests\_db object.
%
\item[Usage:]~%
\begin{lyxcode}%
[params, param\_names, tests, test\_names] = readDBItems(obj)
%
\end{lyxcode}%
%
\item[Description:]%
This is a specific method to convert from cip\_traceset\_dataset to
 a params\_tests\_db, or a subclass. Output of this function can be 
 directly fed to the constructor of a params\_tests\_db or a subclass.
%%
\item[Parameters:]~
\begin{description}%
\item[\texttt{obj}:]
 A physiol\_cip\_traceset\_fileset 
\end{description}%
%
\item[Returns:
]~

	params, param\_names, tests, test\_names: See params\_tests\_db.
%
%
\item[See also:]%
\hyperlink{ref_params_tests_db}{\texttt{params\_tests\_db}}%
\ (p.~\pageref{ref_params_tests_db})%
\index[funcref]{params_tests_db@\fidxl{params\_tests\_db}}%
, \hyperlink{ref_params_tests_fileset}{\texttt{params\_tests\_fileset}}%
\ (p.~\pageref{ref_params_tests_fileset})%
\index[funcref]{params_tests_fileset@\fidxl{params\_tests\_fileset}}%
, \hyperlink{ref_itemResultsRow
	    testNames}{\texttt{itemResultsRow
	    testNames}}%
\ (p.~\pageref{ref_itemResultsRow
	    testNames})%
\index[funcref]{itemResultsRow
	    testNames@\fidxl{itemResultsRow
	    testNames}}%
, \hyperlink{ref_paramNames}{\texttt{paramNames}}%
\ (p.~\pageref{ref_paramNames})%
\index[funcref]{paramNames@\fidxl{paramNames}}%
, \hyperlink{ref_physiol_cip_traceset_fileset}{\texttt{physiol\_cip\_traceset\_fileset}}%
\ (p.~\pageref{ref_physiol_cip_traceset_fileset})%
\index[funcref]{physiol_cip_traceset_fileset@\fidxl{physiol\_cip\_traceset\_fileset}}%
%
\item[Author:]%
Cengiz Gunay <cgunay@emory.edu>, 2005/01/28
%
\end{description}
\methodline%
\subsection{Class \texttt{cluster\_db}}%
\index[funcref]{cluster_db@\fidxl{cluster\_db}|boldhyperpage}%
\label{ref_cluster_db}%
\hypertarget{ref_cluster_db}{}%
\subsubsection[Constructor \texttt{cluster\_db}]{Constructor \texttt{cluster\_db/cluster\_db}}%
\index[funcref]{cluster_db@\fidxl{cluster\_db}!cluster_db@\fidxl{cluster\_db}}%
\label{ref_cluster_db__cluster_db}%
\hypertarget{ref_cluster_db__cluster_db}{}%
\begin{description}
\item[Summary:]A database of cluster centroids generated by a clustering algorithm from a rows of orig\_db.
%
\item[Usage:]~%
\begin{lyxcode}%
a\_cluster\_db = cluster\_db(data, col\_names, orig\_db, cluster\_idx, id, props)
%
\end{lyxcode}%
%
\item[Description:]%
This is a subclass of tests\_db. Use one of the clustering methods in 
 tests\_db, such as kmeansCluster, to get an instance of this class.
%%
\item[Parameters:]~
\begin{description}%
\item[\texttt{data}:]
 Database contents.
\item[\texttt{col\_names}:]
 The column names.
\item[\texttt{orig\_db}:]
 DB whose rows are clustered.
\item[\texttt{cluster\_idx}:]
 Array of cluster numbers that correspond to each row in orig\_db.
\item[\texttt{id}:]
 An identifying string.
\item[\texttt{props}:]
 A structure with any optional properties.
\begin{description}%
\item[\texttt{sumDistances}:]
 Total distance of elements within each cluster.
\item[\texttt{distanceMeasure}:]
 Measure used to find clusters (Default='correlation')
\end{description}%
\end{description}%
%
\item[Returns a structure object with the following fields:
]~

	tests\_db, 
	orig\_db: original DB from which clusters were obtained, 
	cluster\_idx: Array associating rows of orig\_db to each cluster here.
	props.
%
%
\item[See also:]%
\hyperlink{ref_tests_db}{\texttt{tests\_db}}%
\ (p.~\pageref{ref_tests_db})%
\index[funcref]{tests_db@\fidxl{tests\_db}}%
, \hyperlink{ref_tests_db__kmeansCluster}{\texttt{tests\_db/kmeansCluster}}%
\ (p.~\pageref{ref_tests_db__kmeansCluster})%
\index[funcref]{tests_db@\fidxl{tests\_db}!kmeansCluster@\fidxl{kmeansCluster}}%
%
\item[Author:]%
Cengiz Gunay <cgunay@emory.edu>, 2005/04/08
%
\end{description}
\methodline%
\subsubsection[Method \texttt{display}]{Method \texttt{cluster\_db/display}}%
\index[funcref]{cluster_db@\fidxl{cluster\_db}!display@\fidxl{display}}%
\label{ref_cluster_db__display}%
\hypertarget{ref_cluster_db__display}{}%
\begin{description}
%
%
%
%
%
%
%
\item[Author:]%
Cengiz Gunay <cgunay@emory.edu>, 2004/08/04
%
\end{description}
\methodline%
\subsubsection[Method \texttt{get}]{Method \texttt{cluster\_db/get}}%
\index[funcref]{cluster_db@\fidxl{cluster\_db}!get@\fidxl{get}}%
\label{ref_cluster_db__get}%
\hypertarget{ref_cluster_db__get}{}%
\begin{description}
\item[Summary:]Defines generic attribute retrieval for objects.
%
%
%
%
%
%
%
\item[Author:]%
Cengiz Gunay <cgunay@emory.edu>, 2004/09/14
%
\end{description}
\methodline%
\subsubsection[Method \texttt{plotHist}]{Method \texttt{cluster\_db/plotHist}}%
\index[funcref]{cluster_db@\fidxl{cluster\_db}!plotHist@\fidxl{plotHist}}%
\label{ref_cluster_db__plotHist}%
\hypertarget{ref_cluster_db__plotHist}{}%
\begin{description}
\item[Summary:]Creates a histogram plot showing the clustering memberships.
%
\item[Usage:]~%
\begin{lyxcode}%
a\_plot = plotHist(a\_cluster\_db, title\_str)
%
\end{lyxcode}%
%
%
\item[Parameters:]~
\begin{description}%
\item[\texttt{a\_cluster\_db}:]
 A cluster\_db object.
\item[\texttt{title\_str}:]
 (Optional) String to append to plot title.
\end{description}%
%
\item[Returns:
]~

	a\_plot: A plot\_abstract object that can be plotted.
%
%
\item[See also:]%
\hyperlink{ref_plot_abstract}{\texttt{plot\_abstract}}%
\ (p.~\pageref{ref_plot_abstract})%
\index[funcref]{plot_abstract@\fidxl{plot\_abstract}}%
, \hyperlink{ref_plotFigure}{\texttt{plotFigure}}%
\ (p.~\pageref{ref_plotFigure})%
\index[funcref]{plotFigure@\fidxl{plotFigure}}%
, \hyperlink{ref_histogram_db}{\texttt{histogram\_db}}%
\ (p.~\pageref{ref_histogram_db})%
\index[funcref]{histogram_db@\fidxl{histogram\_db}}%
, \hyperlink{ref_histogram_db__plot_abstract}{\texttt{histogram\_db/plot\_abstract}}%
\ (p.~\pageref{ref_histogram_db__plot_abstract})%
\index[funcref]{histogram_db@\fidxl{histogram\_db}!plot_abstract@\fidxl{plot\_abstract}}%
%
\item[Author:]%
Cengiz Gunay <cgunay@emory.edu>, 2005/04/08
%
\end{description}
\methodline%
\subsubsection[Method \texttt{plotQuality}]{Method \texttt{cluster\_db/plotQuality}}%
\index[funcref]{cluster_db@\fidxl{cluster\_db}!plotQuality@\fidxl{plotQuality}}%
\label{ref_cluster_db__plotQuality}%
\hypertarget{ref_cluster_db__plotQuality}{}%
\begin{description}
\item[Summary:]Creates a plot\_abstract of the silhouette plot showing the clustering quality.
%
\item[Usage:]~%
\begin{lyxcode}%
a\_plot = plotQuality(a\_cluster\_db, title\_str)
%
\end{lyxcode}%
%
%
\item[Parameters:]~
\begin{description}%
\item[\texttt{a\_cluster\_db}:]
 A cluster\_db object.
\item[\texttt{title\_str}:]
 (Optional) String to append to plot title.
\end{description}%
%
\item[Returns:
]~

	a\_plot: A plot\_abstract object that can be plotted.
%
%
\item[See also:]%
\hyperlink{ref_plot_abstract}{\texttt{plot\_abstract}}%
\ (p.~\pageref{ref_plot_abstract})%
\index[funcref]{plot_abstract@\fidxl{plot\_abstract}}%
, \hyperlink{ref_plotFigure}{\texttt{plotFigure}}%
\ (p.~\pageref{ref_plotFigure})%
\index[funcref]{plotFigure@\fidxl{plotFigure}}%
, \hyperlink{ref_silhouette}{\texttt{silhouette}}%
\ (p.~\pageref{ref_silhouette})%
\index[funcref]{silhouette@\fidxl{silhouette}}%
%
\item[Author:]%
Cengiz Gunay <cgunay@emory.edu>, 2005/04/08
%
\end{description}
\methodline%
\subsubsection[Method \texttt{plot\_abstract}]{Method \texttt{cluster\_db/plot\_abstract}}%
\index[funcref]{cluster_db@\fidxl{cluster\_db}!plot_abstract@\fidxl{plot\_abstract}}%
\label{ref_cluster_db__plot_abstract}%
\hypertarget{ref_cluster_db__plot_abstract}{}%
\begin{description}
\item[Summary:]Creates a vertical plot\_stack of silhouette and membership histograms for the clusters.
%
\item[Usage:]~%
\begin{lyxcode}%
a\_plot = plot\_abstract(a\_cluster\_db, title\_str)
%
\end{lyxcode}%
%
%
\item[Parameters:]~
\begin{description}%
\item[\texttt{a\_cluster\_db}:]
 A cluster\_db object.
\item[\texttt{title\_str}:]
 (Optional) String to append to plot title.
\end{description}%
%
\item[Returns:
]~

	a\_plot: A plot\_abstract object that can be plotted.
%
%
\item[See also:]%
\hyperlink{ref_cluster_db__plotQuality}{\texttt{cluster\_db/plotQuality}}%
\ (p.~\pageref{ref_cluster_db__plotQuality})%
\index[funcref]{cluster_db@\fidxl{cluster\_db}!plotQuality@\fidxl{plotQuality}}%
, \hyperlink{ref_cluster_db__plotHist}{\texttt{cluster\_db/plotHist}}%
\ (p.~\pageref{ref_cluster_db__plotHist})%
\index[funcref]{cluster_db@\fidxl{cluster\_db}!plotHist@\fidxl{plotHist}}%
%
\item[Author:]%
Cengiz Gunay <cgunay@emory.edu>, 2005/04/08
%
\end{description}
\methodline%
\subsection{Class \texttt{corrcoefs\_db}}%
\index[funcref]{corrcoefs_db@\fidxl{corrcoefs\_db}|boldhyperpage}%
\label{ref_corrcoefs_db}%
\hypertarget{ref_corrcoefs_db}{}%
\subsubsection[Constructor \texttt{corrcoefs\_db}]{Constructor \texttt{corrcoefs\_db/corrcoefs\_db}}%
\index[funcref]{corrcoefs_db@\fidxl{corrcoefs\_db}!corrcoefs_db@\fidxl{corrcoefs\_db}}%
\label{ref_corrcoefs_db__corrcoefs_db}%
\hypertarget{ref_corrcoefs_db__corrcoefs_db}{}%
\begin{description}
\item[Summary:]A database of correlation coefficients generated from 
		a column of another database.
%
\item[Usage:]~%
\begin{lyxcode}%
a\_coef\_db = corrcoefs\_db(col\_name, coefs, coef\_names, pages, id, props)
%
\end{lyxcode}%
%
\item[Description:]%
This is a subclass of tests\_3d\_db. Allows generating a plot, etc.
%%
\item[Parameters:]~
\begin{description}%
\item[\texttt{col\_name}:]
 The column with which the others are correlated.
\item[\texttt{coefs}:]
 Matrix where each column has another coefficient.
\item[\texttt{coef\_names}:]
 Cell array of column names corresponding to coefficients.
\item[\texttt{pages}:]
 Column vector of page indices pointing to the tests\_3d\_db.
\item[\texttt{id}:]
 An identifying string.
\item[\texttt{props}:]
 A structure with any optional properties.
\end{description}%
%
\item[Returns a structure object with the following fields:
]~

	tests\_db.
%
%
\item[See also:]%
\hyperlink{ref_tests_db}{\texttt{tests\_db}}%
\ (p.~\pageref{ref_tests_db})%
\index[funcref]{tests_db@\fidxl{tests\_db}}%
, \hyperlink{ref_plot_simple}{\texttt{plot\_simple}}%
\ (p.~\pageref{ref_plot_simple})%
\index[funcref]{plot_simple@\fidxl{plot\_simple}}%
, \hyperlink{ref_tests_db__histogram}{\texttt{tests\_db/histogram}}%
\ (p.~\pageref{ref_tests_db__histogram})%
\index[funcref]{tests_db@\fidxl{tests\_db}!histogram@\fidxl{histogram}}%
%
\item[Author:]%
Cengiz Gunay <cgunay@emory.edu>, 2004/10/06
%
\end{description}
\methodline%
\subsection{Class \texttt{current\_clamp}}%
\index[funcref]{current_clamp@\fidxl{current\_clamp}|boldhyperpage}%
\label{ref_current_clamp}%
\hypertarget{ref_current_clamp}{}%
\subsubsection[Constructor \texttt{current\_clamp}]{Constructor \texttt{current\_clamp/current\_clamp}}%
\index[funcref]{current_clamp@\fidxl{current\_clamp}!current_clamp@\fidxl{current\_clamp}}%
\label{ref_current_clamp__current_clamp}%
\hypertarget{ref_current_clamp__current_clamp}{}%
\begin{description}
\item[Summary:]Current clamp object with current and voltage traces.
%
%
\item[Description:]%
Subclasses the voltage\_clamp object that uses the generic trace object
 to store voltage clamp I, V data. Inherits the common methods defined
 in voltage\_clamp and trace.
%%
\item[Parameters:]~
\begin{description}%
\item[\texttt{a\_vc}:]
 An existing voltage\_clamp object that actually contains

current-clamp data.
\end{description}%
%
\item[Returns a structure object with the following fields:
]~

   voltage\_clamp: voltage\_clamp object.
%
%
\item[See also:]%
\hyperlink{ref_trace}{\texttt{trace}}%
\ (p.~\pageref{ref_trace})%
\index[funcref]{trace@\fidxl{trace}}%
, \hyperlink{ref_period}{\texttt{period}}%
\ (p.~\pageref{ref_period})%
\index[funcref]{period@\fidxl{period}}%
%
\item[Author:]%
Cengiz Gunay <cgunay@emory.edu>, 2010/02/05
%
\end{description}
\methodline%
\subsubsection[Method \texttt{cip\_trace}]{Method \texttt{current\_clamp/cip\_trace}}%
\index[funcref]{current_clamp@\fidxl{current\_clamp}!cip_trace@\fidxl{cip\_trace}}%
\label{ref_current_clamp__cip_trace}%
\hypertarget{ref_current_clamp__cip_trace}{}%
\begin{description}
\item[Summary:]Return a cip\_trace object of the desired current step.
%
\item[Usage:]~%
\begin{lyxcode}%
a\_ct = cip\_trace(a\_cc, cip\_num, props)
%
\end{lyxcode}%
%
%
\item[Parameters:]~
\begin{description}%
\item[\texttt{a\_cc}:]
 A cip\_trace object.
\item[\texttt{cip\_num}:]
 Index of CIP level.
\item[\texttt{props}:]
 A structure with any optional properties.
\begin{description}%
\item[\texttt{stepNum}:]
 Current step to get results for (default=2).
\end{description}%
\end{description}%
%
\item[Returns:
]~

   a\_ct: A cip\_trace object with voltage.
   cip\_level\_pA: applied current magnitude [pA].
%
%
\item[See also:]%
\hyperlink{ref_cip_trace}{\texttt{cip\_trace}}%
\ (p.~\pageref{ref_cip_trace})%
\index[funcref]{cip_trace@\fidxl{cip\_trace}}%
, \hyperlink{ref_trace}{\texttt{trace}}%
\ (p.~\pageref{ref_trace})%
\index[funcref]{trace@\fidxl{trace}}%
, \hyperlink{ref_spike_shape}{\texttt{spike\_shape}}%
\ (p.~\pageref{ref_spike_shape})%
\index[funcref]{spike_shape@\fidxl{spike\_shape}}%
%
\item[Author:]%
Cengiz Gunay <cgunay@emory.edu>, 2011/02/22
%
\end{description}
\methodline%
\subsubsection[Method \texttt{get}]{Method \texttt{current\_clamp/get}}%
\index[funcref]{current_clamp@\fidxl{current\_clamp}!get@\fidxl{get}}%
\label{ref_current_clamp__get}%
\hypertarget{ref_current_clamp__get}{}%
\begin{description}
\item[Summary:]Defines generic attribute retrieval for objects.
%
%
%
%
%
%
%
\item[Author:]%
Cengiz Gunay <cgunay@emory.edu>, 2004/09/14
%
\end{description}
\methodline%
\subsubsection[Method \texttt{getResults}]{Method \texttt{current\_clamp/getResults}}%
\index[funcref]{current_clamp@\fidxl{current\_clamp}!getResults@\fidxl{getResults}}%
\label{ref_current_clamp__getResults}%
\hypertarget{ref_current_clamp__getResults}{}%
\begin{description}
\item[Summary:]Extract measurement results from all current steps.
%
\item[Usage:]~%
\begin{lyxcode}%
[results profs] = getResults(a\_cc)
%
\end{lyxcode}%
%
%
\item[Parameters:]~
\begin{description}%
\item[\texttt{a\_cc}:]
 A cip\_trace object.
\item[\texttt{props}:]
 A structure with any optional properties.
\begin{description}%
\item[\texttt{stepNum}:]
 Current step to get results for (default=2).
\end{description}%
\end{description}%
%
\item[Returns:
]~

   results: A structure associating test names with result values.
   profs: Cell array of results\_profile objects for each current step.
%
%
\item[See also:]%
\hyperlink{ref_cip_trace}{\texttt{cip\_trace}}%
\ (p.~\pageref{ref_cip_trace})%
\index[funcref]{cip_trace@\fidxl{cip\_trace}}%
, \hyperlink{ref_trace}{\texttt{trace}}%
\ (p.~\pageref{ref_trace})%
\index[funcref]{trace@\fidxl{trace}}%
, \hyperlink{ref_spike_shape}{\texttt{spike\_shape}}%
\ (p.~\pageref{ref_spike_shape})%
\index[funcref]{spike_shape@\fidxl{spike\_shape}}%
%
\item[Author:]%
Cengiz Gunay <cgunay@emory.edu>, 2011/02/22
%
\end{description}
\methodline%
\subsubsection[Method \texttt{params\_tests\_db}]{Method \texttt{current\_clamp/params\_tests\_db}}%
\index[funcref]{current_clamp@\fidxl{current\_clamp}!params_tests_db@\fidxl{params\_tests\_db}}%
\label{ref_current_clamp__params_tests_db}%
\hypertarget{ref_current_clamp__params_tests_db}{}%
\begin{description}
\item[Summary:]Create a database of measurement results changing with applied current.
%
\item[Usage:]~%
\begin{lyxcode}%
a\_db = params\_tests\_db(a\_cc, props)
%
\end{lyxcode}%
%
\item[Description:]%
Selects cip\_level\_pA as the only database parameter. 
%%
\item[Parameters:]~
\begin{description}%
\item[\texttt{a\_cc}:]
 A cip\_trace object.
\item[\texttt{props}:]
 A structure with any optional properties.
\begin{description}%
\item[\texttt{stepNum}:]
 Current period to get results for. Choose 1 for the

initial period, 2 for the pulse period (default) and so on).
\item[\texttt{paramsStruct}:]
 Contains parameter names and values that are constant

for these traces.
\item[\texttt{paramsVary}:]
 Contains parameter names and their varying values for

each of these traces in a structure array (e.g.,
struct('Na', {10, 50 ,100}))
\end{description}%
\end{description}%
%
\item[Returns:
]~

   a\_db: A params\_tests\_db with results collected from getResults
   profs: (Optional) Cell array of cip\_trace\_allspikes\_profile objects for all current steps.
%
%
\item[See also:]%
\hyperlink{ref_getResults}{\texttt{getResults}}%
\ (p.~\pageref{ref_getResults})%
\index[funcref]{getResults@\fidxl{getResults}}%
, \hyperlink{ref_cip_trace}{\texttt{cip\_trace}}%
\ (p.~\pageref{ref_cip_trace})%
\index[funcref]{cip_trace@\fidxl{cip\_trace}}%
, \hyperlink{ref_trace}{\texttt{trace}}%
\ (p.~\pageref{ref_trace})%
\index[funcref]{trace@\fidxl{trace}}%
, \hyperlink{ref_spike_shape}{\texttt{spike\_shape}}%
\ (p.~\pageref{ref_spike_shape})%
\index[funcref]{spike_shape@\fidxl{spike\_shape}}%
%
\item[Author:]%
Cengiz Gunay <cgunay@emory.edu>, 2011/02/23
%
\end{description}
\methodline%
\subsubsection[Method \texttt{set}]{Method \texttt{current\_clamp/set}}%
\index[funcref]{current_clamp@\fidxl{current\_clamp}!set@\fidxl{set}}%
\label{ref_current_clamp__set}%
\hypertarget{ref_current_clamp__set}{}%
\begin{description}
\item[Summary:]Generic method for setting object attributes.
%
%
%
%
%
%
%
\item[Author:]%
Cengiz Gunay <cgunay@emory.edu>, 2004/10/08
%
\end{description}
\methodline%
\subsection{Class \texttt{dataset\_db\_bundle}}%
\index[funcref]{dataset_db_bundle@\fidxl{dataset\_db\_bundle}|boldhyperpage}%
\label{ref_dataset_db_bundle}%
\hypertarget{ref_dataset_db_bundle}{}%
\subsubsection[Constructor \texttt{dataset\_db\_bundle}]{Constructor \texttt{dataset\_db\_bundle/dataset\_db\_bundle}}%
\index[funcref]{dataset_db_bundle@\fidxl{dataset\_db\_bundle}!dataset_db_bundle@\fidxl{dataset\_db\_bundle}}%
\label{ref_dataset_db_bundle__dataset_db_bundle}%
\hypertarget{ref_dataset_db_bundle__dataset_db_bundle}{}%
\begin{description}
\item[Summary:]The dataset and the DB created from it bundled together.
%
\item[Usage:]~%
\begin{lyxcode}%
a\_bundle = dataset\_db\_bundle(a\_dataset, a\_db, a\_joined\_db, props)
%
\end{lyxcode}%
%
\item[Description:]%
This class is made to enable operations that require seamless connection
 between the high-level (joined) DB and the raw data. The raw DB is only
 required to make a connection to the dataset. Therefore it only needs to
 contain columns necessary to make this connection (e.g., ItemIndex) and
 other columns can be discarded to save space. The raw DB corresponds
 row-to-row to the dataset. The joined DB is a higher-level database where
 multiple rows from the raw DB is combined into single rows that represent
 entities (trials, neurons, etc). This is achieved with a function like
 mergePages or mergeMultipleCIPsInOne. There may be several steps for this
 process, which can be specified as a function handle in the joinDBfunc
 property.
%%
\item[Parameters:]~
\begin{description}%
\item[\texttt{a\_dataset}:]
 A params\_tests\_dataset object or a subclass.
\item[\texttt{a\_db}:]
 The raw tests\_db object (or a subclass) created from the dataset.
\item[\texttt{a\_joined\_db}:]
 The processed DB created from the raw DB.
\item[\texttt{props}:]
 A structure with any optional properties.
\begin{description}%
\item[\texttt{joinDBfunc}:]
 A function(a\_db) to be called that can generate

a\_joined\_db from it.
\end{description}%
\end{description}%
%
\item[Returns a structure object with the following fields:
]~

	dataset, db, joined\_db, props.
%
%
\item[See also:]%
\hyperlink{ref_tests_db}{\texttt{tests\_db}}%
\ (p.~\pageref{ref_tests_db})%
\index[funcref]{tests_db@\fidxl{tests\_db}}%
, \hyperlink{ref_params_tests_dataset}{\texttt{params\_tests\_dataset}}%
\ (p.~\pageref{ref_params_tests_dataset})%
\index[funcref]{params_tests_dataset@\fidxl{params\_tests\_dataset}}%
%
\item[Author:]%
Cengiz Gunay <cgunay@emory.edu>, 2005/12/13
%
\end{description}
\methodline%
\subsubsection[Method \texttt{constrainedMeasuresPreset}]{Method \texttt{dataset\_db\_bundle/constrainedMeasuresPreset}}%
\index[funcref]{dataset_db_bundle@\fidxl{dataset\_db\_bundle}!constrainedMeasuresPreset@\fidxl{constrainedMeasuresPreset}}%
\label{ref_dataset_db_bundle__constrainedMeasuresPreset}%
\hypertarget{ref_dataset_db_bundle__constrainedMeasuresPreset}{}%
\begin{description}
\item[Summary:]Returns a dataset\_db\_bundle with constrained measures according to chosen preset.
%
\item[Usage:]~%
\begin{lyxcode}%
[a\_bundle test\_names] = constrainedMeasuresPreset(a\_bundle, preset, props)
%
\end{lyxcode}%
%
%
\item[Parameters:]~
\begin{description}%
\item[\texttt{a\_bundle}:]
 A dataset\_db\_bundle object.
\item[\texttt{preset}:]
 Choose preset measure list (default=1).
\item[\texttt{props}:]
 A structure with any optional properties.
\end{description}%
%
\item[Returns:
]~

	a\_bundle: Modified bundle.
%
%
\item[See also:]%
\hyperlink{ref_physiol_bundle__constrainedMeasuresPreset}{\texttt{physiol\_bundle/constrainedMeasuresPreset}}%
\ (p.~\pageref{ref_physiol_bundle__constrainedMeasuresPreset})%
\index[funcref]{physiol_bundle@\fidxl{physiol\_bundle}!constrainedMeasuresPreset@\fidxl{constrainedMeasuresPreset}}%
%
\item[Author:]%
Cengiz Gunay <cgunay@emory.edu>, 2006/06/13
%
\end{description}
\methodline%
\subsubsection[Method \texttt{ctFromRows}]{Method \texttt{dataset\_db\_bundle/ctFromRows}}%
\index[funcref]{dataset_db_bundle@\fidxl{dataset\_db\_bundle}!ctFromRows@\fidxl{ctFromRows}}%
\label{ref_dataset_db_bundle__ctFromRows}%
\hypertarget{ref_dataset_db_bundle__ctFromRows}{}%
\begin{description}
\item[Summary:]Loads a cip\_trace object from a raw data file in the a\_bundle.
%
\item[Usage:]~%
\begin{lyxcode}%
a\_cip\_trace = ctFromRows(a\_bundle, a\_db, cip\_levels, props)
%
\end{lyxcode}%
%
\item[Description:]%
This method is not implemented for the generic dataset\_db\_bundle class. See 
 subclass implementations.
%%
\item[Parameters:]~
\begin{description}%
\item[\texttt{a\_bundle}:]
 A dataset\_db\_bundle object.
\item[\texttt{a\_db}:]
 A DB created by the dataset in the a\_bundle to read the neuron index numbers from.
\item[\texttt{cip\_levels}:]
 A column vector of CIP-levels to be loaded.
\item[\texttt{props}:]
 A structure with any optional properties.

(passed to a\_bundle.dataset/cip\_trace)
\end{description}%
%
\item[Returns:
]~

	a\_cip\_trace: One or more cip\_trace objects that hold the raw data.
%
%
\item[See also:]%
\hyperlink{ref_model_ct_bundle__ctFromRows}{\texttt{model\_ct\_bundle/ctFromRows}}%
\ (p.~\pageref{ref_model_ct_bundle__ctFromRows})%
\index[funcref]{model_ct_bundle@\fidxl{model\_ct\_bundle}!ctFromRows@\fidxl{ctFromRows}}%
, \hyperlink{ref_physiol_bundle__ctFromRows}{\texttt{physiol\_bundle/ctFromRows}}%
\ (p.~\pageref{ref_physiol_bundle__ctFromRows})%
\index[funcref]{physiol_bundle@\fidxl{physiol\_bundle}!ctFromRows@\fidxl{ctFromRows}}%
%
\item[Author:]%
Cengiz Gunay <cgunay@emory.edu>, 2006/10/11
%
\end{description}
\methodline%
\subsubsection[Method \texttt{display}]{Method \texttt{dataset\_db\_bundle/display}}%
\index[funcref]{dataset_db_bundle@\fidxl{dataset\_db\_bundle}!display@\fidxl{display}}%
\label{ref_dataset_db_bundle__display}%
\hypertarget{ref_dataset_db_bundle__display}{}%
\begin{description}
%
%
%
%
%
%
%
\item[Author:]%
Cengiz Gunay <cgunay@emory.edu>, 2004/08/04
%
\end{description}
\methodline%
\subsubsection[Method \texttt{get}]{Method \texttt{dataset\_db\_bundle/get}}%
\index[funcref]{dataset_db_bundle@\fidxl{dataset\_db\_bundle}!get@\fidxl{get}}%
\label{ref_dataset_db_bundle__get}%
\hypertarget{ref_dataset_db_bundle__get}{}%
\begin{description}
\item[Summary:]Defines generic attribute retrieval for objects.
%
%
%
%
%
%
%
\item[Author:]%
Cengiz Gunay <cgunay@emory.edu>, 2004/09/14
%
\end{description}
\methodline%
\subsubsection[Method \texttt{getNeuronRowIndex}]{Method \texttt{dataset\_db\_bundle/getNeuronRowIndex}}%
\index[funcref]{dataset_db_bundle@\fidxl{dataset\_db\_bundle}!getNeuronRowIndex@\fidxl{getNeuronRowIndex}}%
\label{ref_dataset_db_bundle__getNeuronRowIndex}%
\hypertarget{ref_dataset_db_bundle__getNeuronRowIndex}{}%
\begin{description}
\item[Summary:]Returns the neuron index from bundle.
%
\item[Usage:]~%
\begin{lyxcode}%
a\_row\_index = getNeuronRowIndex(a\_bundle, an\_index, props)
%
\end{lyxcode}%
%
\item[Description:]%
This is a polymorphic method. Therefor it is not defined for this class, 
 but see subclasses of dataset\_db\_bundle for its more meaningful implementations.
%%
\item[Parameters:]~
\begin{description}%
\item[\texttt{a\_bundle}:]
 A dataset\_db\_bundle subclass object.
\item[\texttt{an\_index}:]
 An index number of neuron, or a DB row containing this.
\item[\texttt{props}:]
 A structure with any optional properties.
\end{description}%
%
\item[Returns:
]~

	a\_row\_index: A row index of neuron in a\_bundle.joined\_db.
%
\item[Example:]~
\begin{lyxcode} >> displayRows(mbundle.joined\_db(getNeuronRowIndex(mbundle, 98364), :))
\\%
\end{lyxcode}
%
\item[See also:]%
\hyperlink{ref_dataset_db_bundle}{\texttt{dataset\_db\_bundle}}%
\ (p.~\pageref{ref_dataset_db_bundle})%
\index[funcref]{dataset_db_bundle@\fidxl{dataset\_db\_bundle}}%
%
\item[Author:]%
Cengiz Gunay <cgunay@emory.edu>, 2006/06/09
%
\end{description}
\methodline%
\subsubsection[Method \texttt{matchingRow}]{Method \texttt{dataset\_db\_bundle/matchingRow}}%
\index[funcref]{dataset_db_bundle@\fidxl{dataset\_db\_bundle}!matchingRow@\fidxl{matchingRow}}%
\label{ref_dataset_db_bundle__matchingRow}%
\hypertarget{ref_dataset_db_bundle__matchingRow}{}%
\begin{description}
\item[Summary:]Creates a criterion database for matching the tests of a row.
%
\item[Usage:]~%
\begin{lyxcode}%
crit\_db = matchingRow(a\_bundle, row, props)
%
\end{lyxcode}%
%
\item[Description:]%
Copies selected test values from row as the first row into the 
 criterion db. Adds a second row for the STD of each column in the db.
%%
\item[Parameters:]~
\begin{description}%
\item[\texttt{a\_bundle}:]
 A tests\_db object.
\item[\texttt{row}:]
 A row index to match.
\item[\texttt{props}:]
 A structure with any optional properties.
\begin{description}%
\item[\texttt{distDB}:]
 Take the standard deviation from this db instead.
\end{description}%
\end{description}%
%
\item[Returns:
]~

	crit\_db: A tests\_db with two rows for values and STDs.
%
\item[Example:]~
\begin{lyxcode}        physiol\_bundle has an overloaded matchingRow method that
\\%
        takes the TracesetIndex as argument:
\\%
        >> crit\_db = matchingRow(pbundle, 61)
\\%
\end{lyxcode}
%
\item[See also:]%
\hyperlink{ref_rankMatching}{\texttt{rankMatching}}%
\ (p.~\pageref{ref_rankMatching})%
\index[funcref]{rankMatching@\fidxl{rankMatching}}%
, \hyperlink{ref_tests_db}{\texttt{tests\_db}}%
\ (p.~\pageref{ref_tests_db})%
\index[funcref]{tests_db@\fidxl{tests\_db}}%
, \hyperlink{ref_tests2cols}{\texttt{tests2cols}}%
\ (p.~\pageref{ref_tests2cols})%
\index[funcref]{tests2cols@\fidxl{tests2cols}}%
%
\item[Author:]%
Cengiz Gunay <cgunay@emory.edu>, 2005/12/21
%
\end{description}
\methodline%
\subsubsection[Method \texttt{plotfICurve}]{Method \texttt{dataset\_db\_bundle/plotfICurve}}%
\index[funcref]{dataset_db_bundle@\fidxl{dataset\_db\_bundle}!plotfICurve@\fidxl{plotfICurve}}%
\label{ref_dataset_db_bundle__plotfICurve}%
\hypertarget{ref_dataset_db_bundle__plotfICurve}{}%
\begin{description}
\item[Summary:]Generates a f-I curve doc\_plot for neuron at given an\_index in a\_bundle.
%
\item[Usage:]~%
\begin{lyxcode}%
a\_plot = plotfICurve(a\_bundle, trial\_num, props)
%
\end{lyxcode}%
%
%
\item[Parameters:]~
\begin{description}%
\item[\texttt{a\_bundle}:]
 A dataset\_db\_bundle object.
\item[\texttt{an\_index}:]
 An index with which to address the a\_bundle.
\item[\texttt{props}:]
 A structure with any optional properties.
\begin{description}%
\item[\texttt{shortCaption}:]
 This appears in the figure caption.
\item[\texttt{plotMStats}:]
 If set, add the a\_bundle stats plot.
\item[\texttt{captionToStats}:]
 Use this as its legend label. 
\item[\texttt{quiet}:]
 if given, no title is produced

(passed to plot\_superpose)
\end{description}%
\end{description}%
%
\item[Returns:
]~

	a\_plot: A plot\_superpose that contains a f-I curve plot.
%
\item[Example:]~
\begin{lyxcode} >> a\_p = plotfICurve(r, 1);
\\%
 >> plotFigure(a\_p, 'The f-I curve of best matching model');
\\%
\end{lyxcode}
%
\item[See also:]%
\hyperlink{ref_plot_abstract}{\texttt{plot\_abstract}}%
\ (p.~\pageref{ref_plot_abstract})%
\index[funcref]{plot_abstract@\fidxl{plot\_abstract}}%
, \hyperlink{ref_plot_superpose}{\texttt{plot\_superpose}}%
\ (p.~\pageref{ref_plot_superpose})%
\index[funcref]{plot_superpose@\fidxl{plot\_superpose}}%
%
\item[Author:]%
Cengiz Gunay <cgunay@emory.edu>, 2006/01/16
%
\end{description}
\methodline%
\subsubsection[Method \texttt{profileFromRows}]{Method \texttt{dataset\_db\_bundle/profileFromRows}}%
\index[funcref]{dataset_db_bundle@\fidxl{dataset\_db\_bundle}!profileFromRows@\fidxl{profileFromRows}}%
\label{ref_dataset_db_bundle__profileFromRows}%
\hypertarget{ref_dataset_db_bundle__profileFromRows}{}%
\begin{description}
\item[Summary:]Loads a cip\_trace object from a raw data file in the a\_bundle.
%
\item[Usage:]~%
\begin{lyxcode}%
a\_prof = profileFromRows(a\_bundle, indices, props)
%
\end{lyxcode}%
%
%
\item[Parameters:]~
\begin{description}%
\item[\texttt{a\_bundle}:]
 A dataset\_db\_bundle object.
\item[\texttt{indices}:]
 Array of trial/ItemIndex numbers, or a db that contains rows of them.
\item[\texttt{props}:]
 A structure with any optional properties.
\begin{description}%
\item[\texttt{indexName}:]
 Column to take the ItemIndex from.

(passed to params\_tests\_dataset/loadItemProfile)
\end{description}%
\end{description}%
%
\item[Returns:
]~

   a\_prof: One or more results\_profile objects.
%
%
\item[See also:]%
\hyperlink{ref_params_tests_dataset__loadItemProfile}{\texttt{params\_tests\_dataset/loadItemProfile}}%
\ (p.~\pageref{ref_params_tests_dataset__loadItemProfile})%
\index[funcref]{params_tests_dataset@\fidxl{params\_tests\_dataset}!loadItemProfile@\fidxl{loadItemProfile}}%
%
\item[Author:]%
Cengiz Gunay <cgunay@emory.edu>, 2014/06/19
%
\end{description}
\methodline%
\subsubsection[Method \texttt{rankingReportTeX}]{Method \texttt{dataset\_db\_bundle/rankingReportTeX}}%
\index[funcref]{dataset_db_bundle@\fidxl{dataset\_db\_bundle}!rankingReportTeX@\fidxl{rankingReportTeX}}%
\label{ref_dataset_db_bundle__rankingReportTeX}%
\hypertarget{ref_dataset_db_bundle__rankingReportTeX}{}%
\begin{description}
\item[Summary:]Generates a report by comparing a\_bundle with the given match criteria, crit\_db from crit\_bundle.
%
\item[Usage:]~%
\begin{lyxcode}%
tex\_string = rankingReportTeX(a\_bundle, crit\_bundle, crit\_db, props)
%
\end{lyxcode}%
%
\item[Description:]%
Generates a LaTeX document with:
	- (optional) Raw traces compared with some best matches at different distances
	- Values of some top matching a\_db rows and match errors in a floating table.
	- colored-plot of measure errors for some top matches.
	- Parameter distributions of 50 best matches as a bar graph.
%%
\item[Parameters:]~
\begin{description}%
\item[\texttt{a\_bundle}:]
 A dataset\_db\_bundle object that contains the DB to compare rows from.
\item[\texttt{crit\_bundle}:]
 A dataset\_db\_bundle object that contains the criterion dataset.
\item[\texttt{crit\_db}:]
 A tests\_db object holding the match criterion tests and STDs

which can be created with matchingRow.
\item[\texttt{props}:]
 A structure with any optional properties.
\begin{description}%
\item[\texttt{caption}:]
 Identification of the criterion db (not needed/used?).
\item[\texttt{num\_matches}:]
 Number of best matches to display (default=10).
\item[\texttt{rotate}:]
 Rotation angle for best matches table (default=90).
\end{description}%
\end{description}%
%
\item[Returns:
]~

	tex\_string: LaTeX document string.
%
%
\item[See also:]%
\hyperlink{ref_displayRowsTeX}{\texttt{displayRowsTeX}}%
\ (p.~\pageref{ref_displayRowsTeX})%
\index[funcref]{displayRowsTeX@\fidxl{displayRowsTeX}}%
%
\item[Author:]%
Cengiz Gunay <cgunay@emory.edu>, 2005/12/13
%
\end{description}
\methodline%
\subsubsection[Method \texttt{reportNeuron}]{Method \texttt{dataset\_db\_bundle/reportNeuron}}%
\index[funcref]{dataset_db_bundle@\fidxl{dataset\_db\_bundle}!reportNeuron@\fidxl{reportNeuron}}%
\label{ref_dataset_db_bundle__reportNeuron}%
\hypertarget{ref_dataset_db_bundle__reportNeuron}{}%
\begin{description}
\item[Summary:]Generates a report of neuron at given an\_index of a\_bundle.
%
\item[Usage:]~%
\begin{lyxcode}%
a\_doc = reportNeuron(a\_bundle, an\_index, props)
%
\end{lyxcode}%
%
\item[Description:]%
Generates a report document with preset layouts of annotated plots of
 the selected neuron. See reportLayout below for presets.
%%
\item[Parameters:]~
\begin{description}%
\item[\texttt{a\_bundle}:]
 a dataset\_db\_bundle object which contains the neuron
\item[\texttt{an\_index}:]
 The index to pass to ctFromRows method of a\_bundle.
\item[\texttt{props}:]
 A structure with any optional properties.
\begin{description}%
\item[\texttt{reportLayout}:]
 Allows choosing one of predefined report types (strings):
\begin{description}%
\item[\texttt{1}:]
 Only +/- 100 pA traces in one plot (default).

1a/b: Either one of the +/- 100 pA traces in one plot.
\item[\texttt{2}:]
 Only +/- 100 pA traces and spike shapes in one horiz. plot.
\item[\texttt{2a}:]
 +/- 100 pA traces, f-I curve and spike shapes in one horiz. plot.
\item[\texttt{3}:]
 +100 pA raw trace and rate profile stacked vertically.
\item[\texttt{3b}:]
 -100 pA raw trace and rate profile stacked vertically.
\item[\texttt{4}:]
 Horiz stack of +/- 100 pA raw trace with rate profiles underneath.
\item[\texttt{5}:]
 5-piece trace, spike shape, f-I curve, f-t curve quad-plot.
\end{description}%
\item[\texttt{numTraces}:]
 Limit number of traces to show in plot (>=1).
\item[\texttt{traces}:]
 List of acceptable traces to load.
\item[\texttt{traceAxisLimits}:]
 If given, use these limits for trace plots.
\item[\texttt{rateAxisLimits}:]
 If given, use these limits for rate plots.
\item[\texttt{fIAxisLimits}:]
 If given, use these limits for fIcurve plots.
\item[\texttt{fIstats}:]
 Add a fI-stats plot in addition to the curve.
\item[\texttt{sshapeAxisLimits}:]
 If given, use these limits for spike shape plots.
\item[\texttt{sshapeResults}:]
 If 1, plot measures on the spike shape (default=1).
\end{description}%
\end{description}%
%
\item[Returns:
]~

 	a\_doc: A doc\_generate object, or a subclass, that can be plotted, or
		printed as a PS or PDF file.
%
\item[Example:]~
\begin{lyxcode} >> printTeXFile(reportNeuron(mbundle, 2222), 'a.tex')
\\%
 or:
\\%
 >> plotFigure(get(reportNeuron(mbundle, 2222), 'plot'))
\\%
 or if the result is one or many doc\_plot objects:
\\%
 >> plot(reportNeuron(mbundle, 2222:2224, struct('reportLayout', '5')))
\\%
\end{lyxcode}
%
\item[See also:]%
\hyperlink{ref_doc_multi}{\texttt{doc\_multi}}%
\ (p.~\pageref{ref_doc_multi})%
\index[funcref]{doc_multi@\fidxl{doc\_multi}}%
, \hyperlink{ref_doc_generate}{\texttt{doc\_generate}}%
\ (p.~\pageref{ref_doc_generate})%
\index[funcref]{doc_generate@\fidxl{doc\_generate}}%
, \hyperlink{ref_doc_generate__printTeXFile}{\texttt{doc\_generate/printTeXFile}}%
\ (p.~\pageref{ref_doc_generate__printTeXFile})%
\index[funcref]{doc_generate@\fidxl{doc\_generate}!printTeXFile@\fidxl{printTeXFile}}%
%
\item[Author:]%
Cengiz Gunay <cgunay@emory.edu>, 2006/01/24
%
\end{description}
\methodline%
\subsubsection[Method \texttt{set}]{Method \texttt{dataset\_db\_bundle/set}}%
\index[funcref]{dataset_db_bundle@\fidxl{dataset\_db\_bundle}!set@\fidxl{set}}%
\label{ref_dataset_db_bundle__set}%
\hypertarget{ref_dataset_db_bundle__set}{}%
\begin{description}
\item[Summary:]Generic method for setting object attributes.
%
%
%
%
%
%
%
\item[Author:]%
Cengiz Gunay <cgunay@emory.edu>, 2004/10/08
%
\end{description}
\methodline%
\subsubsection[Method \texttt{simNewParams}]{Method \texttt{dataset\_db\_bundle/simNewParams}}%
\index[funcref]{dataset_db_bundle@\fidxl{dataset\_db\_bundle}!simNewParams@\fidxl{simNewParams}}%
\label{ref_dataset_db_bundle__simNewParams}%
\hypertarget{ref_dataset_db_bundle__simNewParams}{}%
\begin{description}
\item[Summary:]Simulates new parameter set and adds it to dataset and DB.
%
\item[Usage:]~%
\begin{lyxcode}%
[a\_bundle, a\_new\_db, a\_new\_joined\_db] = simNewParams(a\_bundle, a\_param\_row\_db, props)
%
\end{lyxcode}%
%
%
\item[Parameters:]~
\begin{description}%
\item[\texttt{a\_bundle}:]
 A dataset\_db\_bundle object.
\item[\texttt{a\_db}:]
 Rows with new parameters.
\item[\texttt{props}:]
 A structure with any optional properties.
\begin{description}%
\item[\texttt{simFunc}:]
 Handle to function to call for simulating the parameter

row db (e.g., @(row\_db)).
\item[\texttt{trial}:]
 Trial number for new parameter set. It also indicates

parallel mode, which does not alter existing files, but instead
creates new files with this trial number suffix.
\item[\texttt{writePar}:]
 If 1, create a new par file (default=0).
\end{description}%
\end{description}%
%
\item[Returns:
]~

   a\_bundle: The bundle with updated parameters and measures.
   a\_new\_db: Newly created rows.
   a\_new\_joined\_db: Newly created rows joined.
%
%
\item[See also:]%
\hyperlink{ref_params_tests_dataset__loadItemProfile}{\texttt{params\_tests\_dataset/loadItemProfile}}%
\ (p.~\pageref{ref_params_tests_dataset__loadItemProfile})%
\index[funcref]{params_tests_dataset@\fidxl{params\_tests\_dataset}!loadItemProfile@\fidxl{loadItemProfile}}%
%
\item[Author:]%
Cengiz Gunay <cgunay@emory.edu>, 2014/06/23
%
\end{description}
\methodline%
\subsubsection[Method \texttt{subsasgn}]{Method \texttt{dataset\_db\_bundle/subsasgn}}%
\index[funcref]{dataset_db_bundle@\fidxl{dataset\_db\_bundle}!subsasgn@\fidxl{subsasgn}}%
\label{ref_dataset_db_bundle__subsasgn}%
\hypertarget{ref_dataset_db_bundle__subsasgn}{}%
\begin{description}
\item[Summary:]Defines generic index-based assignment for objects.
%
%
%
%
%
%
%
\item[Author:]%
Cengiz Gunay <cgunay@emory.edu>, 2006/02/06
%
\end{description}
\methodline%
\subsubsection[Method \texttt{subsref}]{Method \texttt{dataset\_db\_bundle/subsref}}%
\index[funcref]{dataset_db_bundle@\fidxl{dataset\_db\_bundle}!subsref@\fidxl{subsref}}%
\label{ref_dataset_db_bundle__subsref}%
\hypertarget{ref_dataset_db_bundle__subsref}{}%
\begin{description}
\item[Summary:]Defines indexing for tests\_db objects for () and . operations. 
%
\item[Usage:]~%
\begin{lyxcode}%
obj = obj(rows, tests)
 obj = obj.attribute
%
\end{lyxcode}%
%
\item[Description:]%
Returns attributes or selects the given test columns and rows
 and returns in a new tests\_db object.
%%
\item[Parameters:]~
\begin{description}%
\item[\texttt{obj}:]
 A tests\_db object.
\item[\texttt{rows}:]
 A logical or index vector of rows. If ':', all rows.
\item[\texttt{tests}:]
 Cell array of test names or column indices. If ':', all tests.
\item[\texttt{attribute}:]
 A tests\_db class attribute.
\end{description}%
%
\item[Returns:
]~

	obj: The new tests\_db object.
%
%
\item[See also:]%
\hyperlink{ref_subsref}{\texttt{subsref}}%
\ (p.~\pageref{ref_subsref})%
\index[funcref]{subsref@\fidxl{subsref}}%
, \hyperlink{ref_tests_db}{\texttt{tests\_db}}%
\ (p.~\pageref{ref_tests_db})%
\index[funcref]{tests_db@\fidxl{tests\_db}}%
%
\item[Author:]%
Cengiz Gunay <cgunay@emory.edu>, 2004/09/17
%
\end{description}
\methodline%
\subsection{Class \texttt{doc\_generate}}%
\index[funcref]{doc_generate@\fidxl{doc\_generate}|boldhyperpage}%
\label{ref_doc_generate}%
\hypertarget{ref_doc_generate}{}%
\subsubsection[Constructor \texttt{doc\_generate}]{Constructor \texttt{doc\_generate/doc\_generate}}%
\index[funcref]{doc_generate@\fidxl{doc\_generate}!doc_generate@\fidxl{doc\_generate}}%
\label{ref_doc_generate__doc_generate}%
\hypertarget{ref_doc_generate__doc_generate}{}%
\begin{description}
\item[Summary:]Generic class to help generate printed or annotated documents with results.
%
\item[Usage:]~%
\begin{lyxcode}%
a\_doc = doc\_generate(text\_string, id, props)
%
\end{lyxcode}%
%
\item[Description:]%
This constitutes the base class for other doc\_ classes. For convenience,
 this class holds a text\_string to be printed when the document is generated
 with the printTeXFile option.
%%
\item[Parameters:]~
\begin{description}%
\item[\texttt{text\_string}:]
 Contents of this document.
\item[\texttt{id}:]
 An identifying string.
\item[\texttt{props}:]
 A structure with any optional properties.
\end{description}%
%
\item[Returns a structure object with the following fields:
]~

	text, id, props.
%
%
\item[See also:]%
\hyperlink{ref_doc_plot}{\texttt{doc\_plot}}%
\ (p.~\pageref{ref_doc_plot})%
\index[funcref]{doc_plot@\fidxl{doc\_plot}}%
, \hyperlink{ref_doc_multi}{\texttt{doc\_multi}}%
\ (p.~\pageref{ref_doc_multi})%
\index[funcref]{doc_multi@\fidxl{doc\_multi}}%
%
\item[Author:]%
Cengiz Gunay <cgunay@emory.edu>, 2006/01/17
%
\end{description}
\methodline%
\subsubsection[Method \texttt{display}]{Method \texttt{doc\_generate/display}}%
\index[funcref]{doc_generate@\fidxl{doc\_generate}!display@\fidxl{display}}%
\label{ref_doc_generate__display}%
\hypertarget{ref_doc_generate__display}{}%
\begin{description}
%
%
%
%
%
%
%
\item[Author:]%
Cengiz Gunay <cgunay@emory.edu>, 2004/08/04
%
\end{description}
\methodline%
\subsubsection[Method \texttt{get}]{Method \texttt{doc\_generate/get}}%
\index[funcref]{doc_generate@\fidxl{doc\_generate}!get@\fidxl{get}}%
\label{ref_doc_generate__get}%
\hypertarget{ref_doc_generate__get}{}%
\begin{description}
\item[Summary:]Defines generic attribute retrieval for objects.
%
%
%
%
%
%
%
\item[Author:]%
Cengiz Gunay <cgunay@emory.edu>, 2004/09/14
%
\end{description}
\methodline%
\subsubsection[Method \texttt{getTeXString}]{Method \texttt{doc\_generate/getTeXString}}%
\index[funcref]{doc_generate@\fidxl{doc\_generate}!getTeXString@\fidxl{getTeXString}}%
\label{ref_doc_generate__getTeXString}%
\hypertarget{ref_doc_generate__getTeXString}{}%
\begin{description}
\item[Summary:]Returns the TeX representation for the document.
%
\item[Usage:]~%
\begin{lyxcode}%
tex\_string = getTeXString(a\_doc, props)
%
\end{lyxcode}%
%
\item[Description:]%
This is an abstract placeholder for this method. It specifies what this 
 method should do in the subclasses that implement it. This method should
 create all the auxiliary files needed by the document. The generated tex\_string
 should be ready to be visualized.
%%
\item[Parameters:]~
\begin{description}%
\item[\texttt{a\_doc}:]
 A tests\_db object.
\item[\texttt{props}:]
 A structure with any optional properties.
\end{description}%
%
\item[Returns:
]~

	tex\_string: A string that contains TeX commands, which upon writing to a file,
	  can be interpreted by the TeX engine to produce a document.
%
\item[Example:]~
\begin{lyxcode}        doc\_plot has an overloaded getTeXString method:
\\%
        >> tex\_string = getTeXString(a\_doc\_plot)
\\%
        >> string2File(tex\_string, 'my\_doc.tex')
\\%
        then my\_doc.tex can be used by including from a valid LaTeX document.
\\%
\end{lyxcode}
%
\item[See also:]%
\hyperlink{ref_doc_generate}{\texttt{doc\_generate}}%
\ (p.~\pageref{ref_doc_generate})%
\index[funcref]{doc_generate@\fidxl{doc\_generate}}%
, \hyperlink{ref_doc_plot}{\texttt{doc\_plot}}%
\ (p.~\pageref{ref_doc_plot})%
\index[funcref]{doc_plot@\fidxl{doc\_plot}}%
%
\item[Author:]%
Cengiz Gunay <cgunay@emory.edu>, 2006/01/17
%
\end{description}
\methodline%
\subsubsection[Method \texttt{printTeXFile}]{Method \texttt{doc\_generate/printTeXFile}}%
\index[funcref]{doc_generate@\fidxl{doc\_generate}!printTeXFile@\fidxl{printTeXFile}}%
\label{ref_doc_generate__printTeXFile}%
\hypertarget{ref_doc_generate__printTeXFile}{}%
\begin{description}
\item[Summary:]Creates a TeX file with the contents of this document.
%
\item[Usage:]~%
\begin{lyxcode}%
printTeXFile(a\_doc, filename, props)
%
\end{lyxcode}%
%
\item[Description:]%
Calls getTeXString to generate the contents. The filename is adjusted with 
 a call to properFilename to generate an acceptable TeX filename. TeX-specific
 should only be added at this point or at getTeXString, because before we want
 the object to be a generic document container.
%%
\item[Parameters:]~
\begin{description}%
\item[\texttt{a\_doc}:]
 A tests\_db object.
\item[\texttt{filename}:]
 To write the TeX string.
\item[\texttt{props}:]
 A structure with any optional properties.
\begin{description}%
\item[\texttt{docDir}:]
 Directory in which to create TeX file.

(passed to getTeXString)
\end{description}%
\end{description}%
%
\item[Returns:
]~

	tex\_string: A string that contains TeX commands, which upon writing to a file,
	  can be interpreted by the TeX engine to produce a document.
%
\item[Example:]~
\begin{lyxcode}        >> a\_doc = doc\_plot(a\_plot, 'Results from cell.', 'Results.', struct, ''); 
\\%
        >> printTeXFile(a\_doc, 'my\_doc.tex')
\\%
        then my\_doc.tex can be used by including from a valid LaTeX document.
\\%
\end{lyxcode}
%
\item[See also:]%
\hyperlink{ref_doc_generate}{\texttt{doc\_generate}}%
\ (p.~\pageref{ref_doc_generate})%
\index[funcref]{doc_generate@\fidxl{doc\_generate}}%
, \hyperlink{ref_doc_plot}{\texttt{doc\_plot}}%
\ (p.~\pageref{ref_doc_plot})%
\index[funcref]{doc_plot@\fidxl{doc\_plot}}%
, \hyperlink{ref_string2File}{\texttt{string2File}}%
\ (p.~\pageref{ref_string2File})%
\index[funcref]{string2File@\fidxl{string2File}}%
, \hyperlink{ref_properFilename}{\texttt{properFilename}}%
\ (p.~\pageref{ref_properFilename})%
\index[funcref]{properFilename@\fidxl{properFilename}}%
%
\item[Author:]%
Cengiz Gunay <cgunay@emory.edu>, 2006/01/17
%
\end{description}
\methodline%
\subsubsection[Method \texttt{set}]{Method \texttt{doc\_generate/set}}%
\index[funcref]{doc_generate@\fidxl{doc\_generate}!set@\fidxl{set}}%
\label{ref_doc_generate__set}%
\hypertarget{ref_doc_generate__set}{}%
\begin{description}
\item[Summary:]Generic method for setting object attributes.
%
%
%
%
%
%
%
\item[Author:]%
Cengiz Gunay <cgunay@emory.edu>, 2004/10/08
%
\end{description}
\methodline%
\subsubsection[Method \texttt{subsref}]{Method \texttt{doc\_generate/subsref}}%
\index[funcref]{doc_generate@\fidxl{doc\_generate}!subsref@\fidxl{subsref}}%
\label{ref_doc_generate__subsref}%
\hypertarget{ref_doc_generate__subsref}{}%
\begin{description}
\item[Summary:]Defines generic indexing for objects.
%
%
%
%
%
%
%
%
\end{description}
\methodline%
\subsection{Class \texttt{doc\_multi}}%
\index[funcref]{doc_multi@\fidxl{doc\_multi}|boldhyperpage}%
\label{ref_doc_multi}%
\hypertarget{ref_doc_multi}{}%
\subsubsection[Constructor \texttt{doc\_multi}]{Constructor \texttt{doc\_multi/doc\_multi}}%
\index[funcref]{doc_multi@\fidxl{doc\_multi}!doc_multi@\fidxl{doc\_multi}}%
\label{ref_doc_multi__doc_multi}%
\hypertarget{ref_doc_multi__doc_multi}{}%
\begin{description}
\item[Summary:]A document that is composed of multiple other doc\_generate objects.
%
\item[Usage:]~%
\begin{lyxcode}%
a\_doc = doc\_multi(docs, id, props)
%
\end{lyxcode}%
%
%
\item[Parameters:]~
\begin{description}%
\item[\texttt{docs}:]
 A vector of doc\_generate objects.
\item[\texttt{id}:]
 An identifying string.
\item[\texttt{props}:]
 A structure with any optional properties.
\end{description}%
%
\item[Returns a structure object with the following fields:
]~

	docs, doc\_generate.
%
\item[Example:]~
\begin{lyxcode} >> mydoc = doc\_multi([doc\_plot(a\_plot1), doc\_plot(a\_plot2)], 'Two plots')
\\%
 >> printTeXFile(mydoc, 'two\_plots.tex')
\\%
\end{lyxcode}
%
\item[See also:]%
\hyperlink{ref_doc_generate}{\texttt{doc\_generate}}%
\ (p.~\pageref{ref_doc_generate})%
\index[funcref]{doc_generate@\fidxl{doc\_generate}}%
, \hyperlink{ref_getTeXString}{\texttt{getTeXString}}%
\ (p.~\pageref{ref_getTeXString})%
\index[funcref]{getTeXString@\fidxl{getTeXString}}%
, \hyperlink{ref_doc_generate__printTeXFile}{\texttt{doc\_generate/printTeXFile}}%
\ (p.~\pageref{ref_doc_generate__printTeXFile})%
\index[funcref]{doc_generate@\fidxl{doc\_generate}!printTeXFile@\fidxl{printTeXFile}}%
%
\item[Author:]%
Cengiz Gunay <cgunay@emory.edu>, 2006/01/17
%
\end{description}
\methodline%
\subsubsection[Method \texttt{get}]{Method \texttt{doc\_multi/get}}%
\index[funcref]{doc_multi@\fidxl{doc\_multi}!get@\fidxl{get}}%
\label{ref_doc_multi__get}%
\hypertarget{ref_doc_multi__get}{}%
\begin{description}
\item[Summary:]Defines generic attribute retrieval for objects.
%
%
%
%
%
%
%
\item[Author:]%
Cengiz Gunay <cgunay@emory.edu>, 2004/09/14
%
\end{description}
\methodline%
\subsubsection[Method \texttt{getTeXString}]{Method \texttt{doc\_multi/getTeXString}}%
\index[funcref]{doc_multi@\fidxl{doc\_multi}!getTeXString@\fidxl{getTeXString}}%
\label{ref_doc_multi__getTeXString}%
\hypertarget{ref_doc_multi__getTeXString}{}%
\begin{description}
\item[Summary:]Returns the TeX representation for the document.
%
\item[Usage:]~%
\begin{lyxcode}%
tex\_string = getTeXString(a\_doc, props)
%
\end{lyxcode}%
%
\item[Description:]%
Concatenates TeX representations of doc\_generate, or subclass, objects it contains.
%%
\item[Parameters:]~
\begin{description}%
\item[\texttt{a\_doc}:]
 A tests\_db object.
\item[\texttt{props}:]
 A structure with any optional properties.
\end{description}%
%
\item[Returns:
]~

	tex\_string: A string that contains TeX commands, which upon writing to a file,
	  can be interpreted by the TeX engine to produce a document.
%
\item[Example:]~
\begin{lyxcode}        doc\_plot has an overloaded getTeXString method:
\\%
        >> tex\_string = getTeXString(a\_doc\_plot)
\\%
        >> string2File(tex\_string, 'my\_doc.tex')
\\%
        then my\_doc.tex can be used by including from a valid LaTeX document.
\\%
\end{lyxcode}
%
\item[See also:]%
\hyperlink{ref_doc_generate}{\texttt{doc\_generate}}%
\ (p.~\pageref{ref_doc_generate})%
\index[funcref]{doc_generate@\fidxl{doc\_generate}}%
, \hyperlink{ref_doc_plot}{\texttt{doc\_plot}}%
\ (p.~\pageref{ref_doc_plot})%
\index[funcref]{doc_plot@\fidxl{doc\_plot}}%
%
\item[Author:]%
Cengiz Gunay <cgunay@emory.edu>, 2006/01/17
%
\end{description}
\methodline%
\subsubsection[Method \texttt{set}]{Method \texttt{doc\_multi/set}}%
\index[funcref]{doc_multi@\fidxl{doc\_multi}!set@\fidxl{set}}%
\label{ref_doc_multi__set}%
\hypertarget{ref_doc_multi__set}{}%
\begin{description}
\item[Summary:]Generic method for setting object attributes.
%
%
%
%
%
%
%
\item[Author:]%
Cengiz Gunay <cgunay@emory.edu>, 2004/10/08
%
\end{description}
\methodline%
\subsection{Class \texttt{doc\_plot}}%
\index[funcref]{doc_plot@\fidxl{doc\_plot}|boldhyperpage}%
\label{ref_doc_plot}%
\hypertarget{ref_doc_plot}{}%
\subsubsection[Constructor \texttt{doc\_plot}]{Constructor \texttt{doc\_plot/doc\_plot}}%
\index[funcref]{doc_plot@\fidxl{doc\_plot}!doc_plot@\fidxl{doc\_plot}}%
\label{ref_doc_plot__doc_plot}%
\hypertarget{ref_doc_plot__doc_plot}{}%
\begin{description}
\item[Summary:]Generates a formatted plot for printing, annotated with captions.
%
\item[Usage:]~%
\begin{lyxcode}%
a\_doc = doc\_plot(a\_plot, caption, plot\_filename, float\_props, id, props)
%
\end{lyxcode}%
%
\item[Description:]%
The generated file may take an extension according to chosen format.
%%
\item[Parameters:]~
\begin{description}%
\item[\texttt{a\_plot}:]
 A plot\_abstract ready to be visualized.
\item[\texttt{caption}:]
 Long caption to appear under the figure.
\item[\texttt{plot\_filename}:]
  Filename to be generated from the plot.
\item[\texttt{float\_props}:]
 Formatting instructions passed to TeXfloat. 
\item[\texttt{id}:]
 An identifying string.
\item[\texttt{props}:]
 A structure with any optional properties.
\begin{description}%
\item[\texttt{orient}:]
 Passed to the orient command before printing to figure file.

(others passed to doc\_plot/getTeXString and TeXfloat)
\end{description}%
\end{description}%
%
\item[Returns a structure object with the following fields:
]~

	plot, caption, plot\_filename, float\_props, doc\_generate.
%
\item[Example:]~
\begin{lyxcode}   >> a\_doc = doc\_plot(plotData(my\_cip\_trace), 'My CIP trace. Very interesting.', ...
\\%
                       'trace1', struct, 'first doc');
\\%
   >> printTeXFile(a\_doc, 'my\_doc.tex'); % it will pop-up the figure now
\\%
\end{lyxcode}
%
\item[See also:]%
\hyperlink{ref_doc_generate}{\texttt{doc\_generate}}%
\ (p.~\pageref{ref_doc_generate})%
\index[funcref]{doc_generate@\fidxl{doc\_generate}}%
, \hyperlink{ref_TeXfloat}{\texttt{TeXfloat}}%
\ (p.~\pageref{ref_TeXfloat})%
\index[funcref]{TeXfloat@\fidxl{TeXfloat}}%
%
\item[Author:]%
Cengiz Gunay <cgunay@emory.edu>, 2006/01/17
%
\end{description}
\methodline%
\subsubsection[Method \texttt{get}]{Method \texttt{doc\_plot/get}}%
\index[funcref]{doc_plot@\fidxl{doc\_plot}!get@\fidxl{get}}%
\label{ref_doc_plot__get}%
\hypertarget{ref_doc_plot__get}{}%
\begin{description}
\item[Summary:]Defines generic attribute retrieval for objects.
%
%
%
%
%
%
%
\item[Author:]%
Cengiz Gunay <cgunay@emory.edu>, 2004/09/14
%
\end{description}
\methodline%
\subsubsection[Method \texttt{getTeXString}]{Method \texttt{doc\_plot/getTeXString}}%
\index[funcref]{doc_plot@\fidxl{doc\_plot}!getTeXString@\fidxl{getTeXString}}%
\label{ref_doc_plot__getTeXString}%
\hypertarget{ref_doc_plot__getTeXString}{}%
\begin{description}
\item[Summary:]Returns the TeX representation for the plot document.
%
\item[Usage:]~%
\begin{lyxcode}%
tex\_string = getTeXString(a\_doc, props)
%
\end{lyxcode}%
%
\item[Description:]%
Plots, prints EPS files and generates the necessary LaTeX code.
%%
\item[Parameters:]~
\begin{description}%
\item[\texttt{a\_doc}:]
 A doc\_plot object.
\item[\texttt{props}:]
 A structure with any optional properties.
\begin{description}%
\item[\texttt{docDir}:]
 Base directory for files.
\end{description}%
\item[\texttt{plotRelDir}:]
 Subdirectory for plot files. $\backslash$input commands will

include this directory.
(passed to TeXfloat)
\end{description}%
%
\item[Returns:
]~

	tex\_string: A string that contains TeX commands, which upon writing to a file,
	  can be interpreted by the TeX engine to produce a document.
%
\item[Example:]~
\begin{lyxcode}        doc\_plot has an overloaded getTeXString method:
\\%
        >> tex\_string = getTeXString(a\_doc\_plot)
\\%
        >> string2File(tex\_string, 'my\_doc.tex')
\\%
        then my\_doc.tex can be used by including from a valid LaTeX document.
\\%
\end{lyxcode}
%
\item[See also:]%
\hyperlink{ref_doc_generate}{\texttt{doc\_generate}}%
\ (p.~\pageref{ref_doc_generate})%
\index[funcref]{doc_generate@\fidxl{doc\_generate}}%
, \hyperlink{ref_doc_plot}{\texttt{doc\_plot}}%
\ (p.~\pageref{ref_doc_plot})%
\index[funcref]{doc_plot@\fidxl{doc\_plot}}%
%
\item[Author:]%
Cengiz Gunay <cgunay@emory.edu>, 2006/01/17
%
\end{description}
\methodline%
\subsubsection[Method \texttt{plot}]{Method \texttt{doc\_plot/plot}}%
\index[funcref]{doc_plot@\fidxl{doc\_plot}!plot@\fidxl{plot}}%
\label{ref_doc_plot__plot}%
\hypertarget{ref_doc_plot__plot}{}%
\begin{description}
\item[Summary:]Default plot method to preview the contained plot in a new figure.
%
\item[Usage:]~%
\begin{lyxcode}%
figure\_handle = plot(a\_doc, props)
%
\end{lyxcode}%
%
\item[Description:]%
Only generate the contained plot for previewing. Opens a new figure.
%%
\item[Parameters:]~
\begin{description}%
\item[\texttt{a\_doc}:]
 A doc\_plot object.
\item[\texttt{props}:]
 A structure with any optional properties.
\end{description}%
%
\item[Returns:
]~

	figure\_handle: Handle of newly opened figure.
%
\item[Example:]~
\begin{lyxcode}        >> figure\_handle = plot(a\_doc\_plot)
\\%
\end{lyxcode}
%
\item[See also:]%
\hyperlink{ref_plot_abstract__plotFigure}{\texttt{plot\_abstract/plotFigure}}%
\ (p.~\pageref{ref_plot_abstract__plotFigure})%
\index[funcref]{plot_abstract@\fidxl{plot\_abstract}!plotFigure@\fidxl{plotFigure}}%
, \hyperlink{ref_doc_generate}{\texttt{doc\_generate}}%
\ (p.~\pageref{ref_doc_generate})%
\index[funcref]{doc_generate@\fidxl{doc\_generate}}%
, \hyperlink{ref_doc_plot}{\texttt{doc\_plot}}%
\ (p.~\pageref{ref_doc_plot})%
\index[funcref]{doc_plot@\fidxl{doc\_plot}}%
%
\item[Author:]%
Cengiz Gunay <cgunay@emory.edu>, 2006/01/17
%
\end{description}
\methodline%
\subsubsection[Method \texttt{plot\_abstract}]{Method \texttt{doc\_plot/plot\_abstract}}%
\index[funcref]{doc_plot@\fidxl{doc\_plot}!plot_abstract@\fidxl{plot\_abstract}}%
\label{ref_doc_plot__plot_abstract}%
\hypertarget{ref_doc_plot__plot_abstract}{}%
\begin{description}
\item[Summary:]Returns the plot\_abstract object within the doc\_plot.
%
\item[Usage:]~%
\begin{lyxcode}%
a\_plot = plot\_abstract(a\_doc, title\_str, props)
%
\end{lyxcode}%
%
\item[Description:]%
If a\_doc is a vector, returns a vector of plot\_abstract objects.
%%
\item[Parameters:]~
\begin{description}%
\item[\texttt{a\_doc}:]
 A doc\_plot object.
\item[\texttt{title\_str}:]
 (Optional) String to replace plot title.
\item[\texttt{props}:]
 A structure with any optional properties.

(rest passed to plot\_abstract.)
\end{description}%
%
\item[Returns:
]~

	a\_plot: A plot\_abstract object or vector that can be visualized.
%
%
\item[See also:]%
\hyperlink{ref_doc_plot__plot}{\texttt{doc\_plot/plot}}%
\ (p.~\pageref{ref_doc_plot__plot})%
\index[funcref]{doc_plot@\fidxl{doc\_plot}!plot@\fidxl{plot}}%
, \hyperlink{ref_plot_abstract}{\texttt{plot\_abstract}}%
\ (p.~\pageref{ref_plot_abstract})%
\index[funcref]{plot_abstract@\fidxl{plot\_abstract}}%
%
\item[Author:]%
Cengiz Gunay <cgunay@emory.edu>, 2004/11/17
%
\end{description}
\methodline%
\subsubsection[Method \texttt{set}]{Method \texttt{doc\_plot/set}}%
\index[funcref]{doc_plot@\fidxl{doc\_plot}!set@\fidxl{set}}%
\label{ref_doc_plot__set}%
\hypertarget{ref_doc_plot__set}{}%
\begin{description}
\item[Summary:]Generic method for setting object attributes.
%
%
%
%
%
%
%
\item[Author:]%
Cengiz Gunay <cgunay@emory.edu>, 2004/10/08
%
\end{description}
\methodline%
\subsection{Class \texttt{histogram\_db}}%
\index[funcref]{histogram_db@\fidxl{histogram\_db}|boldhyperpage}%
\label{ref_histogram_db}%
\hypertarget{ref_histogram_db}{}%
\subsubsection[Constructor \texttt{histogram\_db}]{Constructor \texttt{histogram\_db/histogram\_db}}%
\index[funcref]{histogram_db@\fidxl{histogram\_db}!histogram_db@\fidxl{histogram\_db}}%
\label{ref_histogram_db__histogram_db}%
\hypertarget{ref_histogram_db__histogram_db}{}%
\begin{description}
\item[Summary:]A database of histogram values generated for a column of another database.
%
\item[Usage:]~%
\begin{lyxcode}%
a\_hist\_db = histogram\_db(col\_name, bins, hist\_results, id, props)
%
\end{lyxcode}%
%
\item[Description:]%
This is a subclass of tests\_db. Allows generating a histogram plot,
 etc. The histogram count is entered as a column named histVal.
%%
\item[Parameters:]~
\begin{description}%
\item[\texttt{col\_name}:]
 The column name of the histogrammed value.
\item[\texttt{bins}:]
 The values for which the histogram values are calculated.
\item[\texttt{hist\_results}:]
 A column vector of histogram values.
\item[\texttt{id}:]
 An identifying string.
\item[\texttt{props}:]
 A structure with any optional properties.
\end{description}%
%
\item[Returns a structure object with the following fields:
]~

	tests\_db, props.
%
\item[Example:]~
\begin{lyxcode} >> [hist\_results, bins] = hist(my\_data);
\\%
 >> a\_hist\_db = histogram\_db('firing\_rate', bins, hist\_results, 'rate histogram db');
\\%
 >> plot(a\_hist\_db);
\\%
\end{lyxcode}
%
\item[See also:]%
\hyperlink{ref_tests_db}{\texttt{tests\_db}}%
\ (p.~\pageref{ref_tests_db})%
\index[funcref]{tests_db@\fidxl{tests\_db}}%
, \hyperlink{ref_plot_simple}{\texttt{plot\_simple}}%
\ (p.~\pageref{ref_plot_simple})%
\index[funcref]{plot_simple@\fidxl{plot\_simple}}%
, \hyperlink{ref_tests_db__histogram}{\texttt{tests\_db/histogram}}%
\ (p.~\pageref{ref_tests_db__histogram})%
\index[funcref]{tests_db@\fidxl{tests\_db}!histogram@\fidxl{histogram}}%
%
\item[Author:]%
Cengiz Gunay <cgunay@emory.edu>, 2004/09/20
%
\end{description}
\methodline%
\subsubsection[Method \texttt{calcKLhist}]{Method \texttt{histogram\_db/calcKLhist}}%
\index[funcref]{histogram_db@\fidxl{histogram\_db}!calcKLhist@\fidxl{calcKLhist}}%
\label{ref_histogram_db__calcKLhist}%
\hypertarget{ref_histogram_db__calcKLhist}{}%
\begin{description}
\item[Summary:]Calculates the Kullback-Leibler divergence between two histograms with same bins.
%
\item[Usage:]~%
\begin{lyxcode}%
kl\_bits = calcKLhist(two\_hist\_dbs, props)
%
\end{lyxcode}%
%
\item[Description:]%
Histograms must have same bins! See example how to use
 tests\_db/histogram to get same bins from multiple DBs.
%%
\item[Parameters:]~
\begin{description}%
\item[\texttt{two\_hist\_dbs}:]
 Array of two histogram\_db objects to calculate KL divergence.
\item[\texttt{props}:]
 Structure with optional parameters.
\begin{description}%
\item[\texttt{sym}:]
 calculate symmetric KL divergence. Can be 'sum' for sum of

divergences, 'avg' for  average, and 'res' for resistor average.
\end{description}%
\end{description}%
%
\item[Returns:
]~

   kl\_bits: The calculated non-negative divergence in bits.
%
\item[Example:]~
\begin{lyxcode} % Find 100-bin histograms of column var1 from two DBs and calculate
\\%
 % their KL divergence
\\%
 >> kl\_bits = calcKLhist(histogram([one\_db, another\_db], 'var1', 100))
\\%
\end{lyxcode}
%
\item[See also:]%
\hyperlink{ref_histogram_db}{\texttt{histogram\_db}}%
\ (p.~\pageref{ref_histogram_db})%
\index[funcref]{histogram_db@\fidxl{histogram\_db}}%
, \hyperlink{ref_calcKLmodel}{\texttt{calcKLmodel}}%
\ (p.~\pageref{ref_calcKLmodel})%
\index[funcref]{calcKLmodel@\fidxl{calcKLmodel}}%
, \hyperlink{ref_tests_db__histogram}{\texttt{tests\_db/histogram}}%
\ (p.~\pageref{ref_tests_db__histogram})%
\index[funcref]{tests_db@\fidxl{tests\_db}!histogram@\fidxl{histogram}}%
%
\item[Author:]%
Cengiz Gunay <cgunay@emory.edu>, 2009/04/03
%
\end{description}
\methodline%
\subsubsection[Method \texttt{calcKLmodel}]{Method \texttt{histogram\_db/calcKLmodel}}%
\index[funcref]{histogram_db@\fidxl{histogram\_db}!calcKLmodel@\fidxl{calcKLmodel}}%
\label{ref_histogram_db__calcKLmodel}%
\hypertarget{ref_histogram_db__calcKLmodel}{}%
\begin{description}
\item[Summary:]Calculates the Kullback-Leibler divergence of the histogram to a model distribution.
%
\item[Usage:]~%
\begin{lyxcode}%
[mode\_val, mode\_mag] = calcKLmodel(a\_hist\_db)
%
\end{lyxcode}%
%
%
\item[Parameters:]~
\begin{description}%
\item[\texttt{a\_hist\_db}:]
 A histogram\_db object.
\item[\texttt{dist\_model}:]
 Structure that contains the distribution parameters. Must
\begin{description}%
\item[\texttt{be one of these}:]

\item[\texttt{Normal distribution}:]
 struct('dist', 'norm', 'mu', mean, 'sigma', var)
\item[\texttt{Poisson distribution}:]
 struct('dist', 'pois', 'lambda', l)
\item[\texttt{Exponential distribution}:]
 struct('dist', 'exp', 'mu', m)
\item[\texttt{Uniform distribution}:]
 struct('dist', 'uni')
\end{description}%
\item[\texttt{props}:]
 Structure with optional parameters.
\begin{description}%
\item[\texttt{plot}:]
 If 1, return a plot\_abstract object with both distributions.
\end{description}%
\end{description}%
%
\item[Returns:
]~

   kl\_bits: The calculated divergence in bits.
%
%
\item[See also:]%
\hyperlink{ref_histogram_db}{\texttt{histogram\_db}}%
\ (p.~\pageref{ref_histogram_db})%
\index[funcref]{histogram_db@\fidxl{histogram\_db}}%
%
\item[Author:]%
Cengiz Gunay <cgunay@emory.edu>, 2009/03/24
%
\end{description}
\methodline%
\subsubsection[Method \texttt{calcMode}]{Method \texttt{histogram\_db/calcMode}}%
\index[funcref]{histogram_db@\fidxl{histogram\_db}!calcMode@\fidxl{calcMode}}%
\label{ref_histogram_db__calcMode}%
\hypertarget{ref_histogram_db__calcMode}{}%
\begin{description}
\item[Summary:]Finds the mode of the distribution, that is, the bin with the highest value.
%
\item[Usage:]~%
\begin{lyxcode}%
[mode\_val, mode\_mag] = calcMode(a\_hist\_db)
%
\end{lyxcode}%
%
%
\item[Parameters:]~
\begin{description}%
\item[\texttt{a\_hist\_db}:]
 A histogram\_db object.
\end{description}%
%
\item[Returns:
]~

	mode\_val: The center of the bin that has most members.
	mode\_mag: The value of the histogram bin.
%
%
\item[See also:]%
\hyperlink{ref_histogram_db}{\texttt{histogram\_db}}%
\ (p.~\pageref{ref_histogram_db})%
\index[funcref]{histogram_db@\fidxl{histogram\_db}}%
%
\item[Author:]%
Cengiz Gunay <cgunay@emory.edu>, 2005/04/27
%
\end{description}
\methodline%
\subsubsection[Method \texttt{get}]{Method \texttt{histogram\_db/get}}%
\index[funcref]{histogram_db@\fidxl{histogram\_db}!get@\fidxl{get}}%
\label{ref_histogram_db__get}%
\hypertarget{ref_histogram_db__get}{}%
\begin{description}
\item[Summary:]Defines generic attribute retrieval for objects.
%
%
%
%
%
%
%
\item[Author:]%
Cengiz Gunay <cgunay@emory.edu>, 2004/09/14
%
\end{description}
\methodline%
\subsubsection[Method \texttt{plotEqSpaced}]{Method \texttt{histogram\_db/plotEqSpaced}}%
\index[funcref]{histogram_db@\fidxl{histogram\_db}!plotEqSpaced@\fidxl{plotEqSpaced}}%
\label{ref_histogram_db__plotEqSpaced}%
\hypertarget{ref_histogram_db__plotEqSpaced}{}%
\begin{description}
\item[Summary:]Generates a histogram plot where the values are equally spaced on the x-axis. For use with non-linear parameter values.
%
\item[Usage:]~%
\begin{lyxcode}%
a\_plot = plotEqSpaced(a\_hist\_db, title\_str, props)
%
\end{lyxcode}%
%
\item[Description:]%
Generates a plot\_simple object from this histogram.
%%
\item[Parameters:]~
\begin{description}%
\item[\texttt{a\_hist\_db}:]
 A histogram\_db object.
\item[\texttt{title\_str}:]
 Optional title string.
\item[\texttt{props}:]
 Optional properties passed to plot\_abstract.
\begin{description}%
\item[\texttt{quiet}:]
 If 1, don't include database name on title.
\item[\texttt{skipXnum}:]
 Skip every this many values to fit labels on X-axis (default=1).
\item[\texttt{totalXnum}:]
 Skip to fit this number of total X-axis tick labels.
\end{description}%
\end{description}%
%
\item[Returns:
]~

   a\_plot: A object of plot\_abstract or one of its subclasses.
%
%
\item[See also:]%
\hyperlink{ref_plot_abstract}{\texttt{plot\_abstract}}%
\ (p.~\pageref{ref_plot_abstract})%
\index[funcref]{plot_abstract@\fidxl{plot\_abstract}}%
, \hyperlink{ref_plot_simple}{\texttt{plot\_simple}}%
\ (p.~\pageref{ref_plot_simple})%
\index[funcref]{plot_simple@\fidxl{plot\_simple}}%
%
\item[Author:]%
Cengiz Gunay <cgunay@emory.edu>, 2004/09/22
%
\end{description}
\methodline%
\subsubsection[Method \texttt{plotPages}]{Method \texttt{histogram\_db/plotPages}}%
\index[funcref]{histogram_db@\fidxl{histogram\_db}!plotPages@\fidxl{plotPages}}%
\label{ref_histogram_db__plotPages}%
\hypertarget{ref_histogram_db__plotPages}{}%
\begin{description}
\item[Summary:]Generates a plot containing subplots from each page of histograms.
%
\item[Usage:]~%
\begin{lyxcode}%
a\_plot = plotPages(a\_hist\_db, title\_str, props)
%
\end{lyxcode}%
%
\item[Description:]%
For each page of the histogram, a histogram is placed in a subplot.
%%
\item[Parameters:]~
\begin{description}%
\item[\texttt{a\_hist\_db}:]
 A histogram\_db object.
\item[\texttt{title\_str}:]
 (Optional) String to append to plot title.
\item[\texttt{props}:]
 A structure with any optional properties, passed to plot\_stack.
\begin{description}%
\item[\texttt{an\_orient}:]
 Stack orientation. One of 'x', or 'y' (Default='y'; vertical).
\item[\texttt{quiet}:]
 If 1, only display given title\_str.
\end{description}%
\end{description}%
%
\item[Returns:
]~

	a\_plot: A object of plot\_abstract or one of its subclasses.
%
%
\item[See also:]%
\hyperlink{ref_plotPages}{\texttt{plotPages}}%
\ (p.~\pageref{ref_plotPages})%
\index[funcref]{plotPages@\fidxl{plotPages}}%
, \hyperlink{ref_plot_simple}{\texttt{plot\_simple}}%
\ (p.~\pageref{ref_plot_simple})%
\index[funcref]{plot_simple@\fidxl{plot\_simple}}%
%
\item[Author:]%
Cengiz Gunay <cgunay@emory.edu>, 2004/10/04
%
\end{description}
\methodline%
\subsubsection[Method \texttt{plotRowMatrix}]{Method \texttt{histogram\_db/plotRowMatrix}}%
\index[funcref]{histogram_db@\fidxl{histogram\_db}!plotRowMatrix@\fidxl{plotRowMatrix}}%
\label{ref_histogram_db__plotRowMatrix}%
\hypertarget{ref_histogram_db__plotRowMatrix}{}%
\begin{description}
\item[Summary:]Generates a subplot matrix of measure columns versus rows of databases. 
%
\item[Usage:]~%
\begin{lyxcode}%
a\_plot = plotRowMatrix(hist\_dbs, title\_str, props)
%
\end{lyxcode}%
%
\item[Description:]%
Each row in the hist\_dbs is assumed to come from a different DB. Columns represent histograms 
 of different measurements. The plot is made such that histograms in each row have the same
 maximal count, and histograms in each column have the same axis limits.
%%
\item[Parameters:]~
\begin{description}%
\item[\texttt{hist\_dbs}:]
 A matrix of histogram\_db objects.
\item[\texttt{title\_str}:]
 Title to go at the top.
\item[\texttt{props}:]
 A structure with any optional properties.
\begin{description}%
\item[\texttt{rowLabels}:]
 Cell array of y-axis labels for each row.

(rest passed to histogram\_db/plot\_abstract)
\end{description}%
\end{description}%
%
\item[Returns:
]~

	a\_plot: A object of plot\_abstract or one of its subclasses.
%
%
\item[See also:]%
\hyperlink{ref_plot_abstract}{\texttt{plot\_abstract}}%
\ (p.~\pageref{ref_plot_abstract})%
\index[funcref]{plot_abstract@\fidxl{plot\_abstract}}%
%
\item[Author:]%
Cengiz Gunay <cgunay@emory.edu>, 2006/11/22
%
\end{description}
\methodline%
\subsubsection[Method \texttt{plot\_abstract}]{Method \texttt{histogram\_db/plot\_abstract}}%
\index[funcref]{histogram_db@\fidxl{histogram\_db}!plot_abstract@\fidxl{plot\_abstract}}%
\label{ref_histogram_db__plot_abstract}%
\hypertarget{ref_histogram_db__plot_abstract}{}%
\begin{description}
\item[Summary:]Creates a bar plot from the histogram.
%
\item[Usage:]~%
\begin{lyxcode}%
a\_plot = plot\_abstract(a\_hist\_db, title\_str, props)
%
\end{lyxcode}%
%
\item[Description:]%
Generates a plot\_simple object from this histogram.
%%
\item[Parameters:]~
\begin{description}%
\item[\texttt{a\_hist\_db}:]
 A histogram\_db object.
\item[\texttt{title\_str}:]
 Optional title string.
\item[\texttt{props}:]
 Optional properties passed to plot\_abstract.
\begin{description}%
\item[\texttt{command}:]
 Plot command (Optional, default='bar')
\item[\texttt{endZeros}:]
 Prefix and suffix bins with zero values to make a

smooth plot.
\item[\texttt{lineSpec}:]
 Line specification passed to bar command.
\item[\texttt{logScale}:]
 If 1, use logarithmic y-scale.
\item[\texttt{shading}:]
 'faceted' (default) or 'flat'.
\item[\texttt{barWidth}:]
 Controls spacing between bars (see width argument for the

bar command; default=0.8).
\item[\texttt{quiet}:]
 If 1, don't include database name on title.
\end{description}%
\end{description}%
%
\item[Returns:
]~

	a\_plot: A object of plot\_abstract or one of its subclasses.
%
%
\item[See also:]%
\hyperlink{ref_plot_abstract}{\texttt{plot\_abstract}}%
\ (p.~\pageref{ref_plot_abstract})%
\index[funcref]{plot_abstract@\fidxl{plot\_abstract}}%
, \hyperlink{ref_plot_simple}{\texttt{plot\_simple}}%
\ (p.~\pageref{ref_plot_simple})%
\index[funcref]{plot_simple@\fidxl{plot\_simple}}%
%
\item[Author:]%
Cengiz Gunay <cgunay@emory.edu>, 2004/09/22
%
\end{description}
\methodline%
\subsubsection[Method \texttt{subsref}]{Method \texttt{histogram\_db/subsref}}%
\index[funcref]{histogram_db@\fidxl{histogram\_db}!subsref@\fidxl{subsref}}%
\label{ref_histogram_db__subsref}%
\hypertarget{ref_histogram_db__subsref}{}%
\begin{description}
\item[Summary:]Defines generic indexing for objects.
%
%
%
%
%
%
%
%
\end{description}
\methodline%
\subsection{Class \texttt{mesh\_amira}}%
\index[funcref]{mesh_amira@\fidxl{mesh\_amira}|boldhyperpage}%
\label{ref_mesh_amira}%
\hypertarget{ref_mesh_amira}{}%
\subsubsection[Constructor \texttt{mesh\_amira}]{Constructor \texttt{mesh\_amira/mesh\_amira}}%
\index[funcref]{mesh_amira@\fidxl{mesh\_amira}!mesh_amira@\fidxl{mesh\_amira}}%
\label{ref_mesh_amira__mesh_amira}%
\hypertarget{ref_mesh_amira__mesh_amira}{}%
\begin{description}
\item[Summary:]Contains 3D mesh data imported from the Amira software.
%
\item[Usage:]~%
\begin{lyxcode}%
a\_mesh = mesh\_amira(filename, id, props)
%
\end{lyxcode}%
%
\item[Description:]%
By default it reads an ASCII formatted v2 type Amira file and expects
 only vertex and edge information for representing 3D reconstructions
 of neurons.
%%
\item[Parameters:]~
\begin{description}%
\item[\texttt{filename}:]
 Amira mesh file (*.am).
\item[\texttt{id}:]
 Identification string.
\item[\texttt{props}:]
 A structure with any optional properties.
\begin{description}%
\item[\texttt{isVerbose}:]
 If 1, produce verbose info while loading Amira file (for

debugging).
\end{description}%
\end{description}%
%
\item[Returns a structure object with the following fields:
]~

   nVertices, nEdges, nOrigins: Number of vertices, edges and origins, resp.,
   vertices: 3D coordinates of each vertex,
   neighborCount: Number of neighbors for each vertex,
   raddi: Radius of each vertex,
   neighborList: Index of vertices that are neighboring to each vertex,
   Origins: Indices of vertices marked as 'origin',
   origFilename, id, props
%
%
\item[See also:]%
\hyperlink{ref_private__loadAmiraMesh}{\texttt{private/loadAmiraMesh}}%
\ (p.~\pageref{ref_private__loadAmiraMesh})%
\index[funcref]{private@\fidxl{private}!loadAmiraMesh@\fidxl{loadAmiraMesh}}%
%
\item[Author:]%
Cengiz Gunay <cgunay@emory.edu>, 2012/02/03
%
\end{description}
\methodline%
\subsubsection[Method \texttt{exportMorphML}]{Method \texttt{mesh\_amira/exportMorphML}}%
\index[funcref]{mesh_amira@\fidxl{mesh\_amira}!exportMorphML@\fidxl{exportMorphML}}%
\label{ref_mesh_amira__exportMorphML}%
\hypertarget{ref_mesh_amira__exportMorphML}{}%
\begin{description}
\item[Summary:]Export mesh to MorphML XML format.
%
\item[Usage:]~%
\begin{lyxcode}%
a\_dom = exportMorphML(a\_mesh, props)
%
\end{lyxcode}%
%
%
\item[Parameters:]~
\begin{description}%
\item[\texttt{a\_mesh}:]
 A mesh\_amira object.
\item[\texttt{props}:]
 A structure with any optional properties.
\end{description}%
%
\item[Returns:
]~

   a\_dom: A Matlab DomNode object.
%
\item[Example:]~
\begin{lyxcode} >> a\_mesh = mesh\_amira('my\_amira.am', 'Neuron 1')
\\%
 >> xmlwrite('neuron.xml', exportMorphML(a\_mesh))
\\%
\end{lyxcode}
%
\item[See also:]%
%
\item[Author:]%
Cengiz Gunay <cgunay@emory.edu>, 2012/02/03
%
\end{description}
\methodline%
\subsubsection[Method \texttt{plot\_abstract}]{Method \texttt{mesh\_amira/plot\_abstract}}%
\index[funcref]{mesh_amira@\fidxl{mesh\_amira}!plot_abstract@\fidxl{plot\_abstract}}%
\label{ref_mesh_amira__plot_abstract}%
\hypertarget{ref_mesh_amira__plot_abstract}{}%
\begin{description}
\item[Summary:]Prepare a plot the of the 3D mesh.
%
\item[Usage:]~%
\begin{lyxcode}%
a\_p = plot\_abstract(a\_mesh, title\_str, props)
%
\end{lyxcode}%
%
\item[Description:]%
Can be stacked or superposed with other plot objects.
%%
\item[Parameters:]~
\begin{description}%
\item[\texttt{a\_mesh}:]
 A voltage clamp object.
\item[\texttt{title\_str}:]
 (Optional) Text to appear in the plot title.
\item[\texttt{props}:]
 A structure with any optional properties.
\begin{description}%
\item[\texttt{quiet}:]
 If 1, only use given title\_str.
\end{description}%
\end{description}%
%
\item[Returns:
]~

   a\_p: A plot\_abstract object.
%
\item[Example:]~
\begin{lyxcode} >> a\_mesh = mesh\_amira('my\_amira.am', 'Neuron 1')
\\%
 >> plotFigure(plot\_abstract(a\_mesh, ' - side view'))
\\%
\end{lyxcode}
%
\item[See also:]%
\hyperlink{ref_plotFigure}{\texttt{plotFigure}}%
\ (p.~\pageref{ref_plotFigure})%
\index[funcref]{plotFigure@\fidxl{plotFigure}}%
, \hyperlink{ref_plot_superpose}{\texttt{plot\_superpose}}%
\ (p.~\pageref{ref_plot_superpose})%
\index[funcref]{plot_superpose@\fidxl{plot\_superpose}}%
, \hyperlink{ref_plot_stack}{\texttt{plot\_stack}}%
\ (p.~\pageref{ref_plot_stack})%
\index[funcref]{plot_stack@\fidxl{plot\_stack}}%
%
\item[Author:]%
Cengiz Gunay <cgunay@emory.edu>, 2012/02/03
%
\end{description}
\methodline%
\subsection{Class \texttt{model\_ct\_bundle}}%
\index[funcref]{model_ct_bundle@\fidxl{model\_ct\_bundle}|boldhyperpage}%
\label{ref_model_ct_bundle}%
\hypertarget{ref_model_ct_bundle}{}%
\subsubsection[Constructor \texttt{model\_ct\_bundle}]{Constructor \texttt{model\_ct\_bundle/model\_ct\_bundle}}%
\index[funcref]{model_ct_bundle@\fidxl{model\_ct\_bundle}!model_ct_bundle@\fidxl{model\_ct\_bundle}}%
\label{ref_model_ct_bundle__model_ct_bundle}%
\hypertarget{ref_model_ct_bundle__model_ct_bundle}{}%
\begin{description}
\item[Summary:]The model cip\_trace dataset and the DB created from it bundled together.
%
\item[Usage:]~%
\begin{lyxcode}%
a\_bundle = model\_ct\_bundle(a\_dataset, a\_db, a\_joined\_db, props)
%
\end{lyxcode}%
%
\item[Description:]%
This is a subclass of dataset\_db\_bundle, specialized for model datasets. 
%%
\item[Parameters:]~
\begin{description}%
\item[\texttt{a\_dataset}:]
 A params\_cip\_trace\_fileset object.
\item[\texttt{a\_db}:]
 The raw params\_tests\_db object created from the dataset. It only needs

to have the pAcip, trial, and ItemIndex columns.
\item[\texttt{a\_joined\_db}:]
 The one-model-per-line DB created from the raw DB.
\item[\texttt{props}:]
 A structure with any optional properties.
\end{description}%
%
\item[Returns a structure object with the following fields:
]~

	dataset\_db\_bundle.
%
\item[Example:]~
\begin{lyxcode} >> a\_joined\_db = mergeMultipleCIPsInOne(a\_db, ...)
\\%
 >> a\_bundle = model\_ct\_bundle(a\_params\_cip\_trace\_fileset, a\_db, a\_joined\_db)
\\%
\end{lyxcode}
%
\item[See also:]%
\hyperlink{ref_dataset_db_bundle}{\texttt{dataset\_db\_bundle}}%
\ (p.~\pageref{ref_dataset_db_bundle})%
\index[funcref]{dataset_db_bundle@\fidxl{dataset\_db\_bundle}}%
, \hyperlink{ref_tests_db}{\texttt{tests\_db}}%
\ (p.~\pageref{ref_tests_db})%
\index[funcref]{tests_db@\fidxl{tests\_db}}%
, \hyperlink{ref_params_tests_dataset}{\texttt{params\_tests\_dataset}}%
\ (p.~\pageref{ref_params_tests_dataset})%
\index[funcref]{params_tests_dataset@\fidxl{params\_tests\_dataset}}%
, \hyperlink{ref_params_tests_db__mergeMultipleCIPsInOne}{\texttt{params\_tests\_db/mergeMultipleCIPsInOne}}%
\ (p.~\pageref{ref_params_tests_db__mergeMultipleCIPsInOne})%
\index[funcref]{params_tests_db@\fidxl{params\_tests\_db}!mergeMultipleCIPsInOne@\fidxl{mergeMultipleCIPsInOne}}%
%
\item[Author:]%
Cengiz Gunay <cgunay@emory.edu>, 2005/12/13
%
\end{description}
\methodline%
\subsubsection[Method \texttt{addToDB}]{Method \texttt{model\_ct\_bundle/addToDB}}%
\index[funcref]{model_ct_bundle@\fidxl{model\_ct\_bundle}!addToDB@\fidxl{addToDB}}%
\label{ref_model_ct_bundle__addToDB}%
\hypertarget{ref_model_ct_bundle__addToDB}{}%
\begin{description}
\item[Summary:]Concatenate to existing DB in the bundle.
%
\item[Usage:]~%
\begin{lyxcode}%
a\_mbundle = addToDB(a\_mbundle, a\_raw\_db, props)
%
\end{lyxcode}%
%
\item[Description:]%
If joinedDb is not given in props, calls treatSimDB to get the joined\_db from this raw DB. 
 Then concats to both db and joined\_db in bundle.
%%
\item[Parameters:]~
\begin{description}%
\item[\texttt{a\_mbundle}:]
 A model\_ct\_bundle object.
\item[\texttt{a\_crit\_bundle}:]
 A physiol\_bundle having a crit\_db as its joined\_db.
\item[\texttt{props}:]
 A structure with any optional properties.
\begin{description}%
\item[\texttt{joinedDb}:]
 The joined version of a\_raw\_db.
\item[\texttt{dataset}:]
 If given, this one is used to replace the fileset in the bundle.
\end{description}%
\end{description}%
%
\item[Returns:
]~

	a\_mbundle: a model\_ct\_bundle object containing the added DB.
%
\item[Example:]~
\begin{lyxcode} >> mbundle = addToDB(mbundle, params\_tests\_db(mfileset, [19684:59956]))
\\%
\end{lyxcode}
%
\item[See also:]%
\hyperlink{ref_params_tests_fileset__addFiles}{\texttt{params\_tests\_fileset/addFiles}}%
\ (p.~\pageref{ref_params_tests_fileset__addFiles})%
\index[funcref]{params_tests_fileset@\fidxl{params\_tests\_fileset}!addFiles@\fidxl{addFiles}}%
, \hyperlink{ref_multi_fileset_gpsim_cns2005__addFileDir}{\texttt{multi\_fileset\_gpsim\_cns2005/addFileDir}}%
\ (p.~\pageref{ref_multi_fileset_gpsim_cns2005__addFileDir})%
\index[funcref]{multi_fileset_gpsim_cns2005@\fidxl{multi\_fileset\_gpsim\_cns2005}!addFileDir@\fidxl{addFileDir}}%
%
\item[Author:]%
Cengiz Gunay <cgunay@emory.edu>, 2006/02/06
%
\end{description}
\methodline%
\subsubsection[Method \texttt{collectPhysiolMatches}]{Method \texttt{model\_ct\_bundle/collectPhysiolMatches}}%
\index[funcref]{model_ct_bundle@\fidxl{model\_ct\_bundle}!collectPhysiolMatches@\fidxl{collectPhysiolMatches}}%
\label{ref_model_ct_bundle__collectPhysiolMatches}%
\hypertarget{ref_model_ct_bundle__collectPhysiolMatches}{}%
\begin{description}
\item[Summary:]Compare model DB to given physiol criteria and return some top matches.
%
\item[Usage:]~%
\begin{lyxcode}%
row\_index = collectPhysiolMatches(a\_mbundle, a\_crit\_bundle, props)
%
\end{lyxcode}%
%
%
\item[Parameters:]~
\begin{description}%
\item[\texttt{a\_mbundle}:]
 A model\_ct\_bundle object.
\item[\texttt{a\_crit\_bundle}:]
 A physiol\_bundle object that holds the criterion neuron.
\item[\texttt{props}:]
 A structure with any optional properties.
\begin{description}%
\item[\texttt{showTopmost}:]
 Number of top matching models to return (default=50)
\end{description}%
\end{description}%
%
\item[Returns: 
]~

	row\_index: Row indices of best matching models.
%
%
\item[See also:]%
%
\item[Author:]%
Cengiz Gunay <cgunay@emory.edu>, 2006/01/18
%
\end{description}
\methodline%
\subsubsection[Method \texttt{ctFromRows}]{Method \texttt{model\_ct\_bundle/ctFromRows}}%
\index[funcref]{model_ct_bundle@\fidxl{model\_ct\_bundle}!ctFromRows@\fidxl{ctFromRows}}%
\label{ref_model_ct_bundle__ctFromRows}%
\hypertarget{ref_model_ct_bundle__ctFromRows}{}%
\begin{description}
\item[Summary:]Loads a cip\_trace object from a raw data file in the a\_mbundle.
%
\item[Usage:]~%
\begin{lyxcode}%
a\_cip\_trace = ctFromRows(a\_mbundle, a\_db|trials, cip\_levels, props)
%
\end{lyxcode}%
%
\item[Description:]%
This is an overloaded method.
%%
\item[Parameters:]~
\begin{description}%
\item[\texttt{a\_mbundle}:]
 A model\_ct\_bundle object.
\item[\texttt{a\_db}:]
 A DB created by the dataset in the a\_mbundle to read the trial numbers from.
\item[\texttt{trials}:]
 A column vector with trial numbers.
\item[\texttt{cip\_levels}:]
 A column vector of CIP-levels to be loaded.
\item[\texttt{props}:]
 A structure with any optional properties.

(passed to a\_mbundle.dataset/cip\_trace)
\end{description}%
%
\item[Returns:
]~

	a\_cip\_trace: One or more cip\_trace objects that hold the raw data.
%
%
\item[See also:]%
\hyperlink{ref_params_cip_trace_fileset__ctFromRows}{\texttt{params\_cip\_trace\_fileset/ctFromRows}}%
\ (p.~\pageref{ref_params_cip_trace_fileset__ctFromRows})%
\index[funcref]{params_cip_trace_fileset@\fidxl{params\_cip\_trace\_fileset}!ctFromRows@\fidxl{ctFromRows}}%
%
\item[Author:]%
Cengiz Gunay <cgunay@emory.edu>, 2005/07/13
%
\end{description}
\methodline%
\subsubsection[Method \texttt{get}]{Method \texttt{model\_ct\_bundle/get}}%
\index[funcref]{model_ct_bundle@\fidxl{model\_ct\_bundle}!get@\fidxl{get}}%
\label{ref_model_ct_bundle__get}%
\hypertarget{ref_model_ct_bundle__get}{}%
\begin{description}
\item[Summary:]Defines generic attribute retrieval for objects.
%
%
%
%
%
%
%
\item[Author:]%
Cengiz Gunay <cgunay@emory.edu>, 2004/09/14
%
\end{description}
\methodline%
\subsubsection[Method \texttt{getNeuronLabel}]{Method \texttt{model\_ct\_bundle/getNeuronLabel}}%
\index[funcref]{model_ct_bundle@\fidxl{model\_ct\_bundle}!getNeuronLabel@\fidxl{getNeuronLabel}}%
\label{ref_model_ct_bundle__getNeuronLabel}%
\hypertarget{ref_model_ct_bundle__getNeuronLabel}{}%
\begin{description}
\item[Summary:]Constructs the neuron label from bundle.
%
\item[Usage:]~%
\begin{lyxcode}%
a\_label = getNeuronLabel(a\_bundle, trial\_num, props)
%
\end{lyxcode}%
%
%
\item[Parameters:]~
\begin{description}%
\item[\texttt{a\_bundle}:]
 A physiol\_cip\_traceset\_fileset object.
\item[\texttt{trial\_num}:]
 The trial number of model neuron.
\item[\texttt{props}:]
 A structure with any optional properties.
\end{description}%
%
\item[Returns:
]~

	a\_label: A string label identifying selected neuron in bundle.
%
%
\item[See also:]%
\hyperlink{ref_dataset_db_bundle}{\texttt{dataset\_db\_bundle}}%
\ (p.~\pageref{ref_dataset_db_bundle})%
\index[funcref]{dataset_db_bundle@\fidxl{dataset\_db\_bundle}}%
%
\item[Author:]%
Cengiz Gunay <cgunay@emory.edu>, 2006/05/26
%
\end{description}
\methodline%
\subsubsection[Method \texttt{getNeuronRowIndex}]{Method \texttt{model\_ct\_bundle/getNeuronRowIndex}}%
\index[funcref]{model_ct_bundle@\fidxl{model\_ct\_bundle}!getNeuronRowIndex@\fidxl{getNeuronRowIndex}}%
\label{ref_model_ct_bundle__getNeuronRowIndex}%
\hypertarget{ref_model_ct_bundle__getNeuronRowIndex}{}%
\begin{description}
\item[Summary:]Returns the neuron index from bundle.
%
\item[Usage:]~%
\begin{lyxcode}%
a\_row\_index = getNeuronRowIndex(a\_bundle, trial\_num, props)
%
\end{lyxcode}%
%
%
\item[Parameters:]~
\begin{description}%
\item[\texttt{a\_bundle}:]
 A model\_ct\_bundle object.
\item[\texttt{trial\_num}:]
 The trial number of model neuron, or a DB row containing this.
\item[\texttt{props}:]
 A structure with any optional properties.
\end{description}%
%
\item[Returns:
]~

	a\_row\_index: A row index of neuron in a\_bundle.joined\_db.
%
%
\item[See also:]%
\hyperlink{ref_dataset_db_bundle}{\texttt{dataset\_db\_bundle}}%
\ (p.~\pageref{ref_dataset_db_bundle})%
\index[funcref]{dataset_db_bundle@\fidxl{dataset\_db\_bundle}}%
%
\item[Author:]%
Cengiz Gunay <cgunay@emory.edu>, 2006/06/09
%
\end{description}
\methodline%
\subsubsection[Method \texttt{getTrialNum}]{Method \texttt{model\_ct\_bundle/getTrialNum}}%
\index[funcref]{model_ct_bundle@\fidxl{model\_ct\_bundle}!getTrialNum@\fidxl{getTrialNum}}%
\label{ref_model_ct_bundle__getTrialNum}%
\hypertarget{ref_model_ct_bundle__getTrialNum}{}%
\begin{description}
\item[Summary:]Extracts identifying neuron trial number from DB.
%
\item[Usage:]~%
\begin{lyxcode}%
trial\_num = getTrialNum(a\_bundle, a\_db|trial\_num, props)
%
\end{lyxcode}%
%
%
\item[Parameters:]~
\begin{description}%
\item[\texttt{a\_bundle}:]
 A physiol\_cip\_traceset\_fileset object.
\item[\texttt{a\_db}:]
 DB rows representing deisred model neuron(s).
\item[\texttt{trial\_num}:]
 Trial numbers. If specified, this function does nothing but return them.
\item[\texttt{props}:]
 A structure with any optional properties.
\end{description}%
%
\item[Returns:
]~

	trial\_num: The trial number(s) identifying selected neuron(s) in bundle.
%
%
\item[See also:]%
\hyperlink{ref_dataset_db_bundle}{\texttt{dataset\_db\_bundle}}%
\ (p.~\pageref{ref_dataset_db_bundle})%
\index[funcref]{dataset_db_bundle@\fidxl{dataset\_db\_bundle}}%
%
\item[Author:]%
Cengiz Gunay <cgunay@emory.edu>, 2006/05/26
%
\end{description}
\methodline%
\subsubsection[Method \texttt{plotComparefICurve}]{Method \texttt{model\_ct\_bundle/plotComparefICurve}}%
\index[funcref]{model_ct_bundle@\fidxl{model\_ct\_bundle}!plotComparefICurve@\fidxl{plotComparefICurve}}%
\label{ref_model_ct_bundle__plotComparefICurve}%
\hypertarget{ref_model_ct_bundle__plotComparefICurve}{}%
\begin{description}
\item[Summary:]Generates a f-I curve doc\_plot comparing m\_trial and to\_index.
%
\item[Usage:]~%
\begin{lyxcode}%
a\_plot = plotComparefICurve(m\_bundle, m\_trial, to\_bundle, to\_index, props)
%
\end{lyxcode}%
%
\item[Description:]%
Note that this is not a general method. to\_bundle should have been able to accept
 any type of bundle. Most probably this method is redundant and deprecated.
%%
\item[Parameters:]~
\begin{description}%
\item[\texttt{m\_bundle}:]
 A model\_ct\_bundle object.
\item[\texttt{m\_trial}:]
 Trial number of model.
\item[\texttt{to\_bundle}:]
 A physiol\_bundle object.
\item[\texttt{to\_index}:]
 TracesetIndex of neuron.
\item[\texttt{props}:]
 A structure with any optional properties.
\begin{description}%
\item[\texttt{shortCaption}:]
 This appears in the figure caption.
\item[\texttt{plotMStats}:]
 If set, add the m\_bundle stats plot.
\item[\texttt{plotToStats}:]
 If set, add the to\_bundle stats plot.
\item[\texttt{captionToStats}:]
 Use this as its legend label. 
\item[\texttt{quiet}:]
 if given, no title is produced

(passed to plot\_superpose)
\end{description}%
\end{description}%
%
\item[Returns:
]~

	a\_plot: A plot\_superpose that contains a f-I curve plot.
%
\item[Example:]~
\begin{lyxcode} >> a\_p = plotComparefICurve(r, 1);
\\%
 >> plotFigure(a\_p, 'The f-I curve of best matching model');
\\%
\end{lyxcode}
%
\item[See also:]%
\hyperlink{ref_plot_abstract}{\texttt{plot\_abstract}}%
\ (p.~\pageref{ref_plot_abstract})%
\index[funcref]{plot_abstract@\fidxl{plot\_abstract}}%
, \hyperlink{ref_plot_superpose}{\texttt{plot\_superpose}}%
\ (p.~\pageref{ref_plot_superpose})%
\index[funcref]{plot_superpose@\fidxl{plot\_superpose}}%
%
\item[Author:]%
Cengiz Gunay <cgunay@emory.edu>, 2006/01/16
%
\end{description}
\methodline%
\subsubsection[Method \texttt{plotCompareRanks}]{Method \texttt{model\_ct\_bundle/plotCompareRanks}}%
\index[funcref]{model_ct_bundle@\fidxl{model\_ct\_bundle}!plotCompareRanks@\fidxl{plotCompareRanks}}%
\label{ref_model_ct_bundle__plotCompareRanks}%
\hypertarget{ref_model_ct_bundle__plotCompareRanks}{}%
\begin{description}
\item[Summary:]Generates a plots of given ranks from the ranked\_bundle.
%
\item[Usage:]~%
\begin{lyxcode}%
plots = plotCompareRanks(m\_bundle, p\_bundle, a\_ranked\_db, ranks, props)
%
\end{lyxcode}%
%
%
\item[Parameters:]~
\begin{description}%
\item[\texttt{m\_bundle}:]
 A model\_ct\_bundle object.
\item[\texttt{p\_bundle}:]
 A dataset\_db\_bundle object that originated the criterion.
\item[\texttt{a\_ranked\_db}:]
 A ranked\_db generated from ranking m\_bundle.
\item[\texttt{ranks}:]
 Vector of rank indices for which to generate the plots.
\item[\texttt{props}:]
 A structure with any optional properties.
\end{description}%
%
\item[Returns:
]~

	plots: A structure that contains the joined\_db, and the plot vectors 
	  trace\_d100\_plots and trace\_h100\_plots.
%
\item[Example:]~
\begin{lyxcode} >> plots = plotCompareRanks(r, 1:10);
\\%
 >> plotFigure(plots.trace\_d100\_plots(1), 'The best matching +100 pA CIP trace');
\\%
\end{lyxcode}
%
\item[See also:]%
%
\item[Author:]%
Cengiz Gunay <cgunay@emory.edu>, 2006/01/16
%
\end{description}
\methodline%
\subsubsection[Method \texttt{rankMatching}]{Method \texttt{model\_ct\_bundle/rankMatching}}%
\index[funcref]{model_ct_bundle@\fidxl{model\_ct\_bundle}!rankMatching@\fidxl{rankMatching}}%
\label{ref_model_ct_bundle__rankMatching}%
\hypertarget{ref_model_ct_bundle__rankMatching}{}%
\begin{description}
\item[Summary:]Create a ranked\_db from given criterion db.
%
\item[Usage:]~%
\begin{lyxcode}%
a\_ranked\_db = rankMatching(a\_mbundle, a\_crit\_db, props)
%
\end{lyxcode}%
%
%
\item[Parameters:]~
\begin{description}%
\item[\texttt{a\_mbundle}:]
 A model\_ct\_bundle object.
\item[\texttt{a\_crit\_db}:]
 A crit\_db created by a matchingRow method.
\item[\texttt{props}:]
 A structure with any optional properties.

(passed to tests\_db/rankMatching)
\end{description}%
%
\item[Returns:
]~

	a\_ranked\_db: a ranked\_db object containing the rankings.
%
%
\item[See also:]%
\hyperlink{ref_tests_db__rankMatching}{\texttt{tests\_db/rankMatching}}%
\ (p.~\pageref{ref_tests_db__rankMatching})%
\index[funcref]{tests_db@\fidxl{tests\_db}!rankMatching@\fidxl{rankMatching}}%
, \hyperlink{ref_ranked_db}{\texttt{ranked\_db}}%
\ (p.~\pageref{ref_ranked_db})%
\index[funcref]{ranked_db@\fidxl{ranked\_db}}%
%
\item[Author:]%
Cengiz Gunay <cgunay@emory.edu>, 2006/01/18
%
\end{description}
\methodline%
\subsubsection[Method \texttt{reportCompareModelToPhysiolNeuron}]{Method \texttt{model\_ct\_bundle/reportCompareModelToPhysiolNeuron}}%
\index[funcref]{model_ct_bundle@\fidxl{model\_ct\_bundle}!reportCompareModelToPhysiolNeuron@\fidxl{reportCompareModelToPhysiolNeuron}}%
\label{ref_model_ct_bundle__reportCompareModelToPhysiolNeuron}%
\hypertarget{ref_model_ct_bundle__reportCompareModelToPhysiolNeuron}{}%
\begin{description}
\item[Summary:]Generates a report by comparing given model neuron to given physiol neuron.
%
\item[Usage:]~%
\begin{lyxcode}%
a\_doc\_multi = reportCompareModelToPhysiolNeuron(m\_bundle, trial\_num, p\_bundle, 
						  traceset\_index, props)
%
\end{lyxcode}%
%
\item[Description:]%
Generates a report document with:
	- Figure displaying raw traces of the physiol neuron compared with the model neuron
	- Figure comparing f-I curves of the two neurons.
	- Figure comparing spont and pulse spike shapes of the two neurons.
%%
\item[Parameters:]~
\begin{description}%
\item[\texttt{m\_bundle, p\_bundle}:]
 dataset\_db\_bundle objects of the model and physiology neurons.
\item[\texttt{trial\_num}:]
 Trial number of desired model neuron in m\_bundle.
\item[\texttt{traceset\_index}:]
 TracesetIndex of desired neuron in p\_bundle.
\item[\texttt{props}:]
 A structure with any optional properties.
\begin{description}%
\item[\texttt{horizRow}:]
 If defined, create a row-figure with all plots.
\item[\texttt{numPhysTraces}:]
 Number of physiology traces to show in plot (>=1).
\end{description}%
\end{description}%
%
\item[Returns:
]~

	a\_doc\_multi: A doc\_multi object that can be printed as a PS or PDF file.
%
\item[Example:]~
\begin{lyxcode} >> printTeXFile(reportCompareModelToPhysiolNeuron(mbundle, 2222, pbundle, 34), 'a.tex')
\\%
\end{lyxcode}
%
\item[See also:]%
\hyperlink{ref_doc_multi}{\texttt{doc\_multi}}%
\ (p.~\pageref{ref_doc_multi})%
\index[funcref]{doc_multi@\fidxl{doc\_multi}}%
, \hyperlink{ref_doc_generate}{\texttt{doc\_generate}}%
\ (p.~\pageref{ref_doc_generate})%
\index[funcref]{doc_generate@\fidxl{doc\_generate}}%
, \hyperlink{ref_doc_generate__printTeXFile}{\texttt{doc\_generate/printTeXFile}}%
\ (p.~\pageref{ref_doc_generate__printTeXFile})%
\index[funcref]{doc_generate@\fidxl{doc\_generate}!printTeXFile@\fidxl{printTeXFile}}%
%
\item[Author:]%
Cengiz Gunay <cgunay@emory.edu>, 2006/01/24
%
\end{description}
\methodline%
\subsubsection[Method \texttt{reportRankingToPhysiolNeuronsTeXFile}]{Method \texttt{model\_ct\_bundle/reportRankingToPhysiolNeuronsTeXFile}}%
\index[funcref]{model_ct_bundle@\fidxl{model\_ct\_bundle}!reportRankingToPhysiolNeuronsTeXFile@\fidxl{reportRankingToPhysiolNeuronsTeXFile}}%
\label{ref_model_ct_bundle__reportRankingToPhysiolNeuronsTeXFile}%
\hypertarget{ref_model_ct_bundle__reportRankingToPhysiolNeuronsTeXFile}{}%
\begin{description}
\item[Summary:]Compare model DB to given physiol criterion and create a report.
%
\item[Usage:]~%
\begin{lyxcode}%
tex\_filename = reportRankingToPhysiolNeuronsTeXFile(m\_bundle, p\_bundle, a\_crit\_db, props)
%
\end{lyxcode}%
%
\item[Description:]%
A LaTeX report is generated 
 following the example in physiol\_bundle/matchingRow. The filename contains the neuron
 name, followed by the traceset index as an identifier of pharmacological applications,
 as in gpd0421c\_s34. 
%%
\item[Parameters:]~
\begin{description}%
\item[\texttt{m\_bundle}:]
 A model\_ct\_bundle object.
\item[\texttt{p\_bundle}:]
 A physiol\_bundle object.
\item[\texttt{a\_crit\_db}:]
 The criterion neuron chosen with a matchingRow method.
\item[\texttt{props}:]
 A structure with any optional properties.
\begin{description}%
\item[\texttt{filenameSuffix}:]
 Append this identifier to the TeX filename.

(others passed to rankMatching)
\end{description}%
\end{description}%
%
\item[Returns: 
]~

	tex\_filename: Name of LaTeX file generated.
%
%
\item[See also:]%
\hyperlink{ref_tests_db__rankMatching}{\texttt{tests\_db/rankMatching}}%
\ (p.~\pageref{ref_tests_db__rankMatching})%
\index[funcref]{tests_db@\fidxl{tests\_db}!rankMatching@\fidxl{rankMatching}}%
, \hyperlink{ref_physiol_cip_traceset__cip_trace}{\texttt{physiol\_cip\_traceset/cip\_trace}}%
\ (p.~\pageref{ref_physiol_cip_traceset__cip_trace})%
\index[funcref]{physiol_cip_traceset@\fidxl{physiol\_cip\_traceset}!cip_trace@\fidxl{cip\_trace}}%
, \hyperlink{ref_physiol_bundle__matchingRow}{\texttt{physiol\_bundle/matchingRow}}%
\ (p.~\pageref{ref_physiol_bundle__matchingRow})%
\index[funcref]{physiol_bundle@\fidxl{physiol\_bundle}!matchingRow@\fidxl{matchingRow}}%
%
\item[Author:]%
Cengiz Gunay <cgunay@emory.edu>, 2006/01/18
%
\end{description}
\methodline%
\subsubsection[Method \texttt{set}]{Method \texttt{model\_ct\_bundle/set}}%
\index[funcref]{model_ct_bundle@\fidxl{model\_ct\_bundle}!set@\fidxl{set}}%
\label{ref_model_ct_bundle__set}%
\hypertarget{ref_model_ct_bundle__set}{}%
\begin{description}
\item[Summary:]Generic method for setting object attributes.
%
%
%
%
%
%
%
\item[Author:]%
Cengiz Gunay <cgunay@emory.edu>, 2004/10/08
%
\end{description}
\methodline%
\subsection{Class \texttt{model\_data\_vcs}}%
\index[funcref]{model_data_vcs@\fidxl{model\_data\_vcs}|boldhyperpage}%
\label{ref_model_data_vcs}%
\hypertarget{ref_model_data_vcs}{}%
\subsubsection[Constructor \texttt{model\_data\_vcs}]{Constructor \texttt{model\_data\_vcs/model\_data\_vcs}}%
\index[funcref]{model_data_vcs@\fidxl{model\_data\_vcs}!model_data_vcs@\fidxl{model\_data\_vcs}}%
\label{ref_model_data_vcs__model_data_vcs}%
\hypertarget{ref_model_data_vcs__model_data_vcs}{}%
\begin{description}
\item[Summary:]Combines model description that fits a voltage clamp data.
%
\item[Usage:]~%
\begin{lyxcode}%
a\_md = model\_data\_vcs(model\_f, data\_vc, id, props)
%
\end{lyxcode}%
%
\item[Description:]%
For tasks such as plotting comparison of model to data and generating
 initial fits for model. Can also have a GUI here for fitting.
%%
\item[Parameters:]~
\begin{description}%
\item[\texttt{model\_f}:]
 Model as a param\_func object or a NeuroFit file name. 

Make sure to specify required param\_I\_Neurofit props below. 
\item[\texttt{data\_vc}:]
 Data as a voltage\_clamp object directly or in a MAT file or

from an ABF file.
\item[\texttt{id}:]
 Identification string.
\item[\texttt{props}:]
 A structure with any optional properties.
\end{description}%
%
\item[Returns a structure object with the following fields:
]~

   model\_f, data\_vc,
   model\_vc: Obtained by simulating model.
%
%
\item[See also:]%
\hyperlink{ref_voltage_clamp}{\texttt{voltage\_clamp}}%
\ (p.~\pageref{ref_voltage_clamp})%
\index[funcref]{voltage_clamp@\fidxl{voltage\_clamp}}%
, \hyperlink{ref_param_func}{\texttt{param\_func}}%
\ (p.~\pageref{ref_param_func})%
\index[funcref]{param_func@\fidxl{param\_func}}%
%
\item[Author:]%
Cengiz Gunay <cgunay@emory.edu>, 2010/10/11
%
\end{description}
\methodline%
\subsubsection[Method \texttt{convertTauFromSpline}]{Method \texttt{model\_data\_vcs/convertTauFromSpline}}%
\index[funcref]{model_data_vcs@\fidxl{model\_data\_vcs}!convertTauFromSpline@\fidxl{convertTauFromSpline}}%
\label{ref_model_data_vcs__convertTauFromSpline}%
\hypertarget{ref_model_data_vcs__convertTauFromSpline}{}%
\begin{description}
\item[Summary:]Converts m and h tau functions from spline to Hodgkin-Huxley form.
%
\item[Usage:]~%
\begin{lyxcode}%
a\_md = convertTauFromSpline(a\_md, props)
%
\end{lyxcode}%
%
%
\item[Parameters:]~
\begin{description}%
\item[\texttt{a\_md}:]
 A model\_data\_vcs object.
\item[\texttt{props}:]
 A structure with any optional properties.
\begin{description}%
\item[\texttt{vRange}:]
 Voltage values to evaluate spline function (default=-30:60)
\end{description}%
\end{description}%
%
\item[Returns:
]~

   a\_md: (updated)
%
\item[Example:]~
\begin{lyxcode} >> a\_test\_md = convertTauFromSpline(a\_md)
\\%
\end{lyxcode}
%
\item[See also:]%
\hyperlink{ref_model_data_vcs}{\texttt{model\_data\_vcs}}%
\ (p.~\pageref{ref_model_data_vcs})%
\index[funcref]{model_data_vcs@\fidxl{model\_data\_vcs}}%
, \hyperlink{ref_voltage_clamp}{\texttt{voltage\_clamp}}%
\ (p.~\pageref{ref_voltage_clamp})%
\index[funcref]{voltage_clamp@\fidxl{voltage\_clamp}}%
, \hyperlink{ref_plot_abstract}{\texttt{plot\_abstract}}%
\ (p.~\pageref{ref_plot_abstract})%
\index[funcref]{plot_abstract@\fidxl{plot\_abstract}}%
, \hyperlink{ref_plotFigure}{\texttt{plotFigure}}%
\ (p.~\pageref{ref_plotFigure})%
\index[funcref]{plotFigure@\fidxl{plotFigure}}%
%
\item[Author:]%
Cengiz Gunay <cgunay@emory.edu>, 2011/05/31
%
\end{description}
\methodline%
\subsubsection[Method \texttt{display}]{Method \texttt{model\_data\_vcs/display}}%
\index[funcref]{model_data_vcs@\fidxl{model\_data\_vcs}!display@\fidxl{display}}%
\label{ref_model_data_vcs__display}%
\hypertarget{ref_model_data_vcs__display}{}%
\begin{description}
\item[Summary:]Return string representation of object.
%
\item[Usage:]~%
\begin{lyxcode}%
str = display(a\_md)
%
\end{lyxcode}%
%
%
\item[Parameters:]~
\begin{description}%
\item[\texttt{a\_md}:]
 A model\_data\_vcs object.
\end{description}%
%
\item[Returns:
]~

   str: Readable string
%
%
\item[See also:]%
\hyperlink{ref_model_data_vcs}{\texttt{model\_data\_vcs}}%
\ (p.~\pageref{ref_model_data_vcs})%
\index[funcref]{model_data_vcs@\fidxl{model\_data\_vcs}}%
%
\item[Author:]%
Cengiz Gunay <cgunay@emory.edu>, 2011/05/31
%
\end{description}
\methodline%
\subsubsection[Method \texttt{fit}]{Method \texttt{model\_data\_vcs/fit}}%
\index[funcref]{model_data_vcs@\fidxl{model\_data\_vcs}!fit@\fidxl{fit}}%
\label{ref_model_data_vcs__fit}%
\hypertarget{ref_model_data_vcs__fit}{}%
\begin{description}
\item[Summary:]Fit model to data.
%
\item[Usage:]~%
\begin{lyxcode}%
[a\_md a\_doc] = fit(a\_md, title\_str, props)
%
\end{lyxcode}%
%
\item[Description:]%
Caveat: what you see in the plots may be different that what the
 algorithm is comparing for the fits if you provide fitRange or
 fitRangeRel because the plots will simulate the whole range
 anyway. Therefore the plots will always have a more correct output than
 the fitting algorithm -- especially if you selected a too narrow
 simlation range that doesn't let the model react to voltage levels.
%%
\item[Parameters:]~
\begin{description}%
\item[\texttt{a\_md}:]
 A model\_data\_vcs object.
\item[\texttt{props}:]
 A structure with any optional properties.
\begin{description}%
\item[\texttt{fitRange}:]
 Start and end times [ms] of simulated range used for

optimization. Note that simulated range must include prior
stimulation steps to set state of diff. eqs.
\item[\texttt{fitRangeRel}:]
 Like fitRange, but relative to first voltage step

[ms]. Specify any other voltage step as the first
element. See getTimeRelStep.
\item[\texttt{outRangeRel}:]
 Only use this range of the simulated data for

fitting. Defined same as fitRangeRel. Multiple rows can contain
separate ranges are patched together.
\item[\texttt{fitLevels}:]
 Indices of voltage/current levels to use from clamp

data. If empty, not fit is done.
\item[\texttt{dispParams}:]
 If non-zero, display params every once this many iterations.
\item[\texttt{dispPlot}:]
 If non-zero, update a plot of the fit at end of this many

iterations. A zero means no plot will be produced at the
end.
\item[\texttt{saveModelFile}:]
 If given, save the model every dispParams iteration.
\item[\texttt{saveModelAutoNum}:]
 Use this as a base number and use saveModelFile as sprintf formatted

string that includes a number string (e.g., '%d') and increment it
until a non-existing file name is found.
\item[\texttt{savePlotFile}:]
 If given, save the plot to this file every dispPlot iteration as the model file.
\item[\texttt{plotMd}:]
 model\_data\_vcs or subclass object to be used for

plots. This reuses the data in the md and only updates the model
parameters in the plot.
\item[\texttt{quiet}:]
 If 1, do not include cell name on title.
\end{description}%
\end{description}%
%
\item[Returns:
]~

   a\_md: Updated model\_data\_vcs object with fit.
   a\_doc: A doc\_plot object containing the annotated figure.
%
\item[Example:]~
\begin{lyxcode} >> a\_md = ...
\\%
    fit(a\_md, '', ...
\\%
      struct('fitRangeRel', [-.2 165], 'fitLevels', 1:5, ...
\\%
             'dispParams', 5, ...
\\%
             'optimset', struct('Display', 'iter')));
\\%
\end{lyxcode}
%
\item[See also:]%
\hyperlink{ref_param_I_v}{\texttt{param\_I\_v}}%
\ (p.~\pageref{ref_param_I_v})%
\index[funcref]{param_I_v@\fidxl{param\_I\_v}}%
, \hyperlink{ref_param_func}{\texttt{param\_func}}%
\ (p.~\pageref{ref_param_func})%
\index[funcref]{param_func@\fidxl{param\_func}}%
, \hyperlink{ref_doc_plot}{\texttt{doc\_plot}}%
\ (p.~\pageref{ref_doc_plot})%
\index[funcref]{doc_plot@\fidxl{doc\_plot}}%
, \hyperlink{ref_voltage_clamp__getTimeRelStep}{\texttt{voltage\_clamp/getTimeRelStep}}%
\ (p.~\pageref{ref_voltage_clamp__getTimeRelStep})%
\index[funcref]{voltage_clamp@\fidxl{voltage\_clamp}!getTimeRelStep@\fidxl{getTimeRelStep}}%
%
\item[Author:]%
Cengiz Gunay <cgunay@emory.edu>, 2010/10/12
%
\end{description}
\methodline%
\subsubsection[Method \texttt{get}]{Method \texttt{model\_data\_vcs/get}}%
\index[funcref]{model_data_vcs@\fidxl{model\_data\_vcs}!get@\fidxl{get}}%
\label{ref_model_data_vcs__get}%
\hypertarget{ref_model_data_vcs__get}{}%
\begin{description}
\item[Summary:]Defines generic attribute retrieval for objects.
%
%
%
%
%
%
%
\item[Author:]%
Cengiz Gunay <cgunay@emory.edu>, 2004/09/14
%
\end{description}
\methodline%
\subsubsection[Method \texttt{plot}]{Method \texttt{model\_data\_vcs/plot}}%
\index[funcref]{model_data_vcs@\fidxl{model\_data\_vcs}!plot@\fidxl{plot}}%
\label{ref_model_data_vcs__plot}%
\hypertarget{ref_model_data_vcs__plot}{}%
\begin{description}
\item[Summary:]Generic method to plot a tests\_db or a subclass. Requires a 
	plot\_abstract method to be defined for this object.
%
\item[Usage:]~%
\begin{lyxcode}%
h = plot(a\_obj, title\_str, props)
%
\end{lyxcode}%
%
%
\item[Parameters:]~
\begin{description}%
\item[\texttt{a\_obj}:]
 An object.
\item[\texttt{title\_str}:]
 (Optional) String to append to plot title.
\item[\texttt{props}:]
 A structure with any optional properties, passed to plot\_abstract.
\end{description}%
%
\item[Returns:
]~

	h: The figure handle created.
%
%
\item[See also:]%
\hyperlink{ref_plot_abstract}{\texttt{plot\_abstract}}%
\ (p.~\pageref{ref_plot_abstract})%
\index[funcref]{plot_abstract@\fidxl{plot\_abstract}}%
, \hyperlink{ref_plotFigure}{\texttt{plotFigure}}%
\ (p.~\pageref{ref_plotFigure})%
\index[funcref]{plotFigure@\fidxl{plotFigure}}%
%
\item[Author:]%
Cengiz Gunay <cgunay@emory.edu>, 2004/10/06
%
\end{description}
\methodline%
\subsubsection[Method \texttt{plotDataCompare}]{Method \texttt{model\_data\_vcs/plotDataCompare}}%
\index[funcref]{model_data_vcs@\fidxl{model\_data\_vcs}!plotDataCompare@\fidxl{plotDataCompare}}%
\label{ref_model_data_vcs__plotDataCompare}%
\hypertarget{ref_model_data_vcs__plotDataCompare}{}%
\begin{description}
\item[Summary:]Superpose model and data raw traces.
%
\item[Usage:]~%
\begin{lyxcode}%
a\_p = plotDataCompare(a\_md, title\_str, props)
%
\end{lyxcode}%
%
%
\item[Parameters:]~
\begin{description}%
\item[\texttt{a\_md}:]
 A model\_data\_vcs object.
\item[\texttt{title\_str}:]
 (Optional) Text to appear in the plot title.
\item[\texttt{props}:]
 A structure with any optional properties.
\begin{description}%
\item[\texttt{quiet}:]
 If 1, only use given title\_str.
\item[\texttt{zoom}:]
 Zoom into activation or inactivation parts if 'act' or

'inact', resp.
\item[\texttt{skipStep}:]
 Number of voltage steps to skip at the start for zoom (default=0).
\item[\texttt{showSub}:]
 also plot subtracted current
\item[\texttt{showV}:]
 also plot voltage protocol.
\item[\texttt{levels}:]
 Only plot these current and voltage levels
\item[\texttt{colorLevels}:]
 Cycle colors every this number.
\item[\texttt{axisLimits}:]
 Set current traces to these limits unless 'zoom' prop

is specified.
\item[\texttt{iLimits}:]
 If specified, override axisLimits y-axis values with these

only for the data plot (not the subtraction plot).
\end{description}%
\end{description}%
%
\item[Returns:
]~

   a\_p: A plot\_abstract object.
%
\item[Example:]~
\begin{lyxcode} >> a\_md = model\_data\_vcs(model, data\_vc)
\\%
 >> plotFigure(plotDataCompare(a\_md, 'my model'))
\\%
\end{lyxcode}
%
\item[See also:]%
\hyperlink{ref_model_data_vcs}{\texttt{model\_data\_vcs}}%
\ (p.~\pageref{ref_model_data_vcs})%
\index[funcref]{model_data_vcs@\fidxl{model\_data\_vcs}}%
, \hyperlink{ref_voltage_clamp}{\texttt{voltage\_clamp}}%
\ (p.~\pageref{ref_voltage_clamp})%
\index[funcref]{voltage_clamp@\fidxl{voltage\_clamp}}%
, \hyperlink{ref_plot_abstract}{\texttt{plot\_abstract}}%
\ (p.~\pageref{ref_plot_abstract})%
\index[funcref]{plot_abstract@\fidxl{plot\_abstract}}%
, \hyperlink{ref_plotFigure}{\texttt{plotFigure}}%
\ (p.~\pageref{ref_plotFigure})%
\index[funcref]{plotFigure@\fidxl{plotFigure}}%
%
\item[Author:]%
Cengiz Gunay <cgunay@emory.edu>, 2010/10/12
%
\end{description}
\methodline%
\subsubsection[Method \texttt{plotDataModelSub}]{Method \texttt{model\_data\_vcs/plotDataModelSub}}%
\index[funcref]{model_data_vcs@\fidxl{model\_data\_vcs}!plotDataModelSub@\fidxl{plotDataModelSub}}%
\label{ref_model_data_vcs__plotDataModelSub}%
\hypertarget{ref_model_data_vcs__plotDataModelSub}{}%
\begin{description}
\item[Summary:]Plot model traces subtracted from raw data.
%
\item[Usage:]~%
\begin{lyxcode}%
a\_p = plotDataModelSub(a\_md, title\_str, props)
%
\end{lyxcode}%
%
%
\item[Parameters:]~
\begin{description}%
\item[\texttt{a\_md}:]
 A model\_data\_vcs object.
\item[\texttt{title\_str}:]
 (Optional) Text to appear in the plot title.
\item[\texttt{props}:]
 A structure with any optional properties.
\begin{description}%
\item[\texttt{quiet}:]
 If 1, only use given title\_str.
\item[\texttt{zoom}:]
 Zoom into activation or inactivation parts if 'act' or

'inact', resp.
\item[\texttt{skipStep}:]
 Number of voltage steps to skip at the start for zoom (default=0).
\item[\texttt{levels}:]
 Only plot these current and voltage levels
\item[\texttt{colorLevels}:]
 Cycle colors every this number.
\item[\texttt{axisLimits}:]
 Set current traces to these limits unless 'zoom' prop

is specified.
(Rest passed to voltage\_clamp/plot\_abstract)
\end{description}%
\end{description}%
%
\item[Returns:
]~

   a\_p: A plot\_abstract object.
%
\item[Example:]~
\begin{lyxcode} >> a\_md = model\_data\_vcs(model, data\_vc)
\\%
 >> plotFigure(plotDataModelSub(a\_md, 'my model'))
\\%
\end{lyxcode}
%
\item[See also:]%
\hyperlink{ref_model_data_vcs}{\texttt{model\_data\_vcs}}%
\ (p.~\pageref{ref_model_data_vcs})%
\index[funcref]{model_data_vcs@\fidxl{model\_data\_vcs}}%
, \hyperlink{ref_voltage_clamp}{\texttt{voltage\_clamp}}%
\ (p.~\pageref{ref_voltage_clamp})%
\index[funcref]{voltage_clamp@\fidxl{voltage\_clamp}}%
, \hyperlink{ref_voltage_clamp__plot_abstract}{\texttt{voltage\_clamp/plot\_abstract}}%
\ (p.~\pageref{ref_voltage_clamp__plot_abstract})%
\index[funcref]{voltage_clamp@\fidxl{voltage\_clamp}!plot_abstract@\fidxl{plot\_abstract}}%
, \hyperlink{ref_plot_abstract}{\texttt{plot\_abstract}}%
\ (p.~\pageref{ref_plot_abstract})%
\index[funcref]{plot_abstract@\fidxl{plot\_abstract}}%
, \hyperlink{ref_plotFigure}{\texttt{plotFigure}}%
\ (p.~\pageref{ref_plotFigure})%
\index[funcref]{plotFigure@\fidxl{plotFigure}}%
%
\item[Author:]%
Cengiz Gunay <cgunay@emory.edu>, 2010/10/21
%
\end{description}
\methodline%
\subsubsection[Method \texttt{plotModelInfs}]{Method \texttt{model\_data\_vcs/plotModelInfs}}%
\index[funcref]{model_data_vcs@\fidxl{model\_data\_vcs}!plotModelInfs@\fidxl{plotModelInfs}}%
\label{ref_model_data_vcs__plotModelInfs}%
\hypertarget{ref_model_data_vcs__plotModelInfs}{}%
\begin{description}
\item[Summary:]Plot model m\_inf and h\_inf curves.
%
\item[Usage:]~%
\begin{lyxcode}%
a\_p = plotModelInfs(a\_md, title\_str, props)
%
\end{lyxcode}%
%
%
\item[Parameters:]~
\begin{description}%
\item[\texttt{a\_md}:]
 A model\_data\_vcs object.
\item[\texttt{title\_str}:]
 (Optional) Text to appear in the plot title.
\item[\texttt{props}:]
 A structure with any optional properties.
\begin{description}%
\item[\texttt{quiet}:]
 If 1, only use given title\_str.
\end{description}%
\end{description}%
%
\item[Returns:
]~

   a\_p: A plot\_abstract object.
%
\item[Example:]~
\begin{lyxcode} >> a\_md = model\_data\_vcs(model, data\_vc)
\\%
 >> plotFigure(plotModelInfs(a\_md, 'my model'))
\\%
\end{lyxcode}
%
\item[See also:]%
\hyperlink{ref_model_data_vcs}{\texttt{model\_data\_vcs}}%
\ (p.~\pageref{ref_model_data_vcs})%
\index[funcref]{model_data_vcs@\fidxl{model\_data\_vcs}}%
, \hyperlink{ref_voltage_clamp}{\texttt{voltage\_clamp}}%
\ (p.~\pageref{ref_voltage_clamp})%
\index[funcref]{voltage_clamp@\fidxl{voltage\_clamp}}%
, \hyperlink{ref_plot_abstract}{\texttt{plot\_abstract}}%
\ (p.~\pageref{ref_plot_abstract})%
\index[funcref]{plot_abstract@\fidxl{plot\_abstract}}%
, \hyperlink{ref_plotFigure}{\texttt{plotFigure}}%
\ (p.~\pageref{ref_plotFigure})%
\index[funcref]{plotFigure@\fidxl{plotFigure}}%
%
\item[Author:]%
Cengiz Gunay <cgunay@emory.edu>, 2010/10/12
%
\end{description}
\methodline%
\subsubsection[Method \texttt{plotModelTaus}]{Method \texttt{model\_data\_vcs/plotModelTaus}}%
\index[funcref]{model_data_vcs@\fidxl{model\_data\_vcs}!plotModelTaus@\fidxl{plotModelTaus}}%
\label{ref_model_data_vcs__plotModelTaus}%
\hypertarget{ref_model_data_vcs__plotModelTaus}{}%
\begin{description}
\item[Summary:]Plot I/V curves comparing model and data.
%
\item[Usage:]~%
\begin{lyxcode}%
a\_p = plotModelTaus(a\_md, title\_str, props)
%
\end{lyxcode}%
%
%
\item[Parameters:]~
\begin{description}%
\item[\texttt{a\_md}:]
 A model\_data\_vcs object.
\item[\texttt{title\_str}:]
 (Optional) Text to appear in the plot title.
\item[\texttt{props}:]
 A structure with any optional properties.
\begin{description}%
\item[\texttt{quiet}:]
 If 1, only use given title\_str.
\end{description}%
\end{description}%
%
\item[Returns:
]~

   a\_p: A plot\_abstract object.
%
\item[Example:]~
\begin{lyxcode} >> a\_md = model\_data\_vcs(model, data\_vc)
\\%
 >> plotFigure(plotModelTaus(a\_md, 'I/V curves'))
\\%
\end{lyxcode}
%
\item[See also:]%
\hyperlink{ref_model_data_vcs}{\texttt{model\_data\_vcs}}%
\ (p.~\pageref{ref_model_data_vcs})%
\index[funcref]{model_data_vcs@\fidxl{model\_data\_vcs}}%
, \hyperlink{ref_voltage_clamp}{\texttt{voltage\_clamp}}%
\ (p.~\pageref{ref_voltage_clamp})%
\index[funcref]{voltage_clamp@\fidxl{voltage\_clamp}}%
, \hyperlink{ref_plot_abstract}{\texttt{plot\_abstract}}%
\ (p.~\pageref{ref_plot_abstract})%
\index[funcref]{plot_abstract@\fidxl{plot\_abstract}}%
, \hyperlink{ref_plotFigure}{\texttt{plotFigure}}%
\ (p.~\pageref{ref_plotFigure})%
\index[funcref]{plotFigure@\fidxl{plotFigure}}%
%
\item[Author:]%
Cengiz Gunay <cgunay@emory.edu>, 2010/10/11
%
\end{description}
\methodline%
\subsubsection[Method \texttt{plotPeaksCompare}]{Method \texttt{model\_data\_vcs/plotPeaksCompare}}%
\index[funcref]{model_data_vcs@\fidxl{model\_data\_vcs}!plotPeaksCompare@\fidxl{plotPeaksCompare}}%
\label{ref_model_data_vcs__plotPeaksCompare}%
\hypertarget{ref_model_data_vcs__plotPeaksCompare}{}%
\begin{description}
\item[Summary:]Plot I/V curves comparing model and data.
%
\item[Usage:]~%
\begin{lyxcode}%
a\_p = plotPeaksCompare(a\_md, title\_str, props)
%
\end{lyxcode}%
%
%
\item[Parameters:]~
\begin{description}%
\item[\texttt{a\_md}:]
 A model\_data\_vcs object.
\item[\texttt{title\_str}:]
 (Optional) Text to appear in the plot title.
\item[\texttt{props}:]
 A structure with any optional properties.
\begin{description}%
\item[\texttt{quiet}:]
 If 1, only use given title\_str.
\item[\texttt{skipStep}:]
 Number of voltage steps to skip at the start (default=0).
\end{description}%
\end{description}%
%
\item[Returns:
]~

   a\_p: A plot\_abstract object.
%
\item[Example:]~
\begin{lyxcode} >> a\_md = model\_data\_vcs(model, data\_vc)
\\%
 >> plotFigure(plotPeaksCompare(a\_md, 'I/V curves'))
\\%
\end{lyxcode}
%
\item[See also:]%
\hyperlink{ref_model_data_vcs}{\texttt{model\_data\_vcs}}%
\ (p.~\pageref{ref_model_data_vcs})%
\index[funcref]{model_data_vcs@\fidxl{model\_data\_vcs}}%
, \hyperlink{ref_voltage_clamp}{\texttt{voltage\_clamp}}%
\ (p.~\pageref{ref_voltage_clamp})%
\index[funcref]{voltage_clamp@\fidxl{voltage\_clamp}}%
, \hyperlink{ref_plot_abstract}{\texttt{plot\_abstract}}%
\ (p.~\pageref{ref_plot_abstract})%
\index[funcref]{plot_abstract@\fidxl{plot\_abstract}}%
, \hyperlink{ref_plotFigure}{\texttt{plotFigure}}%
\ (p.~\pageref{ref_plotFigure})%
\index[funcref]{plotFigure@\fidxl{plotFigure}}%
%
\item[Author:]%
Cengiz Gunay <cgunay@emory.edu>, 2010/10/11
%
\end{description}
\methodline%
\subsubsection[Method \texttt{plot\_abstract}]{Method \texttt{model\_data\_vcs/plot\_abstract}}%
\index[funcref]{model_data_vcs@\fidxl{model\_data\_vcs}!plot_abstract@\fidxl{plot\_abstract}}%
\label{ref_model_data_vcs__plot_abstract}%
\hypertarget{ref_model_data_vcs__plot_abstract}{}%
\begin{description}
\item[Summary:]Superpose model and data raw traces.
%
\item[Usage:]~%
\begin{lyxcode}%
a\_p = plot\_abstract(a\_md, title\_str, props)
%
\end{lyxcode}%
%
%
\item[Parameters:]~
\begin{description}%
\item[\texttt{a\_md}:]
 A model\_data\_vcs object.
\item[\texttt{title\_str}:]
 (Optional) Text to appear in the plot title.
\item[\texttt{props}:]
 A structure with any optional properties.
\begin{description}%
\item[\texttt{quiet}:]
 If 1, only use given title\_str.
\item[\texttt{zoom}:]
 Zoom into activation or inactivation parts if 'act' or

'inact', resp. Can be a cell to have multiple of these.
\item[\texttt{skipStep}:]
 Number of voltage steps to skip at the start for zoom (default=0).
\item[\texttt{show}:]
 'sub' for subtracted current, and 'v' for voltage trace at

the bottom row.
\item[\texttt{levels}:]
 Only plot these current and voltage levels
\item[\texttt{colorLevels}:]
 Cycle colors every this number.
\item[\texttt{axisLimits}:]
 Set current traces to these limits unless 'zoom' prop

is specified. If it has multiple rows, create multiple data
plots for each set of limits.
\item[\texttt{vLimits}:]
 If given, limit all voltage plot X axes to these.
\item[\texttt{iLimits}:]
 If specified, override axisLimits y-axis values with these

only for the data plot (not the subtraction plot).
\item[\texttt{dataPlotProps}:]
 Props passed to plotDataCompare.
\end{description}%
\end{description}%
%
\item[Returns:
]~

   a\_p: A plot\_abstract object.
%
\item[Example:]~
\begin{lyxcode} >> a\_md = model\_data\_vcs(model, data\_vc)
\\%
 >> plotFigure(plot\_abstract(a\_md, 'my model'))
\\%
\end{lyxcode}
%
\item[See also:]%
\hyperlink{ref_model_data_vcs}{\texttt{model\_data\_vcs}}%
\ (p.~\pageref{ref_model_data_vcs})%
\index[funcref]{model_data_vcs@\fidxl{model\_data\_vcs}}%
, \hyperlink{ref_voltage_clamp}{\texttt{voltage\_clamp}}%
\ (p.~\pageref{ref_voltage_clamp})%
\index[funcref]{voltage_clamp@\fidxl{voltage\_clamp}}%
, \hyperlink{ref_plot_abstract}{\texttt{plot\_abstract}}%
\ (p.~\pageref{ref_plot_abstract})%
\index[funcref]{plot_abstract@\fidxl{plot\_abstract}}%
, \hyperlink{ref_plotFigure}{\texttt{plotFigure}}%
\ (p.~\pageref{ref_plotFigure})%
\index[funcref]{plotFigure@\fidxl{plotFigure}}%
%
\item[Author:]%
Cengiz Gunay <cgunay@emory.edu>, 2010/10/12
%
\end{description}
\methodline%
\subsubsection[Method \texttt{selectFitParams}]{Method \texttt{model\_data\_vcs/selectFitParams}}%
\index[funcref]{model_data_vcs@\fidxl{model\_data\_vcs}!selectFitParams@\fidxl{selectFitParams}}%
\label{ref_model_data_vcs__selectFitParams}%
\hypertarget{ref_model_data_vcs__selectFitParams}{}%
\begin{description}
\item[Summary:]Constrain or release model parameters for fast current.
%
\item[Usage:]~%
\begin{lyxcode}%
a\_md = selectFitParams(a\_md, select\_what, fit\_nofit, props)
%
\end{lyxcode}%
%
%
\item[Parameters:]~
\begin{description}%
\item[\texttt{a\_md}:]
 A model\_data\_vcs object.
\item[\texttt{select\_what}:]
 One of 'passive', 'fast', 'fastInact'
\item[\texttt{fit\_nofit}:]
 1 for including in fits and 0 for not.
\item[\texttt{props}:]
 A structure with any optional properties.
\end{description}%
%
\item[Returns:
]~

   a\_md: Updated object.
%
\item[Example:]~
\begin{lyxcode} >> a\_md = selectFitParams(model\_data\_vcs(model, data\_vc))
\\%
\end{lyxcode}
%
\item[See also:]%
\hyperlink{ref_model_data_vcs}{\texttt{model\_data\_vcs}}%
\ (p.~\pageref{ref_model_data_vcs})%
\index[funcref]{model_data_vcs@\fidxl{model\_data\_vcs}}%
, \hyperlink{ref_model_data_vcs__fit}{\texttt{model\_data\_vcs/fit}}%
\ (p.~\pageref{ref_model_data_vcs__fit})%
\index[funcref]{model_data_vcs@\fidxl{model\_data\_vcs}!fit@\fidxl{fit}}%
, \hyperlink{ref_voltage_clamp}{\texttt{voltage\_clamp}}%
\ (p.~\pageref{ref_voltage_clamp})%
\index[funcref]{voltage_clamp@\fidxl{voltage\_clamp}}%
, \hyperlink{ref_plot_abstract}{\texttt{plot\_abstract}}%
\ (p.~\pageref{ref_plot_abstract})%
\index[funcref]{plot_abstract@\fidxl{plot\_abstract}}%
, \hyperlink{ref_plotFigure}{\texttt{plotFigure}}%
\ (p.~\pageref{ref_plotFigure})%
\index[funcref]{plotFigure@\fidxl{plotFigure}}%
%
\item[Author:]%
Cengiz Gunay <cgunay@emory.edu>, 2010/10/23
%
\end{description}
\methodline%
\subsubsection[Method \texttt{set}]{Method \texttt{model\_data\_vcs/set}}%
\index[funcref]{model_data_vcs@\fidxl{model\_data\_vcs}!set@\fidxl{set}}%
\label{ref_model_data_vcs__set}%
\hypertarget{ref_model_data_vcs__set}{}%
\begin{description}
\item[Summary:]Generic method for setting object attributes.
%
%
%
%
%
%
%
\item[Author:]%
Cengiz Gunay <cgunay@emory.edu>, 2004/10/08
%
\end{description}
\methodline%
\subsubsection[Method \texttt{subsasgn}]{Method \texttt{model\_data\_vcs/subsasgn}}%
\index[funcref]{model_data_vcs@\fidxl{model\_data\_vcs}!subsasgn@\fidxl{subsasgn}}%
\label{ref_model_data_vcs__subsasgn}%
\hypertarget{ref_model_data_vcs__subsasgn}{}%
\begin{description}
\item[Summary:]Defines generic index-based assignment for objects.
%
%
%
%
%
%
%
\item[Author:]%
Cengiz Gunay <cgunay@emory.edu>, 2006/02/06
%
\end{description}
\methodline%
\subsubsection[Method \texttt{subsref}]{Method \texttt{model\_data\_vcs/subsref}}%
\index[funcref]{model_data_vcs@\fidxl{model\_data\_vcs}!subsref@\fidxl{subsref}}%
\label{ref_model_data_vcs__subsref}%
\hypertarget{ref_model_data_vcs__subsref}{}%
\begin{description}
\item[Summary:]Defines indexing for tests\_db objects for () and . operations. 
%
\item[Usage:]~%
\begin{lyxcode}%
obj = obj(rows, tests)
 obj = obj.attribute
%
\end{lyxcode}%
%
\item[Description:]%
Returns attributes or selects the given test columns and rows
 and returns in a new tests\_db object.
%%
\item[Parameters:]~
\begin{description}%
\item[\texttt{obj}:]
 A tests\_db object.
\item[\texttt{rows}:]
 A logical or index vector of rows. If ':', all rows.
\item[\texttt{tests}:]
 Cell array of test names or column indices. If ':', all tests.
\item[\texttt{attribute}:]
 A tests\_db class attribute.
\end{description}%
%
\item[Returns:
]~

	obj: The new tests\_db object.
%
%
\item[See also:]%
\hyperlink{ref_subsref}{\texttt{subsref}}%
\ (p.~\pageref{ref_subsref})%
\index[funcref]{subsref@\fidxl{subsref}}%
, \hyperlink{ref_tests_db}{\texttt{tests\_db}}%
\ (p.~\pageref{ref_tests_db})%
\index[funcref]{tests_db@\fidxl{tests\_db}}%
%
\item[Author:]%
Cengiz Gunay <cgunay@emory.edu>, 2004/09/17
%
\end{description}
\methodline%
\subsubsection[Method \texttt{updateModel}]{Method \texttt{model\_data\_vcs/updateModel}}%
\index[funcref]{model_data_vcs@\fidxl{model\_data\_vcs}!updateModel@\fidxl{updateModel}}%
\label{ref_model_data_vcs__updateModel}%
\hypertarget{ref_model_data_vcs__updateModel}{}%
\begin{description}
\item[Summary:]Simulate and save new model into object.
%
\item[Usage:]~%
\begin{lyxcode}%
a\_md = updateModel(a\_md, model\_f, props)
%
\end{lyxcode}%
%
\item[Description:]%
Simulates the model to update the model\_vc contained.
%%
\item[Parameters:]~
\begin{description}%
\item[\texttt{a\_md}:]
 A model\_data\_vcs object.
\item[\texttt{model\_f}:]
 (optional) param\_func or subclass object that holds the new

model function. If not given, existing model is simulated.
\item[\texttt{props}:]
 A structure with any optional properties.

(passed to voltage\_clamp/simModel)
\end{description}%
%
\item[Returns:
]~

   a\_md: Updated object.
%
\item[Example:]~
\begin{lyxcode} >> a\_md = model\_data\_vcs(model\_f, data\_vc)
\\%
 >> a\_md = updateModel(a\_md, new\_model\_f))
\\%
\end{lyxcode}
%
\item[See also:]%
\hyperlink{ref_model_data_vcs}{\texttt{model\_data\_vcs}}%
\ (p.~\pageref{ref_model_data_vcs})%
\index[funcref]{model_data_vcs@\fidxl{model\_data\_vcs}}%
%
\item[Author:]%
Cengiz Gunay <cgunay@emory.edu>, 2010/10/14
%
\end{description}
\methodline%
\subsection{Class \texttt{model\_ranked\_to\_physiol\_bundle}}%
\index[funcref]{model_ranked_to_physiol_bundle@\fidxl{model\_ranked\_to\_physiol\_bundle}|boldhyperpage}%
\label{ref_model_ranked_to_physiol_bundle}%
\hypertarget{ref_model_ranked_to_physiol_bundle}{}%
\subsubsection[Constructor \texttt{model\_ranked\_to\_physiol\_bundle}]{Constructor \texttt{model\_ranked\_to\_physiol\_bundle/model\_ranked\_to\_physiol\_bundle}}%
\index[funcref]{model_ranked_to_physiol_bundle@\fidxl{model\_ranked\_to\_physiol\_bundle}!model_ranked_to_physiol_bundle@\fidxl{model\_ranked\_to\_physiol\_bundle}}%
\label{ref_model_ranked_to_physiol_bundle__model_ranked_to_physiol_bundle}%
\hypertarget{ref_model_ranked_to_physiol_bundle__model_ranked_to_physiol_bundle}{}%
\begin{description}
\item[Summary:]A DB bundled with its dataset, ranked to a physiology DB bundle.
%
\item[Usage:]~%
\begin{lyxcode}%
r\_bundle = model\_ranked\_to\_physiol\_bundle(a\_dataset, a\_db, a\_ranked\_db, a\_crit\_bundle, props)
%
\end{lyxcode}%
%
\item[Description:]%
This is a subclass of model\_ct\_bundle, specialized for model datasets. 
%%
\item[Parameters:]~
\begin{description}%
\item[\texttt{a\_dataset}:]
 A params\_cip\_trace\_fileset object.
\item[\texttt{a\_db}:]
 The raw params\_tests\_db object created from the dataset. It only needs

to have the pAcip, trial, and ItemIndex columns.
\item[\texttt{a\_ranked\_db}:]
 The one-model-per-line DB created from the raw DB.
\item[\texttt{a\_crit\_bundle}:]
 The bundle object associated with crit\_db that caused the ranking in a\_ranked\_db.
\item[\texttt{props}:]
 A structure with any optional properties.
\end{description}%
%
\item[Returns a structure object with the following fields:
]~

	crit\_bundle, model\_ct\_bundle.
%
%
\item[See also:]%
\hyperlink{ref_model_ct_bundle}{\texttt{model\_ct\_bundle}}%
\ (p.~\pageref{ref_model_ct_bundle})%
\index[funcref]{model_ct_bundle@\fidxl{model\_ct\_bundle}}%
, \hyperlink{ref_ranked_db}{\texttt{ranked\_db}}%
\ (p.~\pageref{ref_ranked_db})%
\index[funcref]{ranked_db@\fidxl{ranked\_db}}%
, \hyperlink{ref_params_tests_dataset}{\texttt{params\_tests\_dataset}}%
\ (p.~\pageref{ref_params_tests_dataset})%
\index[funcref]{params_tests_dataset@\fidxl{params\_tests\_dataset}}%
%
\item[Author:]%
Cengiz Gunay <cgunay@emory.edu>, 2005/12/13
%
\end{description}
\methodline%
\subsubsection[Method \texttt{comparisonReport}]{Method \texttt{model\_ranked\_to\_physiol\_bundle/comparisonReport}}%
\index[funcref]{model_ranked_to_physiol_bundle@\fidxl{model\_ranked\_to\_physiol\_bundle}!comparisonReport@\fidxl{comparisonReport}}%
\label{ref_model_ranked_to_physiol_bundle__comparisonReport}%
\hypertarget{ref_model_ranked_to_physiol_bundle__comparisonReport}{}%
\begin{description}
\item[Summary:]OBSOLETE - Generates a report by comparing r\_bundle with the given match criteria, crit\_db from crit\_bundle.
%
\item[Usage:]~%
\begin{lyxcode}%
a\_doc\_multi = comparisonReport(r\_bundle, crit\_bundle, crit\_db, props)
%
\end{lyxcode}%
%
\item[Description:]%
Generates a LaTeX document with:
	- (optional) Raw traces compared with some best matches at different distances
	- Values of some top matching a\_db rows and match errors in a floating table.
	- colored-plot of measure errors for some top matches.
	- Parameter distributions of 50 best matches as a bar graph.
%%
\item[Parameters:]~
\begin{description}%
\item[\texttt{r\_bundle}:]
 A dataset\_db\_bundle object that contains the DB to compare rows from.
\item[\texttt{crit\_bundle}:]
 A dataset\_db\_bundle object that contains the criterion dataset.
\item[\texttt{crit\_db}:]
 A tests\_db object holding the match criterion tests and STDs

which can be created with matchingRow.
\item[\texttt{props}:]
 A structure with any optional properties.
\begin{description}%
\item[\texttt{caption}:]
 Identification of the criterion db (not needed/used?).
\item[\texttt{num\_matches}:]
 Number of best matches to display (default=10).
\item[\texttt{rotate}:]
 Rotation angle for best matches table (default=90).
\end{description}%
\end{description}%
%
\item[Returns:
]~

	tex\_string: LaTeX document string.
%
%
\item[See also:]%
\hyperlink{ref_displayRowsTeX}{\texttt{displayRowsTeX}}%
\ (p.~\pageref{ref_displayRowsTeX})%
\index[funcref]{displayRowsTeX@\fidxl{displayRowsTeX}}%
%
\item[Author:]%
Cengiz Gunay <cgunay@emory.edu>, 2006/01/17
%
\end{description}
\methodline%
\subsubsection[Method \texttt{plotCompareRanks}]{Method \texttt{model\_ranked\_to\_physiol\_bundle/plotCompareRanks}}%
\index[funcref]{model_ranked_to_physiol_bundle@\fidxl{model\_ranked\_to\_physiol\_bundle}!plotCompareRanks@\fidxl{plotCompareRanks}}%
\label{ref_model_ranked_to_physiol_bundle__plotCompareRanks}%
\hypertarget{ref_model_ranked_to_physiol_bundle__plotCompareRanks}{}%
\begin{description}
\item[Summary:]OBSOLETE - Generates a plots of given ranks from the ranked\_bundle.
%
\item[Usage:]~%
\begin{lyxcode}%
plots = plotCompareRanks(r\_bundle, crit\_bundle, crit\_db, props)
%
\end{lyxcode}%
%
%
\item[Parameters:]~
\begin{description}%
\item[\texttt{r\_bundle}:]
 A ranked\_bundle object.
\item[\texttt{ranks}:]
 Vector of rank indices for which to generate the plots.
\item[\texttt{props}:]
 A structure with any optional properties.
\end{description}%
%
\item[Returns:
]~

	plots: A structure that contains the joined\_db, and the plot vectors 
	  trace\_d100\_plots and trace\_h100\_plots.
%
\item[Example:]~
\begin{lyxcode} >> plots = plotCompareRanks(r, 1:10);
\\%
 >> plotFigure(plots.trace\_d100\_plots(1), 'The best matching +100 pA CIP trace');
\\%
\end{lyxcode}
%
\item[See also:]%
%
\item[Author:]%
Cengiz Gunay <cgunay@emory.edu>, 2006/01/16
%
\end{description}
\methodline%
\subsubsection[Method \texttt{plotfICurve}]{Method \texttt{model\_ranked\_to\_physiol\_bundle/plotfICurve}}%
\index[funcref]{model_ranked_to_physiol_bundle@\fidxl{model\_ranked\_to\_physiol\_bundle}!plotfICurve@\fidxl{plotfICurve}}%
\label{ref_model_ranked_to_physiol_bundle__plotfICurve}%
\hypertarget{ref_model_ranked_to_physiol_bundle__plotfICurve}{}%
\begin{description}
%
\item[Usage:]~%
\begin{lyxcode}%
a\_doc = docfICurve(r\_bundle, crit\_bundle, crit\_db, props)
%
\end{lyxcode}%
%
%
\item[Parameters:]~
\begin{description}%
\item[\texttt{r\_bundle}:]
 A ranked\_bundle object.
\item[\texttt{rank\_num}:]
 Rank index for which to generate the a\_doc.
\item[\texttt{props}:]
 A structure with any optional properties.
\end{description}%
%
\item[Returns:
]~

	a\_doc: A doc\_plot that contains a f-I curve plot and associated captions.
%
\item[Example:]~
\begin{lyxcode} >> a\_d = docfICurve(r, 1);
\\%
 >> plot(a\_d, 'The f-I curve of best matching model');
\\%
\end{lyxcode}
%
\item[See also:]%
\hyperlink{ref_doc_generate}{\texttt{doc\_generate}}%
\ (p.~\pageref{ref_doc_generate})%
\index[funcref]{doc_generate@\fidxl{doc\_generate}}%
, \hyperlink{ref_doc_plot}{\texttt{doc\_plot}}%
\ (p.~\pageref{ref_doc_plot})%
\index[funcref]{doc_plot@\fidxl{doc\_plot}}%
%
\item[Author:]%
Cengiz Gunay <cgunay@emory.edu>, 2006/01/16
%
\end{description}
\methodline%
\subsection{Class \texttt{params\_cip\_trace\_fileset}}%
\index[funcref]{params_cip_trace_fileset@\fidxl{params\_cip\_trace\_fileset}|boldhyperpage}%
\label{ref_params_cip_trace_fileset}%
\hypertarget{ref_params_cip_trace_fileset}{}%
\subsubsection[Constructor \texttt{params\_cip\_trace\_fileset}]{Constructor \texttt{params\_cip\_trace\_fileset/params\_cip\_trace\_fileset}}%
\index[funcref]{params_cip_trace_fileset@\fidxl{params\_cip\_trace\_fileset}!params_cip_trace_fileset@\fidxl{params\_cip\_trace\_fileset}}%
\label{ref_params_cip_trace_fileset__params_cip_trace_fileset}%
\hypertarget{ref_params_cip_trace_fileset__params_cip_trace_fileset}{}%
\begin{description}
\item[Summary:]Description of a raw dataset consisting of cip\_trace files varying 
	with parameter values.
%
\item[Usage:]~%
\begin{lyxcode}%
obj = params\_cip\_trace\_fileset(file\_pattern, dt, dy, 
				 pulse\_time\_start, pulse\_time\_width, id, props)
%
\end{lyxcode}%
%
\item[Description:]%
This is a subclass of params\_tests\_fileset.
%%
\item[Parameters:]~
\begin{description}%
\item[\texttt{file\_pattern}:]
 File pattern mathing all files to be loaded.
\item[\texttt{dt}:]
 Time resolution [s]
\item[\texttt{dy}:]
 y-axis resolution [ISI (V, A, etc.)]
\item[\texttt{pulse\_time\_start, pulse\_time\_width}:]


Start and width of the pulse [dt]
\item[\texttt{id}:]
 An identification string
\item[\texttt{props}:]
 A structure with any optional properties.
\begin{description}%
\item[\texttt{profile\_method\_name}:]
 Use this profile method that takes a

cip\_trace object and returns a results\_profile class. It
can be 'cip\_trace\_profile' (default, but outdated) or
'getProfileAllSpikes' (more current).
(All other props are passed to cip\_trace objects)
\end{description}%
\end{description}%
%
\item[Returns a structure object with the following fields:
]~

	params\_tests\_fileset,
	pulse\_time\_start, pulse\_time\_width.
%
\item[Example:]~
\begin{lyxcode}        >> fileset = params\_cip\_trace\_fileset('/home/abc/data/*.bin', 1e-4, 1e-3, 20001, 10000, 'sim dataset gpsc0501', struct('trace\_time\_start', 10001, 'type', 'sim', 'scale\_y', 1e3))
\\%
\end{lyxcode}
%
\item[See also:]%
\hyperlink{ref_params_tests_fileset}{\texttt{params\_tests\_fileset}}%
\ (p.~\pageref{ref_params_tests_fileset})%
\index[funcref]{params_tests_fileset@\fidxl{params\_tests\_fileset}}%
, \hyperlink{ref_params_tests_db}{\texttt{params\_tests\_db}}%
\ (p.~\pageref{ref_params_tests_db})%
\index[funcref]{params_tests_db@\fidxl{params\_tests\_db}}%
%
\item[Author:]%
Cengiz Gunay <cgunay@emory.edu>, 2004/09/14
%
\end{description}
\methodline%
\subsubsection[Method \texttt{cip\_trace}]{Method \texttt{params\_cip\_trace\_fileset/cip\_trace}}%
\index[funcref]{params_cip_trace_fileset@\fidxl{params\_cip\_trace\_fileset}!cip_trace@\fidxl{cip\_trace}}%
\label{ref_params_cip_trace_fileset__cip_trace}%
\hypertarget{ref_params_cip_trace_fileset__cip_trace}{}%
\begin{description}
\item[Summary:]Loads raw cip\_traces for each given file\_index in this fileset.
%
\item[Usage:]~%
\begin{lyxcode}%
a\_cip\_trace = cip\_trace(fileset, file\_index|a\_db, props)
%
\end{lyxcode}%
%
%
\item[Parameters:]~
\begin{description}%
\item[\texttt{fileset}:]
 A params\_tests\_fileset.
\item[\texttt{file\_index}:]
 A single or array of indices of files in fileset.
\item[\texttt{a\_db}:]
 A DB created by this fileset to read the item indices from.
\item[\texttt{props}:]
 A structure with any optional properties.
\begin{description}%
\item[\texttt{neuronLabel}:]
 Used for annotation purposes.
\end{description}%
\end{description}%
%
\item[Returns:
]~

	a\_cip\_trace: A cip\_trace object.
%
%
\item[See also:]%
\hyperlink{ref_cip_trace}{\texttt{cip\_trace}}%
\ (p.~\pageref{ref_cip_trace})%
\index[funcref]{cip_trace@\fidxl{cip\_trace}}%
, \hyperlink{ref_params_tests_fileset}{\texttt{params\_tests\_fileset}}%
\ (p.~\pageref{ref_params_tests_fileset})%
\index[funcref]{params_tests_fileset@\fidxl{params\_tests\_fileset}}%
%
\item[Author:]%
Cengiz Gunay <cgunay@emory.edu>, 2004/09/13
%
\end{description}
\methodline%
\subsubsection[Method \texttt{cip\_trace\_profile}]{Method \texttt{params\_cip\_trace\_fileset/cip\_trace\_profile}}%
\index[funcref]{params_cip_trace_fileset@\fidxl{params\_cip\_trace\_fileset}!cip_trace_profile@\fidxl{cip\_trace\_profile}}%
\label{ref_params_cip_trace_fileset__cip_trace_profile}%
\hypertarget{ref_params_cip_trace_fileset__cip_trace_profile}{}%
\begin{description}
\item[Summary:]Loads a raw cip\_trace\_profile given a file\_index 
		      to this fileset.
%
\item[Usage:]~%
\begin{lyxcode}%
a\_cip\_trace\_profile = cip\_trace\_profile(fileset, file\_index)
%
\end{lyxcode}%
%
%
\item[Parameters:]~
\begin{description}%
\item[\texttt{fileset}:]
 A params\_tests\_fileset.
\item[\texttt{file\_index}:]
 Index of file in fileset.
\end{description}%
%
\item[Returns:
]~

	a\_cip\_trace\_profile: A cip\_trace\_profile object.
%
%
\item[See also:]%
\hyperlink{ref_cip_trace_profile}{\texttt{cip\_trace\_profile}}%
\ (p.~\pageref{ref_cip_trace_profile})%
\index[funcref]{cip_trace_profile@\fidxl{cip\_trace\_profile}}%
, \hyperlink{ref_params_tests_fileset}{\texttt{params\_tests\_fileset}}%
\ (p.~\pageref{ref_params_tests_fileset})%
\index[funcref]{params_tests_fileset@\fidxl{params\_tests\_fileset}}%
%
\item[Author:]%
Cengiz Gunay <cgunay@emory.edu>, 2004/09/14
%
\end{description}
\methodline%
\subsubsection[Method \texttt{ctFromRows}]{Method \texttt{params\_cip\_trace\_fileset/ctFromRows}}%
\index[funcref]{params_cip_trace_fileset@\fidxl{params\_cip\_trace\_fileset}!ctFromRows@\fidxl{ctFromRows}}%
\label{ref_params_cip_trace_fileset__ctFromRows}%
\hypertarget{ref_params_cip_trace_fileset__ctFromRows}{}%
\begin{description}
\item[Summary:]Loads a cip\_trace object from raw data files in the fileset.
%
\item[Usage:]~%
\begin{lyxcode}%
a\_cip\_trace = ctFromRows(m\_fileset, m\_dball, a\_db|itemIndices, cip\_levels, props)
%
\end{lyxcode}%
%
%
\item[Parameters:]~
\begin{description}%
\item[\texttt{m\_fileset}:]
 A physiol\_cip\_traceset\_fileset object.
\item[\texttt{m\_dball}:]
 A DB created by this fileset that contains the trial, pAcip, and ItemIndex cols.
\item[\texttt{a\_db}:]
 A DB that has one trial for each cip\_trace to be loaded.
\item[\texttt{itemIndices}:]
 A column vector with ItemIndex numbers.
\item[\texttt{cip\_levels}:]
 A column vector of CIP-levels to be loaded.
\item[\texttt{props}:]
 A structure with any optional properties.
\begin{description}%
\item[\texttt{neuronLabel}:]
 appropriate unique neuron label generated by the bundle.

(passed to params\_cip\_trace\_fileset/cip\_trace)
\end{description}%
\end{description}%
%
\item[Returns:
]~

	a\_cip\_trace: One or more cip\_trace objects that hold the raw data.
%
%
\item[See also:]%
\hyperlink{ref_loadItemProfile}{\texttt{loadItemProfile}}%
\ (p.~\pageref{ref_loadItemProfile})%
\index[funcref]{loadItemProfile@\fidxl{loadItemProfile}}%
, \hyperlink{ref_physiol_cip_traceset__cip_trace}{\texttt{physiol\_cip\_traceset/cip\_trace}}%
\ (p.~\pageref{ref_physiol_cip_traceset__cip_trace})%
\index[funcref]{physiol_cip_traceset@\fidxl{physiol\_cip\_traceset}!cip_trace@\fidxl{cip\_trace}}%
%
\item[Author:]%
Cengiz Gunay <cgunay@emory.edu>, 2005/07/13
%
\end{description}
\methodline%
\subsubsection[Method \texttt{display}]{Method \texttt{params\_cip\_trace\_fileset/display}}%
\index[funcref]{params_cip_trace_fileset@\fidxl{params\_cip\_trace\_fileset}!display@\fidxl{display}}%
\label{ref_params_cip_trace_fileset__display}%
\hypertarget{ref_params_cip_trace_fileset__display}{}%
\begin{description}
%
%
%
%
%
%
%
\item[Author:]%
Cengiz Gunay <cgunay@emory.edu>, 2004/08/04
%
\end{description}
\methodline%
\subsubsection[Method \texttt{get}]{Method \texttt{params\_cip\_trace\_fileset/get}}%
\index[funcref]{params_cip_trace_fileset@\fidxl{params\_cip\_trace\_fileset}!get@\fidxl{get}}%
\label{ref_params_cip_trace_fileset__get}%
\hypertarget{ref_params_cip_trace_fileset__get}{}%
\begin{description}
\item[Summary:]Defines generic attribute retrieval for objects.
%
%
%
%
%
%
%
\item[Author:]%
Cengiz Gunay <cgunay@emory.edu>, 2004/09/14
%
\end{description}
\methodline%
\subsubsection[Method \texttt{loadItemProfile}]{Method \texttt{params\_cip\_trace\_fileset/loadItemProfile}}%
\index[funcref]{params_cip_trace_fileset@\fidxl{params\_cip\_trace\_fileset}!loadItemProfile@\fidxl{loadItemProfile}}%
\label{ref_params_cip_trace_fileset__loadItemProfile}%
\hypertarget{ref_params_cip_trace_fileset__loadItemProfile}{}%
\begin{description}
\item[Summary:]Loads a cip\_trace\_profile object from a raw data file in the fileset.
%
\item[Usage:]~%
\begin{lyxcode}%
[params\_row, tests\_row] = loadItemProfile(fileset, file\_index)
%
\end{lyxcode}%
%
%
\item[Parameters:]~
\begin{description}%
\item[\texttt{fileset}:]
 A params\_tests\_fileset.
\item[\texttt{file\_index}:]
 Index of file in fileset.
\end{description}%
%
\item[Returns:
]~

	a\_profile: A profile object that implements the getResults method.
%
%
\item[See also:]%
\hyperlink{ref_itemResultsRow}{\texttt{itemResultsRow}}%
\ (p.~\pageref{ref_itemResultsRow})%
\index[funcref]{itemResultsRow@\fidxl{itemResultsRow}}%
, \hyperlink{ref_params_tests_fileset}{\texttt{params\_tests\_fileset}}%
\ (p.~\pageref{ref_params_tests_fileset})%
\index[funcref]{params_tests_fileset@\fidxl{params\_tests\_fileset}}%
, \hyperlink{ref_paramNames}{\texttt{paramNames}}%
\ (p.~\pageref{ref_paramNames})%
\index[funcref]{paramNames@\fidxl{paramNames}}%
, \hyperlink{ref_testNames}{\texttt{testNames}}%
\ (p.~\pageref{ref_testNames})%
\index[funcref]{testNames@\fidxl{testNames}}%
%
\item[Author:]%
Cengiz Gunay <cgunay@emory.edu>, 2004/09/14
%
\end{description}
\methodline%
\subsubsection[Method \texttt{set}]{Method \texttt{params\_cip\_trace\_fileset/set}}%
\index[funcref]{params_cip_trace_fileset@\fidxl{params\_cip\_trace\_fileset}!set@\fidxl{set}}%
\label{ref_params_cip_trace_fileset__set}%
\hypertarget{ref_params_cip_trace_fileset__set}{}%
\begin{description}
\item[Summary:]Generic method for setting object attributes.
%
%
%
%
%
%
%
\item[Author:]%
Cengiz Gunay <cgunay@emory.edu>, 2004/10/08
%
\end{description}
\methodline%
\subsection{Class \texttt{params\_results\_profile}}%
\index[funcref]{params_results_profile@\fidxl{params\_results\_profile}|boldhyperpage}%
\label{ref_params_results_profile}%
\hypertarget{ref_params_results_profile}{}%
\subsubsection[Constructor \texttt{params\_results\_profile}]{Constructor \texttt{params\_results\_profile/params\_results\_profile}}%
\index[funcref]{params_results_profile@\fidxl{params\_results\_profile}!params_results_profile@\fidxl{params\_results\_profile}}%
\label{ref_params_results_profile__params_results_profile}%
\hypertarget{ref_params_results_profile__params_results_profile}{}%
\begin{description}
\item[Summary:]Profile with parameters and results together.
%
%
\item[Description:]%
This is a subclass of results\_profile, improved by including
 parameter names and values. Should make it easier to code dataset
 classes. Usage 1 is for convenience, same information is contained in
 results\_obj in Usage 2.
%%
\item[Parameters:]~
\begin{description}%
\item[\texttt{params}:]
 Structure with parameter names and values.
\item[\texttt{results}:]
 Structure with result names and values (Usage 1).
\item[\texttt{results\_obj}:]
 A results\_profile object with test results.
\item[\texttt{id}:]
 Identification string (Usage 1).
\item[\texttt{props}:]
 A structure with any optional properties (Usage 1).
\end{description}%
%
\item[Returns a structure object with the following fields:
]~

   params, results (results\_obj above)
%
%
\item[See also:]%
\hyperlink{ref_results_profile}{\texttt{results\_profile}}%
\ (p.~\pageref{ref_results_profile})%
\index[funcref]{results_profile@\fidxl{results\_profile}}%
, \hyperlink{ref_params_tests_dataset}{\texttt{params\_tests\_dataset}}%
\ (p.~\pageref{ref_params_tests_dataset})%
\index[funcref]{params_tests_dataset@\fidxl{params\_tests\_dataset}}%
%
\item[Author:]%
Cengiz Gunay <cgunay@emory.edu>, 2011/07/05
%
\end{description}
\methodline%
\subsubsection[Method \texttt{get}]{Method \texttt{params\_results\_profile/get}}%
\index[funcref]{params_results_profile@\fidxl{params\_results\_profile}!get@\fidxl{get}}%
\label{ref_params_results_profile__get}%
\hypertarget{ref_params_results_profile__get}{}%
\begin{description}
\item[Summary:]Defines generic attribute retrieval for objects.
%
%
%
%
%
%
%
\item[Author:]%
Cengiz Gunay <cgunay@emory.edu>, 2004/09/14
%
\end{description}
\methodline%
\subsection{Class \texttt{params\_tests\_dataset}}%
\index[funcref]{params_tests_dataset@\fidxl{params\_tests\_dataset}|boldhyperpage}%
\label{ref_params_tests_dataset}%
\hypertarget{ref_params_tests_dataset}{}%
\subsubsection[Constructor \texttt{params\_tests\_dataset}]{Constructor \texttt{params\_tests\_dataset/params\_tests\_dataset}}%
\index[funcref]{params_tests_dataset@\fidxl{params\_tests\_dataset}!params_tests_dataset@\fidxl{params\_tests\_dataset}}%
\label{ref_params_tests_dataset__params_tests_dataset}%
\hypertarget{ref_params_tests_dataset__params_tests_dataset}{}%
\begin{description}
\item[Summary:]Contains a set of data objects or files of raw data varying with parameter values.
%
\item[Usage:]~%
\begin{lyxcode}%
obj = params\_tests\_dataset(list, dt, dy, id, props)
%
\end{lyxcode}%
%
\item[Description:]%
This is an abstract base class for keeping dataset information separate
 from the parameters-results database (params\_tests\_db). The list contents
 can be filenames or objects (such as cip\_traces) from which to get the raw data.
 The dataset should have all the necessary information to create a db when
 needed. This is an abstract class, thet it it cannot act on its own. Only 
 fully implemented subclasses can actually hold datasets. See methods below.
%%
\item[Parameters:]~
\begin{description}%
\item[\texttt{list}:]
 Array of dataset items (filenames, objects, etc.).
\item[\texttt{dt}:]
 Time resolution [s]
\item[\texttt{dy}:]
 y-axis resolution [integral V, A, etc.]
\item[\texttt{id}:]
 An identification string.
\item[\texttt{props}:]
 A structure with any optional properties.
\begin{description}%
\item[\texttt{type}:]
 type of file (default = '')
\item[\texttt{loadItemProfileFunc}:]
 Function name or handle to be called as with

(dataset, index, param\_row, props) to load a dataset item during database
creation and return a results\_profile. Changing this property allows creating
different databases from same dataset. It also
allows loading a novel dataset through this generic class.
\end{description}%
\end{description}%
%
\item[Returns a structure object with the following fields:
]~

	list, dt, dy, id, props (see above).
%
%
\item[See also:]%
\hyperlink{ref_params_tests_db}{\texttt{params\_tests\_db}}%
\ (p.~\pageref{ref_params_tests_db})%
\index[funcref]{params_tests_db@\fidxl{params\_tests\_db}}%
, \hyperlink{ref_params_tests_fileset}{\texttt{params\_tests\_fileset}}%
\ (p.~\pageref{ref_params_tests_fileset})%
\index[funcref]{params_tests_fileset@\fidxl{params\_tests\_fileset}}%
, \hyperlink{ref_cip_traces_dataset}{\texttt{cip\_traces\_dataset}}%
\ (p.~\pageref{ref_cip_traces_dataset})%
\index[funcref]{cip_traces_dataset@\fidxl{cip\_traces\_dataset}}%
%
\item[Author:]%
Cengiz Gunay <cgunay@emory.edu>, 2004/12/02
%
\end{description}
\methodline%
\subsubsection[Method \texttt{addItem}]{Method \texttt{params\_tests\_dataset/addItem}}%
\index[funcref]{params_tests_dataset@\fidxl{params\_tests\_dataset}!addItem@\fidxl{addItem}}%
\label{ref_params_tests_dataset__addItem}%
\hypertarget{ref_params_tests_dataset__addItem}{}%
\begin{description}
\item[Summary:]Returns the new dataset with the added item.
%
\item[Usage:]~%
\begin{lyxcode}%
dataset = addItem(dataset, item)
%
\end{lyxcode}%
%
\item[Description:]%
Note that, this is NOT the way to create a dataset. It is only intended for 
 small additions to an existing dataset. This method is too slow
 for creating large datasets. The normal method for creating datasets is
 providing the full list of items to the class constructor.
%%
\item[Parameters:]~
\begin{description}%
\item[\texttt{dataset}:]
 A params\_tests\_dataset.
\item[\texttt{item}:]
 New item to add in dataset.
\end{description}%
%
\item[Returns:
]~

	dataset: With the added item.
%
%
\item[See also:]%
\hyperlink{ref_itemResultsRow}{\texttt{itemResultsRow}}%
\ (p.~\pageref{ref_itemResultsRow})%
\index[funcref]{itemResultsRow@\fidxl{itemResultsRow}}%
, \hyperlink{ref_params_tests_dataset}{\texttt{params\_tests\_dataset}}%
\ (p.~\pageref{ref_params_tests_dataset})%
\index[funcref]{params_tests_dataset@\fidxl{params\_tests\_dataset}}%
, \hyperlink{ref_paramNames}{\texttt{paramNames}}%
\ (p.~\pageref{ref_paramNames})%
\index[funcref]{paramNames@\fidxl{paramNames}}%
, \hyperlink{ref_testNames}{\texttt{testNames}}%
\ (p.~\pageref{ref_testNames})%
\index[funcref]{testNames@\fidxl{testNames}}%
%
\item[Author:]%
Cengiz Gunay <cgunay@emory.edu>, 2005/01/25
%
\end{description}
\methodline%
\subsubsection[Method \texttt{display}]{Method \texttt{params\_tests\_dataset/display}}%
\index[funcref]{params_tests_dataset@\fidxl{params\_tests\_dataset}!display@\fidxl{display}}%
\label{ref_params_tests_dataset__display}%
\hypertarget{ref_params_tests_dataset__display}{}%
\begin{description}
%
%
%
%
%
%
%
\item[Author:]%
Cengiz Gunay <cgunay@emory.edu>, 2004/08/04
%
\end{description}
\methodline%
\subsubsection[Method \texttt{get}]{Method \texttt{params\_tests\_dataset/get}}%
\index[funcref]{params_tests_dataset@\fidxl{params\_tests\_dataset}!get@\fidxl{get}}%
\label{ref_params_tests_dataset__get}%
\hypertarget{ref_params_tests_dataset__get}{}%
\begin{description}
\item[Summary:]Defines generic attribute retrieval for objects.
%
%
%
%
%
%
%
\item[Author:]%
Cengiz Gunay <cgunay@emory.edu>, 2004/09/14
%
\end{description}
\methodline%
\subsubsection[Method \texttt{getItem}]{Method \texttt{params\_tests\_dataset/getItem}}%
\index[funcref]{params_tests_dataset@\fidxl{params\_tests\_dataset}!getItem@\fidxl{getItem}}%
\label{ref_params_tests_dataset__getItem}%
\hypertarget{ref_params_tests_dataset__getItem}{}%
\begin{description}
\item[Summary:]Returns the dataset item at given index.
%
\item[Usage:]~%
\begin{lyxcode}%
item = getItem(dataset, index)
%
\end{lyxcode}%
%
%
\item[Parameters:]~
\begin{description}%
\item[\texttt{dataset}:]
 A params\_tests\_dataset.
\item[\texttt{index}:]
 Index of item in dataset.
\end{description}%
%
\item[Returns:
]~

	item: Object, filename, etc.
%
%
\item[See also:]%
\hyperlink{ref_itemResultsRow}{\texttt{itemResultsRow}}%
\ (p.~\pageref{ref_itemResultsRow})%
\index[funcref]{itemResultsRow@\fidxl{itemResultsRow}}%
, \hyperlink{ref_params_tests_dataset}{\texttt{params\_tests\_dataset}}%
\ (p.~\pageref{ref_params_tests_dataset})%
\index[funcref]{params_tests_dataset@\fidxl{params\_tests\_dataset}}%
, \hyperlink{ref_paramNames}{\texttt{paramNames}}%
\ (p.~\pageref{ref_paramNames})%
\index[funcref]{paramNames@\fidxl{paramNames}}%
, \hyperlink{ref_testNames}{\texttt{testNames}}%
\ (p.~\pageref{ref_testNames})%
\index[funcref]{testNames@\fidxl{testNames}}%
%
\item[Author:]%
Cengiz Gunay <cgunay@emory.edu>, 2004/12/03
%
\end{description}
\methodline%
\subsubsection[Method \texttt{getItemParams}]{Method \texttt{params\_tests\_dataset/getItemParams}}%
\index[funcref]{params_tests_dataset@\fidxl{params\_tests\_dataset}!getItemParams@\fidxl{getItemParams}}%
\label{ref_params_tests_dataset__getItemParams}%
\hypertarget{ref_params_tests_dataset__getItemParams}{}%
\begin{description}
\item[Summary:]Get the parameter values of a dataset item.
%
\item[Usage:]~%
\begin{lyxcode}%
params\_row = getItemParams(dataset, index, a\_profile)
%
\end{lyxcode}%
%
\item[Description:]%
This method can retrieve the item parameters by using either the 
 dataset and the index to find the item or simply by using
 the item profile, a\_profile.
%%
\item[Parameters:]~
\begin{description}%
\item[\texttt{dataset}:]
 A params\_tests\_dataset.
\item[\texttt{index}:]
 Index of item in dataset.
\item[\texttt{a\_profile}:]
 An item profile.
\end{description}%
%
\item[Returns:
]~

	params\_row: Parameter values in the same order of paramNames
%
%
\item[See also:]%
\hyperlink{ref_itemResultsRow}{\texttt{itemResultsRow}}%
\ (p.~\pageref{ref_itemResultsRow})%
\index[funcref]{itemResultsRow@\fidxl{itemResultsRow}}%
, \hyperlink{ref_params_tests_dataset}{\texttt{params\_tests\_dataset}}%
\ (p.~\pageref{ref_params_tests_dataset})%
\index[funcref]{params_tests_dataset@\fidxl{params\_tests\_dataset}}%
, \hyperlink{ref_paramNames}{\texttt{paramNames}}%
\ (p.~\pageref{ref_paramNames})%
\index[funcref]{paramNames@\fidxl{paramNames}}%
, \hyperlink{ref_testNames}{\texttt{testNames}}%
\ (p.~\pageref{ref_testNames})%
\index[funcref]{testNames@\fidxl{testNames}}%
%
\item[Author:]%
Cengiz Gunay <cgunay@emory.edu>, 2004/09/10
%
\end{description}
\methodline%
\subsubsection[Method \texttt{itemResultsRow}]{Method \texttt{params\_tests\_dataset/itemResultsRow}}%
\index[funcref]{params_tests_dataset@\fidxl{params\_tests\_dataset}!itemResultsRow@\fidxl{itemResultsRow}}%
\label{ref_params_tests_dataset__itemResultsRow}%
\hypertarget{ref_params_tests_dataset__itemResultsRow}{}%
\begin{description}
\item[Summary:]Analyze data from the dataset and return its parameter and test values.
%
\item[Usage:]~%
\begin{lyxcode}%
[params\_row, tests\_row] = itemResultsRow(dataset, index)
%
\end{lyxcode}%
%
\item[Description:]%
This method is designed to be reused from subclasses as long as the
 loadItemProfile method is properly overloaded. Adds an Index
 column to the DB to keep track of raw data items after shuffling.
%%
\item[Parameters:]~
\begin{description}%
\item[\texttt{dataset}:]
 A params\_tests\_dataset.
\item[\texttt{index}:]
 Index of file in dataset.
\item[\texttt{props}:]
 A structure with any optional properties.

(passed to loadItemProfileFunc)
\end{description}%
%
\item[Returns:
]~

   params\_row: Parameter values in the same order of paramNames
   tests\_row: Test values in the same order with testNames
%
%
\item[See also:]%
\hyperlink{ref_loadItemProfile}{\texttt{loadItemProfile}}%
\ (p.~\pageref{ref_loadItemProfile})%
\index[funcref]{loadItemProfile@\fidxl{loadItemProfile}}%
, \hyperlink{ref_params_tests_dataset}{\texttt{params\_tests\_dataset}}%
\ (p.~\pageref{ref_params_tests_dataset})%
\index[funcref]{params_tests_dataset@\fidxl{params\_tests\_dataset}}%
, \hyperlink{ref_paramNames}{\texttt{paramNames}}%
\ (p.~\pageref{ref_paramNames})%
\index[funcref]{paramNames@\fidxl{paramNames}}%
, \hyperlink{ref_testNames}{\texttt{testNames}}%
\ (p.~\pageref{ref_testNames})%
\index[funcref]{testNames@\fidxl{testNames}}%
%
\item[Author:]%
Cengiz Gunay <cgunay@emory.edu>, 2004/09/10
%
\end{description}
\methodline%
\subsubsection[Method \texttt{loadItemProfile}]{Method \texttt{params\_tests\_dataset/loadItemProfile}}%
\index[funcref]{params_tests_dataset@\fidxl{params\_tests\_dataset}!loadItemProfile@\fidxl{loadItemProfile}}%
\label{ref_params_tests_dataset__loadItemProfile}%
\hypertarget{ref_params_tests_dataset__loadItemProfile}{}%
\begin{description}
\item[Summary:]Generates a results\_profile object from a dataset item.
%
\item[Usage:]~%
\begin{lyxcode}%
a\_profile = loadItemProfile(dataset, item\_index)
%
\end{lyxcode}%
%
\item[Description:]%
If getResults returns a params\_results\_profile, then implementing
 paramNames and getItemParams become unecessary.
%%
\item[Parameters:]~
\begin{description}%
\item[\texttt{dataset}:]
 A params\_tests\_dataset object.
\item[\texttt{item\_index}:]
 Index of item in dataset.
\item[\texttt{params\_row}:]
 Parameter values for this item (default=[])
\item[\texttt{props}:]
 Struct with optional properties.
\end{description}%
%
\item[Returns:
]~

   a\_profile: A profile object that implements the getResults method.
%
%
\item[See also:]%
\hyperlink{ref_itemResultsRow}{\texttt{itemResultsRow}}%
\ (p.~\pageref{ref_itemResultsRow})%
\index[funcref]{itemResultsRow@\fidxl{itemResultsRow}}%
, \hyperlink{ref_params_tests_fileset}{\texttt{params\_tests\_fileset}}%
\ (p.~\pageref{ref_params_tests_fileset})%
\index[funcref]{params_tests_fileset@\fidxl{params\_tests\_fileset}}%
, \hyperlink{ref_paramNames}{\texttt{paramNames}}%
\ (p.~\pageref{ref_paramNames})%
\index[funcref]{paramNames@\fidxl{paramNames}}%
, \hyperlink{ref_testNames}{\texttt{testNames}}%
\ (p.~\pageref{ref_testNames})%
\index[funcref]{testNames@\fidxl{testNames}}%
%
\item[Author:]%
Cengiz Gunay <cgunay@emory.edu>, 2011/07/05
%
\end{description}
\methodline%
\subsubsection[Method \texttt{params\_tests\_db}]{Method \texttt{params\_tests\_dataset/params\_tests\_db}}%
\index[funcref]{params_tests_dataset@\fidxl{params\_tests\_dataset}!params_tests_db@\fidxl{params\_tests\_db}}%
\label{ref_params_tests_dataset__params_tests_db}%
\hypertarget{ref_params_tests_dataset__params_tests_db}{}%
\begin{description}
\item[Summary:]Generates a params\_tests\_db object from the dataset.
%
\item[Usage:]~%
\begin{lyxcode}%
db\_obj = params\_tests\_db(obj, items, props)
%
\end{lyxcode}%
%
\item[Description:]%
This is a converter method to convert from params\_tests\_dataset to
 params\_tests\_db. Uses readDBItems to read the files.
 A customized subclass should provide the correct 
 paramNames, testNames, and itemResultsRow functions. Adds a ItemIndex
 column to the DB to keep track of raw data files after shuffling.
%%
\item[Parameters:]~
\begin{description}%
\item[\texttt{obj}:]
 A params\_tests\_dataset object.
\item[\texttt{items}:]
 (Optional) List of item indices to use to create the db.
\item[\texttt{props}:]
 Any optional params to pass to params\_tests\_db.
\end{description}%
%
\item[Returns:
]~

	db\_obj: A params\_tests\_db object.
%
%
\item[See also:]%
\hyperlink{ref_readDBItems}{\texttt{readDBItems}}%
\ (p.~\pageref{ref_readDBItems})%
\index[funcref]{readDBItems@\fidxl{readDBItems}}%
, \hyperlink{ref_params_tests_db}{\texttt{params\_tests\_db}}%
\ (p.~\pageref{ref_params_tests_db})%
\index[funcref]{params_tests_db@\fidxl{params\_tests\_db}}%
, \hyperlink{ref_params_tests_dataset}{\texttt{params\_tests\_dataset}}%
\ (p.~\pageref{ref_params_tests_dataset})%
\index[funcref]{params_tests_dataset@\fidxl{params\_tests\_dataset}}%
, \hyperlink{ref_itemResultsRow
	    testNames}{\texttt{itemResultsRow
	    testNames}}%
\ (p.~\pageref{ref_itemResultsRow
	    testNames})%
\index[funcref]{itemResultsRow
	    testNames@\fidxl{itemResultsRow
	    testNames}}%
, \hyperlink{ref_paramNames}{\texttt{paramNames}}%
\ (p.~\pageref{ref_paramNames})%
\index[funcref]{paramNames@\fidxl{paramNames}}%
%
\item[Author:]%
Cengiz Gunay <cgunay@emory.edu>, 2004/09/09
%
\end{description}
\methodline%
\subsubsection[Method \texttt{readDBItems}]{Method \texttt{params\_tests\_dataset/readDBItems}}%
\index[funcref]{params_tests_dataset@\fidxl{params\_tests\_dataset}!readDBItems@\fidxl{readDBItems}}%
\label{ref_params_tests_dataset__readDBItems}%
\hypertarget{ref_params_tests_dataset__readDBItems}{}%
\begin{description}
\item[Summary:]Reads all items to generate a params\_tests\_db object.
%
\item[Usage:]~%
\begin{lyxcode}%
[params, param\_names, tests, test\_names] = readDBItems(obj, items)
%
\end{lyxcode}%
%
\item[Description:]%
This is a generic method to convert from params\_tests\_fileset to
 a params\_tests\_db, or a subclass. This method depends on the  
 paramNames, testNames, and itemResultsRow functions. 
 Outputs of this function can be directly fed to the constructor of
 a params\_tests\_db or a subclass.
%%
\item[Parameters:]~
\begin{description}%
\item[\texttt{obj}:]
 A params\_tests\_fileset object.
\item[\texttt{items}:]
 (Optional) List of item indices to use to create the db.
\end{description}%
%
\item[Returns:
]~

	params, param\_names, tests, test\_names: See params\_tests\_db.
%
%
\item[See also:]%
\hyperlink{ref_params_tests_db}{\texttt{params\_tests\_db}}%
\ (p.~\pageref{ref_params_tests_db})%
\index[funcref]{params_tests_db@\fidxl{params\_tests\_db}}%
, \hyperlink{ref_params_tests_fileset}{\texttt{params\_tests\_fileset}}%
\ (p.~\pageref{ref_params_tests_fileset})%
\index[funcref]{params_tests_fileset@\fidxl{params\_tests\_fileset}}%
, \hyperlink{ref_itemResultsRow
	    testNames}{\texttt{itemResultsRow
	    testNames}}%
\ (p.~\pageref{ref_itemResultsRow
	    testNames})%
\index[funcref]{itemResultsRow
	    testNames@\fidxl{itemResultsRow
	    testNames}}%
, \hyperlink{ref_paramNames}{\texttt{paramNames}}%
\ (p.~\pageref{ref_paramNames})%
\index[funcref]{paramNames@\fidxl{paramNames}}%
%
\item[Author:]%
Cengiz Gunay <cgunay@emory.edu>, 2004/11/24
%
\end{description}
\methodline%
\subsubsection[Method \texttt{set}]{Method \texttt{params\_tests\_dataset/set}}%
\index[funcref]{params_tests_dataset@\fidxl{params\_tests\_dataset}!set@\fidxl{set}}%
\label{ref_params_tests_dataset__set}%
\hypertarget{ref_params_tests_dataset__set}{}%
\begin{description}
\item[Summary:]Generic method for setting object attributes.
%
%
%
%
%
%
%
\item[Author:]%
Cengiz Gunay <cgunay@emory.edu>, 2004/10/08
%
\end{description}
\methodline%
\subsubsection[Method \texttt{setProp}]{Method \texttt{params\_tests\_dataset/setProp}}%
\index[funcref]{params_tests_dataset@\fidxl{params\_tests\_dataset}!setProp@\fidxl{setProp}}%
\label{ref_params_tests_dataset__setProp}%
\hypertarget{ref_params_tests_dataset__setProp}{}%
\begin{description}
\item[Summary:]Generic method for setting optional object properties.
%
\item[Usage:]~%
\begin{lyxcode}%
obj = setProp(obj, prop1, val1, prop2, val2, ...)
%
\end{lyxcode}%
%
\item[Description:]%
Modifies or adds property values. As many property name-value 
 pairs can be specified.
%%
\item[Parameters:]~
\begin{description}%
\item[\texttt{obj}:]
 Any object that has a props field.
\item[\texttt{attr}:]
 Property name
\item[\texttt{val}:]
 Property value.
\end{description}%
%
\item[Returns:
]~

	obj: The new object with the updated properties.
%
%
\item[See also:]%
%
\item[Author:]%
Cengiz Gunay <cgunay@emory.edu>, 2004/11/22
%
\end{description}
\methodline%
\subsubsection[Method \texttt{subsasgn}]{Method \texttt{params\_tests\_dataset/subsasgn}}%
\index[funcref]{params_tests_dataset@\fidxl{params\_tests\_dataset}!subsasgn@\fidxl{subsasgn}}%
\label{ref_params_tests_dataset__subsasgn}%
\hypertarget{ref_params_tests_dataset__subsasgn}{}%
\begin{description}
\item[Summary:]Defines generic index-based assignment for objects.
%
%
%
%
%
%
%
\item[Author:]%
Cengiz Gunay <cgunay@emory.edu>, 2006/02/06
%
\end{description}
\methodline%
\subsubsection[Method \texttt{subsref}]{Method \texttt{params\_tests\_dataset/subsref}}%
\index[funcref]{params_tests_dataset@\fidxl{params\_tests\_dataset}!subsref@\fidxl{subsref}}%
\label{ref_params_tests_dataset__subsref}%
\hypertarget{ref_params_tests_dataset__subsref}{}%
\begin{description}
\item[Summary:]Defines generic indexing for objects.
%
%
%
%
%
%
%
%
\end{description}
\methodline%
\subsubsection[Method \texttt{testNames}]{Method \texttt{params\_tests\_dataset/testNames}}%
\index[funcref]{params_tests_dataset@\fidxl{params\_tests\_dataset}!testNames@\fidxl{testNames}}%
\label{ref_params_tests_dataset__testNames}%
\hypertarget{ref_params_tests_dataset__testNames}{}%
\begin{description}
\item[Summary:]Returns the ordered names of tests for this dataset.
%
\item[Usage:]~%
\begin{lyxcode}%
[test\_names a\_prof] = testNames(dataset, item)
%
\end{lyxcode}%
%
\item[Description:]%
Looks at the results of the first file to find the test names.
%%
\item[Parameters:]~
\begin{description}%
\item[\texttt{dataset}:]
 A params\_tests\_dataset.
\item[\texttt{item}:]
 (Optional) If given, read names by loading item at this index.
\end{description}%
%
\item[Returns:
]~

   test\_names: Cell array with ordered parameter names.
   a\_prof: Profile of the item read to find test names.
%
%
\item[See also:]%
\hyperlink{ref_params_tests_dataset}{\texttt{params\_tests\_dataset}}%
\ (p.~\pageref{ref_params_tests_dataset})%
\index[funcref]{params_tests_dataset@\fidxl{params\_tests\_dataset}}%
, \hyperlink{ref_paramNames}{\texttt{paramNames}}%
\ (p.~\pageref{ref_paramNames})%
\index[funcref]{paramNames@\fidxl{paramNames}}%
, \hyperlink{ref_testNames}{\texttt{testNames}}%
\ (p.~\pageref{ref_testNames})%
\index[funcref]{testNames@\fidxl{testNames}}%
%
\item[Author:]%
Cengiz Gunay <cgunay@emory.edu>, 2004/09/10
%
\end{description}
\methodline%
\subsection{Class \texttt{params\_tests\_db}}%
\index[funcref]{params_tests_db@\fidxl{params\_tests\_db}|boldhyperpage}%
\label{ref_params_tests_db}%
\hypertarget{ref_params_tests_db}{}%
\subsubsection[Constructor \texttt{params\_tests\_db}]{Constructor \texttt{params\_tests\_db/params\_tests\_db}}%
\index[funcref]{params_tests_db@\fidxl{params\_tests\_db}!params_tests_db@\fidxl{params\_tests\_db}}%
\label{ref_params_tests_db__params_tests_db}%
\hypertarget{ref_params_tests_db__params_tests_db}{}%
\begin{description}
\item[Summary:]A generic database of test results varying with parameter values, organized in a matrix format.
%
%
\item[Description:]%
This is a subclass of tests\_db. Defines all operations on this
 structure so that subclasses can use them.
%%
\item[Parameters:]~
\begin{description}%
\item[\texttt{num\_params}:]
 Number of parameters.
\item[\texttt{a\_tests\_db}:]
 A tests\_db upon which to build the params\_tests\_db.
\item[\texttt{props}:]
 A structure with any optional properties.
\end{description}%
%
\item[Returns a structure object with the following fields:
]~

	tests\_db
	num\_params: Number of variable parameters in simulations.
%
%
\item[See also:]%
\hyperlink{ref_tests_db}{\texttt{tests\_db}}%
\ (p.~\pageref{ref_tests_db})%
\index[funcref]{tests_db@\fidxl{tests\_db}}%
, \hyperlink{ref_test_variable_db (N__I)}{\texttt{test\_variable\_db (N/I)}}%
\ (p.~\pageref{ref_test_variable_db (N__I)})%
\index[funcref]{test_variable_db (N@\fidxl{test\_variable\_db (N}!I)@\fidxl{I)}}%
%
\item[Author:]%
Cengiz Gunay <cgunay@emory.edu>, 2004/09/08
%
\end{description}
\methodline%
\subsubsection[Method \texttt{addColumns}]{Method \texttt{params\_tests\_db/addColumns}}%
\index[funcref]{params_tests_db@\fidxl{params\_tests\_db}!addColumns@\fidxl{addColumns}}%
\label{ref_params_tests_db__addColumns}%
\hypertarget{ref_params_tests_db__addColumns}{}%
\begin{description}
\item[Summary:]Inserts new columns to tests\_db.
%
%
\item[Description:]%
Delegates to tests\_db/addColumns, but maintains parameter
 columns for the 2nd usage.
%%
\item[Parameters:]~
\begin{description}%
\item[\texttt{obj, b\_obj}:]
 A tests\_db object.
\item[\texttt{test\_names}:]
 A single string or a cell array of test names to be added.
\item[\texttt{test\_columns}:]
 Data matrix of columns to be added.
\end{description}%
%
\item[Returns:
]~

   obj: The tests\_db object that includes the new columns.
%
%
\item[See also:]%
\hyperlink{ref_tests_db__addColumns}{\texttt{tests\_db/addColumns}}%
\ (p.~\pageref{ref_tests_db__addColumns})%
\index[funcref]{tests_db@\fidxl{tests\_db}!addColumns@\fidxl{addColumns}}%
, \hyperlink{ref_allocateRows}{\texttt{allocateRows}}%
\ (p.~\pageref{ref_allocateRows})%
\index[funcref]{allocateRows@\fidxl{allocateRows}}%
, \hyperlink{ref_tests_db}{\texttt{tests\_db}}%
\ (p.~\pageref{ref_tests_db})%
\index[funcref]{tests_db@\fidxl{tests\_db}}%
%
\item[Author:]%
Cengiz Gunay <cgunay@emory.edu>, 2015/05/18
%
\end{description}
\methodline%
\subsubsection[Method \texttt{addParams}]{Method \texttt{params\_tests\_db/addParams}}%
\index[funcref]{params_tests_db@\fidxl{params\_tests\_db}!addParams@\fidxl{addParams}}%
\label{ref_params_tests_db__addParams}%
\hypertarget{ref_params_tests_db__addParams}{}%
\begin{description}
\item[Summary:]Inserts new parameter columns to tests\_db.
%
\item[Usage:]~%
\begin{lyxcode}%
obj = addParams(obj, param\_names, param\_columns)
%
\end{lyxcode}%
%
\item[Description:]%
Adds new columns to the database and returns the new DB.
   This operation is expensive in the sense that the whole database matrix
   needs to be enlarged just to add a 
   single new column. The method of allocating a matrix, filling it up, and
   then providing it to the tests\_db constructor is the preferred method 
   of creating tests\_db objects. This method may be used for 
   measures obtained by operating on raw measures.
%%
\item[Parameters:]~
\begin{description}%
\item[\texttt{obj}:]
 A tests\_db object.
\item[\texttt{param\_names}:]
 A cell array of param names to be added.
\item[\texttt{param\_columns}:]
 Data matrix of columns to be added.
\end{description}%
%
\item[Returns:
]~

	obj: The tests\_db object that includes the new columns.
%
%
\item[See also:]%
\hyperlink{ref_allocateRows}{\texttt{allocateRows}}%
\ (p.~\pageref{ref_allocateRows})%
\index[funcref]{allocateRows@\fidxl{allocateRows}}%
, \hyperlink{ref_tests_db}{\texttt{tests\_db}}%
\ (p.~\pageref{ref_tests_db})%
\index[funcref]{tests_db@\fidxl{tests\_db}}%
%
\item[Author:]%
Cengiz Gunay <cgunay@emory.edu>, 2005/10/11
%
\end{description}
\methodline%
\subsubsection[Method \texttt{crossProd}]{Method \texttt{params\_tests\_db/crossProd}}%
\index[funcref]{params_tests_db@\fidxl{params\_tests\_db}!crossProd@\fidxl{crossProd}}%
\label{ref_params_tests_db__crossProd}%
\hypertarget{ref_params_tests_db__crossProd}{}%
\begin{description}
\item[Summary:]Create a DB by taking the cross product of two database row sets.
%
\item[Usage:]~%
\begin{lyxcode}%
cross\_db = crossProd(a\_db, b\_db)
%
\end{lyxcode}%
%
\item[Description:]%
Overloaded function to maintain correct number of parameters after
 cross product operation. See original in tests\_db/crossProd.
%%
\item[Parameters:]~
\begin{description}%
\item[\texttt{a\_db, b\_db}:]
 A tests\_db object.
\end{description}%
%
\item[Returns:
]~

	cross\_db: The tests\_db object with all combinations of rows.
%
%
\item[See also:]%
\hyperlink{ref_tests_db__crossProd}{\texttt{tests\_db/crossProd}}%
\ (p.~\pageref{ref_tests_db__crossProd})%
\index[funcref]{tests_db@\fidxl{tests\_db}!crossProd@\fidxl{crossProd}}%
%
\item[Author:]%
Cengiz Gunay <cgunay@emory.edu>, 2005/10/11
%
\end{description}
\methodline%
\subsubsection[Method \texttt{delColumns}]{Method \texttt{params\_tests\_db/delColumns}}%
\index[funcref]{params_tests_db@\fidxl{params\_tests\_db}!delColumns@\fidxl{delColumns}}%
\label{ref_params_tests_db__delColumns}%
\hypertarget{ref_params_tests_db__delColumns}{}%
\begin{description}
\item[Summary:]Deletes columns from tests\_db.
%
\item[Usage:]~%
\begin{lyxcode}%
index = delColumns(obj, tests)
%
\end{lyxcode}%
%
\item[Description:]%
Overloaded function that maintains correct number of parameters. See
 original tests\_db/delColumns.
%%
\item[Parameters:]~
\begin{description}%
\item[\texttt{obj}:]
 A tests\_db object.
\item[\texttt{tests}:]
 Numbers or names of tests (see tests2cols)
\end{description}%
%
\item[Returns:
]~

	obj: The tests\_db object that is missing the columns.
%
%
\item[See also:]%
\hyperlink{ref_tests_db__delColumns}{\texttt{tests\_db/delColumns}}%
\ (p.~\pageref{ref_tests_db__delColumns})%
\index[funcref]{tests_db@\fidxl{tests\_db}!delColumns@\fidxl{delColumns}}%
%
\item[Author:]%
Cengiz Gunay <cgunay@emory.edu>, 2005/10/11
%
\end{description}
\methodline%
\subsubsection[Method \texttt{display}]{Method \texttt{params\_tests\_db/display}}%
\index[funcref]{params_tests_db@\fidxl{params\_tests\_db}!display@\fidxl{display}}%
\label{ref_params_tests_db__display}%
\hypertarget{ref_params_tests_db__display}{}%
\begin{description}
%
%
%
%
%
%
%
\item[Author:]%
Cengiz Gunay <cgunay@emory.edu>, 2004/08/04
%
\end{description}
\methodline%
\subsubsection[Method \texttt{displayRankingsTeX}]{Method \texttt{params\_tests\_db/displayRankingsTeX}}%
\index[funcref]{params_tests_db@\fidxl{params\_tests\_db}!displayRankingsTeX@\fidxl{displayRankingsTeX}}%
\label{ref_params_tests_db__displayRankingsTeX}%
\hypertarget{ref_params_tests_db__displayRankingsTeX}{}%
\begin{description}
\item[Summary:]Generates and displays a ranking DB by comparing rows of a\_db with the given match criteria.
%
\item[Usage:]~%
\begin{lyxcode}%
tex\_string = displayRankingsTeX(a\_db, crit\_db, props)
%
\end{lyxcode}%
%
\item[Description:]%
Generates a LaTeX document with:
	- Values of 10 best matching a\_db rows in a floating table.
	- (optional) Raw traces compared with some best matches at different distances
	- Parameter distributions of 50 best matches as a bar graph.
%%
\item[Parameters:]~
\begin{description}%
\item[\texttt{a\_db}:]
 A params\_tests\_db object to compare rows from.
\item[\texttt{crit\_db}:]
 A tests\_db object holding the match criterion tests and STDs

which can be created with matchingRow.
\item[\texttt{props}:]
 A structure with any optional properties.
\begin{description}%
\item[\texttt{caption}:]
 Identification of the criterion db (not needed/used?).
\item[\texttt{a\_dataset}:]
 Dataset for a\_db.
\item[\texttt{a\_dball}:]
 The non-joined DB for for a\_db.
\item[\texttt{crit\_dataset}:]
 Dataset for crit\_db.
\item[\texttt{crit\_dball}:]
 Dataset for crit\_db.
\item[\texttt{num\_matches}:]
 Number of best matches to display (default=10).
\item[\texttt{rotate}:]
 Rotation angle for best matches table (default=90).
\end{description}%
\end{description}%
%
\item[Returns:
]~

	tex\_string: LaTeX document string.
%
%
\item[See also:]%
\hyperlink{ref_rankVsDB}{\texttt{rankVsDB}}%
\ (p.~\pageref{ref_rankVsDB})%
\index[funcref]{rankVsDB@\fidxl{rankVsDB}}%
, \hyperlink{ref_displayRowsTeX}{\texttt{displayRowsTeX}}%
\ (p.~\pageref{ref_displayRowsTeX})%
\index[funcref]{displayRowsTeX@\fidxl{displayRowsTeX}}%
%
\item[Author:]%
Cengiz Gunay <cgunay@emory.edu>, 2004/10/20
%
\end{description}
\methodline%
\subsubsection[Method \texttt{fillMissingParams}]{Method \texttt{params\_tests\_db/fillMissingParams}}%
\index[funcref]{params_tests_db@\fidxl{params\_tests\_db}!fillMissingParams@\fidxl{fillMissingParams}}%
\label{ref_params_tests_db__fillMissingParams}%
\hypertarget{ref_params_tests_db__fillMissingParams}{}%
\begin{description}
\item[Summary:]Add missing columns as params with given default fill value.
%
\item[Usage:]~%
\begin{lyxcode}%
db = fillMissingParams(db, col\_names, fill\_value)
%
\end{lyxcode}%
%
%
\item[Parameters:]~
\begin{description}%
\item[\texttt{db}:]
 A tests\_db object.
\item[\texttt{col\_names}:]
 A cell array of param names.
\item[\texttt{fill\_value}:]
 Value to be used for missing columns.
\end{description}%
%
\item[Returns:
]~

	db: The tests\_db object that includes the newly filled columns.
%
%
\item[See also:]%
\hyperlink{ref_tests_db__fillMissingColumns}{\texttt{tests\_db/fillMissingColumns}}%
\ (p.~\pageref{ref_tests_db__fillMissingColumns})%
\index[funcref]{tests_db@\fidxl{tests\_db}!fillMissingColumns@\fidxl{fillMissingColumns}}%
, \hyperlink{ref_params_tests_db__addParams}{\texttt{params\_tests\_db/addParams}}%
\ (p.~\pageref{ref_params_tests_db__addParams})%
\index[funcref]{params_tests_db@\fidxl{params\_tests\_db}!addParams@\fidxl{addParams}}%
, \hyperlink{ref_params_tests_db__unionCat}{\texttt{params\_tests\_db/unionCat}}%
\ (p.~\pageref{ref_params_tests_db__unionCat})%
\index[funcref]{params_tests_db@\fidxl{params\_tests\_db}!unionCat@\fidxl{unionCat}}%
%
\item[Author:]%
Cengiz Gunay <cgunay@emory.edu>, 2008/06/02 
%
\end{description}
\methodline%
\subsubsection[Method \texttt{get}]{Method \texttt{params\_tests\_db/get}}%
\index[funcref]{params_tests_db@\fidxl{params\_tests\_db}!get@\fidxl{get}}%
\label{ref_params_tests_db__get}%
\hypertarget{ref_params_tests_db__get}{}%
\begin{description}
\item[Summary:]Defines generic attribute retrieval for objects.
%
%
%
%
%
%
%
\item[Author:]%
Cengiz Gunay <cgunay@emory.edu>, 2004/09/14
%
\end{description}
\methodline%
\subsubsection[Method \texttt{getDualCIPdb}]{Method \texttt{params\_tests\_db/getDualCIPdb}}%
\index[funcref]{params_tests_db@\fidxl{params\_tests\_db}!getDualCIPdb@\fidxl{getDualCIPdb}}%
\label{ref_params_tests_db__getDualCIPdb}%
\hypertarget{ref_params_tests_db__getDualCIPdb}{}%
\begin{description}
\item[Summary:]Generates a database by merging selected tests of depolarizing and hyperpolarizing cip results.
%
\item[Usage:]~%
\begin{lyxcode}%
a\_db = getDualCIPdb(db, depol\_tests, hyper\_tests, depol\_suffix, hyper\_suffix)
%
\end{lyxcode}%
%
\item[Description:]%
depol\_tests need to have the RowIndex column in it.
%%
\item[Parameters:]~
\begin{description}%
\item[\texttt{db}:]
 A params\_tests\_db object.
\end{description}%
%
\item[Returns:
]~

	a\_db: A params\_tests\_db object of organized values.
%
\item[Example:]~
\begin{lyxcode}        >> control\_phys\_sdb = getDualCIPdb(control\_phys\_db, depol\_tests, hyper\_tests, '', 'Hyp100pA')
\\%
        where depol\_tests and hyper\_tests are cell arrays of selected tests.
\\%
\end{lyxcode}
%
\item[See also:]%
\hyperlink{ref_invarValues}{\texttt{invarValues}}%
\ (p.~\pageref{ref_invarValues})%
\index[funcref]{invarValues@\fidxl{invarValues}}%
, \hyperlink{ref_tests_3D_db}{\texttt{tests\_3D\_db}}%
\ (p.~\pageref{ref_tests_3D_db})%
\index[funcref]{tests_3D_db@\fidxl{tests\_3D\_db}}%
, \hyperlink{ref_corrCoefs}{\texttt{corrCoefs}}%
\ (p.~\pageref{ref_corrCoefs})%
\index[funcref]{corrCoefs@\fidxl{corrCoefs}}%
, \hyperlink{ref_tests_3D_db__plotPair}{\texttt{tests\_3D\_db/plotPair}}%
\ (p.~\pageref{ref_tests_3D_db__plotPair})%
\index[funcref]{tests_3D_db@\fidxl{tests\_3D\_db}!plotPair@\fidxl{plotPair}}%
%
\item[Author:]%
Cengiz Gunay <cgunay@emory.edu>, 2005/01/13
%
\end{description}
\methodline%
\subsubsection[Method \texttt{getParamNames}]{Method \texttt{params\_tests\_db/getParamNames}}%
\index[funcref]{params_tests_db@\fidxl{params\_tests\_db}!getParamNames@\fidxl{getParamNames}}%
\label{ref_params_tests_db__getParamNames}%
\hypertarget{ref_params_tests_db__getParamNames}{}%
\begin{description}
\item[Summary:]Gets parameter column names.
%
\item[Usage:]~%
\begin{lyxcode}%
col\_names = getParamNames(db)
%
\end{lyxcode}%
%
\item[Description:]%
Convenience function that delegates to getColNames.
%%
\item[Parameters:]~
\begin{description}%
\item[\texttt{db}:]
 A params\_tests\_db object.
\end{description}%
%
\item[Returns:
]~

   col\_names: A cell array of strings.
%
%
\item[See also:]%
\hyperlink{ref_tests_db__getColNames}{\texttt{tests\_db/getColNames}}%
\ (p.~\pageref{ref_tests_db__getColNames})%
\index[funcref]{tests_db@\fidxl{tests\_db}!getColNames@\fidxl{getColNames}}%
, \hyperlink{ref_tests_db}{\texttt{tests\_db}}%
\ (p.~\pageref{ref_tests_db})%
\index[funcref]{tests_db@\fidxl{tests\_db}}%
%
\item[Author:]%
Cengiz Gunay <cgunay@emory.edu>, 2008/04/03
%
\end{description}
\methodline%
\subsubsection[Method \texttt{getParamRowIndices}]{Method \texttt{params\_tests\_db/getParamRowIndices}}%
\index[funcref]{params_tests_db@\fidxl{params\_tests\_db}!getParamRowIndices@\fidxl{getParamRowIndices}}%
\label{ref_params_tests_db__getParamRowIndices}%
\hypertarget{ref_params_tests_db__getParamRowIndices}{}%
\begin{description}
\item[Summary:]Returns indices of rows with matching parameter values from rows of this db.
%
\item[Usage:]~%
\begin{lyxcode}%
row\_indices = getParamRowIndices(a\_db, rows, to\_db)
%
\end{lyxcode}%
%
%
\item[Parameters:]~
\begin{description}%
\item[\texttt{a\_db}:]
 A params\_tests\_db object.
\item[\texttt{rows}:]
 rows to find indices for.
\item[\texttt{to\_db}:]
 Where to find the matching rows.
\end{description}%
%
\item[Returns:
]~

	row\_indices: Array of row indices.
%
%
\item[See also:]%
\hyperlink{ref_makeModifiedParamDB}{\texttt{makeModifiedParamDB}}%
\ (p.~\pageref{ref_makeModifiedParamDB})%
\index[funcref]{makeModifiedParamDB@\fidxl{makeModifiedParamDB}}%
, \hyperlink{ref_scanParamAllRows}{\texttt{scanParamAllRows}}%
\ (p.~\pageref{ref_scanParamAllRows})%
\index[funcref]{scanParamAllRows@\fidxl{scanParamAllRows}}%
, \hyperlink{ref_scaleParamsOneRow}{\texttt{scaleParamsOneRow}}%
\ (p.~\pageref{ref_scaleParamsOneRow})%
\index[funcref]{scaleParamsOneRow@\fidxl{scaleParamsOneRow}}%
, \hyperlink{ref_writeParFile}{\texttt{writeParFile}}%
\ (p.~\pageref{ref_writeParFile})%
\index[funcref]{writeParFile@\fidxl{writeParFile}}%
%
\item[Author:]%
Cengiz Gunay <cgunay@emory.edu>, 2005/01/14
%
\end{description}
\methodline%
\subsubsection[Method \texttt{getProfile}]{Method \texttt{params\_tests\_db/getProfile}}%
\index[funcref]{params_tests_db@\fidxl{params\_tests\_db}!getProfile@\fidxl{getProfile}}%
\label{ref_params_tests_db__getProfile}%
\hypertarget{ref_params_tests_db__getProfile}{}%
\begin{description}
\item[Summary:]Create a profile object from a params\_tests\_db by collecting
			 statistics.
%
\item[Usage:]~%
\begin{lyxcode}%
a\_pt\_profile = getProfile(a\_db, props)
%
\end{lyxcode}%
%
\item[Description:]%
Calculates the following results items:
	idx: Name-index pairs for accessing results arrays.
	t\_hists: Cell array of histograms of each test.
	p\_t3ds: Cell array of invariant relations of each parameter with all tests.
	pt\_hists: Cell array of separate test value histograms 
		for uniques value of each parameter.
	p\_stats: Cell array of test stats for each param.
	p\_coefs: Cell array of correlation coefficients 
		for each parameter with all tests.
	pt\_coefs\_hists: Cell matrix of histograms of coefficients from 
		correlations of each parameter with each test.
	pp\_coefs: Cell 3D matrix of mean coefficients from 
		correlations of each parameter with correlation 
		coefficients of each parameter with each test.		
%%
\item[Parameters:]~
\begin{description}%
\item[\texttt{a\_db}:]
 A params\_tests\_db object.
\item[\texttt{props}:]
 A structure with any optional properties.
\end{description}%
%
\item[Returns a params\_tests\_profile object.
]~

%
%
\item[See also:]%
\hyperlink{ref_params_tests_profile}{\texttt{params\_tests\_profile}}%
\ (p.~\pageref{ref_params_tests_profile})%
\index[funcref]{params_tests_profile@\fidxl{params\_tests\_profile}}%
, \hyperlink{ref_results_profile}{\texttt{results\_profile}}%
\ (p.~\pageref{ref_results_profile})%
\index[funcref]{results_profile@\fidxl{results\_profile}}%
, \hyperlink{ref_params_tests_db}{\texttt{params\_tests\_db}}%
\ (p.~\pageref{ref_params_tests_db})%
\index[funcref]{params_tests_db@\fidxl{params\_tests\_db}}%
, \hyperlink{ref_params_tests_fileset}{\texttt{params\_tests\_fileset}}%
\ (p.~\pageref{ref_params_tests_fileset})%
\index[funcref]{params_tests_fileset@\fidxl{params\_tests\_fileset}}%
, \hyperlink{ref_tests_db}{\texttt{tests\_db}}%
\ (p.~\pageref{ref_tests_db})%
\index[funcref]{tests_db@\fidxl{tests\_db}}%
, \hyperlink{ref_tests_3D_db}{\texttt{tests\_3D\_db}}%
\ (p.~\pageref{ref_tests_3D_db})%
\index[funcref]{tests_3D_db@\fidxl{tests\_3D\_db}}%
, \hyperlink{ref_histogram_db}{\texttt{histogram\_db}}%
\ (p.~\pageref{ref_histogram_db})%
\index[funcref]{histogram_db@\fidxl{histogram\_db}}%
, \hyperlink{ref_stats_db}{\texttt{stats\_db}}%
\ (p.~\pageref{ref_stats_db})%
\index[funcref]{stats_db@\fidxl{stats\_db}}%
, \hyperlink{ref_corrcoefs_db}{\texttt{corrcoefs\_db}}%
\ (p.~\pageref{ref_corrcoefs_db})%
\index[funcref]{corrcoefs_db@\fidxl{corrcoefs\_db}}%
%
\item[Author:]%
Cengiz Gunay <cgunay@emory.edu>, 2004/10/13
%
\end{description}
\methodline%
\subsubsection[Method \texttt{invarParam}]{Method \texttt{params\_tests\_db/invarParam}}%
\index[funcref]{params_tests_db@\fidxl{params\_tests\_db}!invarParam@\fidxl{invarParam}}%
\label{ref_params_tests_db__invarParam}%
\hypertarget{ref_params_tests_db__invarParam}{}%
\begin{description}
\item[Summary:]Generates a 3D database of invariant values of a parameter and all test columns. 
%
\item[Usage:]~%
\begin{lyxcode}%
a\_3D\_db = invarParam(db, param, props)
%
\end{lyxcode}%
%
%
\item[Parameters:]~
\begin{description}%
\item[\texttt{db}:]
 A tests\_db object.
\item[\texttt{param}:]
 A parameter name/column number. It can be empty [], meaning to

find all unique combinations of parameters.
\item[\texttt{props}:]
 A structure with any optional properties.
\begin{description}%
\item[\texttt{removeRedun}:]
 If 1 (default), clean database by removing redundant

sets of parameters.
\item[\texttt{removeCol}:]
 If removeRedun == 1, name of parameter column to remove 

if found. Default: 'trial'.
(others passed to tests\_db/invarValues)
\end{description}%
\end{description}%
%
%
%
%
%
\end{description}
\methodline%
\subsubsection[Method \texttt{invarParams}]{Method \texttt{params\_tests\_db/invarParams}}%
\index[funcref]{params_tests_db@\fidxl{params\_tests\_db}!invarParams@\fidxl{invarParams}}%
\label{ref_params_tests_db__invarParams}%
\hypertarget{ref_params_tests_db__invarParams}{}%
\begin{description}
\item[Summary:]Calculates invariant param dbs for all parameters and returns in an array.
%
\item[Usage:]~%
\begin{lyxcode}%
p\_t3ds = invarParams(a\_db)
%
\end{lyxcode}%
%
\item[Description:]%
Skips the 'ItemIndex' test.
%%
\item[Parameters:]~
\begin{description}%
\item[\texttt{a\_db}:]
 A tests\_db object.
\end{description}%
%
\item[Returns:
]~

	p\_t3ds: An array of tests\_3D\_dbs for each param in a\_db.
%
%
\item[See also:]%
\hyperlink{ref_params_tests_profile}{\texttt{params\_tests\_profile}}%
\ (p.~\pageref{ref_params_tests_profile})%
\index[funcref]{params_tests_profile@\fidxl{params\_tests\_profile}}%
%
\item[Author:]%
Cengiz Gunay <cgunay@emory.edu>, 2004/10/17
%
\end{description}
\methodline%
\subsubsection[Method \texttt{joinRows}]{Method \texttt{params\_tests\_db/joinRows}}%
\index[funcref]{params_tests_db@\fidxl{params\_tests\_db}!joinRows@\fidxl{joinRows}}%
\label{ref_params_tests_db__joinRows}%
\hypertarget{ref_params_tests_db__joinRows}{}%
\begin{description}
\item[Summary:]Joins the rows of the given db with rows of with\_db with matching
  	RowIndex values.
%
\item[Usage:]~%
\begin{lyxcode}%
a\_db = joinRows(db, with\_db, props)
%
\end{lyxcode}%
%
\item[Description:]%
Takes the desired columns in with\_db with rows having a 
 row index and joins them next to dedired columns from the current db. 
 Assumes each row index only appears once in with\_db. The created
 db preserves the ordering of with\_db.
%%
\item[Parameters:]~
\begin{description}%
\item[\texttt{db}:]
 A param\_tests\_db object.
\item[\texttt{with\_db}:]
 A tests\_db object with a RowIndex column.
\item[\texttt{props}:]
 A structure with any optional properties.
\begin{description}%
\item[\texttt{indexColName}:]
 (Optional) Name of row index column (default='RowIndex').
\end{description}%
\end{description}%
%
\item[Returns:
]~

	a\_db: A params\_tests\_db object.
%
%
\item[See also:]%
\hyperlink{ref_tests_db}{\texttt{tests\_db}}%
\ (p.~\pageref{ref_tests_db})%
\index[funcref]{tests_db@\fidxl{tests\_db}}%
%
\item[Author:]%
Cengiz Gunay <cgunay@emory.edu>, 2004/10/16
%
\end{description}
\methodline%
\subsubsection[Method \texttt{matchingRow}]{Method \texttt{params\_tests\_db/matchingRow}}%
\index[funcref]{params_tests_db@\fidxl{params\_tests\_db}!matchingRow@\fidxl{matchingRow}}%
\label{ref_params_tests_db__matchingRow}%
\hypertarget{ref_params_tests_db__matchingRow}{}%
\begin{description}
\item[Summary:]Creates a criterion database for matching the tests of a row.
%
\item[Usage:]~%
\begin{lyxcode}%
crit\_db = matchingRow(a\_db, row, props)
%
\end{lyxcode}%
%
\item[Description:]%
Overloaded method for skipping parameter values. STD for param values will be NaNs.
%%
\item[Parameters:]~
\begin{description}%
\item[\texttt{a\_db}:]
 A tests\_db object.
\item[\texttt{row}:]
 A row index to match.
\item[\texttt{props}:]
 A structure with any optional properties.
\begin{description}%
\item[\texttt{distDB}:]
 Take the standard deviation from this db instead.
\end{description}%
\end{description}%
%
\item[Returns:
]~

	crit\_db: A tests\_db with two rows for values and STDs.
%
%
\item[See also:]%
\hyperlink{ref_tests_db__matchingRow}{\texttt{tests\_db/matchingRow}}%
\ (p.~\pageref{ref_tests_db__matchingRow})%
\index[funcref]{tests_db@\fidxl{tests\_db}!matchingRow@\fidxl{matchingRow}}%
, \hyperlink{ref_rankMatching}{\texttt{rankMatching}}%
\ (p.~\pageref{ref_rankMatching})%
\index[funcref]{rankMatching@\fidxl{rankMatching}}%
, \hyperlink{ref_tests_db}{\texttt{tests\_db}}%
\ (p.~\pageref{ref_tests_db})%
\index[funcref]{tests_db@\fidxl{tests\_db}}%
, \hyperlink{ref_tests2cols}{\texttt{tests2cols}}%
\ (p.~\pageref{ref_tests2cols})%
\index[funcref]{tests2cols@\fidxl{tests2cols}}%
%
\item[Author:]%
Cengiz Gunay <cgunay@emory.edu>, 2006/06/13
%
\end{description}
\methodline%
\subsubsection[Method \texttt{meanDuplicateParams}]{Method \texttt{params\_tests\_db/meanDuplicateParams}}%
\index[funcref]{params_tests_db@\fidxl{params\_tests\_db}!meanDuplicateParams@\fidxl{meanDuplicateParams}}%
\label{ref_params_tests_db__meanDuplicateParams}%
\hypertarget{ref_params_tests_db__meanDuplicateParams}{}%
\begin{description}
\item[Summary:]Takes the mean of all measures for rows that have the same parameter columns.
%
\item[Usage:]~%
\begin{lyxcode}%
a\_params\_tests\_db = meanDuplicateParams(db, props)
%
\end{lyxcode}%
%
\item[Description:]%
Calls tests\_db/meanDuplicateRows with params as main\_cols and tests
 as rest\_cols.
%%
\item[Parameters:]~
\begin{description}%
\item[\texttt{db}:]
 A params\_tests\_db object.
\item[\texttt{props}:]
 Structure with optional parameters.
\end{description}%
%
\item[Returns:
]~

	a\_params\_tests\_db: The db object of with the means on page 1 
		    and standard deviations on page 2.
%
%
\item[See also:]%
\hyperlink{ref_tests_db__meanDuplicateRows}{\texttt{tests\_db/meanDuplicateRows}}%
\ (p.~\pageref{ref_tests_db__meanDuplicateRows})%
\index[funcref]{tests_db@\fidxl{tests\_db}!meanDuplicateRows@\fidxl{meanDuplicateRows}}%
, \hyperlink{ref_tests_db__mean}{\texttt{tests\_db/mean}}%
\ (p.~\pageref{ref_tests_db__mean})%
\index[funcref]{tests_db@\fidxl{tests\_db}!mean@\fidxl{mean}}%
, \hyperlink{ref_tests_db__std}{\texttt{tests\_db/std}}%
\ (p.~\pageref{ref_tests_db__std})%
\index[funcref]{tests_db@\fidxl{tests\_db}!std@\fidxl{std}}%
, \hyperlink{ref_sortedUniqueValues}{\texttt{sortedUniqueValues}}%
\ (p.~\pageref{ref_sortedUniqueValues})%
\index[funcref]{sortedUniqueValues@\fidxl{sortedUniqueValues}}%
%
\item[Author:]%
Cengiz Gunay <cgunay@emory.edu>, 2007/12/20
%
\end{description}
\methodline%
\subsubsection[Method \texttt{mergeMultipleCIPsInOne}]{Method \texttt{params\_tests\_db/mergeMultipleCIPsInOne}}%
\index[funcref]{params_tests_db@\fidxl{params\_tests\_db}!mergeMultipleCIPsInOne@\fidxl{mergeMultipleCIPsInOne}}%
\label{ref_params_tests_db__mergeMultipleCIPsInOne}%
\hypertarget{ref_params_tests_db__mergeMultipleCIPsInOne}{}%
\begin{description}
\item[Summary:]Merges multiple rows with different CIP data into one, generating a database of one row per neuron.
%
\item[Usage:]~%
\begin{lyxcode}%
a\_db = mergeMultipleCIPsInOne(db, names\_tests\_cell, index\_col\_name, props)
%
\end{lyxcode}%
%
\item[Description:]%
It calls invarParam to separate DB into pages with different CIP level
 data.  Then uses the names\_tests\_cell to choose tests from each page to be
 merged into the final database row. The tests will be suffixed with the
 field name so that they can be distinguished. RowIndex columns will be
 automatically included, and one of them can be chosen with index\_col\_name
 that has values for all cells. The suffixed for needs to be used to choose
 index\_col\_name, such as 'RowIndex\_H100pA', assuming 'H100pA' was the field
 name in names\_tests\_cell that corresponds to page -100 pA.
%%
\item[Parameters:]~
\begin{description}%
\item[\texttt{db}:]
 A params\_tests\_db object.
\item[\texttt{names\_tests\_cell}:]
 A cell array alternating suffix names and test column vectors.

The order of names correspond to each unique CIP level in db, 
with increasing order.
\item[\texttt{index\_col\_name}:]
 (Optional) Name of row index column 

(default is 'RowIndex' suffixed with the first field name).
\item[\texttt{props}:]
 A structure with any optional properties.
\begin{description}%
\item[\texttt{cipLevels}:]
 In case db is missing some levels, provides a list of 

cip levels that correspond to names\_tests\_cell db. Missing 
levels are replaced with NaN values. DB is filtered to
remove other CIP levels.
\end{description}%
\end{description}%
%
\item[Returns:
]~

	a\_db: A params\_tests\_db object of organized values.
%
\item[Example:]~
\begin{lyxcode}        >> control\_phys\_sdb = 
\\%
             mergeMultipleCIPsInOne(control\_phys\_db, 
\\%
                                     struct('\_H100pA', [1:10], '\_D100pA', [1:10 16:18]), 
\\%
                                     'RowIndex\_H100pA')
\\%
\end{lyxcode}
%
\item[See also:]%
\hyperlink{ref_invarValues}{\texttt{invarValues}}%
\ (p.~\pageref{ref_invarValues})%
\index[funcref]{invarValues@\fidxl{invarValues}}%
, \hyperlink{ref_tests_3D_db}{\texttt{tests\_3D\_db}}%
\ (p.~\pageref{ref_tests_3D_db})%
\index[funcref]{tests_3D_db@\fidxl{tests\_3D\_db}}%
, \hyperlink{ref_corrCoefs}{\texttt{corrCoefs}}%
\ (p.~\pageref{ref_corrCoefs})%
\index[funcref]{corrCoefs@\fidxl{corrCoefs}}%
, \hyperlink{ref_tests_3D_db__plotVarBox}{\texttt{tests\_3D\_db/plotVarBox}}%
\ (p.~\pageref{ref_tests_3D_db__plotVarBox})%
\index[funcref]{tests_3D_db@\fidxl{tests\_3D\_db}!plotVarBox@\fidxl{plotVarBox}}%
%
\item[Author:]%
Cengiz Gunay <cgunay@emory.edu>, 2005/01/13
%
\end{description}
\methodline%
\subsubsection[Method \texttt{onlyRowsTests}]{Method \texttt{params\_tests\_db/onlyRowsTests}}%
\index[funcref]{params_tests_db@\fidxl{params\_tests\_db}!onlyRowsTests@\fidxl{onlyRowsTests}}%
\label{ref_params_tests_db__onlyRowsTests}%
\hypertarget{ref_params_tests_db__onlyRowsTests}{}%
\begin{description}
\item[Summary:]Returns a tests\_db that only contains the desired 
		tests and rows (and pages).
%
\item[Usage:]~%
\begin{lyxcode}%
obj = onlyRowsTests(obj, rows, tests, pages)
%
\end{lyxcode}%
%
\item[Description:]%
Selects the given dimensions and returns in a new tests\_db
 object. Makes sure num\_params remains correct.
%%
\item[Parameters:]~
\begin{description}%
\item[\texttt{obj}:]
 A tests\_db object.
\item[\texttt{rows, tests}:]
 A logical or index vector of rows, or cell array of

names of rows. If ':', all rows. For names, regular expressions are
supported if quoted with slashes (e.g., '/a.*/'). See tests2idx.
\item[\texttt{pages}:]
 (Optional) A logical or index vector of pages. ':' for all pages.
\end{description}%
%
\item[Returns:
]~

	obj: The new tests\_db object.
%
%
\item[See also:]%
\hyperlink{ref_subsref}{\texttt{subsref}}%
\ (p.~\pageref{ref_subsref})%
\index[funcref]{subsref@\fidxl{subsref}}%
, \hyperlink{ref_tests_db}{\texttt{tests\_db}}%
\ (p.~\pageref{ref_tests_db})%
\index[funcref]{tests_db@\fidxl{tests\_db}}%
, \hyperlink{ref_test2idx}{\texttt{test2idx}}%
\ (p.~\pageref{ref_test2idx})%
\index[funcref]{test2idx@\fidxl{test2idx}}%
%
\item[Author:]%
Cengiz Gunay <cgunay@emory.edu>, 2004/09/17
%
\end{description}
\methodline%
\subsubsection[Method \texttt{paramsCoefs}]{Method \texttt{params\_tests\_db/paramsCoefs}}%
\index[funcref]{params_tests_db@\fidxl{params\_tests\_db}!paramsCoefs@\fidxl{paramsCoefs}}%
\label{ref_params_tests_db__paramsCoefs}%
\hypertarget{ref_params_tests_db__paramsCoefs}{}%
\begin{description}
\item[Summary:]Calculates a corrcoefs\_db for each param and collects them in a cell array.
%
\item[Usage:]~%
\begin{lyxcode}%
p\_coefs = paramsCoefs(a\_db, p\_t3ds)
%
\end{lyxcode}%
%
\item[Description:]%
Skips the 'ItemIndex' test.
%%
\item[Parameters:]~
\begin{description}%
\item[\texttt{a\_db}:]
 A tests\_db object.
\item[\texttt{p\_t3ds}:]
 Cell array of invariant parameter databases.
\end{description}%
%
\item[Returns:
]~

	p\_coefs: A cell array of corrcoefs\_dbs for each param in a\_db.
%
%
\item[See also:]%
\hyperlink{ref_params_tests_profile}{\texttt{params\_tests\_profile}}%
\ (p.~\pageref{ref_params_tests_profile})%
\index[funcref]{params_tests_profile@\fidxl{params\_tests\_profile}}%
%
\item[Author:]%
Cengiz Gunay <cgunay@emory.edu>, 2004/10/17
%
\end{description}
\methodline%
\subsubsection[Method \texttt{paramsHists}]{Method \texttt{params\_tests\_db/paramsHists}}%
\index[funcref]{params_tests_db@\fidxl{params\_tests\_db}!paramsHists@\fidxl{paramsHists}}%
\label{ref_params_tests_db__paramsHists}%
\hypertarget{ref_params_tests_db__paramsHists}{}%
\begin{description}
\item[Summary:]Calculates histograms for all parameters and returns in a 
		cell array.
%
\item[Usage:]~%
\begin{lyxcode}%
p\_hists = paramsHists(a\_db, props)
%
\end{lyxcode}%
%
\item[Description:]%
Useful for looking at subset databases and find out what parameter
 values are used most. Skips the 'ItemIndex' column(s). Finds unique values
 of each parameter to make histograms.
%%
\item[Parameters:]~
\begin{description}%
\item[\texttt{a\_db}:]
 A tests\_db object.
\item[\texttt{props}:]
 A structure with any optional properties.
\begin{description}%
\item[\texttt{paramVals}:]
 Array of histogram bins to use as parameter values.
\end{description}%
\end{description}%
%
\item[Returns:
]~

   p\_hists: An array of histograms for each parameter in a\_db.
%
%
\item[See also:]%
\hyperlink{ref_params_tests_profile}{\texttt{params\_tests\_profile}}%
\ (p.~\pageref{ref_params_tests_profile})%
\index[funcref]{params_tests_profile@\fidxl{params\_tests\_profile}}%
%
\item[Author:]%
Cengiz Gunay <cgunay@emory.edu>, 2004/10/20
%
\end{description}
\methodline%
\subsubsection[Method \texttt{paramsParamsCoefs}]{Method \texttt{params\_tests\_db/paramsParamsCoefs}}%
\index[funcref]{params_tests_db@\fidxl{params\_tests\_db}!paramsParamsCoefs@\fidxl{paramsParamsCoefs}}%
\label{ref_params_tests_db__paramsParamsCoefs}%
\hypertarget{ref_params_tests_db__paramsParamsCoefs}{}%
\begin{description}
\item[Summary:]Calculates a corrcoefs\_db for each param from correlations of variant params and invariant param coefs and collects them in a cell array.
%
\item[Usage:]~%
\begin{lyxcode}%
pp\_coefs = paramsParamsCoefs(a\_db, p\_t3ds, p\_coefs)
%
\end{lyxcode}%
%
\item[Description:]%
Skips the 'ItemIndex' test.
%%
\item[Parameters:]~
\begin{description}%
\item[\texttt{a\_db}:]
 A tests\_db object.
\item[\texttt{p\_t3ds}:]
 Cell array of invariant parameter databases.
\item[\texttt{p\_coefs}:]
 Cell array of tests coefficients for each parameter.
\end{description}%
%
\item[Returns:
]~

	pp\_coefs: A cell array of corrcoefs\_dbs for each param 
		  combination in a\_db.
%
%
\item[See also:]%
\hyperlink{ref_params_tests_profile}{\texttt{params\_tests\_profile}}%
\ (p.~\pageref{ref_params_tests_profile})%
\index[funcref]{params_tests_profile@\fidxl{params\_tests\_profile}}%
%
\item[Author:]%
Cengiz Gunay <cgunay@emory.edu>, 2004/10/17
%
\end{description}
\methodline%
\subsubsection[Method \texttt{paramsTestsCoefsHists}]{Method \texttt{params\_tests\_db/paramsTestsCoefsHists}}%
\index[funcref]{params_tests_db@\fidxl{params\_tests\_db}!paramsTestsCoefsHists@\fidxl{paramsTestsCoefsHists}}%
\label{ref_params_tests_db__paramsTestsCoefsHists}%
\hypertarget{ref_params_tests_db__paramsTestsCoefsHists}{}%
\begin{description}
\item[Summary:]Calculates histograms for all pairs of params 
		  and tests coefficients and returns in a cell array.
%
\item[Usage:]~%
\begin{lyxcode}%
pt\_coefs\_hists = paramsTestsCoefsHists(a\_db, p\_coefs)
%
\end{lyxcode}%
%
\item[Description:]%
Skips the 'ItemIndex' test.
%%
\item[Parameters:]~
\begin{description}%
\item[\texttt{a\_db}:]
 A tests\_db object.
\item[\texttt{p\_coefs}:]
 Cell array of tests coefficients for each parameter.
\end{description}%
%
\item[Returns:
]~

	pt\_coefs\_hists: A cell array of corrcoefs\_dbs for each param in a\_db.
%
%
\item[See also:]%
\hyperlink{ref_params_tests_profile}{\texttt{params\_tests\_profile}}%
\ (p.~\pageref{ref_params_tests_profile})%
\index[funcref]{params_tests_profile@\fidxl{params\_tests\_profile}}%
%
\item[Author:]%
Cengiz Gunay <cgunay@emory.edu>, 2004/10/17
%
\end{description}
\methodline%
\subsubsection[Method \texttt{plotParamsHists}]{Method \texttt{params\_tests\_db/plotParamsHists}}%
\index[funcref]{params_tests_db@\fidxl{params\_tests\_db}!plotParamsHists@\fidxl{plotParamsHists}}%
\label{ref_params_tests_db__plotParamsHists}%
\hypertarget{ref_params_tests_db__plotParamsHists}{}%
\begin{description}
\item[Summary:]Create a horizontal plot\_stack of parameter histograms.
%
\item[Usage:]~%
\begin{lyxcode}%
a\_ps = plotParamsHists(a\_db, title\_str, props)
%
\end{lyxcode}%
%
\item[Description:]%
Skips the 'ItemIndex' test.
%%
\item[Parameters:]~
\begin{description}%
\item[\texttt{a\_db}:]
 A params\_tests\_db object.
\item[\texttt{title\_str}:]
 (Optional) A string to be concatanated to the title.
\item[\texttt{props}:]
 A structure with any optional properties.
\begin{description}%
\item[\texttt{quiet}:]
 Do not display the DB id on the plot title.
\item[\texttt{barAxisProps}:]
 passed to plotEqSpaced for each bar axis.
\item[\texttt{stackAxisLimits}:]
 Axis limits for plot\_stack (default=[NaN NaN 0 Inf]).

(Others passed to paramsHists, plotEqSpaced, and plot\_stack)
\end{description}%
\end{description}%
%
\item[Returns:
]~

   a\_ps: A horizontal plot\_stack of plots
%
%
\item[See also:]%
\hyperlink{ref_plot_stack}{\texttt{plot\_stack}}%
\ (p.~\pageref{ref_plot_stack})%
\index[funcref]{plot_stack@\fidxl{plot\_stack}}%
, \hyperlink{ref_paramsHists}{\texttt{paramsHists}}%
\ (p.~\pageref{ref_paramsHists})%
\index[funcref]{paramsHists@\fidxl{paramsHists}}%
, \hyperlink{ref_plotEqSpaced}{\texttt{plotEqSpaced}}%
\ (p.~\pageref{ref_plotEqSpaced})%
\index[funcref]{plotEqSpaced@\fidxl{plotEqSpaced}}%
%
\item[Author:]%
Cengiz Gunay <cgunay@emory.edu>, 2005/04/07
%
\end{description}
\methodline%
\subsubsection[Method \texttt{plotVarBoxMatrix}]{Method \texttt{params\_tests\_db/plotVarBoxMatrix}}%
\index[funcref]{params_tests_db@\fidxl{params\_tests\_db}!plotVarBoxMatrix@\fidxl{plotVarBoxMatrix}}%
\label{ref_params_tests_db__plotVarBoxMatrix}%
\hypertarget{ref_params_tests_db__plotVarBoxMatrix}{}%
\begin{description}
\item[Summary:]Create a stack of parameter-test variation plots 
		organized in a matrix.
%
\item[Usage:]~%
\begin{lyxcode}%
a\_plot\_stack = plotVarBoxMatrix(a\_db, p\_t3ds)
%
\end{lyxcode}%
%
\item[Description:]%
Skips the 'ItemIndex' test.
%%
\item[Parameters:]~
\begin{description}%
\item[\texttt{a\_db}:]
 A tests\_db object.
\item[\texttt{p\_t3ds}:]
 Cell array of invariant parameter databases.
\end{description}%
%
\item[Returns:
]~

	a\_plot\_stack: A plot\_stack with the plots organized in matrix form
%
%
\item[See also:]%
\hyperlink{ref_params_tests_profile}{\texttt{params\_tests\_profile}}%
\ (p.~\pageref{ref_params_tests_profile})%
\index[funcref]{params_tests_profile@\fidxl{params\_tests\_profile}}%
, \hyperlink{ref_plotVar}{\texttt{plotVar}}%
\ (p.~\pageref{ref_plotVar})%
\index[funcref]{plotVar@\fidxl{plotVar}}%
%
\item[Author:]%
Cengiz Gunay <cgunay@emory.edu>, 2004/10/17
%
\end{description}
\methodline%
\subsubsection[Method \texttt{rankVsAllDB}]{Method \texttt{params\_tests\_db/rankVsAllDB}}%
\index[funcref]{params_tests_db@\fidxl{params\_tests\_db}!rankVsAllDB@\fidxl{rankVsAllDB}}%
\label{ref_params_tests_db__rankVsAllDB}%
\hypertarget{ref_params_tests_db__rankVsAllDB}{}%
\begin{description}
\item[Summary:]Generates ranking DBs by comparing rows of a\_db with each row of to\_db.
%
\item[Usage:]~%
\begin{lyxcode}%
tex\_string = rankVsAllDB(a\_db, to\_db, a\_dataset, to\_dataset)
%
\end{lyxcode}%
%
\item[Description:]%
Distance is each measure difference divided by the STD in to\_db, squared and
 summed. Returned DB contains only the selected to\_tests and the parameters
 from initial DB.
%%
\item[Parameters:]~
\begin{description}%
\item[\texttt{a\_db}:]
 A params\_tests\_db object to compare rows from.
\item[\texttt{to\_db}:]
 A tests\_db object to compare it with.
\item[\texttt{a\_dataset}:]
 Dataset for a\_db.
\item[\texttt{to\_dataset}:]
 Dataset for crit\_db.
\end{description}%
%
\item[Returns:
]~

	ranked\_dbs: Array of created DBs with original rows and a distance 
		   measure, in ascending order. 
	tex\_string: A LaTeX string for all tables created.
%
%
\item[See also:]%
\hyperlink{ref_rankVsDB}{\texttt{rankVsDB}}%
\ (p.~\pageref{ref_rankVsDB})%
\index[funcref]{rankVsDB@\fidxl{rankVsDB}}%
, \hyperlink{ref_matchingRow}{\texttt{matchingRow}}%
\ (p.~\pageref{ref_matchingRow})%
\index[funcref]{matchingRow@\fidxl{matchingRow}}%
, \hyperlink{ref_rankMatching}{\texttt{rankMatching}}%
\ (p.~\pageref{ref_rankMatching})%
\index[funcref]{rankMatching@\fidxl{rankMatching}}%
, \hyperlink{ref_joinRows}{\texttt{joinRows}}%
\ (p.~\pageref{ref_joinRows})%
\index[funcref]{joinRows@\fidxl{joinRows}}%
%
\item[Author:]%
Cengiz Gunay <cgunay@emory.edu>, 2004/12/10
%
\end{description}
\methodline%
\subsubsection[Method \texttt{rankVsDB}]{Method \texttt{params\_tests\_db/rankVsDB}}%
\index[funcref]{params_tests_db@\fidxl{params\_tests\_db}!rankVsDB@\fidxl{rankVsDB}}%
\label{ref_params_tests_db__rankVsDB}%
\hypertarget{ref_params_tests_db__rankVsDB}{}%
\begin{description}
\item[Summary:]Generates a ranking DB by comparing rows of this db with the given test criteria.
%
\item[Usage:]~%
\begin{lyxcode}%
a\_ranked\_db = rankVsDB(a\_db, crit\_db)
%
\end{lyxcode}%
%
\item[Description:]%
Distance is each measure difference divided by the STD in to\_db, squared and
 summed. Returned DB contains only the selected tests from crit\_db and the parameters
 from initial a\_db.
%%
\item[Parameters:]~
\begin{description}%
\item[\texttt{a\_db}:]
 A params\_tests\_db object to compare rows from.
\item[\texttt{crit\_db}:]
 A tests\_db object holding the match criterion tests and STDs

which can be created with matchingRow.
\end{description}%
%
\item[Returns:
]~

	a\_ranked\_db: The created DB with original rows and a distance measure, 
		   in ascending order. 
%
%
\item[See also:]%
\hyperlink{ref_matchingRow}{\texttt{matchingRow}}%
\ (p.~\pageref{ref_matchingRow})%
\index[funcref]{matchingRow@\fidxl{matchingRow}}%
, \hyperlink{ref_rankMatching}{\texttt{rankMatching}}%
\ (p.~\pageref{ref_rankMatching})%
\index[funcref]{rankMatching@\fidxl{rankMatching}}%
, \hyperlink{ref_joinRows}{\texttt{joinRows}}%
\ (p.~\pageref{ref_joinRows})%
\index[funcref]{joinRows@\fidxl{joinRows}}%
%
\item[Author:]%
Cengiz Gunay <cgunay@emory.edu>, 2004/10/20
%
\end{description}
\methodline%
\subsubsection[Method \texttt{reIndexNeurons}]{Method \texttt{params\_tests\_db/reIndexNeurons}}%
\index[funcref]{params_tests_db@\fidxl{params\_tests\_db}!reIndexNeurons@\fidxl{reIndexNeurons}}%
\label{ref_params_tests_db__reIndexNeurons}%
\hypertarget{ref_params_tests_db__reIndexNeurons}{}%
\begin{description}
\item[Summary:]Re-index neurons with accending numbers. 
%
\item[Usage:]~%
\begin{lyxcode}%
new\_db = reIndexNeurons(a\_db, startNum, colName)
%
\end{lyxcode}%
%
\item[Description:]%
This is useful for avoiding NeuronId conflicts when concatenating two
 databases. It can also remove the unused number 'hole' (e.g. after
 deleting rows) and make the indices continuous.
%%
\item[Parameters:]~
\begin{description}%
\item[\texttt{a\_db}:]
 a tests\_db object
\item[\texttt{startNum}:]
 the starting index number (default = 1)
\item[\texttt{colName}:]
 the column name or number of neuron index. (default = 'NeuronId')
\end{description}%
%
\item[Returns:
]~

   new\_db: a new database with new neuron index.
%
%
\item[See also:]%
\hyperlink{ref_physiol_bundle}{\texttt{physiol\_bundle}}%
\ (p.~\pageref{ref_physiol_bundle})%
\index[funcref]{physiol_bundle@\fidxl{physiol\_bundle}}%
, \hyperlink{ref_tests_db__physiol_bundle}{\texttt{tests\_db/physiol\_bundle}}%
\ (p.~\pageref{ref_tests_db__physiol_bundle})%
\index[funcref]{tests_db@\fidxl{tests\_db}!physiol_bundle@\fidxl{physiol\_bundle}}%
%
\item[Author:]%
Li Su. 03/21/2008
%
\end{description}
\methodline%
\subsubsection[Method \texttt{scanParamAllRows}]{Method \texttt{params\_tests\_db/scanParamAllRows}}%
\index[funcref]{params_tests_db@\fidxl{params\_tests\_db}!scanParamAllRows@\fidxl{scanParamAllRows}}%
\label{ref_params_tests_db__scanParamAllRows}%
\hypertarget{ref_params_tests_db__scanParamAllRows}{}%
\begin{description}
\item[Summary:]OBSOLETE (use instead varyParams) - Scans given parameter range for each row in DB.
%
\item[Usage:]~%
\begin{lyxcode}%
a\_params\_db = scanParamAllRows(a\_db, param, min\_val, max\_val, num\_levels, props)
%
\end{lyxcode}%
%
\item[Description:]%
Produces rows by replacing the desired parameter value, in all rows of DB, 
 with num\_levels values between the given boundaries, min\_val and max\_val. 
 This results in a DB with num\_levels times more rows than the original DB. 
 Then, writeParFile can be used to generate a parameter file from 
 this DB to drive new simulations.
%%
\item[Parameters:]~
\begin{description}%
\item[\texttt{a\_db}:]
 A params\_tests\_db object whose first row is subject to modifications.
\item[\texttt{param}:]
 The parameter to be varied (see tests2cols for param description).
\item[\texttt{min\_val, max\_val}:]
 The low and high boundaries for the parameter value.
\item[\texttt{num\_levels}:]
 Number of levels to produce, including the boundaries.
\item[\texttt{props}:]
 A structure with any optional properties.
\begin{description}%
\item[\texttt{renameTrial}:]
 If given, the 'trial' column is renamed to this name.
\item[\texttt{levelFunc}:]
 Use this function to get the parameter range with 

feval(levelFunc, min\_val, max\_val, num\_levels). Example: 'logLevels'
\end{description}%
\end{description}%
%
\item[Returns:
]~

	a\_params\_db: A db only with params.
%
\item[Example:]~
\begin{lyxcode} Sets NaF to given range with 100 levels:
\\%
 >> naf\_rows\_db = scanParamAllRows(a\_db(desired\_rows, :), 'NaF', 0, 1000, 100);
\\%
\end{lyxcode}
%
\item[See also:]%
\hyperlink{ref_writeParFile}{\texttt{writeParFile}}%
\ (p.~\pageref{ref_writeParFile})%
\index[funcref]{writeParFile@\fidxl{writeParFile}}%
, \hyperlink{ref_scaleParamsOneRow}{\texttt{scaleParamsOneRow}}%
\ (p.~\pageref{ref_scaleParamsOneRow})%
\index[funcref]{scaleParamsOneRow@\fidxl{scaleParamsOneRow}}%
, \hyperlink{ref_ranked_db__blockedDistances}{\texttt{ranked\_db/blockedDistances}}%
\ (p.~\pageref{ref_ranked_db__blockedDistances})%
\index[funcref]{ranked_db@\fidxl{ranked\_db}!blockedDistances@\fidxl{blockedDistances}}%
, \hyperlink{ref_getParamRowIndices}{\texttt{getParamRowIndices}}%
\ (p.~\pageref{ref_getParamRowIndices})%
\index[funcref]{getParamRowIndices@\fidxl{getParamRowIndices}}%
, \hyperlink{ref_logLevels}{\texttt{logLevels}}%
\ (p.~\pageref{ref_logLevels})%
\index[funcref]{logLevels@\fidxl{logLevels}}%
%
\item[Author:]%
Cengiz Gunay <cgunay@emory.edu>, 2006/02/16
%
\end{description}
\methodline%
\subsubsection[Method \texttt{set}]{Method \texttt{params\_tests\_db/set}}%
\index[funcref]{params_tests_db@\fidxl{params\_tests\_db}!set@\fidxl{set}}%
\label{ref_params_tests_db__set}%
\hypertarget{ref_params_tests_db__set}{}%
\begin{description}
\item[Summary:]Generic method for setting object attributes.
%
%
%
%
%
%
%
\item[Author:]%
Cengiz Gunay <cgunay@emory.edu>, 2004/10/08
%
\end{description}
\methodline%
\subsubsection[Method \texttt{subsref}]{Method \texttt{params\_tests\_db/subsref}}%
\index[funcref]{params_tests_db@\fidxl{params\_tests\_db}!subsref@\fidxl{subsref}}%
\label{ref_params_tests_db__subsref}%
\hypertarget{ref_params_tests_db__subsref}{}%
\begin{description}
\item[Summary:]Defines generic indexing for objects.
%
%
%
%
%
%
%
%
\end{description}
\methodline%
\subsubsection[Method \texttt{testsHists}]{Method \texttt{params\_tests\_db/testsHists}}%
\index[funcref]{params_tests_db@\fidxl{params\_tests\_db}!testsHists@\fidxl{testsHists}}%
\label{ref_params_tests_db__testsHists}%
\hypertarget{ref_params_tests_db__testsHists}{}%
\begin{description}
\item[Summary:]Calculates histograms for all tests and returns them in a cell array.
%
\item[Usage:]~%
\begin{lyxcode}%
t\_hists = testsHists(a\_db, num\_bins)
%
\end{lyxcode}%
%
\item[Description:]%
Skips the 'ItemIndex' test.
%%
\item[Parameters:]~
\begin{description}%
\item[\texttt{a\_db}:]
 One or more tests\_db objects in an array.
\item[\texttt{num\_bins}:]
 Number of histogram bins (Optional, default=100), or

vector of histogram bin centers.
\end{description}%
%
\item[Returns:
]~

	t\_hists: An array of histograms for each test in a\_db.
%
%
\item[See also:]%
\hyperlink{ref_params_tests_profile}{\texttt{params\_tests\_profile}}%
\ (p.~\pageref{ref_params_tests_profile})%
\index[funcref]{params_tests_profile@\fidxl{params\_tests\_profile}}%
%
\item[Author:]%
Cengiz Gunay <cgunay@emory.edu>, 2004/10/17
%
\end{description}
\methodline%
\subsubsection[Method \texttt{unionCat}]{Method \texttt{params\_tests\_db/unionCat}}%
\index[funcref]{params_tests_db@\fidxl{params\_tests\_db}!unionCat@\fidxl{unionCat}}%
\label{ref_params_tests_db__unionCat}%
\hypertarget{ref_params_tests_db__unionCat}{}%
\begin{description}
\item[Summary:]Vertically concatenate two or more databases with different parameters or tests.
%
\item[Usage:]~%
\begin{lyxcode}%
a\_db = unionCat(db, with\_db, ...)
%
\end{lyxcode}%
%
\item[Description:]%
The parameters and tests in the result are a union of both. Adds 0 for
 parameter and NaN for tests in the rows which didn't have the additional
 columns before.
%%
\item[Parameters:]~

db, with\_db, ...: tests\_db objects to be concatenated together.
%
%
%
%
\item[Author:]%
Li Su, 2008-04-10
%
\end{description}
\methodline%
\subsubsection[Method \texttt{unionCatTwo}]{Method \texttt{params\_tests\_db/unionCatTwo}}%
\index[funcref]{params_tests_db@\fidxl{params\_tests\_db}!unionCatTwo@\fidxl{unionCatTwo}}%
\label{ref_params_tests_db__unionCatTwo}%
\hypertarget{ref_params_tests_db__unionCatTwo}{}%
\begin{description}
%
\item[Usage:]~%
\begin{lyxcode}%
a\_db = unionCat(db, with\_db, props)
%
\end{lyxcode}%
%
\item[Description:]%
The parameters and tests in the result are a union of both. Adds 0 for
 parameter and NaN for tests in the rows which didn't have the additional
 columns before.
%%
\item[Parameters:]~
\begin{description}%
\item[\texttt{db, with\_db}:]
 tests\_db objects to be concatenated together.
\item[\texttt{props}:]
 A structure with any optional properties.
\begin{description}%
\item[\texttt{offsetTracesets}:]
 If 1, make sure the TracesetIndex column is

offset to non-overlapping values in the concatenated
databases (default=0).
\end{description}%
\end{description}%
%
%
%
%
\item[Author:]%
Li Su, 2008-04-10
%
\end{description}
\methodline%
\subsubsection[Method \texttt{varyParams}]{Method \texttt{params\_tests\_db/varyParams}}%
\index[funcref]{params_tests_db@\fidxl{params\_tests\_db}!varyParams@\fidxl{varyParams}}%
\label{ref_params_tests_db__varyParams}%
\hypertarget{ref_params_tests_db__varyParams}{}%
\begin{description}
\item[Summary:]Varies chosen parameters in all rows by given levels.
%
\item[Usage:]~%
\begin{lyxcode}%
a\_params\_db = varyParamsOneRow(a\_db, params, levels, props)
%
\end{lyxcode}%
%
\item[Description:]%
Produces new rows by either multiplying or replacing the desired params
 with each value in levels. Thus, the newly created parameter db will be
 size of levels times bigger. Columns other than parameers will be
 pruned. Then, writeParFile can be used to generate a parameter file
 from this DB to drive new simulations.
%%
\item[Parameters:]~
\begin{description}%
\item[\texttt{a\_db}:]
 A params\_tests\_db object.
\item[\texttt{params}:]
 Parameters to be varied (see tests2cols).
\item[\texttt{levels}:]
 Column vector of parameter values to multiply (1=unity)

or to replace parameters with (see 'replace' prop).
\item[\texttt{props}:]
 A structure with any optional properties.
\begin{description}%
\item[\texttt{replace}:]
 Replace parameter values with levels instead of scaling.
\end{description}%
\end{description}%
%
\item[Returns:
]~

   a\_params\_db: A db only with params.
%
\item[Example:]~
\begin{lyxcode} Blocks NaF from 0%-100% with 10% increments:
\\%
 >> naf\_rows\_db = varyParams(a\_db(desired\_row, :), 'NaF', 0:0.1:1);
\\%
\end{lyxcode}
%
\item[See also:]%
\hyperlink{ref_writeParFile}{\texttt{writeParFile}}%
\ (p.~\pageref{ref_writeParFile})%
\index[funcref]{writeParFile@\fidxl{writeParFile}}%
, \hyperlink{ref_ranked_db__blockedDistances}{\texttt{ranked\_db/blockedDistances}}%
\ (p.~\pageref{ref_ranked_db__blockedDistances})%
\index[funcref]{ranked_db@\fidxl{ranked\_db}!blockedDistances@\fidxl{blockedDistances}}%
, \hyperlink{ref_getParamRowIndices}{\texttt{getParamRowIndices}}%
\ (p.~\pageref{ref_getParamRowIndices})%
\index[funcref]{getParamRowIndices@\fidxl{getParamRowIndices}}%
%
\item[Author:]%
Cengiz Gunay <cgunay@emory.edu>, 2006/02/16
%
\end{description}
\methodline%
\subsubsection[Method \texttt{writeParFile}]{Method \texttt{params\_tests\_db/writeParFile}}%
\index[funcref]{params_tests_db@\fidxl{params\_tests\_db}!writeParFile@\fidxl{writeParFile}}%
\label{ref_params_tests_db__writeParFile}%
\hypertarget{ref_params_tests_db__writeParFile}{}%
\begin{description}
\item[Summary:]Creates or appends to text file all the parameter values in a\_db.
%
\item[Usage:]~%
\begin{lyxcode}%
writeParFile(a\_db, filename, props)
%
\end{lyxcode}%
%
\item[Description:]%
Creates a text file that has a set of parameter values for each row. The
 first line is a header that contains number of parameters and total
 rows. If the file exists, the data is appended, but the header is NOT
 updated for efficiency considerations. Optionally, a parameter description
 file can be created that contains one parameter name per row. These files
 can be processed with various utilities to control simulations.
%%
\item[Parameters:]~
\begin{description}%
\item[\texttt{a\_db}:]
 A params\_tests\_db object.
\item[\texttt{filename}:]
 Genesis parameter file to be created.
\item[\texttt{props}:]
 A structure with any optional properties.
\begin{description}%
\item[\texttt{trialStart}:]
 If given, adds/replaces the trial parameter and counts forward.
\item[\texttt{makeParamDesc}:]
 If 1, put the parameter names in a parameter description file with

with a .txt extension.
\item[\texttt{noAppend}:]
 If given, do not append to file even if it exists
\end{description}%
\end{description}%
%
\item[Returns:
]~

   nothing.
%
\item[Example:]~
\begin{lyxcode}>> naf\_rows\_db = scanParamAllRows(a\_db(desired\_rows, :), 'NaF', 0, 1000, 100);
\\%
>> writeParFile(naf\_rows\_db, 'naf.par')
\\%
\end{lyxcode}
%
\item[See also:]%
\hyperlink{ref_scanParamAllRows}{\texttt{scanParamAllRows}}%
\ (p.~\pageref{ref_scanParamAllRows})%
\index[funcref]{scanParamAllRows@\fidxl{scanParamAllRows}}%
, \hyperlink{ref_scaleParamsOneRow}{\texttt{scaleParamsOneRow}}%
\ (p.~\pageref{ref_scaleParamsOneRow})%
\index[funcref]{scaleParamsOneRow@\fidxl{scaleParamsOneRow}}%
, \hyperlink{ref_https:__}{\texttt{https://github.com/cengique/param-search-neuro}}%
\ (p.~\pageref{ref_https:__})%
\index[funcref]{https:@\fidxl{https:}!@\fidxl{}}%
%
\item[Author:]%
Cengiz Gunay <cgunay@emory.edu>, 2005/03/13
%
\end{description}
\methodline%
\subsection{Class \texttt{params\_tests\_fileset}}%
\index[funcref]{params_tests_fileset@\fidxl{params\_tests\_fileset}|boldhyperpage}%
\label{ref_params_tests_fileset}%
\hypertarget{ref_params_tests_fileset}{}%
\subsubsection[Constructor \texttt{params\_tests\_fileset}]{Constructor \texttt{params\_tests\_fileset/params\_tests\_fileset}}%
\index[funcref]{params_tests_fileset@\fidxl{params\_tests\_fileset}!params_tests_fileset@\fidxl{params\_tests\_fileset}}%
\label{ref_params_tests_fileset__params_tests_fileset}%
\hypertarget{ref_params_tests_fileset__params_tests_fileset}{}%
\begin{description}
\item[Summary:]Description of a set of data files of raw data varying with parameter values.
%
\item[Usage:]~%
\begin{lyxcode}%
obj = params\_tests\_fileset(file\_pattern, dt, dy, id, props)
%
\end{lyxcode}%
%
\item[Description:]%
This is a subclass of params\_tests\_dataset. This class is used to
 generate params\_tests\_db objects and keep a connection to the raw
 data files. This class only keeps names of files and loads raw data
 files whenever it's requested. A database object can easily be
 generated using the convertion methods.  Most methods defined here
 can be used as-is, however some should be overloaded in subclasses.
 The specific methods are loadItemProfile.
%%
\item[Parameters:]~
\begin{description}%
\item[\texttt{file\_pattern}:]
 File pattern, or cell array of patterns, matching all 

files to be loaded.
\item[\texttt{dt}:]
 Time resolution [s]
\item[\texttt{dy}:]
 y-axis resolution [ISI (V, A, etc.)]
\item[\texttt{id}:]
 An identification string
\item[\texttt{props}:]
 A structure with any optional properties.
\begin{description}%
\item[\texttt{num\_params}:]
 Number of parameters that appear in filenames

(auto-detected by default; see props for
parseFilenameNamesVals).
\item[\texttt{fileParamsRegexp}:]
 Regular expression to find parameter names and

values instead of the default function
parseFilenameNamesVals. It will look for named capture
variables 'name' and 'val'. See the 'names' option to regexp.
\item[\texttt{param\_trial\_name}:]
 Use this name on the filename as the 'trial' parameter.
\item[\texttt{trial\_hash}:]
 Structure to get integer indices from non-integer trial numbers as key.
\item[\texttt{trialHashFunc}:]
 Produces structure key from trial number and precision (see num2str).
\item[\texttt{param\_row\_filename}:]
 If given, the 'trial' parameter will be used

to address rows from this file and acquire parameters.
\item[\texttt{param\_rows}:]
 Instead of a file, just give parameters in this matrix.
\item[\texttt{param\_desc\_filename}:]
 Contains the parameter range descriptions one per 

each row. The parameter names are acquired from this file.
\item[\texttt{param\_names}:]
 Cell array of parameter names corresponding to the 

param\_row\_filename columns can be specified as an alternative to
specifying param\_desc\_filename. These names are not for the 
parameters present in the data filename.
\item[\texttt{profile\_method\_name}:]
 It can be one of the profile-creating methods in this

class. E.g., 'trace\_profile', 'srp\_trace\_profile',
etc. OBSOLOTE: see loadItemProfileFunc prop in
params\_tests\_dataset.
(Others passed to params\_tests\_dataset and parseFilenameNamesVals)
\end{description}%
\end{description}%
%
\item[Returns a structure object with the following fields:
]~

   params\_tests\_dataset,
   path: The pathname to files.
%
%
\item[See also:]%
\hyperlink{ref_params_tests_db}{\texttt{params\_tests\_db}}%
\ (p.~\pageref{ref_params_tests_db})%
\index[funcref]{params_tests_db@\fidxl{params\_tests\_db}}%
, \hyperlink{ref_tests_db}{\texttt{tests\_db}}%
\ (p.~\pageref{ref_tests_db})%
\index[funcref]{tests_db@\fidxl{tests\_db}}%
, \hyperlink{ref_params_tests_dataset}{\texttt{params\_tests\_dataset}}%
\ (p.~\pageref{ref_params_tests_dataset})%
\index[funcref]{params_tests_dataset@\fidxl{params\_tests\_dataset}}%
, \hyperlink{ref_regexp}{\texttt{regexp}}%
\ (p.~\pageref{ref_regexp})%
\index[funcref]{regexp@\fidxl{regexp}}%
%
\item[Author:]%
Cengiz Gunay <cgunay@emory.edu>, 2004/09/09
%
\end{description}
\methodline%
\subsubsection[Method \texttt{addFiles}]{Method \texttt{params\_tests\_fileset/addFiles}}%
\index[funcref]{params_tests_fileset@\fidxl{params\_tests\_fileset}!addFiles@\fidxl{addFiles}}%
\label{ref_params_tests_fileset__addFiles}%
\hypertarget{ref_params_tests_fileset__addFiles}{}%
\begin{description}
\item[Summary:]Adds to existing list of files in set.
%
\item[Usage:]~%
\begin{lyxcode}%
[a\_fileset, index\_list] = addFiles(a\_fileset, file\_pattern, props)
%
\end{lyxcode}%
%
%
\item[Parameters:]~
\begin{description}%
\item[\texttt{a\_fileset}:]
 A params\_tests\_fileset object.
\item[\texttt{file\_pattern}:]
 File pattern, or cell array of patterns, matching additional files.
\item[\texttt{props}:]
 A structure with any optional properties.
\begin{description}%
\item[\texttt{param\_row\_filename}:]
 Update parameters from here. The 'trial' parameter is used

to address rows from this file and acquire parameters.
\end{description}%
\end{description}%
%
\item[Returns:
]~

	a\_fileset: The augmented fileset object.
	index\_list: The vector of index numbers of the new files added. Can be used
		to selectively load the new files into a DB using params\_test\_db.
%
%
\item[See also:]%
\hyperlink{ref_params_tests_fileset}{\texttt{params\_tests\_fileset}}%
\ (p.~\pageref{ref_params_tests_fileset})%
\index[funcref]{params_tests_fileset@\fidxl{params\_tests\_fileset}}%
, \hyperlink{ref_params_tests_dataset__params_test_db.}{\texttt{params\_tests\_dataset/params\_test\_db.}}%
\ (p.~\pageref{ref_params_tests_dataset__params_test_db.})%
\index[funcref]{params_tests_dataset@\fidxl{params\_tests\_dataset}!params_test_db.@\fidxl{params\_test\_db.}}%
%
\item[Author:]%
Cengiz Gunay <cgunay@emory.edu>, 2006/02/01
%
\end{description}
\methodline%
\subsubsection[Method \texttt{display}]{Method \texttt{params\_tests\_fileset/display}}%
\index[funcref]{params_tests_fileset@\fidxl{params\_tests\_fileset}!display@\fidxl{display}}%
\label{ref_params_tests_fileset__display}%
\hypertarget{ref_params_tests_fileset__display}{}%
\begin{description}
%
%
%
%
%
%
%
\item[Author:]%
Cengiz Gunay <cgunay@emory.edu>, 2004/08/04
%
\end{description}
\methodline%
\subsubsection[Method \texttt{get}]{Method \texttt{params\_tests\_fileset/get}}%
\index[funcref]{params_tests_fileset@\fidxl{params\_tests\_fileset}!get@\fidxl{get}}%
\label{ref_params_tests_fileset__get}%
\hypertarget{ref_params_tests_fileset__get}{}%
\begin{description}
\item[Summary:]Defines generic attribute retrieval for objects.
%
%
%
%
%
%
%
\item[Author:]%
Cengiz Gunay <cgunay@emory.edu>, 2004/09/14
%
\end{description}
\methodline%
\subsubsection[Method \texttt{getItemParams}]{Method \texttt{params\_tests\_fileset/getItemParams}}%
\index[funcref]{params_tests_fileset@\fidxl{params\_tests\_fileset}!getItemParams@\fidxl{getItemParams}}%
\label{ref_params_tests_fileset__getItemParams}%
\hypertarget{ref_params_tests_fileset__getItemParams}{}%
\begin{description}
\item[Summary:]Get the parameter values of a fileset item.
%
\item[Usage:]~%
\begin{lyxcode}%
params\_row = getItemParams(fileset, index)
%
\end{lyxcode}%
%
%
\item[Parameters:]~
\begin{description}%
\item[\texttt{fileset}:]
 A params\_tests\_fileset.
\item[\texttt{index}:]
 Index of item in fileset.
\end{description}%
%
\item[Returns:
]~

	params\_row: Parameter values in the same order of paramNames
%
%
\item[See also:]%
\hyperlink{ref_itemResultsRow}{\texttt{itemResultsRow}}%
\ (p.~\pageref{ref_itemResultsRow})%
\index[funcref]{itemResultsRow@\fidxl{itemResultsRow}}%
, \hyperlink{ref_params_tests_dataset}{\texttt{params\_tests\_dataset}}%
\ (p.~\pageref{ref_params_tests_dataset})%
\index[funcref]{params_tests_dataset@\fidxl{params\_tests\_dataset}}%
, \hyperlink{ref_paramNames}{\texttt{paramNames}}%
\ (p.~\pageref{ref_paramNames})%
\index[funcref]{paramNames@\fidxl{paramNames}}%
, \hyperlink{ref_testNames}{\texttt{testNames}}%
\ (p.~\pageref{ref_testNames})%
\index[funcref]{testNames@\fidxl{testNames}}%
%
\item[Author:]%
Cengiz Gunay <cgunay@emory.edu>, 2004/12/03
%
\end{description}
\methodline%
\subsubsection[Method \texttt{loadItemProfile}]{Method \texttt{params\_tests\_fileset/loadItemProfile}}%
\index[funcref]{params_tests_fileset@\fidxl{params\_tests\_fileset}!loadItemProfile@\fidxl{loadItemProfile}}%
\label{ref_params_tests_fileset__loadItemProfile}%
\hypertarget{ref_params_tests_fileset__loadItemProfile}{}%
\begin{description}
\item[Summary:]Loads a profile object from a raw data file in the fileset.
%
\item[Usage:]~%
\begin{lyxcode}%
a\_profile = loadItemProfile(fileset, file\_index)
%
\end{lyxcode}%
%
\item[Description:]%
First it looks for the prop loadItemProfileFunc to load the
 profile. Otherwise, it assumes fileset items can be loaded as traces. and
 runs props.profile\_method\_name if available, or simply creates a
 trace\_profile object. Subclasses should overload this function to load the
 specific profile object they desire. The profile class should define a
 getResults method which is used in the params\_tests\_dataset/itemResultsRow
 method.
%%
\item[Parameters:]~
\begin{description}%
\item[\texttt{fileset}:]
 A params\_tests\_fileset.
\item[\texttt{file\_index}:]
 Index of file in fileset.
\item[\texttt{params\_row}:]
 Parameter values for this item (default=[])
\item[\texttt{props}:]
 A structure with any optional properties.

(passed to loadItemProfileFunc)
\end{description}%
%
\item[Returns:
]~

   a\_profile: A results\_profile object that implements the getResults method.
%
%
\item[See also:]%
\hyperlink{ref_results_profile}{\texttt{results\_profile}}%
\ (p.~\pageref{ref_results_profile})%
\index[funcref]{results_profile@\fidxl{results\_profile}}%
, \hyperlink{ref_params_tests_dataset__itemResultsRow}{\texttt{params\_tests\_dataset/itemResultsRow}}%
\ (p.~\pageref{ref_params_tests_dataset__itemResultsRow})%
\index[funcref]{params_tests_dataset@\fidxl{params\_tests\_dataset}!itemResultsRow@\fidxl{itemResultsRow}}%
%
\item[Author:]%
Cengiz Gunay <cgunay@emory.edu>, 2004/09/14
%
\end{description}
\methodline%
\subsubsection[Method \texttt{paramNames}]{Method \texttt{params\_tests\_fileset/paramNames}}%
\index[funcref]{params_tests_fileset@\fidxl{params\_tests\_fileset}!paramNames@\fidxl{paramNames}}%
\label{ref_params_tests_fileset__paramNames}%
\hypertarget{ref_params_tests_fileset__paramNames}{}%
\begin{description}
\item[Summary:]Returns the ordered names of parameters for this fileset.
%
\item[Usage:]~%
\begin{lyxcode}%
param\_names = paramNames(fileset, item)
%
\end{lyxcode}%
%
\item[Description:]%
Looks at the filename of the first file to find the parameter names.
%%
\item[Parameters:]~
\begin{description}%
\item[\texttt{fileset}:]
 A params\_tests\_fileset.
\item[\texttt{item}:]
 (Optional) If given, read param names by loading item at this index.
\end{description}%
%
\item[Returns:
]~

	params\_names: Cell array with ordered parameter names.
%
%
\item[See also:]%
\hyperlink{ref_params_tests_fileset}{\texttt{params\_tests\_fileset}}%
\ (p.~\pageref{ref_params_tests_fileset})%
\index[funcref]{params_tests_fileset@\fidxl{params\_tests\_fileset}}%
, \hyperlink{ref_testNames}{\texttt{testNames}}%
\ (p.~\pageref{ref_testNames})%
\index[funcref]{testNames@\fidxl{testNames}}%
, \hyperlink{ref_parseFilenameNamesVals}{\texttt{parseFilenameNamesVals}}%
\ (p.~\pageref{ref_parseFilenameNamesVals})%
\index[funcref]{parseFilenameNamesVals@\fidxl{parseFilenameNamesVals}}%
%
\item[Author:]%
Cengiz Gunay <cgunay@emory.edu>, 2004/09/10
%
\end{description}
\methodline%
\subsubsection[Method \texttt{set}]{Method \texttt{params\_tests\_fileset/set}}%
\index[funcref]{params_tests_fileset@\fidxl{params\_tests\_fileset}!set@\fidxl{set}}%
\label{ref_params_tests_fileset__set}%
\hypertarget{ref_params_tests_fileset__set}{}%
\begin{description}
\item[Summary:]Generic method for setting object attributes.
%
%
%
%
%
%
%
\item[Author:]%
Cengiz Gunay <cgunay@emory.edu>, 2004/10/08
%
\end{description}
\methodline%
\subsubsection[Method \texttt{trace}]{Method \texttt{params\_tests\_fileset/trace}}%
\index[funcref]{params_tests_fileset@\fidxl{params\_tests\_fileset}!trace@\fidxl{trace}}%
\label{ref_params_tests_fileset__trace}%
\hypertarget{ref_params_tests_fileset__trace}{}%
\begin{description}
\item[Summary:]Loads a raw trace given a file\_index to this fileset.
%
\item[Usage:]~%
\begin{lyxcode}%
a\_trace = trace(fileset, file\_index)
%
\end{lyxcode}%
%
%
\item[Parameters:]~
\begin{description}%
\item[\texttt{fileset}:]
 A params\_tests\_fileset.
\item[\texttt{file\_index}:]
 Index of file in fileset.
\end{description}%
%
\item[Returns:
]~

	a\_trace: A trace object.
%
%
\item[See also:]%
\hyperlink{ref_trace}{\texttt{trace}}%
\ (p.~\pageref{ref_trace})%
\index[funcref]{trace@\fidxl{trace}}%
, \hyperlink{ref_params_tests_fileset}{\texttt{params\_tests\_fileset}}%
\ (p.~\pageref{ref_params_tests_fileset})%
\index[funcref]{params_tests_fileset@\fidxl{params\_tests\_fileset}}%
%
\item[Author:]%
Cengiz Gunay <cgunay@emory.edu>, 2004/09/13
%
\end{description}
\methodline%
\subsubsection[Method \texttt{trace\_profile}]{Method \texttt{params\_tests\_fileset/trace\_profile}}%
\index[funcref]{params_tests_fileset@\fidxl{params\_tests\_fileset}!trace_profile@\fidxl{trace\_profile}}%
\label{ref_params_tests_fileset__trace_profile}%
\hypertarget{ref_params_tests_fileset__trace_profile}{}%
\begin{description}
\item[Summary:]Loads a raw trace\_profile given a file\_index to this fileset.
%
\item[Usage:]~%
\begin{lyxcode}%
a\_trace\_profile = trace\_profile(fileset, file\_index)
%
\end{lyxcode}%
%
%
\item[Parameters:]~
\begin{description}%
\item[\texttt{fileset}:]
 A params\_tests\_fileset.
\item[\texttt{file\_index}:]
 Index of file in fileset.
\end{description}%
%
\item[Returns:
]~

	a\_trace\_profile: A trace\_profile object.
%
%
\item[See also:]%
\hyperlink{ref_trace_profile}{\texttt{trace\_profile}}%
\ (p.~\pageref{ref_trace_profile})%
\index[funcref]{trace_profile@\fidxl{trace\_profile}}%
, \hyperlink{ref_params_tests_fileset}{\texttt{params\_tests\_fileset}}%
\ (p.~\pageref{ref_params_tests_fileset})%
\index[funcref]{params_tests_fileset@\fidxl{params\_tests\_fileset}}%
%
\item[Author:]%
Cengiz Gunay <cgunay@emory.edu>, 2004/09/13
%
\end{description}
\methodline%
\subsection{Class \texttt{params\_tests\_profile}}%
\index[funcref]{params_tests_profile@\fidxl{params\_tests\_profile}|boldhyperpage}%
\label{ref_params_tests_profile}%
\hypertarget{ref_params_tests_profile}{}%
\subsubsection[Constructor \texttt{params\_tests\_profile}]{Constructor \texttt{params\_tests\_profile/params\_tests\_profile}}%
\index[funcref]{params_tests_profile@\fidxl{params\_tests\_profile}!params_tests_profile@\fidxl{params\_tests\_profile}}%
\label{ref_params_tests_profile__params_tests_profile}%
\hypertarget{ref_params_tests_profile__params_tests_profile}{}%
\begin{description}
\item[Summary:]Holds the results profile from a params\_tests\_db.
%
\item[Usage:]~%
\begin{lyxcode}%
a\_pt\_profile = params\_tests\_profile(results, a\_db, props)
%
\end{lyxcode}%
%
%
\item[Parameters:]~
\begin{description}%
\item[\texttt{a\_db}:]
 A params\_tests\_db object.
\item[\texttt{results}:]
 A structure containing test results.
\item[\texttt{props}:]
 A structure with any optional properties.
\end{description}%
%
\item[Returns a structure object with the following fields:
]~

	results\_profile: Contains results of tests.
	db: The params\_tests\_db.
	props.
%
%
\item[See also:]%
\hyperlink{ref_results_profile}{\texttt{results\_profile}}%
\ (p.~\pageref{ref_results_profile})%
\index[funcref]{results_profile@\fidxl{results\_profile}}%
, \hyperlink{ref_params_tests_db__params_tests_profile}{\texttt{params\_tests\_db/params\_tests\_profile}}%
\ (p.~\pageref{ref_params_tests_db__params_tests_profile})%
\index[funcref]{params_tests_db@\fidxl{params\_tests\_db}!params_tests_profile@\fidxl{params\_tests\_profile}}%
%
\item[Author:]%
Cengiz Gunay <cgunay@emory.edu>, 2004/10/13
%
\end{description}
\methodline%
\subsubsection[Method \texttt{get}]{Method \texttt{params\_tests\_profile/get}}%
\index[funcref]{params_tests_profile@\fidxl{params\_tests\_profile}!get@\fidxl{get}}%
\label{ref_params_tests_profile__get}%
\hypertarget{ref_params_tests_profile__get}{}%
\begin{description}
\item[Summary:]Defines generic attribute retrieval for objects.
%
%
%
%
%
%
%
\item[Author:]%
Cengiz Gunay <cgunay@emory.edu>, 2004/09/14
%
\end{description}
\methodline%
\subsection{Class \texttt{period}}%
\index[funcref]{period@\fidxl{period}|boldhyperpage}%
\label{ref_period}%
\hypertarget{ref_period}{}%
\subsubsection[Constructor \texttt{period}]{Constructor \texttt{period/period}}%
\index[funcref]{period@\fidxl{period}!period@\fidxl{period}}%
\label{ref_period__period}%
\hypertarget{ref_period__period}{}%
\begin{description}
\item[Summary:]Start and end times of a period in terms of the dt of the trace to which belongs.
%
\item[Usage:]~%
\begin{lyxcode}%
obj = period(start\_time, end\_time, dt)
%
\end{lyxcode}%
%
%
\item[Parameters:]~
\begin{description}%
\item[\texttt{start\_time, end\_time}:]
 Inclusive period [dt]. If end\_time is missing

and start\_time has two values, the second one is used as
end\_time. 
\item[\texttt{dt}:]
 If provided, interpret start\_time and end\_time in seconds

and convert them using this dt.
\end{description}%
%
\item[Returns a structure object with the following fields:
]~

   start\_time, end\_time.
%
%
\item[See also:]%
\hyperlink{ref_trace}{\texttt{trace}}%
\ (p.~\pageref{ref_trace})%
\index[funcref]{trace@\fidxl{trace}}%
, \hyperlink{ref_spikes}{\texttt{spikes}}%
\ (p.~\pageref{ref_spikes})%
\index[funcref]{spikes@\fidxl{spikes}}%
, \hyperlink{ref_spike_shape}{\texttt{spike\_shape}}%
\ (p.~\pageref{ref_spike_shape})%
\index[funcref]{spike_shape@\fidxl{spike\_shape}}%
%
\item[Author:]%
Cengiz Gunay <cgunay@emory.edu>, 2004/07/30
%
\end{description}
\methodline%
\subsubsection[Method \texttt{array}]{Method \texttt{period/array}}%
\index[funcref]{period@\fidxl{period}!array@\fidxl{array}}%
\label{ref_period__array}%
\hypertarget{ref_period__array}{}%
\begin{description}
\item[Summary:]Converts the period to an index array. 
%
\item[Usage:]~%
\begin{lyxcode}%
an\_array = array(period)
%
\end{lyxcode}%
%
%
\item[Parameters:]~
\begin{description}%
\item[\texttt{period}:]
 A period object
\end{description}%
%
\item[Returns:
]~

   an\_array: An array of dt indices contained within the period.
%
%
\item[See also:]%
\hyperlink{ref_period}{\texttt{period}}%
\ (p.~\pageref{ref_period})%
\index[funcref]{period@\fidxl{period}}%
, \hyperlink{ref_trace}{\texttt{trace}}%
\ (p.~\pageref{ref_trace})%
\index[funcref]{trace@\fidxl{trace}}%
%
\item[Author:]%
Cengiz Gunay <cengique@users.sf.net>, 2010/03/29
%
\end{description}
\methodline%
\subsubsection[Method \texttt{char}]{Method \texttt{period/char}}%
\index[funcref]{period@\fidxl{period}!char@\fidxl{char}}%
\label{ref_period__char}%
\hypertarget{ref_period__char}{}%
\begin{description}
%
%
%
%
%
%
%
\item[Author:]%
Cengiz Gunay <cgunay@emory.edu>, 2004/08/04
%
\end{description}
\methodline%
\subsubsection[Method \texttt{display}]{Method \texttt{period/display}}%
\index[funcref]{period@\fidxl{period}!display@\fidxl{display}}%
\label{ref_period__display}%
\hypertarget{ref_period__display}{}%
\begin{description}
%
%
%
%
%
%
%
\item[Author:]%
Cengiz Gunay <cgunay@emory.edu>, 2004/08/04
%
\end{description}
\methodline%
\subsubsection[Method \texttt{get}]{Method \texttt{period/get}}%
\index[funcref]{period@\fidxl{period}!get@\fidxl{get}}%
\label{ref_period__get}%
\hypertarget{ref_period__get}{}%
\begin{description}
\item[Summary:]Defines generic attribute retrieval for objects.
%
%
%
%
%
%
%
\item[Author:]%
Cengiz Gunay <cgunay@emory.edu>, 2004/09/14
%
\end{description}
\methodline%
\subsubsection[Method \texttt{set}]{Method \texttt{period/set}}%
\index[funcref]{period@\fidxl{period}!set@\fidxl{set}}%
\label{ref_period__set}%
\hypertarget{ref_period__set}{}%
\begin{description}
\item[Summary:]Generic method for setting object attributes.
%
%
%
%
%
%
%
\item[Author:]%
Cengiz Gunay <cgunay@emory.edu>, 2004/10/08
%
\end{description}
\methodline%
\subsubsection[Method \texttt{SpikeTimesinPeriod}]{Method \texttt{period/SpikeTimesinPeriod}}%
\index[funcref]{period@\fidxl{period}!SpikeTimesinPeriod@\fidxl{SpikeTimesinPeriod}}%
\label{ref_period__SpikeTimesinPeriod}%
\hypertarget{ref_period__SpikeTimesinPeriod}{}%
\begin{description}
%
\item[Usage:]~%
\begin{lyxcode}%
SpkTimes=Interval(times, period)
%
\end{lyxcode}%
%
%
\item[Parameters:]~
\begin{description}%
\item[\texttt{times}:]
 an array of spike times.
\item[\texttt{period}:]
 A period object
\end{description}%
%
\item[Returns:
]~

	the\_period: The cropped set of spike times that fall within a period.
%
%
\item[See also:]%
\hyperlink{ref_period}{\texttt{period}}%
\ (p.~\pageref{ref_period})%
\index[funcref]{period@\fidxl{period}}%
, \hyperlink{ref_cip_trace}{\texttt{cip\_trace}}%
\ (p.~\pageref{ref_cip_trace})%
\index[funcref]{cip_trace@\fidxl{cip\_trace}}%
, \hyperlink{ref_trace}{\texttt{trace}}%
\ (p.~\pageref{ref_trace})%
\index[funcref]{trace@\fidxl{trace}}%
, \hyperlink{ref_spikes}{\texttt{spikes}}%
\ (p.~\pageref{ref_spikes})%
\index[funcref]{spikes@\fidxl{spikes}}%
%
\item[Author:]%
Tom Sangrey, 2006/01/26
%
\end{description}
\methodline%
\subsubsection[Method \texttt{subsref}]{Method \texttt{period/subsref}}%
\index[funcref]{period@\fidxl{period}!subsref@\fidxl{subsref}}%
\label{ref_period__subsref}%
\hypertarget{ref_period__subsref}{}%
\begin{description}
\item[Summary:]Defines generic indexing for objects.
%
%
%
%
%
%
%
%
\end{description}
\methodline%
\subsection{Class \texttt{physiol\_bundle}}%
\index[funcref]{physiol_bundle@\fidxl{physiol\_bundle}|boldhyperpage}%
\label{ref_physiol_bundle}%
\hypertarget{ref_physiol_bundle}{}%
\subsubsection[Constructor \texttt{physiol\_bundle}]{Constructor \texttt{physiol\_bundle/physiol\_bundle}}%
\index[funcref]{physiol_bundle@\fidxl{physiol\_bundle}!physiol_bundle@\fidxl{physiol\_bundle}}%
\label{ref_physiol_bundle__physiol_bundle}%
\hypertarget{ref_physiol_bundle__physiol_bundle}{}%
\begin{description}
\item[Summary:]The physiology dataset and the DB created from it bundled together.
%
\item[Usage:]~%
\begin{lyxcode}%
a\_bundle = physiol\_bundle(a\_cell, props)
%
\end{lyxcode}%
%
\item[Description:]%
This is a subclass of dataset\_db\_bundle, specialized for physiology datasets. 
%%
\item[Parameters:]~
\begin{description}%
\item[\texttt{a\_cell}:]
 A cell array that contains the following elements:
\item[\texttt{a\_dataset}:]
 A cell-enclosed physiol\_cip\_traceset\_fileset object.
\item[\texttt{a\_db}:]
 The raw params\_tests\_db object created from the dataset. 

It only needs to have the pAcip, pAbias, TracesetIndex, and ItemIndex columns.
\item[\texttt{a\_joined\_db}:]
 The one-treatment-per-line DB created from the raw DB.
\item[\texttt{props}:]
 A structure with any optional properties.
\begin{description}%
\item[\texttt{controlDB}:]
 Use this as the ontrol DB rather than computing.
\end{description}%
\end{description}%
%
\item[Returns a structure object with the following fields:
]~

	dataset\_db\_bundle, 
	joined\_control\_db: DB of control neurons (no pharmacological applications).
%
%
\item[See also:]%
\hyperlink{ref_dataset_db_bundle}{\texttt{dataset\_db\_bundle}}%
\ (p.~\pageref{ref_dataset_db_bundle})%
\index[funcref]{dataset_db_bundle@\fidxl{dataset\_db\_bundle}}%
, \hyperlink{ref_tests_db}{\texttt{tests\_db}}%
\ (p.~\pageref{ref_tests_db})%
\index[funcref]{tests_db@\fidxl{tests\_db}}%
, \hyperlink{ref_params_tests_dataset}{\texttt{params\_tests\_dataset}}%
\ (p.~\pageref{ref_params_tests_dataset})%
\index[funcref]{params_tests_dataset@\fidxl{params\_tests\_dataset}}%
%
\item[Author:]%
Cengiz Gunay <cgunay@emory.edu>, 2005/12/13
%
\end{description}
\methodline%
\subsubsection[Method \texttt{bestMatchAllNeurons}]{Method \texttt{physiol\_bundle/bestMatchAllNeurons}}%
\index[funcref]{physiol_bundle@\fidxl{physiol\_bundle}!bestMatchAllNeurons@\fidxl{bestMatchAllNeurons}}%
\label{ref_physiol_bundle__bestMatchAllNeurons}%
\hypertarget{ref_physiol_bundle__bestMatchAllNeurons}{}%
\begin{description}
\item[Summary:]Finds the best match among given database for each physiology neuron.
%
\item[Usage:]~%
\begin{lyxcode}%
all\_ranks\_db = bestMatchAllNeurons(p\_bundle, joined\_db, props)
%
\end{lyxcode}%
%
\item[Description:]%
Returns a database of best matching entries from joined\_db for each
 entry in p\_bundle.joined\_control\_db.
%%
\item[Parameters:]~
\begin{description}%
\item[\texttt{p\_bundle}:]
 A physiol\_bundle object.
\item[\texttt{joined\_db}:]
 A database with neuron representations to rank against

neurons.
\item[\texttt{props}:]
 A structure with any optional properties.

(passed to rankMatching)
\end{description}%
%
\item[Returns:
]~

	all\_ranks\_db: DB of best matching from joined\_db. Each row
		corresponds to p\_bundle.joined\_control\_db rows.
%
\item[Example:]~
\begin{lyxcode} >> all\_ranks\_db = ...
\\%
        bestMatchAllNeurons(constrainedMeasuresPreset(pbundle2, 6), mbundle\_maxcond.joined\_db)
\\%
 >> plotXRows(all\_ranks\_db, 'Distance', 'maxcond DB distance per neuron', 'maxcond', ...
\\%
              struct('LineStyle', '-', 'quiet', 1, 'PaperPosition', [0 0 4 3]))
\\%
\end{lyxcode}
%
\item[See also:]%
\hyperlink{ref_tests_db__rankMatching}{\texttt{tests\_db/rankMatching}}%
\ (p.~\pageref{ref_tests_db__rankMatching})%
\index[funcref]{tests_db@\fidxl{tests\_db}!rankMatching@\fidxl{rankMatching}}%
, \hyperlink{ref_tests_db__matchingRow}{\texttt{tests\_db/matchingRow}}%
\ (p.~\pageref{ref_tests_db__matchingRow})%
\index[funcref]{tests_db@\fidxl{tests\_db}!matchingRow@\fidxl{matchingRow}}%
%
\item[Author:]%
Cengiz Gunay <cgunay@emory.edu>, 2007/05/24
%
\end{description}
\methodline%
\subsubsection[Method \texttt{constrainedMeasuresPreset}]{Method \texttt{physiol\_bundle/constrainedMeasuresPreset}}%
\index[funcref]{physiol_bundle@\fidxl{physiol\_bundle}!constrainedMeasuresPreset@\fidxl{constrainedMeasuresPreset}}%
\label{ref_physiol_bundle__constrainedMeasuresPreset}%
\hypertarget{ref_physiol_bundle__constrainedMeasuresPreset}{}%
\begin{description}
\item[Summary:]Returns a physiol\_bundle with constrained measures according to chosen preset.
%
\item[Usage:]~%
\begin{lyxcode}%
[a\_pbundle test\_names] = constrainedMeasuresPreset(a\_pbundle, preset, props)
%
\end{lyxcode}%
%
%
\item[Parameters:]~
\begin{description}%
\item[\texttt{a\_pbundle}:]
 A physiol\_cip\_traceset\_fileset object.
\item[\texttt{preset}:]
 Choose preset measure list (default=1).
\item[\texttt{props}:]
 A structure with any optional properties.
\end{description}%
%
\item[Returns:
]~

	a\_pbundle: One or more cip\_trace object that holds the raw data.
%
%
\item[See also:]%
\hyperlink{ref_loadItemProfile}{\texttt{loadItemProfile}}%
\ (p.~\pageref{ref_loadItemProfile})%
\index[funcref]{loadItemProfile@\fidxl{loadItemProfile}}%
, \hyperlink{ref_physiol_cip_traceset__cip_trace}{\texttt{physiol\_cip\_traceset/cip\_trace}}%
\ (p.~\pageref{ref_physiol_cip_traceset__cip_trace})%
\index[funcref]{physiol_cip_traceset@\fidxl{physiol\_cip\_traceset}!cip_trace@\fidxl{cip\_trace}}%
%
\item[Author:]%
Cengiz Gunay <cgunay@emory.edu>, 2006/01/19
%
\end{description}
\methodline%
\subsubsection[Method \texttt{ctFromRows}]{Method \texttt{physiol\_bundle/ctFromRows}}%
\index[funcref]{physiol_bundle@\fidxl{physiol\_bundle}!ctFromRows@\fidxl{ctFromRows}}%
\label{ref_physiol_bundle__ctFromRows}%
\hypertarget{ref_physiol_bundle__ctFromRows}{}%
\begin{description}
\item[Summary:]Loads a cip\_trace object from a raw data file in the a\_pbundle.
%
\item[Usage:]~%
\begin{lyxcode}%
a\_cip\_trace = ctFromRows(a\_pbundle, a\_db|traceset\_idx, cip\_levels, props)
%
\end{lyxcode}%
%
%
\item[Parameters:]~
\begin{description}%
\item[\texttt{a\_pbundle}:]
 A physiol\_cip\_traceset\_fileset object.
\item[\texttt{a\_db}:]
 A DB created by this fileset to read the traceset indices from.
\item[\texttt{traceset\_idx}:]
 A column vector with traceset indices.
\item[\texttt{cip\_levels}:]
 A column vector of CIP-levels to be loaded.
\item[\texttt{props}:]
 A structure with any optional properties.
\begin{description}%
\item[\texttt{traces}:]
 column vector of trace indices to load.
\item[\texttt{showParamsList}:]
 Cell array of params or treatments to include in the id field.
\end{description}%
\end{description}%
%
\item[Returns:
]~

	a\_cip\_trace: One or more cip\_trace object that holds the raw data.
%
%
\item[See also:]%
\hyperlink{ref_loadItemProfile}{\texttt{loadItemProfile}}%
\ (p.~\pageref{ref_loadItemProfile})%
\index[funcref]{loadItemProfile@\fidxl{loadItemProfile}}%
, \hyperlink{ref_physiol_cip_traceset__cip_trace}{\texttt{physiol\_cip\_traceset/cip\_trace}}%
\ (p.~\pageref{ref_physiol_cip_traceset__cip_trace})%
\index[funcref]{physiol_cip_traceset@\fidxl{physiol\_cip\_traceset}!cip_trace@\fidxl{cip\_trace}}%
%
\item[Author:]%
Cengiz Gunay <cgunay@emory.edu>, 2005/07/13
%
\end{description}
\methodline%
\subsubsection[Method \texttt{get}]{Method \texttt{physiol\_bundle/get}}%
\index[funcref]{physiol_bundle@\fidxl{physiol\_bundle}!get@\fidxl{get}}%
\label{ref_physiol_bundle__get}%
\hypertarget{ref_physiol_bundle__get}{}%
\begin{description}
\item[Summary:]Defines generic attribute retrieval for objects.
%
%
%
%
%
%
%
\item[Author:]%
Cengiz Gunay <cgunay@emory.edu>, 2004/09/14
%
\end{description}
\methodline%
\subsubsection[Method \texttt{getNeuronLabel}]{Method \texttt{physiol\_bundle/getNeuronLabel}}%
\index[funcref]{physiol_bundle@\fidxl{physiol\_bundle}!getNeuronLabel@\fidxl{getNeuronLabel}}%
\label{ref_physiol_bundle__getNeuronLabel}%
\hypertarget{ref_physiol_bundle__getNeuronLabel}{}%
\begin{description}
\item[Summary:]Constructs the neuron label from dataset.
%
\item[Usage:]~%
\begin{lyxcode}%
a\_label = getNeuronLabel(a\_bundle, traceset\_index, props)
%
\end{lyxcode}%
%
%
\item[Parameters:]~
\begin{description}%
\item[\texttt{a\_bundle}:]
 A physiol\_cip\_traceset\_fileset object.
\item[\texttt{traceset\_index}:]
 The traceset index of neuron.
\item[\texttt{props}:]
 A structure with any optional properties.
\end{description}%
%
\item[Returns:
]~

	a\_label: A string label identifying selected neuron in bundle.
%
%
\item[See also:]%
\hyperlink{ref_dataset_db_bundle}{\texttt{dataset\_db\_bundle}}%
\ (p.~\pageref{ref_dataset_db_bundle})%
\index[funcref]{dataset_db_bundle@\fidxl{dataset\_db\_bundle}}%
%
\item[Author:]%
Cengiz Gunay <cgunay@emory.edu>, 2006/05/05
%
\end{description}
\methodline%
\subsubsection[Method \texttt{getNeuronRowIndex}]{Method \texttt{physiol\_bundle/getNeuronRowIndex}}%
\index[funcref]{physiol_bundle@\fidxl{physiol\_bundle}!getNeuronRowIndex@\fidxl{getNeuronRowIndex}}%
\label{ref_physiol_bundle__getNeuronRowIndex}%
\hypertarget{ref_physiol_bundle__getNeuronRowIndex}{}%
\begin{description}
\item[Summary:]Returns the neuron index from bundle.
%
\item[Usage:]~%
\begin{lyxcode}%
a\_row\_index = getNeuronRowIndex(a\_bundle, traceset\_index, props)
%
\end{lyxcode}%
%
%
\item[Parameters:]~
\begin{description}%
\item[\texttt{a\_bundle}:]
 A physiol\_bundle object.
\item[\texttt{traceset\_index}:]
 The TracesetIndex number of neuron, or a DB row containing this.
\item[\texttt{props}:]
 A structure with any optional properties.
\end{description}%
%
\item[Returns:
]~

	a\_row\_index: A row index of neuron in a\_bundle.joined\_db.
%
%
\item[See also:]%
\hyperlink{ref_dataset_db_bundle__getNeuronRowIndex}{\texttt{dataset\_db\_bundle/getNeuronRowIndex}}%
\ (p.~\pageref{ref_dataset_db_bundle__getNeuronRowIndex})%
\index[funcref]{dataset_db_bundle@\fidxl{dataset\_db\_bundle}!getNeuronRowIndex@\fidxl{getNeuronRowIndex}}%
%
\item[Author:]%
Cengiz Gunay <cgunay@emory.edu>, 2006/06/09
%
\end{description}
\methodline%
\subsubsection[Method \texttt{matchingControlNeuron}]{Method \texttt{physiol\_bundle/matchingControlNeuron}}%
\index[funcref]{physiol_bundle@\fidxl{physiol\_bundle}!matchingControlNeuron@\fidxl{matchingControlNeuron}}%
\label{ref_physiol_bundle__matchingControlNeuron}%
\hypertarget{ref_physiol_bundle__matchingControlNeuron}{}%
\begin{description}
\item[Summary:]Creates a criterion database for matching the neuron at traceset\_index.
%
\item[Usage:]~%
\begin{lyxcode}%
a\_crit\_bundle = matchingControlNeuron(a\_bundle, neuron\_id, props)
%
\end{lyxcode}%
%
\item[Description:]%
Copies selected test values from row as the first row into the 
 criterion db. Adds a second row for the STD of each column in the db.
%%
\item[Parameters:]~
\begin{description}%
\item[\texttt{a\_bundle}:]
 A physiol\_bundle object.
\item[\texttt{neuron\_id}:]
 A NeuronId of the neuron to match.
\item[\texttt{props}:]
 A structure with any optional properties.
\end{description}%
%
\item[Returns:
]~

	a\_crit\_bundle: A tests\_db with two rows for values and STDs.
%
\item[Example:]~
\begin{lyxcode}        Matches gpd0421c from cip\_traces\_all\_axoclamp.txt:
\\%
        >> a\_crit\_bundle = matchingControlNeuron(pbundle, 33)
\\%
        (see example in matchingRow)
\\%
\end{lyxcode}
%
\item[See also:]%
\hyperlink{ref_rankMatching}{\texttt{rankMatching}}%
\ (p.~\pageref{ref_rankMatching})%
\index[funcref]{rankMatching@\fidxl{rankMatching}}%
, \hyperlink{ref_tests_db}{\texttt{tests\_db}}%
\ (p.~\pageref{ref_tests_db})%
\index[funcref]{tests_db@\fidxl{tests\_db}}%
, \hyperlink{ref_tests2cols}{\texttt{tests2cols}}%
\ (p.~\pageref{ref_tests2cols})%
\index[funcref]{tests2cols@\fidxl{tests2cols}}%
%
\item[Author:]%
Cengiz Gunay <cgunay@emory.edu>, 2005/12/21
%
\end{description}
\methodline%
\subsubsection[Method \texttt{matchingRow}]{Method \texttt{physiol\_bundle/matchingRow}}%
\index[funcref]{physiol_bundle@\fidxl{physiol\_bundle}!matchingRow@\fidxl{matchingRow}}%
\label{ref_physiol_bundle__matchingRow}%
\hypertarget{ref_physiol_bundle__matchingRow}{}%
\begin{description}
\item[Summary:]Creates a criterion database for matching the neuron at traceset\_index.
%
\item[Usage:]~%
\begin{lyxcode}%
a\_crit\_db = matchingRow(p\_bundle, traceset\_index, props)
%
\end{lyxcode}%
%
\item[Description:]%
Copies selected test values from row as the first row into the 
 criterion db. Adds a second row for the STD of each column in the db.
%%
\item[Parameters:]~
\begin{description}%
\item[\texttt{p\_bundle}:]
 A physiol\_bundle object.
\item[\texttt{traceset\_index}:]
 A TracesetIndex of the neuron and treatments to match.
\item[\texttt{props}:]
 A structure with any optional properties.
\end{description}%
%
\item[Returns:
]~

	a\_crit\_db: A tests\_db with two rows for values and STDs.
%
\item[Example:]~
\begin{lyxcode}        physiol\_bundle has an overloaded matchingRow method that
\\%
        takes the TracesetIndex as argument:
\\%
        >> a\_crit\_bundle = matchingRow(pbundle, 61)
\\%
        >> a\_ranked\_bundle = rankMatching(mbundle, a\_crit\_bundle);
\\%
        >> printTeXFile(comparisonReport(a\_ranked\_bundle), 'my\_report.tex')
\\%
\end{lyxcode}
%
\item[See also:]%
\hyperlink{ref_rankMatching}{\texttt{rankMatching}}%
\ (p.~\pageref{ref_rankMatching})%
\index[funcref]{rankMatching@\fidxl{rankMatching}}%
, \hyperlink{ref_tests_db__matchingRow}{\texttt{tests\_db/matchingRow}}%
\ (p.~\pageref{ref_tests_db__matchingRow})%
\index[funcref]{tests_db@\fidxl{tests\_db}!matchingRow@\fidxl{matchingRow}}%
%
\item[Author:]%
Cengiz Gunay <cgunay@emory.edu>, 2005/12/21
%
\end{description}
\methodline%
\subsubsection[Method \texttt{mergeBundles}]{Method \texttt{physiol\_bundle/mergeBundles}}%
\index[funcref]{physiol_bundle@\fidxl{physiol\_bundle}!mergeBundles@\fidxl{mergeBundles}}%
\label{ref_physiol_bundle__mergeBundles}%
\hypertarget{ref_physiol_bundle__mergeBundles}{}%
\begin{description}
\item[Summary:]Merges two bundles together by adding w\_bundle to p\_bundle.
%
\item[Usage:]~%
\begin{lyxcode}%
p\_bundle = mergeBundles(p\_bundle, w\_bundle, props)
%
\end{lyxcode}%
%
%
\item[Parameters:]~
\begin{description}%
\item[\texttt{p\_bundle, w\_bundle}:]
 physiol\_bundle objects.
\item[\texttt{props}:]
 A structure with any optional properties.
\end{description}%
%
\item[Returns:
]~

	p\_bundle: The merged p\_bundle.
%
\item[Example:]~
\begin{lyxcode} >> p\_bundle = mergeBundles(pbundle, another\_bundle)
\\%
\end{lyxcode}
%
\item[See also:]%
\hyperlink{ref_rankMatching}{\texttt{rankMatching}}%
\ (p.~\pageref{ref_rankMatching})%
\index[funcref]{rankMatching@\fidxl{rankMatching}}%
, \hyperlink{ref_tests_db__mergeBundles}{\texttt{tests\_db/mergeBundles}}%
\ (p.~\pageref{ref_tests_db__mergeBundles})%
\index[funcref]{tests_db@\fidxl{tests\_db}!mergeBundles@\fidxl{mergeBundles}}%
%
\item[Author:]%
Cengiz Gunay <cgunay@emory.edu>, 2008/05/18
%
\end{description}
\methodline%
\subsubsection[Method \texttt{plotfICurveStats}]{Method \texttt{physiol\_bundle/plotfICurveStats}}%
\index[funcref]{physiol_bundle@\fidxl{physiol\_bundle}!plotfICurveStats@\fidxl{plotfICurveStats}}%
\label{ref_physiol_bundle__plotfICurveStats}%
\hypertarget{ref_physiol_bundle__plotfICurveStats}{}%
\begin{description}
\item[Summary:]Generates a f-I curve mean-std plot of physiology DB.
%
\item[Usage:]~%
\begin{lyxcode}%
a\_plot = plotfICurveStats(p\_bundle, title\_str, props)
%
\end{lyxcode}%
%
%
\item[Parameters:]~
\begin{description}%
\item[\texttt{p\_bundle}:]
 A physiol\_bundle object.
\item[\texttt{title\_str}:]
 (Optional) String to append to plot title.
\item[\texttt{props}:]
 A structure with any optional properties.
\begin{description}%
\item[\texttt{quiet}:]
 if given, no title is produced

(passed to plot\_superpose)
\end{description}%
\end{description}%
%
\item[Returns:
]~

	a\_plot: An f-I curve plot.
%
\item[Example:]~
\begin{lyxcode} >> plotFigure(plotfICurveStats(pbundle));
\\%
\end{lyxcode}
%
\item[See also:]%
\hyperlink{ref_dataset_db_bundle__plotfICurve}{\texttt{dataset\_db\_bundle/plotfICurve}}%
\ (p.~\pageref{ref_dataset_db_bundle__plotfICurve})%
\index[funcref]{dataset_db_bundle@\fidxl{dataset\_db\_bundle}!plotfICurve@\fidxl{plotfICurve}}%
, \hyperlink{ref_plot_abstract}{\texttt{plot\_abstract}}%
\ (p.~\pageref{ref_plot_abstract})%
\index[funcref]{plot_abstract@\fidxl{plot\_abstract}}%
, \hyperlink{ref_plot_superpose}{\texttt{plot\_superpose}}%
\ (p.~\pageref{ref_plot_superpose})%
\index[funcref]{plot_superpose@\fidxl{plot\_superpose}}%
%
\item[Author:]%
Cengiz Gunay <cgunay@emory.edu>, 2006/06/16
%
\end{description}
\methodline%
\subsubsection[Method \texttt{set}]{Method \texttt{physiol\_bundle/set}}%
\index[funcref]{physiol_bundle@\fidxl{physiol\_bundle}!set@\fidxl{set}}%
\label{ref_physiol_bundle__set}%
\hypertarget{ref_physiol_bundle__set}{}%
\begin{description}
\item[Summary:]Generic method for setting object attributes.
%
%
%
%
%
%
%
\item[Author:]%
Cengiz Gunay <cgunay@emory.edu>, 2004/10/08
%
\end{description}
\methodline%
\subsection{Class \texttt{physiol\_cip\_traceset}}%
\index[funcref]{physiol_cip_traceset@\fidxl{physiol\_cip\_traceset}|boldhyperpage}%
\label{ref_physiol_cip_traceset}%
\hypertarget{ref_physiol_cip_traceset}{}%
\subsubsection[Constructor \texttt{physiol\_cip\_traceset}]{Constructor \texttt{physiol\_cip\_traceset/physiol\_cip\_traceset}}%
\index[funcref]{physiol_cip_traceset@\fidxl{physiol\_cip\_traceset}!physiol_cip_traceset@\fidxl{physiol\_cip\_traceset}}%
\label{ref_physiol_cip_traceset__physiol_cip_traceset}%
\hypertarget{ref_physiol_cip_traceset__physiol_cip_traceset}{}%
\begin{description}
\item[Summary:]Dataset of cip traces from same PCDX file.
%
\item[Usage:]~%
\begin{lyxcode}%
obj = physiol\_cip\_traceset(trace\_str, data\_src, chaninfo, dt, dy, treatments, neuron\_id, props);
%
\end{lyxcode}%
%
\item[Description:]%
This is a subclass of params\_tests\_dataset. Each trace varies in bias, 
 pulse times and cip magnitude.
%%
\item[Parameters:]~
\begin{description}%
\item[\texttt{trace\_str}:]
 Trace list in the format for loadtraces or just a Matlab vector.
\item[\texttt{data\_src}:]
 Absolute path of PCDX data source.
\item[\texttt{chaninfo}:]
 4-element array containing vchan, ichan, vgain, igain
\begin{description}%
\item[\texttt{vchan, ichan}:]
 Current and voltage channels.
\item[\texttt{vgain, igain}:]
 External gain factors for voltage channel and current 

channel
(vgain does NOT include the 10X amplification from the Axoclamp,
so vgain = 1 would mean no additional amplification beyond the 10X.)
\end{description}%
\item[\texttt{dt}:]
 Time resolution [s].
\item[\texttt{dy}:]
 Y-axis resolution [V] or [A].
\item[\texttt{treatments}:]
 Structure containing the names and concentrations

of compounds.
\item[\texttt{neuron\_id}:]
 Neuron name.
\item[\texttt{props}:]
 A structure with any optional properties.
\begin{description}%
\item[\texttt{nsHDF5}:]
 For NeuroSAGE HDF5 files, processing is faster if the output

of ns\_open\_file is given here. Must be defined to allow
special NeuroSAGE processing.
\item[\texttt{profile\_method\_name}:]
 Use this cip\_trace method to return a

profile (Default: 'getProfileAllSpikes').
\item[\texttt{cip\_list}:]
 Vector of cip levels to which the current trace will be matched.

(All other props are passed to cip\_trace objects)
\end{description}%
\end{description}%
%
\item[Returns a structure object with the following fields:
]~

	params\_tests\_dataset,
	data\_src, ichan, vchan, vgain, igain, treatments.
%
%
\item[See also:]%
\hyperlink{ref_cip_traces}{\texttt{cip\_traces}}%
\ (p.~\pageref{ref_cip_traces})%
\index[funcref]{cip_traces@\fidxl{cip\_traces}}%
, \hyperlink{ref_params_tests_dataset}{\texttt{params\_tests\_dataset}}%
\ (p.~\pageref{ref_params_tests_dataset})%
\index[funcref]{params_tests_dataset@\fidxl{params\_tests\_dataset}}%
, \hyperlink{ref_params_tests_db}{\texttt{params\_tests\_db}}%
\ (p.~\pageref{ref_params_tests_db})%
\index[funcref]{params_tests_db@\fidxl{params\_tests\_db}}%
%
\item[Author:]%
Cengiz Gunay <cgunay@emory.edu> and Thomas Sangrey, 2005/01/17
%
\end{description}
\methodline%
\subsubsection[Method \texttt{CIPform}]{Method \texttt{physiol\_cip\_traceset/CIPform}}%
\index[funcref]{physiol_cip_traceset@\fidxl{physiol\_cip\_traceset}!CIPform@\fidxl{CIPform}}%
\label{ref_physiol_cip_traceset__CIPform}%
\hypertarget{ref_physiol_cip_traceset__CIPform}{}%
\begin{description}
\item[Summary:]Extracts current bias and pulse information from the current channel.
%
\item[Usage:]~%
\begin{lyxcode}%
[ciptype, on, off, finish, bias, pulse] = ns\_CIPform(traceset,trace\_index)
%
\end{lyxcode}%
%
%
\item[Parameters:]~
\begin{description}%
\item[\texttt{traceset}:]
 A physiol\_cip\_traceset object.
\item[\texttt{trace\_index}:]
 Index of item in traceset
\end{description}%
%
%
%
\item[See also:]%
\hyperlink{ref_cip_traces}{\texttt{cip\_traces}}%
\ (p.~\pageref{ref_cip_traces})%
\index[funcref]{cip_traces@\fidxl{cip\_traces}}%
, \hyperlink{ref_params_tests_dataset}{\texttt{params\_tests\_dataset}}%
\ (p.~\pageref{ref_params_tests_dataset})%
\index[funcref]{params_tests_dataset@\fidxl{params\_tests\_dataset}}%
, \hyperlink{ref_params_tests_db}{\texttt{params\_tests\_db}}%
\ (p.~\pageref{ref_params_tests_db})%
\index[funcref]{params_tests_db@\fidxl{params\_tests\_db}}%
%
\item[Author:]%
Thomas Sangrey, 2005
%
\end{description}
\methodline%
\subsubsection[Method \texttt{cip\_trace}]{Method \texttt{physiol\_cip\_traceset/cip\_trace}}%
\index[funcref]{physiol_cip_traceset@\fidxl{physiol\_cip\_traceset}!cip_trace@\fidxl{cip\_trace}}%
\label{ref_physiol_cip_traceset__cip_trace}%
\hypertarget{ref_physiol_cip_traceset__cip_trace}{}%
\begin{description}
\item[Summary:]Loads a cip\_trace object from a raw data file in the traceset.
%
\item[Usage:]~%
\begin{lyxcode}%
a\_cip\_trace = cip\_trace(traceset, trace\_index, props)
%
\end{lyxcode}%
%
%
\item[Parameters:]~
\begin{description}%
\item[\texttt{traceset}:]
 A physiol\_cip\_traceset object.
\item[\texttt{trace\_index}:]
 Index of file in traceset.
\item[\texttt{props}:]
 A structure with any optional properties.
\begin{description}%
\item[\texttt{showParamsList}:]
 Cell array of params to add to id field.
\item[\texttt{showName}:]
 Show the name of the cell in the id field (default=1).
\item[\texttt{TracesetIndex}:]
 Indicates in the id field.
\end{description}%
\end{description}%
%
\item[Returns:
]~

	a\_cip\_trace: A cip\_trace object that holds the raw data.
%
%
\item[See also:]%
\hyperlink{ref_itemResultsRow}{\texttt{itemResultsRow}}%
\ (p.~\pageref{ref_itemResultsRow})%
\index[funcref]{itemResultsRow@\fidxl{itemResultsRow}}%
, \hyperlink{ref_params_tests_fileset}{\texttt{params\_tests\_fileset}}%
\ (p.~\pageref{ref_params_tests_fileset})%
\index[funcref]{params_tests_fileset@\fidxl{params\_tests\_fileset}}%
, \hyperlink{ref_paramNames}{\texttt{paramNames}}%
\ (p.~\pageref{ref_paramNames})%
\index[funcref]{paramNames@\fidxl{paramNames}}%
, \hyperlink{ref_testNames}{\texttt{testNames}}%
\ (p.~\pageref{ref_testNames})%
\index[funcref]{testNames@\fidxl{testNames}}%
%
\item[Author:]%
Cengiz Gunay <cgunay@emory.edu>, 2005/07/13
%
\end{description}
\methodline%
\subsubsection[Method \texttt{cip\_trace\_profile}]{Method \texttt{physiol\_cip\_traceset/cip\_trace\_profile}}%
\index[funcref]{physiol_cip_traceset@\fidxl{physiol\_cip\_traceset}!cip_trace_profile@\fidxl{cip\_trace\_profile}}%
\label{ref_physiol_cip_traceset__cip_trace_profile}%
\hypertarget{ref_physiol_cip_traceset__cip_trace_profile}{}%
\begin{description}
\item[Summary:]Loads a cip\_trace\_profile object from a raw data file in the traceset.
%
\item[Usage:]~%
\begin{lyxcode}%
a\_profile = cip\_trace\_profile(traceset, trace\_index)
%
\end{lyxcode}%
%
%
\item[Parameters:]~
\begin{description}%
\item[\texttt{traceset}:]
 A physiol\_cip\_traceset object.
\item[\texttt{trace\_index}:]
 Index of file in traceset.
\end{description}%
%
\item[Returns:
]~

	a\_profile: A profile object that implements the getResults method.
%
%
\item[See also:]%
\hyperlink{ref_itemResultsRow}{\texttt{itemResultsRow}}%
\ (p.~\pageref{ref_itemResultsRow})%
\index[funcref]{itemResultsRow@\fidxl{itemResultsRow}}%
, \hyperlink{ref_params_tests_fileset}{\texttt{params\_tests\_fileset}}%
\ (p.~\pageref{ref_params_tests_fileset})%
\index[funcref]{params_tests_fileset@\fidxl{params\_tests\_fileset}}%
, \hyperlink{ref_paramNames}{\texttt{paramNames}}%
\ (p.~\pageref{ref_paramNames})%
\index[funcref]{paramNames@\fidxl{paramNames}}%
, \hyperlink{ref_testNames}{\texttt{testNames}}%
\ (p.~\pageref{ref_testNames})%
\index[funcref]{testNames@\fidxl{testNames}}%
%
\item[Author:]%
Cengiz Gunay <cgunay@emory.edu> and Thomas Sangrey, 2005/01/18
%
\end{description}
\methodline%
\subsubsection[Method \texttt{get}]{Method \texttt{physiol\_cip\_traceset/get}}%
\index[funcref]{physiol_cip_traceset@\fidxl{physiol\_cip\_traceset}!get@\fidxl{get}}%
\label{ref_physiol_cip_traceset__get}%
\hypertarget{ref_physiol_cip_traceset__get}{}%
\begin{description}
\item[Summary:]Defines generic attribute retrieval for objects.
%
%
%
%
%
%
%
\item[Author:]%
Cengiz Gunay <cgunay@emory.edu>, 2004/09/14
%
\end{description}
\methodline%
\subsubsection[Method \texttt{getItemParams}]{Method \texttt{physiol\_cip\_traceset/getItemParams}}%
\index[funcref]{physiol_cip_traceset@\fidxl{physiol\_cip\_traceset}!getItemParams@\fidxl{getItemParams}}%
\label{ref_physiol_cip_traceset__getItemParams}%
\hypertarget{ref_physiol_cip_traceset__getItemParams}{}%
\begin{description}
\item[Summary:]Get the parameter values of a dataset item.
%
\item[Usage:]~%
\begin{lyxcode}%
params\_row = getItemParams(dataset, index, a\_profile)
%
\end{lyxcode}%
%
%
\item[Parameters:]~
\begin{description}%
\item[\texttt{dataset}:]
 A params\_tests\_dataset.
\item[\texttt{index}:]
 Index of item in dataset.
\item[\texttt{a\_profile}:]
 cip\_trace\_profile object
\end{description}%
%
\item[Returns:
]~

	params\_row: Parameter values in the same order of paramNames
%
%
\item[See also:]%
\hyperlink{ref_itemResultsRow}{\texttt{itemResultsRow}}%
\ (p.~\pageref{ref_itemResultsRow})%
\index[funcref]{itemResultsRow@\fidxl{itemResultsRow}}%
, \hyperlink{ref_params_tests_dataset}{\texttt{params\_tests\_dataset}}%
\ (p.~\pageref{ref_params_tests_dataset})%
\index[funcref]{params_tests_dataset@\fidxl{params\_tests\_dataset}}%
, \hyperlink{ref_paramNames}{\texttt{paramNames}}%
\ (p.~\pageref{ref_paramNames})%
\index[funcref]{paramNames@\fidxl{paramNames}}%
, \hyperlink{ref_testNames}{\texttt{testNames}}%
\ (p.~\pageref{ref_testNames})%
\index[funcref]{testNames@\fidxl{testNames}}%
%
\item[Author:]%
Cengiz Gunay <cgunay@emory.edu>, 2004/12/06
%
\end{description}
\methodline%
\subsubsection[Method \texttt{itemResultsRow}]{Method \texttt{physiol\_cip\_traceset/itemResultsRow}}%
\index[funcref]{physiol_cip_traceset@\fidxl{physiol\_cip\_traceset}!itemResultsRow@\fidxl{itemResultsRow}}%
\label{ref_physiol_cip_traceset__itemResultsRow}%
\hypertarget{ref_physiol_cip_traceset__itemResultsRow}{}%
\begin{description}
\item[Summary:]Processes a raw data file from the dataset and return
		its parameter and test values.
%
\item[Usage:]~%
\begin{lyxcode}%
[params\_row, tests\_row] = itemResultsRow(dataset, index)
%
\end{lyxcode}%
%
\item[Description:]%
This method is designed to be reused from subclasses as long as the
 loadItemProfile method is properly overloaded. Adds an Index
 column to the DB to keep track of raw data items after shuffling.
%%
\item[Parameters:]~
\begin{description}%
\item[\texttt{dataset}:]
 A params\_tests\_dataset.
\item[\texttt{index}:]
 Index of file in dataset.
\end{description}%
%
\item[Returns:
]~

	params\_row: Parameter values in the same order of paramNames
	tests\_row: Test values in the same order with testNames
%
%
\item[See also:]%
\hyperlink{ref_loadItemProfile}{\texttt{loadItemProfile}}%
\ (p.~\pageref{ref_loadItemProfile})%
\index[funcref]{loadItemProfile@\fidxl{loadItemProfile}}%
, \hyperlink{ref_params_tests_dataset}{\texttt{params\_tests\_dataset}}%
\ (p.~\pageref{ref_params_tests_dataset})%
\index[funcref]{params_tests_dataset@\fidxl{params\_tests\_dataset}}%
, \hyperlink{ref_paramNames}{\texttt{paramNames}}%
\ (p.~\pageref{ref_paramNames})%
\index[funcref]{paramNames@\fidxl{paramNames}}%
, \hyperlink{ref_testNames}{\texttt{testNames}}%
\ (p.~\pageref{ref_testNames})%
\index[funcref]{testNames@\fidxl{testNames}}%
%
\item[Author:]%
Cengiz Gunay <cgunay@emory.edu>, 2004/09/10
%
\end{description}
\methodline%
\subsubsection[Method \texttt{loadItemProfile}]{Method \texttt{physiol\_cip\_traceset/loadItemProfile}}%
\index[funcref]{physiol_cip_traceset@\fidxl{physiol\_cip\_traceset}!loadItemProfile@\fidxl{loadItemProfile}}%
\label{ref_physiol_cip_traceset__loadItemProfile}%
\hypertarget{ref_physiol_cip_traceset__loadItemProfile}{}%
\begin{description}
\item[Summary:]Loads a cip\_trace\_profile object from a raw data file in the traceset.
%
\item[Usage:]~%
\begin{lyxcode}%
a\_profile = loadItemProfile(traceset, trace\_index)
%
\end{lyxcode}%
%
%
\item[Parameters:]~
\begin{description}%
\item[\texttt{traceset}:]
 A physiol\_cip\_traceset object.
\item[\texttt{trace\_index}:]
 Index of file in traceset.
\end{description}%
%
\item[Returns:
]~

	a\_profile: A profile object that implements the getResults method.
%
%
\item[See also:]%
\hyperlink{ref_itemResultsRow}{\texttt{itemResultsRow}}%
\ (p.~\pageref{ref_itemResultsRow})%
\index[funcref]{itemResultsRow@\fidxl{itemResultsRow}}%
, \hyperlink{ref_params_tests_fileset}{\texttt{params\_tests\_fileset}}%
\ (p.~\pageref{ref_params_tests_fileset})%
\index[funcref]{params_tests_fileset@\fidxl{params\_tests\_fileset}}%
, \hyperlink{ref_paramNames}{\texttt{paramNames}}%
\ (p.~\pageref{ref_paramNames})%
\index[funcref]{paramNames@\fidxl{paramNames}}%
, \hyperlink{ref_testNames}{\texttt{testNames}}%
\ (p.~\pageref{ref_testNames})%
\index[funcref]{testNames@\fidxl{testNames}}%
%
\item[Author:]%
Cengiz Gunay <cgunay@emory.edu>, 2004/09/14
%
\end{description}
\methodline%
\subsubsection[Method \texttt{paramNames}]{Method \texttt{physiol\_cip\_traceset/paramNames}}%
\index[funcref]{physiol_cip_traceset@\fidxl{physiol\_cip\_traceset}!paramNames@\fidxl{paramNames}}%
\label{ref_physiol_cip_traceset__paramNames}%
\hypertarget{ref_physiol_cip_traceset__paramNames}{}%
\begin{description}
\item[Summary:]Returns the parameter names for this traceset.
%
\item[Usage:]~%
\begin{lyxcode}%
param\_names = paramNames(traceset)
%
\end{lyxcode}%
%
\item[Description:]%
Looks at the filename of the first file to find the parameter names.
%%
\item[Parameters:]~
\begin{description}%
\item[\texttt{traceset}:]
 A params\_tests\_dataset.
\end{description}%
%
\item[Returns:
]~

	param\_names: Cell array with ordered parameter names.
%
%
\item[See also:]%
\hyperlink{ref_params_tests_dataset}{\texttt{params\_tests\_dataset}}%
\ (p.~\pageref{ref_params_tests_dataset})%
\index[funcref]{params_tests_dataset@\fidxl{params\_tests\_dataset}}%
, \hyperlink{ref_paramNames}{\texttt{paramNames}}%
\ (p.~\pageref{ref_paramNames})%
\index[funcref]{paramNames@\fidxl{paramNames}}%
, \hyperlink{ref_testNames}{\texttt{testNames}}%
\ (p.~\pageref{ref_testNames})%
\index[funcref]{testNames@\fidxl{testNames}}%
%
\item[Author:]%
Cengiz Gunay <cgunay@emory.edu>, 2004/12/06
%
\end{description}
\methodline%
\subsubsection[Method \texttt{set}]{Method \texttt{physiol\_cip\_traceset/set}}%
\index[funcref]{physiol_cip_traceset@\fidxl{physiol\_cip\_traceset}!set@\fidxl{set}}%
\label{ref_physiol_cip_traceset__set}%
\hypertarget{ref_physiol_cip_traceset__set}{}%
\begin{description}
\item[Summary:]Generic method for setting object attributes.
%
%
%
%
%
%
%
\item[Author:]%
Cengiz Gunay <cgunay@emory.edu>, 2004/10/08
%
\end{description}
\methodline%
\subsubsection[Method \texttt{setProp}]{Method \texttt{physiol\_cip\_traceset/setProp}}%
\index[funcref]{physiol_cip_traceset@\fidxl{physiol\_cip\_traceset}!setProp@\fidxl{setProp}}%
\label{ref_physiol_cip_traceset__setProp}%
\hypertarget{ref_physiol_cip_traceset__setProp}{}%
\begin{description}
\item[Summary:]Generic method for setting optional object properties.
%
\item[Usage:]~%
\begin{lyxcode}%
obj = setProp(obj, prop1, val1, prop2, val2, ...)
%
\end{lyxcode}%
%
\item[Description:]%
Modifies or adds property values. As many property name-value 
 pairs can be specified.
%%
\item[Parameters:]~
\begin{description}%
\item[\texttt{obj}:]
 Any object that has a props field.
\item[\texttt{attr}:]
 Property name
\item[\texttt{val}:]
 Property value.
\end{description}%
%
\item[Returns:
]~

	obj: The new object with the updated properties.
%
%
\item[See also:]%
%
\item[Author:]%
Cengiz Gunay <cgunay@emory.edu>, 2004/11/22
%
\end{description}
\methodline%
\subsection{Class \texttt{physiol\_cip\_traceset\_fileset}}%
\index[funcref]{physiol_cip_traceset_fileset@\fidxl{physiol\_cip\_traceset\_fileset}|boldhyperpage}%
\label{ref_physiol_cip_traceset_fileset}%
\hypertarget{ref_physiol_cip_traceset_fileset}{}%
\subsubsection[Constructor \texttt{physiol\_cip\_traceset\_fileset}]{Constructor \texttt{physiol\_cip\_traceset\_fileset/physiol\_cip\_traceset\_fileset}}%
\index[funcref]{physiol_cip_traceset_fileset@\fidxl{physiol\_cip\_traceset\_fileset}!physiol_cip_traceset_fileset@\fidxl{physiol\_cip\_traceset\_fileset}}%
\label{ref_physiol_cip_traceset_fileset__physiol_cip_traceset_fileset}%
\hypertarget{ref_physiol_cip_traceset_fileset__physiol_cip_traceset_fileset}{}%
\begin{description}
\item[Summary:]Physiological fileset of traceset objects (concatenated).
%
\item[Usage:]~%
\begin{lyxcode}%
obj = physiol\_cip\_traceset\_fileset(traceset\_items, dt, dy, props)
%
\end{lyxcode}%
%
\item[Description:]%
This is a subclass of params\_tests\_dataset. It contains a set of
 physiol\_cip\_traceset items that are tied to physical data sources. Each
 traceset can load a set of traces for an experimental recording. Most
 flexible usage is obtained when the input traceset\_items is given as a
 cell array of physiol\_cip\_traceset objects. These objects can each link to
 PCDX or NeuroSAGE HDF5 files independent of each other. A regular Matlab
 script can be used to create such a cell array. If a function is defined
 to return such an array, it can be passed as
 traceset\_items. Alternatively, the cell array can be constructed from an
 ASCII file as described below, such as for deprecated PCDX data files.
%%
\item[Parameters:]~
\begin{description}%
\item[\texttt{traceset\_items}:]
 It can be a function handle, cell array or filename

string. Function should return a cell array of physiol\_cip\_traceset
items. Alternatively a preconstructed cell array can be provided directly.
If it is an ASCII filename, then it should contain the following tab-delimited items:
1. Neuron ID (name to associate with the neuron). If left blank, use
the filename with the '.all' extension removed.
2. The absolute path of the data file
3. The trace numbers to load, space-delimited (e.g. 1-21 24 26 27)
4. Vchan: voltage channel number
5. Ichan: current channel number
6. Vgain: external gain on voltage channel IN ADDITION to the 10X that
automatically comes from the Axoclamp 2B.
7. Igain: external gain on current channel.
8. Pairs of condition names and molar concentrations in any order
e.g.: TTX       1e-8    apamin  2e-7    picrotoxin      1e-4
\end{description}%
%
\item[Returns a structure object with the following fields:
]~

	neuron\_idx: A structure that points from neuron names to NeuronId numbers.
	params\_tests\_dataset
%
%
\item[See also:]%
\hyperlink{ref_physiol_cip_traceset}{\texttt{physiol\_cip\_traceset}}%
\ (p.~\pageref{ref_physiol_cip_traceset})%
\index[funcref]{physiol_cip_traceset@\fidxl{physiol\_cip\_traceset}}%
, \hyperlink{ref_params_tests_dataset}{\texttt{params\_tests\_dataset}}%
\ (p.~\pageref{ref_params_tests_dataset})%
\index[funcref]{params_tests_dataset@\fidxl{params\_tests\_dataset}}%
, \hyperlink{ref_params_tests_db}{\texttt{params\_tests\_db}}%
\ (p.~\pageref{ref_params_tests_db})%
\index[funcref]{params_tests_db@\fidxl{params\_tests\_db}}%
%
\item[Author:]%
Cengiz Gunay <cgunay@emory.edu> and Thomas Sangrey, 2005/01/17
%
\end{description}
\methodline%
\subsubsection[Method \texttt{cip\_trace}]{Method \texttt{physiol\_cip\_traceset\_fileset/cip\_trace}}%
\index[funcref]{physiol_cip_traceset_fileset@\fidxl{physiol\_cip\_traceset\_fileset}!cip_trace@\fidxl{cip\_trace}}%
\label{ref_physiol_cip_traceset_fileset__cip_trace}%
\hypertarget{ref_physiol_cip_traceset_fileset__cip_trace}{}%
\begin{description}
\item[Summary:]Loads a cip\_trace object from a raw data file in the fileset.
%
%
%
\item[Parameters:]~
\begin{description}%
\item[\texttt{fileset}:]
 A physiol\_cip\_traceset\_fileset object.
\item[\texttt{traceset\_index}:]
 Index of traceset item in this fileset (corresponds 

to row in cells\_filename) to find the cell information.
\item[\texttt{trace\_index}:]
 Index of item in the traceset.
\item[\texttt{a\_db}:]
 A DB created by this fileset to read the traceset and item indices from.
\item[\texttt{props}:]
 A structure with any optional properties, passed to physiol\_cip\_traceset/cip\_trace.
\end{description}%
%
\item[Returns:
]~

	a\_cip\_trace: One or more cip\_trace object that holds the raw data.
%
%
\item[See also:]%
\hyperlink{ref_loadItemProfile}{\texttt{loadItemProfile}}%
\ (p.~\pageref{ref_loadItemProfile})%
\index[funcref]{loadItemProfile@\fidxl{loadItemProfile}}%
, \hyperlink{ref_physiol_cip_traceset__cip_trace}{\texttt{physiol\_cip\_traceset/cip\_trace}}%
\ (p.~\pageref{ref_physiol_cip_traceset__cip_trace})%
\index[funcref]{physiol_cip_traceset@\fidxl{physiol\_cip\_traceset}!cip_trace@\fidxl{cip\_trace}}%
%
\item[Author:]%
Cengiz Gunay <cgunay@emory.edu>, 2005/07/13
%
\end{description}
\methodline%
\subsubsection[Method \texttt{display}]{Method \texttt{physiol\_cip\_traceset\_fileset/display}}%
\index[funcref]{physiol_cip_traceset_fileset@\fidxl{physiol\_cip\_traceset\_fileset}!display@\fidxl{display}}%
\label{ref_physiol_cip_traceset_fileset__display}%
\hypertarget{ref_physiol_cip_traceset_fileset__display}{}%
\begin{description}
%
%
%
%
%
%
%
\item[Author:]%
Cengiz Gunay <cgunay@emory.edu>, 2004/08/04
%
\end{description}
\methodline%
\subsubsection[Method \texttt{get}]{Method \texttt{physiol\_cip\_traceset\_fileset/get}}%
\index[funcref]{physiol_cip_traceset_fileset@\fidxl{physiol\_cip\_traceset\_fileset}!get@\fidxl{get}}%
\label{ref_physiol_cip_traceset_fileset__get}%
\hypertarget{ref_physiol_cip_traceset_fileset__get}{}%
\begin{description}
\item[Summary:]Defines generic attribute retrieval for objects.
%
%
%
%
%
%
%
\item[Author:]%
Cengiz Gunay <cgunay@emory.edu>, 2004/09/14
%
\end{description}
\methodline%
\subsubsection[Method \texttt{loadItemProfile}]{Method \texttt{physiol\_cip\_traceset\_fileset/loadItemProfile}}%
\index[funcref]{physiol_cip_traceset_fileset@\fidxl{physiol\_cip\_traceset\_fileset}!loadItemProfile@\fidxl{loadItemProfile}}%
\label{ref_physiol_cip_traceset_fileset__loadItemProfile}%
\hypertarget{ref_physiol_cip_traceset_fileset__loadItemProfile}{}%
\begin{description}
\item[Summary:]Loads a cip\_trace\_profile object from a raw data file in the fileset.
%
\item[Usage:]~%
\begin{lyxcode}%
a\_profile = loadItemProfile(fileset, traceset\_index, trace\_index)
%
\end{lyxcode}%
%
%
\item[Parameters:]~
\begin{description}%
\item[\texttt{fileset}:]
     A physiol\_cip\_traceset object.
\item[\texttt{traceset\_index }:]
  Index of traceset item in this fileset (corresponds 

to row in cells\_filename) to use grab the cell information.
\item[\texttt{trace\_index}:]
 Index of item in the traceset.
\end{description}%
%
\item[Returns:
]~

	a\_profile: A profile object that implements the getResults method.
%
%
\item[See also:]%
\hyperlink{ref_itemResultsRow}{\texttt{itemResultsRow}}%
\ (p.~\pageref{ref_itemResultsRow})%
\index[funcref]{itemResultsRow@\fidxl{itemResultsRow}}%
, \hyperlink{ref_params_tests_fileset}{\texttt{params\_tests\_fileset}}%
\ (p.~\pageref{ref_params_tests_fileset})%
\index[funcref]{params_tests_fileset@\fidxl{params\_tests\_fileset}}%
, \hyperlink{ref_paramNames}{\texttt{paramNames}}%
\ (p.~\pageref{ref_paramNames})%
\index[funcref]{paramNames@\fidxl{paramNames}}%
, \hyperlink{ref_testNames}{\texttt{testNames}}%
\ (p.~\pageref{ref_testNames})%
\index[funcref]{testNames@\fidxl{testNames}}%
%
\item[Author:]%
Cengiz Gunay <cgunay@emory.edu>, 2004/09/14 and Tom Sangrey
%
\end{description}
\methodline%
\subsubsection[Method \texttt{mergeFilesets}]{Method \texttt{physiol\_cip\_traceset\_fileset/mergeFilesets}}%
\index[funcref]{physiol_cip_traceset_fileset@\fidxl{physiol\_cip\_traceset\_fileset}!mergeFilesets@\fidxl{mergeFilesets}}%
\label{ref_physiol_cip_traceset_fileset__mergeFilesets}%
\hypertarget{ref_physiol_cip_traceset_fileset__mergeFilesets}{}%
\begin{description}
\item[Summary:]Concatenates two physiol\_cip\_traceset\_fileset objects.
%
\item[Usage:]~%
\begin{lyxcode}%
[a\_fileset, traceset\_offset, neuron\_id\_offset] = mergeFilesets(a\_fileset, w\_fileset)
%
\end{lyxcode}%
%
\item[Description:]%
Concatenates the list contents, and combines the neuron\_idx
 structures. The properties such as dt, dy and props are retained from
 first object.
%%
\item[Parameters:]~
\begin{description}%
\item[\texttt{a\_fileset, w\_fileset}:]
 Two physiol\_cip\_traceset\_fileset objects without

overlapping neuron\_id items.
\end{description}%
%
\item[Returns:
]~

   a\_fileset: The new object with combined contents.
%
%
\item[See also:]%
\hyperlink{ref_physiol_cip_traceset_fileset}{\texttt{physiol\_cip\_traceset\_fileset}}%
\ (p.~\pageref{ref_physiol_cip_traceset_fileset})%
\index[funcref]{physiol_cip_traceset_fileset@\fidxl{physiol\_cip\_traceset\_fileset}}%
%
\item[Author:]%
Cengiz Gunay <cgunay@emory.edu>, 2008/05/18
%
\end{description}
\methodline%
\subsubsection[Method \texttt{neuronNameFromId}]{Method \texttt{physiol\_cip\_traceset\_fileset/neuronNameFromId}}%
\index[funcref]{physiol_cip_traceset_fileset@\fidxl{physiol\_cip\_traceset\_fileset}!neuronNameFromId@\fidxl{neuronNameFromId}}%
\label{ref_physiol_cip_traceset_fileset__neuronNameFromId}%
\hypertarget{ref_physiol_cip_traceset_fileset__neuronNameFromId}{}%
\begin{description}
\item[Summary:]Returns string neuron names from a list of neuron ids.
%
\item[Usage:]~%
\begin{lyxcode}%
name\_strs = neuronNameFromId(fileset, neuron\_ids)
%
\end{lyxcode}%
%
%
\item[Parameters:]~
\begin{description}%
\item[\texttt{fileset}:]
 A physiol\_cip\_traceset\_fileset object.
\item[\texttt{neuron\_ids}:]
 One or more neuron ids in a vector.
\end{description}%
%
\item[Returns:
]~

	name\_strs: Cell array of neuron names corresponding to the ids given.
%
%
\item[See also:]%
\hyperlink{ref_physiol_cip_traceset_fileset}{\texttt{physiol\_cip\_traceset\_fileset}}%
\ (p.~\pageref{ref_physiol_cip_traceset_fileset})%
\index[funcref]{physiol_cip_traceset_fileset@\fidxl{physiol\_cip\_traceset\_fileset}}%
%
\item[Author:]%
Cengiz Gunay <cgunay@emory.edu>, 2007/11/16
%
\end{description}
\methodline%
\subsubsection[Method \texttt{physiol\_bundle}]{Method \texttt{physiol\_cip\_traceset\_fileset/physiol\_bundle}}%
\index[funcref]{physiol_cip_traceset_fileset@\fidxl{physiol\_cip\_traceset\_fileset}!physiol_bundle@\fidxl{physiol\_bundle}}%
\label{ref_physiol_cip_traceset_fileset__physiol_bundle}%
\hypertarget{ref_physiol_cip_traceset_fileset__physiol_bundle}{}%
\begin{description}
\item[Summary:]Loads the database and then creates the physiol\_bundle object.
%
\item[Usage:]~%
\begin{lyxcode}%
a\_pbundle = physiol\_bundle(fileset, props)
%
\end{lyxcode}%
%
\item[Description:]%
Calls params\_tests\_db to get the db, and then calls
 tests\_db/physiol\_bundle to do transformations.
%%
\item[Parameters:]~
\begin{description}%
\item[\texttt{fileset}:]
 A physiol\_cip\_traceset\_fileset object.
\item[\texttt{props}:]
 A structure with any optional properties.

(Passed to tests\_db/physiol\_bundle)
\end{description}%
%
\item[Returns:
]~

	a\_physiol\_bundle: One or more physiol\_bundle object that holds the raw data.
%
%
\item[See also:]%
\hyperlink{ref_tests_db__physiol_bundle}{\texttt{tests\_db/physiol\_bundle}}%
\ (p.~\pageref{ref_tests_db__physiol_bundle})%
\index[funcref]{tests_db@\fidxl{tests\_db}!physiol_bundle@\fidxl{physiol\_bundle}}%
%
\item[Author:]%
Cengiz Gunay <cgunay@emory.edu>, 2007/12/21
%
\end{description}
\methodline%
\subsubsection[Method \texttt{readDBItems}]{Method \texttt{physiol\_cip\_traceset\_fileset/readDBItems}}%
\index[funcref]{physiol_cip_traceset_fileset@\fidxl{physiol\_cip\_traceset\_fileset}!readDBItems@\fidxl{readDBItems}}%
\label{ref_physiol_cip_traceset_fileset__readDBItems}%
\hypertarget{ref_physiol_cip_traceset_fileset__readDBItems}{}%
\begin{description}
\item[Summary:]Reads all items to generate a params\_tests\_db object.
%
\item[Usage:]~%
\begin{lyxcode}%
[params, param\_names, tests, test\_names] = readDBItems(obj, items)
%
\end{lyxcode}%
%
\item[Description:]%
This is a specific method to convert from physiol\_cip\_traceset\_fileset to
 a params\_tests\_db, or a subclass. 
 Outputs of this function can be directly fed to the constructor of
 a params\_tests\_db or a subclass.
%%
\item[Parameters:]~
\begin{description}%
\item[\texttt{obj}:]
 A physiol\_cip\_traceset\_fileset 
\item[\texttt{items}:]
 (Optional) List of item indices to use to create the db.
\end{description}%
%
\item[Returns:
]~

	params, param\_names, tests, test\_names: See params\_tests\_db.
%
%
\item[See also:]%
\hyperlink{ref_params_tests_db}{\texttt{params\_tests\_db}}%
\ (p.~\pageref{ref_params_tests_db})%
\index[funcref]{params_tests_db@\fidxl{params\_tests\_db}}%
, \hyperlink{ref_params_tests_fileset}{\texttt{params\_tests\_fileset}}%
\ (p.~\pageref{ref_params_tests_fileset})%
\index[funcref]{params_tests_fileset@\fidxl{params\_tests\_fileset}}%
, \hyperlink{ref_itemResultsRow}{\texttt{itemResultsRow}}%
\ (p.~\pageref{ref_itemResultsRow})%
\index[funcref]{itemResultsRow@\fidxl{itemResultsRow}}%
%
%
\end{description}
\methodline%
\subsubsection[Method \texttt{set}]{Method \texttt{physiol\_cip\_traceset\_fileset/set}}%
\index[funcref]{physiol_cip_traceset_fileset@\fidxl{physiol\_cip\_traceset\_fileset}!set@\fidxl{set}}%
\label{ref_physiol_cip_traceset_fileset__set}%
\hypertarget{ref_physiol_cip_traceset_fileset__set}{}%
\begin{description}
\item[Summary:]Generic method for setting object attributes.
%
%
%
%
%
%
%
\item[Author:]%
Cengiz Gunay <cgunay@emory.edu>, 2004/10/08
%
\end{description}
\methodline%
\subsubsection[Method \texttt{setProp}]{Method \texttt{physiol\_cip\_traceset\_fileset/setProp}}%
\index[funcref]{physiol_cip_traceset_fileset@\fidxl{physiol\_cip\_traceset\_fileset}!setProp@\fidxl{setProp}}%
\label{ref_physiol_cip_traceset_fileset__setProp}%
\hypertarget{ref_physiol_cip_traceset_fileset__setProp}{}%
\begin{description}
\item[Summary:]Generic method for setting optional object properties.
%
\item[Usage:]~%
\begin{lyxcode}%
obj = setProp(obj, prop1, val1, prop2, val2, ...)
%
\end{lyxcode}%
%
\item[Description:]%
Modifies or adds property values. As many property name-value 
 pairs can be specified.
%%
\item[Parameters:]~
\begin{description}%
\item[\texttt{obj}:]
 Any object that has a props field.
\item[\texttt{attr}:]
 Property name
\item[\texttt{val}:]
 Property value.
\end{description}%
%
\item[Returns:
]~

	obj: The new object with the updated properties.
%
%
\item[See also:]%
%
\item[Author:]%
Cengiz Gunay <cgunay@emory.edu>, 2004/11/22
%
\end{description}
\methodline%
\subsubsection[Method \texttt{vertcat}]{Method \texttt{physiol\_cip\_traceset\_fileset/vertcat}}%
\index[funcref]{physiol_cip_traceset_fileset@\fidxl{physiol\_cip\_traceset\_fileset}!vertcat@\fidxl{vertcat}}%
\label{ref_physiol_cip_traceset_fileset__vertcat}%
\hypertarget{ref_physiol_cip_traceset_fileset__vertcat}{}%
\begin{description}
\item[Summary:]Concatenates multiple physiol\_cip\_traceset\_fileset objects.
%
\item[Usage:]~%
\begin{lyxcode}%
obj = vertcat(obj, obj2)
%
\end{lyxcode}%
%
\item[Description:]%
Concatenates the list contents, and combines the neuron\_idx
 structures. The properties such as dt, dy and props are retained from
 first object.
%%
\item[Parameters:]~
\begin{description}%
\item[\texttt{obj, obj2}:]
 Two physiol\_cip\_traceset\_fileset objects without

overlapping neuron\_id items.
\end{description}%
%
\item[Returns:
]~

   obj: The new object with combined contents.
%
%
\item[See also:]%
\hyperlink{ref_physiol_cip_traceset_fileset}{\texttt{physiol\_cip\_traceset\_fileset}}%
\ (p.~\pageref{ref_physiol_cip_traceset_fileset})%
\index[funcref]{physiol_cip_traceset_fileset@\fidxl{physiol\_cip\_traceset\_fileset}}%
%
\item[Author:]%
Cengiz Gunay <cgunay@emory.edu>, 2008/01/13
%
\end{description}
\methodline%
\subsection{Class \texttt{plot\_abstract}}%
\index[funcref]{plot_abstract@\fidxl{plot\_abstract}|boldhyperpage}%
\label{ref_plot_abstract}%
\hypertarget{ref_plot_abstract}{}%
\subsubsection[Constructor \texttt{plot\_abstract}]{Constructor \texttt{plot\_abstract/plot\_abstract}}%
\index[funcref]{plot_abstract@\fidxl{plot\_abstract}!plot_abstract@\fidxl{plot\_abstract}}%
\label{ref_plot_abstract__plot_abstract}%
\hypertarget{ref_plot_abstract__plot_abstract}{}%
\begin{description}
\item[Summary:]A plot that can be directly visualized or included in subplots.
%
\item[Usage:]~%
\begin{lyxcode}%
obj = plot\_abstract(data, axis\_labels, title, legend, command, props)
%
\end{lyxcode}%
%
\item[Description:]%
Base class that holds the necessary data to draw a plot. This data
 can then be used to generate different plots. Subclasses define specific
 plots with additional data. Subclasses should conform to the standard 
 that the series of commands found in plotFigure should produce a valid
 figure.
%%
\item[Parameters:]~
\begin{description}%
\item[\texttt{data}:]
 A cell array of data arrays (x, y, z, etc.) that can be 

fed to plot commands.
\item[\texttt{axis\_labels}:]
 Cell array of axis label strings.
\item[\texttt{title}:]
 Plot description string.
\item[\texttt{legend}:]
 Cell array of descriptions for each item plotted.
\item[\texttt{command}:]
 Plotting command to use (Optional, default='plot')
\item[\texttt{props}:]
 A structure with any optional properties.
\begin{description}%
\item[\texttt{axisLimits}:]
 Sets axis limits of non-NaN values in vector.
\item[\texttt{tightLimits}:]
 If 1, issues an "axis tight" command (default=0)
\item[\texttt{border}:]
 Relative size of border spacing around axis, between 0 - 1. (default=0)

If a scalar, equal border on all sides, give a four-element vector 
[left bottom right top] to define borders for each side.
\item[\texttt{colormap}:]
 Figure colormap passed to the colormap function. If 

function handle, its output is passed instead.
\item[\texttt{grid}:]
 Display dashed grid in background.
\item[\texttt{noXLabel}:]
 No X-axis label.
\item[\texttt{noYLabel}:]
 No Y-axis label.
\item[\texttt{noTitle}:]
 No title.
\item[\texttt{rotateXLabel}:]
 Rotates the X-axis label for smaller width.
\item[\texttt{rotateYLabel}:]
 Rotates the Y-axis label for smaller width.
\item[\texttt{numXTicks}:]
 Number of ticks on X-axis.
\item[\texttt{formatXTickLabels}:]
 The sprintf format string for tick labels.
\item[\texttt{XTick, YTick}:]
 Point locations for axis ticks.
\item[\texttt{XTickLabel, YTickLabel}:]
 Axis tick labels.
\item[\texttt{ColorOrder}:]
 Set the ColorOrder of the axis.
\item[\texttt{LineStyleOrder}:]
 Set the LineStyleOrder of the axis.
\item[\texttt{legendLocation}:]
 Passed to legend(..., 'location', legendLocation).
\item[\texttt{legendOrientation}:]
 Passed to legend(..., 'orientation', legendLocation).
\item[\texttt{noLegends}:]
 If exists, no legends are displayed.
\item[\texttt{axisProps}:]
 Passed to set properties of the axis drawn.
\item[\texttt{plotProps}:]
 Passed to set properties of the plot drawn.
\item[\texttt{figureProps}:]
 Passed to set properties of the figure drawn.
\item[\texttt{PaperPosition}:]
 Sets the figure property for printing at this size.
\item[\texttt{resizeControl}:]
 If 0, drawing after resize is disabled and prints at screen 

size, if 1 (default), redraws figure after each resize event and 
prints at PaperPosition size.
\item[\texttt{fixedSize}:]
 Vector of width and height in inches passed to

PaperPosition property. Implies resizeControl=0.
\end{description}%
\end{description}%
%
\item[Returns a structure object with the following fields:
]~

	data, axis\_labels, title, legend, command, props
%
%
\item[See also:]%
\hyperlink{ref_plot_abstract__plot}{\texttt{plot\_abstract/plot}}%
\ (p.~\pageref{ref_plot_abstract__plot})%
\index[funcref]{plot_abstract@\fidxl{plot\_abstract}!plot@\fidxl{plot}}%
, \hyperlink{ref_plot_abstract__plotFigure}{\texttt{plot\_abstract/plotFigure}}%
\ (p.~\pageref{ref_plot_abstract__plotFigure})%
\index[funcref]{plot_abstract@\fidxl{plot\_abstract}!plotFigure@\fidxl{plotFigure}}%
%
\item[Author:]%
Cengiz Gunay <cgunay@emory.edu>, 2004/09/22
%
\end{description}
\methodline%
\subsubsection[Method \texttt{axis}]{Method \texttt{plot\_abstract/axis}}%
\index[funcref]{plot_abstract@\fidxl{plot\_abstract}!axis@\fidxl{axis}}%
\label{ref_plot_abstract__axis}%
\hypertarget{ref_plot_abstract__axis}{}%
\begin{description}
\item[Summary:]Returns the estimated axis ranges of this plot according to its data.
%
\item[Usage:]~%
\begin{lyxcode}%
ranges = axis(a\_plot)
%
\end{lyxcode}%
%
%
\item[Parameters:]~
\begin{description}%
\item[\texttt{a\_plot}:]
 A plot\_abstract object, or a subclass object.
\end{description}%
%
\item[Returns:
]~

	ranges: The ranges as a vector in the same way 'axis' would return.
%
%
\item[See also:]%
\hyperlink{ref_plot_abstract}{\texttt{plot\_abstract}}%
\ (p.~\pageref{ref_plot_abstract})%
\index[funcref]{plot_abstract@\fidxl{plot\_abstract}}%
, \hyperlink{ref_plot_abstract__plot}{\texttt{plot\_abstract/plot}}%
\ (p.~\pageref{ref_plot_abstract__plot})%
\index[funcref]{plot_abstract@\fidxl{plot\_abstract}!plot@\fidxl{plot}}%
%
\item[Author:]%
Cengiz Gunay <cgunay@emory.edu>, 2004/10/13
%
\end{description}
\methodline%
\subsubsection[Method \texttt{decorate}]{Method \texttt{plot\_abstract/decorate}}%
\index[funcref]{plot_abstract@\fidxl{plot\_abstract}!decorate@\fidxl{decorate}}%
\label{ref_plot_abstract__decorate}%
\hypertarget{ref_plot_abstract__decorate}{}%
\begin{description}
\item[Summary:]Places decorations (titles, labels, ticks, etc.) on the plot.
%
\item[Usage:]~%
\begin{lyxcode}%
handles = decorate(a\_plot, plot\_handles)
%
\end{lyxcode}%
%
%
\item[Parameters:]~
\begin{description}%
\item[\texttt{a\_plot}:]
 A plot\_abstract object, or a subclass object.
\item[\texttt{plot\_handles}:]
 Handles of plots already drawn (structure returned by

plot\_abstract/plot). 
\end{description}%
%
\item[Returns:
]~

   handles: Structure with handles of all graphical objects drawn.
%
%
\item[See also:]%
\hyperlink{ref_plot_abstract}{\texttt{plot\_abstract}}%
\ (p.~\pageref{ref_plot_abstract})%
\index[funcref]{plot_abstract@\fidxl{plot\_abstract}}%
, \hyperlink{ref_plot_abstract__plot}{\texttt{plot\_abstract/plot}}%
\ (p.~\pageref{ref_plot_abstract__plot})%
\index[funcref]{plot_abstract@\fidxl{plot\_abstract}!plot@\fidxl{plot}}%
%
\item[Author:]%
Cengiz Gunay <cgunay@emory.edu>, 2004/09/22
%
\end{description}
\methodline%
\subsubsection[Method \texttt{display}]{Method \texttt{plot\_abstract/display}}%
\index[funcref]{plot_abstract@\fidxl{plot\_abstract}!display@\fidxl{display}}%
\label{ref_plot_abstract__display}%
\hypertarget{ref_plot_abstract__display}{}%
\begin{description}
%
%
%
%
%
%
%
\item[Author:]%
Cengiz Gunay <cgunay@emory.edu>, 2004/08/04
%
\end{description}
\methodline%
\subsubsection[Method \texttt{get}]{Method \texttt{plot\_abstract/get}}%
\index[funcref]{plot_abstract@\fidxl{plot\_abstract}!get@\fidxl{get}}%
\label{ref_plot_abstract__get}%
\hypertarget{ref_plot_abstract__get}{}%
\begin{description}
\item[Summary:]Defines generic attribute retrieval for objects.
%
%
%
%
%
%
%
\item[Author:]%
Cengiz Gunay <cgunay@emory.edu>, 2004/09/14
%
\end{description}
\methodline%
\subsubsection[Method \texttt{matrixPlots}]{Method \texttt{plot\_abstract/matrixPlots}}%
\index[funcref]{plot_abstract@\fidxl{plot\_abstract}!matrixPlots@\fidxl{matrixPlots}}%
\label{ref_plot_abstract__matrixPlots}%
\hypertarget{ref_plot_abstract__matrixPlots}{}%
\begin{description}
\item[Summary:]Organize multiple plots in a matrix formation.
%
\item[Usage:]~%
\begin{lyxcode}%
a\_plot = matrixPlots(plots, axis\_labels, title\_str, props)
%
\end{lyxcode}%
%
%
\item[Parameters:]~
\begin{description}%
\item[\texttt{plots}:]
 Array of plot\_abstract or subclass objects.
\item[\texttt{axis\_labels}:]
 Cell array of axis label strings (optional, taken from plots).
\item[\texttt{title\_str}:]
 Plot description string (optional, taken from plots).
\item[\texttt{props}:]
 A structure with any optional properties passed to the Y stack\_plot.
\begin{description}%
\item[\texttt{titlesPos, yLabelsPos, yTicksPos}:]
 if specified, passed to the X stack\_plots.
\item[\texttt{rotateYLabel}:]
 if specified, passed to the X stack\_plots.
\item[\texttt{axisLimits}:]
 if specified, passed to the X stack\_plots.
\item[\texttt{goldratio}:]
 try to make the figure in this aspect ratio.
\item[\texttt{width, height}:]
 if specified, make the figure have this many plots in 

corresponding dimension.
\end{description}%
\end{description}%
%
\item[Returns:
]~

	a\_plot: A plot\_abstract object.
%
%
\item[See also:]%
\hyperlink{ref_plot_abstract}{\texttt{plot\_abstract}}%
\ (p.~\pageref{ref_plot_abstract})%
\index[funcref]{plot_abstract@\fidxl{plot\_abstract}}%
, \hyperlink{ref_plot_abstract__plot}{\texttt{plot\_abstract/plot}}%
\ (p.~\pageref{ref_plot_abstract__plot})%
\index[funcref]{plot_abstract@\fidxl{plot\_abstract}!plot@\fidxl{plot}}%
, \hyperlink{ref_plot_abstract__plotFigure}{\texttt{plot\_abstract/plotFigure}}%
\ (p.~\pageref{ref_plot_abstract__plotFigure})%
\index[funcref]{plot_abstract@\fidxl{plot\_abstract}!plotFigure@\fidxl{plotFigure}}%
%
\item[Author:]%
Cengiz Gunay <cgunay@emory.edu>, 2004/12/07
%
\end{description}
\methodline%
\subsubsection[Method \texttt{openAxis}]{Method \texttt{plot\_abstract/openAxis}}%
\index[funcref]{plot_abstract@\fidxl{plot\_abstract}!openAxis@\fidxl{openAxis}}%
\label{ref_plot_abstract__openAxis}%
\hypertarget{ref_plot_abstract__openAxis}{}%
\begin{description}
\item[Summary:]Calculates the extents for the axis of this plot and opens it.
%
\item[Usage:]~%
\begin{lyxcode}%
[axis\_handle, layout\_axis] = openAxis(a\_plot, layout\_axis)
%
\end{lyxcode}%
%
%
\item[Parameters:]~
\begin{description}%
\item[\texttt{a\_plot}:]
 A plot\_abstract object, or a subclass object.
\item[\texttt{layout\_axis}:]
 The axis position to layout this plot (Optional). 

If NaN, doesn't open a new axis.
\end{description}%
%
\item[Returns:
]~

	handles: Handles of graphical objects drawn.
%
%
\item[See also:]%
\hyperlink{ref_plot_abstract}{\texttt{plot\_abstract}}%
\ (p.~\pageref{ref_plot_abstract})%
\index[funcref]{plot_abstract@\fidxl{plot\_abstract}}%
%
\item[Author:]%
Cengiz Gunay <cgunay@emory.edu>, 2004/09/22
%
\end{description}
\methodline%
\subsubsection[Method \texttt{plot}]{Method \texttt{plot\_abstract/plot}}%
\index[funcref]{plot_abstract@\fidxl{plot\_abstract}!plot@\fidxl{plot}}%
\label{ref_plot_abstract__plot}%
\hypertarget{ref_plot_abstract__plot}{}%
\begin{description}
\item[Summary:]Draws this plot in the current axis.
%
\item[Usage:]~%
\begin{lyxcode}%
handles = plot(a\_plot, layout\_axis)
%
\end{lyxcode}%
%
%
\item[Parameters:]~
\begin{description}%
\item[\texttt{a\_plot}:]
 A plot\_abstract object, or a subclass object.
\item[\texttt{layout\_axis}:]
 The axis position to layout this plot (Optional). 

If NaN, doesn't open a new axis.
\end{description}%
%
\item[Returns:
]~

	handles: Handles of graphical objects drawn.
%
%
\item[See also:]%
\hyperlink{ref_plot_abstract}{\texttt{plot\_abstract}}%
\ (p.~\pageref{ref_plot_abstract})%
\index[funcref]{plot_abstract@\fidxl{plot\_abstract}}%
%
\item[Author:]%
Cengiz Gunay <cgunay@emory.edu>, 2004/09/22
%
\end{description}
\methodline%
\subsubsection[Method \texttt{plotFigure}]{Method \texttt{plot\_abstract/plotFigure}}%
\index[funcref]{plot_abstract@\fidxl{plot\_abstract}!plotFigure@\fidxl{plotFigure}}%
\label{ref_plot_abstract__plotFigure}%
\hypertarget{ref_plot_abstract__plotFigure}{}%
\begin{description}
\item[Summary:]Draws this plot alone in a new figure window.
%
\item[Usage:]~%
\begin{lyxcode}%
[handle, plot\_handles] = plotFigure(a\_plot, title\_str, props)
%
\end{lyxcode}%
%
%
\item[Parameters:]~
\begin{description}%
\item[\texttt{a\_plot}:]
 A plot\_abstract object, or a subclass object.
\item[\texttt{title\_str}:]
 (Optional) String to append to plot title.
\item[\texttt{props}:]
 A structure with any optional properties.
\begin{description}%
\item[\texttt{figureHandle}:]
 Use this figure instead of opening a new one.
\item[\texttt{delayOpen}:]
 Wait this many seconds for figure to materialize 

(0.5s is good workaround for Compiz-fusion bug)
\end{description}%
\end{description}%
%
\item[Returns:
]~

   handle: Handle of new figure.
   plot\_handles: Structure with all plotted data and decorations.
%
%
\item[See also:]%
\hyperlink{ref_plot_abstract}{\texttt{plot\_abstract}}%
\ (p.~\pageref{ref_plot_abstract})%
\index[funcref]{plot_abstract@\fidxl{plot\_abstract}}%
, \hyperlink{ref_plot_abstract__plot}{\texttt{plot\_abstract/plot}}%
\ (p.~\pageref{ref_plot_abstract__plot})%
\index[funcref]{plot_abstract@\fidxl{plot\_abstract}!plot@\fidxl{plot}}%
, \hyperlink{ref_plot_abstract__decorate}{\texttt{plot\_abstract/decorate}}%
\ (p.~\pageref{ref_plot_abstract__decorate})%
\index[funcref]{plot_abstract@\fidxl{plot\_abstract}!decorate@\fidxl{decorate}}%
%
\item[Author:]%
Cengiz Gunay <cgunay@emory.edu>, 2004/09/22
%
\end{description}
\methodline%
\subsubsection[Method \texttt{set}]{Method \texttt{plot\_abstract/set}}%
\index[funcref]{plot_abstract@\fidxl{plot\_abstract}!set@\fidxl{set}}%
\label{ref_plot_abstract__set}%
\hypertarget{ref_plot_abstract__set}{}%
\begin{description}
\item[Summary:]Generic method for setting object attributes.
%
%
%
%
%
%
%
\item[Author:]%
Cengiz Gunay <cgunay@emory.edu>, 2004/10/08
%
\end{description}
\methodline%
\subsubsection[Method \texttt{setProp}]{Method \texttt{plot\_abstract/setProp}}%
\index[funcref]{plot_abstract@\fidxl{plot\_abstract}!setProp@\fidxl{setProp}}%
\label{ref_plot_abstract__setProp}%
\hypertarget{ref_plot_abstract__setProp}{}%
\begin{description}
\item[Summary:]Generic method for setting optional object properties.
%
\item[Usage:]~%
\begin{lyxcode}%
obj = setProp(obj, prop1, val1, prop2, val2, ...)
%
\end{lyxcode}%
%
\item[Description:]%
Modifies or adds property values. As many property name-value 
 pairs can be specified.
%%
\item[Parameters:]~
\begin{description}%
\item[\texttt{obj}:]
 Any object that has a props field.
\item[\texttt{attr}:]
 Property name
\item[\texttt{val}:]
 Property value.
\end{description}%
%
\item[Returns:
]~

	obj: The new object with the updated properties.
%
%
\item[See also:]%
%
\item[Author:]%
Cengiz Gunay <cgunay@emory.edu>, 2004/11/22
%
\end{description}
\methodline%
\subsubsection[Method \texttt{subsasgn}]{Method \texttt{plot\_abstract/subsasgn}}%
\index[funcref]{plot_abstract@\fidxl{plot\_abstract}!subsasgn@\fidxl{subsasgn}}%
\label{ref_plot_abstract__subsasgn}%
\hypertarget{ref_plot_abstract__subsasgn}{}%
\begin{description}
\item[Summary:]Defines generic index-based assignment for objects.
%
%
%
%
%
%
%
\item[Author:]%
Cengiz Gunay <cgunay@emory.edu>, 2006/02/06
%
\end{description}
\methodline%
\subsubsection[Method \texttt{subsref}]{Method \texttt{plot\_abstract/subsref}}%
\index[funcref]{plot_abstract@\fidxl{plot\_abstract}!subsref@\fidxl{subsref}}%
\label{ref_plot_abstract__subsref}%
\hypertarget{ref_plot_abstract__subsref}{}%
\begin{description}
\item[Summary:]Defines generic indexing for objects.
%
%
%
%
%
%
%
%
\end{description}
\methodline%
\subsubsection[Method \texttt{superposePlots}]{Method \texttt{plot\_abstract/superposePlots}}%
\index[funcref]{plot_abstract@\fidxl{plot\_abstract}!superposePlots@\fidxl{superposePlots}}%
\label{ref_plot_abstract__superposePlots}%
\hypertarget{ref_plot_abstract__superposePlots}{}%
\begin{description}
\item[Summary:]Superpose multiple plots with common command onto a single axis.
%
\item[Usage:]~%
\begin{lyxcode}%
a\_plot = superposePlots(plots, axis\_labels, title\_str, command, props)
%
\end{lyxcode}%
%
\item[Description:]%
The plot decoration will be taken from the last plot in the list, 
 with the exception of legend labels.
%%
\item[Parameters:]~
\begin{description}%
\item[\texttt{plots}:]
 Array of plot\_abstract or subclass objects.
\item[\texttt{axis\_labels}:]
 Cell array of axis label strings (optional, taken from plots).
\item[\texttt{title\_str}:]
 Plot description string (optional, taken from plots).
\item[\texttt{command}:]
 Plotting command to use (optional, taken from plots)
\item[\texttt{props}:]
 A structure with any optional properties.
\begin{description}%
\item[\texttt{noLegends}:]
 If exists, no legends are created.
\end{description}%
\end{description}%
%
\item[Returns:
]~

	a\_plot: A plot\_abstract object.
%
%
\item[See also:]%
\hyperlink{ref_plot_abstract}{\texttt{plot\_abstract}}%
\ (p.~\pageref{ref_plot_abstract})%
\index[funcref]{plot_abstract@\fidxl{plot\_abstract}}%
, \hyperlink{ref_plot_abstract__plot}{\texttt{plot\_abstract/plot}}%
\ (p.~\pageref{ref_plot_abstract__plot})%
\index[funcref]{plot_abstract@\fidxl{plot\_abstract}!plot@\fidxl{plot}}%
, \hyperlink{ref_plot_abstract__plotFigure}{\texttt{plot\_abstract/plotFigure}}%
\ (p.~\pageref{ref_plot_abstract__plotFigure})%
\index[funcref]{plot_abstract@\fidxl{plot\_abstract}!plotFigure@\fidxl{plotFigure}}%
%
\item[Author:]%
Cengiz Gunay <cgunay@emory.edu>, 2004/09/23
%
\end{description}
\methodline%
\subsection{Class \texttt{plot\_bars}}%
\index[funcref]{plot_bars@\fidxl{plot\_bars}|boldhyperpage}%
\label{ref_plot_bars}%
\hypertarget{ref_plot_bars}{}%
\subsubsection[Constructor \texttt{plot\_bars}]{Constructor \texttt{plot\_bars/plot\_bars}}%
\index[funcref]{plot_bars@\fidxl{plot\_bars}!plot_bars@\fidxl{plot\_bars}}%
\label{ref_plot_bars__plot_bars}%
\hypertarget{ref_plot_bars__plot_bars}{}%
\begin{description}
\item[Summary:]Bar plot with error lines in individual axes for each variable.
%
\item[Usage:]~%
\begin{lyxcode}%
a\_plot = plot\_bars(mid\_vals, lo\_vals, hi\_vals, n\_vals, x\_labels, y\_labels, ...
		     title\_str, axis\_limits, props)
%
\end{lyxcode}%
%
\item[Description:]%
Additional rows of data will result in grouped bars in each axis. If all
 data is given as a column vector, then they will appear in a single
 axis. plot\_bars is a subclass of plot\_stack. The plot\_abstract/plot
 command can be used to plot this data. Rows of *\_vals will create grouped
 bars, columns will create new axes.
%%
\item[Parameters:]~
\begin{description}%
\item[\texttt{mid\_vals}:]
 Middle points of error bars.
\item[\texttt{lo\_vals}:]
 Low points of error bars as difference from mid\_vals.
\item[\texttt{hi\_vals}:]
 High points of error bars as difference from mid\_vals.
\item[\texttt{n\_vals}:]
 Number of samples used for the statistic (Optional).
\item[\texttt{x\_labels, y\_labels}:]
 Axis labels for each bar group. Must match with data columns.
\item[\texttt{title\_str}:]
 Plot description.
\item[\texttt{axis\_limits}:]
 If given, all plots contained will have these axis limits.
\item[\texttt{props}:]
 A structure with any optional properties.
\begin{description}%
\item[\texttt{dispBarsLines}:]
 Choose between using 'bars' or 'lines' to connect the errorbars.
\item[\texttt{dispErrorbars}:]
 If 1, display errorbars for lo\_vals and hi\_vals deviation from mid\_vals 

(default=1).
\item[\texttt{dispInnerBars}:]
 If 1, an inner bar extends from the base to hi\_vals

(default=0). Mutually exclusive with
dispInnerBars. It will make the larger bars blank.
\item[\texttt{dispNvals}:]
 If 1, display n\_vals on top of each bar (default=1).
\item[\texttt{groupValues}:]
 List of within-group labels passed to XTickLabels,

instead of just a sequence of numbers.
\item[\texttt{groupValuesLoc}:]
 If 1, use specified group values as the location of bars

or errorbars. By default locations are set arbitrarily as 1:n.
\item[\texttt{truncateDecDigits}:]
 Truncate labels to this many decimal digits.
\item[\texttt{barAxisProps}:]
 props passed to plot\_abstract objects with bar commands
\item[\texttt{barWidth}:]
 Controls spacing between bars (see width argument for the

bar command; default=0.8).
\end{description}%
\end{description}%
%
\item[Returns a structure object with the following fields:
]~

   plot\_abstract
%
%
\item[See also:]%
\hyperlink{ref_plot_abstract}{\texttt{plot\_abstract}}%
\ (p.~\pageref{ref_plot_abstract})%
\index[funcref]{plot_abstract@\fidxl{plot\_abstract}}%
, \hyperlink{ref_plot_abstract__plot}{\texttt{plot\_abstract/plot}}%
\ (p.~\pageref{ref_plot_abstract__plot})%
\index[funcref]{plot_abstract@\fidxl{plot\_abstract}!plot@\fidxl{plot}}%
%
\item[Author:]%
Cengiz Gunay <cgunay@emory.edu>, 2004/10/07
%
\end{description}
\methodline%
\subsubsection[Method \texttt{set}]{Method \texttt{plot\_bars/set}}%
\index[funcref]{plot_bars@\fidxl{plot\_bars}!set@\fidxl{set}}%
\label{ref_plot_bars__set}%
\hypertarget{ref_plot_bars__set}{}%
\begin{description}
\item[Summary:]Generic method for setting object attributes.
%
%
%
%
%
%
%
\item[Author:]%
Cengiz Gunay <cgunay@emory.edu>, 2004/10/08
%
\end{description}
\methodline%
\subsection{Class \texttt{plot\_errorbar}}%
\index[funcref]{plot_errorbar@\fidxl{plot\_errorbar}|boldhyperpage}%
\label{ref_plot_errorbar}%
\hypertarget{ref_plot_errorbar}{}%
\subsubsection[Constructor \texttt{plot\_errorbar}]{Constructor \texttt{plot\_errorbar/plot\_errorbar}}%
\index[funcref]{plot_errorbar@\fidxl{plot\_errorbar}!plot_errorbar@\fidxl{plot\_errorbar}}%
\label{ref_plot_errorbar__plot_errorbar}%
\hypertarget{ref_plot_errorbar__plot_errorbar}{}%
\begin{description}
\item[Summary:]Generic errorbar plot.
%
\item[Usage:]~%
\begin{lyxcode}%
a\_plot = plot\_errorbar(x\_vals, mid\_vals, lo\_vals, hi\_vals, line\_spec, 
			 axis\_labels, title, legend, props)
%
\end{lyxcode}%
%
\item[Description:]%
Subclass of plot\_abstract. The plot\_abstract/plot command can be used to
 plot this data. Needed to create this as a separate class to have the
 axis ranges method to measure the errorbars.
%%
\item[Parameters:]~
\begin{description}%
\item[\texttt{x\_vals}:]
 X coordinates of errorbars.
\item[\texttt{mid\_vals}:]
 Middle points of error bars.
\item[\texttt{lo\_vals}:]
 Low points of error bars.
\item[\texttt{hi\_vals}:]
 High points of error bars.
\item[\texttt{line\_spec}:]
 Plot line spec to be passed to errorbar
\item[\texttt{axis\_labels}:]
 Cell array for X, Y axis labels.
\item[\texttt{title}:]
 Plot description.
\item[\texttt{legend}:]
 For multiple errorbar plots (matrix form), description of each plot.
\item[\texttt{props}:]
 A structure with any optional properties to be passed to plot\_abstract.
\end{description}%
%
\item[Returns a structure object with the following fields:
]~

	plot\_abstract.
%
%
\item[See also:]%
\hyperlink{ref_plot_abstract}{\texttt{plot\_abstract}}%
\ (p.~\pageref{ref_plot_abstract})%
\index[funcref]{plot_abstract@\fidxl{plot\_abstract}}%
, \hyperlink{ref_plot_abstract__plot}{\texttt{plot\_abstract/plot}}%
\ (p.~\pageref{ref_plot_abstract__plot})%
\index[funcref]{plot_abstract@\fidxl{plot\_abstract}!plot@\fidxl{plot}}%
%
\item[Author:]%
Cengiz Gunay <cgunay@emory.edu>, 2004/10/07
%
\end{description}
\methodline%
\subsubsection[Method \texttt{axis}]{Method \texttt{plot\_errorbar/axis}}%
\index[funcref]{plot_errorbar@\fidxl{plot\_errorbar}!axis@\fidxl{axis}}%
\label{ref_plot_errorbar__axis}%
\hypertarget{ref_plot_errorbar__axis}{}%
\begin{description}
\item[Summary:]Returns the estimated axis ranges of this plot according to its data.
%
\item[Usage:]~%
\begin{lyxcode}%
ranges = axis(a\_plot)
%
\end{lyxcode}%
%
%
\item[Parameters:]~
\begin{description}%
\item[\texttt{a\_plot}:]
 A plot\_abstract object, or a subclass object.
\end{description}%
%
\item[Returns:
]~

	ranges: The ranges as a vector in the same way 'axis' would return.
%
%
\item[See also:]%
\hyperlink{ref_plot_abstract}{\texttt{plot\_abstract}}%
\ (p.~\pageref{ref_plot_abstract})%
\index[funcref]{plot_abstract@\fidxl{plot\_abstract}}%
, \hyperlink{ref_plot_abstract__plot}{\texttt{plot\_abstract/plot}}%
\ (p.~\pageref{ref_plot_abstract__plot})%
\index[funcref]{plot_abstract@\fidxl{plot\_abstract}!plot@\fidxl{plot}}%
%
\item[Author:]%
Cengiz Gunay <cgunay@emory.edu>, 2004/10/13
%
\end{description}
\methodline%
\subsubsection[Method \texttt{get}]{Method \texttt{plot\_errorbar/get}}%
\index[funcref]{plot_errorbar@\fidxl{plot\_errorbar}!get@\fidxl{get}}%
\label{ref_plot_errorbar__get}%
\hypertarget{ref_plot_errorbar__get}{}%
\begin{description}
\item[Summary:]Defines generic attribute retrieval for objects.
%
%
%
%
%
%
%
\item[Author:]%
Cengiz Gunay <cgunay@emory.edu>, 2004/09/14
%
\end{description}
\methodline%
\subsection{Class \texttt{plot\_errorbars}}%
\index[funcref]{plot_errorbars@\fidxl{plot\_errorbars}|boldhyperpage}%
\label{ref_plot_errorbars}%
\hypertarget{ref_plot_errorbars}{}%
\subsubsection[Constructor \texttt{plot\_errorbars}]{Constructor \texttt{plot\_errorbars/plot\_errorbars}}%
\index[funcref]{plot_errorbars@\fidxl{plot\_errorbars}!plot_errorbars@\fidxl{plot\_errorbars}}%
\label{ref_plot_errorbars__plot_errorbars}%
\hypertarget{ref_plot_errorbars__plot_errorbars}{}%
\begin{description}
\item[Summary:]Plots distributions of variables with errorbars in separate axes.
%
\item[Usage:]~%
\begin{lyxcode}%
a\_plot = plot\_errorbars(labels, mid\_vals, lo\_vals, hi\_vals, labels, 
			 title, axis\_limits, props)
%
\end{lyxcode}%
%
\item[Description:]%
Subclass of plot\_stack. The plot\_abstract/plot command can be used to
 plot this data. Each of mid\_vals, lo\_vals, and hi\_vals plot its rows in
 the same axis and columns in different axes.
%%
\item[Parameters:]~
\begin{description}%
\item[\texttt{labels}:]
 Labels of parameters to appear at bottom of each errorbar.
\item[\texttt{mid\_vals}:]
 Middle points of error bars.
\item[\texttt{lo\_vals}:]
 Low points of error bars.
\item[\texttt{hi\_vals}:]
 High points of error bars.
\item[\texttt{title}:]
 Plot description.
\item[\texttt{axis\_limits}:]
 If given, all plots contained will have these axis limits.
\item[\texttt{props}:]
 A structure with any optional properties.
\end{description}%
%
\item[Returns a structure object with the following fields:
]~

	plot\_abstract, labels.
%
%
\item[See also:]%
\hyperlink{ref_plot_abstract}{\texttt{plot\_abstract}}%
\ (p.~\pageref{ref_plot_abstract})%
\index[funcref]{plot_abstract@\fidxl{plot\_abstract}}%
, \hyperlink{ref_plot_abstract__plot}{\texttt{plot\_abstract/plot}}%
\ (p.~\pageref{ref_plot_abstract__plot})%
\index[funcref]{plot_abstract@\fidxl{plot\_abstract}!plot@\fidxl{plot}}%
%
\item[Author:]%
Cengiz Gunay <cgunay@emory.edu>, 2004/10/07
%
\end{description}
\methodline%
\subsection{Class \texttt{plot\_image}}%
\index[funcref]{plot_image@\fidxl{plot\_image}|boldhyperpage}%
\label{ref_plot_image}%
\hypertarget{ref_plot_image}{}%
\subsubsection[Constructor \texttt{plot\_image}]{Constructor \texttt{plot\_image/plot\_image}}%
\index[funcref]{plot_image@\fidxl{plot\_image}!plot_image@\fidxl{plot\_image}}%
\label{ref_plot_image__plot_image}%
\hypertarget{ref_plot_image__plot_image}{}%
\begin{description}
\item[Summary:]Generic image plot.
%
\item[Usage:]~%
\begin{lyxcode}%
a\_plot = plot\_image(image\_data, axis\_labels, colorbar\_label, title\_str, props)
%
\end{lyxcode}%
%
\item[Description:]%
Subclass of plot\_abstract. The plot\_abstract/plot command can be used to
 plot this data. Needed to create this as a separate class to have the
 axis ranges method implemented and take advantage of plot\_abstract props.
%%
\item[Parameters:]~
\begin{description}%
\item[\texttt{image\_data}:]
 2D matrix with image data.
\item[\texttt{axis\_labels}:]
 Cell array for X, Y axis labels.
\item[\texttt{colorbar\_label}:]
 String to appear next to colorbar.
\item[\texttt{title\_str}:]
 Plot description.
\item[\texttt{props}:]
 A structure with any optional properties.
\begin{description}%
\item[\texttt{colorbar}:]
 If defined, show colorbar next to plot.
\item[\texttt{numGrads}:]
 Number of poles in the colormap gradient. If 1 (default),

it will be a monocolor gradient (e.g., gray-level); if 2, it will
be a dual-color gradient (e.g., blue to red) with a black
crossing point. This point is determined by the minValue
(below). Default numGrads=2, if negative values exist in
image\_data after scaling.
\item[\texttt{minValue,maxValue}:]
 Use these value to scale the data by 

(image\_data - minValue) / (maxValue - minValue). 
Otherwise, its min and max is used.
\item[\texttt{colormap}:]
 Colormap vector, function name or handle to colormap (e.g., 'jet').
\item[\texttt{numColors}:]
 Number of colors in colormap.
\item[\texttt{NaNcolor}:]
 Color to be used for NaN entries (default: [1 1 1])

(Rest passed to plotImage.)
\end{description}%
\end{description}%
%
\item[Returns a structure object with the following fields:
]~

	plot\_abstract.
%
\item[Example:]~
\begin{lyxcode} >> plotFigure(plot\_image(rand(5), {'r1', 'r2'}, 'rand', 'random matrix'))
\\%
\end{lyxcode}
%
\item[See also:]%
\hyperlink{ref_plot_abstract}{\texttt{plot\_abstract}}%
\ (p.~\pageref{ref_plot_abstract})%
\index[funcref]{plot_abstract@\fidxl{plot\_abstract}}%
, \hyperlink{ref_plot_abstract__plot}{\texttt{plot\_abstract/plot}}%
\ (p.~\pageref{ref_plot_abstract__plot})%
\index[funcref]{plot_abstract@\fidxl{plot\_abstract}!plot@\fidxl{plot}}%
%
\item[Author:]%
Cengiz Gunay <cgunay@emory.edu>, 2008/04/15
%
\end{description}
\methodline%
\subsubsection[Method \texttt{axis}]{Method \texttt{plot\_image/axis}}%
\index[funcref]{plot_image@\fidxl{plot\_image}!axis@\fidxl{axis}}%
\label{ref_plot_image__axis}%
\hypertarget{ref_plot_image__axis}{}%
\begin{description}
\item[Summary:]Returns the estimated axis ranges of this plot according to its data.
%
\item[Usage:]~%
\begin{lyxcode}%
ranges = axis(a\_plot)
%
\end{lyxcode}%
%
%
\item[Parameters:]~
\begin{description}%
\item[\texttt{a\_plot}:]
 A plot\_abstract, or subclass, object.
\end{description}%
%
\item[Returns:
]~

	ranges: The ranges as a vector in the same way 'axis' would return.
%
%
\item[See also:]%
\hyperlink{ref_plot_abstract}{\texttt{plot\_abstract}}%
\ (p.~\pageref{ref_plot_abstract})%
\index[funcref]{plot_abstract@\fidxl{plot\_abstract}}%
, \hyperlink{ref_plot_abstract__plot}{\texttt{plot\_abstract/plot}}%
\ (p.~\pageref{ref_plot_abstract__plot})%
\index[funcref]{plot_abstract@\fidxl{plot\_abstract}!plot@\fidxl{plot}}%
, \hyperlink{ref_axis}{\texttt{axis}}%
\ (p.~\pageref{ref_axis})%
\index[funcref]{axis@\fidxl{axis}}%
%
\item[Author:]%
Cengiz Gunay <cgunay@emory.edu>, 2004/10/13
%
\end{description}
\methodline%
\subsubsection[Method \texttt{get}]{Method \texttt{plot\_image/get}}%
\index[funcref]{plot_image@\fidxl{plot\_image}!get@\fidxl{get}}%
\label{ref_plot_image__get}%
\hypertarget{ref_plot_image__get}{}%
\begin{description}
\item[Summary:]Defines generic attribute retrieval for objects.
%
%
%
%
%
%
%
\item[Author:]%
Cengiz Gunay <cgunay@emory.edu>, 2004/09/14
%
\end{description}
\methodline%
\subsection{Class \texttt{plot\_inset}}%
\index[funcref]{plot_inset@\fidxl{plot\_inset}|boldhyperpage}%
\label{ref_plot_inset}%
\hypertarget{ref_plot_inset}{}%
\subsubsection[Constructor \texttt{plot\_inset}]{Constructor \texttt{plot\_inset/plot\_inset}}%
\index[funcref]{plot_inset@\fidxl{plot\_inset}!plot_inset@\fidxl{plot\_inset}}%
\label{ref_plot_inset__plot_inset}%
\hypertarget{ref_plot_inset__plot_inset}{}%
\begin{description}
\item[Summary:]Superpose multiple plots with individual axis at arbitrary locations.
%
\item[Usage:]~%
\begin{lyxcode}%
a\_plot = plot\_inset(plots, axis\_locations, title\_str, props)
%
\end{lyxcode}%
%
\item[Description:]%
Subclass of plot\_abstract. Contains other plot\_abstract objects or
 subclasses thereof to be layout in arbitaray format. Allows overlapping
 and therefore good for insets and special plots.
%%
\item[Parameters:]~
\begin{description}%
\item[\texttt{plots}:]
 Cell array of plot\_abstract or subclass objects.
\item[\texttt{axis\_locations}:]
 Matrix of four-element vectors for each given plot.
\item[\texttt{title\_str}:]
 Title to go on top of the stack
\item[\texttt{props}:]
 A structure with any optional properties.
\begin{description}%
\item[\texttt{positioning}:]
 axis\_locations interpreted as 'absolute' values or

'relative' to the 1st plot (default='absolute'). 
Relative positioning doesn't work well.
\end{description}%
\end{description}%
%
\item[Returns a structure object with the following fields:
]~

	plot\_abstract, plots, axis\_locations.
%
%
\item[See also:]%
\hyperlink{ref_plot_abstract}{\texttt{plot\_abstract}}%
\ (p.~\pageref{ref_plot_abstract})%
\index[funcref]{plot_abstract@\fidxl{plot\_abstract}}%
, \hyperlink{ref_plot_abstract__plotFigure}{\texttt{plot\_abstract/plotFigure}}%
\ (p.~\pageref{ref_plot_abstract__plotFigure})%
\index[funcref]{plot_abstract@\fidxl{plot\_abstract}!plotFigure@\fidxl{plotFigure}}%
%
\item[Author:]%
Cengiz Gunay <cgunay@emory.edu>, 2007/06/05
%
\end{description}
\methodline%
\subsubsection[Method \texttt{get}]{Method \texttt{plot\_inset/get}}%
\index[funcref]{plot_inset@\fidxl{plot\_inset}!get@\fidxl{get}}%
\label{ref_plot_inset__get}%
\hypertarget{ref_plot_inset__get}{}%
\begin{description}
\item[Summary:]Defines generic attribute retrieval for objects.
%
%
%
%
%
%
%
\item[Author:]%
Cengiz Gunay <cgunay@emory.edu>, 2004/09/14
%
\end{description}
\methodline%
\subsubsection[Method \texttt{plot}]{Method \texttt{plot\_inset/plot}}%
\index[funcref]{plot_inset@\fidxl{plot\_inset}!plot@\fidxl{plot}}%
\label{ref_plot_inset__plot}%
\hypertarget{ref_plot_inset__plot}{}%
\begin{description}
\item[Summary:]Superposes contained plots in their own axes.
%
\item[Usage:]~%
\begin{lyxcode}%
handles = plot(a\_plot, layout\_axis)
%
\end{lyxcode}%
%
%
\item[Parameters:]~
\begin{description}%
\item[\texttt{a\_plot}:]
 A plot\_superpose object.
\item[\texttt{layout\_axis}:]
 The axis position to layout this plot (Optional). 
\end{description}%
%
\item[Returns:
]~

	handles: Handles of graphical objects drawn.
%
%
\item[See also:]%
\hyperlink{ref_plot_abstract}{\texttt{plot\_abstract}}%
\ (p.~\pageref{ref_plot_abstract})%
\index[funcref]{plot_abstract@\fidxl{plot\_abstract}}%
%
\item[Author:]%
Cengiz Gunay <cgunay@emory.edu>, 2007/06/08
%
\end{description}
\methodline%
\subsubsection[Method \texttt{set}]{Method \texttt{plot\_inset/set}}%
\index[funcref]{plot_inset@\fidxl{plot\_inset}!set@\fidxl{set}}%
\label{ref_plot_inset__set}%
\hypertarget{ref_plot_inset__set}{}%
\begin{description}
\item[Summary:]Generic method for setting object attributes.
%
%
%
%
%
%
%
\item[Author:]%
Cengiz Gunay <cgunay@emory.edu>, 2004/10/08
%
\end{description}
\methodline%
\subsection{Class \texttt{plot\_simple}}%
\index[funcref]{plot_simple@\fidxl{plot\_simple}|boldhyperpage}%
\label{ref_plot_simple}%
\hypertarget{ref_plot_simple}{}%
\subsubsection[Constructor \texttt{plot\_simple}]{Constructor \texttt{plot\_simple/plot\_simple}}%
\index[funcref]{plot_simple@\fidxl{plot\_simple}!plot_simple@\fidxl{plot\_simple}}%
\label{ref_plot_simple__plot_simple}%
\hypertarget{ref_plot_simple__plot_simple}{}%
\begin{description}
\item[Summary:]Abstract description of a single plot.
%
\item[Usage:]~%
\begin{lyxcode}%
a\_plot = plot\_simple(data\_x, data\_y, title, 
		       label\_x, label\_y, legend, command, props)
%
\end{lyxcode}%
%
\item[Description:]%
Subclass of plot\_abstract. The plot\_abstract/plot command can be used to
 plot this data.
%%
\item[Parameters:]~
\begin{description}%
\item[\texttt{data\_x}:]
 X-axis values for the plot.
\item[\texttt{data\_y}:]
 Y-axis values for the plot.
\item[\texttt{title}:]
 Plot description.
\item[\texttt{label\_x}:]
 X-axis label string.
\item[\texttt{label\_y}:]
 Y-axis label string.
\item[\texttt{legend}:]
 Short description of data points.
\item[\texttt{command}:]
 Plotting command to use (Optional, default='plot')
\item[\texttt{props}:]
 A structure with any optional properties.
\end{description}%
%
\item[Returns a structure object with the following fields:
]~

	plot\_abstract.
%
%
\item[See also:]%
\hyperlink{ref_plot_abstract}{\texttt{plot\_abstract}}%
\ (p.~\pageref{ref_plot_abstract})%
\index[funcref]{plot_abstract@\fidxl{plot\_abstract}}%
, \hyperlink{ref_plot_abstract__plot}{\texttt{plot\_abstract/plot}}%
\ (p.~\pageref{ref_plot_abstract__plot})%
\index[funcref]{plot_abstract@\fidxl{plot\_abstract}!plot@\fidxl{plot}}%
%
\item[Author:]%
Cengiz Gunay <cgunay@emory.edu>, 2004/09/22
%
\end{description}
\methodline%
\subsubsection[Method \texttt{get}]{Method \texttt{plot\_simple/get}}%
\index[funcref]{plot_simple@\fidxl{plot\_simple}!get@\fidxl{get}}%
\label{ref_plot_simple__get}%
\hypertarget{ref_plot_simple__get}{}%
\begin{description}
\item[Summary:]Defines generic attribute retrieval for objects.
%
%
%
%
%
%
%
\item[Author:]%
Cengiz Gunay <cgunay@emory.edu>, 2004/09/14
%
\end{description}
\methodline%
\subsubsection[Method \texttt{set}]{Method \texttt{plot\_simple/set}}%
\index[funcref]{plot_simple@\fidxl{plot\_simple}!set@\fidxl{set}}%
\label{ref_plot_simple__set}%
\hypertarget{ref_plot_simple__set}{}%
\begin{description}
\item[Summary:]Generic method for setting object attributes.
%
%
%
%
%
%
%
\item[Author:]%
Cengiz Gunay <cgunay@emory.edu>, 2004/10/08
%
\end{description}
\methodline%
\subsection{Class \texttt{plot\_stack}}%
\index[funcref]{plot_stack@\fidxl{plot\_stack}|boldhyperpage}%
\label{ref_plot_stack}%
\hypertarget{ref_plot_stack}{}%
\subsubsection[Constructor \texttt{plot\_stack}]{Constructor \texttt{plot\_stack/plot\_stack}}%
\index[funcref]{plot_stack@\fidxl{plot\_stack}!plot_stack@\fidxl{plot\_stack}}%
\label{ref_plot_stack__plot_stack}%
\hypertarget{ref_plot_stack__plot_stack}{}%
\begin{description}
\item[Summary:]A horizontal or vertical stack of plots.
%
\item[Usage:]~%
\begin{lyxcode}%
a\_plot = plot\_stack(plots, axis\_limits, orientation, title\_str, props)
%
\end{lyxcode}%
%
\item[Description:]%
Subclass of plot\_abstract. Contains other plot\_abstract objects or
 subclasses thereof to be layout in stack format. 
%%
\item[Parameters:]~
\begin{description}%
\item[\texttt{plots}:]
 Cell array of plot\_abstract or subclass objects.
\item[\texttt{axis\_limits}:]
 If given, all plots contained will have these axis

limits. In this vector, NaNs are untouched, Infs are
replaced by minimal and maximal ranges of the
stacked plots.
\item[\texttt{orientation}:]
 Stack orientation 'x' for horizontal, 'y' for vertical, etc.
\item[\texttt{title\_str}:]
 Title to go on top of the stack
\item[\texttt{props}:]
 A structure with any optional properties.
\begin{description}%
\item[\texttt{yLabelsPos}:]
 'left' means only put y-axis label to leftmost plot.
\item[\texttt{yTicksPos}:]
 'left' means only put y-axis ticks to leftmost plot.
\item[\texttt{xLabelsPos}:]
 'bottom' means only put x-axis label to lowest plot.
\item[\texttt{xTicksPos}:]
 'bottom' means only put x-axis ticks to lowest plot.
\item[\texttt{titlesPos}:]
 'top' means only put title to top plot.
\item[\texttt{relaxedLimits}:]
 Add 10% to all axis limits, overriding Matlab's layout
\item[\texttt{relativeSizes}:]
 An array specifying relative size of each plot with one value.

(Example: relativeSizes=[1 2] makes second plot twice wider than first.)
\end{description}%
\end{description}%
%
\item[Returns a structure object with the following fields:
]~

	plot\_abstract, plots, axis\_limits, orient.
%
%
\item[See also:]%
\hyperlink{ref_plot_abstract}{\texttt{plot\_abstract}}%
\ (p.~\pageref{ref_plot_abstract})%
\index[funcref]{plot_abstract@\fidxl{plot\_abstract}}%
, \hyperlink{ref_plot_abstract__plotFigure}{\texttt{plot\_abstract/plotFigure}}%
\ (p.~\pageref{ref_plot_abstract__plotFigure})%
\index[funcref]{plot_abstract@\fidxl{plot\_abstract}!plotFigure@\fidxl{plotFigure}}%
%
\item[Author:]%
Cengiz Gunay <cgunay@emory.edu>, 2004/10/04
%
\end{description}
\methodline%
\subsubsection[Method \texttt{decorate}]{Method \texttt{plot\_stack/decorate}}%
\index[funcref]{plot_stack@\fidxl{plot\_stack}!decorate@\fidxl{decorate}}%
\label{ref_plot_stack__decorate}%
\hypertarget{ref_plot_stack__decorate}{}%
\begin{description}
\item[Summary:]No additional decorations for stacked plots.
%
\item[Usage:]~%
\begin{lyxcode}%
a\_histogram\_db = decorate(a\_plot)
%
\end{lyxcode}%
%
%
\item[Parameters:]~
\begin{description}%
\item[\texttt{a\_plot}:]
 A plot\_abstract object, or a subclass object.
\end{description}%
%
\item[Returns:
]~

	handles: Handles of graphical objects drawn.
%
%
\item[See also:]%
\hyperlink{ref_plot_abstract}{\texttt{plot\_abstract}}%
\ (p.~\pageref{ref_plot_abstract})%
\index[funcref]{plot_abstract@\fidxl{plot\_abstract}}%
, \hyperlink{ref_plot_abstract__plot}{\texttt{plot\_abstract/plot}}%
\ (p.~\pageref{ref_plot_abstract__plot})%
\index[funcref]{plot_abstract@\fidxl{plot\_abstract}!plot@\fidxl{plot}}%
%
\item[Author:]%
Cengiz Gunay <cgunay@emory.edu>, 2004/10/04
%
\end{description}
\methodline%
\subsubsection[Method \texttt{display}]{Method \texttt{plot\_stack/display}}%
\index[funcref]{plot_stack@\fidxl{plot\_stack}!display@\fidxl{display}}%
\label{ref_plot_stack__display}%
\hypertarget{ref_plot_stack__display}{}%
\begin{description}
%
%
%
%
%
%
%
\item[Author:]%
Cengiz Gunay <cgunay@emory.edu>, 2004/08/04
%
\end{description}
\methodline%
\subsubsection[Method \texttt{get}]{Method \texttt{plot\_stack/get}}%
\index[funcref]{plot_stack@\fidxl{plot\_stack}!get@\fidxl{get}}%
\label{ref_plot_stack__get}%
\hypertarget{ref_plot_stack__get}{}%
\begin{description}
\item[Summary:]Defines generic attribute retrieval for objects.
%
%
%
%
%
%
%
\item[Author:]%
Cengiz Gunay <cgunay@emory.edu>, 2004/09/14
%
\end{description}
\methodline%
\subsubsection[Method \texttt{plot}]{Method \texttt{plot\_stack/plot}}%
\index[funcref]{plot_stack@\fidxl{plot\_stack}!plot@\fidxl{plot}}%
\label{ref_plot_stack__plot}%
\hypertarget{ref_plot_stack__plot}{}%
\begin{description}
\item[Summary:]Draws this plot in the current axis or at the position in
	layout\_axis.
%
\item[Usage:]~%
\begin{lyxcode}%
handles = plot(a\_plot, layout\_axis)
%
\end{lyxcode}%
%
%
\item[Parameters:]~
\begin{description}%
\item[\texttt{a\_plot}:]
 A plot\_abstract object, or a subclass object.
\item[\texttt{layout\_axis}:]
 The axis position to layout this plot (Optional). 
\end{description}%
%
\item[Returns:
]~

	handles: Handles of graphical objects drawn.
%
%
\item[See also:]%
\hyperlink{ref_plot_stack}{\texttt{plot\_stack}}%
\ (p.~\pageref{ref_plot_stack})%
\index[funcref]{plot_stack@\fidxl{plot\_stack}}%
, \hyperlink{ref_plot_abstract}{\texttt{plot\_abstract}}%
\ (p.~\pageref{ref_plot_abstract})%
\index[funcref]{plot_abstract@\fidxl{plot\_abstract}}%
%
\item[Author:]%
Cengiz Gunay <cgunay@emory.edu>, 2004/10/04
%
\end{description}
\methodline%
\subsubsection[Method \texttt{set}]{Method \texttt{plot\_stack/set}}%
\index[funcref]{plot_stack@\fidxl{plot\_stack}!set@\fidxl{set}}%
\label{ref_plot_stack__set}%
\hypertarget{ref_plot_stack__set}{}%
\begin{description}
\item[Summary:]Generic method for setting object attributes.
%
%
%
%
%
%
%
\item[Author:]%
Cengiz Gunay <cgunay@emory.edu>, 2004/10/08
%
\end{description}
\methodline%
\subsubsection[Method \texttt{superposePlots}]{Method \texttt{plot\_stack/superposePlots}}%
\index[funcref]{plot_stack@\fidxl{plot\_stack}!superposePlots@\fidxl{superposePlots}}%
\label{ref_plot_stack__superposePlots}%
\hypertarget{ref_plot_stack__superposePlots}{}%
\begin{description}
\item[Summary:]Superpose multiple plot\_stack objects that contain exact same contents.
%
\item[Usage:]~%
\begin{lyxcode}%
a\_plot = superposePlots(plots, axis\_labels, title\_str, command, props)
%
\end{lyxcode}%
%
\item[Description:]%
The plot decoration will be taken from the last plot in the list, 
 with the exception of legend labels.
%%
\item[Parameters:]~
\begin{description}%
\item[\texttt{plots}:]
 Array of plot\_stack objects.
\item[\texttt{axis\_labels}:]
 Cell array of axis label strings (optional, taken from plots).
\item[\texttt{title\_str}:]
 Plot description string (optional, taken from plots).
\item[\texttt{command}:]
 Plotting command to use (optional, taken from plots)
\item[\texttt{props}:]
 A structure with any optional properties.
\begin{description}%
\item[\texttt{noLegends}:]
 If exists, no legends are created.
\end{description}%
\end{description}%
%
\item[Returns:
]~

	a\_plot: A plot\_stack object.
%
%
\item[See also:]%
\hyperlink{ref_plot_abstract}{\texttt{plot\_abstract}}%
\ (p.~\pageref{ref_plot_abstract})%
\index[funcref]{plot_abstract@\fidxl{plot\_abstract}}%
, \hyperlink{ref_plot_abstract__plot}{\texttt{plot\_abstract/plot}}%
\ (p.~\pageref{ref_plot_abstract__plot})%
\index[funcref]{plot_abstract@\fidxl{plot\_abstract}!plot@\fidxl{plot}}%
, \hyperlink{ref_plot_abstract__plotFigure}{\texttt{plot\_abstract/plotFigure}}%
\ (p.~\pageref{ref_plot_abstract__plotFigure})%
\index[funcref]{plot_abstract@\fidxl{plot\_abstract}!plotFigure@\fidxl{plotFigure}}%
%
\item[Author:]%
Cengiz Gunay <cgunay@emory.edu>, 2006/06/14
%
\end{description}
\methodline%
\subsection{Class \texttt{plot\_superpose}}%
\index[funcref]{plot_superpose@\fidxl{plot\_superpose}|boldhyperpage}%
\label{ref_plot_superpose}%
\hypertarget{ref_plot_superpose}{}%
\subsubsection[Constructor \texttt{plot\_superpose}]{Constructor \texttt{plot\_superpose/plot\_superpose}}%
\index[funcref]{plot_superpose@\fidxl{plot\_superpose}!plot_superpose@\fidxl{plot\_superpose}}%
\label{ref_plot_superpose__plot_superpose}%
\hypertarget{ref_plot_superpose__plot_superpose}{}%
\begin{description}
\item[Summary:]Multiple plot\_abstract objects superposed on the same axis.
%
\item[Usage:]~%
\begin{lyxcode}%
obj = plot\_superpose(plots, axis\_labels, title\_str, props)
%
\end{lyxcode}%
%
\item[Description:]%
Subclass of plot\_abstract. Contains multiple plot\_abstract objects to be
 plotted on the same axis. This is different than the
 plot\_abstract/superpose, where only using the same plot command is
 allowed.  Here, each plot\_abstract can have its own special plotting
 command. Subclasses of plot\_abstract is also allowed here. The decorations
 comes from this object and not children plots. This behavior is different
 than plot\_stack, where each plot has its own decorations. If you want each
 plot to have its own axis (e.g. an inset, or plot with multiple axis
 labels) then you should use plot\_inset. For convenience, the props of
 first plot is inherited by plot\_superpose objects. For instance, in the
 example below, the fixedSize property is used by plotFigure because
 it's in the first subplot.
%%
\item[Parameters:]~
\begin{description}%
\item[\texttt{plots}:]
 Cell array of plot\_abstract or subclass objects.
\item[\texttt{axis\_labels}:]
 Cell array of axis label strings.
\item[\texttt{title\_str}:]
 Plot description string.
\item[\texttt{props}:]
 A structure with any optional properties (passed to

plot\_abstract).
\begin{description}%
\item[\texttt{noCombine}:]
 Do not auto-combine plots with same properties

(default=0). This is especially important for plot\_bars.
\end{description}%
\end{description}%
%
\item[Returns a structure object with the following fields:
]~

	plot\_abstract, plots
%
\item[Example:]~
\begin{lyxcode} >> a\_p = plot\_abstract({[0 0], [1 2]}, {}, '', {}, 'plot',
\\%
      struct('fixedSize', [3 2]));
\\%
 >> a\_p2 = plot\_abstract(...)
\\%
 >> a\_p3 = plot\_abstract(...)
\\%
 >> plotFigure(plot\_superpose({a\_p1, a\_p2, a\_p3}))
\\%
\end{lyxcode}
%
\item[See also:]%
\hyperlink{ref_plot_abstract__superpose}{\texttt{plot\_abstract/superpose}}%
\ (p.~\pageref{ref_plot_abstract__superpose})%
\index[funcref]{plot_abstract@\fidxl{plot\_abstract}!superpose@\fidxl{superpose}}%
, \hyperlink{ref_plot_superpose__plot}{\texttt{plot\_superpose/plot}}%
\ (p.~\pageref{ref_plot_superpose__plot})%
\index[funcref]{plot_superpose@\fidxl{plot\_superpose}!plot@\fidxl{plot}}%
%
\item[Author:]%
Cengiz Gunay <cgunay@emory.edu>, 2004/09/22
%
\end{description}
\methodline%
\subsubsection[Method \texttt{axis}]{Method \texttt{plot\_superpose/axis}}%
\index[funcref]{plot_superpose@\fidxl{plot\_superpose}!axis@\fidxl{axis}}%
\label{ref_plot_superpose__axis}%
\hypertarget{ref_plot_superpose__axis}{}%
\begin{description}
\item[Summary:]Returns the maximal axis ranges according to superposed subplots.
%
\item[Usage:]~%
\begin{lyxcode}%
ranges = axis(a\_plot)
%
\end{lyxcode}%
%
%
\item[Parameters:]~
\begin{description}%
\item[\texttt{a\_plot}:]
 A plot\_abstract object, or a subclass object.
\end{description}%
%
\item[Returns:
]~

	ranges: The ranges as a vector in the same way 'axis' would return.
%
%
\item[See also:]%
\hyperlink{ref_plot_abstract}{\texttt{plot\_abstract}}%
\ (p.~\pageref{ref_plot_abstract})%
\index[funcref]{plot_abstract@\fidxl{plot\_abstract}}%
, \hyperlink{ref_plot_abstract__plot}{\texttt{plot\_abstract/plot}}%
\ (p.~\pageref{ref_plot_abstract__plot})%
\index[funcref]{plot_abstract@\fidxl{plot\_abstract}!plot@\fidxl{plot}}%
%
\item[Author:]%
Cengiz Gunay <cgunay@emory.edu>, 2006/05/22
%
\end{description}
\methodline%
\subsubsection[Method \texttt{decorate}]{Method \texttt{plot\_superpose/decorate}}%
\index[funcref]{plot_superpose@\fidxl{plot\_superpose}!decorate@\fidxl{decorate}}%
\label{ref_plot_superpose__decorate}%
\hypertarget{ref_plot_superpose__decorate}{}%
\begin{description}
\item[Summary:]Places decorations using the first plot of the superposed plots.
%
\item[Usage:]~%
\begin{lyxcode}%
handles = decorate(a\_plot, plot\_handles)
%
\end{lyxcode}%
%
%
\item[Parameters:]~
\begin{description}%
\item[\texttt{a\_plot}:]
 A plot\_abstract object, or a subclass object.
\item[\texttt{plot\_handles}:]
 Handles of plots already drawn (structure returned by

plot\_superpose/plot). 
\end{description}%
%
\item[Returns:
]~

   handles: Handles of graphical objects drawn.
%
%
\item[See also:]%
\hyperlink{ref_plot_abstract}{\texttt{plot\_abstract}}%
\ (p.~\pageref{ref_plot_abstract})%
\index[funcref]{plot_abstract@\fidxl{plot\_abstract}}%
, \hyperlink{ref_plot_abstract__plot}{\texttt{plot\_abstract/plot}}%
\ (p.~\pageref{ref_plot_abstract__plot})%
\index[funcref]{plot_abstract@\fidxl{plot\_abstract}!plot@\fidxl{plot}}%
%
\item[Author:]%
Cengiz Gunay <cgunay@emory.edu>, 2005/04/11
%
\end{description}
\methodline%
\subsubsection[Method \texttt{display}]{Method \texttt{plot\_superpose/display}}%
\index[funcref]{plot_superpose@\fidxl{plot\_superpose}!display@\fidxl{display}}%
\label{ref_plot_superpose__display}%
\hypertarget{ref_plot_superpose__display}{}%
\begin{description}
%
%
%
%
%
%
%
\item[Author:]%
Cengiz Gunay <cgunay@emory.edu>, 2004/08/04
%
\end{description}
\methodline%
\subsubsection[Method \texttt{get}]{Method \texttt{plot\_superpose/get}}%
\index[funcref]{plot_superpose@\fidxl{plot\_superpose}!get@\fidxl{get}}%
\label{ref_plot_superpose__get}%
\hypertarget{ref_plot_superpose__get}{}%
\begin{description}
\item[Summary:]Defines generic attribute retrieval for objects.
%
%
%
%
%
%
%
\item[Author:]%
Cengiz Gunay <cgunay@emory.edu>, 2004/09/14
%
\end{description}
\methodline%
\subsubsection[Method \texttt{plot}]{Method \texttt{plot\_superpose/plot}}%
\index[funcref]{plot_superpose@\fidxl{plot\_superpose}!plot@\fidxl{plot}}%
\label{ref_plot_superpose__plot}%
\hypertarget{ref_plot_superpose__plot}{}%
\begin{description}
\item[Summary:]Draws this plot in the current axis.
%
\item[Usage:]~%
\begin{lyxcode}%
handles = plot(a\_plot, layout\_axis)
%
\end{lyxcode}%
%
%
\item[Parameters:]~
\begin{description}%
\item[\texttt{a\_plot}:]
 A plot\_superpose object.
\item[\texttt{layout\_axis}:]
 The axis position to layout this plot (Optional). 
\end{description}%
%
\item[Returns:
]~

	handles: Handles of graphical objects drawn.
%
%
\item[See also:]%
\hyperlink{ref_plot_abstract}{\texttt{plot\_abstract}}%
\ (p.~\pageref{ref_plot_abstract})%
\index[funcref]{plot_abstract@\fidxl{plot\_abstract}}%
%
\item[Author:]%
Cengiz Gunay <cgunay@emory.edu>, 2005/04/08
%
\end{description}
\methodline%
\subsubsection[Method \texttt{set}]{Method \texttt{plot\_superpose/set}}%
\index[funcref]{plot_superpose@\fidxl{plot\_superpose}!set@\fidxl{set}}%
\label{ref_plot_superpose__set}%
\hypertarget{ref_plot_superpose__set}{}%
\begin{description}
\item[Summary:]Generic method for setting object attributes.
%
%
%
%
%
%
%
\item[Author:]%
Cengiz Gunay <cgunay@emory.edu>, 2004/10/08
%
\end{description}
\methodline%
\subsubsection[Method \texttt{superposePlots}]{Method \texttt{plot\_superpose/superposePlots}}%
\index[funcref]{plot_superpose@\fidxl{plot\_superpose}!superposePlots@\fidxl{superposePlots}}%
\label{ref_plot_superpose__superposePlots}%
\hypertarget{ref_plot_superpose__superposePlots}{}%
\begin{description}
\item[Summary:]Superpose multiple plot\_superpose objects by merging them into one.
%
\item[Usage:]~%
\begin{lyxcode}%
a\_plot = superposePlots(plots, axis\_labels, title\_str, command, props)
%
\end{lyxcode}%
%
%
\item[Parameters:]~
\begin{description}%
\item[\texttt{plots}:]
 Array of plot\_superpose objects.
\item[\texttt{axis\_labels}:]
 Cell array of axis label strings (optional, taken from plots).
\item[\texttt{title\_str}:]
 Plot description string (optional, taken from plots).
\item[\texttt{command}:]
 Plotting command to use (optional, taken from plots)
\item[\texttt{props}:]
 A structure with any optional properties.
\begin{description}%
\item[\texttt{noLegends}:]
 If exists, no legends are created.
\end{description}%
\end{description}%
%
\item[Returns:
]~

	a\_plot: A plot\_superpose object.
%
%
\item[See also:]%
\hyperlink{ref_plot_abstract__superposePlots}{\texttt{plot\_abstract/superposePlots}}%
\ (p.~\pageref{ref_plot_abstract__superposePlots})%
\index[funcref]{plot_abstract@\fidxl{plot\_abstract}!superposePlots@\fidxl{superposePlots}}%
, \hyperlink{ref_plot_stack__superposePlots}{\texttt{plot\_stack/superposePlots}}%
\ (p.~\pageref{ref_plot_stack__superposePlots})%
\index[funcref]{plot_stack@\fidxl{plot\_stack}!superposePlots@\fidxl{superposePlots}}%
%
\item[Author:]%
Cengiz Gunay <cgunay@emory.edu>, 2006/06/14
%
\end{description}
\methodline%
\subsection{Class \texttt{ranked\_db}}%
\index[funcref]{ranked_db@\fidxl{ranked\_db}|boldhyperpage}%
\label{ref_ranked_db}%
\hypertarget{ref_ranked_db}{}%
\subsubsection[Constructor \texttt{ranked\_db}]{Constructor \texttt{ranked\_db/ranked\_db}}%
\index[funcref]{ranked_db@\fidxl{ranked\_db}!ranked_db@\fidxl{ranked\_db}}%
\label{ref_ranked_db__ranked_db}%
\hypertarget{ref_ranked_db__ranked_db}{}%
\begin{description}
\item[Summary:]A database of distance values generated by ranking rows of orig\_db with the criterion in crit\_db.
%
\item[Usage:]~%
\begin{lyxcode}%
a\_ranked\_db = ranked\_db(data, col\_names, orig\_db, crit\_db, id, props)
%
\end{lyxcode}%
%
\item[Description:]%
This is a subclass of tests\_db. It should contain a Distance column. A
 more general ranked db class may be needed later. Use the rankMatching method
 to get an instance of this class.
%%
\item[Parameters:]~
\begin{description}%
\item[\texttt{data}:]
 Database contents.
\item[\texttt{col\_names}:]
 The column names.
\item[\texttt{orig\_db}:]
 DB whose rows are ranked.
\item[\texttt{crit\_db}:]
 The criterion DB used for generating the ranking scores.
\item[\texttt{id}:]
 An identifying string.
\item[\texttt{props}:]
 A structure with any optional properties.
\begin{description}%
\item[\texttt{tolerateNaNs}:]
 If 0, rows with any NaN values are skipped (default=1).
\end{description}%
\end{description}%
%
\item[Returns a structure object with the following fields:
]~

	tests\_db, orig\_db, crit\_db, props.
%
%
\item[See also:]%
\hyperlink{ref_tests_db}{\texttt{tests\_db}}%
\ (p.~\pageref{ref_tests_db})%
\index[funcref]{tests_db@\fidxl{tests\_db}}%
, \hyperlink{ref_tests_db__rankMatching}{\texttt{tests\_db/rankMatching}}%
\ (p.~\pageref{ref_tests_db__rankMatching})%
\index[funcref]{tests_db@\fidxl{tests\_db}!rankMatching@\fidxl{rankMatching}}%
, \hyperlink{ref_tests_db__matchingRow}{\texttt{tests\_db/matchingRow}}%
\ (p.~\pageref{ref_tests_db__matchingRow})%
\index[funcref]{tests_db@\fidxl{tests\_db}!matchingRow@\fidxl{matchingRow}}%
%
\item[Author:]%
Cengiz Gunay <cgunay@emory.edu>, 2004/12/21
%
\end{description}
\methodline%
\subsubsection[Method \texttt{blockedDistances}]{Method \texttt{ranked\_db/blockedDistances}}%
\index[funcref]{ranked_db@\fidxl{ranked\_db}!blockedDistances@\fidxl{blockedDistances}}%
\label{ref_ranked_db__blockedDistances}%
\hypertarget{ref_ranked_db__blockedDistances}{}%
\begin{description}
\item[Summary:]Creates a db of distances to blocked versions of top ranks.
%
\item[Usage:]~%
\begin{lyxcode}%
[a\_db, ranked\_dbs] = 
   blockedDistances(a\_ranked\_db, rows, blocked\_db, blocked\_param\_indices, 
	  	     block\_levels, crit\_db)
%
\end{lyxcode}%
%
%
\item[Parameters:]~
\begin{description}%
\item[\texttt{a\_ranked\_db}:]
 A ranked\_db object.
\item[\texttt{rows}:]
 Use the given row rankings.
\item[\texttt{blocked\_db}:]
 db with blocked versions of original ranks.
\item[\texttt{blocked\_param\_indices}:]
 Indices of parameters to be blocked.
\item[\texttt{block\_levels}:]
 Number of parameter levels for blocking.
\item[\texttt{crit\_db}:]
 Calculate distance from this criterion.
\end{description}%
%
\item[Returns:
]~

	a\_db: A tests\_db object with the matrix of distances.
	ranked\_dbs: A cell array of ranked\_dbs for each row.
%
\item[Example:]~
\begin{lyxcode}        >> dist\_matx\_db = blockedDistances(rankMatching(super\_db, matchingRow(rsuper\_phys\_db, 20)), 1:5, super\_blocker\_db, [1 2], 10, matchingRow(rsuper\_phys\_db, 21))
\\%
\end{lyxcode}
%
\item[See also:]%
\hyperlink{ref_makeModifiedParamDB}{\texttt{makeModifiedParamDB}}%
\ (p.~\pageref{ref_makeModifiedParamDB})%
\index[funcref]{makeModifiedParamDB@\fidxl{makeModifiedParamDB}}%
, \hyperlink{ref_getParamRowIndices}{\texttt{getParamRowIndices}}%
\ (p.~\pageref{ref_getParamRowIndices})%
\index[funcref]{getParamRowIndices@\fidxl{getParamRowIndices}}%
%
\item[Author:]%
Cengiz Gunay <cgunay@emory.edu>, 2005/01/14
%
\end{description}
\methodline%
\subsubsection[Method \texttt{displayRows}]{Method \texttt{ranked\_db/displayRows}}%
\index[funcref]{ranked_db@\fidxl{ranked\_db}!displayRows@\fidxl{displayRows}}%
\label{ref_ranked_db__displayRows}%
\hypertarget{ref_ranked_db__displayRows}{}%
\begin{description}
\item[Summary:]Displays rows of rankings together with errors associated with each measure.
%
\item[Usage:]~%
\begin{lyxcode}%
s = displayRows(db, rows, props)
%
\end{lyxcode}%
%
%
\item[Parameters:]~
\begin{description}%
\item[\texttt{db}:]
 A tests\_db object.
\item[\texttt{rows}:]
 Indices of rows in db.
\item[\texttt{props}:]
 Struct with optional properties.
\begin{description}%
\item[\texttt{originalCols}:]
 Column pattern to limit result of joinOriginal.
\end{description}%
\end{description}%
%
\item[Returns:
]~

   s: A structure of column name and value pairs.
%
%
\item[See also:]%
\hyperlink{ref_tests_db}{\texttt{tests\_db}}%
\ (p.~\pageref{ref_tests_db})%
\index[funcref]{tests_db@\fidxl{tests\_db}}%
%
\item[Author:]%
Cengiz Gunay <cgunay@emory.edu>, 2004/09/15
%
\end{description}
\methodline%
\subsubsection[Method \texttt{get}]{Method \texttt{ranked\_db/get}}%
\index[funcref]{ranked_db@\fidxl{ranked\_db}!get@\fidxl{get}}%
\label{ref_ranked_db__get}%
\hypertarget{ref_ranked_db__get}{}%
\begin{description}
\item[Summary:]Defines generic attribute retrieval for objects.
%
%
%
%
%
%
%
\item[Author:]%
Cengiz Gunay <cgunay@emory.edu>, 2004/09/14
%
\end{description}
\methodline%
\subsubsection[Method \texttt{getDistMatrix}]{Method \texttt{ranked\_db/getDistMatrix}}%
\index[funcref]{ranked_db@\fidxl{ranked\_db}!getDistMatrix@\fidxl{getDistMatrix}}%
\label{ref_ranked_db__getDistMatrix}%
\hypertarget{ref_ranked_db__getDistMatrix}{}%
\begin{description}
\item[Summary:]Create a matrix of total errors from the ranked DB.
%
\item[Usage:]~%
\begin{lyxcode}%
distmatx = getDistMatrix(db, rows, col\_size, props)
%
\end{lyxcode}%
%
\item[Description:]%
The col\_size parameter is used to find the number of rows that make up the 
 x-dimension of the matrix.
%%
\item[Parameters:]~
\begin{description}%
\item[\texttt{db}:]
 A tests\_db object.
\item[\texttt{rows}:]
 Indices of rows in db after joining (and sorting).
\item[\texttt{col\_size}:]
 Number of rows to take from DB to form the columns of matrix plot.
\item[\texttt{props}:]
 A structure with any optional properties.
\begin{description}%
\item[\texttt{sortBy}:]
 If specified, db is sorted after being joined with original using this column.
\end{description}%
\end{description}%
%
\item[Returns:
]~

	a\_plot: A plot\_abstract object.
%
%
\item[See also:]%
\hyperlink{ref_tests_db}{\texttt{tests\_db}}%
\ (p.~\pageref{ref_tests_db})%
\index[funcref]{tests_db@\fidxl{tests\_db}}%
, \hyperlink{ref_plot_abstract}{\texttt{plot\_abstract}}%
\ (p.~\pageref{ref_plot_abstract})%
\index[funcref]{plot_abstract@\fidxl{plot\_abstract}}%
%
\item[Author:]%
Cengiz Gunay <cgunay@emory.edu>, 2005/12/12
%
\end{description}
\methodline%
\subsubsection[Method \texttt{joinOriginal}]{Method \texttt{ranked\_db/joinOriginal}}%
\index[funcref]{ranked_db@\fidxl{ranked\_db}!joinOriginal@\fidxl{joinOriginal}}%
\label{ref_ranked_db__joinOriginal}%
\hypertarget{ref_ranked_db__joinOriginal}{}%
\begin{description}
\item[Summary:]Joins the distance values to the original db rows with matching row indices.
%
\item[Usage:]~%
\begin{lyxcode}%
a\_db = joinOriginal(a\_ranked\_db, rows, props)
%
\end{lyxcode}%
%
\item[Description:]%
Takes the parameter columns from orig\_db and all tests from
 crit\_db. Therefore z-score values in a\_ranked\_db are replaced with
 original metric magnitudes. Alternatively, the scores can be preserved in
 the output by specifying the keepScores prop.
%%
\item[Parameters:]~
\begin{description}%
\item[\texttt{a\_ranked\_db}:]
 A ranked\_db object.
\item[\texttt{rows}:]
 Join only the given rows.
\item[\texttt{props}:]
 A structure with any optional properties.
\begin{description}%
\item[\texttt{includeIndices}:]
 Also joins ItemIndex columns from the original db.
\item[\texttt{origCols}:]
 Also join these columns from the original DB.
\item[\texttt{rankedCols}:]
 Also join these columns from a\_ranked\_db.
\item[\texttt{keepScores}:]
 Keep tests (metrics) from a\_ranked\_db instead of original db.
\end{description}%
\end{description}%
%
\item[Returns:
]~

   a\_db: A params\_tests\_db object (same type as a\_ranked\_db.orig\_db) containing 
	the desired rows in ascending order of distance.
%
%
\item[See also:]%
\hyperlink{ref_tests_db}{\texttt{tests\_db}}%
\ (p.~\pageref{ref_tests_db})%
\index[funcref]{tests_db@\fidxl{tests\_db}}%
%
\item[Author:]%
Cengiz Gunay <cgunay@emory.edu>, 2004/12/21
%
\end{description}
\methodline%
\subsubsection[Method \texttt{plotCompareDistMatx}]{Method \texttt{ranked\_db/plotCompareDistMatx}}%
\index[funcref]{ranked_db@\fidxl{ranked\_db}!plotCompareDistMatx@\fidxl{plotCompareDistMatx}}%
\label{ref_ranked_db__plotCompareDistMatx}%
\hypertarget{ref_ranked_db__plotCompareDistMatx}{}%
\begin{description}
\item[Summary:]Compare differences and correlations of distance matrices from two ranked DBs.
%
\item[Usage:]~%
\begin{lyxcode}%
a\_plot = plotCompareDistMatx(db, rows, col\_size, col\_name, num\_col\_labels, 
			  row\_name, num\_row\_labels, title\_str, props)
%
\end{lyxcode}%
%
\item[Description:]%
Produces three plots: (1) distance difference matrix, (2) 2D cross-correlogram, 
 and (3) repeated 1D cross-correlogram for each row.
%%
\item[Parameters:]~
\begin{description}%
\item[\texttt{db, w\_db}:]
 The ranked\_db objects to be compared.
\item[\texttt{rows}:]
 Indices of rows in db after joining (and sorting) for both DBs.
\item[\texttt{col\_size}:]
 Number of rows to take from DB to form the columns of matrix plot.
\item[\texttt{col\_name, row\_name}:]
 DB column to use fot the figure column and row, respectively.
\item[\texttt{num\_col\_labels, num\_row\_labels}:]
 Number of labels to put on each axis.
\item[\texttt{title\_str}:]
 If non-empty, replaces generic title with db name. 
\item[\texttt{props}:]
 A structure with any optional properties.
\begin{description}%
\item[\texttt{sortBy}:]
 If specified, db is sorted after being joined with original using this column.
\item[\texttt{colorbar}:]
 Put a colorbar on the figure.

(also passed to plot\_abstract)
\end{description}%
\end{description}%
%
\item[Returns:
]~

	a\_plot: A plot\_abstract object.
%
%
\item[See also:]%
\hyperlink{ref_tests_db}{\texttt{tests\_db}}%
\ (p.~\pageref{ref_tests_db})%
\index[funcref]{tests_db@\fidxl{tests\_db}}%
, \hyperlink{ref_plot_abstract}{\texttt{plot\_abstract}}%
\ (p.~\pageref{ref_plot_abstract})%
\index[funcref]{plot_abstract@\fidxl{plot\_abstract}}%
%
\item[Author:]%
Cengiz Gunay <cgunay@emory.edu>, 2005/12/12
%
\end{description}
\methodline%
\subsubsection[Method \texttt{plotDistMatrix}]{Method \texttt{ranked\_db/plotDistMatrix}}%
\index[funcref]{ranked_db@\fidxl{ranked\_db}!plotDistMatrix@\fidxl{plotDistMatrix}}%
\label{ref_ranked_db__plotDistMatrix}%
\hypertarget{ref_ranked_db__plotDistMatrix}{}%
\begin{description}
\item[Summary:]Create a color-coded matrix plot of with total errors from the ranked DB.
%
\item[Usage:]~%
\begin{lyxcode}%
a\_plot = plotDistMatrix(db, rows, col\_size, col\_name, num\_col\_labels, 
			  row\_name, num\_row\_labels, title\_str, props)
%
\end{lyxcode}%
%
\item[Description:]%
The col\_size parameter is used to find the number of rows that make up the 
 x-dimension of the color matrix plot.
%%
\item[Parameters:]~
\begin{description}%
\item[\texttt{db}:]
 A ranked\_db object.
\item[\texttt{rows}:]
 Indices of rows in db after joining (and sorting).
\item[\texttt{col\_size}:]
 Number of rows to take from DB to form the columns of matrix plot.
\item[\texttt{col\_name, row\_name}:]
 DB column to use for the figure column and row, respectively.
\item[\texttt{num\_col\_labels, num\_row\_labels}:]
 Number of labels to put on each axis.
\item[\texttt{title\_str}:]
 If non-empty, replaces generic title with db name. 
\item[\texttt{props}:]
 A structure with any optional properties.
\begin{description}%
\item[\texttt{sortBy}:]
 If specified, db is sorted after being joined with original using this column.
\item[\texttt{colorbar}:]
 Put a colorbar on the figure.

(also passed to plot\_abstract)
\end{description}%
\end{description}%
%
\item[Returns:
]~

	a\_plot: A plot\_abstract object.
%
\item[Example:]~
\begin{lyxcode} >> plotFigure(plotDistMatrix(scored\_blocked\_sk\_gps0503b\_control\_db, ':', 10, 'SK', 10, 'trial', 10, 'gps0503b (control), preset 6 - top 50 matches', struct('sortBy', 'trial', 'colorbar', 1, 'PaperPosition', [0 0 5 3])));
\\%
\end{lyxcode}
%
\item[See also:]%
\hyperlink{ref_ranked_db}{\texttt{ranked\_db}}%
\ (p.~\pageref{ref_ranked_db})%
\index[funcref]{ranked_db@\fidxl{ranked\_db}}%
, \hyperlink{ref_plot_abstract}{\texttt{plot\_abstract}}%
\ (p.~\pageref{ref_plot_abstract})%
\index[funcref]{plot_abstract@\fidxl{plot\_abstract}}%
, \hyperlink{ref_getDistMatrix}{\texttt{getDistMatrix}}%
\ (p.~\pageref{ref_getDistMatrix})%
\index[funcref]{getDistMatrix@\fidxl{getDistMatrix}}%
, \hyperlink{ref_plotCompareDistMatx}{\texttt{plotCompareDistMatx}}%
\ (p.~\pageref{ref_plotCompareDistMatx})%
\index[funcref]{plotCompareDistMatx@\fidxl{plotCompareDistMatx}}%
%
\item[Author:]%
Cengiz Gunay <cgunay@emory.edu>, 2005/12/12
%
\end{description}
\methodline%
\subsubsection[Method \texttt{plotRowErrors}]{Method \texttt{ranked\_db/plotRowErrors}}%
\index[funcref]{ranked_db@\fidxl{ranked\_db}!plotRowErrors@\fidxl{plotRowErrors}}%
\label{ref_ranked_db__plotRowErrors}%
\hypertarget{ref_ranked_db__plotRowErrors}{}%
\begin{description}
\item[Summary:]Create plot of rankings with errors associated with each measure color-coded.
%
\item[Usage:]~%
\begin{lyxcode}%
a\_plot = plotRowErrors(a\_ranked\_db, rows, props)
%
\end{lyxcode}%
%
%
\item[Parameters:]~
\begin{description}%
\item[\texttt{a\_ranked\_db}:]
 A ranked\_db object.
\item[\texttt{rows}:]
 Indices of rows in a\_ranked\_db.
\item[\texttt{title\_str}:]
 (Optional) String to append to plot title.
\item[\texttt{props}:]
 A structure with any optional properties.
\begin{description}%
\item[\texttt{sortMeasures}:]
 If specified, measure order is determined with increasing 

overall distance.
\item[\texttt{RowName}:]
 Label to show on X-axis (default='Ranks')
\item[\texttt{rowSteps}:]
 Steps to jump in labeling rows on the x-axis.
\item[\texttt{superposeDistances}:]
 Superpose a white-colored distance line plot.
\item[\texttt{colorbar}:]
 Put a colorbar on the figure.

(rest passed to plot\_abstract)
\end{description}%
\end{description}%
%
\item[Returns:
]~

	a\_plot: A plot\_abstract object.
%
%
\item[See also:]%
\hyperlink{ref_ranked_db}{\texttt{ranked\_db}}%
\ (p.~\pageref{ref_ranked_db})%
\index[funcref]{ranked_db@\fidxl{ranked\_db}}%
, \hyperlink{ref_tests_db__rankMatching}{\texttt{tests\_db/rankMatching}}%
\ (p.~\pageref{ref_tests_db__rankMatching})%
\index[funcref]{tests_db@\fidxl{tests\_db}!rankMatching@\fidxl{rankMatching}}%
, \hyperlink{ref_plot_abstract}{\texttt{plot\_abstract}}%
\ (p.~\pageref{ref_plot_abstract})%
\index[funcref]{plot_abstract@\fidxl{plot\_abstract}}%
, \hyperlink{ref_plotImage}{\texttt{plotImage}}%
\ (p.~\pageref{ref_plotImage})%
\index[funcref]{plotImage@\fidxl{plotImage}}%
%
\item[Author:]%
Cengiz Gunay <cgunay@emory.edu>, 2005/12/12
%
\end{description}
\methodline%
\subsubsection[Method \texttt{renameColumns}]{Method \texttt{ranked\_db/renameColumns}}%
\index[funcref]{ranked_db@\fidxl{ranked\_db}!renameColumns@\fidxl{renameColumns}}%
\label{ref_ranked_db__renameColumns}%
\hypertarget{ref_ranked_db__renameColumns}{}%
\begin{description}
\item[Summary:]Rename an existing column or columns.
%
\item[Usage:]~%
\begin{lyxcode}%
a\_db = renameColumns(a\_db, test\_names, new\_names)
%
\end{lyxcode}%
%
\item[Description:]%
This method is an overloaded method for ranked\_db that keeps consistent
 the column names of the ranked, criterion and original DBs. The other
 DBs are not renamed for the Distance and RowIndex columns.
%%
\item[Parameters:]~
\begin{description}%
\item[\texttt{a\_db}:]
 A ranked\_db object.
\item[\texttt{test\_names}:]
 A cell array of existing test names.
\item[\texttt{new\_names}:]
 New names to replace existing ones.
\end{description}%
%
\item[Returns:
]~

	a\_db: The ranked\_db object that includes the new columns.
%
%
\item[See also:]%
\hyperlink{ref_tests_db__renameColumns}{\texttt{tests\_db/renameColumns}}%
\ (p.~\pageref{ref_tests_db__renameColumns})%
\index[funcref]{tests_db@\fidxl{tests\_db}!renameColumns@\fidxl{renameColumns}}%
%
\item[Author:]%
Cengiz Gunay <cgunay@emory.edu>, 2006/06/07
%
\end{description}
\methodline%
\subsubsection[Method \texttt{set}]{Method \texttt{ranked\_db/set}}%
\index[funcref]{ranked_db@\fidxl{ranked\_db}!set@\fidxl{set}}%
\label{ref_ranked_db__set}%
\hypertarget{ref_ranked_db__set}{}%
\begin{description}
\item[Summary:]Generic method for setting object attributes.
%
%
%
%
%
%
%
\item[Author:]%
Cengiz Gunay <cgunay@emory.edu>, 2004/10/08
%
\end{description}
\methodline%
\subsubsection[Method \texttt{subsref}]{Method \texttt{ranked\_db/subsref}}%
\index[funcref]{ranked_db@\fidxl{ranked\_db}!subsref@\fidxl{subsref}}%
\label{ref_ranked_db__subsref}%
\hypertarget{ref_ranked_db__subsref}{}%
\begin{description}
\item[Summary:]Defines generic indexing for objects.
%
%
%
%
%
%
%
%
\end{description}
\methodline%
\subsection{Class \texttt{results\_profile}}%
\index[funcref]{results_profile@\fidxl{results\_profile}|boldhyperpage}%
\label{ref_results_profile}%
\hypertarget{ref_results_profile}{}%
\subsubsection[Constructor \texttt{results\_profile}]{Constructor \texttt{results\_profile/results\_profile}}%
\index[funcref]{results_profile@\fidxl{results\_profile}!results_profile@\fidxl{results\_profile}}%
\label{ref_results_profile__results_profile}%
\hypertarget{ref_results_profile__results_profile}{}%
\begin{description}
\item[Summary:]Creates and collects result profiles for data objects.
%
\item[Usage:]~%
\begin{lyxcode}%
obj = results\_profile(results, id, props)
%
\end{lyxcode}%
%
\item[Description:]%
This is the base class for all profile classes.
%%
\item[Parameters:]~
\begin{description}%
\item[\texttt{results}:]
 A structure containing test results.
\item[\texttt{id}:]
 Identification string.
\item[\texttt{props}:]
 A structure with any optional properties.
\end{description}%
%
\item[Returns a structure object with the following fields:
]~

	results, id, props.
%
%
\item[See also:]%
\hyperlink{ref_trace_profile}{\texttt{trace\_profile}}%
\ (p.~\pageref{ref_trace_profile})%
\index[funcref]{trace_profile@\fidxl{trace\_profile}}%
, \hyperlink{ref_cip_trace_profile}{\texttt{cip\_trace\_profile}}%
\ (p.~\pageref{ref_cip_trace_profile})%
\index[funcref]{cip_trace_profile@\fidxl{cip\_trace\_profile}}%
%
\item[Author:]%
Cengiz Gunay <cgunay@emory.edu>, 2004/09/14
%
\end{description}
\methodline%
\subsubsection[Method \texttt{display}]{Method \texttt{results\_profile/display}}%
\index[funcref]{results_profile@\fidxl{results\_profile}!display@\fidxl{display}}%
\label{ref_results_profile__display}%
\hypertarget{ref_results_profile__display}{}%
\begin{description}
%
%
%
%
%
%
%
\item[Author:]%
Cengiz Gunay <cgunay@emory.edu>, 2004/08/04
%
\end{description}
\methodline%
\subsubsection[Method \texttt{get}]{Method \texttt{results\_profile/get}}%
\index[funcref]{results_profile@\fidxl{results\_profile}!get@\fidxl{get}}%
\label{ref_results_profile__get}%
\hypertarget{ref_results_profile__get}{}%
\begin{description}
\item[Summary:]Defines generic attribute retrieval for objects.
%
%
%
%
%
%
%
\item[Author:]%
Cengiz Gunay <cgunay@emory.edu>, 2004/09/14
%
\end{description}
\methodline%
\subsubsection[Method \texttt{getResults}]{Method \texttt{results\_profile/getResults}}%
\index[funcref]{results_profile@\fidxl{results\_profile}!getResults@\fidxl{getResults}}%
\label{ref_results_profile__getResults}%
\hypertarget{ref_results_profile__getResults}{}%
\begin{description}
\item[Summary:]Return the results profile structure.
%
\item[Usage:]~%
\begin{lyxcode}%
results = getResults(p)
%
\end{lyxcode}%
%
%
\item[Parameters:]~
\begin{description}%
\item[\texttt{p}:]
 A result\_profile object.
\end{description}%
%
\item[Returns:
]~

	results: A structure associating test names to values.
%
%
\item[See also:]%
\hyperlink{ref_results_profile}{\texttt{results\_profile}}%
\ (p.~\pageref{ref_results_profile})%
\index[funcref]{results_profile@\fidxl{results\_profile}}%
%
\item[Author:]%
Cengiz Gunay <cgunay@emory.edu>, 2004/09/14
%
\end{description}
\methodline%
\subsubsection[Method \texttt{plot}]{Method \texttt{results\_profile/plot}}%
\index[funcref]{results_profile@\fidxl{results\_profile}!plot@\fidxl{plot}}%
\label{ref_results_profile__plot}%
\hypertarget{ref_results_profile__plot}{}%
\begin{description}
\item[Summary:]Generic method to plot a tests\_db or a subclass. Requires a 
	plot\_abstract method to be defined for this object.
%
\item[Usage:]~%
\begin{lyxcode}%
h = plot(a\_tests\_db, title\_str, props)
%
\end{lyxcode}%
%
%
\item[Parameters:]~
\begin{description}%
\item[\texttt{a\_tests\_db}:]
 A histogram\_db object.
\item[\texttt{title\_str}:]
 (Optional) String to append to plot title.
\item[\texttt{props}:]
 Optional properties passed to plot\_abstract.
\end{description}%
%
\item[Returns:
]~

	h: The figure handle created.
%
%
\item[See also:]%
\hyperlink{ref_plot_abstract}{\texttt{plot\_abstract}}%
\ (p.~\pageref{ref_plot_abstract})%
\index[funcref]{plot_abstract@\fidxl{plot\_abstract}}%
, \hyperlink{ref_plotFigure}{\texttt{plotFigure}}%
\ (p.~\pageref{ref_plotFigure})%
\index[funcref]{plotFigure@\fidxl{plotFigure}}%
%
\item[Author:]%
Cengiz Gunay <cgunay@emory.edu>, 2004/10/06
%
\end{description}
\methodline%
\subsubsection[Method \texttt{subsref}]{Method \texttt{results\_profile/subsref}}%
\index[funcref]{results_profile@\fidxl{results\_profile}!subsref@\fidxl{subsref}}%
\label{ref_results_profile__subsref}%
\hypertarget{ref_results_profile__subsref}{}%
\begin{description}
\item[Summary:]Defines generic indexing for objects.
%
%
%
%
%
%
%
%
\end{description}
\methodline%
\subsection{Class \texttt{script\_array}}%
\index[funcref]{script_array@\fidxl{script\_array}|boldhyperpage}%
\label{ref_script_array}%
\hypertarget{ref_script_array}{}%
\subsubsection[Constructor \texttt{script\_array}]{Constructor \texttt{script\_array/script\_array}}%
\index[funcref]{script_array@\fidxl{script\_array}!script_array@\fidxl{script\_array}}%
\label{ref_script_array__script_array}%
\hypertarget{ref_script_array__script_array}{}%
\begin{description}
\item[Summary:]Generic class that provides the scripts for a repetitive array job.
%
\item[Usage:]~%
\begin{lyxcode}%
obj = script\_array(num\_runs, id, props)
%
\end{lyxcode}%
%
\item[Description:]%
This is the base class for all script\_array classes. Runs the runJob method as 
 num\_runs many times. Run initiated by calling runFirst, but the final
 result will be returned by runLast.
%%
\item[Parameters:]~
\begin{description}%
\item[\texttt{num\_runs}:]
 The number of times the runJob script should be evoked.
\item[\texttt{id}:]
 Identification string.
\item[\texttt{props}:]
 A structure with any optional properties.
\begin{description}%
\item[\texttt{runJobFunc}:]
 A function name or handle to be used instead of default

runJob.
\item[\texttt{parallel}:]
 If 1, run jobs in parallel using parfor.
\end{description}%
\end{description}%
%
\item[Returns a structure object with the following fields:
]~

	num\_runs, id, props.
%
\item[Example:]~
\begin{lyxcode} >> func1 = inline('x\textasciicircum{}2')
\\%
 >> runFirst(script\_array(10, 'squares numbers up to 10'), struct('runJobFunc', func1))
\\%
 ans = [  1]    [  4]    [  9]    [ 16]    [ 25]    [ 36]    [ 49]    [ 64]    [ 81]    [100]
\\%
\end{lyxcode}
%
\item[See also:]%
\hyperlink{ref_runFirst}{\texttt{runFirst}}%
\ (p.~\pageref{ref_runFirst})%
\index[funcref]{runFirst@\fidxl{runFirst}}%
, \hyperlink{ref_runLast}{\texttt{runLast}}%
\ (p.~\pageref{ref_runLast})%
\index[funcref]{runLast@\fidxl{runLast}}%
, \hyperlink{ref_runJob}{\texttt{runJob}}%
\ (p.~\pageref{ref_runJob})%
\index[funcref]{runJob@\fidxl{runJob}}%
, \hyperlink{ref_parfor}{\texttt{parfor}}%
\ (p.~\pageref{ref_parfor})%
\index[funcref]{parfor@\fidxl{parfor}}%
%
\item[Author:]%
Cengiz Gunay <cgunay@emory.edu>, 2006/02/01
%
\end{description}
\methodline%
\subsubsection[Method \texttt{get}]{Method \texttt{script\_array/get}}%
\index[funcref]{script_array@\fidxl{script\_array}!get@\fidxl{get}}%
\label{ref_script_array__get}%
\hypertarget{ref_script_array__get}{}%
\begin{description}
\item[Summary:]Defines generic attribute retrieval for objects.
%
%
%
%
%
%
%
\item[Author:]%
Cengiz Gunay <cgunay@emory.edu>, 2004/09/14
%
\end{description}
\methodline%
\subsubsection[Method \texttt{runFirst}]{Method \texttt{script\_array/runFirst}}%
\index[funcref]{script_array@\fidxl{script\_array}!runFirst@\fidxl{runFirst}}%
\label{ref_script_array__runFirst}%
\hypertarget{ref_script_array__runFirst}{}%
\begin{description}
\item[Summary:]Method to be called at beginning of script\_array jobs.
%
\item[Usage:]~%
\begin{lyxcode}%
job\_results = runFirst(a\_script\_array)
%
\end{lyxcode}%
%
\item[Description:]%
This method initiates the script\_array jobs. It loops and calls runJob and 
 finally calls runLast.
%%
\item[Parameters:]~
\begin{description}%
\item[\texttt{a\_script\_array}:]
 A script\_array object.
\end{description}%
%
\item[Returns:
]~

	job\_results: A cell array of results collected from each item of the vector jobs.
%
\item[Example:]~
\begin{lyxcode} >> results = runFirst(script\_array(10, 'this one does nothing for 10 times'));
\\%
\end{lyxcode}
%
\item[See also:]%
\hyperlink{ref_runLast}{\texttt{runLast}}%
\ (p.~\pageref{ref_runLast})%
\index[funcref]{runLast@\fidxl{runLast}}%
, \hyperlink{ref_runJob}{\texttt{runJob}}%
\ (p.~\pageref{ref_runJob})%
\index[funcref]{runJob@\fidxl{runJob}}%
%
\item[Author:]%
Cengiz Gunay <cgunay@emory.edu>, 2006/02/01
%
\end{description}
\methodline%
\subsubsection[Method \texttt{runJob}]{Method \texttt{script\_array/runJob}}%
\index[funcref]{script_array@\fidxl{script\_array}!runJob@\fidxl{runJob}}%
\label{ref_script_array__runJob}%
\hypertarget{ref_script_array__runJob}{}%
\begin{description}
\item[Summary:]Method to be called for each of the script\_array jobs.
%
\item[Usage:]~%
\begin{lyxcode}%
job\_result = runJob(a\_script\_array, vector\_index)
%
\end{lyxcode}%
%
\item[Description:]%
This method is provided as a placeholder and does nothing. If the run\_job\_func
 property is defined, it will call that function.
%%
\item[Parameters:]~
\begin{description}%
\item[\texttt{a\_script\_array}:]
 A script\_array object.
\item[\texttt{vector\_index}:]
 The index within the vector job.
\end{description}%
%
\item[Returns:
]~

   job\_result: Any output produced by the job.
%
\item[Example:]~
\begin{lyxcode} See real example in script\_array. Call the 5th job:
\\%
 >> runJob(script\_array(10, 'this one does nothing for 10 times'), 5);
\\%
\end{lyxcode}
%
\item[See also:]%
\hyperlink{ref_runLast}{\texttt{runLast}}%
\ (p.~\pageref{ref_runLast})%
\index[funcref]{runLast@\fidxl{runLast}}%
, \hyperlink{ref_runFirst}{\texttt{runFirst}}%
\ (p.~\pageref{ref_runFirst})%
\index[funcref]{runFirst@\fidxl{runFirst}}%
%
\item[Author:]%
Cengiz Gunay <cgunay@emory.edu>, 2006/02/01
%
\end{description}
\methodline%
\subsubsection[Method \texttt{runLast}]{Method \texttt{script\_array/runLast}}%
\index[funcref]{script_array@\fidxl{script\_array}!runLast@\fidxl{runLast}}%
\label{ref_script_array__runLast}%
\hypertarget{ref_script_array__runLast}{}%
\begin{description}
\item[Summary:]Method to be called last after the script\_array jobs.
%
\item[Usage:]~%
\begin{lyxcode}%
job\_results = runLast(a\_script\_array, job\_results)
%
\end{lyxcode}%
%
\item[Description:]%
This method is provided as a placeholder and does nothing. It can filter-out the
 results returned from the jobs run. Normally it is invoked internally by the runFirst
 method, after running and collecting results from the vector jobs with the runJob method.
%%
\item[Parameters:]~
\begin{description}%
\item[\texttt{a\_script\_array}:]
 A script\_array object.
\item[\texttt{job\_results}:]
 The index within the vector job.
\end{description}%
%
\item[Returns:
]~

   job\_results: Any output produced by the job.
%
\item[Example:]~
\begin{lyxcode} Call it directly:
\\%
 >> runLast(script\_array(10, 'this one does nothing for 10 times'), {});
\\%
\end{lyxcode}
%
\item[See also:]%
\hyperlink{ref_runJob}{\texttt{runJob}}%
\ (p.~\pageref{ref_runJob})%
\index[funcref]{runJob@\fidxl{runJob}}%
, \hyperlink{ref_runFirst}{\texttt{runFirst}}%
\ (p.~\pageref{ref_runFirst})%
\index[funcref]{runFirst@\fidxl{runFirst}}%
%
\item[Author:]%
Cengiz Gunay <cgunay@emory.edu>, 2006/02/01
%
\end{description}
\methodline%
\subsubsection[Method \texttt{set}]{Method \texttt{script\_array/set}}%
\index[funcref]{script_array@\fidxl{script\_array}!set@\fidxl{set}}%
\label{ref_script_array__set}%
\hypertarget{ref_script_array__set}{}%
\begin{description}
\item[Summary:]Generic method for setting object attributes.
%
%
%
%
%
%
%
\item[Author:]%
Cengiz Gunay <cgunay@emory.edu>, 2006/02/06
%
\end{description}
\methodline%
\subsubsection[Method \texttt{subsasgn}]{Method \texttt{script\_array/subsasgn}}%
\index[funcref]{script_array@\fidxl{script\_array}!subsasgn@\fidxl{subsasgn}}%
\label{ref_script_array__subsasgn}%
\hypertarget{ref_script_array__subsasgn}{}%
\begin{description}
\item[Summary:]Defines generic index-based assignment for objects.
%
%
%
%
%
%
%
\item[Author:]%
Cengiz Gunay <cgunay@emory.edu>, 2006/02/06
%
\end{description}
\methodline%
\subsubsection[Method \texttt{subsref}]{Method \texttt{script\_array/subsref}}%
\index[funcref]{script_array@\fidxl{script\_array}!subsref@\fidxl{subsref}}%
\label{ref_script_array__subsref}%
\hypertarget{ref_script_array__subsref}{}%
\begin{description}
\item[Summary:]Defines generic indexing for objects.
%
%
%
%
%
%
%
\item[Author:]%
Cengiz Gunay <cgunay@emory.edu>, 2004/08/04
%
\end{description}
\methodline%
\subsection{Class \texttt{script\_array\_for\_cluster}}%
\index[funcref]{script_array_for_cluster@\fidxl{script\_array\_for\_cluster}|boldhyperpage}%
\label{ref_script_array_for_cluster}%
\hypertarget{ref_script_array_for_cluster}{}%
\subsubsection[Constructor \texttt{script\_array\_for\_cluster}]{Constructor \texttt{script\_array\_for\_cluster/script\_array\_for\_cluster}}%
\index[funcref]{script_array_for_cluster@\fidxl{script\_array\_for\_cluster}!script_array_for_cluster@\fidxl{script\_array\_for\_cluster}}%
\label{ref_script_array_for_cluster__script_array_for_cluster}%
\hypertarget{ref_script_array_for_cluster__script_array_for_cluster}{}%
\begin{description}
\item[Summary:]Generic class defining a repetitive vector job to be run on a Sun Grid Engine (SGE) computing cluster.
%
\item[Usage:]~%
\begin{lyxcode}%
a\_script\_cluster = script\_array\_for\_cluster(num\_runs, sge\_wrapper\_script, id, props)
%
\end{lyxcode}%
%
\item[Description:]%
This is a subclass of the script\_array class. The runFirst method spawns num\_runs
 copies of the runJob method in parallel on the cluster, followed by the invocation 
 of the runLast method.
%%
\item[Parameters:]~
\begin{description}%
\item[\texttt{num\_runs}:]
 The number of times the runJob script should be evoked.
\item[\texttt{sge\_wrapper\_script}:]
 A script that can be submitted with qsub and can execute arbitrary

Matlab commands on the cluster nodes. It can have qsub options prepended to it
such as '-p -100 -q all.q <abs\_path\_to>/sge\_matlab.sh'.
\item[\texttt{id}:]
 Identification string.
\item[\texttt{props}:]
 A structure with any optional properties.
\begin{description}%
\item[\texttt{notifyByMail}:]
 An SGE notification email is sent to this address after lastJob.

(others passed to script\_array)
\end{description}%
\end{description}%
%
\item[Returns a structure object with the following fields:
]~

	num\_runs, id, props.
%
%
\item[See also:]%
\hyperlink{ref_runFirst}{\texttt{runFirst}}%
\ (p.~\pageref{ref_runFirst})%
\index[funcref]{runFirst@\fidxl{runFirst}}%
, \hyperlink{ref_runLast}{\texttt{runLast}}%
\ (p.~\pageref{ref_runLast})%
\index[funcref]{runLast@\fidxl{runLast}}%
, \hyperlink{ref_runJob}{\texttt{runJob}}%
\ (p.~\pageref{ref_runJob})%
\index[funcref]{runJob@\fidxl{runJob}}%
, \hyperlink{ref_script_array}{\texttt{script\_array}}%
\ (p.~\pageref{ref_script_array})%
\index[funcref]{script_array@\fidxl{script\_array}}%
%
\item[Author:]%
Cengiz Gunay <cgunay@emory.edu>, 2006/02/02
%
\end{description}
\methodline%
\subsubsection[Method \texttt{get}]{Method \texttt{script\_array\_for\_cluster/get}}%
\index[funcref]{script_array_for_cluster@\fidxl{script\_array\_for\_cluster}!get@\fidxl{get}}%
\label{ref_script_array_for_cluster__get}%
\hypertarget{ref_script_array_for_cluster__get}{}%
\begin{description}
\item[Summary:]Defines generic attribute retrieval for objects.
%
%
%
%
%
%
%
\item[Author:]%
Cengiz Gunay <cgunay@emory.edu>, 2004/09/14
%
\end{description}
\methodline%
\subsubsection[Method \texttt{runFirst}]{Method \texttt{script\_array\_for\_cluster/runFirst}}%
\index[funcref]{script_array_for_cluster@\fidxl{script\_array\_for\_cluster}!runFirst@\fidxl{runFirst}}%
\label{ref_script_array_for_cluster__runFirst}%
\hypertarget{ref_script_array_for_cluster__runFirst}{}%
\begin{description}
\item[Summary:]Method to be called at beginning of script\_array\_for\_cluster jobs.
%
\item[Usage:]~%
\begin{lyxcode}%
job\_results = runFirst(a\_script\_cluster)
%
\end{lyxcode}%
%
\item[Description:]%
This method initiates the script\_array\_for\_cluster jobs. It submits an SGE
 vector job for running each runJob and finally runLast. There is no way of
 collecting outputs from individual runJob calls.
%%
\item[Parameters:]~
\begin{description}%
\item[\texttt{a\_script\_cluster}:]
 A script\_array\_for\_cluster object.
\end{description}%
%
\item[Returns:
]~

	job\_results: A cell array of results collected from each item of the vector jobs.
%
\item[Example:]~
\begin{lyxcode} >> runFirst(script\_array\_for\_cluster(10, 'this one does nothing for 10 times'));
\\%
\end{lyxcode}
%
\item[See also:]%
\hyperlink{ref_script_array_for_cluster}{\texttt{script\_array\_for\_cluster}}%
\ (p.~\pageref{ref_script_array_for_cluster})%
\index[funcref]{script_array_for_cluster@\fidxl{script\_array\_for\_cluster}}%
%
\item[Author:]%
Cengiz Gunay <cgunay@emory.edu>, 2006/02/01
%
\end{description}
\methodline%
\subsubsection[Method \texttt{set}]{Method \texttt{script\_array\_for\_cluster/set}}%
\index[funcref]{script_array_for_cluster@\fidxl{script\_array\_for\_cluster}!set@\fidxl{set}}%
\label{ref_script_array_for_cluster__set}%
\hypertarget{ref_script_array_for_cluster__set}{}%
\begin{description}
\item[Summary:]Generic method for setting object attributes.
%
%
%
%
%
%
%
\item[Author:]%
Cengiz Gunay <cgunay@emory.edu>, 2006/02/06
%
\end{description}
\methodline%
\subsection{Class \texttt{script\_array\_loaddb}}%
\index[funcref]{script_array_loaddb@\fidxl{script\_array\_loaddb}|boldhyperpage}%
\label{ref_script_array_loaddb}%
\hypertarget{ref_script_array_loaddb}{}%
\subsubsection[Constructor \texttt{script\_array\_loaddb}]{Constructor \texttt{script\_array\_loaddb/script\_array\_loaddb}}%
\index[funcref]{script_array_loaddb@\fidxl{script\_array\_loaddb}!script_array_loaddb@\fidxl{script\_array\_loaddb}}%
\label{ref_script_array_loaddb__script_array_loaddb}%
\hypertarget{ref_script_array_loaddb__script_array_loaddb}{}%
\begin{description}
\item[Summary:]Loads dataset items into a database by partitioning into parallel tasks.
%
\item[Usage:]~%
\begin{lyxcode}%
a\_s = script\_array\_loaddb(num\_runs, a\_dataset, id, props)
%
\end{lyxcode}%
%
\item[Description:]%
This is a subclass of the script\_array class. It will analyze and load
 items in a params\_tests\_dataset (and subclass) objects by partitioning it
 into num\_runs pieces, to be loaded in parallel on multicore
 machines. Partitioned loading of databases can also be beneficial in
 serial (by setting prop 'parallel' to 0) for large datasets, such that
 problematic items (that crash) do not hinder the loading of the rest of
 the dataset.
%%
\item[Parameters:]~
\begin{description}%
\item[\texttt{num\_runs}:]
 The number of times the runJob script should be evoked.
\item[\texttt{sge\_wrapper\_script}:]
 A script that can be submitted with qsub and can execute arbitrary

Matlab commands on the cluster nodes. It can have qsub options prepended to it
such as '-p -100 -q all.q <abs\_path\_to>/sge\_matlab.sh'.
\item[\texttt{id}:]
 Identification string.
\item[\texttt{props}:]
 A structure with any optional properties.
\begin{description}%
\item[\texttt{items}:]
 If specified, only load items in this horizontal vector.

(others passed to script\_array)
\end{description}%
\end{description}%
%
\item[Returns a structure object with the following fields:
]~

	dataset, script\_array.
%
%
\item[See also:]%
\hyperlink{ref_runFirst}{\texttt{runFirst}}%
\ (p.~\pageref{ref_runFirst})%
\index[funcref]{runFirst@\fidxl{runFirst}}%
, \hyperlink{ref_runLast}{\texttt{runLast}}%
\ (p.~\pageref{ref_runLast})%
\index[funcref]{runLast@\fidxl{runLast}}%
, \hyperlink{ref_runJob}{\texttt{runJob}}%
\ (p.~\pageref{ref_runJob})%
\index[funcref]{runJob@\fidxl{runJob}}%
, \hyperlink{ref_script_array}{\texttt{script\_array}}%
\ (p.~\pageref{ref_script_array})%
\index[funcref]{script_array@\fidxl{script\_array}}%
%
\item[Author:]%
Cengiz Gunay <cgunay@emory.edu>, 2014/04/02
%
\end{description}
\methodline%
\subsubsection[Method \texttt{runJob}]{Method \texttt{script\_array\_loaddb/runJob}}%
\index[funcref]{script_array_loaddb@\fidxl{script\_array\_loaddb}!runJob@\fidxl{runJob}}%
\label{ref_script_array_loaddb__runJob}%
\hypertarget{ref_script_array_loaddb__runJob}{}%
\begin{description}
\item[Summary:]Loads one piece of the database.
%
\item[Usage:]~%
\begin{lyxcode}%
a\_db\_piece = runJob(a\_s, vector\_index)
%
\end{lyxcode}%
%
\item[Description:]%
Load the part of the database calculated by the index. Uses
 the params\_tests\_dataset/params\_tests\_db method to load the database.
%%
\item[Parameters:]~
\begin{description}%
\item[\texttt{a\_s}:]
 A script\_array\_loaddb object.
\item[\texttt{vector\_index}:]
 The index within the vector job.
\end{description}%
%
\item[Returns:
]~

   a\_db\_piece: Piece of loaded database.
%
%
\item[See also:]%
\hyperlink{ref_script_array_loaddb}{\texttt{script\_array\_loaddb}}%
\ (p.~\pageref{ref_script_array_loaddb})%
\index[funcref]{script_array_loaddb@\fidxl{script\_array\_loaddb}}%
, \hyperlink{ref_runLast}{\texttt{runLast}}%
\ (p.~\pageref{ref_runLast})%
\index[funcref]{runLast@\fidxl{runLast}}%
, \hyperlink{ref_runFirst}{\texttt{runFirst}}%
\ (p.~\pageref{ref_runFirst})%
\index[funcref]{runFirst@\fidxl{runFirst}}%
%
\item[Author:]%
Cengiz Gunay <cgunay@emory.edu>, 2014/04/02
%
\end{description}
\methodline%
\subsubsection[Method \texttt{runLast}]{Method \texttt{script\_array\_loaddb/runLast}}%
\index[funcref]{script_array_loaddb@\fidxl{script\_array\_loaddb}!runLast@\fidxl{runLast}}%
\label{ref_script_array_loaddb__runLast}%
\hypertarget{ref_script_array_loaddb__runLast}{}%
\begin{description}
\item[Summary:]Method to be called last after the script\_array jobs.
%
\item[Usage:]~%
\begin{lyxcode}%
a\_db = runLast(a\_script\_array, job\_results)
%
\end{lyxcode}%
%
\item[Description:]%
Combines the separately loaded databases found in job\_results into a
 final one and returns it.
%%
\item[Parameters:]~
\begin{description}%
\item[\texttt{a\_script\_array}:]
 A script\_array object.
\item[\texttt{job\_results}:]
 The index within the vector job.
\end{description}%
%
\item[Returns:
]~

   a\_db: Concatenated database.
%
%
\item[See also:]%
\hyperlink{ref_script_array_loaddb}{\texttt{script\_array\_loaddb}}%
\ (p.~\pageref{ref_script_array_loaddb})%
\index[funcref]{script_array_loaddb@\fidxl{script\_array\_loaddb}}%
, \hyperlink{ref_runJob}{\texttt{runJob}}%
\ (p.~\pageref{ref_runJob})%
\index[funcref]{runJob@\fidxl{runJob}}%
, \hyperlink{ref_runFirst}{\texttt{runFirst}}%
\ (p.~\pageref{ref_runFirst})%
\index[funcref]{runFirst@\fidxl{runFirst}}%
%
\item[Author:]%
Cengiz Gunay <cgunay@emory.edu>, 2014/04/02
%
\end{description}
\methodline%
\subsection{Class \texttt{script\_factory}}%
\index[funcref]{script_factory@\fidxl{script\_factory}|boldhyperpage}%
\label{ref_script_factory}%
\hypertarget{ref_script_factory}{}%
\subsubsection[Constructor \texttt{script\_factory}]{Constructor \texttt{script\_factory/script\_factory}}%
\index[funcref]{script_factory@\fidxl{script\_factory}!script_factory@\fidxl{script\_factory}}%
\label{ref_script_factory__script_factory}%
\hypertarget{ref_script_factory__script_factory}{}%
\begin{description}
\item[Summary:]Generic class to automatically create a set of scripts.
%
\item[Usage:]~%
\begin{lyxcode}%
obj = script\_factory(num\_scripts, out\_name, id, props)
%
\end{lyxcode}%
%
\item[Description:]%
This is the base class for all script\_factory classes.
%%
\item[Parameters:]~
\begin{description}%
\item[\texttt{num\_scripts}:]
 Number of scripts to create.
\item[\texttt{out\_name}:]
 The file name for the output scripts. A '%d' in the

filename corresponds to the script number.
\item[\texttt{id}:]
 Identification string.
\item[\texttt{props}:]
 A structure with any optional properties.
\end{description}%
%
\item[Returns a structure object with the following fields:
]~

	num\_scripts, out\_name, id, props.
%
%
\item[See also:]%
\hyperlink{ref_script_factory__writeScripts}{\texttt{script\_factory/writeScripts}}%
\ (p.~\pageref{ref_script_factory__writeScripts})%
\index[funcref]{script_factory@\fidxl{script\_factory}!writeScripts@\fidxl{writeScripts}}%
%
\item[Author:]%
Cengiz Gunay <cgunay@emory.edu>, 2005/11/28
%
\end{description}
\methodline%
\subsubsection[Method \texttt{get}]{Method \texttt{script\_factory/get}}%
\index[funcref]{script_factory@\fidxl{script\_factory}!get@\fidxl{get}}%
\label{ref_script_factory__get}%
\hypertarget{ref_script_factory__get}{}%
\begin{description}
\item[Summary:]Defines generic attribute retrieval for objects.
%
%
%
%
%
%
%
\item[Author:]%
Cengiz Gunay <cgunay@emory.edu>, 2004/09/14
%
\end{description}
\methodline%
\subsection{Class \texttt{spike\_shape}}%
\index[funcref]{spike_shape@\fidxl{spike\_shape}|boldhyperpage}%
\label{ref_spike_shape}%
\hypertarget{ref_spike_shape}{}%
\subsubsection[Constructor \texttt{spike\_shape}]{Constructor \texttt{spike\_shape/spike\_shape}}%
\index[funcref]{spike_shape@\fidxl{spike\_shape}!spike_shape@\fidxl{spike\_shape}}%
\label{ref_spike_shape__spike_shape}%
\hypertarget{ref_spike_shape__spike_shape}{}%
\begin{description}
\item[Summary:]An action potential shape trace.
%
\item[Usage:]~%
\begin{lyxcode}%
obj = spike\_shape(data, dt, dy, id)
%
\end{lyxcode}%
%
%
\item[Parameters:]~
\begin{description}%
\item[\texttt{data}:]
 A vector of data points containing the spike shape.
\item[\texttt{dt}:]
 Time resolution [s].
\item[\texttt{dy}:]
 y-axis resolution [ISI (V, A, etc.)]
\item[\texttt{id}:]
 Identification string.
\item[\texttt{props}:]
 A structure with any optional properties.
\begin{description}%
\item[\texttt{baseline}:]
 Resting potential.
\item[\texttt{threshold}:]
 Spike threshold.
\item[\texttt{init\_Vm\_method}:]
 Method to obtain spike initiation voltage.

1- maximum acceleration point
2- threshold crossing of acceleration (needs threshold)
3- threshold crossing of slope (needs threshold)
4- maximum acceleration in phase space
(optionally specify maximal threshold as init\_threshold)
5- point of maximum curvature, when slope is between 
init\_lo\_thr and init\_hi\_thr
6- local maximum of second derivative in the phase space
nearest slope crossing init\_threshold
7- threshold crossing of interpolated slope (needs threshold)
8- maximum curvature in phase-plane
9- Combined curvature and inflection method in time-domain.
\item[\texttt{init\_threshold}:]
 Spike initiation threshold (deriv or accel).

(see above methods and implementation in calcInitVm)
\item[\texttt{init\_lo\_thr, init\_hi\_thr}:]
 Low and high thresholds for slope.
\end{description}%
\end{description}%
%
\item[Returns a structure object with the following fields:
]~

	trace, props.
%
%
\item[See also:]%
\hyperlink{ref_trace__spike_shape}{\texttt{trace/spike\_shape}}%
\ (p.~\pageref{ref_trace__spike_shape})%
\index[funcref]{trace@\fidxl{trace}!spike_shape@\fidxl{spike\_shape}}%
, \hyperlink{ref_trace__analyzeSpikesInPeriod}{\texttt{trace/analyzeSpikesInPeriod}}%
\ (p.~\pageref{ref_trace__analyzeSpikesInPeriod})%
\index[funcref]{trace@\fidxl{trace}!analyzeSpikesInPeriod@\fidxl{analyzeSpikesInPeriod}}%
, \hyperlink{ref_trace}{\texttt{trace}}%
\ (p.~\pageref{ref_trace})%
\index[funcref]{trace@\fidxl{trace}}%
, \hyperlink{ref_spikes}{\texttt{spikes}}%
\ (p.~\pageref{ref_spikes})%
\index[funcref]{spikes@\fidxl{spikes}}%
, \hyperlink{ref_period}{\texttt{period}}%
\ (p.~\pageref{ref_period})%
\index[funcref]{period@\fidxl{period}}%
%
\item[Author:]%
Cengiz Gunay <cgunay@emory.edu>, 2004/07/30
%
\end{description}
\methodline%
\subsubsection[Method \texttt{calcInitVm}]{Method \texttt{spike\_shape/calcInitVm}}%
\index[funcref]{spike_shape@\fidxl{spike\_shape}!calcInitVm@\fidxl{calcInitVm}}%
\label{ref_spike_shape__calcInitVm}%
\hypertarget{ref_spike_shape__calcInitVm}{}%
\begin{description}
\item[Summary:]Calculates spike threshold related measures of the spike\_shape, s. 
%
\item[Usage:]~%
\begin{lyxcode}%
[init\_val, init\_idx, rise\_time, amplitude,
  peak\_mag, peak\_idx, max\_d1o, a\_plot] = 
	calcInitVm(s, max\_idx, min\_idx)
%
\end{lyxcode}%
%
%
\item[Parameters:]~
\begin{description}%
\item[\texttt{s}:]
 A spike\_shape object.
\item[\texttt{max\_idx}:]
 The index of the maximal point of the spike\_shape [dt].
\item[\texttt{min\_idx}:]
 The index of the minimal point of the spike\_shape [dt].
\item[\texttt{plotit}:]
 If non-zero, plot a graph annotating the test results 

(optional).
\end{description}%
%
\item[Returns:
]~

	init\_val: The potential value [dy].
	init\_idx: Its index in the spike\_shape [dt].
	rise\_time: Time from initiation to maximum [dt].
	amplitude: Magnitude from initiation to max [dy].
	peak\_mag: Peak value [dy].
	peak\_idx: Extrapolated spike peak index [dt].
	max\_d1o: Maximal value of first voltage derivative [dy].
	a\_plot: plot\_abstract, if requested.
%
%
\item[See also:]%
\hyperlink{ref_spike_shape}{\texttt{spike\_shape}}%
\ (p.~\pageref{ref_spike_shape})%
\index[funcref]{spike_shape@\fidxl{spike\_shape}}%
%
\item[Author:]%
Cengiz Gunay <cgunay@emory.edu>, 2004/08/02
%
\end{description}
\methodline%
\subsubsection[Method \texttt{calcInitVmLtdMaxCurv}]{Method \texttt{spike\_shape/calcInitVmLtdMaxCurv}}%
\index[funcref]{spike_shape@\fidxl{spike\_shape}!calcInitVmLtdMaxCurv@\fidxl{calcInitVmLtdMaxCurv}}%
\label{ref_spike_shape__calcInitVmLtdMaxCurv}%
\hypertarget{ref_spike_shape__calcInitVmLtdMaxCurv}{}%
\begin{description}
\item[Summary:]Calculates the action potential threshold using the maximum of the curvature equation only in the limited range given with two voltage slope thresholds.
%
\item[Usage:]~%
\begin{lyxcode}%
[init\_idx, a\_plot] = calcInitVmLtdMaxCurv(s, max\_idx, min\_idx, lo\_thr, hi\_thr, plotit)
%
\end{lyxcode}%
%
\item[Description:]%
Point of maximum curvature: Kp = V''[1 + (V')\textasciicircum{}2]\textasciicircum{}(-3/2)
 Taken from Sekerli, Del Negro, Lee and Butera. 
 IEEE Trans. Biomed. Eng., 51(9): 1665-71, 2004.
%%
\item[Parameters:]~
\begin{description}%
\item[\texttt{s}:]
 A spike\_shape object.
\item[\texttt{max\_idx}:]
 The index of the maximal point of the spike\_shape [dt].
\item[\texttt{min\_idx}:]
 The index of the minimal point of the spike\_shape [dt].
\item[\texttt{lo\_thr, hi\_thr}:]
 Lower and higher thresholds for time derivative of voltage.
\item[\texttt{plotit}:]
 If non-zero, plot a graph annotating the test results 

(optional).
\end{description}%
%
\item[Returns:
]~

	init\_idx: AP threshold index in the spike\_shape [dt].
	a\_plot: plot\_abstract, if requested.
%
%
\item[See also:]%
\hyperlink{ref_calcInitVm}{\texttt{calcInitVm}}%
\ (p.~\pageref{ref_calcInitVm})%
\index[funcref]{calcInitVm@\fidxl{calcInitVm}}%
%
\item[Author:]%
Cengiz Gunay <cgunay@emory.edu>, 2004/11/19
%
\end{description}
\methodline%
\subsubsection[Method \texttt{calcInitVmMaxCurvature}]{Method \texttt{spike\_shape/calcInitVmMaxCurvature}}%
\index[funcref]{spike_shape@\fidxl{spike\_shape}!calcInitVmMaxCurvature@\fidxl{calcInitVmMaxCurvature}}%
\label{ref_spike_shape__calcInitVmMaxCurvature}%
\hypertarget{ref_spike_shape__calcInitVmMaxCurvature}{}%
\begin{description}
\item[Summary:]Calculates the action potential threshold using the
			maximum of the curvature equation.
%
\item[Usage:]~%
\begin{lyxcode}%
[init\_idx, a\_plot] = calcInitVmMaxCurvature(s, max\_idx, min\_idx, plotit)
%
\end{lyxcode}%
%
\item[Description:]%
Point of maximum curvature: Kp = V''[1 + (V')\textasciicircum{}2]\textasciicircum{}(-3/2)
 Taken from Sekerli, Del Negro, Lee and Butera. 
 IEEE Trans. Biomed. Eng., 51(9): 1665-71, 2004.
%%
\item[Parameters:]~
\begin{description}%
\item[\texttt{s}:]
 A spike\_shape object.
\item[\texttt{max\_idx}:]
 The index of the maximal point of the spike\_shape [dt].
\item[\texttt{min\_idx}:]
 The index of the minimal point of the spike\_shape [dt].
\item[\texttt{plotit}:]
 If non-zero, plot a graph annotating the test results 

(optional).
\end{description}%
%
\item[Returns:
]~

	init\_idx: AP threshold index in the spike\_shape [dt].
	a\_plot: plot\_abstract, if requested.
%
%
\item[See also:]%
\hyperlink{ref_calcInitVm}{\texttt{calcInitVm}}%
\ (p.~\pageref{ref_calcInitVm})%
\index[funcref]{calcInitVm@\fidxl{calcInitVm}}%
%
\item[Author:]%
Cengiz Gunay <cgunay@emory.edu>, 2004/11/19
%
\end{description}
\methodline%
\subsubsection[Method \texttt{calcInitVmMaxCurvPhasePlane}]{Method \texttt{spike\_shape/calcInitVmMaxCurvPhasePlane}}%
\index[funcref]{spike_shape@\fidxl{spike\_shape}!calcInitVmMaxCurvPhasePlane@\fidxl{calcInitVmMaxCurvPhasePlane}}%
\label{ref_spike_shape__calcInitVmMaxCurvPhasePlane}%
\hypertarget{ref_spike_shape__calcInitVmMaxCurvPhasePlane}{}%
\begin{description}
\item[Summary:]Calculates the voltage at the maximum curvature in the phase plane as action potential threshold.
%
\item[Usage:]~%
\begin{lyxcode}%
[init\_idx, max\_d1o, a\_plot, fail\_cond] = 
	calcInitVmMaxCurvPhasePlane(s, max\_idx, min\_idx, plotit)
%
\end{lyxcode}%
%
\item[Description:]%
First take the phase-plane v'-v from the beginning to max(v'). Then regulate 
 intervals by interpolation. Point of maximum curvature: Kp = V''[1 + (V')\textasciicircum{}2]\textasciicircum{}(-3/2)
 Taken from Sekerli, Del Negro, Lee and Butera. 
 IEEE Trans. Biomed. Eng., 51(9): 1665-71, 2004.
%%
\item[Parameters:]~
\begin{description}%
\item[\texttt{s}:]
 A spike\_shape object.
\item[\texttt{max\_idx}:]
 The index of the maximal point of the spike\_shape [dt].
\item[\texttt{min\_idx}:]
 The index of the minimal point of the spike\_shape [dt].
\item[\texttt{plotit}:]
 If non-zero, plot a graph annotating the test results 

(optional).
\end{description}%
%
\item[Returns:
]~

	init\_idx: AP threshold index in the spike\_shape [dt].
	max\_d1o: Maximal value of first voltage derivative [dy].
	a\_plot: plot\_abstract, if requested.
	fail\_cond: True if this algorithm fails to be trustable.
%
%
\item[See also:]%
\hyperlink{ref_calcInitVm}{\texttt{calcInitVm}}%
\ (p.~\pageref{ref_calcInitVm})%
\index[funcref]{calcInitVm@\fidxl{calcInitVm}}%
%
\item[Author:]%
Cengiz Gunay <cgunay@emory.edu>, 2005/04/12
%
\end{description}
\methodline%
\subsubsection[Method \texttt{calcInitVmSekerliV2}]{Method \texttt{spike\_shape/calcInitVmSekerliV2}}%
\index[funcref]{spike_shape@\fidxl{spike\_shape}!calcInitVmSekerliV2@\fidxl{calcInitVmSekerliV2}}%
\label{ref_spike_shape__calcInitVmSekerliV2}%
\hypertarget{ref_spike_shape__calcInitVmSekerliV2}{}%
\begin{description}
\item[Summary:]Calculates the action potential threshold using the maximum second derivative of the phase space of voltage-time slope versus voltage.
%
\item[Usage:]~%
\begin{lyxcode}%
[init\_idx, a\_plot] = calcInitVmSekerliV2(s, max\_idx, min\_idx, plotit)
%
\end{lyxcode}%
%
%
\item[Parameters:]~
\begin{description}%
\item[\texttt{s}:]
 A spike\_shape object.
\item[\texttt{max\_idx}:]
 The index of the maximal point of the spike\_shape [dt].
\item[\texttt{min\_idx}:]
 The index of the minimal point of the spike\_shape [dt].
\item[\texttt{plotit}:]
 If non-zero, plot a graph annotating the test results 

(optional).
\end{description}%
%
\item[Returns:
]~

	init\_idx: Its index in the spike\_shape [dt].
	a\_plot: plot\_abstract, if requested.
%
%
\item[See also:]%
\hyperlink{ref_calcInitVm}{\texttt{calcInitVm}}%
\ (p.~\pageref{ref_calcInitVm})%
\index[funcref]{calcInitVm@\fidxl{calcInitVm}}%
%
\item[Author:]%
Cengiz Gunay <cgunay@emory.edu>, 2004/11/18
%
\end{description}
\methodline%
\subsubsection[Method \texttt{calcInitVmSlopeThreshold}]{Method \texttt{spike\_shape/calcInitVmSlopeThreshold}}%
\index[funcref]{spike_shape@\fidxl{spike\_shape}!calcInitVmSlopeThreshold@\fidxl{calcInitVmSlopeThreshold}}%
\label{ref_spike_shape__calcInitVmSlopeThreshold}%
\hypertarget{ref_spike_shape__calcInitVmSlopeThreshold}{}%
\begin{description}
\item[Summary:]Calculates the AP threshold using the slope threhold crossing.
%
\item[Usage:]~%
\begin{lyxcode}%
[init\_idx, a\_plot] = calcInitVmSlopeThreshold(s, max\_idx, min\_idx, thr, plotit)
%
\end{lyxcode}%
%
%
\item[Parameters:]~
\begin{description}%
\item[\texttt{s}:]
 A spike\_shape object.
\item[\texttt{max\_idx}:]
 The index of the maximal point of the spike\_shape [dt].
\item[\texttt{min\_idx}:]
 The index of the minimal point of the spike\_shape [dt].
\item[\texttt{thr}:]
 Threshold for time derivative of voltage.
\item[\texttt{plotit}:]
 If non-zero, plot a graph annotating the test results 

(optional).
\end{description}%
%
\item[Returns:
]~

	init\_idx: AP threshold index in the spike\_shape [dt].
	a\_plot: plot\_abstract, if requested.
%
%
\item[See also:]%
\hyperlink{ref_calcInitVm}{\texttt{calcInitVm}}%
\ (p.~\pageref{ref_calcInitVm})%
\index[funcref]{calcInitVm@\fidxl{calcInitVm}}%
%
\item[Author:]%
Cengiz Gunay <cgunay@emory.edu>, 2004/11/24
%
\end{description}
\methodline%
\subsubsection[Method \texttt{calcInitVmSlopeThresholdSupsample}]{Method \texttt{spike\_shape/calcInitVmSlopeThresholdSupsample}}%
\index[funcref]{spike_shape@\fidxl{spike\_shape}!calcInitVmSlopeThresholdSupsample@\fidxl{calcInitVmSlopeThresholdSupsample}}%
\label{ref_spike_shape__calcInitVmSlopeThresholdSupsample}%
\hypertarget{ref_spike_shape__calcInitVmSlopeThresholdSupsample}{}%
\begin{description}
\item[Summary:]Estimates the AP threshold as the first slope threshold crossing by first supersampling the data using cubic spline interpolation.
%
\item[Usage:]~%
\begin{lyxcode}%
[init\_idx, a\_plot] = calcInitVmSlopeThresholdSupsample(s, max\_idx, min\_idx, thr, plotit)
%
\end{lyxcode}%
%
%
\item[Parameters:]~
\begin{description}%
\item[\texttt{s}:]
 A spike\_shape object.
\item[\texttt{max\_idx}:]
 The index of the maximal point of the spike\_shape [dt].
\item[\texttt{min\_idx}:]
 The index of the minimal point of the spike\_shape [dt].
\item[\texttt{thr}:]
 Threshold for time derivative of voltage.
\item[\texttt{plotit}:]
 If non-zero, plot a graph annotating the test results 

(optional).
\end{description}%
%
\item[Returns:
]~

	init\_idx: AP threshold index in the spike\_shape [dt].
	a\_plot: plot\_abstract, if requested.
%
%
\item[See also:]%
\hyperlink{ref_calcInitVm}{\texttt{calcInitVm}}%
\ (p.~\pageref{ref_calcInitVm})%
\index[funcref]{calcInitVm@\fidxl{calcInitVm}}%
%
\item[Author:]%
Cengiz Gunay <cgunay@emory.edu>, 2005/03/23
%
\end{description}
\methodline%
\subsubsection[Method \texttt{calcInitVmV2PPLocal}]{Method \texttt{spike\_shape/calcInitVmV2PPLocal}}%
\index[funcref]{spike_shape@\fidxl{spike\_shape}!calcInitVmV2PPLocal@\fidxl{calcInitVmV2PPLocal}}%
\label{ref_spike_shape__calcInitVmV2PPLocal}%
\hypertarget{ref_spike_shape__calcInitVmV2PPLocal}{}%
\begin{description}
\item[Summary:]Calculates the action potential threshold by finding the local second derivative maximum in voltage-time slope versus voltage phase plane, nearest a slope threshold crossing.
%
\item[Usage:]~%
\begin{lyxcode}%
[init\_idx, a\_plot] = calcInitVmV2PPLocal(s, max\_idx, min\_idx, lo\_thr, plotit)
%
\end{lyxcode}%
%
%
\item[Parameters:]~
\begin{description}%
\item[\texttt{s}:]
 A spike\_shape object.
\item[\texttt{max\_idx}:]
 The index of the maximal point of the spike\_shape [dt].
\item[\texttt{min\_idx}:]
 The index of the minimal point of the spike\_shape [dt].
\item[\texttt{lo\_thr}:]
 Lower threshold for time voltage slope.
\item[\texttt{plotit}:]
 If non-zero, plot a graph annotating the test results 

(optional).
\end{description}%
%
\item[Returns:
]~

	init\_idx: Its index in the spike\_shape [dt].
	a\_plot: plot\_abstract, if requested.
%
%
\item[See also:]%
\hyperlink{ref_calcInitVm}{\texttt{calcInitVm}}%
\ (p.~\pageref{ref_calcInitVm})%
\index[funcref]{calcInitVm@\fidxl{calcInitVm}}%
%
\item[Author:]%
Cengiz Gunay <cgunay@emory.edu>, 2004/11/18
%
\end{description}
\methodline%
\subsubsection[Method \texttt{calcInitVmV3hKpTinterp}]{Method \texttt{spike\_shape/calcInitVmV3hKpTinterp}}%
\index[funcref]{spike_shape@\fidxl{spike\_shape}!calcInitVmV3hKpTinterp@\fidxl{calcInitVmV3hKpTinterp}}%
\label{ref_spike_shape__calcInitVmV3hKpTinterp}%
\hypertarget{ref_spike_shape__calcInitVmV3hKpTinterp}{}%
\begin{description}
\item[Summary:]Calculates candidates for action potential threshold using the first three time-domain derivatives.
%
\item[Usage:]~%
\begin{lyxcode}%
[init\_idx, a\_plot] = 
   calcInitVmV3hKpTinterp(s, max\_idx, min\_idx, lo\_thr, hi\_thr, plotit)
%
\end{lyxcode}%
%
\item[Description:]%
First uses interpolation to increase time points. Calculates h,
 the second derivative of phase-plane (d\textasciicircum{}2 v'/dv\textasciicircum{}2), in terms of 
 time-domain derivatives. Also calculates Kp = V''[1 + (V')\textasciicircum{}2]\textasciicircum{}(-3/2), 
 the curvature. The maxima of these functions are used as candidates 
 for AP thresholds.
%%
\item[Parameters:]~
\begin{description}%
\item[\texttt{s}:]
 A spike\_shape object.
\item[\texttt{max\_idx}:]
 The index of the maximal point of the spike\_shape [dt].
\item[\texttt{min\_idx}:]
 The index of the minimal point of the spike\_shape [dt].
\item[\texttt{lo\_thr, hi\_thr}:]
 Lower and higher thresholds for time derivative of voltage.
\item[\texttt{plotit}:]
 If non-zero, plot a graph annotating the test results 

(optional).
\end{description}%
%
\item[Returns:
]~

	init\_idx: Indices of threshold candidates in the spike\_shape [dt].
	a\_plot: plot\_abstract, if requested.
%
%
\item[See also:]%
\hyperlink{ref_calcInitVm}{\texttt{calcInitVm}}%
\ (p.~\pageref{ref_calcInitVm})%
\index[funcref]{calcInitVm@\fidxl{calcInitVm}}%
%
\item[Author:]%
Cengiz Gunay <cgunay@emory.edu>, 2004/11/18
%
\end{description}
\methodline%
\subsubsection[Method \texttt{calcMaxVm}]{Method \texttt{spike\_shape/calcMaxVm}}%
\index[funcref]{spike_shape@\fidxl{spike\_shape}!calcMaxVm@\fidxl{calcMaxVm}}%
\label{ref_spike_shape__calcMaxVm}%
\hypertarget{ref_spike_shape__calcMaxVm}{}%
\begin{description}
\item[Summary:]Calculates the maximal value of the spike\_shape, s. 
%
\item[Usage:]~%
\begin{lyxcode}%
[max\_val, max\_idx] = calcMaxVm(s)
%
\end{lyxcode}%
%
%
\item[Parameters:]~
\begin{description}%
\item[\texttt{s}:]
 A spike\_shape object.
\end{description}%
%
\item[Returns:
]~

	max\_val: The max value.
	max\_idx: Its index in the spike\_shape [dt].
%
%
\item[See also:]%
\hyperlink{ref_period}{\texttt{period}}%
\ (p.~\pageref{ref_period})%
\index[funcref]{period@\fidxl{period}}%
, \hyperlink{ref_spike_shape}{\texttt{spike\_shape}}%
\ (p.~\pageref{ref_spike_shape})%
\index[funcref]{spike_shape@\fidxl{spike\_shape}}%
, \hyperlink{ref_trace__calcMax}{\texttt{trace/calcMax}}%
\ (p.~\pageref{ref_trace__calcMax})%
\index[funcref]{trace@\fidxl{trace}!calcMax@\fidxl{calcMax}}%
%
\item[Author:]%
Cengiz Gunay <cgunay@emory.edu>, 2004/08/02
%
\end{description}
\methodline%
\subsubsection[Method \texttt{calcMinVm}]{Method \texttt{spike\_shape/calcMinVm}}%
\index[funcref]{spike_shape@\fidxl{spike\_shape}!calcMinVm@\fidxl{calcMinVm}}%
\label{ref_spike_shape__calcMinVm}%
\hypertarget{ref_spike_shape__calcMinVm}{}%
\begin{description}
\item[Summary:]Calculates the minimal value of the spike\_shape, s. 
%
\item[Usage:]~%
\begin{lyxcode}%
[min\_val, min\_idx, max\_min\_time] = calcMinVm(s, max\_idx)
%
\end{lyxcode}%
%
%
\item[Parameters:]~
\begin{description}%
\item[\texttt{s}:]
 A spike\_shape object.
\item[\texttt{max\_idx}:]
 The index of the maximal point of the spike\_shape [dt].
\end{description}%
%
\item[Returns:
]~

	min\_val: The min value [dy].
	min\_idx: Its index in the spike\_shape [dt].
	max\_min\_time: Time from max to min [dt].
%
%
\item[See also:]%
\hyperlink{ref_period}{\texttt{period}}%
\ (p.~\pageref{ref_period})%
\index[funcref]{period@\fidxl{period}}%
, \hyperlink{ref_spike_shape}{\texttt{spike\_shape}}%
\ (p.~\pageref{ref_spike_shape})%
\index[funcref]{spike_shape@\fidxl{spike\_shape}}%
, \hyperlink{ref_trace__calcMin}{\texttt{trace/calcMin}}%
\ (p.~\pageref{ref_trace__calcMin})%
\index[funcref]{trace@\fidxl{trace}!calcMin@\fidxl{calcMin}}%
%
\item[Author:]%
Cengiz Gunay <cgunay@emory.edu>, 2004/08/02
%
\end{description}
\methodline%
\subsubsection[Method \texttt{calcWidthFall}]{Method \texttt{spike\_shape/calcWidthFall}}%
\index[funcref]{spike_shape@\fidxl{spike\_shape}!calcWidthFall@\fidxl{calcWidthFall}}%
\label{ref_spike_shape__calcWidthFall}%
\hypertarget{ref_spike_shape__calcWidthFall}{}%
\begin{description}
\item[Summary:]Calculates the spike width and fall information of the spike\_shape, s. 
%
\item[Usage:]~%
\begin{lyxcode}%
[base\_width, half\_width, half\_Vm, fall\_time, min\_idx, min\_val, 
  max\_ahp, ahp\_decay\_constant, dahp\_mag, dahp\_idx] = ...
      calcWidthFall(s, max\_idx, max\_val, init\_idx, init\_val)
%
\end{lyxcode}%
%
\item[Description:]%
max\_* can be the peak\_* from calcInitVm.
%%
\item[Parameters:]~
\begin{description}%
\item[\texttt{s}:]
 A spike\_shape object.
\item[\texttt{max\_idx}:]
 The index of the maximal point [dt].
\item[\texttt{max\_val}:]
 The value of the maximal point [dy].
\item[\texttt{init\_idx}:]
 The index of spike initiation point [dt].
\item[\texttt{init\_val}:]
 The value of spike initiation point [dy].
\item[\texttt{fixed\_Vm}:]
 The desired height for width calculation [V].
\end{description}%
%
\item[Returns:
]~

	base\_width: Width of spike at base [dt]
	half\_width: Width of spike at half\_Vm [dt]
	half\_Vm: Half height of spike [dy]
	fall\_time: Time from peak to initialization level [dt].
	min\_idx: The index of the minimal point of the spike\_shape [dt].
	max\_ahp: Magnitude from initiation to minimum [dy].
	ahp\_decay\_constant: Approximation to refractory decay after maxAHP [dt].
	dahp\_mag: Magnitude of the double AHP peak
	dahp\_mag: Index of the double AHP peak
%
%
\item[See also:]%
\hyperlink{ref_spike_shape}{\texttt{spike\_shape}}%
\ (p.~\pageref{ref_spike_shape})%
\index[funcref]{spike_shape@\fidxl{spike\_shape}}%
%
\item[Author:]%
Cengiz Gunay <cgunay@emory.edu>, 2004/08/02
%
\end{description}
\methodline%
\subsubsection[Method \texttt{display}]{Method \texttt{spike\_shape/display}}%
\index[funcref]{spike_shape@\fidxl{spike\_shape}!display@\fidxl{display}}%
\label{ref_spike_shape__display}%
\hypertarget{ref_spike_shape__display}{}%
\begin{description}
%
%
%
%
%
%
%
\item[Author:]%
Cengiz Gunay <cgunay@emory.edu>, 2004/08/04
%
\end{description}
\methodline%
\subsubsection[Method \texttt{get}]{Method \texttt{spike\_shape/get}}%
\index[funcref]{spike_shape@\fidxl{spike\_shape}!get@\fidxl{get}}%
\label{ref_spike_shape__get}%
\hypertarget{ref_spike_shape__get}{}%
\begin{description}
\item[Summary:]Defines generic attribute retrieval for objects.
%
%
%
%
%
%
%
\item[Author:]%
Cengiz Gunay <cgunay@emory.edu>, 2004/09/14
%
\end{description}
\methodline%
\subsubsection[Method \texttt{getResults}]{Method \texttt{spike\_shape/getResults}}%
\index[funcref]{spike_shape@\fidxl{spike\_shape}!getResults@\fidxl{getResults}}%
\label{ref_spike_shape__getResults}%
\hypertarget{ref_spike_shape__getResults}{}%
\begin{description}
\item[Summary:]Runs all tests defined by this class and return them in a 
		structure.
%
\item[Usage:]~%
\begin{lyxcode}%
[results, a\_plot] = getResults(s, plotit)
%
\end{lyxcode}%
%
%
\item[Parameters:]~
\begin{description}%
\item[\texttt{s}:]
 A spike\_shape object.
\item[\texttt{plotit}:]
 If non-zero, plot a graph annotating the test results 

(optional).
\end{description}%
%
\item[Returns:
]~

	results: A structure associating test names to values in ms and mV.
	a\_plot: plot\_abstract, if requested.
%
%
\item[See also:]%
\hyperlink{ref_spike_shape}{\texttt{spike\_shape}}%
\ (p.~\pageref{ref_spike_shape})%
\index[funcref]{spike_shape@\fidxl{spike\_shape}}%
%
\item[Author:]%
Cengiz Gunay <cgunay@emory.edu>, 2004/08/02
%
\end{description}
\methodline%
\subsubsection[Method \texttt{plotCompareMethods}]{Method \texttt{spike\_shape/plotCompareMethods}}%
\index[funcref]{spike_shape@\fidxl{spike\_shape}!plotCompareMethods@\fidxl{plotCompareMethods}}%
\label{ref_spike_shape__plotCompareMethods}%
\hypertarget{ref_spike_shape__plotCompareMethods}{}%
\begin{description}
\item[Summary:]Creates a multi-plot comparing different action potential
			threshold finding methods.
%
\item[Usage:]~%
\begin{lyxcode}%
a\_plot = plotCompareMethods(s, title\_str)
%
\end{lyxcode}%
%
%
\item[Parameters:]~
\begin{description}%
\item[\texttt{s}:]
 A spike\_shape object.
\item[\texttt{title\_str}:]
 Title suffix (optional).
\end{description}%
%
\item[Returns:
]~

	a\_plot: A plot\_abstract object that can be visualized.
%
%
\item[See also:]%
\hyperlink{ref_spike_shape}{\texttt{spike\_shape}}%
\ (p.~\pageref{ref_spike_shape})%
\index[funcref]{spike_shape@\fidxl{spike\_shape}}%
, \hyperlink{ref_plot_abstract}{\texttt{plot\_abstract}}%
\ (p.~\pageref{ref_plot_abstract})%
\index[funcref]{plot_abstract@\fidxl{plot\_abstract}}%
%
\item[Author:]%
Cengiz Gunay <cgunay@emory.edu>, 2004/11/19
%
\end{description}
\methodline%
\subsubsection[Method \texttt{plotCompareMethodsSimple}]{Method \texttt{spike\_shape/plotCompareMethodsSimple}}%
\index[funcref]{spike_shape@\fidxl{spike\_shape}!plotCompareMethodsSimple@\fidxl{plotCompareMethodsSimple}}%
\label{ref_spike_shape__plotCompareMethodsSimple}%
\hypertarget{ref_spike_shape__plotCompareMethodsSimple}{}%
\begin{description}
\item[Summary:]Creates a multi-plot comparing different action potential
			threshold finding methods.
%
\item[Usage:]~%
\begin{lyxcode}%
a\_plot = plotCompareMethodsSimple(s, title\_str)
%
\end{lyxcode}%
%
%
\item[Parameters:]~
\begin{description}%
\item[\texttt{s}:]
 A spike\_shape object.
\item[\texttt{title\_str}:]
 Title suffix (optional).
\end{description}%
%
\item[Returns:
]~

	a\_plot: A plot\_abstract object that can be visualized.
%
%
\item[See also:]%
\hyperlink{ref_spike_shape}{\texttt{spike\_shape}}%
\ (p.~\pageref{ref_spike_shape})%
\index[funcref]{spike_shape@\fidxl{spike\_shape}}%
, \hyperlink{ref_plot_abstract}{\texttt{plot\_abstract}}%
\ (p.~\pageref{ref_plot_abstract})%
\index[funcref]{plot_abstract@\fidxl{plot\_abstract}}%
%
\item[Author:]%
Cengiz Gunay <cgunay@emory.edu>, 2004/11/19
%
\end{description}
\methodline%
\subsubsection[Method \texttt{plotPP}]{Method \texttt{spike\_shape/plotPP}}%
\index[funcref]{spike_shape@\fidxl{spike\_shape}!plotPP@\fidxl{plotPP}}%
\label{ref_spike_shape__plotPP}%
\hypertarget{ref_spike_shape__plotPP}{}%
\begin{description}
\item[Summary:]Plots the dV/dt vs. V phase-plane representation of the spike shape.
%
\item[Usage:]~%
\begin{lyxcode}%
a\_plot = plotPP(s)
%
\end{lyxcode}%
%
%
\item[Parameters:]~
\begin{description}%
\item[\texttt{s}:]
 A spike\_shape object.
\end{description}%
%
\item[Returns:
]~

	a\_plot: A plot\_abstract object that can be visualized.
%
%
\item[See also:]%
\hyperlink{ref_spike_shape}{\texttt{spike\_shape}}%
\ (p.~\pageref{ref_spike_shape})%
\index[funcref]{spike_shape@\fidxl{spike\_shape}}%
, \hyperlink{ref_plot_abstract}{\texttt{plot\_abstract}}%
\ (p.~\pageref{ref_plot_abstract})%
\index[funcref]{plot_abstract@\fidxl{plot\_abstract}}%
%
\item[Author:]%
Cengiz Gunay <cgunay@emory.edu>, 2004/11/16
%
\end{description}
\methodline%
\subsubsection[Method \texttt{plotResults}]{Method \texttt{spike\_shape/plotResults}}%
\index[funcref]{spike_shape@\fidxl{spike\_shape}!plotResults@\fidxl{plotResults}}%
\label{ref_spike_shape__plotResults}%
\hypertarget{ref_spike_shape__plotResults}{}%
\begin{description}
\item[Summary:]Plots the spike shape annotated with result characteristics.
%
\item[Usage:]~%
\begin{lyxcode}%
a\_plot = plotResults(s, title\_str, props)
%
\end{lyxcode}%
%
%
\item[Parameters:]~
\begin{description}%
\item[\texttt{s}:]
 A spike\_shape object.
\end{description}%
%
\item[Returns:
]~

	a\_plot: A plot\_abstract object that can be visualized.
	title\_str: (Optional) String to append to plot title.
	props: A structure with any optional properties, passed to trace/plotData.
%
%
\item[See also:]%
\hyperlink{ref_spike_shape}{\texttt{spike\_shape}}%
\ (p.~\pageref{ref_spike_shape})%
\index[funcref]{spike_shape@\fidxl{spike\_shape}}%
, \hyperlink{ref_plot_abstract}{\texttt{plot\_abstract}}%
\ (p.~\pageref{ref_plot_abstract})%
\index[funcref]{plot_abstract@\fidxl{plot\_abstract}}%
%
\item[Author:]%
Cengiz Gunay <cgunay@emory.edu>, 2004/11/17
%
\end{description}
\methodline%
\subsubsection[Method \texttt{plotTPP}]{Method \texttt{spike\_shape/plotTPP}}%
\index[funcref]{spike_shape@\fidxl{spike\_shape}!plotTPP@\fidxl{plotTPP}}%
\label{ref_spike_shape__plotTPP}%
\hypertarget{ref_spike_shape__plotTPP}{}%
\begin{description}
\item[Summary:]Plots the dV/dt vs. V phase-plane representation of the spike shape.
%
\item[Usage:]~%
\begin{lyxcode}%
a\_plot = plotTPP(s)
%
\end{lyxcode}%
%
\item[Description:]%
Uses the Taylor series estimation for finding the derivative dV/dt.
%%
\item[Parameters:]~
\begin{description}%
\item[\texttt{s}:]
 A spike\_shape object.
\end{description}%
%
\item[Returns:
]~

	a\_plot: A plot\_abstract object that can be visualized.
%
%
\item[See also:]%
\hyperlink{ref_spike_shape}{\texttt{spike\_shape}}%
\ (p.~\pageref{ref_spike_shape})%
\index[funcref]{spike_shape@\fidxl{spike\_shape}}%
, \hyperlink{ref_plot_abstract}{\texttt{plot\_abstract}}%
\ (p.~\pageref{ref_plot_abstract})%
\index[funcref]{plot_abstract@\fidxl{plot\_abstract}}%
, \hyperlink{ref_diffT}{\texttt{diffT}}%
\ (p.~\pageref{ref_diffT})%
\index[funcref]{diffT@\fidxl{diffT}}%
%
\item[Author:]%
Cengiz Gunay <cgunay@emory.edu>, 2004/11/16
%
\end{description}
\methodline%
\subsubsection[Method \texttt{set}]{Method \texttt{spike\_shape/set}}%
\index[funcref]{spike_shape@\fidxl{spike\_shape}!set@\fidxl{set}}%
\label{ref_spike_shape__set}%
\hypertarget{ref_spike_shape__set}{}%
\begin{description}
\item[Summary:]Generic method for setting object attributes.
%
%
%
%
%
%
%
\item[Author:]%
Cengiz Gunay <cgunay@emory.edu>, 2004/10/08
%
\end{description}
\methodline%
\subsection{Class \texttt{spike\_shape\_profile}}%
\index[funcref]{spike_shape_profile@\fidxl{spike\_shape\_profile}|boldhyperpage}%
\label{ref_spike_shape_profile}%
\hypertarget{ref_spike_shape_profile}{}%
\subsubsection[Constructor \texttt{spike\_shape\_profile}]{Constructor \texttt{spike\_shape\_profile/spike\_shape\_profile}}%
\index[funcref]{spike_shape_profile@\fidxl{spike\_shape\_profile}!spike_shape_profile@\fidxl{spike\_shape\_profile}}%
\label{ref_spike_shape_profile__spike_shape_profile}%
\hypertarget{ref_spike_shape_profile__spike_shape_profile}{}%
\begin{description}
\item[Summary:]Holds the results profile from a spike\_shape object.
%
\item[Usage:]~%
\begin{lyxcode}%
a\_ss\_profile = spike\_shape\_profile(results, a\_spike\_shape, props)
%
\end{lyxcode}%
%
%
\item[Parameters:]~
\begin{description}%
\item[\texttt{results}:]
 A structure containing test results.
\item[\texttt{a\_spike\_shape}:]
 A spike\_shape object.
\item[\texttt{props}:]
 A structure with any optional properties.
\end{description}%
%
\item[Returns a structure object with the following fields:
]~

	results\_profile: Contains results of tests.
	spike\_shape: The spike\_shape object from which results were obtained.
	props.
%
%
\item[See also:]%
\hyperlink{ref_results_profile}{\texttt{results\_profile}}%
\ (p.~\pageref{ref_results_profile})%
\index[funcref]{results_profile@\fidxl{results\_profile}}%
%
\item[Author:]%
Cengiz Gunay <cgunay@emory.edu>, 2005/08/17
%
\end{description}
\methodline%
\subsubsection[Method \texttt{get}]{Method \texttt{spike\_shape\_profile/get}}%
\index[funcref]{spike_shape_profile@\fidxl{spike\_shape\_profile}!get@\fidxl{get}}%
\label{ref_spike_shape_profile__get}%
\hypertarget{ref_spike_shape_profile__get}{}%
\begin{description}
\item[Summary:]Defines generic attribute retrieval for objects.
%
%
%
%
%
%
%
\item[Author:]%
Cengiz Gunay <cgunay@emory.edu>, 2004/09/14
%
\end{description}
\methodline%
\subsubsection[Method \texttt{plot\_abstract}]{Method \texttt{spike\_shape\_profile/plot\_abstract}}%
\index[funcref]{spike_shape_profile@\fidxl{spike\_shape\_profile}!plot_abstract@\fidxl{plot\_abstract}}%
\label{ref_spike_shape_profile__plot_abstract}%
\hypertarget{ref_spike_shape_profile__plot_abstract}{}%
\begin{description}
\item[Summary:]Plots the spike shape with measurements marked in red.
%
\item[Usage:]~%
\begin{lyxcode}%
a\_plot = plot\_abstract(s, props)
%
\end{lyxcode}%
%
%
\item[Parameters:]~
\begin{description}%
\item[\texttt{s}:]
 A spike\_shape object.
\item[\texttt{props}:]
 A structure with any optional properties.
\begin{description}%
\item[\texttt{absolute\_peak\_time}:]
 Shift the peak to this point on the plot.
\item[\texttt{no\_plot\_spike}:]
 Do not plot the spike shape first.
\end{description}%
\end{description}%
%
\item[Returns:
]~

	a\_plot: A plot\_abstract object that can be visualized.
%
%
\item[See also:]%
\hyperlink{ref_spike_shape}{\texttt{spike\_shape}}%
\ (p.~\pageref{ref_spike_shape})%
\index[funcref]{spike_shape@\fidxl{spike\_shape}}%
, \hyperlink{ref_plot_abstract}{\texttt{plot\_abstract}}%
\ (p.~\pageref{ref_plot_abstract})%
\index[funcref]{plot_abstract@\fidxl{plot\_abstract}}%
%
\item[Author:]%
Cengiz Gunay <cgunay@emory.edu>, 2005/08/17
%
\end{description}
\methodline%
\subsection{Class \texttt{spikes}}%
\index[funcref]{spikes@\fidxl{spikes}|boldhyperpage}%
\label{ref_spikes}%
\hypertarget{ref_spikes}{}%
\subsubsection[Constructor \texttt{spikes}]{Constructor \texttt{spikes/spikes}}%
\index[funcref]{spikes@\fidxl{spikes}!spikes@\fidxl{spikes}}%
\label{ref_spikes__spikes}%
\hypertarget{ref_spikes__spikes}{}%
\begin{description}
\item[Summary:]Spike times from a trace.
%
\item[Usage:]~%
\begin{lyxcode}%
obj = spikes(times, num\_samples, dt, id)
%
\end{lyxcode}%
%
%
\item[Parameters:]~
\begin{description}%
\item[\texttt{times}:]
 The spike times [dt].
\item[\texttt{num\_samples}:]
 Number of samples in the original trace.
\item[\texttt{dt}:]
 Time resolution [s].
\item[\texttt{id}:]
 Identification string.
\end{description}%
%
\item[Returns a structure object with the following fields:
]~

	times, num\_samples, dt, id.
%
%
\item[See also:]%
\hyperlink{ref_trace__spikes}{\texttt{trace/spikes}}%
\ (p.~\pageref{ref_trace__spikes})%
\index[funcref]{trace@\fidxl{trace}!spikes@\fidxl{spikes}}%
, \hyperlink{ref_trace}{\texttt{trace}}%
\ (p.~\pageref{ref_trace})%
\index[funcref]{trace@\fidxl{trace}}%
, \hyperlink{ref_spike_shape}{\texttt{spike\_shape}}%
\ (p.~\pageref{ref_spike_shape})%
\index[funcref]{spike_shape@\fidxl{spike\_shape}}%
, \hyperlink{ref_period}{\texttt{period}}%
\ (p.~\pageref{ref_period})%
\index[funcref]{period@\fidxl{period}}%
%
\item[Author:]%
Cengiz Gunay <cgunay@emory.edu>, 2004/07/30
%
\end{description}
\methodline%
\subsubsection[Method \texttt{addSpikes}]{Method \texttt{spikes/addSpikes}}%
\index[funcref]{spikes@\fidxl{spikes}!addSpikes@\fidxl{addSpikes}}%
\label{ref_spikes__addSpikes}%
\hypertarget{ref_spikes__addSpikes}{}%
\begin{description}
%
\item[Usage:]~%
\begin{lyxcode}%
s = addSpike(s, times)
%
\end{lyxcode}%
%
%
\item[Parameters:]~
\begin{description}%
\item[\texttt{s}:]
 A spikes object.
\item[\texttt{times}:]
 Times of spikes to add
\end{description}%
%
\item[Returns:
]~

	s: The updated object.
%
%
\item[See also:]%
\hyperlink{ref_spikes}{\texttt{spikes}}%
\ (p.~\pageref{ref_spikes})%
\index[funcref]{spikes@\fidxl{spikes}}%
%
\item[Author:]%
Cengiz Gunay <cgunay@emory.edu>, 2005/08/16
%
\end{description}
\methodline%
\subsubsection[Method \texttt{display}]{Method \texttt{spikes/display}}%
\index[funcref]{spikes@\fidxl{spikes}!display@\fidxl{display}}%
\label{ref_spikes__display}%
\hypertarget{ref_spikes__display}{}%
\begin{description}
%
%
%
%
%
%
%
\item[Author:]%
Cengiz Gunay <cgunay@emory.edu>, 2004/08/04
%
\end{description}
\methodline%
\subsubsection[Method \texttt{get}]{Method \texttt{spikes/get}}%
\index[funcref]{spikes@\fidxl{spikes}!get@\fidxl{get}}%
\label{ref_spikes__get}%
\hypertarget{ref_spikes__get}{}%
\begin{description}
\item[Summary:]Defines generic attribute retrieval for objects.
%
%
%
%
%
%
%
\item[Author:]%
Cengiz Gunay <cgunay@emory.edu>, 2004/09/14
%
\end{description}
\methodline%
\subsubsection[Method \texttt{getISIs}]{Method \texttt{spikes/getISIs}}%
\index[funcref]{spikes@\fidxl{spikes}!getISIs@\fidxl{getISIs}}%
\label{ref_spikes__getISIs}%
\hypertarget{ref_spikes__getISIs}{}%
\begin{description}
\item[Summary:]Calculates the firing rate of the spikes found in the given 
		period with an averaged inter-spike-interval approach.
%
\item[Usage:]~%
\begin{lyxcode}%
isi = getISIs(s, period)
%
\end{lyxcode}%
%
%
\item[Parameters:]~
\begin{description}%
\item[\texttt{s}:]
 A spikes object.
\item[\texttt{period}:]
 The period where spikes were found (optional)
\end{description}%
%
\item[Returns:
]~

	isi: Inter-spike-interval vector [dt]
%
%
\item[See also:]%
\hyperlink{ref_trace}{\texttt{trace}}%
\ (p.~\pageref{ref_trace})%
\index[funcref]{trace@\fidxl{trace}}%
, \hyperlink{ref_spikes}{\texttt{spikes}}%
\ (p.~\pageref{ref_spikes})%
\index[funcref]{spikes@\fidxl{spikes}}%
, \hyperlink{ref_period}{\texttt{period}}%
\ (p.~\pageref{ref_period})%
\index[funcref]{period@\fidxl{period}}%
%
\item[Author:]%
Cengiz Gunay <cgunay@emory.edu>, 2004/03/09
%
\end{description}
\methodline%
\subsubsection[Method \texttt{getResults}]{Method \texttt{spikes/getResults}}%
\index[funcref]{spikes@\fidxl{spikes}!getResults@\fidxl{getResults}}%
\label{ref_spikes__getResults}%
\hypertarget{ref_spikes__getResults}{}%
\begin{description}
\item[Summary:]Runs all tests defined by this class and return them in a 
		structure.
%
\item[Usage:]~%
\begin{lyxcode}%
results = getResults(s)
%
\end{lyxcode}%
%
%
\item[Parameters:]~
\begin{description}%
\item[\texttt{s}:]
 A spikes object.
\end{description}%
%
\item[Returns:
]~

	results: A structure associating test names to values 
		in ms and mV (or mA).
%
%
\item[See also:]%
\hyperlink{ref_spikes}{\texttt{spikes}}%
\ (p.~\pageref{ref_spikes})%
\index[funcref]{spikes@\fidxl{spikes}}%
%
\item[Author:]%
Cengiz Gunay <cgunay@emory.edu>, 2004/09/13
%
\end{description}
\methodline%
\subsubsection[Method \texttt{intoPeriod}]{Method \texttt{spikes/intoPeriod}}%
\index[funcref]{spikes@\fidxl{spikes}!intoPeriod@\fidxl{intoPeriod}}%
\label{ref_spikes__intoPeriod}%
\hypertarget{ref_spikes__intoPeriod}{}%
\begin{description}
\item[Summary:]Shifts the spikes times to be within the given period.
%
\item[Usage:]~%
\begin{lyxcode}%
obj = intoPeriod(s, a\_period)
%
\end{lyxcode}%
%
\item[Description:]%
Assuming this spikes object's length fits into the given period, it shifts
 all times to start from the beginning of the given period. This may be used
 to reconstruct the original spikes object from subperiods that were cut out
 previously, using the withinPeriod method.
%%
\item[Parameters:]~
\begin{description}%
\item[\texttt{s}:]
 A spikes object.
\item[\texttt{a\_period}:]
 The desired period 
\end{description}%
%
\item[Returns:
]~

	obj: A spikes object
%
%
\item[See also:]%
\hyperlink{ref_spikes}{\texttt{spikes}}%
\ (p.~\pageref{ref_spikes})%
\index[funcref]{spikes@\fidxl{spikes}}%
, \hyperlink{ref_period}{\texttt{period}}%
\ (p.~\pageref{ref_period})%
\index[funcref]{period@\fidxl{period}}%
%
\item[Author:]%
Cengiz Gunay <cgunay@emory.edu>, 2004/07/31
%
\end{description}
\methodline%
\subsubsection[Method \texttt{ISICV}]{Method \texttt{spikes/ISICV}}%
\index[funcref]{spikes@\fidxl{spikes}!ISICV@\fidxl{ISICV}}%
\label{ref_spikes__ISICV}%
\hypertarget{ref_spikes__ISICV}{}%
\begin{description}
\item[Summary:]Calculates the coefficient of variation (CV) of the 
	inter-spike-intervals (ISI).
%
\item[Usage:]~%
\begin{lyxcode}%
cv = ISICV(s, a\_period)
%
\end{lyxcode}%
%
%
\item[Parameters:]~
\begin{description}%
\item[\texttt{s}:]
 A spikes object.
\item[\texttt{a\_period}:]
 The period where spikes were found (optional)
\end{description}%
%
\item[Returns:
]~

	cv: Coefficient of variation.
%
%
\item[See also:]%
\hyperlink{ref_spikes}{\texttt{spikes}}%
\ (p.~\pageref{ref_spikes})%
\index[funcref]{spikes@\fidxl{spikes}}%
, \hyperlink{ref_period}{\texttt{period}}%
\ (p.~\pageref{ref_period})%
\index[funcref]{period@\fidxl{period}}%
%
\item[Author:]%
Cengiz Gunay <cgunay@emory.edu>, 2004/09/13
%
\end{description}
\methodline%
\subsubsection[Method \texttt{periodWhole}]{Method \texttt{spikes/periodWhole}}%
\index[funcref]{spikes@\fidxl{spikes}!periodWhole@\fidxl{periodWhole}}%
\label{ref_spikes__periodWhole}%
\hypertarget{ref_spikes__periodWhole}{}%
\begin{description}
\item[Summary:]Returns the boundaries of the whole period of spikes, s. 
%
\item[Usage:]~%
\begin{lyxcode}%
whole\_period = periodWhole(s)
%
\end{lyxcode}%
%
%
\item[Parameters:]~
\begin{description}%
\item[\texttt{s}:]
 A spikes object.
\end{description}%
%
%
%
\item[See also:]%
\hyperlink{ref_period}{\texttt{period}}%
\ (p.~\pageref{ref_period})%
\index[funcref]{period@\fidxl{period}}%
, \hyperlink{ref_spikes}{\texttt{spikes}}%
\ (p.~\pageref{ref_spikes})%
\index[funcref]{spikes@\fidxl{spikes}}%
%
\item[Author:]%
Cengiz Gunay <cgunay@emory.edu>, 2004/07/30
%
\end{description}
\methodline%
\subsubsection[Method \texttt{plot}]{Method \texttt{spikes/plot}}%
\index[funcref]{spikes@\fidxl{spikes}!plot@\fidxl{plot}}%
\label{ref_spikes__plot}%
\hypertarget{ref_spikes__plot}{}%
\begin{description}
\item[Summary:]Plots spikes.
%
\item[Usage:]~%
\begin{lyxcode}%
h = plot(t)
%
\end{lyxcode}%
%
%
\item[Parameters:]~
\begin{description}%
\item[\texttt{t}:]
 A spikes object.
\item[\texttt{title\_str}:]
 (Optional) String to append to plot title.
\end{description}%
%
\item[Returns:
]~

	h: Handle to figure object.
%
%
\item[See also:]%
\hyperlink{ref_spikes}{\texttt{spikes}}%
\ (p.~\pageref{ref_spikes})%
\index[funcref]{spikes@\fidxl{spikes}}%
, \hyperlink{ref_plot_abstract}{\texttt{plot\_abstract}}%
\ (p.~\pageref{ref_plot_abstract})%
\index[funcref]{plot_abstract@\fidxl{plot\_abstract}}%
%
\item[Author:]%
Cengiz Gunay <cgunay@emory.edu>, 2004/08/04
%
\end{description}
\methodline%
\subsubsection[Method \texttt{plotData}]{Method \texttt{spikes/plotData}}%
\index[funcref]{spikes@\fidxl{spikes}!plotData@\fidxl{plotData}}%
\label{ref_spikes__plotData}%
\hypertarget{ref_spikes__plotData}{}%
\begin{description}
\item[Summary:]Plots a spikes object.
%
\item[Usage:]~%
\begin{lyxcode}%
a\_plot = plotData(s, title\_str)
%
\end{lyxcode}%
%
\item[Description:]%
If s is a vector of spikes objects, returns a vector of plot objects.
%%
\item[Parameters:]~
\begin{description}%
\item[\texttt{s}:]
 A spikes object.
\end{description}%
%
\item[Returns:
]~

	a\_plot: A plot\_abstract object that can be visualized.
	title\_str: (Optional) String to append to plot title.
%
%
\item[See also:]%
\hyperlink{ref_trace}{\texttt{trace}}%
\ (p.~\pageref{ref_trace})%
\index[funcref]{trace@\fidxl{trace}}%
, \hyperlink{ref_plot_abstract}{\texttt{plot\_abstract}}%
\ (p.~\pageref{ref_plot_abstract})%
\index[funcref]{plot_abstract@\fidxl{plot\_abstract}}%
%
\item[Author:]%
Cengiz Gunay <cgunay@emory.edu>, 2005/10/21
%
\end{description}
\methodline%
\subsubsection[Method \texttt{plotFreqVsTime}]{Method \texttt{spikes/plotFreqVsTime}}%
\index[funcref]{spikes@\fidxl{spikes}!plotFreqVsTime@\fidxl{plotFreqVsTime}}%
\label{ref_spikes__plotFreqVsTime}%
\hypertarget{ref_spikes__plotFreqVsTime}{}%
\begin{description}
\item[Summary:]Plots a frequency-time graph from the spikes object.
%
\item[Usage:]~%
\begin{lyxcode}%
a\_plot = plotFreqVsTime(s, title\_str, props)
%
\end{lyxcode}%
%
\item[Description:]%
If s is a vector of spikes objects, returns a vector of plot objects.
%%
\item[Parameters:]~
\begin{description}%
\item[\texttt{s}:]
 A spikes object.
\item[\texttt{title\_str}:]
 (Optional) String to append to plot title.
\item[\texttt{props}:]
 A structure with any optional properties.
\begin{description}%
\item[\texttt{timeScale}:]
 's' for seconds, or 'ms' for milliseconds.
\item[\texttt{type}:]
 If 'simple' plots 1/is for each spike time, 

'manhattan' uses flat lines of 1/isi height between spike times (default).
(others passed to plot\_abstract)
\end{description}%
\end{description}%
%
\item[Returns:
]~

	a\_plot: A plot\_abstract object that can be visualized.
	title\_str: (Optional) String to append to plot title.
%
%
\item[See also:]%
\hyperlink{ref_trace}{\texttt{trace}}%
\ (p.~\pageref{ref_trace})%
\index[funcref]{trace@\fidxl{trace}}%
, \hyperlink{ref_plot_abstract}{\texttt{plot\_abstract}}%
\ (p.~\pageref{ref_plot_abstract})%
\index[funcref]{plot_abstract@\fidxl{plot\_abstract}}%
%
\item[Author:]%
Cengiz Gunay <cgunay@emory.edu>, 2006/05/05
%
\end{description}
\methodline%
\subsubsection[Method \texttt{plotISIs}]{Method \texttt{spikes/plotISIs}}%
\index[funcref]{spikes@\fidxl{spikes}!plotISIs@\fidxl{plotISIs}}%
\label{ref_spikes__plotISIs}%
\hypertarget{ref_spikes__plotISIs}{}%
\begin{description}
\item[Summary:]Plots a spikes object.
%
\item[Usage:]~%
\begin{lyxcode}%
a\_plot = plotISIs(s, title\_str)
%
\end{lyxcode}%
%
\item[Description:]%
If s is a vector of spikes objects, returns a vector of plot objects.
%%
\item[Parameters:]~
\begin{description}%
\item[\texttt{s}:]
 A spikes object.
\end{description}%
%
\item[Returns:
]~

	a\_plot: A plot\_abstract object that can be visualized.
	title\_str: (Optional) String to append to plot title.
%
%
\item[See also:]%
\hyperlink{ref_trace}{\texttt{trace}}%
\ (p.~\pageref{ref_trace})%
\index[funcref]{trace@\fidxl{trace}}%
, \hyperlink{ref_plot_abstract}{\texttt{plot\_abstract}}%
\ (p.~\pageref{ref_plot_abstract})%
\index[funcref]{plot_abstract@\fidxl{plot\_abstract}}%
%
\item[Author:]%
Cengiz Gunay <cgunay@emory.edu>, 2005/10/21
%
\end{description}
\methodline%
\subsubsection[Method \texttt{set}]{Method \texttt{spikes/set}}%
\index[funcref]{spikes@\fidxl{spikes}!set@\fidxl{set}}%
\label{ref_spikes__set}%
\hypertarget{ref_spikes__set}{}%
\begin{description}
\item[Summary:]Generic method for setting object attributes.
%
%
%
%
%
%
%
\item[Author:]%
Cengiz Gunay <cgunay@emory.edu>, 2004/10/08
%
\end{description}
\methodline%
\subsubsection[Method \texttt{SFA}]{Method \texttt{spikes/SFA}}%
\index[funcref]{spikes@\fidxl{spikes}!SFA@\fidxl{SFA}}%
\label{ref_spikes__SFA}%
\hypertarget{ref_spikes__SFA}{}%
\begin{description}
\item[Summary:]Calculates the spike frequency accommodation (SFA) of the 
	inter-spike-intervals (ISI).
%
\item[Usage:]~%
\begin{lyxcode}%
sfa = SFA(s, a\_period)
%
\end{lyxcode}%
%
\item[Description:]%
SFA is the ration of the last ISI to the first ISI in the period.
%%
\item[Parameters:]~
\begin{description}%
\item[\texttt{s}:]
 A spikes object.
\item[\texttt{a\_period}:]
 The period where spikes were found (optional)
\end{description}%
%
\item[Returns:
]~

	sfa: Spike frequency accommodation.
%
%
\item[See also:]%
\hyperlink{ref_spikes}{\texttt{spikes}}%
\ (p.~\pageref{ref_spikes})%
\index[funcref]{spikes@\fidxl{spikes}}%
, \hyperlink{ref_period}{\texttt{period}}%
\ (p.~\pageref{ref_period})%
\index[funcref]{period@\fidxl{period}}%
%
\item[Author:]%
Cengiz Gunay <cgunay@emory.edu>, 2004/09/13
%
\end{description}
\methodline%
\subsubsection[Method \texttt{spikeAmpSlope}]{Method \texttt{spikes/spikeAmpSlope}}%
\index[funcref]{spikes@\fidxl{spikes}!spikeAmpSlope@\fidxl{spikeAmpSlope}}%
\label{ref_spikes__spikeAmpSlope}%
\hypertarget{ref_spikes__spikeAmpSlope}{}%
\begin{description}
\item[Summary:]Calculates the time constant and steady-state value
		      of the spike amplitude for slow inactivating decays.
%
\item[Usage:]~%
\begin{lyxcode}%
[a\_tau, da\_inf] = spikeAmpSlope(a\_spikes, a\_trace, a\_period)
%
\end{lyxcode}%
%
%
\item[Parameters:]~
\begin{description}%
\item[\texttt{a\_spikes}:]
 A spikes object.
\item[\texttt{a\_trace}:]
 A trace object.
\item[\texttt{a\_period}:]
 The desired period (optional)
\end{description}%
%
\item[Returns:
]~

	a\_tau: Approximate amplitude decay constant.
	da\_inf: Delta change in final spike peak value from initial.
%
%
\item[See also:]%
\hyperlink{ref_period}{\texttt{period}}%
\ (p.~\pageref{ref_period})%
\index[funcref]{period@\fidxl{period}}%
, \hyperlink{ref_spikes}{\texttt{spikes}}%
\ (p.~\pageref{ref_spikes})%
\index[funcref]{spikes@\fidxl{spikes}}%
, \hyperlink{ref_trace}{\texttt{trace}}%
\ (p.~\pageref{ref_trace})%
\index[funcref]{trace@\fidxl{trace}}%
%
\item[Author:]%
Cengiz Gunay <cgunay@emory.edu>, 2004/09/15
%
\end{description}
\methodline%
\subsubsection[Method \texttt{spikeRate}]{Method \texttt{spikes/spikeRate}}%
\index[funcref]{spikes@\fidxl{spikes}!spikeRate@\fidxl{spikeRate}}%
\label{ref_spikes__spikeRate}%
\hypertarget{ref_spikes__spikeRate}{}%
\begin{description}
\item[Summary:]Calculates the average firing rate [Hz] of the given spike train.
%
\item[Usage:]~%
\begin{lyxcode}%
freq = spikeRate(s, a\_period)
%
\end{lyxcode}%
%
%
\item[Parameters:]~
\begin{description}%
\item[\texttt{s}:]
 A spikes object.
\item[\texttt{a\_period}:]
 The period where spikes were found (optional)
\end{description}%
%
\item[Returns:
]~

	freq: Firing rate [Hz]
%
%
\item[See also:]%
\hyperlink{ref_spikes}{\texttt{spikes}}%
\ (p.~\pageref{ref_spikes})%
\index[funcref]{spikes@\fidxl{spikes}}%
, \hyperlink{ref_period}{\texttt{period}}%
\ (p.~\pageref{ref_period})%
\index[funcref]{period@\fidxl{period}}%
%
\item[Author:]%
Cengiz Gunay <cgunay@emory.edu>, 2004/03/09
%
\end{description}
\methodline%
\subsubsection[Method \texttt{spikeRateISI}]{Method \texttt{spikes/spikeRateISI}}%
\index[funcref]{spikes@\fidxl{spikes}!spikeRateISI@\fidxl{spikeRateISI}}%
\label{ref_spikes__spikeRateISI}%
\hypertarget{ref_spikes__spikeRateISI}{}%
\begin{description}
\item[Summary:]Calculates the firing rate of the spikes found in the given 
		period with an averaged inter-spike-interval approach.
%
\item[Usage:]~%
\begin{lyxcode}%
freq = spikeRateISI(s, trace\_index, times, period)
%
\end{lyxcode}%
%
%
\item[Parameters:]~
\begin{description}%
\item[\texttt{s}:]
 A spikes object.
\item[\texttt{period}:]
 The period where spikes were found (optional)
\end{description}%
%
\item[Returns:
]~

	freq: Firing rate [Hz]
%
%
\item[See also:]%
\hyperlink{ref_trace}{\texttt{trace}}%
\ (p.~\pageref{ref_trace})%
\index[funcref]{trace@\fidxl{trace}}%
, \hyperlink{ref_spikes}{\texttt{spikes}}%
\ (p.~\pageref{ref_spikes})%
\index[funcref]{spikes@\fidxl{spikes}}%
, \hyperlink{ref_period}{\texttt{period}}%
\ (p.~\pageref{ref_period})%
\index[funcref]{period@\fidxl{period}}%
%
\item[Author:]%
Cengiz Gunay <cgunay@emory.edu>, 2004/03/09
%
\end{description}
\methodline%
\subsubsection[Method \texttt{subsref}]{Method \texttt{spikes/subsref}}%
\index[funcref]{spikes@\fidxl{spikes}!subsref@\fidxl{subsref}}%
\label{ref_spikes__subsref}%
\hypertarget{ref_spikes__subsref}{}%
\begin{description}
\item[Summary:]Defines generic indexing for objects.
%
%
%
%
%
%
%
\item[Author:]%
Cengiz Gunay <cgunay@emory.edu>, 2004/08/04
%
\end{description}
\methodline%
\subsubsection[Method \texttt{vertcat}]{Method \texttt{spikes/vertcat}}%
\index[funcref]{spikes@\fidxl{spikes}!vertcat@\fidxl{vertcat}}%
\label{ref_spikes__vertcat}%
\hypertarget{ref_spikes__vertcat}{}%
\begin{description}
\item[Summary:]Vertical concatanation [a\_spikes;with\_spikes;...] operator.
%
\item[Usage:]~%
\begin{lyxcode}%
a\_spikes = vertcat(a\_spikes, with\_spikes, ...)
%
\end{lyxcode}%
%
\item[Description:]%
Concatanates spike times of with\_spikes with that of a\_spikes. Overrides the built-in
 vertcat function that is called when [a\_spikes;with\_spikes] is executed.
%%
\item[Parameters:]~

a\_spikes, with\_spikes, ...: Spikes objects.
%
\item[Returns:
]~

	a\_spikes: A tests\_spikes that contains times of all given spikes objects.
%
%
\item[See also:]%
\hyperlink{ref_vertcat}{\texttt{vertcat}}%
\ (p.~\pageref{ref_vertcat})%
\index[funcref]{vertcat@\fidxl{vertcat}}%
, \hyperlink{ref_spikes}{\texttt{spikes}}%
\ (p.~\pageref{ref_spikes})%
\index[funcref]{spikes@\fidxl{spikes}}%
%
\item[Author:]%
Cengiz Gunay <cgunay@emory.edu>, 2005/08/16
%
\end{description}
\methodline%
\subsubsection[Method \texttt{withinPeriod}]{Method \texttt{spikes/withinPeriod}}%
\index[funcref]{spikes@\fidxl{spikes}!withinPeriod@\fidxl{withinPeriod}}%
\label{ref_spikes__withinPeriod}%
\hypertarget{ref_spikes__withinPeriod}{}%
\begin{description}
\item[Summary:]Returns a spikes object valid only within the given period, subtracts the offset.
%
\item[Usage:]~%
\begin{lyxcode}%
obj = withinPeriod(s, a\_period)
%
\end{lyxcode}%
%
%
\item[Parameters:]~
\begin{description}%
\item[\texttt{s}:]
 A spikes object.
\item[\texttt{a\_period}:]
 The desired period 
\end{description}%
%
\item[Returns:
]~

	obj: A spikes object
%
%
\item[See also:]%
\hyperlink{ref_spikes}{\texttt{spikes}}%
\ (p.~\pageref{ref_spikes})%
\index[funcref]{spikes@\fidxl{spikes}}%
, \hyperlink{ref_period}{\texttt{period}}%
\ (p.~\pageref{ref_period})%
\index[funcref]{period@\fidxl{period}}%
%
\item[Author:]%
Cengiz Gunay <cgunay@emory.edu>, 2004/07/31
%
\end{description}
\methodline%
\subsubsection[Method \texttt{withinPeriodWOffset}]{Method \texttt{spikes/withinPeriodWOffset}}%
\index[funcref]{spikes@\fidxl{spikes}!withinPeriodWOffset@\fidxl{withinPeriodWOffset}}%
\label{ref_spikes__withinPeriodWOffset}%
\hypertarget{ref_spikes__withinPeriodWOffset}{}%
\begin{description}
\item[Summary:]Returns a spikes object valid only within the given period, keeps the offset.
%
\item[Usage:]~%
\begin{lyxcode}%
obj = withinPeriodWOffset(s, a\_period)
%
\end{lyxcode}%
%
%
\item[Parameters:]~
\begin{description}%
\item[\texttt{s}:]
 A spikes object.
\item[\texttt{a\_period}:]
 The desired period 
\end{description}%
%
\item[Returns:
]~

	obj: A spikes object
%
%
\item[See also:]%
\hyperlink{ref_spikes}{\texttt{spikes}}%
\ (p.~\pageref{ref_spikes})%
\index[funcref]{spikes@\fidxl{spikes}}%
, \hyperlink{ref_period}{\texttt{period}}%
\ (p.~\pageref{ref_period})%
\index[funcref]{period@\fidxl{period}}%
%
\item[Author:]%
Cengiz Gunay <cgunay@emory.edu>, 2005/05/09
%
\end{description}
\methodline%
\subsection{Class \texttt{spikes\_db}}%
\index[funcref]{spikes_db@\fidxl{spikes\_db}|boldhyperpage}%
\label{ref_spikes_db}%
\hypertarget{ref_spikes_db}{}%
\subsubsection[Constructor \texttt{spikes\_db}]{Constructor \texttt{spikes\_db/spikes\_db}}%
\index[funcref]{spikes_db@\fidxl{spikes\_db}!spikes_db@\fidxl{spikes\_db}}%
\label{ref_spikes_db__spikes_db}%
\hypertarget{ref_spikes_db__spikes_db}{}%
\begin{description}
\item[Summary:]A database of spike shape results obtained from a period in a trace.
%
\item[Usage:]~%
\begin{lyxcode}%
a\_spikes\_db = spikes\_db(data, col\_names, a\_trace, a\_period, id, props)
%
\end{lyxcode}%
%
\item[Description:]%
This is a subclass of tests\_db. Use trace/analyzeSpikesInPeriod to 
 get an instance of this class.
%%
\item[Parameters:]~
\begin{description}%
\item[\texttt{data}:]
 Database contents.
\item[\texttt{col\_names}:]
 The column names.
\item[\texttt{a\_trace}:]
 The trace where the spikes were found.
\item[\texttt{a\_period}:]
 The period inside a\_trace where spikes were found.
\item[\texttt{id}:]
 An identifying string.
\item[\texttt{props}:]
 A structure with any optional properties.
\end{description}%
%
\item[Returns a structure object with the following fields:
]~

	tests\_db, trace, period, props.
%
%
\item[See also:]%
\hyperlink{ref_tests_db}{\texttt{tests\_db}}%
\ (p.~\pageref{ref_tests_db})%
\index[funcref]{tests_db@\fidxl{tests\_db}}%
, \hyperlink{ref_trace}{\texttt{trace}}%
\ (p.~\pageref{ref_trace})%
\index[funcref]{trace@\fidxl{trace}}%
, \hyperlink{ref_period}{\texttt{period}}%
\ (p.~\pageref{ref_period})%
\index[funcref]{period@\fidxl{period}}%
, \hyperlink{ref_trace__analyzeSpikesInPeriod}{\texttt{trace/analyzeSpikesInPeriod}}%
\ (p.~\pageref{ref_trace__analyzeSpikesInPeriod})%
\index[funcref]{trace@\fidxl{trace}!analyzeSpikesInPeriod@\fidxl{analyzeSpikesInPeriod}}%
%
\item[Author:]%
Cengiz Gunay <cgunay@emory.edu>, 2005/08/17
%
\end{description}
\methodline%
\subsubsection[Method \texttt{plot\_abstract}]{Method \texttt{spikes\_db/plot\_abstract}}%
\index[funcref]{spikes_db@\fidxl{spikes\_db}!plot_abstract@\fidxl{plot\_abstract}}%
\label{ref_spikes_db__plot_abstract}%
\hypertarget{ref_spikes_db__plot_abstract}{}%
\begin{description}
\item[Summary:]Visualizes the spikes\_db by marking spike shapes measurements on the trace plot.
%
\item[Usage:]~%
\begin{lyxcode}%
a\_pm = plot\_abstract(a\_db, title\_str, props)
%
\end{lyxcode}%
%
%
\item[Parameters:]~
\begin{description}%
\item[\texttt{a\_db}:]
 A spikes\_db object.
\item[\texttt{title\_str}:]
 (Optional) A string to be concatanated to the title.
\item[\texttt{props}:]
 A structure with any optional properties passed to trace/plotData.
\end{description}%
%
\item[Returns:
]~

	a\_pm: A trace plot.
%
%
\item[See also:]%
\hyperlink{ref_plot_abstract__plot_abstract}{\texttt{plot\_abstract/plot\_abstract}}%
\ (p.~\pageref{ref_plot_abstract__plot_abstract})%
\index[funcref]{plot_abstract@\fidxl{plot\_abstract}!plot_abstract@\fidxl{plot\_abstract}}%
, \hyperlink{ref_tests_db__plot_abstract}{\texttt{tests\_db/plot\_abstract}}%
\ (p.~\pageref{ref_tests_db__plot_abstract})%
\index[funcref]{tests_db@\fidxl{tests\_db}!plot_abstract@\fidxl{plot\_abstract}}%
, \hyperlink{ref_plotFigure}{\texttt{plotFigure}}%
\ (p.~\pageref{ref_plotFigure})%
\index[funcref]{plotFigure@\fidxl{plotFigure}}%
%
\item[Author:]%
Cengiz Gunay <cgunay@emory.edu>, 2005/08/17
%
\end{description}
\methodline%
\subsection{Class \texttt{sql\_portal}}%
\index[funcref]{sql_portal@\fidxl{sql\_portal}|boldhyperpage}%
\label{ref_sql_portal}%
\hypertarget{ref_sql_portal}{}%
\subsubsection[Constructor \texttt{sql\_portal}]{Constructor \texttt{sql\_portal/sql\_portal}}%
\index[funcref]{sql_portal@\fidxl{sql\_portal}!sql_portal@\fidxl{sql\_portal}}%
\label{ref_sql_portal__sql_portal}%
\hypertarget{ref_sql_portal__sql_portal}{}%
\begin{description}
\item[Summary:]For import and export to external SQL engines using the Matlab Database Toolbox.
%
\item[Usage:]~%
\begin{lyxcode}%
obj = sql\_portal(db\_conn, id, props)
%
\end{lyxcode}%
%
\item[Description:]%
Uses the Database (DB) Toolbox's connection object to read and write to
 external SQL engines.
%%
\item[Parameters:]~
\begin{description}%
\item[\texttt{db\_conn}:]
 An object of the database class of the DB Toolbox.
\item[\texttt{id}:]
 An identifying string.
\item[\texttt{props}:]
 A structure with any optional properties.
\end{description}%
%
\item[Returns a structure object with the following fields:
]~

	db\_conn: The DB Toolbox connection object.
	id, props.
%
%
\item[See also:]%
\hyperlink{ref_tests_db}{\texttt{tests\_db}}%
\ (p.~\pageref{ref_tests_db})%
\index[funcref]{tests_db@\fidxl{tests\_db}}%
, \hyperlink{ref_database}{\texttt{database}}%
\ (p.~\pageref{ref_database})%
\index[funcref]{database@\fidxl{database}}%
%
\item[Author:]%
Cengiz Gunay <cgunay@emory.edu>, 2007/11/29
%
\end{description}
\methodline%
\subsubsection[Method \texttt{get}]{Method \texttt{sql\_portal/get}}%
\index[funcref]{sql_portal@\fidxl{sql\_portal}!get@\fidxl{get}}%
\label{ref_sql_portal__get}%
\hypertarget{ref_sql_portal__get}{}%
\begin{description}
\item[Summary:]Defines generic attribute retrieval for objects.
%
%
%
%
%
%
%
\item[Author:]%
Cengiz Gunay <cgunay@emory.edu>, 2004/09/14
%
\end{description}
\methodline%
\subsubsection[Method \texttt{sql\_statement}]{Method \texttt{sql\_portal/sql\_statement}}%
\index[funcref]{sql_portal@\fidxl{sql\_portal}!sql_statement@\fidxl{sql\_statement}}%
\label{ref_sql_portal__sql_statement}%
\hypertarget{ref_sql_portal__sql_statement}{}%
\begin{description}
\item[Summary:]Run an SQL statement and discard its results.
%
\item[Usage:]~%
\begin{lyxcode}%
response\_str = sql\_statement(a\_sql\_portal, statement\_string, props)
%
\end{lyxcode}%
%
\item[Description:]%
This function is for sending SQL statements that do not return any data,
 for such functions as inserting data into a database, creating views, and
 running administration commands. See sql\_portal/tests\_db to import select
 query results into Pandora.
%%
\item[Parameters:]~
\begin{description}%
\item[\texttt{a\_sql\_portal}:]
 A sql\_portal object.
\item[\texttt{statement\_string}:]
 An SQL statement that does not return data.
\item[\texttt{props}:]
 A structure with any optional properties.
\end{description}%
%
\item[Returns:
]~

	response\_str: Response string from the SQL engine.
%
%
\item[See also:]%
\hyperlink{ref_sql_portal__tests_db}{\texttt{sql\_portal/tests\_db}}%
\ (p.~\pageref{ref_sql_portal__tests_db})%
\index[funcref]{sql_portal@\fidxl{sql\_portal}!tests_db@\fidxl{tests\_db}}%
, \hyperlink{ref_database}{\texttt{database}}%
\ (p.~\pageref{ref_database})%
\index[funcref]{database@\fidxl{database}}%
%
\item[Author:]%
Cengiz Gunay <cgunay@emory.edu>, 2008/04/18
%
\end{description}
\methodline%
\subsubsection[Method \texttt{sql\_table}]{Method \texttt{sql\_portal/sql\_table}}%
\index[funcref]{sql_portal@\fidxl{sql\_portal}!sql_table@\fidxl{sql\_table}}%
\label{ref_sql_portal__sql_table}%
\hypertarget{ref_sql_portal__sql_table}{}%
\begin{description}
\item[Summary:]Create an SQL table from the contents of a tests\_db object.
%
\item[Usage:]~%
\begin{lyxcode}%
sql\_table(a\_sql\_portal, a\_tests\_db, table\_label, props)
%
\end{lyxcode}%
%
\item[Description:]%
Converter function to get a tests\_db object properly annotated with
 the metadata obtained from the results of the executed SQL
 query. Currently this function is limited to importing numeric data only.
%%
\item[Parameters:]~
\begin{description}%
\item[\texttt{a\_sql\_portal}:]
 A sql\_portal object.
\item[\texttt{a\_tests\_db}:]
 A tests\_db object.
\item[\texttt{table\_label}:]
 Name of the newly created table.
\item[\texttt{props}:]
 A structure with any optional properties.
\end{description}%
%
\item[Returns:
]~

%
%
\item[See also:]%
\hyperlink{ref_tests_db__tests_db}{\texttt{tests\_db/tests\_db}}%
\ (p.~\pageref{ref_tests_db__tests_db})%
\index[funcref]{tests_db@\fidxl{tests\_db}!tests_db@\fidxl{tests\_db}}%
, \hyperlink{ref_database}{\texttt{database}}%
\ (p.~\pageref{ref_database})%
\index[funcref]{database@\fidxl{database}}%
, \hyperlink{ref_sql_portal__tests_db}{\texttt{sql\_portal/tests\_db}}%
\ (p.~\pageref{ref_sql_portal__tests_db})%
\index[funcref]{sql_portal@\fidxl{sql\_portal}!tests_db@\fidxl{tests\_db}}%
%
\item[Author:]%
Cengiz Gunay <cgunay@emory.edu>, 2008/12/17
%
\end{description}
\methodline%
\subsubsection[Method \texttt{subsref}]{Method \texttt{sql\_portal/subsref}}%
\index[funcref]{sql_portal@\fidxl{sql\_portal}!subsref@\fidxl{subsref}}%
\label{ref_sql_portal__subsref}%
\hypertarget{ref_sql_portal__subsref}{}%
\begin{description}
\item[Summary:]Defines generic indexing for objects.
%
%
%
%
%
%
%
%
\end{description}
\methodline%
\subsubsection[Method \texttt{tests\_db}]{Method \texttt{sql\_portal/tests\_db}}%
\index[funcref]{sql_portal@\fidxl{sql\_portal}!tests_db@\fidxl{tests\_db}}%
\label{ref_sql_portal__tests_db}%
\hypertarget{ref_sql_portal__tests_db}{}%
\begin{description}
\item[Summary:]Create a tests\_db object from the results of a SQL query.
%
\item[Usage:]~%
\begin{lyxcode}%
a\_db = tests\_db(a\_sql\_portal, query\_string, query\_id, props)
%
\end{lyxcode}%
%
\item[Description:]%
Converter function to get a tests\_db object properly annotated with
 the metadata obtained from the results of the executed SQL
 query. Currently this function is limited to importing numeric data only.
%%
\item[Parameters:]~
\begin{description}%
\item[\texttt{a\_sql\_portal}:]
 A sql\_portal object.
\item[\texttt{query\_string}:]
 An SQL query returning numeric results.
\item[\texttt{query\_id}:]
 Identifier associated witht the query, to be passed

into the tests\_db object.
\item[\texttt{props}:]
 A structure with any optional properties passed to tests\_db.
\end{description}%
%
\item[Returns:
]~

	a\_db: A tests\_db object.
%
%
\item[See also:]%
\hyperlink{ref_tests_db}{\texttt{tests\_db}}%
\ (p.~\pageref{ref_tests_db})%
\index[funcref]{tests_db@\fidxl{tests\_db}}%
, \hyperlink{ref_database}{\texttt{database}}%
\ (p.~\pageref{ref_database})%
\index[funcref]{database@\fidxl{database}}%
%
\item[Author:]%
Cengiz Gunay <cgunay@emory.edu>, 2007/11/29
%
\end{description}
\methodline%
\subsection{Class \texttt{stats\_db}}%
\index[funcref]{stats_db@\fidxl{stats\_db}|boldhyperpage}%
\label{ref_stats_db}%
\hypertarget{ref_stats_db}{}%
\subsubsection[Constructor \texttt{stats\_db}]{Constructor \texttt{stats\_db/stats\_db}}%
\index[funcref]{stats_db@\fidxl{stats\_db}!stats_db@\fidxl{stats\_db}}%
\label{ref_stats_db__stats_db}%
\hypertarget{ref_stats_db__stats_db}{}%
\begin{description}
\item[Summary:]A database of rows corresponding to statistical distribution
		properties of tests. Multiple pages can be used to
		indicate another dimension.
%
\item[Usage:]~%
\begin{lyxcode}%
a\_stats\_db = stats\_db(test\_results, col\_names, row\_names, page\_names, 
			id, props)
%
\end{lyxcode}%
%
\item[Description:]%
This is a subclass of tests\_3D\_db. Allows generating a plot, etc.
%%
\item[Parameters:]~
\begin{description}%
\item[\texttt{test\_results}:]
 The 3-d array of rows, columns, and pages. Will*/

also accept a tests\_db object, but row\_names must also be supplied.
\item[\texttt{col\_names}:]
 Test names in this db.
\item[\texttt{row\_names}:]
 Statistical test names for each row (e.g., mean, STD, SE, etc.)
\item[\texttt{page\_names}:]
 Meaning of each separate page of data 

(e.g., a different invariant parameter).
\item[\texttt{id}:]
 An identifying string.
\item[\texttt{props}:]
 A structure with any optional properties.
\begin{description}%
\item[\texttt{axis\_limits}:]
 Limits in the form of [xmin xmax ymin ymax]

for errorbar axes.
\item[\texttt{yTicksPos}:]
 'left' means only put y-axis ticks to leftmost plot.
\item[\texttt{xTicksPos}:]
 'bottom' means only put x-axis ticks to lowest plot.
\end{description}%
\end{description}%
%
\item[Returns a structure object with the following fields:
]~

	tests\_3D\_db.
%
%
\item[See also:]%
\hyperlink{ref_tests_3D_db}{\texttt{tests\_3D\_db}}%
\ (p.~\pageref{ref_tests_3D_db})%
\index[funcref]{tests_3D_db@\fidxl{tests\_3D\_db}}%
, \hyperlink{ref_plot_abstract}{\texttt{plot\_abstract}}%
\ (p.~\pageref{ref_plot_abstract})%
\index[funcref]{plot_abstract@\fidxl{plot\_abstract}}%
%
\item[Author:]%
Cengiz Gunay <cgunay@emory.edu>, 2004/10/07
%
\end{description}
\methodline%
\subsubsection[Method \texttt{compareStats}]{Method \texttt{stats\_db/compareStats}}%
\index[funcref]{stats_db@\fidxl{stats\_db}!compareStats@\fidxl{compareStats}}%
\label{ref_stats_db__compareStats}%
\hypertarget{ref_stats_db__compareStats}{}%
\begin{description}
\item[Summary:]Merges multiple stats\_dbs into pages of a single stats\_db for comparison.
%
\item[Usage:]~%
\begin{lyxcode}%
a\_mult\_stats\_db = compareStats(a\_stats\_db, a\_2nd\_stats\_db, ...)
%
\end{lyxcode}%
%
\item[Description:]%
Generates a plot\_simple object from this histogram.
%%
\item[Parameters:]~
\begin{description}%
\item[\texttt{a\_stats\_db}:]
 A stats\_db object.
\end{description}%
%
\item[Returns:
]~

	a\_mult\_stats\_db: A multi-page stats\_db.
%
%
\item[See also:]%
\hyperlink{ref_plot_abstract}{\texttt{plot\_abstract}}%
\ (p.~\pageref{ref_plot_abstract})%
\index[funcref]{plot_abstract@\fidxl{plot\_abstract}}%
, \hyperlink{ref_plot_simple}{\texttt{plot\_simple}}%
\ (p.~\pageref{ref_plot_simple})%
\index[funcref]{plot_simple@\fidxl{plot\_simple}}%
%
\item[Author:]%
Cengiz Gunay <cgunay@emory.edu>, 2004/10/08
%
\end{description}
\methodline%
\subsubsection[Method \texttt{get}]{Method \texttt{stats\_db/get}}%
\index[funcref]{stats_db@\fidxl{stats\_db}!get@\fidxl{get}}%
\label{ref_stats_db__get}%
\hypertarget{ref_stats_db__get}{}%
\begin{description}
\item[Summary:]Defines generic attribute retrieval for objects.
%
%
%
%
%
%
%
\item[Author:]%
Cengiz Gunay <cgunay@emory.edu>, 2004/09/14
%
\end{description}
\methodline%
\subsubsection[Method \texttt{onlyRowsTests}]{Method \texttt{stats\_db/onlyRowsTests}}%
\index[funcref]{stats_db@\fidxl{stats\_db}!onlyRowsTests@\fidxl{onlyRowsTests}}%
\label{ref_stats_db__onlyRowsTests}%
\hypertarget{ref_stats_db__onlyRowsTests}{}%
\begin{description}
\item[Summary:]Returns a tests\_db that only contains the desired 
		tests and rows (and pages).
%
\item[Usage:]~%
\begin{lyxcode}%
obj = onlyRowsTests(obj, rows, tests, pages)
%
\end{lyxcode}%
%
\item[Description:]%
Selects the given dimensions and returns in a new tests\_db object.
%%
\item[Parameters:]~
\begin{description}%
\item[\texttt{obj}:]
 A tests\_db object.
\item[\texttt{rows}:]
 A logical or index vector of rows. If ':', all rows.
\item[\texttt{tests}:]
 Cell array of test names or column indices. If ':', all tests.
\item[\texttt{pages}:]
 (Optional) A logical or index vector of pages. ':' for all pages.
\end{description}%
%
\item[Returns:
]~

	obj: The new tests\_db object.
%
%
\item[See also:]%
\hyperlink{ref_subsref}{\texttt{subsref}}%
\ (p.~\pageref{ref_subsref})%
\index[funcref]{subsref@\fidxl{subsref}}%
, \hyperlink{ref_tests_db}{\texttt{tests\_db}}%
\ (p.~\pageref{ref_tests_db})%
\index[funcref]{tests_db@\fidxl{tests\_db}}%
%
\item[Author:]%
Cengiz Gunay <cgunay@emory.edu>, 2004/09/17
%
\end{description}
\methodline%
\subsubsection[Method \texttt{plotColorVar}]{Method \texttt{stats\_db/plotColorVar}}%
\index[funcref]{stats_db@\fidxl{stats\_db}!plotColorVar@\fidxl{plotColorVar}}%
\label{ref_stats_db__plotColorVar}%
\hypertarget{ref_stats_db__plotColorVar}{}%
\begin{description}
\item[Summary:]Create a color-plot of parameter-test variations in a matrix.
%
\item[Usage:]~%
\begin{lyxcode}%
a\_plot = plotColorVar(p\_stats, props)
%
\end{lyxcode}%
%
\item[Description:]%
Skips the 'ItemIndex' test.
%%
\item[Parameters:]~
\begin{description}%
\item[\texttt{p\_stats}:]
 Array of invariant parameter databases obtained from

calling tests\_3D\_db/paramsTestsHistsStats.
\item[\texttt{title\_str}:]
 (Optional) String to append to plot title.
\item[\texttt{props}:]
 A structure with any optional properties, passed to plot\_stack.
\begin{description}%
\item[\texttt{plotMethod}:]
 'plotVar' uses stats\_db/plotVar (default)

'plot\_bars' uses stats\_db/plot\_bars
\end{description}%
\end{description}%
%
\item[Returns:
]~

	a\_plot: A plot\_abstract with the color plot
%
%
\item[See also:]%
\hyperlink{ref_paramsTestsHistsStats}{\texttt{paramsTestsHistsStats}}%
\ (p.~\pageref{ref_paramsTestsHistsStats})%
\index[funcref]{paramsTestsHistsStats@\fidxl{paramsTestsHistsStats}}%
, \hyperlink{ref_params_tests_profile}{\texttt{params\_tests\_profile}}%
\ (p.~\pageref{ref_params_tests_profile})%
\index[funcref]{params_tests_profile@\fidxl{params\_tests\_profile}}%
, \hyperlink{ref_plotVar.}{\texttt{plotVar.}}%
\ (p.~\pageref{ref_plotVar.})%
\index[funcref]{plotVar.@\fidxl{plotVar.}}%
%
\item[Author:]%
Cengiz Gunay <cgunay@emory.edu>, 2004/10/17
%
\end{description}
\methodline%
\subsubsection[Method \texttt{plotVar}]{Method \texttt{stats\_db/plotVar}}%
\index[funcref]{stats_db@\fidxl{stats\_db}!plotVar@\fidxl{plotVar}}%
\label{ref_stats_db__plotVar}%
\hypertarget{ref_stats_db__plotVar}{}%
\begin{description}
\item[Summary:]Generates a plot of the variation between two tests.
%
\item[Usage:]~%
\begin{lyxcode}%
a\_plot = plotVar(a\_stats\_db, test1, test2, props)
%
\end{lyxcode}%
%
\item[Description:]%
Creates a plot description where the mean values are used for solid lines
 and the std values of test2 is indicated with errorbars. It is assumed that 
 each page of the stats\_db contains a value to be matched.
%%
\item[Parameters:]~
\begin{description}%
\item[\texttt{a\_stats\_db}:]
 A stats\_db object.
\item[\texttt{test1}:]
 Test column for the x-axis, only mean values are used.
\item[\texttt{test2}:]
 Test column for the y-axis, std values are indicated with errorbars.
\item[\texttt{title\_str}:]
 (Optional) String to append to plot title.
\item[\texttt{props}:]
 Optional properties.
\begin{description}%
\item[\texttt{plotType}:]
 1, only errorbars (default); 2, errorbars extending from bars.
\item[\texttt{quiet}:]
 If 1, only display given title\_str.

(rest passed to plot\_abstract)
\end{description}%
\end{description}%
%
\item[Returns:
]~

	a\_plot: A plot\_abstract object or one of its subclasses.
%
%
\item[See also:]%
\hyperlink{ref_plotVar}{\texttt{plotVar}}%
\ (p.~\pageref{ref_plotVar})%
\index[funcref]{plotVar@\fidxl{plotVar}}%
, \hyperlink{ref_plot_simple}{\texttt{plot\_simple}}%
\ (p.~\pageref{ref_plot_simple})%
\index[funcref]{plot_simple@\fidxl{plot\_simple}}%
%
\item[Author:]%
Cengiz Gunay <cgunay@emory.edu>, 2004/10/13
%
\end{description}
\methodline%
\subsubsection[Method \texttt{plotVarMatrix}]{Method \texttt{stats\_db/plotVarMatrix}}%
\index[funcref]{stats_db@\fidxl{stats\_db}!plotVarMatrix@\fidxl{plotVarMatrix}}%
\label{ref_stats_db__plotVarMatrix}%
\hypertarget{ref_stats_db__plotVarMatrix}{}%
\begin{description}
\item[Summary:]Create a stack of parameter-test variation plots organized in a matrix.
%
\item[Usage:]~%
\begin{lyxcode}%
a\_plot\_stack = plotVarMatrix(p\_stats, props)
%
\end{lyxcode}%
%
\item[Description:]%
Skips the 'ItemIndex' test.
%%
\item[Parameters:]~
\begin{description}%
\item[\texttt{p\_stats}:]
 Array of invariant parameter databases obtained from

calling tests\_3D\_db/paramsTestsHistsStats.
\item[\texttt{props}:]
 A structure with any optional properties, passed to plot\_stack.
\begin{description}%
\item[\texttt{plotMethod}:]
 'plotVar' uses stats\_db/plotVar (default)

'plot\_bars' uses stats\_db/plot\_bars
\item[\texttt{rotateYLabel}:]
 Rotate row labels this much (default=60).
\end{description}%
\end{description}%
%
\item[Returns:
]~

	a\_plot\_stack: A plot\_stack with the plots organized in matrix form
%
%
\item[See also:]%
\hyperlink{ref_paramsTestsHistsStats}{\texttt{paramsTestsHistsStats}}%
\ (p.~\pageref{ref_paramsTestsHistsStats})%
\index[funcref]{paramsTestsHistsStats@\fidxl{paramsTestsHistsStats}}%
, \hyperlink{ref_params_tests_profile}{\texttt{params\_tests\_profile}}%
\ (p.~\pageref{ref_params_tests_profile})%
\index[funcref]{params_tests_profile@\fidxl{params\_tests\_profile}}%
, \hyperlink{ref_plotVar.}{\texttt{plotVar.}}%
\ (p.~\pageref{ref_plotVar.})%
\index[funcref]{plotVar.@\fidxl{plotVar.}}%
%
\item[Author:]%
Cengiz Gunay <cgunay@emory.edu>, 2004/10/17
%
\end{description}
\methodline%
\subsubsection[Method \texttt{plotYTests}]{Method \texttt{stats\_db/plotYTests}}%
\index[funcref]{stats_db@\fidxl{stats\_db}!plotYTests@\fidxl{plotYTests}}%
\label{ref_stats_db__plotYTests}%
\hypertarget{ref_stats_db__plotYTests}{}%
\begin{description}
\item[Summary:]Create an errorbar plot of database stats measures against given X-axis values.
%
\item[Usage:]~%
\begin{lyxcode}%
a\_p = plotYTests(a\_stats\_db, x\_vals, tests, axis\_labels, title\_str, short\_title, command, props)
%
\end{lyxcode}%
%
%
\item[Parameters:]~
\begin{description}%
\item[\texttt{a\_stats\_db}:]
 A params\_tests\_db object.
\item[\texttt{x\_vals}:]
 A vector of X-axis values.
\item[\texttt{tests}:]
 A vector or cell array of columns to correspond to each value from x\_vals.
\item[\texttt{title\_str}:]
 (Optional) A string to be concatanated to the title.
\item[\texttt{short\_title}:]
 (Optional) Few words that may appear in legends of multiplot.
\item[\texttt{command}:]
 (Optional) Command to do the plotting with (default: 'plot')
\item[\texttt{props}:]
 A structure with any optional properties.
\begin{description}%
\item[\texttt{LineStyle}:]
 Plot line style to use. (default: 'd-')
\item[\texttt{quiet}:]
 If 1, don't include database name on title.
\end{description}%
\end{description}%
%
\item[Returns:
]~

	a\_p: A plot\_abstract.
%
\item[Example:]~
\begin{lyxcode} >> a\_p = plotYTests(a\_stats\_db, [0 40 100 200], ...
\\%
                      {'IniSpontSpikeRateISI\_0pA', 'PulseIni100msSpikeRateISI\_D40pA', ...
\\%
                       'PulseIni100msSpikeRateISI\_D100pA', 'PulseIni100msSpikeRateISI\_D200pA'}, ...
\\%
                      {'current pulse [pA]', 'firing rate [Hz]'}, ', f-I curves', 'neuron 1');
\\%
 >> plotFigure(a\_p);
\\%
\end{lyxcode}
%
\item[See also:]%
\hyperlink{ref_plotFigure}{\texttt{plotFigure}}%
\ (p.~\pageref{ref_plotFigure})%
\index[funcref]{plotFigure@\fidxl{plotFigure}}%
%
\item[Author:]%
Cengiz Gunay <cgunay@emory.edu>, 2006/01/23
%
\end{description}
\methodline%
\subsubsection[Method \texttt{plot\_abstract}]{Method \texttt{stats\_db/plot\_abstract}}%
\index[funcref]{stats_db@\fidxl{stats\_db}!plot_abstract@\fidxl{plot\_abstract}}%
\label{ref_stats_db__plot_abstract}%
\hypertarget{ref_stats_db__plot_abstract}{}%
\begin{description}
\item[Summary:]Generates an error bar graph for each db columns. 
%
\item[Usage:]~%
\begin{lyxcode}%
a\_plot = plot\_abstract(a\_stats\_db, title\_str, props)
%
\end{lyxcode}%
%
\item[Description:]%
Generates a plot\_simple object from this histogram. Looks for 'mean',
 'min', 'max', and 'STD' labels in the row\_idx for drawing the
 errorbars. Each column of a\_stats\_db is shown in a separate
 axis. Values from multiple pages of a\_stats\_db are shown as distinct
 points in the axis.
%%
\item[Parameters:]~
\begin{description}%
\item[\texttt{a\_stats\_db}:]
 A histogram\_db object.
\item[\texttt{title\_str}:]
 A title string on the plot
\item[\texttt{props}:]
 A structure with any optional properties.
\end{description}%
%
\item[Returns:
]~

	a\_plot: A object of plot\_abstract or one of its subclasses.
%
%
\item[See also:]%
\hyperlink{ref_plot_abstract}{\texttt{plot\_abstract}}%
\ (p.~\pageref{ref_plot_abstract})%
\index[funcref]{plot_abstract@\fidxl{plot\_abstract}}%
, \hyperlink{ref_plot_simple}{\texttt{plot\_simple}}%
\ (p.~\pageref{ref_plot_simple})%
\index[funcref]{plot_simple@\fidxl{plot\_simple}}%
%
\item[Author:]%
Cengiz Gunay <cgunay@emory.edu>, 2004/10/08
%
\end{description}
\methodline%
\subsubsection[Method \texttt{plot\_bars}]{Method \texttt{stats\_db/plot\_bars}}%
\index[funcref]{stats_db@\fidxl{stats\_db}!plot_bars@\fidxl{plot\_bars}}%
\label{ref_stats_db__plot_bars}%
\hypertarget{ref_stats_db__plot_bars}{}%
\begin{description}
\item[Summary:]Creates a bar graph with errorbars for each db column. 
%
\item[Usage:]~%
\begin{lyxcode}%
a\_plot = plot\_bars(a\_stats\_db, title\_str, props)
%
\end{lyxcode}%
%
\item[Description:]%
Looks for 'min', 'max', and 'STD' labels in the row\_idx for drawing the errorbars. 
 Each page of the DB will produce grouped bars.
%%
\item[Parameters:]~
\begin{description}%
\item[\texttt{a\_stats\_db}:]
 A stats\_db object.
\item[\texttt{title\_str}:]
 The plot title.
\item[\texttt{props}:]
 A structure with any optional properties.
\begin{description}%
\item[\texttt{pageVariable}:]
 The column used for denoting page values.
\item[\texttt{axis\_limits}:]
 Passed as argument to plot\_bars/plot\_bars.

(passed to plot\_bars/plot\_bars)
\end{description}%
\end{description}%
%
\item[Returns:
]~

	a\_plot: A object of plot\_bars or one of its subclasses.
%
%
\item[See also:]%
\hyperlink{ref_plot_abstract}{\texttt{plot\_abstract}}%
\ (p.~\pageref{ref_plot_abstract})%
\index[funcref]{plot_abstract@\fidxl{plot\_abstract}}%
, \hyperlink{ref_plot_bars__plot_bars}{\texttt{plot\_bars/plot\_bars}}%
\ (p.~\pageref{ref_plot_bars__plot_bars})%
\index[funcref]{plot_bars@\fidxl{plot\_bars}!plot_bars@\fidxl{plot\_bars}}%
%
\item[Author:]%
Cengiz Gunay <cgunay@emory.edu>, 2004/10/08
%
\end{description}
\methodline%
\subsubsection[Method \texttt{plot\_bars\_ax}]{Method \texttt{stats\_db/plot\_bars\_ax}}%
\index[funcref]{stats_db@\fidxl{stats\_db}!plot_bars_ax@\fidxl{plot\_bars\_ax}}%
\label{ref_stats_db__plot_bars_ax}%
\hypertarget{ref_stats_db__plot_bars_ax}{}%
\begin{description}
\item[Summary:]Bar plot with extending errorbars for all columns in the same axis.
%
\item[Usage:]~%
\begin{lyxcode}%
a\_plot = plot\_bars\_ax(a\_tests\_db, row, props)
%
\end{lyxcode}%
%
\item[Description:]%
Differs from stats\_db/plot\_bars because it does not open a new axis
 for each column. This is only suitable if all columns have similar extents.
%%
\item[Parameters:]~
\begin{description}%
\item[\texttt{a\_stats\_db}:]
 A stats\_db object.
\item[\texttt{title\_str}:]
 Optional title string.
\item[\texttt{props}:]
 A structure with any optional properties.
\begin{description}%
\item[\texttt{putLabels}:]
 Put special column name labels.
\end{description}%
\end{description}%
%
\item[Returns:
]~

	a\_plot: A plot\_abstract object that can be plotted.
%
%
\item[See also:]%
\hyperlink{ref_plot_abstract}{\texttt{plot\_abstract}}%
\ (p.~\pageref{ref_plot_abstract})%
\index[funcref]{plot_abstract@\fidxl{plot\_abstract}}%
, \hyperlink{ref_plotFigure}{\texttt{plotFigure}}%
\ (p.~\pageref{ref_plotFigure})%
\index[funcref]{plotFigure@\fidxl{plotFigure}}%
, \hyperlink{ref_stats_db__plot_bars}{\texttt{stats\_db/plot\_bars}}%
\ (p.~\pageref{ref_stats_db__plot_bars})%
\index[funcref]{stats_db@\fidxl{stats\_db}!plot_bars@\fidxl{plot\_bars}}%
%
\item[Author:]%
Cengiz Gunay <cgunay@emory.edu>, 2008/04/17
%
\end{description}
\methodline%
\subsubsection[Method \texttt{set}]{Method \texttt{stats\_db/set}}%
\index[funcref]{stats_db@\fidxl{stats\_db}!set@\fidxl{set}}%
\label{ref_stats_db__set}%
\hypertarget{ref_stats_db__set}{}%
\begin{description}
\item[Summary:]Generic method for setting object attributes.
%
%
%
%
%
%
%
\item[Author:]%
Cengiz Gunay <cgunay@emory.edu>, 2004/10/08
%
\end{description}
\methodline%
\subsubsection[Method \texttt{subsref}]{Method \texttt{stats\_db/subsref}}%
\index[funcref]{stats_db@\fidxl{stats\_db}!subsref@\fidxl{subsref}}%
\label{ref_stats_db__subsref}%
\hypertarget{ref_stats_db__subsref}{}%
\begin{description}
\item[Summary:]Defines generic indexing for objects.
%
%
%
%
%
%
%
%
\end{description}
\methodline%
\subsection{Class \texttt{tests\_3D\_db}}%
\index[funcref]{tests_3D_db@\fidxl{tests\_3D\_db}|boldhyperpage}%
\label{ref_tests_3D_db}%
\hypertarget{ref_tests_3D_db}{}%
\subsubsection[Constructor \texttt{tests\_3D\_db}]{Constructor \texttt{tests\_3D\_db/tests\_3D\_db}}%
\index[funcref]{tests_3D_db@\fidxl{tests\_3D\_db}!tests_3D_db@\fidxl{tests\_3D\_db}}%
\label{ref_tests_3D_db__tests_3D_db}%
\hypertarget{ref_tests_3D_db__tests_3D_db}{}%
\begin{description}
\item[Summary:]A database multiple pages with rows of test columns. 
		Each page may represent aspects of the data that are
		different, but not defined in this object.
%
\item[Usage:]~%
\begin{lyxcode}%
a\_3D\_db = tests\_3D\_db(data, col\_names, row\_names, page\_names, id, props)
%
\end{lyxcode}%
%
\item[Description:]%
This is a subclass of tests\_db. Usually it contains a RowIndex
 column that points to an original db from which this data originated. 
 The row indices can be used to reach the values associated with different
 pages of information contained in this object.
%%
\item[Parameters:]~
\begin{description}%
\item[\texttt{data}:]
 The 3-d vector of rows, columns, and pages.
\item[\texttt{col\_names}:]
 Colun names of the database.
\item[\texttt{id}:]
 An identifying string.
\item[\texttt{props}:]
 A structure with any optional properties.
\begin{description}%
\item[\texttt{invarName}:]
 Name of the invariant parameter for this db.
\end{description}%
\end{description}%
%
\item[Returns a structure object with the following fields:
]~

	tests\_db, page\_idx.
%
%
\item[See also:]%
\hyperlink{ref_tests_db}{\texttt{tests\_db}}%
\ (p.~\pageref{ref_tests_db})%
\index[funcref]{tests_db@\fidxl{tests\_db}}%
, \hyperlink{ref_tests_db__invarValues}{\texttt{tests\_db/invarValues}}%
\ (p.~\pageref{ref_tests_db__invarValues})%
\index[funcref]{tests_db@\fidxl{tests\_db}!invarValues@\fidxl{invarValues}}%
%
\item[Author:]%
Cengiz Gunay <cgunay@emory.edu>, 2004/09/30
%
\end{description}
\methodline%
\subsubsection[Method \texttt{addPages}]{Method \texttt{tests\_3D\_db/addPages}}%
\index[funcref]{tests_3D_db@\fidxl{tests\_3D\_db}!addPages@\fidxl{addPages}}%
\label{ref_tests_3D_db__addPages}%
\hypertarget{ref_tests_3D_db__addPages}{}%
\begin{description}
\item[Summary:]Inserts new pages (third dimension) to a tests\_3D\_db object.
%
%
\item[Description:]%
Adds new third dimension pages to the database and returns the new DB.
 Usage 2 concatanates two DBs pagewise. This operation is 
 expensive in the sense that the whole database matrix needs to be 
 enlarged just to add a single new page. The method of allocating
 a matrix, filling it up, and then providing it to the tests\_db 
 constructor is the preferred method of creating tests\_db objects. 
 This method may be used for measures obtained by operating on raw measures.
%%
\item[Parameters:]~
\begin{description}%
\item[\texttt{obj, b\_obj}:]
 A tests\_db object.
\item[\texttt{page\_names}:]
 A single string or a cell array of page names to be added.
\item[\texttt{page\_data}:]
 Data matrix of pages to be added.
\end{description}%
%
\item[Returns:
]~

   obj: The tests\_db object that includes the new columns.
%
%
\item[See also:]%
\hyperlink{ref_tests_db}{\texttt{tests\_db}}%
\ (p.~\pageref{ref_tests_db})%
\index[funcref]{tests_db@\fidxl{tests\_db}}%
%
\item[Author:]%
Cengiz Gunay <cengique@users.sf.net>, 2017/06/08
%
\end{description}
\methodline%
\subsubsection[Method \texttt{corrCoefs}]{Method \texttt{tests\_3D\_db/corrCoefs}}%
\index[funcref]{tests_3D_db@\fidxl{tests\_3D\_db}!corrCoefs@\fidxl{corrCoefs}}%
\label{ref_tests_3D_db__corrCoefs}%
\hypertarget{ref_tests_3D_db__corrCoefs}{}%
\begin{description}
\item[Summary:]Calculates correlation coefficients by comparing col1 with other cols. 
%
\item[Usage:]~%
\begin{lyxcode}%
a\_coefs\_db = corrCoefs(db, col1, cols, props)
%
\end{lyxcode}%
%
\item[Description:]%
If db has multiple pages, then each page in db produces a row of
 coefficients and matching PageIndex. Assuming the db was created with
 invarValues, this function finds the invariant correlation coefficients
 between its columns. The invariant correlation coefficients are the
 correlation of one column value with another column value when some other
 column values are fixed.  Since there are many occurences of the invariant
 coefficients, a histogram can then be created and returned from the
 created db. The other columns that are fixed are not in this db object,
 but can be reached using the indices to the original db. The page
 number is saved in the created db, so that it can be used to find the page
 from which the coefficient came. Then row indices of the page points to
 original constant column values.
%%
\item[Parameters:]~
\begin{description}%
\item[\texttt{db}:]
 A tests\_db object.
\item[\texttt{col1}:]
 Column to compare.
\item[\texttt{cols}:]
 Columns to be compared with col1.
\item[\texttt{props}:]
 A structure with any optional properties.
\begin{description}%
\item[\texttt{skipCoefs}:]
 If 1, coefficients of less confidence than %95 

will be skipped. (default=1)
\end{description}%
\end{description}%
%
\item[Returns:
]~

	a\_coefs\_db: A corrcoefs\_db of the coefficients and page indices.
%
%
\item[See also:]%
\hyperlink{ref_tests_db}{\texttt{tests\_db}}%
\ (p.~\pageref{ref_tests_db})%
\index[funcref]{tests_db@\fidxl{tests\_db}}%
, \hyperlink{ref_corrcoefs_db}{\texttt{corrcoefs\_db}}%
\ (p.~\pageref{ref_corrcoefs_db})%
\index[funcref]{corrcoefs_db@\fidxl{corrcoefs\_db}}%
%
\item[Author:]%
Cengiz Gunay <cgunay@emory.edu>, 2004/09/30
%
\end{description}
\methodline%
\subsubsection[Method \texttt{diff2D}]{Method \texttt{tests\_3D\_db/diff2D}}%
\index[funcref]{tests_3D_db@\fidxl{tests\_3D\_db}!diff2D@\fidxl{diff2D}}%
\label{ref_tests_3D_db__diff2D}%
\hypertarget{ref_tests_3D_db__diff2D}{}%
\begin{description}
\item[Summary:]Creates a tests\_db by taking the derivative of the given test.
%
\item[Usage:]~%
\begin{lyxcode}%
a\_tests\_db = diff2D(a\_db, test, props)
%
\end{lyxcode}%
%
\item[Description:]%
Applies the diff function to the chosen test, and collapses the middle
 dimension of the 3D DB to create a 2D DB and transposes it. The result is
 that the pages of the 3D DB becomes the rows of the new database, and the
 differenced rows appear as new columns, each named uniquely. The column
 index would correspons to the row index in the 3D DB. A new column
 'PageNumber' is appended to point back to the 3D DB.
%%
\item[Parameters:]~
\begin{description}%
\item[\texttt{a\_db}:]
 A tests\_3D\_db object.
\item[\texttt{test}:]
 Test column.
\item[\texttt{props}:]
 Optional properties.
\end{description}%
%
\item[Returns:
]~

	a\_tests\_db: A tests\_db that holds the requested differences of parameter values.
%
%
\item[See also:]%
\hyperlink{ref_boxplot}{\texttt{boxplot}}%
\ (p.~\pageref{ref_boxplot})%
\index[funcref]{boxplot@\fidxl{boxplot}}%
, \hyperlink{ref_plot_abstract}{\texttt{plot\_abstract}}%
\ (p.~\pageref{ref_plot_abstract})%
\index[funcref]{plot_abstract@\fidxl{plot\_abstract}}%
%
\item[Author:]%
Cengiz Gunay <cgunay@emory.edu>, 2005/05/22
%
\end{description}
\methodline%
\subsubsection[Method \texttt{display}]{Method \texttt{tests\_3D\_db/display}}%
\index[funcref]{tests_3D_db@\fidxl{tests\_3D\_db}!display@\fidxl{display}}%
\label{ref_tests_3D_db__display}%
\hypertarget{ref_tests_3D_db__display}{}%
\begin{description}
%
%
%
%
%
%
%
\item[Author:]%
Cengiz Gunay <cgunay@emory.edu>, 2004/08/04
%
\end{description}
\methodline%
\subsubsection[Method \texttt{flattenPages}]{Method \texttt{tests\_3D\_db/flattenPages}}%
\index[funcref]{tests_3D_db@\fidxl{tests\_3D\_db}!flattenPages@\fidxl{flattenPages}}%
\label{ref_tests_3D_db__flattenPages}%
\hypertarget{ref_tests_3D_db__flattenPages}{}%
\begin{description}
\item[Summary:]Convert from 3D to 2D by simply flattening pages.
%
\item[Usage:]~%
\begin{lyxcode}%
a\_db = flattenPages(a\_db)
%
\end{lyxcode}%
%
\item[Description:]%
Allows to get the original database after doing invarValues or
 invarParams and then using joinRows.
%%
\item[Parameters:]~
\begin{description}%
\item[\texttt{a\_db}:]
 A tests\_3D\_db object.
\end{description}%
%
\item[Returns:
]~

   a\_db: A tests\_db object.
%
%
\item[See also:]%
\hyperlink{ref_tests_db__invarValues}{\texttt{tests\_db/invarValues}}%
\ (p.~\pageref{ref_tests_db__invarValues})%
\index[funcref]{tests_db@\fidxl{tests\_db}!invarValues@\fidxl{invarValues}}%
, \hyperlink{ref_params_tests_db__invarParams}{\texttt{params\_tests\_db/invarParams}}%
\ (p.~\pageref{ref_params_tests_db__invarParams})%
\index[funcref]{params_tests_db@\fidxl{params\_tests\_db}!invarParams@\fidxl{invarParams}}%
, \hyperlink{ref_tests_db__joinRows}{\texttt{tests\_db/joinRows}}%
\ (p.~\pageref{ref_tests_db__joinRows})%
\index[funcref]{tests_db@\fidxl{tests\_db}!joinRows@\fidxl{joinRows}}%
%
\item[Author:]%
Cengiz Gunay <cgunay@emory.edu>, 2014/05/07
%
\end{description}
\methodline%
\subsubsection[Method \texttt{get}]{Method \texttt{tests\_3D\_db/get}}%
\index[funcref]{tests_3D_db@\fidxl{tests\_3D\_db}!get@\fidxl{get}}%
\label{ref_tests_3D_db__get}%
\hypertarget{ref_tests_3D_db__get}{}%
\begin{description}
\item[Summary:]Defines generic attribute retrieval for objects.
%
%
%
%
%
%
%
\item[Author:]%
Cengiz Gunay <cgunay@emory.edu>, 2004/09/14
%
\end{description}
\methodline%
\subsubsection[Method \texttt{histograms}]{Method \texttt{tests\_3D\_db/histograms}}%
\index[funcref]{tests_3D_db@\fidxl{tests\_3D\_db}!histograms@\fidxl{histograms}}%
\label{ref_tests_3D_db__histograms}%
\hypertarget{ref_tests_3D_db__histograms}{}%
\begin{description}
%
\item[Usage:]~%
\begin{lyxcode}%
a\_histogram\_db = histogram(db, col, num\_bins)
%
\end{lyxcode}%
%
\item[Description:]%
If one wants to get histograms of test values for each single value of
 the selected invariant parameter, then swapRowsPages should be done
 first on db.
%%
\item[Parameters:]~
\begin{description}%
\item[\texttt{db}:]
 A tests\_3D\_db object.
\item[\texttt{col}:]
 Column to find the histogram.
\item[\texttt{num\_bins}:]
 Number of histogram bins (Optional, default=100)
\end{description}%
%
\item[Returns:
]~

	a\_histogram\_db: A histogram\_db object containing the histogram.
%
%
\item[See also:]%
\hyperlink{ref_histogram_db}{\texttt{histogram\_db}}%
\ (p.~\pageref{ref_histogram_db})%
\index[funcref]{histogram_db@\fidxl{histogram\_db}}%
, \hyperlink{ref_tests_db}{\texttt{tests\_db}}%
\ (p.~\pageref{ref_tests_db})%
\index[funcref]{tests_db@\fidxl{tests\_db}}%
%
\item[Author:]%
Cengiz Gunay <cgunay@emory.edu>, 2004/10/04
%
\end{description}
\methodline%
\subsubsection[Method \texttt{joinPages}]{Method \texttt{tests\_3D\_db/joinPages}}%
\index[funcref]{tests_3D_db@\fidxl{tests\_3D\_db}!joinPages@\fidxl{joinPages}}%
\label{ref_tests_3D_db__joinPages}%
\hypertarget{ref_tests_3D_db__joinPages}{}%
\begin{description}
\item[Summary:]Joins the rows of the given db to the with\_db rows matching with the PageIndex
 	column.
%
\item[Usage:]~%
\begin{lyxcode}%
a\_db = joinPages(db, with\_db)
%
\end{lyxcode}%
%
\item[Description:]%
Replicates the desired columns in the with\_db with rows having a 
 page index and joins them next to desired columns from the current 3D\_db. Flattens 
 the resulting 3D\_db to become a 2D db. Assumes each page index only 
 appears once in with\_db.
%%
\item[Parameters:]~
\begin{description}%
\item[\texttt{db}:]
 A tests\_3D\_db object.
\item[\texttt{with\_db}:]
 A tests\_db object with a PageIndex column.
\end{description}%
%
\item[Returns:
]~

	a\_db: A tests\_db object.
%
%
\item[See also:]%
\hyperlink{ref_tests_db}{\texttt{tests\_db}}%
\ (p.~\pageref{ref_tests_db})%
\index[funcref]{tests_db@\fidxl{tests\_db}}%
%
\item[Author:]%
Cengiz Gunay <cgunay@emory.edu>, 2004/10/15
%
\end{description}
\methodline%
\subsubsection[Method \texttt{mergePages}]{Method \texttt{tests\_3D\_db/mergePages}}%
\index[funcref]{tests_3D_db@\fidxl{tests\_3D\_db}!mergePages@\fidxl{mergePages}}%
\label{ref_tests_3D_db__mergePages}%
\hypertarget{ref_tests_3D_db__mergePages}{}%
\begin{description}
\item[Summary:]Merges tests from separate pages into a 2D params\_tests\_db.
%
\item[Usage:]~%
\begin{lyxcode}%
a\_db = mergePages(db, page\_tests, page\_suffixes)
%
\end{lyxcode}%
%
\item[Description:]%
Keeps uniqueness by adding suffixes to test names.
 If you're using invarParams, do swapRowsPages, then use joinRows with original db to get
 the parameter values.
%%
\item[Parameters:]~
\begin{description}%
\item[\texttt{db}:]
 A tests\_3D\_db object.
\item[\texttt{page\_tests}:]
 Cell array of list of tests to take from each page.
\item[\texttt{page\_suffixes}:]
 Cell array of suffixes to append to tests from each page.
\end{description}%
%
\item[Returns:
]~

	a\_db: A tests\_db object.
%
%
\item[See also:]%
\hyperlink{ref_tests_db}{\texttt{tests\_db}}%
\ (p.~\pageref{ref_tests_db})%
\index[funcref]{tests_db@\fidxl{tests\_db}}%
, \hyperlink{ref_tests_3D_db}{\texttt{tests\_3D\_db}}%
\ (p.~\pageref{ref_tests_3D_db})%
\index[funcref]{tests_3D_db@\fidxl{tests\_3D\_db}}%
, \hyperlink{ref_tests_db__joinRows}{\texttt{tests\_db/joinRows}}%
\ (p.~\pageref{ref_tests_db__joinRows})%
\index[funcref]{tests_db@\fidxl{tests\_db}!joinRows@\fidxl{joinRows}}%
%
\item[Author:]%
Cengiz Gunay <cgunay@emory.edu>, 2005/01/13
%
\end{description}
\methodline%
\subsubsection[Method \texttt{onlyRowsTests}]{Method \texttt{tests\_3D\_db/onlyRowsTests}}%
\index[funcref]{tests_3D_db@\fidxl{tests\_3D\_db}!onlyRowsTests@\fidxl{onlyRowsTests}}%
\label{ref_tests_3D_db__onlyRowsTests}%
\hypertarget{ref_tests_3D_db__onlyRowsTests}{}%
\begin{description}
\item[Summary:]Returns a tests\_db that only contains the desired 
		tests and rows (and pages).
%
\item[Usage:]~%
\begin{lyxcode}%
obj = onlyRowsTests(obj, rows, tests, pages)
%
\end{lyxcode}%
%
\item[Description:]%
Selects the given dimensions and returns in a new tests\_db object.
%%
\item[Parameters:]~
\begin{description}%
\item[\texttt{obj}:]
 A tests\_db object.
\item[\texttt{rows, tests}:]
 A logical or index vector of rows, or cell array of

names of rows. If ':', all rows. For names, regular expressions are
supported if quoted with slashes (e.g., '/a.*/'). See tests2idx.
\item[\texttt{pages}:]
 (Optional) A logical or index vector of pages. ':' for all pages.
\end{description}%
%
\item[Returns:
]~

	obj: The new tests\_db object.
%
%
\item[See also:]%
\hyperlink{ref_subsref}{\texttt{subsref}}%
\ (p.~\pageref{ref_subsref})%
\index[funcref]{subsref@\fidxl{subsref}}%
, \hyperlink{ref_tests_db}{\texttt{tests\_db}}%
\ (p.~\pageref{ref_tests_db})%
\index[funcref]{tests_db@\fidxl{tests\_db}}%
, \hyperlink{ref_tests2idx}{\texttt{tests2idx}}%
\ (p.~\pageref{ref_tests2idx})%
\index[funcref]{tests2idx@\fidxl{tests2idx}}%
, \hyperlink{ref_regexp}{\texttt{regexp}}%
\ (p.~\pageref{ref_regexp})%
\index[funcref]{regexp@\fidxl{regexp}}%
%
\item[Author:]%
Cengiz Gunay <cgunay@emory.edu>, 2004/09/17
%
\end{description}
\methodline%
\subsubsection[Method \texttt{paramsTestsHistsStats}]{Method \texttt{tests\_3D\_db/paramsTestsHistsStats}}%
\index[funcref]{tests_3D_db@\fidxl{tests\_3D\_db}!paramsTestsHistsStats@\fidxl{paramsTestsHistsStats}}%
\label{ref_tests_3D_db__paramsTestsHistsStats}%
\hypertarget{ref_tests_3D_db__paramsTestsHistsStats}{}%
\begin{description}
\item[Summary:]Calculates histograms and statistics for DB.
%
\item[Usage:]~%
\begin{lyxcode}%
[pt\_hists, p\_stats] = paramsTestsHistsStats(p\_t3ds, props)
%
\end{lyxcode}%
%
\item[Description:]%
Calculates histograms and statistics for all combinations of tests 
 and params and returns them in a cell array. Skips the 'ItemIndex' test.
%%
\item[Parameters:]~
\begin{description}%
\item[\texttt{p\_t3ds}:]
 Array of invariant parameter databases obtained by

calling the params\_tests\_db/invarParams method.
\item[\texttt{props}:]
 Optional properties.
\begin{description}%
\item[\texttt{statsMethod}:]
 method to call to get a stats\_db (default='statsMeanSE')
\item[\texttt{useDiff}:]
 If 1, takes the derivative with diff on the 3D DBs (default=0).
\end{description}%
\end{description}%
%
\item[Returns:
]~

	pt\_hists: An array of 3D histograms for each pair of param 
		  and test.
	p\_stats: An array of stats\_dbs for each param.
%
%
\item[See also:]%
\hyperlink{ref_invarParams}{\texttt{invarParams}}%
\ (p.~\pageref{ref_invarParams})%
\index[funcref]{invarParams@\fidxl{invarParams}}%
, \hyperlink{ref_params_tests_profile}{\texttt{params\_tests\_profile}}%
\ (p.~\pageref{ref_params_tests_profile})%
\index[funcref]{params_tests_profile@\fidxl{params\_tests\_profile}}%
%
\item[Author:]%
Cengiz Gunay <cgunay@emory.edu>, 2004/10/17
%
\end{description}
\methodline%
\subsubsection[Method \texttt{plotParamPairImage}]{Method \texttt{tests\_3D\_db/plotParamPairImage}}%
\index[funcref]{tests_3D_db@\fidxl{tests\_3D\_db}!plotParamPairImage@\fidxl{plotParamPairImage}}%
\label{ref_tests_3D_db__plotParamPairImage}%
\hypertarget{ref_tests_3D_db__plotParamPairImage}{}%
\begin{description}
\item[Summary:]Generates an image plot of variation of a test with two parameters in the first page.
%
\item[Usage:]~%
\begin{lyxcode}%
a\_plot = plotParamPairImage(a\_db, test, title\_str, props)
%
\end{lyxcode}%
%
\item[Description:]%
It is assumed that the 3D DB is created by invariant combinations of two parameters,
 which are the first two columns. Each page of the db must contain a same parameter 
 values. This is the default character of tests\_3D\_db created by 
 params\_tests\_db/invarParam. Parameter values will be enumerated and then an 
 image plot is created.
%%
\item[Parameters:]~
\begin{description}%
\item[\texttt{a\_db}:]
 A tests\_3D\_db object.
\item[\texttt{test}:]
 Test column to take the measure value.
\item[\texttt{title\_str}:]
 (Optional) String to append to plot title.
\item[\texttt{props}:]
 Optional properties to be passed to plot\_abstract.
\begin{description}%
\item[\texttt{truncateDecDigits}:]
 Truncate labels to this many decimal digits.
\item[\texttt{labelSteps}:]
 Skip this many labels between ticks to reduce to total number.
\item[\texttt{maxValue}:]
 Maximal value to normalize colors and to annotate the colorbar.
\end{description}%
\end{description}%
%
\item[Returns:
]~

	a\_plot: A plot\_abstract object or one of its subclasses.
%
\item[Example:]~
\begin{lyxcode} Find relationship of two parameters against a measure:
\\%
 >> plotFigure(plotParamPairImage(invarParam(a\_db, {'NaF', 'KCNQ'}), 'PulseIni100msRest2SpikeRateISI\_D100pA'));
\\%
\end{lyxcode}
%
\item[See also:]%
\hyperlink{ref_params_tests_db__invarParam}{\texttt{params\_tests\_db/invarParam}}%
\ (p.~\pageref{ref_params_tests_db__invarParam})%
\index[funcref]{params_tests_db@\fidxl{params\_tests\_db}!invarParam@\fidxl{invarParam}}%
, \hyperlink{ref_plotImage}{\texttt{plotImage}}%
\ (p.~\pageref{ref_plotImage})%
\index[funcref]{plotImage@\fidxl{plotImage}}%
, \hyperlink{ref_plot_abstract.}{\texttt{plot\_abstract.}}%
\ (p.~\pageref{ref_plot_abstract.})%
\index[funcref]{plot_abstract.@\fidxl{plot\_abstract.}}%
%
\item[Author:]%
Cengiz Gunay <cgunay@emory.edu>, 2004/11/10
%
\end{description}
\methodline%
\subsubsection[Method \texttt{plotScatter}]{Method \texttt{tests\_3D\_db/plotScatter}}%
\index[funcref]{tests_3D_db@\fidxl{tests\_3D\_db}!plotScatter@\fidxl{plotScatter}}%
\label{ref_tests_3D_db__plotScatter}%
\hypertarget{ref_tests_3D_db__plotScatter}{}%
\begin{description}
\item[Summary:]Superpose scatter plots for each page of the database of the given two tests.
%
\item[Usage:]~%
\begin{lyxcode}%
a\_p = plotScatter(a\_db, test1, test2, title\_str, short\_title, props)
%
\end{lyxcode}%
%
\item[Description:]%
If 'warning on verbose' is issued before this, it will display
 regression statistics: R\textasciicircum{}2, F, p, and the error variance.
%%
\item[Parameters:]~
\begin{description}%
\item[\texttt{a\_db}:]
 A tests\_3D\_db object.
\item[\texttt{test1, test2}:]
 X \& Y variables.
\item[\texttt{title\_str}:]
 (Optional) A string to be concatenated to the title.
\item[\texttt{short\_title}:]
 (Optional) Few words that may appear in legends of multiplot.
\item[\texttt{props}:]
 A structure with any optional properties.
\begin{description}%
\item[\texttt{LineStyle}:]
 Plot line style to use. (default: 'x')
\item[\texttt{Regress}:]
 If exists, use these props for plotting the linear regression.
\item[\texttt{quiet}:]
 If 1, don't include database name on title.

(all passed to tests\_db/plotScatter)
\end{description}%
\end{description}%
%
\item[Returns:
]~

	a\_p: A plot\_abstract.
%
%
\item[See also:]%
\hyperlink{ref_tests_db__plotScatter}{\texttt{tests\_db/plotScatter}}%
\ (p.~\pageref{ref_tests_db__plotScatter})%
\index[funcref]{tests_db@\fidxl{tests\_db}!plotScatter@\fidxl{plotScatter}}%
%
\item[Author:]%
Cengiz Gunay <cgunay@emory.edu>, 2005/09/29
%
\end{description}
\methodline%
\subsubsection[Method \texttt{plotVarBox}]{Method \texttt{tests\_3D\_db/plotVarBox}}%
\index[funcref]{tests_3D_db@\fidxl{tests\_3D\_db}!plotVarBox@\fidxl{plotVarBox}}%
\label{ref_tests_3D_db__plotVarBox}%
\hypertarget{ref_tests_3D_db__plotVarBox}{}%
\begin{description}
\item[Summary:]Generates a boxplot of the variation between two tests.
%
\item[Usage:]~%
\begin{lyxcode}%
a\_plot = plotVarBox(a\_db, test1, test2, notch, sym, vert, whis, props)
%
\end{lyxcode}%
%
\item[Description:]%
It is assumed that each page of the db contains a different parameter value.
%%
\item[Parameters:]~
\begin{description}%
\item[\texttt{a\_db}:]
 A tests\_3D\_db object.
\item[\texttt{test1}:]
 Test column for the x-axis, only mean values are used.
\item[\texttt{test2}:]
 Test column for the y-axis, used for boxplot.
\item[\texttt{notch, sym, vert, whis}:]
 See boxplot, defaults = (1, '+', 1, 1.5).
\item[\texttt{props}:]
 Optional properties to be passed to plot\_abstract.
\end{description}%
%
\item[Returns:
]~

	a\_plot: A plot\_abstract object or one of its subclasses.
%
%
\item[See also:]%
\hyperlink{ref_boxplot}{\texttt{boxplot}}%
\ (p.~\pageref{ref_boxplot})%
\index[funcref]{boxplot@\fidxl{boxplot}}%
, \hyperlink{ref_plot_abstract}{\texttt{plot\_abstract}}%
\ (p.~\pageref{ref_plot_abstract})%
\index[funcref]{plot_abstract@\fidxl{plot\_abstract}}%
%
\item[Author:]%
Cengiz Gunay <cgunay@emory.edu>, 2004/11/10
%
\end{description}
\methodline%
\subsubsection[Method \texttt{renamePages}]{Method \texttt{tests\_3D\_db/renamePages}}%
\index[funcref]{tests_3D_db@\fidxl{tests\_3D\_db}!renamePages@\fidxl{renamePages}}%
\label{ref_tests_3D_db__renamePages}%
\hypertarget{ref_tests_3D_db__renamePages}{}%
\begin{description}
\item[Summary:]Rename one or more existing pages.
%
\item[Usage:]~%
\begin{lyxcode}%
a\_db = renamePages(a\_db, old\_names, new\_names)
%
\end{lyxcode}%
%
\item[Description:]%
This is a cheap operation than modifies meta-data kept in object. For
 the regular expression renaming, the old\_names and new\_names
 parameters are passed to the regexprep command after removing the
 delimiting slashes (//). At least one grouping construct ('()') must be
 used in the search pattern such that it can be used in the replacement
 pattern (e.g., '\$1'). See example above. This function uses the generic
 renameIdx that can work on row, column, or page indices.
%%
\item[Parameters:]~
\begin{description}%
\item[\texttt{a\_db}:]
 A tests\_db object.
\item[\texttt{old\_names}:]
 A cell array of existing names, array of numerical indices, or a regular

expression denoted between slashes (e.g., '/(.*)/').
\item[\texttt{new\_names}:]
 New names to replace existing ones OR regular expression

replace string (no slashes, e.g, '\$1\_test'). See regexprep command.
\end{description}%
%
\item[Returns:
]~

   a\_db: The tests\_db object that includes the new pages.
%
\item[Example:]~
\begin{lyxcode} % Renaming a single page:
\\%
 >> new\_db = renamePages(a\_db, 'PulseIni100msSpikeRateISI\_D40pA', 'Firing\_rate');
\\%
 % Renaming an unnamed page:
\\%
 >> new\_db = renamePages(a\_db, 1, 'Firing\_rate');
\\%
 % Renaming using regular expressions: add suffix to all pages
\\%
 >> new\_db = renamePages(a\_db, '/(.*)/', '\$1\_old');
\\%
 % Renaming multiple pages:
\\%
 >> new\_db = renamePages(a\_db, {'a', 'b'}, {'c', 'd'});
\\%
\end{lyxcode}
%
\item[See also:]%
\hyperlink{ref_renameIdx}{\texttt{renameIdx}}%
\ (p.~\pageref{ref_renameIdx})%
\index[funcref]{renameIdx@\fidxl{renameIdx}}%
, \hyperlink{ref_regexprep}{\texttt{regexprep}}%
\ (p.~\pageref{ref_regexprep})%
\index[funcref]{regexprep@\fidxl{regexprep}}%
, \hyperlink{ref_allocateRows}{\texttt{allocateRows}}%
\ (p.~\pageref{ref_allocateRows})%
\index[funcref]{allocateRows@\fidxl{allocateRows}}%
, \hyperlink{ref_tests_db}{\texttt{tests\_db}}%
\ (p.~\pageref{ref_tests_db})%
\index[funcref]{tests_db@\fidxl{tests\_db}}%
%
\item[Author:]%
Cengiz Gunay <cgunay@emory.edu>, 2017/06/09
%
\end{description}
\methodline%
\subsubsection[Method \texttt{set}]{Method \texttt{tests\_3D\_db/set}}%
\index[funcref]{tests_3D_db@\fidxl{tests\_3D\_db}!set@\fidxl{set}}%
\label{ref_tests_3D_db__set}%
\hypertarget{ref_tests_3D_db__set}{}%
\begin{description}
\item[Summary:]Generic method for setting object attributes.
%
%
%
%
%
%
%
\item[Author:]%
Cengiz Gunay <cgunay@emory.edu>, 2004/10/08
%
\end{description}
\methodline%
\subsubsection[Method \texttt{swapColsPages}]{Method \texttt{tests\_3D\_db/swapColsPages}}%
\index[funcref]{tests_3D_db@\fidxl{tests\_3D\_db}!swapColsPages@\fidxl{swapColsPages}}%
\label{ref_tests_3D_db__swapColsPages}%
\hypertarget{ref_tests_3D_db__swapColsPages}{}%
\begin{description}
\item[Summary:]Swaps the column dimension with the page dimension of the
		  tests\_3D\_db.
%
\item[Usage:]~%
\begin{lyxcode}%
a\_3D\_db = swapColsPages(db)
%
\end{lyxcode}%
%
%
\item[Parameters:]~
\begin{description}%
\item[\texttt{db}:]
 A tests\_db object.
\end{description}%
%
\item[Returns:
]~

	a\_3D\_db: A tests\_3D\_db object.
%
%
\item[See also:]%
\hyperlink{ref_tests_db}{\texttt{tests\_db}}%
\ (p.~\pageref{ref_tests_db})%
\index[funcref]{tests_db@\fidxl{tests\_db}}%
%
\item[Author:]%
Cengiz Gunay <cgunay@emory.edu>, 2017/06/09
%
\end{description}
\methodline%
\subsubsection[Method \texttt{swapRowsPages}]{Method \texttt{tests\_3D\_db/swapRowsPages}}%
\index[funcref]{tests_3D_db@\fidxl{tests\_3D\_db}!swapRowsPages@\fidxl{swapRowsPages}}%
\label{ref_tests_3D_db__swapRowsPages}%
\hypertarget{ref_tests_3D_db__swapRowsPages}{}%
\begin{description}
\item[Summary:]Swaps the row dimension with the page dimension of the
		  tests\_3D\_db.
%
\item[Usage:]~%
\begin{lyxcode}%
a\_3D\_db = swapRowsPages(db)
%
\end{lyxcode}%
%
\item[Description:]%
Assuming that this is a invariant parameter and tests relations db, this
 function transposes the data matrix by swapping the pages with rows. Each
 resulting page correspond to a single value of the chosen parameter, with
 each row containing a test result with different combinations of the rest
 of the parameters.
%%
\item[Parameters:]~
\begin{description}%
\item[\texttt{db}:]
 A tests\_db object.
\end{description}%
%
\item[Returns:
]~

	a\_3D\_db: A tests\_3D\_db object.
%
%
\item[See also:]%
\hyperlink{ref_tests_db}{\texttt{tests\_db}}%
\ (p.~\pageref{ref_tests_db})%
\index[funcref]{tests_db@\fidxl{tests\_db}}%
%
\item[Author:]%
Cengiz Gunay <cgunay@emory.edu>, 2004/10/04
%
\end{description}
\methodline%
\subsection{Class \texttt{tests\_db}}%
\index[funcref]{tests_db@\fidxl{tests\_db}|boldhyperpage}%
\label{ref_tests_db}%
\hypertarget{ref_tests_db}{}%
\subsubsection[Constructor \texttt{tests\_db}]{Constructor \texttt{tests\_db/tests\_db}}%
\index[funcref]{tests_db@\fidxl{tests\_db}!tests_db@\fidxl{tests\_db}}%
\label{ref_tests_db__tests_db}%
\hypertarget{ref_tests_db__tests_db}{}%
\begin{description}
\item[Summary:]Construct a numeric database organized in a matrix format.
%
\item[Usage:]~%
\begin{lyxcode}%
obj = tests\_db(test\_results, col\_names, row\_names, id, props)
%
\end{lyxcode}%
%
\item[Description:]%
This is the base database class. Note for loading text files:
 Matlab's dlmread commands is used, and it is unable to handle files that
 have any non-numeric data (except skipped rows). Therefore, those files
 are best filtered with outside tools before importing.
%%
\item[Parameters:]~
\begin{description}%
\item[\texttt{test\_results}:]
 Either a text file (e.g., CSV) name or a matrix that contains

measurement columns and separate observations as rows.
\item[\texttt{col\_names}:]
 Cell array of column names of test\_results.
\item[\texttt{row\_names}:]
 Cell array of row names of test\_results.
\item[\texttt{id}:]
 An identifying string.
\item[\texttt{props}:]
 A structure with any optional properties.
\begin{description}%
\item[\texttt{textDelim}:]
 Delimiter in text file to be used by dlmread (Default:

',' for CSV). Use '' for all whitespace, '$\backslash$t' for tabs.
\item[\texttt{csvArgs}:]
 Cell array of arguments passed to dlmread function (e.g.,

{R, C, [r1 c1 r2 c2]}) for (R)ow and (C)olumn offset/range to read.
\item[\texttt{csvReadColNames}:]
 If 1, first row of the file or at the given offset

(see csvArgs) is used to read column names. Double quotes must be
used consistently.
\item[\texttt{paramDescFile}:]
 Load parameter names from file (one line per name).
\end{description}%
\end{description}%
%
\item[Returns a structure object with the following fields:
]~

   data: The data matrix.
   row\_idx, col\_idx: Structure associating row/column names to indices.
   id, props.
%
%
\item[See also:]%
\hyperlink{ref_params_tests_db}{\texttt{params\_tests\_db}}%
\ (p.~\pageref{ref_params_tests_db})%
\index[funcref]{params_tests_db@\fidxl{params\_tests\_db}}%
, \hyperlink{ref_dlmread}{\texttt{dlmread}}%
\ (p.~\pageref{ref_dlmread})%
\index[funcref]{dlmread@\fidxl{dlmread}}%
%
\item[Author:]%
Cengiz Gunay <cgunay@emory.edu>, 2004/09/01
%
\end{description}
\methodline%
\subsubsection[Method \texttt{abs}]{Method \texttt{tests\_db/abs}}%
\index[funcref]{tests_db@\fidxl{tests\_db}!abs@\fidxl{abs}}%
\label{ref_tests_db__abs}%
\hypertarget{ref_tests_db__abs}{}%
\begin{description}
\item[Summary:]Take absolute value of all db elements.
%
\item[Usage:]~%
\begin{lyxcode}%
a\_db = abs(left\_obj)
%
\end{lyxcode}%
%
%
\item[Parameters:]~
\begin{description}%
\item[\texttt{left\_obj}:]
 A tests\_db object.
\end{description}%
%
\item[Returns:
]~

   a\_db: The resulting tests\_db.
%
%
\item[See also:]%
\hyperlink{ref_abs}{\texttt{abs}}%
\ (p.~\pageref{ref_abs})%
\index[funcref]{abs@\fidxl{abs}}%
, \hyperlink{ref_uop}{\texttt{uop}}%
\ (p.~\pageref{ref_uop})%
\index[funcref]{uop@\fidxl{uop}}%
%
\item[Author:]%
Cengiz Gunay <cgunay@emory.edu>, 2008/01/16
%
\end{description}
\methodline%
\subsubsection[Method \texttt{addColumns}]{Method \texttt{tests\_db/addColumns}}%
\index[funcref]{tests_db@\fidxl{tests\_db}!addColumns@\fidxl{addColumns}}%
\label{ref_tests_db__addColumns}%
\hypertarget{ref_tests_db__addColumns}{}%
\begin{description}
\item[Summary:]Inserts new columns to tests\_db.
%
%
\item[Description:]%
Adds new test columns to the database and returns the new DB.
 Usage 2 concatanates two DBs columnwise. This operation is 
 expensive in the sense that the whole database matrix needs to be 
 enlarged just to add a single new column. The method of allocating
 a matrix, filling it up, and then providing it to the tests\_db 
 constructor is the preferred method of creating tests\_db objects. 
 This method may be used for measures obtained by operating on raw measures.
%%
\item[Parameters:]~
\begin{description}%
\item[\texttt{obj, b\_obj}:]
 A tests\_db object.
\item[\texttt{test\_names}:]
 A single string or a cell array of test names to be added.
\item[\texttt{test\_columns}:]
 Data matrix of columns to be added.
\end{description}%
%
\item[Returns:
]~

   obj: The tests\_db object that includes the new columns.
%
%
\item[See also:]%
\hyperlink{ref_allocateRows}{\texttt{allocateRows}}%
\ (p.~\pageref{ref_allocateRows})%
\index[funcref]{allocateRows@\fidxl{allocateRows}}%
, \hyperlink{ref_tests_db}{\texttt{tests\_db}}%
\ (p.~\pageref{ref_tests_db})%
\index[funcref]{tests_db@\fidxl{tests\_db}}%
%
\item[Author:]%
Cengiz Gunay <cgunay@emory.edu>, 2005/09/30
%
\end{description}
\methodline%
\subsubsection[Method \texttt{addLastRow}]{Method \texttt{tests\_db/addLastRow}}%
\index[funcref]{tests_db@\fidxl{tests\_db}!addLastRow@\fidxl{addLastRow}}%
\label{ref_tests_db__addLastRow}%
\hypertarget{ref_tests_db__addLastRow}{}%
\begin{description}
\item[Summary:]Inserts a row of observations at the end of tests\_db.
%
\item[Usage:]~%
\begin{lyxcode}%
index = addLastRow(obj, row)
%
\end{lyxcode}%
%
\item[Description:]%
Adds a new set of observations to the database and returns its row index.
   This operation is expensive because the whole 
   database matrix needs to be duplicated and resized in order to add a 
   single new row. The method of allocating a matrix, filling it up, and
   then providing it to the tests\_db constructor is the preferred method 
   of creating tests\_db objects.
%%
\item[Parameters:]~
\begin{description}%
\item[\texttt{obj}:]
 A tests\_db object.
\item[\texttt{row}:]
 A row vector that contains values for each DB column.
\end{description}%
%
\item[Returns:
]~

	obj: The tests\_db object that includes the new row.
%
%
\item[See also:]%
\hyperlink{ref_allocateRows}{\texttt{allocateRows}}%
\ (p.~\pageref{ref_allocateRows})%
\index[funcref]{allocateRows@\fidxl{allocateRows}}%
, \hyperlink{ref_addRow}{\texttt{addRow}}%
\ (p.~\pageref{ref_addRow})%
\index[funcref]{addRow@\fidxl{addRow}}%
, \hyperlink{ref_tests_db}{\texttt{tests\_db}}%
\ (p.~\pageref{ref_tests_db})%
\index[funcref]{tests_db@\fidxl{tests\_db}}%
%
\item[Author:]%
Cengiz Gunay <cgunay@emory.edu>, 2004/09/08
%
\end{description}
\methodline%
\subsubsection[Method \texttt{addPages}]{Method \texttt{tests\_db/addPages}}%
\index[funcref]{tests_db@\fidxl{tests\_db}!addPages@\fidxl{addPages}}%
\label{ref_tests_db__addPages}%
\hypertarget{ref_tests_db__addPages}{}%
\begin{description}
\item[Summary:]Inserts new pages (third dimension) to a tests\_db object.
%
%
\item[Description:]%
Adds new third dimension pages to the database and returns the new
 DB. Page names are not maintained in tests\_db; use tests\_3D\_db instead.
 Usage 2 concatanates two DBs pagewise. This operation is expensive in the
 sense that the whole database matrix needs to be enlarged just to add a
 single new page. The method of allocating a matrix, filling it up, and
 then providing it to the tests\_db constructor is the preferred method of
 creating tests\_db objects.  This method may be used for measures obtained
 by operating on raw measures.
%%
\item[Parameters:]~
\begin{description}%
\item[\texttt{obj, b\_obj}:]
 A tests\_db object.
\item[\texttt{page\_names}:]
 A single string or a cell array of page names to be

added. IGNORED in tests\_db.
\item[\texttt{page\_data}:]
 Data matrix of pages to be added.
\end{description}%
%
\item[Returns:
]~

   obj: The tests\_db object that includes the new columns.
%
%
\item[See also:]%
\hyperlink{ref_tests_db}{\texttt{tests\_db}}%
\ (p.~\pageref{ref_tests_db})%
\index[funcref]{tests_db@\fidxl{tests\_db}}%
%
\item[Author:]%
Cengiz Gunay <cengique@users.sf.net>, 2017/06/08
%
\end{description}
\methodline%
\subsubsection[Method \texttt{addRow}]{Method \texttt{tests\_db/addRow}}%
\index[funcref]{tests_db@\fidxl{tests\_db}!addRow@\fidxl{addRow}}%
\label{ref_tests_db__addRow}%
\hypertarget{ref_tests_db__addRow}{}%
\begin{description}
\item[Summary:]Inserts a row of observations to tests\_db at the given row index.
%
\item[Usage:]~%
\begin{lyxcode}%
index = addRow(obj, row, index)
%
\end{lyxcode}%
%
\item[Description:]%
Adds a new set of observations to the database and returns the new DB.
   This operation is expensive in the sense that the whole database matrix
   needs to be copied to be passed to this function just to add a 
   single new row. The method of allocating a matrix, filling it up, and
   then providing it to the tests\_db constructor is the preferred method 
   of creating tests\_db objects.
%%
\item[Parameters:]~
\begin{description}%
\item[\texttt{obj}:]
 A tests\_db object.
\item[\texttt{row}:]
 A row vector that contains values for each DB column.
\item[\texttt{index}:]
 The row index.
\end{description}%
%
\item[Returns:
]~

	obj: The tests\_db object that includes the new row.
%
%
\item[See also:]%
\hyperlink{ref_addLastRow}{\texttt{addLastRow}}%
\ (p.~\pageref{ref_addLastRow})%
\index[funcref]{addLastRow@\fidxl{addLastRow}}%
, \hyperlink{ref_allocateRows}{\texttt{allocateRows}}%
\ (p.~\pageref{ref_allocateRows})%
\index[funcref]{allocateRows@\fidxl{allocateRows}}%
, \hyperlink{ref_tests_db}{\texttt{tests\_db}}%
\ (p.~\pageref{ref_tests_db})%
\index[funcref]{tests_db@\fidxl{tests\_db}}%
%
\item[Author:]%
Cengiz Gunay <cgunay@emory.edu>, 2004/09/08
%
\end{description}
\methodline%
\subsubsection[Method \texttt{allocateRows}]{Method \texttt{tests\_db/allocateRows}}%
\index[funcref]{tests_db@\fidxl{tests\_db}!allocateRows@\fidxl{allocateRows}}%
\label{ref_tests_db__allocateRows}%
\hypertarget{ref_tests_db__allocateRows}{}%
\begin{description}
\item[Summary:]Preallocates a NaN-filled num\_rows rows in tests\_db.
%
\item[Usage:]~%
\begin{lyxcode}%
obj = allocateRows(obj, num\_rows)
%
\end{lyxcode}%
%
\item[Description:]%
Overwrites the database by allocating a new matrix of the desired number
 of rows to speed up filling up the data matrix using
 assignRowsTests. Using addRow after this operation is still expensive.
 The method of allocating a matrix, filling it up, and then providing it to
 the tests\_db constructor is the preferred method of creating tests\_db
 objects.
%%
\item[Parameters:]~
\begin{description}%
\item[\texttt{obj}:]
 A tests\_db object.
\item[\texttt{num\_rows}:]
 The predicted number of observations for this tests\_db.
\end{description}%
%
\item[Returns:
]~

	obj: The new tests\_db object.
%
%
\item[See also:]%
\hyperlink{ref_assignRowsTests}{\texttt{assignRowsTests}}%
\ (p.~\pageref{ref_assignRowsTests})%
\index[funcref]{assignRowsTests@\fidxl{assignRowsTests}}%
, \hyperlink{ref_addRow}{\texttt{addRow}}%
\ (p.~\pageref{ref_addRow})%
\index[funcref]{addRow@\fidxl{addRow}}%
, \hyperlink{ref_setRows}{\texttt{setRows}}%
\ (p.~\pageref{ref_setRows})%
\index[funcref]{setRows@\fidxl{setRows}}%
, \hyperlink{ref_tests_db}{\texttt{tests\_db}}%
\ (p.~\pageref{ref_tests_db})%
\index[funcref]{tests_db@\fidxl{tests\_db}}%
%
\item[Author:]%
Cengiz Gunay <cgunay@emory.edu>, 2004/09/08
%
\end{description}
\methodline%
\subsubsection[Method \texttt{anyRows}]{Method \texttt{tests\_db/anyRows}}%
\index[funcref]{tests_db@\fidxl{tests\_db}!anyRows@\fidxl{anyRows}}%
\label{ref_tests_db__anyRows}%
\hypertarget{ref_tests_db__anyRows}{}%
\begin{description}
\item[Summary:]Returns db rows matching any of the given rows.
%
\item[Usage:]~%
\begin{lyxcode}%
idx = anyRows(db, rows)
%
\end{lyxcode}%
%
\item[Description:]%
The db rows are compared to each row and row indices succeeding any of
 these comparisons are returned.
%%
\item[Parameters:]~
\begin{description}%
\item[\texttt{db}:]
 A tests\_db object.
\item[\texttt{rows}:]
 Row array, matrix or database to be compared with db rows.
\end{description}%
%
\item[Returns:
]~

	idx: A logical column vector of matching db row indices. 
	rows\_idx: Indices of rows entries corresponding to each db
		row. Non-matching entries were left as NaN.
%
\item[Example:]~
\begin{lyxcode}  >> db(anyRows(db(:, 'trial'), [12; 46; 37]), :)
\\%
 returns a db with rows having trial equal to any of the given values.
\\%
\end{lyxcode}
%
\item[See also:]%
\hyperlink{ref_compareRows}{\texttt{compareRows}}%
\ (p.~\pageref{ref_compareRows})%
\index[funcref]{compareRows@\fidxl{compareRows}}%
, \hyperlink{ref_eq}{\texttt{eq}}%
\ (p.~\pageref{ref_eq})%
\index[funcref]{eq@\fidxl{eq}}%
, \hyperlink{ref_tests_db}{\texttt{tests\_db}}%
\ (p.~\pageref{ref_tests_db})%
\index[funcref]{tests_db@\fidxl{tests\_db}}%
%
\item[Author:]%
Cengiz Gunay <cgunay@emory.edu>, 2004/09/17
%
\end{description}
\methodline%
\subsubsection[Method \texttt{approxMappingLIBSVM}]{Method \texttt{tests\_db/approxMappingLIBSVM}}%
\index[funcref]{tests_db@\fidxl{tests\_db}!approxMappingLIBSVM@\fidxl{approxMappingLIBSVM}}%
\label{ref_tests_db__approxMappingLIBSVM}%
\hypertarget{ref_tests_db__approxMappingLIBSVM}{}%
\begin{description}
\item[Summary:]Approximates the desired input-output mapping using a support vector machine (SVM).
%
\item[Usage:]~%
\begin{lyxcode}%
[an\_approx\_db, an\_svm] = approxMappingLIBSVM(a\_db, input\_cols, output\_cols, props)
%
\end{lyxcode}%
%
\item[Description:]%
Uses the LIBSVM package (http://www.csie.ntu.edu.tw/~cjlin/libsvm/). If
 'warning on verbose' is issued prior to running, it provides additional
 debug info.
%%
\item[Parameters:]~
\begin{description}%
\item[\texttt{a\_db}:]
 A tests\_db object.
\item[\texttt{input\_cols, output\_cols}:]
 Input and output columns to be mapped

(see tests2cols for accept column specifications).
\item[\texttt{props}:]
 A structure with any optional properties.
\begin{description}%
\item[\texttt{classProbs}:]
 'prob': use probabilistic sampling to normalize

prior class probabilities.
\item[\texttt{kernel}:]
 Kernel type (default='poly').
\item[\texttt{crossFold}:]
 n to use in libsvm's n-fold cross-validation (default=0;

disabled).
\item[\texttt{testControl}:]
 Ratio of dataset to train the data and rest to test for success

(default=0; disabled).
\item[\texttt{svmCost}:]
 Add the -c option to svmOpts with this value.
\item[\texttt{svmGamma}:]
 Add the -g option to svmOpts with this value.
\item[\texttt{svmOpts}:]
 Passed to LIBSVM overwriting default options (see output

of svm-train for options; default='-s0 -t2 -d2').
(Rest passed to balanceInputProbs and tests\_db)
\end{description}%
\end{description}%
%
\item[Returns:
]~

	an\_approx\_db: A tests\_db object containing the original inputs and
			the approximated outputs.
	an\_svm: The Matlab neural network approximator object.
%
\item[Example:]~
\begin{lyxcode} >> [a\_class\_db, an\_svm = approxMappingLIBSVM(my\_db, {'NaF', 'Kv3'}, {'spike\_width'});
\\%
 >> plotFigure(plot\_superpose({plotScatter(my\_db, 'NaF', 'spike\_width'),
\\%
                                plotScatter(a\_class\_db, 'NaF', 'spike\_width')}))
\\%
\end{lyxcode}
%
\item[See also:]%
\hyperlink{ref_tests_db}{\texttt{tests\_db}}%
\ (p.~\pageref{ref_tests_db})%
\index[funcref]{tests_db@\fidxl{tests\_db}}%
, \hyperlink{ref_newff}{\texttt{newff}}%
\ (p.~\pageref{ref_newff})%
\index[funcref]{newff@\fidxl{newff}}%
%
\item[Author:]%
Cengiz Gunay <cgunay@emory.edu>, 2007/12/12
%
\end{description}
\methodline%
\subsubsection[Method \texttt{approxMappingNNet}]{Method \texttt{tests\_db/approxMappingNNet}}%
\index[funcref]{tests_db@\fidxl{tests\_db}!approxMappingNNet@\fidxl{approxMappingNNet}}%
\label{ref_tests_db__approxMappingNNet}%
\hypertarget{ref_tests_db__approxMappingNNet}{}%
\begin{description}
\item[Summary:]Approximates the desired input-output mapping using a Matlab neural network.
%
\item[Usage:]~%
\begin{lyxcode}%
[an\_approx\_db, a\_nnet] = approxMappingNNet(a\_db, input\_cols, output\_cols, props)
%
\end{lyxcode}%
%
\item[Description:]%
Approximates the mapping between the given inputs to outputs
 using the Matlab Neural Network Toolbox. By default it creates a
 feed-forward network to be trained with a Levenberg-Marquardt training
 algorithm (see newff). Returns and the trained network object and a
 database with output columns obtained from the approximator. The outputs
 can then be compared to the original database to test the success of the
 approximation. If 'warning on verbose' is issued prior to running, it
 provides additional debug info.
%%
\item[Parameters:]~
\begin{description}%
\item[\texttt{a\_db}:]
 A tests\_db object.
\item[\texttt{input\_cols, output\_cols}:]
 Input and output columns to be mapped

(see tests2cols for accept column specifications).
\item[\texttt{props}:]
 A structure with any optional properties.
\begin{description}%
\item[\texttt{nnetFcn}:]
 Neural network classifier function (default='newff')
\item[\texttt{nnetParams}:]
 Cell array of parameters passed to nnetFcn after

inputs and outputs.
\item[\texttt{trainMode}:]
 'batch' or 'incr'.
\item[\texttt{testControl}:]
 Ratio of dataset to train the data and rest to test for success

(default=0; disabled).
\item[\texttt{classProbs}:]
 'prob': use probabilistic sampling to normalize

prior class probabilities.
\item[\texttt{maxEpochs}:]
 maximum number of epochs to train for.

(Rest passed to balanceInputProbs and tests\_db)
\end{description}%
\end{description}%
%
\item[Returns:
]~

	an\_approx\_db: A tests\_db object containing the original inputs and
			the approximated outputs.
	a\_nnet: The Matlab neural network approximator object.
%
\item[Example:]~
\begin{lyxcode} >> [a\_class\_db, a\_nnet = approxMappingNNet(my\_db, {'NaF', 'Kv3'}, {'spike\_width'});
\\%
 >> plotFigure(plot\_superpose({plotScatter(my\_db, 'NaF', 'spike\_width'),
\\%
                                plotScatter(a\_class\_db, 'NaF', 'spike\_width')}))
\\%
\end{lyxcode}
%
\item[See also:]%
\hyperlink{ref_tests_db}{\texttt{tests\_db}}%
\ (p.~\pageref{ref_tests_db})%
\index[funcref]{tests_db@\fidxl{tests\_db}}%
, \hyperlink{ref_newff}{\texttt{newff}}%
\ (p.~\pageref{ref_newff})%
\index[funcref]{newff@\fidxl{newff}}%
%
\item[Author:]%
Cengiz Gunay <cgunay@emory.edu>, 2007/12/12
%
\end{description}
\methodline%
\subsubsection[Method \texttt{approxMappingSVM}]{Method \texttt{tests\_db/approxMappingSVM}}%
\index[funcref]{tests_db@\fidxl{tests\_db}!approxMappingSVM@\fidxl{approxMappingSVM}}%
\label{ref_tests_db__approxMappingSVM}%
\hypertarget{ref_tests_db__approxMappingSVM}{}%
\begin{description}
\item[Summary:]Approximates the desired input-output mapping using a support vector machine (SVM).
%
\item[Usage:]~%
\begin{lyxcode}%
[an\_approx\_db, an\_svm] = approxMappingSVM(a\_db, input\_cols, output\_cols, props)
%
\end{lyxcode}%
%
\item[Description:]%
Uses the SVM-KM package
 (http://asi.insa-rouen.fr/enseignants/~arakotom/toolbox/index.html). If
 'warning on verbose' is issued prior to running, it provides additional
 debug info.
%%
\item[Parameters:]~
\begin{description}%
\item[\texttt{a\_db}:]
 A tests\_db object.
\item[\texttt{input\_cols, output\_cols}:]
 Input and output columns to be mapped

(see tests2cols for accept column specifications).
\item[\texttt{props}:]
 A structure with any optional properties.
\begin{description}%
\item[\texttt{classProbs}:]
 'prob': use probabilistic sampling to normalize

prior class probabilities.
\item[\texttt{kernel}:]
 Kernel type (default='poly').

(Rest passed to balanceInputProbs and tests\_db)
\end{description}%
\end{description}%
%
\item[Returns:
]~

	an\_approx\_db: A tests\_db object containing the original inputs and
			the approximated outputs.
	an\_svm: The Matlab neural network approximator object.
%
\item[Example:]~
\begin{lyxcode} >> [a\_class\_db, an\_svm = approxMappingSVM(my\_db, {'NaF', 'Kv3'}, {'spike\_width'});
\\%
 >> plotFigure(plot\_superpose({plotScatter(my\_db, 'NaF', 'spike\_width'),
\\%
                                plotScatter(a\_class\_db, 'NaF', 'spike\_width')}))
\\%
\end{lyxcode}
%
\item[See also:]%
\hyperlink{ref_tests_db}{\texttt{tests\_db}}%
\ (p.~\pageref{ref_tests_db})%
\index[funcref]{tests_db@\fidxl{tests\_db}}%
, \hyperlink{ref_newff}{\texttt{newff}}%
\ (p.~\pageref{ref_newff})%
\index[funcref]{newff@\fidxl{newff}}%
%
\item[Author:]%
Cengiz Gunay <cgunay@emory.edu>, 2007/12/12
%
\end{description}
\methodline%
\subsubsection[Method \texttt{assignRowsTests}]{Method \texttt{tests\_db/assignRowsTests}}%
\index[funcref]{tests_db@\fidxl{tests\_db}!assignRowsTests@\fidxl{assignRowsTests}}%
\label{ref_tests_db__assignRowsTests}%
\hypertarget{ref_tests_db__assignRowsTests}{}%
\begin{description}
\item[Summary:]Assign the values to the tests and rows (and pages) of the tests\_db.
%
\item[Usage:]~%
\begin{lyxcode}%
obj = assignRowsTests(obj, val, rows, tests, pages)
%
\end{lyxcode}%
%
\item[Description:]%
Selects the given dimensions and returns in a new tests\_db object.
%%
\item[Parameters:]~
\begin{description}%
\item[\texttt{obj}:]
 A tests\_db object.
\item[\texttt{val}:]
 DB object or data matrix to be assigned to the addressed indices.
\item[\texttt{rows}:]
 A logical or index vector of rows. If ':', all rows.
\item[\texttt{tests}:]
 Cell array of test names or column indices. If ':', all tests.
\item[\texttt{pages}:]
 (Optional) A logical or index vector of pages. ':' for all pages.
\end{description}%
%
\item[Returns:
]~

	obj: The new tests\_db object.
%
%
\item[See also:]%
\hyperlink{ref_subsref}{\texttt{subsref}}%
\ (p.~\pageref{ref_subsref})%
\index[funcref]{subsref@\fidxl{subsref}}%
, \hyperlink{ref_tests_db}{\texttt{tests\_db}}%
\ (p.~\pageref{ref_tests_db})%
\index[funcref]{tests_db@\fidxl{tests\_db}}%
%
\item[Author:]%
Cengiz Gunay <cgunay@emory.edu>, 2006/02/08
%
\end{description}
\methodline%
\subsubsection[Method \texttt{checkConsistentCols}]{Method \texttt{tests\_db/checkConsistentCols}}%
\index[funcref]{tests_db@\fidxl{tests\_db}!checkConsistentCols@\fidxl{checkConsistentCols}}%
\label{ref_tests_db__checkConsistentCols}%
\hypertarget{ref_tests_db__checkConsistentCols}{}%
\begin{description}
\item[Summary:]Check if two DBs have exactly the same columns.
%
\item[Usage:]~%
\begin{lyxcode}%
[col\_names, with\_col\_names] = checkConsistentCols(db, with\_db, props)
%
\end{lyxcode}%
%
%
\item[Parameters:]~
\begin{description}%
\item[\texttt{db}:]
 A tests\_db object.
\item[\texttt{with\_db}:]
 A tests\_db object whose column names are checked for consistency.
\item[\texttt{props}:]
 A structure with any optional properties.
\begin{description}%
\item[\texttt{useCommon}:]
 Tolerate mismatching column names and only return

the common columns.
\end{description}%
\end{description}%
%
\item[Returns:
]~

	col\_names, with\_col\_names: list of column names of each DB.
%
%
\item[See also:]%
\hyperlink{ref_vertcat}{\texttt{vertcat}}%
\ (p.~\pageref{ref_vertcat})%
\index[funcref]{vertcat@\fidxl{vertcat}}%
, \hyperlink{ref_tests_db}{\texttt{tests\_db}}%
\ (p.~\pageref{ref_tests_db})%
\index[funcref]{tests_db@\fidxl{tests\_db}}%
%
\item[Author:]%
Cengiz Gunay <cgunay@emory.edu>, 2007/01/18
%
\end{description}
\methodline%
\subsubsection[Method \texttt{compareRows}]{Method \texttt{tests\_db/compareRows}}%
\index[funcref]{tests_db@\fidxl{tests\_db}!compareRows@\fidxl{compareRows}}%
\label{ref_tests_db__compareRows}%
\hypertarget{ref_tests_db__compareRows}{}%
\begin{description}
\item[Summary:]Returns differing rows of db and the given row(s).
%
\item[Usage:]~%
\begin{lyxcode}%
[idx, compared] = compareRows(db, rows)
%
\end{lyxcode}%
%
\item[Description:]%
It can compare all db rows to corresponding row entries or to a single
 row. For the case with only one entry, returns all db rows that do not
 match the given row in idx, and the result of the differences in
 compared. For the case of multiple rows, rows must have exactly same number
 of rows with db. In both cases, idx must be negated to test for equality.
%%
\item[Parameters:]~
\begin{description}%
\item[\texttt{db}:]
 A tests\_db object.
\item[\texttt{rows}:]
 Row array, matrix or database to be compared with db rows.
\end{description}%
%
\item[Returns:
]~

	idx: A inverted logical column vector of comparison results. 
		(false if db == rows, true otherwise)
	compared: A column vector of differences of each DB row to the
 		given row (i.e., compared = db - rows).
%
\item[Example:]~
\begin{lyxcode}  >> db(db(:, 'trial') > [12]), :)
\\%
 calls gt which calls compareRows to check for equality. Returns a db
\\%
 only containing rows with trial numbers greater than 12.
\\%
\end{lyxcode}
%
\item[See also:]%
\hyperlink{ref_eq}{\texttt{eq}}%
\ (p.~\pageref{ref_eq})%
\index[funcref]{eq@\fidxl{eq}}%
, \hyperlink{ref_anyRows}{\texttt{anyRows}}%
\ (p.~\pageref{ref_anyRows})%
\index[funcref]{anyRows@\fidxl{anyRows}}%
, \hyperlink{ref_ge}{\texttt{ge}}%
\ (p.~\pageref{ref_ge})%
\index[funcref]{ge@\fidxl{ge}}%
, \hyperlink{ref_gt}{\texttt{gt}}%
\ (p.~\pageref{ref_gt})%
\index[funcref]{gt@\fidxl{gt}}%
, \hyperlink{ref_le}{\texttt{le}}%
\ (p.~\pageref{ref_le})%
\index[funcref]{le@\fidxl{le}}%
, \hyperlink{ref_lt}{\texttt{lt}}%
\ (p.~\pageref{ref_lt})%
\index[funcref]{lt@\fidxl{lt}}%
, \hyperlink{ref_tests_db}{\texttt{tests\_db}}%
\ (p.~\pageref{ref_tests_db})%
\index[funcref]{tests_db@\fidxl{tests\_db}}%
%
\item[Author:]%
Cengiz Gunay <cgunay@emory.edu>, 2004/09/17
%
\end{description}
\methodline%
\subsubsection[Method \texttt{corrcoef}]{Method \texttt{tests\_db/corrcoef}}%
\index[funcref]{tests_db@\fidxl{tests\_db}!corrcoef@\fidxl{corrcoef}}%
\label{ref_tests_db__corrcoef}%
\hypertarget{ref_tests_db__corrcoef}{}%
\begin{description}
\item[Summary:]Calculates a correlation coefficient matrix by comparing cols.
%
\item[Usage:]~%
\begin{lyxcode}%
a\_coefs\_db = corrcoef(db, cols, props)
%
\end{lyxcode}%
%
%
\item[Parameters:]~
\begin{description}%
\item[\texttt{db}:]
 A tests\_db object.
\item[\texttt{cols}:]
 Columns to be compared.
\item[\texttt{props}:]
 A structure with any optional properties.
\begin{description}%
\item[\texttt{skipCoefs}:]
 If 1, coefficients of less confidence than %95 

will be skipped. (default=1)
\item[\texttt{alpha}:]
 Skip coefs with p values lower than this (default=0.05).
\item[\texttt{partialCols}:]
 Columns to calculate partial correlations by

controlling for the other columns.
\item[\texttt{bonfer}:]
 Bonferroni correction to alpha value.
\item[\texttt{resample}:]
 Shuffle columns and resample correlations this many times to get

statistics for the null hypothesis.
\end{description}%
\end{description}%
%
\item[Returns:
]~

	a\_coefs\_db: A tests\_3D\_db of the coefficient matrix, and their
		upper/lower limits on different pages.
%
%
\item[See also:]%
\hyperlink{ref_tests_db}{\texttt{tests\_db}}%
\ (p.~\pageref{ref_tests_db})%
\index[funcref]{tests_db@\fidxl{tests\_db}}%
, \hyperlink{ref_corrcoefs_db}{\texttt{corrcoefs\_db}}%
\ (p.~\pageref{ref_corrcoefs_db})%
\index[funcref]{corrcoefs_db@\fidxl{corrcoefs\_db}}%
, \hyperlink{ref_corrcoef}{\texttt{corrcoef}}%
\ (p.~\pageref{ref_corrcoef})%
\index[funcref]{corrcoef@\fidxl{corrcoef}}%
, \hyperlink{ref_partialcorr}{\texttt{partialcorr}}%
\ (p.~\pageref{ref_partialcorr})%
\index[funcref]{partialcorr@\fidxl{partialcorr}}%
%
\item[Author:]%
Cengiz Gunay <cgunay@emory.edu>, 2008/04/25
%
\end{description}
\methodline%
\subsubsection[Method \texttt{cov}]{Method \texttt{tests\_db/cov}}%
\index[funcref]{tests_db@\fidxl{tests\_db}!cov@\fidxl{cov}}%
\label{ref_tests_db__cov}%
\hypertarget{ref_tests_db__cov}{}%
\begin{description}
\item[Summary:]Generates a database of the covariance of given DB.
%
\item[Usage:]~%
\begin{lyxcode}%
a\_cov\_db = cov(db, props)
%
\end{lyxcode}%
%
%
\item[Parameters:]~
\begin{description}%
\item[\texttt{db}:]
 A tests\_db object.
\item[\texttt{props}:]
 A structure with any optional properties.
\begin{description}%
\item[\texttt{keepOrigDB}:]
 Keep db as origDB in the props.

(others passed to tests\_db)
\end{description}%
\end{description}%
%
\item[Returns:
]~

	a\_cov\_db: A tests\_db which contains the covariance matrix.
%
%
\item[See also:]%
\hyperlink{ref_cov}{\texttt{cov}}%
\ (p.~\pageref{ref_cov})%
\index[funcref]{cov@\fidxl{cov}}%
%
\item[Author:]%
Cengiz Gunay <cgunay@emory.edu>, 2007/05/25
%
\end{description}
\methodline%
\subsubsection[Method \texttt{crossProd}]{Method \texttt{tests\_db/crossProd}}%
\index[funcref]{tests_db@\fidxl{tests\_db}!crossProd@\fidxl{crossProd}}%
\label{ref_tests_db__crossProd}%
\hypertarget{ref_tests_db__crossProd}{}%
\begin{description}
\item[Summary:]Create a DB by taking the cross product of two database row sets.
%
\item[Usage:]~%
\begin{lyxcode}%
cross\_db = crossProd(a\_db, b\_db)
%
\end{lyxcode}%
%
\item[Description:]%
This is not a vector cross product operation. Each row of the two DBs are matched 
 and added as a new row to a DB. The end is a DB with all combinations 
 of rows from both DBs. The final DB contains columns of both DBs.
%%
\item[Parameters:]~
\begin{description}%
\item[\texttt{a\_db, b\_db}:]
 A tests\_db object.
\end{description}%
%
\item[Returns:
]~

	cross\_db: The tests\_db object with all combinations of rows.
%
%
\item[See also:]%
\hyperlink{ref_allocateRows}{\texttt{allocateRows}}%
\ (p.~\pageref{ref_allocateRows})%
\index[funcref]{allocateRows@\fidxl{allocateRows}}%
, \hyperlink{ref_tests_db}{\texttt{tests\_db}}%
\ (p.~\pageref{ref_tests_db})%
\index[funcref]{tests_db@\fidxl{tests\_db}}%
%
\item[Author:]%
Cengiz Gunay <cgunay@emory.edu>, 2005/10/11
%
\end{description}
\methodline%
\subsubsection[Method \texttt{dbsize}]{Method \texttt{tests\_db/dbsize}}%
\index[funcref]{tests_db@\fidxl{tests\_db}!dbsize@\fidxl{dbsize}}%
\label{ref_tests_db__dbsize}%
\hypertarget{ref_tests_db__dbsize}{}%
\begin{description}
\item[Summary:]Returns the size of the data matrix of db.
%
\item[Usage:]~%
\begin{lyxcode}%
s = dbsize(db)
%
\end{lyxcode}%
%
%
\item[Parameters:]~
\begin{description}%
\item[\texttt{db}:]
 A tests\_db object.
\end{description}%
%
\item[Returns:
]~

	s: The size values.
%
%
\item[See also:]%
\hyperlink{ref_size}{\texttt{size}}%
\ (p.~\pageref{ref_size})%
\index[funcref]{size@\fidxl{size}}%
, \hyperlink{ref_tests_db}{\texttt{tests\_db}}%
\ (p.~\pageref{ref_tests_db})%
\index[funcref]{tests_db@\fidxl{tests\_db}}%
%
\item[Author:]%
Cengiz Gunay <cgunay@emory.edu>, 2004/10/06
%
\end{description}
\methodline%
\subsubsection[Method \texttt{delColumns}]{Method \texttt{tests\_db/delColumns}}%
\index[funcref]{tests_db@\fidxl{tests\_db}!delColumns@\fidxl{delColumns}}%
\label{ref_tests_db__delColumns}%
\hypertarget{ref_tests_db__delColumns}{}%
\begin{description}
\item[Summary:]Deletes columns from tests\_db.
%
\item[Usage:]~%
\begin{lyxcode}%
obj = delColumns(obj, tests)
%
\end{lyxcode}%
%
\item[Description:]%
Deletes test columns from the database and returns the new DB.
   This operation is expensive in the sense that the whole database matrix
   needs to be copied just to delete a 
   single column. The method of allocating a matrix, filling it up, and
   then providing it to the tests\_db constructor is the preferred method 
   of creating tests\_db objects. This method may be used for 
   measures obtained by operating on raw measures.
%%
\item[Parameters:]~
\begin{description}%
\item[\texttt{obj}:]
 A tests\_db object.
\item[\texttt{tests}:]
 Numbers or names of tests (see tests2cols)
\end{description}%
%
\item[Returns:
]~

	obj: The tests\_db object that is missing the columns.
%
%
\item[See also:]%
\hyperlink{ref_allocateRows}{\texttt{allocateRows}}%
\ (p.~\pageref{ref_allocateRows})%
\index[funcref]{allocateRows@\fidxl{allocateRows}}%
, \hyperlink{ref_tests_db}{\texttt{tests\_db}}%
\ (p.~\pageref{ref_tests_db})%
\index[funcref]{tests_db@\fidxl{tests\_db}}%
%
\item[Author:]%
Cengiz Gunay <cgunay@emory.edu>, 2005/10/06
%
\end{description}
\methodline%
\subsubsection[Method \texttt{diff}]{Method \texttt{tests\_db/diff}}%
\index[funcref]{tests_db@\fidxl{tests\_db}!diff@\fidxl{diff}}%
\label{ref_tests_db__diff}%
\hypertarget{ref_tests_db__diff}{}%
\begin{description}
\item[Summary:]Creates a tests\_db by taking the derivative of all tests.
%
\item[Usage:]~%
\begin{lyxcode}%
a\_db = diff(a\_db, props)
%
\end{lyxcode}%
%
\item[Description:]%
Applies the diff function to whole DB. The resulting DB will have one less row.
%%
\item[Parameters:]~
\begin{description}%
\item[\texttt{a\_db}:]
 A tests\_db object.
\item[\texttt{props}:]
 Optional properties.
\end{description}%
%
\item[Returns:
]~

	a\_db: The resulting tests\_db.
%
%
\item[See also:]%
\hyperlink{ref_diff}{\texttt{diff}}%
\ (p.~\pageref{ref_diff})%
\index[funcref]{diff@\fidxl{diff}}%
, \hyperlink{ref_tests_3D_db__getDiff2DDB}{\texttt{tests\_3D\_db/getDiff2DDB}}%
\ (p.~\pageref{ref_tests_3D_db__getDiff2DDB})%
\index[funcref]{tests_3D_db@\fidxl{tests\_3D\_db}!getDiff2DDB@\fidxl{getDiff2DDB}}%
%
\item[Author:]%
Cengiz Gunay <cgunay@emory.edu>, 2006/05/24
%
\end{description}
\methodline%
\subsubsection[Method \texttt{display}]{Method \texttt{tests\_db/display}}%
\index[funcref]{tests_db@\fidxl{tests\_db}!display@\fidxl{display}}%
\label{ref_tests_db__display}%
\hypertarget{ref_tests_db__display}{}%
\begin{description}
%
%
%
%
%
%
%
\item[Author:]%
Cengiz Gunay <cgunay@emory.edu>, 2004/08/04
%
\end{description}
\methodline%
\subsubsection[Method \texttt{displayRows}]{Method \texttt{tests\_db/displayRows}}%
\index[funcref]{tests_db@\fidxl{tests\_db}!displayRows@\fidxl{displayRows}}%
\label{ref_tests_db__displayRows}%
\hypertarget{ref_tests_db__displayRows}{}%
\begin{description}
\item[Summary:]Displays rows of data with associated column labels.
%
\item[Usage:]~%
\begin{lyxcode}%
s = displayRows(db, rows, pages)
%
\end{lyxcode}%
%
\item[Description:]%
Use transpose() on db to rotate display.
%%
\item[Parameters:]~
\begin{description}%
\item[\texttt{db}:]
 A tests\_db object.
\item[\texttt{rows}:]
 Indices of rows in db.
\item[\texttt{pages}:]
 Pages of db.
\end{description}%
%
\item[Returns:
]~

   s: A cell array of trasposed database contents, prefixed with 
	column names on each row. Meant to be displayed on the screen.
%
%
\item[See also:]%
\hyperlink{ref_tests_db}{\texttt{tests\_db}}%
\ (p.~\pageref{ref_tests_db})%
\index[funcref]{tests_db@\fidxl{tests\_db}}%
%
\item[Author:]%
Cengiz Gunay <cgunay@emory.edu>, 2004/09/15
%
\end{description}
\methodline%
\subsubsection[Method \texttt{displayRowsCSV}]{Method \texttt{tests\_db/displayRowsCSV}}%
\index[funcref]{tests_db@\fidxl{tests\_db}!displayRowsCSV@\fidxl{displayRowsCSV}}%
\label{ref_tests_db__displayRowsCSV}%
\hypertarget{ref_tests_db__displayRowsCSV}{}%
\begin{description}
\item[Summary:]Returns a comma-separated values (CSV) version of the table.
%
\item[Usage:]~%
\begin{lyxcode}%
csv\_string = displayRowsCSV(a\_db, props)
%
\end{lyxcode}%
%
\item[Description:]%
Uses displayRows. See its documentation for details.
%%
\item[Parameters:]~
\begin{description}%
\item[\texttt{a\_db}:]
 A tests\_db object.
\item[\texttt{props}:]
 A structure with any optional properties.
\end{description}%
%
\item[Returns:
]~

   csv\_string: String that can be saved as  a file or copy-pasted into other software.
%
\item[Example:]~
\begin{lyxcode} >> string2File(displayRowsCSV(a\_db(1:10, 4:7)), 'excel-export.csv')
\\%
\end{lyxcode}
%
\item[See also:]%
\hyperlink{ref_displayRows}{\texttt{displayRows}}%
\ (p.~\pageref{ref_displayRows})%
\index[funcref]{displayRows@\fidxl{displayRows}}%
, \hyperlink{ref_displayRowsTeX}{\texttt{displayRowsTeX}}%
\ (p.~\pageref{ref_displayRowsTeX})%
\index[funcref]{displayRowsTeX@\fidxl{displayRowsTeX}}%
%
\item[Author:]%
Cengiz Gunay <cgunay@emory.edu>, 2011/07/07
%
\end{description}
\methodline%
\subsubsection[Method \texttt{displayRowsTeX}]{Method \texttt{tests\_db/displayRowsTeX}}%
\index[funcref]{tests_db@\fidxl{tests\_db}!displayRowsTeX@\fidxl{displayRowsTeX}}%
\label{ref_tests_db__displayRowsTeX}%
\hypertarget{ref_tests_db__displayRowsTeX}{}%
\begin{description}
\item[Summary:]Generates a LaTeX table that lists rows of this DB.
%
\item[Usage:]~%
\begin{lyxcode}%
tex\_string = displayRowsTeX(a\_db, caption, props)
%
\end{lyxcode}%
%
\item[Description:]%
Can be then written to a file for processing by Latex or inclusion in
 other documents.
%%
\item[Parameters:]~
\begin{description}%
\item[\texttt{a\_db}:]
 A tests\_db object.
\item[\texttt{caption}:]
 Table caption.
\item[\texttt{props}:]
 A structure with any optional properties, passed to TeXfloat.
\begin{description}%
\item[\texttt{landscape}:]
 If 1, rotate table 90 degrees and scale to 90% of page height.
\end{description}%
\end{description}%
%
\item[Returns:
]~

   tex\_string: LaTeX string for table float.
%
\item[Example:]~
\begin{lyxcode} >> string2File(displayRowsTeX(a\_db(1:10, 4:7), 'some values',
\\%
                               struct('rotate', 0)), 'table.tex')
\\%
\end{lyxcode}
%
\item[See also:]%
\hyperlink{ref_displayRows}{\texttt{displayRows}}%
\ (p.~\pageref{ref_displayRows})%
\index[funcref]{displayRows@\fidxl{displayRows}}%
, \hyperlink{ref_TeXfloat}{\texttt{TeXfloat}}%
\ (p.~\pageref{ref_TeXfloat})%
\index[funcref]{TeXfloat@\fidxl{TeXfloat}}%
, \hyperlink{ref_cell2TeX}{\texttt{cell2TeX}}%
\ (p.~\pageref{ref_cell2TeX})%
\index[funcref]{cell2TeX@\fidxl{cell2TeX}}%
%
\item[Author:]%
Cengiz Gunay <cgunay@emory.edu>, 2004/12/16
%
\end{description}
\methodline%
\subsubsection[Method \texttt{end}]{Method \texttt{tests\_db/end}}%
\index[funcref]{tests_db@\fidxl{tests\_db}!end@\fidxl{end}}%
\label{ref_tests_db__end}%
\hypertarget{ref_tests_db__end}{}%
\begin{description}
\item[Summary:]Overloaded primitive matlab function, returns maximal dimension size.
%
\item[Usage:]~%
\begin{lyxcode}%
s = end(db, index, total)
%
\end{lyxcode}%
%
%
\item[Parameters:]~
\begin{description}%
\item[\texttt{db}:]
 A tests\_db object.
\end{description}%
%
\item[Returns:
]~

	s: The size.
%
%
\item[See also:]%
\hyperlink{ref_size}{\texttt{size}}%
\ (p.~\pageref{ref_size})%
\index[funcref]{size@\fidxl{size}}%
, \hyperlink{ref_tests_db}{\texttt{tests\_db}}%
\ (p.~\pageref{ref_tests_db})%
\index[funcref]{tests_db@\fidxl{tests\_db}}%
%
\item[Author:]%
Cengiz Gunay <cgunay@emory.edu>, 2004/10/06
%
\end{description}
\methodline%
\subsubsection[Method \texttt{enumerateColumns}]{Method \texttt{tests\_db/enumerateColumns}}%
\index[funcref]{tests_db@\fidxl{tests\_db}!enumerateColumns@\fidxl{enumerateColumns}}%
\label{ref_tests_db__enumerateColumns}%
\hypertarget{ref_tests_db__enumerateColumns}{}%
\begin{description}
\item[Summary:]Replaces each value with an integer pointing to the index of enumerated unique values in a column.
%
\item[Usage:]~%
\begin{lyxcode}%
a\_db = enumerateColumns(a\_db, tests, props)
%
\end{lyxcode}%
%
\item[Description:]%
Finds unique values of each column, and replaces the original values
 with the enumerated indices of these unique values. Useful for normalizing all 
 parameter values in a hypercube.
%%
\item[Parameters:]~
\begin{description}%
\item[\texttt{a\_db}:]
 A tests\_db object.
\item[\texttt{tests}:]
 Array of tests to be enumerated.
\item[\texttt{props}:]
 Optional properties.
\begin{description}%
\item[\texttt{truncateDecDigits}:]
 Use only up to this many decimal digits after the point 

when checking for uniqueness.
\end{description}%
\end{description}%
%
\item[Returns:
]~

	a\_db: The modified DB.
%
\item[Example:]~
\begin{lyxcode} >> enumerated\_db = enumerateColumns(a\_db(:, 1:9));
\\%
\end{lyxcode}
%
\item[See also:]%
\hyperlink{ref_uniqueValues}{\texttt{uniqueValues}}%
\ (p.~\pageref{ref_uniqueValues})%
\index[funcref]{uniqueValues@\fidxl{uniqueValues}}%
%
\item[Author:]%
Cengiz Gunay <cgunay@emory.edu>, 2006/06/14
%
\end{description}
\methodline%
\subsubsection[Method \texttt{eq}]{Method \texttt{tests\_db/eq}}%
\index[funcref]{tests_db@\fidxl{tests\_db}!eq@\fidxl{eq}}%
\label{ref_tests_db__eq}%
\hypertarget{ref_tests_db__eq}{}%
\begin{description}
\item[Summary:]Equality (==) operator. Returns logical indices of db rows 
	that match with given row.
%
\item[Usage:]~%
\begin{lyxcode}%
rows = eq(db, row)
%
\end{lyxcode}%
%
%
\item[Parameters:]~
\begin{description}%
\item[\texttt{db}:]
 A tests\_db object.
\item[\texttt{row}:]
 Row array to be compared with db rows.
\end{description}%
%
\item[Returns:
]~

	rows: A logical or index vector to be used in indexing db objects. 
%
%
\item[See also:]%
\hyperlink{ref_eq}{\texttt{eq}}%
\ (p.~\pageref{ref_eq})%
\index[funcref]{eq@\fidxl{eq}}%
, \hyperlink{ref_tests_db}{\texttt{tests\_db}}%
\ (p.~\pageref{ref_tests_db})%
\index[funcref]{tests_db@\fidxl{tests\_db}}%
%
\item[Author:]%
Cengiz Gunay <cgunay@emory.edu>, 2004/09/17
%
\end{description}
\methodline%
\subsubsection[Method \texttt{factoran}]{Method \texttt{tests\_db/factoran}}%
\index[funcref]{tests_db@\fidxl{tests\_db}!factoran@\fidxl{factoran}}%
\label{ref_tests_db__factoran}%
\hypertarget{ref_tests_db__factoran}{}%
\begin{description}
\item[Summary:]Generates a database of factor loadings obtained from the 
		factor analysis of db with factoran. Each row corresponds
		to a rotated factor and columns represent observed variables.
%
\item[Usage:]~%
\begin{lyxcode}%
a\_factors\_db = factoran(db, num\_factors, props)
%
\end{lyxcode}%
%
\item[Description:]%
Uses the promax method to rotate common factors.
%%
\item[Parameters:]~
\begin{description}%
\item[\texttt{db}:]
 A tests\_db object.
\item[\texttt{num\_factors}:]
 Number of common factors to look for.
\item[\texttt{props}:]
 A structure with any optional properties.
\end{description}%
%
\item[Returns:
]~

	a\_factors\_db: A corrcoefs\_db of the coefficients and page indices.
%
%
\item[See also:]%
\hyperlink{ref_tests_db}{\texttt{tests\_db}}%
\ (p.~\pageref{ref_tests_db})%
\index[funcref]{tests_db@\fidxl{tests\_db}}%
, \hyperlink{ref_corrcoefs_db}{\texttt{corrcoefs\_db}}%
\ (p.~\pageref{ref_corrcoefs_db})%
\index[funcref]{corrcoefs_db@\fidxl{corrcoefs\_db}}%
%
\item[Author:]%
Cengiz Gunay <cgunay@emory.edu>, 2004/11/08
%
\end{description}
\methodline%
\subsubsection[Method \texttt{fillMissingColumns}]{Method \texttt{tests\_db/fillMissingColumns}}%
\index[funcref]{tests_db@\fidxl{tests\_db}!fillMissingColumns@\fidxl{fillMissingColumns}}%
\label{ref_tests_db__fillMissingColumns}%
\hypertarget{ref_tests_db__fillMissingColumns}{}%
\begin{description}
\item[Summary:]Add missing columns with given default fill value.
%
\item[Usage:]~%
\begin{lyxcode}%
db = fillMissingColumns(db, col\_names, fill\_value)
%
\end{lyxcode}%
%
%
\item[Parameters:]~
\begin{description}%
\item[\texttt{db}:]
 A tests\_db object.
\item[\texttt{col\_names}:]
 A cell array of column names.
\item[\texttt{fill\_value}:]
 Value to be used for missing columns.
\end{description}%
%
\item[Returns:
]~

	db: The tests\_db object that includes the newly filled columns.
%
%
\item[See also:]%
\hyperlink{ref_tests_db}{\texttt{tests\_db}}%
\ (p.~\pageref{ref_tests_db})%
\index[funcref]{tests_db@\fidxl{tests\_db}}%
, \hyperlink{ref_addColumns}{\texttt{addColumns}}%
\ (p.~\pageref{ref_addColumns})%
\index[funcref]{addColumns@\fidxl{addColumns}}%
, \hyperlink{ref_params_tests_db__addParams}{\texttt{params\_tests\_db/addParams}}%
\ (p.~\pageref{ref_params_tests_db__addParams})%
\index[funcref]{params_tests_db@\fidxl{params\_tests\_db}!addParams@\fidxl{addParams}}%
, \hyperlink{ref_params_tests_db__unionCat}{\texttt{params\_tests\_db/unionCat}}%
\ (p.~\pageref{ref_params_tests_db__unionCat})%
\index[funcref]{params_tests_db@\fidxl{params\_tests\_db}!unionCat@\fidxl{unionCat}}%
%
\item[Author:]%
Cengiz Gunay <cgunay@emory.edu>, 2008/06/02 
%
\end{description}
\methodline%
\subsubsection[Method \texttt{ge}]{Method \texttt{tests\_db/ge}}%
\index[funcref]{tests_db@\fidxl{tests\_db}!ge@\fidxl{ge}}%
\label{ref_tests_db__ge}%
\hypertarget{ref_tests_db__ge}{}%
\begin{description}
\item[Summary:]Greater or equal to (>=) operator. Returns logical indices of db rows 
	that are greater than or equal to given row.
%
\item[Usage:]~%
\begin{lyxcode}%
rows = ge(db, row)
%
\end{lyxcode}%
%
%
\item[Parameters:]~
\begin{description}%
\item[\texttt{db}:]
 A tests\_db object.
\item[\texttt{row}:]
 Row array to be compared with db rows.
\end{description}%
%
\item[Returns:
]~

	rows: A logical or index vector to be used in indexing db objects. 
%
%
\item[See also:]%
\hyperlink{ref_ge}{\texttt{ge}}%
\ (p.~\pageref{ref_ge})%
\index[funcref]{ge@\fidxl{ge}}%
, \hyperlink{ref_tests_db}{\texttt{tests\_db}}%
\ (p.~\pageref{ref_tests_db})%
\index[funcref]{tests_db@\fidxl{tests\_db}}%
%
\item[Author:]%
Cengiz Gunay <cgunay@emory.edu>, 2004/09/17
%
\end{description}
\methodline%
\subsubsection[Method \texttt{get}]{Method \texttt{tests\_db/get}}%
\index[funcref]{tests_db@\fidxl{tests\_db}!get@\fidxl{get}}%
\label{ref_tests_db__get}%
\hypertarget{ref_tests_db__get}{}%
\begin{description}
\item[Summary:]Defines generic attribute retrieval for objects.
%
%
%
%
%
%
%
\item[Author:]%
Cengiz Gunay <cgunay@emory.edu>, 2004/09/14
%
\end{description}
\methodline%
\subsubsection[Method \texttt{getColNames}]{Method \texttt{tests\_db/getColNames}}%
\index[funcref]{tests_db@\fidxl{tests\_db}!getColNames@\fidxl{getColNames}}%
\label{ref_tests_db__getColNames}%
\hypertarget{ref_tests_db__getColNames}{}%
\begin{description}
\item[Summary:]Gets column names.
%
\item[Usage:]~%
\begin{lyxcode}%
col\_names = getColNames(db, tests)
%
\end{lyxcode}%
%
\item[Description:]%
Performs a light operation without touching the data matrix.
%%
\item[Parameters:]~
\begin{description}%
\item[\texttt{db}:]
 A tests\_db object.
\item[\texttt{tests}:]
 Columns for which to get names (Optional, default = ':')
\end{description}%
%
\item[Returns:
]~

	col\_names: A cell array of strings.
%
%
\item[See also:]%
\hyperlink{ref_getColNames}{\texttt{getColNames}}%
\ (p.~\pageref{ref_getColNames})%
\index[funcref]{getColNames@\fidxl{getColNames}}%
, \hyperlink{ref_tests_db}{\texttt{tests\_db}}%
\ (p.~\pageref{ref_tests_db})%
\index[funcref]{tests_db@\fidxl{tests\_db}}%
%
\item[Author:]%
Cengiz Gunay <cgunay@emory.edu>, 2006/05/24
%
\end{description}
\methodline%
\subsubsection[Method \texttt{groupBy}]{Method \texttt{tests\_db/groupBy}}%
\index[funcref]{tests_db@\fidxl{tests\_db}!groupBy@\fidxl{groupBy}}%
\label{ref_tests_db__groupBy}%
\hypertarget{ref_tests_db__groupBy}{}%
\begin{description}
\item[Summary:]Groups same values of column(s) into separate pages of a 3D db.
%
\item[Usage:]~%
\begin{lyxcode}%
a\_tests\_3D\_db = groupBy(db, cols)
%
\end{lyxcode}%
%
\item[Description:]%
Functionality similar to SQL's GROUP BY keyword. This function
 uses invarValues, but resulting pages will not be sorted.
%%
\item[Parameters:]~
\begin{description}%
\item[\texttt{db}:]
 A tests\_db object.
\item[\texttt{cols}:]
 Columns whose same values will be in one page (see tests2cols

for column representation).
\end{description}%
%
\item[Returns:
]~

	a\_tests\_3D\_db: A tests\_3D\_db object of organized values.
%
\item[Example:]~
\begin{lyxcode} >> a\_db = tests\_db([ ... ], {'par1', 'par2', 'measure1', 'measure2'})
\\%
 % make a page for each value of par1, and list par2 values with assoc. measures:
\\%
 >> a\_3d\_db = groupBy(a\_db, 'par1')
\\%
 >> % get back other columns:
\\%
 >> joined\_3d\_db = joinRows(a\_db, a\_3d\_db)
\\%
 >> displayRows(joined\_3d\_db(:, :, 1))
\\%
\end{lyxcode}
%
\item[See also:]%
\hyperlink{ref_invarValues}{\texttt{invarValues}}%
\ (p.~\pageref{ref_invarValues})%
\index[funcref]{invarValues@\fidxl{invarValues}}%
, \hyperlink{ref_tests_3D_db}{\texttt{tests\_3D\_db}}%
\ (p.~\pageref{ref_tests_3D_db})%
\index[funcref]{tests_3D_db@\fidxl{tests\_3D\_db}}%
, \hyperlink{ref_tests_3D_db__corrCoefs}{\texttt{tests\_3D\_db/corrCoefs}}%
\ (p.~\pageref{ref_tests_3D_db__corrCoefs})%
\index[funcref]{tests_3D_db@\fidxl{tests\_3D\_db}!corrCoefs@\fidxl{corrCoefs}}%
, \hyperlink{ref_tests_3D_db__plotPair}{\texttt{tests\_3D\_db/plotPair}}%
\ (p.~\pageref{ref_tests_3D_db__plotPair})%
\index[funcref]{tests_3D_db@\fidxl{tests\_3D\_db}!plotPair@\fidxl{plotPair}}%
, \hyperlink{ref_joinRows}{\texttt{joinRows}}%
\ (p.~\pageref{ref_joinRows})%
\index[funcref]{joinRows@\fidxl{joinRows}}%
, \hyperlink{ref_tests_3D_db__swapRowsPages}{\texttt{tests\_3D\_db/swapRowsPages}}%
\ (p.~\pageref{ref_tests_3D_db__swapRowsPages})%
\index[funcref]{tests_3D_db@\fidxl{tests\_3D\_db}!swapRowsPages@\fidxl{swapRowsPages}}%
, \hyperlink{ref_tests_3D_db__mergePages}{\texttt{tests\_3D\_db/mergePages}}%
\ (p.~\pageref{ref_tests_3D_db__mergePages})%
\index[funcref]{tests_3D_db@\fidxl{tests\_3D\_db}!mergePages@\fidxl{mergePages}}%
%
\item[Author:]%
Cengiz Gunay <cgunay@emory.edu>, 2008/05/27
%
\end{description}
\methodline%
\subsubsection[Method \texttt{gt}]{Method \texttt{tests\_db/gt}}%
\index[funcref]{tests_db@\fidxl{tests\_db}!gt@\fidxl{gt}}%
\label{ref_tests_db__gt}%
\hypertarget{ref_tests_db__gt}{}%
\begin{description}
\item[Summary:]Greater than (>) operator. Returns logical indices of db rows 
	that are greater than given row.
%
\item[Usage:]~%
\begin{lyxcode}%
rows = gt(db, row)
%
\end{lyxcode}%
%
%
\item[Parameters:]~
\begin{description}%
\item[\texttt{db}:]
 A tests\_db object.
\item[\texttt{row}:]
 Row array to be compared with db rows.
\end{description}%
%
\item[Returns:
]~

	rows: A logical or index vector to be used in indexing db objects. 
%
%
\item[See also:]%
\hyperlink{ref_gt}{\texttt{gt}}%
\ (p.~\pageref{ref_gt})%
\index[funcref]{gt@\fidxl{gt}}%
, \hyperlink{ref_tests_db}{\texttt{tests\_db}}%
\ (p.~\pageref{ref_tests_db})%
\index[funcref]{tests_db@\fidxl{tests\_db}}%
%
\item[Author:]%
Cengiz Gunay <cgunay@emory.edu>, 2004/09/17
%
\end{description}
\methodline%
\subsubsection[Method \texttt{histogram}]{Method \texttt{tests\_db/histogram}}%
\index[funcref]{tests_db@\fidxl{tests\_db}!histogram@\fidxl{histogram}}%
\label{ref_tests_db__histogram}%
\hypertarget{ref_tests_db__histogram}{}%
\begin{description}
\item[Summary:]Returns histogram of chosen database column.
%
\item[Usage:]~%
\begin{lyxcode}%
a\_histogram\_db = histogram(db, col, num\_bins, props)
%
\end{lyxcode}%
%
\item[Description:]%
Generates a histogram\_db object with rows corresponding to histogram
 entries. If an array of DBs is given, finds and uses common histogram bin centers.
%%
\item[Parameters:]~
\begin{description}%
\item[\texttt{db}:]
 A tests\_db object.
\item[\texttt{col}:]
 Column to find the histogram.
\item[\texttt{num\_bins}:]
 Number of histogram bins (Optional, default=100), or

vector of histogram bin centers.
\item[\texttt{props}:]
 A structure with any optional properties.
\begin{description}%
\item[\texttt{normalized}:]
 If 1, normalize histogram counts.
\end{description}%
\end{description}%
%
\item[Returns:
]~

	a\_histogram\_db: A histogram\_db object containing the histogram.
%
\item[Example:]~
\begin{lyxcode} >> a\_hist\_db = histogram(my\_db, 'spike\_width');
\\%
 >> plot(a\_hist\_db);
\\%
\end{lyxcode}
%
\item[See also:]%
\hyperlink{ref_histogram_db}{\texttt{histogram\_db}}%
\ (p.~\pageref{ref_histogram_db})%
\index[funcref]{histogram_db@\fidxl{histogram\_db}}%
, \hyperlink{ref_tests_db}{\texttt{tests\_db}}%
\ (p.~\pageref{ref_tests_db})%
\index[funcref]{tests_db@\fidxl{tests\_db}}%
, \hyperlink{ref_hist}{\texttt{hist}}%
\ (p.~\pageref{ref_hist})%
\index[funcref]{hist@\fidxl{hist}}%
%
\item[Author:]%
Cengiz Gunay <cgunay@emory.edu>, 2004/09/17
%
\end{description}
\methodline%
\subsubsection[Method \texttt{invarValues}]{Method \texttt{tests\_db/invarValues}}%
\index[funcref]{tests_db@\fidxl{tests\_db}!invarValues@\fidxl{invarValues}}%
\label{ref_tests_db__invarValues}%
\hypertarget{ref_tests_db__invarValues}{}%
\begin{description}
\item[Summary:]Finds all sets in which given columns vary while the rest are invariant.
%
\item[Usage:]~%
\begin{lyxcode}%
a\_tests\_3D\_db = invarValues(db, cols, in\_page\_unique\_cols, props)
%
\end{lyxcode}%
%
\item[Description:]%
Useful when trying to find relationships between some columns
 independent of other columns. In a database that contains results of a
 multivariate function, this function can find the effect of one or more
 parameters when other parameters are kept constant (i.e., invariant). Rows
 with the values of the desired columns are separated into the pages of a
 tests\_3D\_db for each unique set of the other column values. These
 invariant values of the other columns are missing from the resulting
 tests\_3D\_db, instead a RowIndex is kept pointing to the db in which they
 can be found. See joinRows for joining the results back with the invariant
 columns.
   If in\_page\_unique\_cols is given, this function by default row-sorts the
 database to ensure that each page has the same parameter values in the
 same rows. This is important because when the rows and pages of database
 is swapped (see tests\_3D\_db/swapRowsPages) each page has the same value of
 the in\_page\_unique\_cols variables. Other functions such as
 tests\_3D\_db/mergePages also depend on this property.
   In databases that contain all unique combinations of certain parameters,
 the resulting 3D database becomes symmetric. However, for databases with
 missing combinations, in\_page\_unique\_cols specifies which columns is used
 to guide which rows of the page to place values found. This function will
 fail if you do not have such a column.  Note: the trial column will be
 ignored before finding invariant values.
%%
\item[Parameters:]~
\begin{description}%
\item[\texttt{db}:]
 A tests\_db object.
\item[\texttt{cols}:]
 Vector of column numbers to find values when others are

invariant. Include result columns here.
\item[\texttt{in\_page\_unique\_cols}:]
 Vector of columns that have the same unique values in each page 

(Optional; used only if database is not symmetric, to ignore 
missing values of in\_page\_unique\_cols)
\item[\texttt{props}:]
 A structure with any optional properties.
\begin{description}%
\item[\texttt{sortPages}:]
 If 1, page-sorts even symmetric databases (default=1).
\end{description}%
\end{description}%
%
\item[Returns:
]~

	a\_tests\_3D\_db: A tests\_3D\_db object of organized values.
%
\item[Example:]~
\begin{lyxcode} >> a\_db = tests\_db([ ... ], {'par1', 'par2', 'measure1', 'measure2'})
\\%
 % make a page for each value of par1, and list par2 values with assoc. measures:
\\%
 >> a\_3d\_db = invarValues(a\_db, [2:4], 'par2')
\\%
 >> % get back other columns:
\\%
 >> joined\_3d\_db = joinRows(a\_db, a\_3d\_db)
\\%
 >> displayRows(joined\_3d\_db(:, :, 1))
\\%
\end{lyxcode}
%
\item[See also:]%
\hyperlink{ref_tests_3D_db}{\texttt{tests\_3D\_db}}%
\ (p.~\pageref{ref_tests_3D_db})%
\index[funcref]{tests_3D_db@\fidxl{tests\_3D\_db}}%
, \hyperlink{ref_tests_3D_db__corrCoefs}{\texttt{tests\_3D\_db/corrCoefs}}%
\ (p.~\pageref{ref_tests_3D_db__corrCoefs})%
\index[funcref]{tests_3D_db@\fidxl{tests\_3D\_db}!corrCoefs@\fidxl{corrCoefs}}%
, \hyperlink{ref_tests_3D_db__plotPair}{\texttt{tests\_3D\_db/plotPair}}%
\ (p.~\pageref{ref_tests_3D_db__plotPair})%
\index[funcref]{tests_3D_db@\fidxl{tests\_3D\_db}!plotPair@\fidxl{plotPair}}%
, \hyperlink{ref_joinRows}{\texttt{joinRows}}%
\ (p.~\pageref{ref_joinRows})%
\index[funcref]{joinRows@\fidxl{joinRows}}%
, \hyperlink{ref_tests_3D_db__swapRowsPages}{\texttt{tests\_3D\_db/swapRowsPages}}%
\ (p.~\pageref{ref_tests_3D_db__swapRowsPages})%
\index[funcref]{tests_3D_db@\fidxl{tests\_3D\_db}!swapRowsPages@\fidxl{swapRowsPages}}%
, \hyperlink{ref_tests_3D_db__mergePages}{\texttt{tests\_3D\_db/mergePages}}%
\ (p.~\pageref{ref_tests_3D_db__mergePages})%
\index[funcref]{tests_3D_db@\fidxl{tests\_3D\_db}!mergePages@\fidxl{mergePages}}%
%
\item[Author:]%
Cengiz Gunay <cgunay@emory.edu>, 2004/09/30
%
\end{description}
\methodline%
\subsubsection[Method \texttt{isinf}]{Method \texttt{tests\_db/isinf}}%
\index[funcref]{tests_db@\fidxl{tests\_db}!isinf@\fidxl{isinf}}%
\label{ref_tests_db__isinf}%
\hypertarget{ref_tests_db__isinf}{}%
\begin{description}
\item[Summary:]Returns logical row indices of Inf-valued columns.
%
\item[Usage:]~%
\begin{lyxcode}%
rows = isinf(db, col)
%
\end{lyxcode}%
%
%
\item[Parameters:]~
\begin{description}%
\item[\texttt{db}:]
 A tests\_db object.
\item[\texttt{col}:]
 Column to check (Optional, default = 1)
\end{description}%
%
\item[Returns:
]~

	rows: A logical column vector of rows.
%
%
\item[See also:]%
\hyperlink{ref_isinf}{\texttt{isinf}}%
\ (p.~\pageref{ref_isinf})%
\index[funcref]{isinf@\fidxl{isinf}}%
, \hyperlink{ref_tests_db}{\texttt{tests\_db}}%
\ (p.~\pageref{ref_tests_db})%
\index[funcref]{tests_db@\fidxl{tests\_db}}%
%
\item[Author:]%
Cengiz Gunay <cgunay@emory.edu>, 2005/08/16
%
\end{description}
\methodline%
\subsubsection[Method \texttt{isnan}]{Method \texttt{tests\_db/isnan}}%
\index[funcref]{tests_db@\fidxl{tests\_db}!isnan@\fidxl{isnan}}%
\label{ref_tests_db__isnan}%
\hypertarget{ref_tests_db__isnan}{}%
\begin{description}
\item[Summary:]Returns logical row indices of NaN-valued columns.
%
\item[Usage:]~%
\begin{lyxcode}%
rows = isnan(db, col)
%
\end{lyxcode}%
%
%
\item[Parameters:]~
\begin{description}%
\item[\texttt{db}:]
 A tests\_db object.
\item[\texttt{col}:]
 Column to check (Optional, default = 1)
\end{description}%
%
\item[Returns:
]~

	rows: A logical column vector of rows.
%
%
\item[See also:]%
\hyperlink{ref_isnan}{\texttt{isnan}}%
\ (p.~\pageref{ref_isnan})%
\index[funcref]{isnan@\fidxl{isnan}}%
, \hyperlink{ref_tests_db}{\texttt{tests\_db}}%
\ (p.~\pageref{ref_tests_db})%
\index[funcref]{tests_db@\fidxl{tests\_db}}%
%
\item[Author:]%
Cengiz Gunay <cgunay@emory.edu>, 2004/10/06
%
\end{description}
\methodline%
\subsubsection[Method \texttt{isnanrows}]{Method \texttt{tests\_db/isnanrows}}%
\index[funcref]{tests_db@\fidxl{tests\_db}!isnanrows@\fidxl{isnanrows}}%
\label{ref_tests_db__isnanrows}%
\hypertarget{ref_tests_db__isnanrows}{}%
\begin{description}
\item[Summary:]Finds rows with any NaN values. Returns logical indices of db rows.
%
\item[Usage:]~%
\begin{lyxcode}%
rows = isnanrows(db)
%
\end{lyxcode}%
%
\item[Description:]%
Some operations need that no NaN values exist in the matrix. This method
 can be used to find and then remove NaN-contaminated rows from DB. Note
 that sometimes no rows can  be found, and some columns should be discarded
 before this operation.
%%
\item[Parameters:]~
\begin{description}%
\item[\texttt{db}:]
 A tests\_db object.
\end{description}%
%
\item[Returns:
]~

	rows: A logical vector to be used in indexing db objects or passed
		through other logical operators. 
%
%
\item[See also:]%
\hyperlink{ref_isnan}{\texttt{isnan}}%
\ (p.~\pageref{ref_isnan})%
\index[funcref]{isnan@\fidxl{isnan}}%
, \hyperlink{ref_tests_db}{\texttt{tests\_db}}%
\ (p.~\pageref{ref_tests_db})%
\index[funcref]{tests_db@\fidxl{tests\_db}}%
%
\item[Author:]%
Cengiz Gunay <cgunay@emory.edu>, 2004/11/08
%
\end{description}
\methodline%
\subsubsection[Method \texttt{joinRows}]{Method \texttt{tests\_db/joinRows}}%
\index[funcref]{tests_db@\fidxl{tests\_db}!joinRows@\fidxl{joinRows}}%
\label{ref_tests_db__joinRows}%
\hypertarget{ref_tests_db__joinRows}{}%
\begin{description}
\item[Summary:]Joins a\_db rows with w\_db rows having matching RowIndex values.
%
\item[Usage:]~%
\begin{lyxcode}%
a\_db = joinRows(a\_db, w\_db, props)
%
\end{lyxcode}%
%
\item[Description:]%
Concatenates columns of rows matching the join condition from the two
 databases. Each row index must appear only once in w\_db. The created db
 preserves the ordering of w\_db. See the multipleIndices option if there
 are several redundant index columns. Multiple pages in w\_db are accepted
 (see keepNaNs option).
   This function is the equivalent of a "right outer join" command in SQL, w\_db
 being the database table on the right.
%%
\item[Parameters:]~
\begin{description}%
\item[\texttt{a\_db}:]
 A tests\_db object.
\item[\texttt{w\_db}:]
 A tests\_db object with a row index column.
\item[\texttt{props}:]
 A structure with any optional properties.
\begin{description}%
\item[\texttt{indexColName}:]
 (Optional) Name of row index column

(default='RowIndex').
\item[\texttt{keepNaNs}:]
 If 1, substitute NaN values for NaN indices. 

(default=1, for multi-page DBs; 0, otherwise).
\item[\texttt{multipleIndices}:]
 If 1, search for substitute RowIndex* columns for

indices with NaN values. It will fail if all indices are
NaNs. (default=0)
\item[\texttt{keepIndex}:]
 If 1, don't delete the indexColName (default='RowIndex')
\end{description}%
\end{description}%
%
\item[Returns:
]~

	a\_db: A tests\_db object.
%
%
\item[See also:]%
\hyperlink{ref_tests_db}{\texttt{tests\_db}}%
\ (p.~\pageref{ref_tests_db})%
\index[funcref]{tests_db@\fidxl{tests\_db}}%
%
\item[Author:]%
Cengiz Gunay <cgunay@emory.edu>, 2004/10/16
%
\end{description}
\methodline%
\subsubsection[Method \texttt{kmeansCluster}]{Method \texttt{tests\_db/kmeansCluster}}%
\index[funcref]{tests_db@\fidxl{tests\_db}!kmeansCluster@\fidxl{kmeansCluster}}%
\label{ref_tests_db__kmeansCluster}%
\hypertarget{ref_tests_db__kmeansCluster}{}%
\begin{description}
\item[Summary:]Generates a database of cluster centers obtained from a k-means cluster analysis with the command kmeans.
%
\item[Usage:]~%
\begin{lyxcode}%
a\_cluster\_db = kmeansCluster(db, num\_clusters, props)
%
\end{lyxcode}%
%
%
\item[Parameters:]~
\begin{description}%
\item[\texttt{db}:]
 A tests\_db object.
\item[\texttt{num\_clusters}:]
 Number of clusters to form.
\item[\texttt{props}:]
 A structure with any optional properties.
\begin{description}%
\item[\texttt{DistanceMeasure}:]
 Choose one appropriate for kmeans.
\end{description}%
\end{description}%
%
\item[Returns:
]~

	a\_cluster\_db: A tests\_db where each row is a cluster center.
	a\_hist\_db: histogram\_db showing cluster membership from original db.
	idx: Cluster indices of each row or original db.
	sum\_distances: Quality of clustering indicated by total distance from
			centroid for each cluster.
%
%
\item[See also:]%
\hyperlink{ref_tests_db}{\texttt{tests\_db}}%
\ (p.~\pageref{ref_tests_db})%
\index[funcref]{tests_db@\fidxl{tests\_db}}%
, \hyperlink{ref_histogram_db}{\texttt{histogram\_db}}%
\ (p.~\pageref{ref_histogram_db})%
\index[funcref]{histogram_db@\fidxl{histogram\_db}}%
%
\item[Author:]%
Cengiz Gunay <cgunay@emory.edu>, 2005/04/06
%
\end{description}
\methodline%
\subsubsection[Method \texttt{le}]{Method \texttt{tests\_db/le}}%
\index[funcref]{tests_db@\fidxl{tests\_db}!le@\fidxl{le}}%
\label{ref_tests_db__le}%
\hypertarget{ref_tests_db__le}{}%
\begin{description}
\item[Summary:]Less or equal (<=) operator. Returns logical indices of db rows 
	that are less than or equal to given row.
%
\item[Usage:]~%
\begin{lyxcode}%
rows = le(db, row)
%
\end{lyxcode}%
%
%
\item[Parameters:]~
\begin{description}%
\item[\texttt{db}:]
 A tests\_db object.
\item[\texttt{row}:]
 Row array to be compared with db rows.
\end{description}%
%
\item[Returns:
]~

	rows: A logical or index vector to be used in indexing db objects. 
%
%
\item[See also:]%
\hyperlink{ref_le}{\texttt{le}}%
\ (p.~\pageref{ref_le})%
\index[funcref]{le@\fidxl{le}}%
, \hyperlink{ref_tests_db}{\texttt{tests\_db}}%
\ (p.~\pageref{ref_tests_db})%
\index[funcref]{tests_db@\fidxl{tests\_db}}%
%
\item[Author:]%
Cengiz Gunay <cgunay@emory.edu>, 2004/09/17
%
\end{description}
\methodline%
\subsubsection[Method \texttt{lt}]{Method \texttt{tests\_db/lt}}%
\index[funcref]{tests_db@\fidxl{tests\_db}!lt@\fidxl{lt}}%
\label{ref_tests_db__lt}%
\hypertarget{ref_tests_db__lt}{}%
\begin{description}
\item[Summary:]Less than (<) operator. Returns logical indices of db rows 
	that are less than given row.
%
\item[Usage:]~%
\begin{lyxcode}%
rows = lt(db, row)
%
\end{lyxcode}%
%
%
\item[Parameters:]~
\begin{description}%
\item[\texttt{db}:]
 A tests\_db object.
\item[\texttt{row}:]
 Row array to be compared with db rows.
\end{description}%
%
\item[Returns:
]~

	rows: A logical or index vector to be used in indexing db objects. 
%
%
\item[See also:]%
\hyperlink{ref_lt}{\texttt{lt}}%
\ (p.~\pageref{ref_lt})%
\index[funcref]{lt@\fidxl{lt}}%
, \hyperlink{ref_tests_db}{\texttt{tests\_db}}%
\ (p.~\pageref{ref_tests_db})%
\index[funcref]{tests_db@\fidxl{tests\_db}}%
%
\item[Author:]%
Cengiz Gunay <cgunay@emory.edu>, 2004/09/17
%
\end{description}
\methodline%
\subsubsection[Method \texttt{matchingRow}]{Method \texttt{tests\_db/matchingRow}}%
\index[funcref]{tests_db@\fidxl{tests\_db}!matchingRow@\fidxl{matchingRow}}%
\label{ref_tests_db__matchingRow}%
\hypertarget{ref_tests_db__matchingRow}{}%
\begin{description}
\item[Summary:]Creates a criterion database for matching the tests of a row.
%
\item[Usage:]~%
\begin{lyxcode}%
crit\_db = matchingRow(db, row, props)
%
\end{lyxcode}%
%
\item[Description:]%
Copies selected test values from row as the first row into the criterion
 db. Adds a second row for the STD of each column in the db.  Calculates the
 covariance for using the Mahalonobis distance in the ranking.
%%
\item[Parameters:]~
\begin{description}%
\item[\texttt{db}:]
 A tests\_db object.
\item[\texttt{row}:]
 A row index to match.
\item[\texttt{props}:]
 A structure with any optional properties.
\begin{description}%
\item[\texttt{distDB}:]
 Take the standard deviation and covariance of this db instead.
\end{description}%
\end{description}%
%
\item[Returns:
]~

	crit\_db: A tests\_db with two rows for values and STDs.
%
\item[Example:]~
\begin{lyxcode}        >> crit\_db = matchingRow(phys\_control\_compare\_db, 
\\%
                find(phys\_control\_compare\_db(:, 'TracesetIndex') == 61))
\\%
\end{lyxcode}
%
\item[See also:]%
\hyperlink{ref_rankMatching}{\texttt{rankMatching}}%
\ (p.~\pageref{ref_rankMatching})%
\index[funcref]{rankMatching@\fidxl{rankMatching}}%
, \hyperlink{ref_tests_db}{\texttt{tests\_db}}%
\ (p.~\pageref{ref_tests_db})%
\index[funcref]{tests_db@\fidxl{tests\_db}}%
, \hyperlink{ref_tests2cols}{\texttt{tests2cols}}%
\ (p.~\pageref{ref_tests2cols})%
\index[funcref]{tests2cols@\fidxl{tests2cols}}%
%
\item[Author:]%
Cengiz Gunay <cgunay@emory.edu>, 2004/12/08
%
\end{description}
\methodline%
\subsubsection[Method \texttt{max}]{Method \texttt{tests\_db/max}}%
\index[funcref]{tests_db@\fidxl{tests\_db}!max@\fidxl{max}}%
\label{ref_tests_db__max}%
\hypertarget{ref_tests_db__max}{}%
\begin{description}
\item[Summary:]Returns the max of the data matrix of a\_db. Ignores NaN and Inf values.
%
\item[Usage:]~%
\begin{lyxcode}%
[a\_db, n, i] = max(a\_db, dim)
%
\end{lyxcode}%
%
\item[Description:]%
Does a recursive operation over dimensions in order to remove NaN and
 Inf values. This takes more time than a straightforward max operation. 
%%
\item[Parameters:]~
\begin{description}%
\item[\texttt{a\_db}:]
 A tests\_db object.
\item[\texttt{dim}:]
 Work down dimension.
\end{description}%
%
\item[Returns:
]~

   a\_db: The DB with one row of max values.
   n: (Optional) Numbers of non-NaN rows included in calculating
	each column.
   i: Indices where the value was found.
%
%
\item[See also:]%
\hyperlink{ref_max}{\texttt{max}}%
\ (p.~\pageref{ref_max})%
\index[funcref]{max@\fidxl{max}}%
, \hyperlink{ref_tests_db}{\texttt{tests\_db}}%
\ (p.~\pageref{ref_tests_db})%
\index[funcref]{tests_db@\fidxl{tests\_db}}%
%
\item[Author:]%
Cengiz Gunay <cgunay@emory.edu>, 2004/10/06
%
\end{description}
\methodline%
\subsubsection[Method \texttt{mean}]{Method \texttt{tests\_db/mean}}%
\index[funcref]{tests_db@\fidxl{tests\_db}!mean@\fidxl{mean}}%
\label{ref_tests_db__mean}%
\hypertarget{ref_tests_db__mean}{}%
\begin{description}
\item[Summary:]Returns the mean of the data matrix of a\_db. Ignores NaN values.
%
\item[Usage:]~%
\begin{lyxcode}%
[a\_db, n] = mean(a\_db, dim)
%
\end{lyxcode}%
%
\item[Description:]%
Does a recursive operation over dimensions in order to remove NaN values.
 This takes more time than a straightforward mean operation. 
%%
\item[Parameters:]~
\begin{description}%
\item[\texttt{a\_db}:]
 A tests\_db object.
\item[\texttt{dim}:]
 Work down dimension.
\end{description}%
%
\item[Returns:
]~

	a\_db: The DB with one row of mean values.
	n: (Optional) Numbers of non-NaN rows included in calculating each column.
%
%
\item[See also:]%
\hyperlink{ref_mean}{\texttt{mean}}%
\ (p.~\pageref{ref_mean})%
\index[funcref]{mean@\fidxl{mean}}%
, \hyperlink{ref_tests_db}{\texttt{tests\_db}}%
\ (p.~\pageref{ref_tests_db})%
\index[funcref]{tests_db@\fidxl{tests\_db}}%
%
\item[Author:]%
Cengiz Gunay <cgunay@emory.edu>, 2004/10/06
%
\end{description}
\methodline%
\subsubsection[Method \texttt{meanDuplicateRows}]{Method \texttt{tests\_db/meanDuplicateRows}}%
\index[funcref]{tests_db@\fidxl{tests\_db}!meanDuplicateRows@\fidxl{meanDuplicateRows}}%
\label{ref_tests_db__meanDuplicateRows}%
\hypertarget{ref_tests_db__meanDuplicateRows}{}%
\begin{description}
\item[Summary:]Row-reduces a db by finding sets of rows with same main\_cols values, and replacing each set with a single row containing main\_cols and the mean of rest\_cols.
%
\item[Usage:]~%
\begin{lyxcode}%
a\_tests\_db = meanDuplicateRows(db, main\_cols, rest\_cols)
%
\end{lyxcode}%
%
\item[Description:]%
The database is sorted for the values of the columns of 
 interest (main\_cols) and all rows with duplicate values of 
 these columns are identified. The rest of the columns (rest\_cols) 
 are averaged and reduced to a single row, and attached to the
 nominal values of main\_cols. Two additional parameter columns will be added to the
 database created. The NumDuplicates column is the the number of duplicates 
 used in the mean operation, and RowIndex is the row number points 
 to the first of a set of duplicate values.
%%
\item[Parameters:]~
\begin{description}%
\item[\texttt{db}:]
 A tests\_db object.
\item[\texttt{main\_cols}:]
 Vector of columns in which to find duplicates.
\item[\texttt{rest\_cols}:]
 Vector of columns to be averaged for duplicate main\_cols.
\end{description}%
%
\item[Returns:
]~

	a\_tests\_db: The db object of with the means on page 1 
		    and standard deviations on page 2.
%
%
\item[See also:]%
\hyperlink{ref_tests_db__mean}{\texttt{tests\_db/mean}}%
\ (p.~\pageref{ref_tests_db__mean})%
\index[funcref]{tests_db@\fidxl{tests\_db}!mean@\fidxl{mean}}%
, \hyperlink{ref_tests_db__std}{\texttt{tests\_db/std}}%
\ (p.~\pageref{ref_tests_db__std})%
\index[funcref]{tests_db@\fidxl{tests\_db}!std@\fidxl{std}}%
, \hyperlink{ref_sortedUniqueValues}{\texttt{sortedUniqueValues}}%
\ (p.~\pageref{ref_sortedUniqueValues})%
\index[funcref]{sortedUniqueValues@\fidxl{sortedUniqueValues}}%
%
\item[Author:]%
Cengiz Gunay <cgunay@emory.edu>, 2004/09/30
%
\end{description}
\methodline%
\subsubsection[Method \texttt{min}]{Method \texttt{tests\_db/min}}%
\index[funcref]{tests_db@\fidxl{tests\_db}!min@\fidxl{min}}%
\label{ref_tests_db__min}%
\hypertarget{ref_tests_db__min}{}%
\begin{description}
\item[Summary:]Returns the min of the data matrix of a\_db. Ignores NaN and Inf values.
%
\item[Usage:]~%
\begin{lyxcode}%
[a\_db, n, i] = min(a\_db, dim)
%
\end{lyxcode}%
%
\item[Description:]%
Does a recursive operation over dimensions in order to remove NaN and
 Inf values. This takes more time than a straightforward min operation. 
%%
\item[Parameters:]~
\begin{description}%
\item[\texttt{a\_db}:]
 A tests\_db object.
\item[\texttt{dim}:]
 Work down dimension.
\end{description}%
%
\item[Returns:
]~

   a\_db: The DB with one row of min values.
   n: (Optional) Numbers of non-NaN rows included in calculating
	each column.
   i: Indices where the value was found.
%
%
\item[See also:]%
\hyperlink{ref_min}{\texttt{min}}%
\ (p.~\pageref{ref_min})%
\index[funcref]{min@\fidxl{min}}%
, \hyperlink{ref_max}{\texttt{max}}%
\ (p.~\pageref{ref_max})%
\index[funcref]{max@\fidxl{max}}%
, \hyperlink{ref_tests_db}{\texttt{tests\_db}}%
\ (p.~\pageref{ref_tests_db})%
\index[funcref]{tests_db@\fidxl{tests\_db}}%
%
\item[Author:]%
Cengiz Gunay <cgunay@emory.edu>, 2004/10/06
%
\end{description}
\methodline%
\subsubsection[Method \texttt{minus}]{Method \texttt{tests\_db/minus}}%
\index[funcref]{tests_db@\fidxl{tests\_db}!minus@\fidxl{minus}}%
\label{ref_tests_db__minus}%
\hypertarget{ref_tests_db__minus}{}%
\begin{description}
\item[Summary:]Subtracts a DB from another or from a scalar.
%
\item[Usage:]~%
\begin{lyxcode}%
a\_db = minus(left\_obj, right\_obj)
%
\end{lyxcode}%
%
\item[Description:]%
If DBs have mismatching columns only the common columns will be kept.
 In any case, the resulting DB columns will be sorted in the order of the
 left-hand-side DB.
%%
\item[Parameters:]~
\begin{description}%
\item[\texttt{left\_obj, right\_obj}:]
 Operands of the subtraction. One must be of type tests\_db

and the other can be a scalar.
\end{description}%
%
\item[Returns:
]~

	a\_db: The resulting tests\_db.
%
%
\item[See also:]%
\hyperlink{ref_minus}{\texttt{minus}}%
\ (p.~\pageref{ref_minus})%
\index[funcref]{minus@\fidxl{minus}}%
%
\item[Author:]%
Cengiz Gunay <cgunay@emory.edu>, 2006/05/24
%
\end{description}
\methodline%
\subsubsection[Method \texttt{mtimes}]{Method \texttt{tests\_db/mtimes}}%
\index[funcref]{tests_db@\fidxl{tests\_db}!mtimes@\fidxl{mtimes}}%
\label{ref_tests_db__mtimes}%
\hypertarget{ref_tests_db__mtimes}{}%
\begin{description}
\item[Summary:]Multiplies the DB with a scalar.
%
\item[Usage:]~%
\begin{lyxcode}%
a\_db = mtimes(left\_obj, right\_obj)
%
\end{lyxcode}%
%
%
\item[Parameters:]~
\begin{description}%
\item[\texttt{left\_obj, right\_obj}:]
 Operands of the multiplication. One or more must be of type tests\_db.
\end{description}%
%
\item[Returns:
]~

	a\_db: The resulting tests\_db.
%
%
\item[See also:]%
\hyperlink{ref_tests_db__times}{\texttt{tests\_db/times}}%
\ (p.~\pageref{ref_tests_db__times})%
\index[funcref]{tests_db@\fidxl{tests\_db}!times@\fidxl{times}}%
, \hyperlink{ref_mtimes}{\texttt{mtimes}}%
\ (p.~\pageref{ref_mtimes})%
\index[funcref]{mtimes@\fidxl{mtimes}}%
%
\item[Author:]%
Cengiz Gunay <cgunay@emory.edu>, 2006/05/24
%
\end{description}
\methodline%
\subsubsection[Method \texttt{ne}]{Method \texttt{tests\_db/ne}}%
\index[funcref]{tests_db@\fidxl{tests\_db}!ne@\fidxl{ne}}%
\label{ref_tests_db__ne}%
\hypertarget{ref_tests_db__ne}{}%
\begin{description}
\item[Summary:]Returns logical indices of db rows that doesn't match with given row.
%
\item[Usage:]~%
\begin{lyxcode}%
rows = ne(db, row)
%
\end{lyxcode}%
%
%
\item[Parameters:]~
\begin{description}%
\item[\texttt{db}:]
 A tests\_db object.
\item[\texttt{row}:]
 Row array to be compared with db rows.
\end{description}%
%
\item[Returns:
]~

	rows: A logical or index vector to be used in indexing db objects. 
%
%
\item[See also:]%
\hyperlink{ref_ne}{\texttt{ne}}%
\ (p.~\pageref{ref_ne})%
\index[funcref]{ne@\fidxl{ne}}%
, \hyperlink{ref_tests_db}{\texttt{tests\_db}}%
\ (p.~\pageref{ref_tests_db})%
\index[funcref]{tests_db@\fidxl{tests\_db}}%
%
\item[Author:]%
Cengiz Gunay <cgunay@emory.edu>, 2004/09/17
%
\end{description}
\methodline%
\subsubsection[Method \texttt{noNaNRows}]{Method \texttt{tests\_db/noNaNRows}}%
\index[funcref]{tests_db@\fidxl{tests\_db}!noNaNRows@\fidxl{noNaNRows}}%
\label{ref_tests_db__noNaNRows}%
\hypertarget{ref_tests_db__noNaNRows}{}%
\begin{description}
\item[Summary:]Returns a DB by removing rows containing any NaN or Inf.
%
\item[Usage:]~%
\begin{lyxcode}%
a\_db = noNaNRows(a\_db)
%
\end{lyxcode}%
%
%
\item[Parameters:]~
\begin{description}%
\item[\texttt{a\_db}:]
 A tests\_db object.
\end{description}%
%
\item[Returns:
]~

	a\_db: DB with missing rows.
%
%
\item[See also:]%
\hyperlink{ref_tests_db__isnanrows}{\texttt{tests\_db/isnanrows}}%
\ (p.~\pageref{ref_tests_db__isnanrows})%
\index[funcref]{tests_db@\fidxl{tests\_db}!isnanrows@\fidxl{isnanrows}}%
%
\item[Author:]%
Cengiz Gunay <cgunay@emory.edu>, 2005/09/21
%
\end{description}
\methodline%
\subsubsection[Method \texttt{onlyRowsTests}]{Method \texttt{tests\_db/onlyRowsTests}}%
\index[funcref]{tests_db@\fidxl{tests\_db}!onlyRowsTests@\fidxl{onlyRowsTests}}%
\label{ref_tests_db__onlyRowsTests}%
\hypertarget{ref_tests_db__onlyRowsTests}{}%
\begin{description}
\item[Summary:]Returns a tests\_db that only contains the desired 
		tests and rows (and pages).
%
\item[Usage:]~%
\begin{lyxcode}%
obj = onlyRowsTests(obj, rows, tests, pages)
%
\end{lyxcode}%
%
\item[Description:]%
Selects the given dimensions and returns in a new tests\_db object.
%%
\item[Parameters:]~
\begin{description}%
\item[\texttt{obj}:]
 A tests\_db object.
\item[\texttt{rows, tests}:]
 A logical or index vector of rows, or cell array of

names of rows. If ':', all rows. For names, regular expressions are
supported if quoted with slashes (e.g., '/a.*/'). See tests2idx.
\item[\texttt{pages}:]
 (Optional) A logical or index vector of pages. ':' for all pages.
\end{description}%
%
\item[Returns:
]~

	obj: The new tests\_db object.
%
%
\item[See also:]%
\hyperlink{ref_subsref}{\texttt{subsref}}%
\ (p.~\pageref{ref_subsref})%
\index[funcref]{subsref@\fidxl{subsref}}%
, \hyperlink{ref_tests_db}{\texttt{tests\_db}}%
\ (p.~\pageref{ref_tests_db})%
\index[funcref]{tests_db@\fidxl{tests\_db}}%
, \hyperlink{ref_tests2idx}{\texttt{tests2idx}}%
\ (p.~\pageref{ref_tests2idx})%
\index[funcref]{tests2idx@\fidxl{tests2idx}}%
, \hyperlink{ref_regexp}{\texttt{regexp}}%
\ (p.~\pageref{ref_regexp})%
\index[funcref]{regexp@\fidxl{regexp}}%
%
\item[Author:]%
Cengiz Gunay <cgunay@emory.edu>, 2004/09/17
%
\end{description}
\methodline%
\subsubsection[Method \texttt{physiol\_bundle}]{Method \texttt{tests\_db/physiol\_bundle}}%
\index[funcref]{tests_db@\fidxl{tests\_db}!physiol_bundle@\fidxl{physiol\_bundle}}%
\label{ref_tests_db__physiol_bundle}%
\hypertarget{ref_tests_db__physiol_bundle}{}%
\begin{description}
\item[Summary:]Create a physiol\_bundle from a raw physiology database.
%
\item[Usage:]~%
\begin{lyxcode}%
a\_pbundle = physiol\_bundle(phys\_dball, phys\_dataset, props)
%
\end{lyxcode}%
%
\item[Description:]%
Removes small bias currents, calculates input resistance by averaging
 negative CIP traces, averages multiple traces with similar treatments,
 selects certain CIP levels collapses its rows to create a
 one-neuron-per-pow database. It includes post-DB calculated columns
 such as rate ratios between spont and recovery periods.
%%
\item[Parameters:]~
\begin{description}%
\item[\texttt{phys\_dball}:]
 A raw database obtained by loading traces from the tracesets.
\item[\texttt{phys\_dataset}:]
 Dataset object passed to physiol\_bundle.
\item[\texttt{props}:]
 Optional parameters.
\begin{description}%
\item[\texttt{weedCols}:]
 Cell array of parameter columns to be weed-out before averaging rows

that are same w.r.t other parameters.
(default={'pulseOn', 'pulseOff', 'traceEnd', 'pAbias', 'ItemIndex'}).
\item[\texttt{drugCols}:]
 Cell array of drug names that need to be zero for the

control db (default={'TTX', 'Apamin', 'EBIO', 'XE991', 'Cadmium', 'drug\_4AP'}).
\end{description}%
\item[\texttt{CIPList}:]
 row array specifying the CIP levels to choose (eliminate the

others), default is an empty array, which means to choose all.
\begin{description}%
\item[\texttt{biasLimit}:]
 Limit in pA, biases larger +/- than which will be

eliminated. (default=30)
\end{description}%
\end{description}%
%
\item[Returns:
]~

	phys\_joined\_db: Final one row per cip and neuron db.
	phys\_joined\_control\_db: Rows where all drug treatments are zero.
	phys\_db: Original db only with parameter and including the weedCols.
%
%
\item[See also:]%
\hyperlink{ref_physiol_bundle}{\texttt{physiol\_bundle}}%
\ (p.~\pageref{ref_physiol_bundle})%
\index[funcref]{physiol_bundle@\fidxl{physiol\_bundle}}%
, \hyperlink{ref_params_tests_db}{\texttt{params\_tests\_db}}%
\ (p.~\pageref{ref_params_tests_db})%
\index[funcref]{params_tests_db@\fidxl{params\_tests\_db}}%
%
\item[Author:]%
Cengiz Gunay <cgunay@emory.edu>, 2007/12/21
%
\end{description}
\methodline%
\subsubsection[Method \texttt{plot}]{Method \texttt{tests\_db/plot}}%
\index[funcref]{tests_db@\fidxl{tests\_db}!plot@\fidxl{plot}}%
\label{ref_tests_db__plot}%
\hypertarget{ref_tests_db__plot}{}%
\begin{description}
\item[Summary:]Generic method to plot a tests\_db or a subclass. Requires a 
	plot\_abstract method to be defined for this object.
%
\item[Usage:]~%
\begin{lyxcode}%
h = plot(a\_tests\_db, title\_str, props)
%
\end{lyxcode}%
%
%
\item[Parameters:]~
\begin{description}%
\item[\texttt{a\_tests\_db}:]
 A histogram\_db object.
\item[\texttt{title\_str}:]
 (Optional) String to append to plot title.
\item[\texttt{props}:]
 A structure with any optional properties, passed to plot\_abstract.
\end{description}%
%
\item[Returns:
]~

	h: The figure handle created.
%
%
\item[See also:]%
\hyperlink{ref_plot_abstract}{\texttt{plot\_abstract}}%
\ (p.~\pageref{ref_plot_abstract})%
\index[funcref]{plot_abstract@\fidxl{plot\_abstract}}%
, \hyperlink{ref_plotFigure}{\texttt{plotFigure}}%
\ (p.~\pageref{ref_plotFigure})%
\index[funcref]{plotFigure@\fidxl{plotFigure}}%
%
\item[Author:]%
Cengiz Gunay <cgunay@emory.edu>, 2004/10/06
%
\end{description}
\methodline%
\subsubsection[Method \texttt{plotBox}]{Method \texttt{tests\_db/plotBox}}%
\index[funcref]{tests_db@\fidxl{tests\_db}!plotBox@\fidxl{plotBox}}%
\label{ref_tests_db__plotBox}%
\hypertarget{ref_tests_db__plotBox}{}%
\begin{description}
\item[Summary:]Creates a boxplot from each column in tests\_db in separate axes.
%
\item[Usage:]~%
\begin{lyxcode}%
a\_plot = plotBox(a\_tests\_db, title\_str, props)
%
\end{lyxcode}%
%
%
\item[Parameters:]~
\begin{description}%
\item[\texttt{a\_tests\_db}:]
 A tests\_db object.
\item[\texttt{title\_str}:]
 Optional title.
\item[\texttt{props}:]
 A structure with any optional properties.
\begin{description}%
\item[\texttt{putLabels}:]
 Put special column name labels.
\item[\texttt{notch}:]
 If 1, put notches on boxplots (default=1).
\item[\texttt{whis}:]
 Whisker size passed to boxplotp (default=1.5);
\end{description}%
\end{description}%
%
\item[Returns:
]~

	a\_plot: A plot\_abstract object that can be plotted.
%
%
\item[See also:]%
\hyperlink{ref_plot_abstract}{\texttt{plot\_abstract}}%
\ (p.~\pageref{ref_plot_abstract})%
\index[funcref]{plot_abstract@\fidxl{plot\_abstract}}%
, \hyperlink{ref_plotFigure}{\texttt{plotFigure}}%
\ (p.~\pageref{ref_plotFigure})%
\index[funcref]{plotFigure@\fidxl{plotFigure}}%
, \hyperlink{ref_boxplotp}{\texttt{boxplotp}}%
\ (p.~\pageref{ref_boxplotp})%
\index[funcref]{boxplotp@\fidxl{boxplotp}}%
%
\item[Author:]%
Cengiz Gunay <cgunay@emory.edu>, 2008/01/16
%
\end{description}
\methodline%
\subsubsection[Method \texttt{plotCircular}]{Method \texttt{tests\_db/plotCircular}}%
\index[funcref]{tests_db@\fidxl{tests\_db}!plotCircular@\fidxl{plotCircular}}%
\label{ref_tests_db__plotCircular}%
\hypertarget{ref_tests_db__plotCircular}{}%
\begin{description}
\item[Summary:]Circular plot.
%
\item[Usage:]~%
\begin{lyxcode}%
a\_p = plotCircular(a\_db, theta\_col, title\_str, short\_title, props)
%
\end{lyxcode}%
%
\item[Description:]%
Radius is taken to be constant on the unit circle.
%%
\item[Parameters:]~
\begin{description}%
\item[\texttt{a\_db}:]
 A tests\_db object.
\item[\texttt{theta\_col}:]
 Column with angle values to plot on circle.
\item[\texttt{title\_str}:]
 (Optional) A string to be concatanated to the title.
\item[\texttt{short\_title}:]
 (Optional) Few words that may appear in legends of multiplot.
\item[\texttt{props}:]
 A structure with any optional properties.
\begin{description}%
\item[\texttt{avgVector}:]
 If 1, plot an average vector from polar coordinates.
\item[\texttt{vectorMarker}:]
 Specify a marker for the average vector (default='.').
\item[\texttt{radius}:]
 The radius at which angles are plotted (default=1).
\item[\texttt{angles1}:]
 If 1, angles are in the range of 0-1, and they will be

converted to radians.
\item[\texttt{jitter}:]
 Add this much random jitter to radius while plotting.
\item[\texttt{quiet}:]
 If 1, don't include database name on title.
\end{description}%
\end{description}%
%
\item[Returns:
]~

   a\_p: A plot\_abstract.
%
%
\item[See also:]%
\hyperlink{ref_polar}{\texttt{polar}}%
\ (p.~\pageref{ref_polar})%
\index[funcref]{polar@\fidxl{polar}}%
, \hyperlink{ref_pol2cart}{\texttt{pol2cart}}%
\ (p.~\pageref{ref_pol2cart})%
\index[funcref]{pol2cart@\fidxl{pol2cart}}%
%
\item[Author:]%
Cengiz Gunay <cgunay@emory.edu>, 2014/07/14
%
\end{description}
\methodline%
\subsubsection[Method \texttt{plotCovar}]{Method \texttt{tests\_db/plotCovar}}%
\index[funcref]{tests_db@\fidxl{tests\_db}!plotCovar@\fidxl{plotCovar}}%
\label{ref_tests_db__plotCovar}%
\hypertarget{ref_tests_db__plotCovar}{}%
\begin{description}
\item[Summary:]Generates an image plot of the covariance-type values in a\_db.
%
\item[Usage:]~%
\begin{lyxcode}%
[a\_plot, shuffle\_idx] = plotCovar(a\_db, title\_str, props)
%
\end{lyxcode}%
%
%
\item[Parameters:]~
\begin{description}%
\item[\texttt{a\_db}:]
 A tests\_db object that resulted from a function like cov.
\item[\texttt{title\_str}:]
 (Optional) String to append to plot title.
\item[\texttt{props}:]
 Optional properties.
\begin{description}%
\item[\texttt{inverse}:]
 If 1, take inverse of the data matrix.
\item[\texttt{corrcoef}:]
 If 1, normalize matrix elements to get corrcoef values.
\item[\texttt{logScale}:]
 If 1, take logarithm of values before plotting.
\item[\texttt{localityIters}:]
 Apply a locality optimization algorithm with

this many iterations.
(rest passed to plot\_image.)
\end{description}%
\end{description}%
%
\item[Returns:
]~

	a\_plot: A plot\_abstract object or one of its subclasses.
       shuffle\_idx: (Optional) Ordering of columns found by locality optimization.
%
\item[Example:]~
\begin{lyxcode} >> plotFigure(plotCovar(cov(get(constrainedMeasuresPreset(pbundle2, 6), 'joined\_control\_db'))));
\\%
\end{lyxcode}
%
\item[See also:]%
\hyperlink{ref_tests_db__cov}{\texttt{tests\_db/cov}}%
\ (p.~\pageref{ref_tests_db__cov})%
\index[funcref]{tests_db@\fidxl{tests\_db}!cov@\fidxl{cov}}%
, \hyperlink{ref_plotImage}{\texttt{plotImage}}%
\ (p.~\pageref{ref_plotImage})%
\index[funcref]{plotImage@\fidxl{plotImage}}%
, \hyperlink{ref_tests_db__matchingRow}{\texttt{tests\_db/matchingRow}}%
\ (p.~\pageref{ref_tests_db__matchingRow})%
\index[funcref]{tests_db@\fidxl{tests\_db}!matchingRow@\fidxl{matchingRow}}%
, \hyperlink{ref_corrcoefs.}{\texttt{corrcoefs.}}%
\ (p.~\pageref{ref_corrcoefs.})%
\index[funcref]{corrcoefs.@\fidxl{corrcoefs.}}%
%
\item[Author:]%
Cengiz Gunay <cgunay@emory.edu>, 2007/05/30
%
\end{description}
\methodline%
\subsubsection[Method \texttt{plotImage}]{Method \texttt{tests\_db/plotImage}}%
\index[funcref]{tests_db@\fidxl{tests\_db}!plotImage@\fidxl{plotImage}}%
\label{ref_tests_db__plotImage}%
\hypertarget{ref_tests_db__plotImage}{}%
\begin{description}
\item[Summary:]Create an image plot of a measure changing with the given two parameters.
%
\item[Usage:]~%
\begin{lyxcode}%
a\_p = plotImage(a\_db, par1, par2, col, title\_str, short\_title, props)
%
\end{lyxcode}%
%
%
\item[Parameters:]~
\begin{description}%
\item[\texttt{a\_db}:]
 A tests\_db object.
\item[\texttt{par1, par2}:]
 X \& Y variables.
\item[\texttt{col}:]
 Plot this column.
\item[\texttt{title\_str}:]
 (Optional) A string to be concatanated to the title.
\item[\texttt{short\_title}:]
 (Optional) Few words that may appear in legends of multiplot.
\item[\texttt{props}:]
 A structure with any optional properties.
\begin{description}%
\item[\texttt{logScale}:]
 If 1, take logarithm of values before plotting.
\item[\texttt{quiet}:]
 If 1, don't include database name on title.
\end{description}%
\end{description}%
%
\item[Returns:
]~

	a\_p: A plot\_abstract.
%
%
\item[See also:]%
\hyperlink{ref_plotScatter}{\texttt{plotScatter}}%
\ (p.~\pageref{ref_plotScatter})%
\index[funcref]{plotScatter@\fidxl{plotScatter}}%
, \hyperlink{ref_plotScatter3D}{\texttt{plotScatter3D}}%
\ (p.~\pageref{ref_plotScatter3D})%
\index[funcref]{plotScatter3D@\fidxl{plotScatter3D}}%
%
\item[Author:]%
Cengiz Gunay <cgunay@emory.edu>, 2009/02/17
%
\end{description}
\methodline%
\subsubsection[Method \texttt{plotParamsCoverage}]{Method \texttt{tests\_db/plotParamsCoverage}}%
\index[funcref]{tests_db@\fidxl{tests\_db}!plotParamsCoverage@\fidxl{plotParamsCoverage}}%
\label{ref_tests_db__plotParamsCoverage}%
\hypertarget{ref_tests_db__plotParamsCoverage}{}%
\begin{description}
\item[Summary:]Lower triangle matrix of parameter-parameter combination scatter plots.
%
\item[Usage:]~%
\begin{lyxcode}%
a\_pm = plotParamsCoverage(a\_db, params, title\_str, props)
%
\end{lyxcode}%
%
%
\item[Parameters:]~
\begin{description}%
\item[\texttt{a\_db}:]
 A tests\_db or params\_tests\_db object. 
\item[\texttt{params}:]
 Columns to be used in the parameter coverage.
\item[\texttt{title\_str}:]
 (Optional) A string to be concatanated to the title.
\item[\texttt{props}:]
 A structure with any optional properties, passed to plot\_abstract.
\begin{description}%
\item[\texttt{colorTest}:]
 Use this column to specify colors of points (see

plotScatter for other props to control behavior).
\item[\texttt{quiet}:]
 Don't put the DB id on the title.

(rest passed to plotScatter)
\end{description}%
\end{description}%
%
\item[Returns:
]~

   a\_pm: Resulting plot\_stack object.
%
%
\item[See also:]%
\hyperlink{ref_plotScatter}{\texttt{plotScatter}}%
\ (p.~\pageref{ref_plotScatter})%
\index[funcref]{plotScatter@\fidxl{plotScatter}}%
%
\item[Author:]%
Cengiz Gunay <cgunay@emory.edu>, 2016/05/27
%
\end{description}
\methodline%
\subsubsection[Method \texttt{plotrow}]{Method \texttt{tests\_db/plotrow}}%
\index[funcref]{tests_db@\fidxl{tests\_db}!plotrow@\fidxl{plotrow}}%
\label{ref_tests_db__plotrow}%
\hypertarget{ref_tests_db__plotrow}{}%
\begin{description}
\item[Summary:]Creates a plot\_abstract describing the desired db row.
%
\item[Usage:]~%
\begin{lyxcode}%
a\_plot = plotrow(a\_tests\_db, row, title\_str, props)
%
\end{lyxcode}%
%
%
\item[Parameters:]~
\begin{description}%
\item[\texttt{a\_tests\_db}:]
 A tests\_db object.
\item[\texttt{row}:]
 Row number to visualize.
\item[\texttt{title\_str}:]
 Optional title string.
\item[\texttt{props}:]
 A structure with any optional properties.
\begin{description}%
\item[\texttt{putLabels}:]
 Put special column name labels.
\end{description}%
\end{description}%
%
\item[Returns:
]~

	a\_plot: A plot\_abstract object that can be plotted.
%
%
\item[See also:]%
\hyperlink{ref_plot_abstract}{\texttt{plot\_abstract}}%
\ (p.~\pageref{ref_plot_abstract})%
\index[funcref]{plot_abstract@\fidxl{plot\_abstract}}%
, \hyperlink{ref_plotFigure}{\texttt{plotFigure}}%
\ (p.~\pageref{ref_plotFigure})%
\index[funcref]{plotFigure@\fidxl{plotFigure}}%
%
\item[Author:]%
Cengiz Gunay <cgunay@emory.edu>, 2004/11/08
%
\end{description}
\methodline%
\subsubsection[Method \texttt{plotrows}]{Method \texttt{tests\_db/plotrows}}%
\index[funcref]{tests_db@\fidxl{tests\_db}!plotrows@\fidxl{plotrows}}%
\label{ref_tests_db__plotrows}%
\hypertarget{ref_tests_db__plotrows}{}%
\begin{description}
\item[Summary:]Creates a plot\_stack describing the db rows.
%
\item[Usage:]~%
\begin{lyxcode}%
a\_plot = plotrows(a\_tests\_db, axis\_limits, orientation, props)
%
\end{lyxcode}%
%
%
\item[Parameters:]~
\begin{description}%
\item[\texttt{a\_tests\_db}:]
 A tests\_db object.
\item[\texttt{axis\_limits}:]
 If given, all plots contained will have these axis limits.
\item[\texttt{orientation}:]
 Stack orientation 'x' for horizontal, 'y' for vertical, etc.
\item[\texttt{title\_str}:]
 Optional title string.
\item[\texttt{props}:]
 A structure with any optional properties passed to plot\_stack.
\end{description}%
%
\item[Returns:
]~

	a\_plot: A plot\_stack object that can be plotted.
%
%
\item[See also:]%
\hyperlink{ref_plot_abstract}{\texttt{plot\_abstract}}%
\ (p.~\pageref{ref_plot_abstract})%
\index[funcref]{plot_abstract@\fidxl{plot\_abstract}}%
, \hyperlink{ref_plotFigure}{\texttt{plotFigure}}%
\ (p.~\pageref{ref_plotFigure})%
\index[funcref]{plotFigure@\fidxl{plotFigure}}%
%
\item[Author:]%
Cengiz Gunay <cgunay@emory.edu>, 2004/11/09
%
\end{description}
\methodline%
\subsubsection[Method \texttt{plotScatter}]{Method \texttt{tests\_db/plotScatter}}%
\index[funcref]{tests_db@\fidxl{tests\_db}!plotScatter@\fidxl{plotScatter}}%
\label{ref_tests_db__plotScatter}%
\hypertarget{ref_tests_db__plotScatter}{}%
\begin{description}
\item[Summary:]Create a scatter plot of the given two tests.
%
\item[Usage:]~%
\begin{lyxcode}%
[a\_p, b] = plotScatter(a\_db, test1, test2, title\_str, short\_title, props)
%
\end{lyxcode}%
%
\item[Description:]%
Newer versions of Matlab (e.g. R2014b) won't allow line styles in
 a scatter plot. To draw lines between points, one can switch to the
 default 'plot' function using the 'command' prop (see above) - but at
 the expense of losing the ability to use different colors for each point.
%%
\item[Parameters:]~
\begin{description}%
\item[\texttt{a\_db}:]
 A tests\_db object.
\item[\texttt{test1, test2}:]
 X \& Y variables.
\item[\texttt{title\_str}:]
 (Optional) A string to be concatanated to the title.
\item[\texttt{short\_title}:]
 (Optional) Few words that may appear in legends of multiplot.
\item[\texttt{props}:]
 A structure with any optional properties.
\begin{description}%
\item[\texttt{LineStyle}:]
 Plot line style to use. (default: 'x')
\item[\texttt{Regress}:]
 If exists, use these as props for plotting linear
\begin{description}%
\item[\texttt{regression and displays statistics}:]
 R\textasciicircum{}2, F, p, and the error variance. 
\end{description}%
\item[\texttt{colorTest}:]
 Use this column as index into colormap.
\item[\texttt{colormap}:]
 Colormap vector, function name or handle to colormap (e.g., 'jet').
\item[\texttt{minValue, maxValue}:]
 Set boundaries for the displayed colors (see

plotColormap). Values outside of the boundaries will be truncated.
\item[\texttt{command}:]
 Plot command to use (default: 'scatter'). If 'plot' is

selected, colormap cannot be used.
\item[\texttt{numColors}:]
 Number of colors desired in colormap (default: 50).
\item[\texttt{quiet}:]
 If 1, don't include database name on title.
\item[\texttt{markerArea}:]
 Passed as the 'area' argument to scatter (default=36).

(Others passed to plotColormap and plot\_abstract).
\end{description}%
\end{description}%
%
\item[Returns:
]~

   a\_p: A plot\_abstract.
   b: A double holding the regression coefficient (optional)
%
%
\item[See also:]%
\hyperlink{ref_plotScatter3D}{\texttt{plotScatter3D}}%
\ (p.~\pageref{ref_plotScatter3D})%
\index[funcref]{plotScatter3D@\fidxl{plotScatter3D}}%
, \hyperlink{ref_plotImage}{\texttt{plotImage}}%
\ (p.~\pageref{ref_plotImage})%
\index[funcref]{plotImage@\fidxl{plotImage}}%
%
\item[Author:]%
Cengiz Gunay <cgunay@emory.edu>, 2005/09/29
%
\end{description}
\methodline%
\subsubsection[Method \texttt{plotScatter3D}]{Method \texttt{tests\_db/plotScatter3D}}%
\index[funcref]{tests_db@\fidxl{tests\_db}!plotScatter3D@\fidxl{plotScatter3D}}%
\label{ref_tests_db__plotScatter3D}%
\hypertarget{ref_tests_db__plotScatter3D}{}%
\begin{description}
\item[Summary:]Create a 3D scatter plot of the given three tests.
%
\item[Usage:]~%
\begin{lyxcode}%
a\_p = plotScatter3D(a\_db, test1, test2, test3, title\_str, short\_title, props)
%
\end{lyxcode}%
%
%
\item[Parameters:]~
\begin{description}%
\item[\texttt{a\_db}:]
 A params\_tests\_db object.
\item[\texttt{test1, test2, test3}:]
 X, Y, \& Z variables.
\item[\texttt{title\_str}:]
 (Optional) A string to be concatanated to the title.
\item[\texttt{short\_title}:]
 (Optional) Few words that may appear in legends of multiplot.
\item[\texttt{props}:]
 A structure with any optional properties.
\begin{description}%
\item[\texttt{LineStyle}:]
 Plot line style to use. (default: 'x')
\item[\texttt{Regress}:]
 Calculate and plot a linear regression.
\item[\texttt{quiet}:]
 If 1, don't include database name on title.
\end{description}%
\end{description}%
%
\item[Returns:
]~

	a\_p: A plot\_abstract.
%
%
\item[See also:]%
%
\item[Author:]%
Cengiz Gunay <cgunay@emory.edu>, 2007/11/30
%
\end{description}
\methodline%
\subsubsection[Method \texttt{plotTestsHistsMatrix}]{Method \texttt{tests\_db/plotTestsHistsMatrix}}%
\index[funcref]{tests_db@\fidxl{tests\_db}!plotTestsHistsMatrix@\fidxl{plotTestsHistsMatrix}}%
\label{ref_tests_db__plotTestsHistsMatrix}%
\hypertarget{ref_tests_db__plotTestsHistsMatrix}{}%
\begin{description}
\item[Summary:]Create a matrix plot of test histograms.
%
\item[Usage:]~%
\begin{lyxcode}%
a\_pm = plotTestsHistsMatrix(a\_db, title\_str, props)
%
\end{lyxcode}%
%
\item[Description:]%
Skips the 'ItemIndex' test. If no axisLimits is given in stackProps, 
 y-ranges are the maximal found from db.
%%
\item[Parameters:]~
\begin{description}%
\item[\texttt{a\_db}:]
 One or more params\_tests\_db object. Multiple databases in an

array will produce vertical stacks.
\item[\texttt{title\_str}:]
 (Optional) A string to be concatanated to the title.
\item[\texttt{props}:]
 A structure with any optional properties, passed to plot\_abstract.
\begin{description}%
\item[\texttt{orient}:]
 Orientation of the plot\_stack. 'x', 'y', or 'matrix' (default).
\item[\texttt{histBins}:]
 Number of histogram bins.
\item[\texttt{quiet}:]
 Don't put the DB id on the title.
\item[\texttt{plotProps}:]
 Props passed to individual plots.
\item[\texttt{stackProps}:]
 Passed to vertical plot stacks.
\end{description}%
\end{description}%
%
\item[Returns:
]~

   a\_pm: A plot\_stack with the plots organized in matrix form.
%
%
\item[See also:]%
\hyperlink{ref_params_tests_profile}{\texttt{params\_tests\_profile}}%
\ (p.~\pageref{ref_params_tests_profile})%
\index[funcref]{params_tests_profile@\fidxl{params\_tests\_profile}}%
, \hyperlink{ref_plotVar}{\texttt{plotVar}}%
\ (p.~\pageref{ref_plotVar})%
\index[funcref]{plotVar@\fidxl{plotVar}}%
%
\item[Author:]%
Cengiz Gunay <cgunay@emory.edu>, 2004/10/17
%
\end{description}
\methodline%
\subsubsection[Method \texttt{plotUITable}]{Method \texttt{tests\_db/plotUITable}}%
\index[funcref]{tests_db@\fidxl{tests\_db}!plotUITable@\fidxl{plotUITable}}%
\label{ref_tests_db__plotUITable}%
\hypertarget{ref_tests_db__plotUITable}{}%
\begin{description}
\item[Summary:]Display rows in figure table element.
%
\item[Usage:]~%
\begin{lyxcode}%
a\_p = plotUITable(a\_db, title\_str, props)
%
\end{lyxcode}%
%
%
\item[Parameters:]~
\begin{description}%
\item[\texttt{a\_db}:]
 A params\_tests\_db object.
\item[\texttt{title\_str}:]
 (Optional) A string to be concatanated to the title.
\item[\texttt{props}:]
 A structure with any optional properties
\end{description}%
%
\item[Returns:
]~

   a\_p: A plot\_abstract.
%
\item[Example:]~
\begin{lyxcode} >> plotFigure(plotUITable(my\_db(1:5, :), 'my DB'))
\\%
\end{lyxcode}
%
\item[See also:]%
\hyperlink{ref_displayRows}{\texttt{displayRows}}%
\ (p.~\pageref{ref_displayRows})%
\index[funcref]{displayRows@\fidxl{displayRows}}%
%
\item[Author:]%
Cengiz Gunay <cgunay@emory.edu>, 2014/10/22
%
\end{description}
\methodline%
\subsubsection[Method \texttt{plotUniquesStats2D}]{Method \texttt{tests\_db/plotUniquesStats2D}}%
\index[funcref]{tests_db@\fidxl{tests\_db}!plotUniquesStats2D@\fidxl{plotUniquesStats2D}}%
\label{ref_tests_db__plotUniquesStats2D}%
\hypertarget{ref_tests_db__plotUniquesStats2D}{}%
\begin{description}
\item[Summary:]2D image plot of the change in column mean for unique values of two other columns.
%
\item[Usage:]~%
\begin{lyxcode}%
an\_image\_plot = plotUniquesStats2D(a\_db, unique\_test1, unique\_test2, 
 					stat\_test, title\_str, props)
%
\end{lyxcode}%
%
%
\item[Parameters:]~
\begin{description}%
\item[\texttt{a\_db}:]
 A tests\_db.
\item[\texttt{unique\_test1, unique\_test2}:]
 Columns whose unique values make up the X

\& Y of the 2D image plot.
\item[\texttt{stat\_test}:]
 Column for which statsMeanSTD will be calculated for each

unique value.
\item[\texttt{props}:]
 A structure with any optional properties.
\begin{description}%
\item[\texttt{popMean}:]
 If specified, plot a dotted line specifying the

population mean. If NaN, calculate from given a\_db.
\item[\texttt{popDev}:]
 Use this value +/- to choose colorbar extents

(default=.3 or 2*STD if popMean=NaN).
\item[\texttt{colorbar}:]
 Show vertical colorbar axis (see plotImage).
\item[\texttt{uniqueVals1,uniqueVals2}:]
 Use these unique values for

unique\_test1,unique\_test2.
\item[\texttt{statsFunc}:]
 tests\_db/stats* method to use (default: statsMeanStd).
\item[\texttt{statsRow}:]
 The row to pick from statsFunc results (default: mean).

(rest passed to plotImage and plot\_abstract).
\end{description}%
\end{description}%
%
\item[Returns:
]~

	an\_image\_plot: A plot\_abstract object to be plotted.
%
\item[Example:]~
\begin{lyxcode} >> plotFigure(plotUniquesStats2D(triplet\_param\_success\_db, ...
\\%
               'F\_tau\_m', 'S\_tau\_m', 'successDefault', ...
\\%
                'accross triplets', 
\\%
                struct('fixedSize', [4 3], 'popMean', NaN, ...
\\%
                'colorbar', 1, 'quiet', 1, 'border', [0.03 0 0.03 0])))
\\%
\end{lyxcode}
%
\item[See also:]%
\hyperlink{ref_tests_db}{\texttt{tests\_db}}%
\ (p.~\pageref{ref_tests_db})%
\index[funcref]{tests_db@\fidxl{tests\_db}}%
, \hyperlink{ref_sortedUniqueValues}{\texttt{sortedUniqueValues}}%
\ (p.~\pageref{ref_sortedUniqueValues})%
\index[funcref]{sortedUniqueValues@\fidxl{sortedUniqueValues}}%
, \hyperlink{ref_statsMeanStd}{\texttt{statsMeanStd}}%
\ (p.~\pageref{ref_statsMeanStd})%
\index[funcref]{statsMeanStd@\fidxl{statsMeanStd}}%
, \hyperlink{ref_plot_abstract}{\texttt{plot\_abstract}}%
\ (p.~\pageref{ref_plot_abstract})%
\index[funcref]{plot_abstract@\fidxl{plot\_abstract}}%
, \hyperlink{ref_plotImage}{\texttt{plotImage}}%
\ (p.~\pageref{ref_plotImage})%
\index[funcref]{plotImage@\fidxl{plotImage}}%
%
\item[Author:]%
Cengiz Gunay <cgunay@emory.edu>, 2008/04/14
%
\end{description}
\methodline%
\subsubsection[Method \texttt{plotUniquesStatsBars}]{Method \texttt{tests\_db/plotUniquesStatsBars}}%
\index[funcref]{tests_db@\fidxl{tests\_db}!plotUniquesStatsBars@\fidxl{plotUniquesStatsBars}}%
\label{ref_tests_db__plotUniquesStatsBars}%
\hypertarget{ref_tests_db__plotUniquesStatsBars}{}%
\begin{description}
\item[Summary:]Creates a mean-STD bar plot of a column for unique values of another column.
%
\item[Usage:]~%
\begin{lyxcode}%
a\_bar\_plot = plotUniquesStatsBars(a\_db, unique\_test, stat\_test, title\_str, props)
%
\end{lyxcode}%
%
%
\item[Parameters:]~
\begin{description}%
\item[\texttt{a\_db}:]
 A tests\_db.
\item[\texttt{unique\_test}:]
 Column for which to generate bars for each of its unique values.
\item[\texttt{stat\_test}:]
 Column for which statsMeanSTD will be calculated for each bar.
\item[\texttt{props}:]
 A structure with any optional properties.
\begin{description}%
\item[\texttt{popMean}:]
 If specified, plot a dotted line specifying the

population mean. If NaN, calculated from a\_db.
\item[\texttt{yLims}:]
 two-element vector for specifying y axis limits showing

interesting part of the bar plot.
\item[\texttt{uniqueVals}:]
 Use these unique values for unique\_test.

(rest passed to plot\_bars [and plot\_superpose if popMean]).
\end{description}%
\end{description}%
%
\item[Returns:
]~

	a\_bar\_plot: A plot\_abstract object to be plotted.
%
\item[Example:]~
\begin{lyxcode} >> plotFigure(plotUniquesStatsBars(triplet\_param\_success\_db, 'F\_tau\_m', ...
\\%
                                'successDefault', 'across triplets', ...
\\%
                                struct('fixedSize', [4 2], 'yLims', [.7 .9], ...
\\%
                                       'popMean', 0.82, 'quiet', 1)))
\\%
\end{lyxcode}
%
\item[See also:]%
\hyperlink{ref_tests_db}{\texttt{tests\_db}}%
\ (p.~\pageref{ref_tests_db})%
\index[funcref]{tests_db@\fidxl{tests\_db}}%
, \hyperlink{ref_sortedUniqueValues}{\texttt{sortedUniqueValues}}%
\ (p.~\pageref{ref_sortedUniqueValues})%
\index[funcref]{sortedUniqueValues@\fidxl{sortedUniqueValues}}%
, \hyperlink{ref_statsMeanStd}{\texttt{statsMeanStd}}%
\ (p.~\pageref{ref_statsMeanStd})%
\index[funcref]{statsMeanStd@\fidxl{statsMeanStd}}%
, \hyperlink{ref_plot_bars}{\texttt{plot\_bars}}%
\ (p.~\pageref{ref_plot_bars})%
\index[funcref]{plot_bars@\fidxl{plot\_bars}}%
%
\item[Author:]%
Cengiz Gunay <cgunay@emory.edu>, 2008/04/14
%
\end{description}
\methodline%
\subsubsection[Method \texttt{plotUniquesStatsStacked3D}]{Method \texttt{tests\_db/plotUniquesStatsStacked3D}}%
\index[funcref]{tests_db@\fidxl{tests\_db}!plotUniquesStatsStacked3D@\fidxl{plotUniquesStatsStacked3D}}%
\label{ref_tests_db__plotUniquesStatsStacked3D}%
\hypertarget{ref_tests_db__plotUniquesStatsStacked3D}{}%
\begin{description}
\item[Summary:]Stack of 2D image plots of a column mean at unique values of three other columns.
%
\item[Usage:]~%
\begin{lyxcode}%
a\_stacked\_plot = plotUniquesStatsStacked3D(a\_db, unique\_test1, unique\_test2, 
 					unique\_test3, stat\_test, title\_str, props)
%
\end{lyxcode}%
%
%
\item[Parameters:]~
\begin{description}%
\item[\texttt{a\_db}:]
 A tests\_db.
\item[\texttt{unique\_test1, unique\_test2}:]
 Columns whose unique values make up the X

\& Y of the 2D image plot.
\item[\texttt{unique\_test3}:]
 Column whose unique values make up stacked dimension.
\item[\texttt{stat\_test}:]
 Column for which statsMeanSTD will be calculated for each

unique value.
\item[\texttt{props}:]
 A structure with any optional properties.

(rest passed to plotUniquesStats2D and plot\_stack).
\end{description}%
%
\item[Returns:
]~

	a\_stacked\_plot: A plot\_abstract object to be plotted.
%
%
\item[See also:]%
\hyperlink{ref_tests_db}{\texttt{tests\_db}}%
\ (p.~\pageref{ref_tests_db})%
\index[funcref]{tests_db@\fidxl{tests\_db}}%
, \hyperlink{ref_sortedUniqueValues}{\texttt{sortedUniqueValues}}%
\ (p.~\pageref{ref_sortedUniqueValues})%
\index[funcref]{sortedUniqueValues@\fidxl{sortedUniqueValues}}%
, \hyperlink{ref_statsMeanStd}{\texttt{statsMeanStd}}%
\ (p.~\pageref{ref_statsMeanStd})%
\index[funcref]{statsMeanStd@\fidxl{statsMeanStd}}%
, \hyperlink{ref_plot_abstract}{\texttt{plot\_abstract}}%
\ (p.~\pageref{ref_plot_abstract})%
\index[funcref]{plot_abstract@\fidxl{plot\_abstract}}%
, \hyperlink{ref_plotImage}{\texttt{plotImage}}%
\ (p.~\pageref{ref_plotImage})%
\index[funcref]{plotImage@\fidxl{plotImage}}%
%
\item[Author:]%
Cengiz Gunay <cgunay@emory.edu>, 2008/04/15
%
\end{description}
\methodline%
\subsubsection[Method \texttt{plotXRows}]{Method \texttt{tests\_db/plotXRows}}%
\index[funcref]{tests_db@\fidxl{tests\_db}!plotXRows@\fidxl{plotXRows}}%
\label{ref_tests_db__plotXRows}%
\hypertarget{ref_tests_db__plotXRows}{}%
\begin{description}
\item[Summary:]Create a scatter plot with a test versus the row numbers on the X-axis.
%
\item[Usage:]~%
\begin{lyxcode}%
a\_p = plotXRows(a\_db, test\_y, title\_str, short\_title, props)
%
\end{lyxcode}%
%
%
\item[Parameters:]~
\begin{description}%
\item[\texttt{a\_db}:]
 A params\_tests\_db object.
\item[\texttt{test\_y}:]
 Y variable.
\item[\texttt{title\_str}:]
 (Optional) A string to be concatanated to the title.
\item[\texttt{short\_title}:]
 (Optional) Few words that may appear in legends of multiplot.
\item[\texttt{props}:]
 A structure with any optional properties passed to plotScatter.
\begin{description}%
\item[\texttt{RowName}:]
 Label to show on X-axis, becomes a db column (default='RowNumber')
\item[\texttt{Vertical}:]
 If provided, put the rows on the Y-axis instead.
\end{description}%
\end{description}%
%
\item[Returns:
]~

   a\_p: A plot\_abstract.
%
%
\item[See also:]%
%
\item[Author:]%
Cengiz Gunay <cgunay@emory.edu>, 2007/01/16
%
\end{description}
\methodline%
\subsubsection[Method \texttt{plotYTests}]{Method \texttt{tests\_db/plotYTests}}%
\index[funcref]{tests_db@\fidxl{tests\_db}!plotYTests@\fidxl{plotYTests}}%
\label{ref_tests_db__plotYTests}%
\hypertarget{ref_tests_db__plotYTests}{}%
\begin{description}
\item[Summary:]Create a plot given database measures against given X-axis values, for each row.
%
\item[Usage:]~%
\begin{lyxcode}%
a\_p = plotYTests(a\_db, x\_vals, tests, axis\_labels, title\_str, short\_title, command, props)
%
\end{lyxcode}%
%
%
\item[Parameters:]~
\begin{description}%
\item[\texttt{a\_db}:]
 A params\_tests\_db object.
\item[\texttt{x\_vals}:]
 A vector of X-axis values.
\item[\texttt{tests}:]
 A vector or cell array of columns to correspond to each value from x\_vals.
\item[\texttt{axis\_labels}:]
 Cell array of X \& Y axis labels.
\item[\texttt{title\_str}:]
 (Optional) A string to be concatanated to the title.
\item[\texttt{short\_title}:]
 (Optional) Few words that may appear in legends of multiplot.
\item[\texttt{command}:]
 (Optional) Command to do the plotting with (default: 'plot')
\item[\texttt{props}:]
 A structure with any optional properties.
\begin{description}%
\item[\texttt{LineStyle}:]
 Plot line style to use. (default: 'd-')
\item[\texttt{ShowErrorbars}:]
 If 1, errorbars are added to each point.
\item[\texttt{StatsDB}:]
 If given, use this stats\_db for the errorbar (default=statsMeanStd(a\_db)).
\item[\texttt{jitterX}:]
 Randomly jitter x-axis locations by this magnitude.
\item[\texttt{quiet}:]
 If 1, don't include database name on title.
\end{description}%
\end{description}%
%
\item[Returns:
]~

   a\_p: A plot\_abstract.
%
\item[Example:]~
\begin{lyxcode} >> a\_p = plotYTests(a\_db\_row, [0 40 100 200], ...
\\%
                      {'IniSpontSpikeRateISI\_0pA', 'PulseIni100msSpikeRateISI\_D40pA', ...
\\%
                       'PulseIni100msSpikeRateISI\_D100pA', 'PulseIni100msSpikeRateISI\_D200pA'}, ...
\\%
                      {'current pulse [pA]', 'firing rate [Hz]'}, ', f-I curves', 'neuron 1');
\\%
 >> plotFigure(a\_p);
\\%
\end{lyxcode}
%
\item[See also:]%
\hyperlink{ref_plotFigure}{\texttt{plotFigure}}%
\ (p.~\pageref{ref_plotFigure})%
\index[funcref]{plotFigure@\fidxl{plotFigure}}%
%
\item[Author:]%
Cengiz Gunay <cgunay@emory.edu>, 2006/01/23
%
\end{description}
\methodline%
\subsubsection[Method \texttt{plot\_abstract}]{Method \texttt{tests\_db/plot\_abstract}}%
\index[funcref]{tests_db@\fidxl{tests\_db}!plot_abstract@\fidxl{plot\_abstract}}%
\label{ref_tests_db__plot_abstract}%
\hypertarget{ref_tests_db__plot_abstract}{}%
\begin{description}
\item[Summary:]Default visualization for a database.
%
\item[Usage:]~%
\begin{lyxcode}%
a\_pm = plot\_abstract(a\_db, title\_str)
%
\end{lyxcode}%
%
\item[Description:]%
Calls plotTestsHistsMatrix. Subclasses should override this method
 to provide their own visualization.
%%
\item[Parameters:]~
\begin{description}%
\item[\texttt{a\_db}:]
 A params\_tests\_db object.
\item[\texttt{title\_str}:]
 (Optional) A string to be concatanated to the title.
\item[\texttt{props}:]
 A structure with any optional properties.
\end{description}%
%
\item[Returns:
]~

	a\_pm: A plot\_stack with the plots organized in matrix form
%
\item[Example:]~
\begin{lyxcode}   >> plot(my\_db, ': first impression')
\\%
 will call this function and send the generated plot to the plotFigure function.
\\%
\end{lyxcode}
%
\item[See also:]%
\hyperlink{ref_plot_abstract__plot_abstract}{\texttt{plot\_abstract/plot\_abstract}}%
\ (p.~\pageref{ref_plot_abstract__plot_abstract})%
\index[funcref]{plot_abstract@\fidxl{plot\_abstract}!plot_abstract@\fidxl{plot\_abstract}}%
, \hyperlink{ref_plotTestsHistsMatrix}{\texttt{plotTestsHistsMatrix}}%
\ (p.~\pageref{ref_plotTestsHistsMatrix})%
\index[funcref]{plotTestsHistsMatrix@\fidxl{plotTestsHistsMatrix}}%
, \hyperlink{ref_plotFigure}{\texttt{plotFigure}}%
\ (p.~\pageref{ref_plotFigure})%
\index[funcref]{plotFigure@\fidxl{plotFigure}}%
%
\item[Author:]%
Cengiz Gunay <cgunay@emory.edu>, 2005/08/17
%
\end{description}
\methodline%
\subsubsection[Method \texttt{plot\_bars}]{Method \texttt{tests\_db/plot\_bars}}%
\index[funcref]{tests_db@\fidxl{tests\_db}!plot_bars@\fidxl{plot\_bars}}%
\label{ref_tests_db__plot_bars}%
\hypertarget{ref_tests_db__plot_bars}{}%
\begin{description}
\item[Summary:]Creates a bar graph comparing all DB rows in groups, with a separate axis for each column.
%
\item[Usage:]~%
\begin{lyxcode}%
a\_plot = plot\_bars(a\_tests\_db, title\_str, props)
%
\end{lyxcode}%
%
%
\item[Parameters:]~
\begin{description}%
\item[\texttt{a\_tests\_db}:]
 A tests\_db object.
\item[\texttt{title\_str}:]
 (Optional) The plot title.
\item[\texttt{props}:]
 A structure with any optional properties.
\end{description}%
%
\item[Returns:
]~

	a\_plot: A object of plot\_bars or one of its subclasses.
%
%
\item[See also:]%
\hyperlink{ref_plot_abstract}{\texttt{plot\_abstract}}%
\ (p.~\pageref{ref_plot_abstract})%
\index[funcref]{plot_abstract@\fidxl{plot\_abstract}}%
, \hyperlink{ref_plot_simple}{\texttt{plot\_simple}}%
\ (p.~\pageref{ref_plot_simple})%
\index[funcref]{plot_simple@\fidxl{plot\_simple}}%
%
\item[Author:]%
Cengiz Gunay <cgunay@emory.edu>, 2006/03/13
%
\end{description}
\methodline%
\subsubsection[Method \texttt{plus}]{Method \texttt{tests\_db/plus}}%
\index[funcref]{tests_db@\fidxl{tests\_db}!plus@\fidxl{plus}}%
\label{ref_tests_db__plus}%
\hypertarget{ref_tests_db__plus}{}%
\begin{description}
\item[Summary:]Adds a DB to another or to a scalar.
%
\item[Usage:]~%
\begin{lyxcode}%
a\_db = plus(left\_obj, right\_obj)
%
\end{lyxcode}%
%
\item[Description:]%
If DBs have mismatching columns only the common columns will be kept.
 In any case, the resulting DB columns will be sorted in the order of the
 left-hand-side DB.
%%
\item[Parameters:]~
\begin{description}%
\item[\texttt{left\_obj, right\_obj}:]
 Operands of the addition. One must be of type tests\_db

and the other can be a scalar or tests\_db.
\end{description}%
%
\item[Returns:
]~

	a\_db: The resulting tests\_db.
%
%
\item[See also:]%
\hyperlink{ref_plus}{\texttt{plus}}%
\ (p.~\pageref{ref_plus})%
\index[funcref]{plus@\fidxl{plus}}%
%
\item[Author:]%
Cengiz Gunay <cgunay@emory.edu>, 2007/12/13
%
\end{description}
\methodline%
\subsubsection[Method \texttt{princomp}]{Method \texttt{tests\_db/princomp}}%
\index[funcref]{tests_db@\fidxl{tests\_db}!princomp@\fidxl{princomp}}%
\label{ref_tests_db__princomp}%
\hypertarget{ref_tests_db__princomp}{}%
\begin{description}
\item[Summary:]Generates a database of the principal components of given DB.
%
\item[Usage:]~%
\begin{lyxcode}%
a\_pca\_db = princomp(db, props)
%
\end{lyxcode}%
%
%
\item[Parameters:]~
\begin{description}%
\item[\texttt{db}:]
 A tests\_db object.
\item[\texttt{props}:]
 A structure with any optional properties.
\begin{description}%
\item[\texttt{normalized}:]
 If specified zscore is used before princomp.
\end{description}%
\end{description}%
%
\item[Returns:
]~

	a\_pca\_db: A tests\_db where each row is a principal component.
%
%
\item[See also:]%
\hyperlink{ref_princomp}{\texttt{princomp}}%
\ (p.~\pageref{ref_princomp})%
\index[funcref]{princomp@\fidxl{princomp}}%
, \hyperlink{ref_zscore}{\texttt{zscore}}%
\ (p.~\pageref{ref_zscore})%
\index[funcref]{zscore@\fidxl{zscore}}%
%
\item[Author:]%
Cengiz Gunay <cgunay@emory.edu>, 2005/09/21
%
\end{description}
\methodline%
\subsubsection[Method \texttt{processDimNonNaNInf}]{Method \texttt{tests\_db/processDimNonNaNInf}}%
\index[funcref]{tests_db@\fidxl{tests\_db}!processDimNonNaNInf@\fidxl{processDimNonNaNInf}}%
\label{ref_tests_db__processDimNonNaNInf}%
\hypertarget{ref_tests_db__processDimNonNaNInf}{}%
\begin{description}
\item[Summary:]Recursively process the specified dimension with the desired function after removing NaNs and Infs.
%
\item[Usage:]~%
\begin{lyxcode}%
[a\_db, n, i] = processDimNonNaNInf(a\_db, dim, a\_func, a\_func\_name)
%
\end{lyxcode}%
%
\item[Description:]%
Does a recursive operation over other dimensions in order to remove
 NaN and Inf values. This takes more time than applying the function directly. 
%%
\item[Parameters:]~
\begin{description}%
\item[\texttt{a\_db}:]
 A tests\_db object.
\item[\texttt{dim}:]
 Work down dimension (see mean).
\item[\texttt{a\_func}:]
 A function name or handle to be passed to feval that

takes the data as the first argument and dimension to
work as second.
\item[\texttt{a\_func\_name}:]
 (Optional) A name to add to the id of a\_db.
\end{description}%
%
\item[Returns:
]~

   a\_db: The DB with one row of max values, with selected dimension
	replaced by the output of the given function.
   n: (Optional) Numbers of used values in each call of a\_func.
   i: (Optional) Indices returned by a\_func.
%
\item[Example:]~
\begin{lyxcode} a\_db = tests\_3D\_db(rand(5, 5, 5));
\\%
 >> b\_db = processDimNonNaNInf(a\_db, 1, 'mean')
\\%
 will find the mean of rows in each page of the random 3D matrix.
\\%
 >> b\_db = processDimNonNaNInf(a\_db, 1, @(x,y)(max(x, [], y)), 'max')
\\%
 more complex function form with 'max'.
\\%
\end{lyxcode}
%
\item[See also:]%
\hyperlink{ref_max}{\texttt{max}}%
\ (p.~\pageref{ref_max})%
\index[funcref]{max@\fidxl{max}}%
, \hyperlink{ref_mean}{\texttt{mean}}%
\ (p.~\pageref{ref_mean})%
\index[funcref]{mean@\fidxl{mean}}%
, \hyperlink{ref_feval}{\texttt{feval}}%
\ (p.~\pageref{ref_feval})%
\index[funcref]{feval@\fidxl{feval}}%
, \hyperlink{ref_tests_db}{\texttt{tests\_db}}%
\ (p.~\pageref{ref_tests_db})%
\index[funcref]{tests_db@\fidxl{tests\_db}}%
%
\item[Author:]%
Cengiz Gunay <cgunay@emory.edu>, 2008/05/27
%
\end{description}
\methodline%
\subsubsection[Method \texttt{rankMatching}]{Method \texttt{tests\_db/rankMatching}}%
\index[funcref]{tests_db@\fidxl{tests\_db}!rankMatching@\fidxl{rankMatching}}%
\label{ref_tests_db__rankMatching}%
\hypertarget{ref_tests_db__rankMatching}{}%
\begin{description}
\item[Summary:]Create a ranking db of row distances of db to given criterion db.
%
\item[Usage:]~%
\begin{lyxcode}%
a\_ranked\_db = rankMatching(db, crit\_db, props)
%
\end{lyxcode}%
%
\item[Description:]%
The crit\_db parameter can be created with the matchingRow
 method. TestWeights modify the importance of each measure. Rows containing
 NaNs can be removed using noNaNRows before calling rankMatching. 
 Warning: existing RowIndex column in db will be replaced with the
 ranking RowIndex used for joinOriginal.
%%
\item[Parameters:]~
\begin{description}%
\item[\texttt{db}:]
 A tests\_db to rank.
\item[\texttt{crit\_db}:]
 A tests\_db object holding the match criterion tests and stds.
\item[\texttt{props}:]
 A structure with any optional properties.
\begin{description}%
\item[\texttt{limitSTD}:]
 Truncate error values at this many STDs.
\item[\texttt{tolerateNaNs}:]
 Multiplied by 3xSTD to replace NaN values. 0 means

to skip NaNs in the distance calculation, scaling sum of
errors by number of non-NaN entries. Negative values
accepted; -1 means -3xSTDs error (default=+1).
\item[\texttt{testWeights}:]
 Structure array associating tests and multiplicative weights.
\item[\texttt{restoreWeights}:]
 Reverse the testWeights application after

calculating distances.
\item[\texttt{topRows}:]
 If given, only return this many of the top rows.
\item[\texttt{useMahal}:]
 Use the Mahalonobis distance from the covariance

matrix in crit\_db.
\end{description}%
\end{description}%
%
\item[Returns:
]~

   a\_ranked\_db: A ranked\_db object.
%
\item[Example:]~
\begin{lyxcode} Select a target row from mesured distribution (e.g., experimental data):
\\%
 >> a\_crit\_db = matchingRow(exp\_db, 12);
\\%
 Rank another database by comparing to selected row:
\\%
 >> a\_ranked\_db = rankMatching(orig\_db, a\_crit\_db);
\\%
 Look at top ranked rows:
\\%
 >> displayRows(a\_ranked\_db(1:5, :))
\\%
\end{lyxcode}
%
\item[See also:]%
\hyperlink{ref_matchingRow}{\texttt{matchingRow}}%
\ (p.~\pageref{ref_matchingRow})%
\index[funcref]{matchingRow@\fidxl{matchingRow}}%
, \hyperlink{ref_joinOriginal}{\texttt{joinOriginal}}%
\ (p.~\pageref{ref_joinOriginal})%
\index[funcref]{joinOriginal@\fidxl{joinOriginal}}%
, \hyperlink{ref_tests_db}{\texttt{tests\_db}}%
\ (p.~\pageref{ref_tests_db})%
\index[funcref]{tests_db@\fidxl{tests\_db}}%
, \hyperlink{ref_noNaNRows}{\texttt{noNaNRows}}%
\ (p.~\pageref{ref_noNaNRows})%
\index[funcref]{noNaNRows@\fidxl{noNaNRows}}%
%
\item[Author:]%
Cengiz Gunay <cgunay@emory.edu>, 2004/12/08
%
\end{description}
\methodline%
\subsubsection[Method \texttt{rdivide}]{Method \texttt{tests\_db/rdivide}}%
\index[funcref]{tests_db@\fidxl{tests\_db}!rdivide@\fidxl{rdivide}}%
\label{ref_tests_db__rdivide}%
\hypertarget{ref_tests_db__rdivide}{}%
\begin{description}
\item[Summary:]Adds a DB to another or to a scalar.
%
\item[Usage:]~%
\begin{lyxcode}%
a\_db = rdivide(left\_obj, right\_obj)
%
\end{lyxcode}%
%
\item[Description:]%
If DBs have mismatching columns only the common columns will be kept.
 In any case, the resulting DB columns will be sorted in the order of the
 left-hand-side DB.
%%
\item[Parameters:]~
\begin{description}%
\item[\texttt{left\_obj, right\_obj}:]
 Operands of the addition. One must be of type tests\_db

and the other can be a scalar or tests\_db.
\end{description}%
%
\item[Returns:
]~

	a\_db: The resulting tests\_db.
%
%
\item[See also:]%
\hyperlink{ref_rdivide}{\texttt{rdivide}}%
\ (p.~\pageref{ref_rdivide})%
\index[funcref]{rdivide@\fidxl{rdivide}}%
%
\item[Author:]%
Cengiz Gunay <cgunay@emory.edu>, 2007/12/13
%
\end{description}
\methodline%
\subsubsection[Method \texttt{renameColumns}]{Method \texttt{tests\_db/renameColumns}}%
\index[funcref]{tests_db@\fidxl{tests\_db}!renameColumns@\fidxl{renameColumns}}%
\label{ref_tests_db__renameColumns}%
\hypertarget{ref_tests_db__renameColumns}{}%
\begin{description}
\item[Summary:]Rename one or more existing columns.
%
\item[Usage:]~%
\begin{lyxcode}%
a\_db = renameColumns(a\_db, old\_names, new\_names)
%
\end{lyxcode}%
%
\item[Description:]%
This is a cheap operation than modifies meta-data kept in object. For
 the regular expression renaming, the old\_names and new\_names
 parameters are passed to the regexprep command after removing the
 delimiting slashes (//). At least one grouping construct ('()') must be
 used in the search pattern such that it can be used in the replacement
 pattern (e.g., '\$1'). See example above. This function uses the generic
 renameIdx that can work on row, column, or page indices.
%%
\item[Parameters:]~
\begin{description}%
\item[\texttt{a\_db}:]
 A tests\_db object.
\item[\texttt{old\_names}:]
 A cell array of existing names, array of numerical indices, or a regular

expression denoted between slashes (e.g., '/(.*)/').
\item[\texttt{new\_names}:]
 New names to replace existing ones OR regular expression

replace string (no slashes, e.g, '\$1\_test'). See regexprep command.
\end{description}%
%
\item[Returns:
]~

   a\_db: The tests\_db object that includes the new columns.
%
\item[Example:]~
\begin{lyxcode} % Renaming a single column:
\\%
 >> new\_db = renameColumns(a\_db, 'PulseIni100msSpikeRateISI\_D40pA', 'Firing\_rate');
\\%
 % Renaming an unnamed column:
\\%
 >> new\_db = renameColumns(a\_db, 1, 'Firing\_rate');
\\%
 % Renaming using regular expressions: add suffix to all columns
\\%
 >> new\_db = renameColumns(a\_db, '/(.*)/', '\$1\_old');
\\%
 % Renaming multiple columns:
\\%
 >> new\_db = renameColumns(a\_db, {'a', 'b'}, {'c', 'd'});
\\%
\end{lyxcode}
%
\item[See also:]%
\hyperlink{ref_renameIdx}{\texttt{renameIdx}}%
\ (p.~\pageref{ref_renameIdx})%
\index[funcref]{renameIdx@\fidxl{renameIdx}}%
, \hyperlink{ref_regexprep}{\texttt{regexprep}}%
\ (p.~\pageref{ref_regexprep})%
\index[funcref]{regexprep@\fidxl{regexprep}}%
, \hyperlink{ref_allocateRows}{\texttt{allocateRows}}%
\ (p.~\pageref{ref_allocateRows})%
\index[funcref]{allocateRows@\fidxl{allocateRows}}%
, \hyperlink{ref_tests_db}{\texttt{tests\_db}}%
\ (p.~\pageref{ref_tests_db})%
\index[funcref]{tests_db@\fidxl{tests\_db}}%
%
\item[Author:]%
Cengiz Gunay <cgunay@emory.edu>, 2017/06/09
%
\end{description}
\methodline%
\subsubsection[Method \texttt{renameRows}]{Method \texttt{tests\_db/renameRows}}%
\index[funcref]{tests_db@\fidxl{tests\_db}!renameRows@\fidxl{renameRows}}%
\label{ref_tests_db__renameRows}%
\hypertarget{ref_tests_db__renameRows}{}%
\begin{description}
\item[Summary:]Rename one or more existing rows.
%
\item[Usage:]~%
\begin{lyxcode}%
a\_db = renameRows(a\_db, old\_names, new\_names)
%
\end{lyxcode}%
%
\item[Description:]%
This is a cheap operation than modifies meta-data kept in object. For
 the regular expression renaming, the old\_names and new\_names
 parameters are passed to the regexprep command after removing the
 delimiting slashes (//). At least one grouping construct ('()') must be
 used in the search pattern such that it can be used in the replacement
 pattern (e.g., '\$1'). See example above. This function uses the generic
 renameIdx that can work on row, column, or page indices.
%%
\item[Parameters:]~
\begin{description}%
\item[\texttt{a\_db}:]
 A tests\_db object.
\item[\texttt{old\_names}:]
 A cell array of existing names, array of numerical indices, or a regular

expression denoted between slashes (e.g., '/(.*)/').
\item[\texttt{new\_names}:]
 New names to replace existing ones OR regular expression

replace string (no slashes, e.g, '\$1\_test'). See regexprep command.
\end{description}%
%
\item[Returns:
]~

   a\_db: The tests\_db object that includes the new rows.
%
\item[Example:]~
\begin{lyxcode} % Renaming a single row:
\\%
 >> new\_db = renameRows(a\_db, 'PulseIni100msSpikeRateISI\_D40pA', 'Firing\_rate');
\\%
 % Renaming an unnamed row:
\\%
 >> new\_db = renameRows(a\_db, 1, 'Firing\_rate');
\\%
 % Renaming using regular expressions: add suffix to all rows
\\%
 >> new\_db = renameRows(a\_db, '/(.*)/', '\$1\_old');
\\%
 % Renaming multiple rows:
\\%
 >> new\_db = renameRows(a\_db, {'a', 'b'}, {'c', 'd'});
\\%
\end{lyxcode}
%
\item[See also:]%
\hyperlink{ref_renameIdx}{\texttt{renameIdx}}%
\ (p.~\pageref{ref_renameIdx})%
\index[funcref]{renameIdx@\fidxl{renameIdx}}%
, \hyperlink{ref_regexprep}{\texttt{regexprep}}%
\ (p.~\pageref{ref_regexprep})%
\index[funcref]{regexprep@\fidxl{regexprep}}%
, \hyperlink{ref_allocateRows}{\texttt{allocateRows}}%
\ (p.~\pageref{ref_allocateRows})%
\index[funcref]{allocateRows@\fidxl{allocateRows}}%
, \hyperlink{ref_tests_db}{\texttt{tests\_db}}%
\ (p.~\pageref{ref_tests_db})%
\index[funcref]{tests_db@\fidxl{tests\_db}}%
%
\item[Author:]%
Cengiz Gunay <cgunay@emory.edu>, 2017/06/09
%
\end{description}
\methodline%
\subsubsection[Method \texttt{rop}]{Method \texttt{tests\_db/rop}}%
\index[funcref]{tests_db@\fidxl{tests\_db}!rop@\fidxl{rop}}%
\label{ref_tests_db__rop}%
\hypertarget{ref_tests_db__rop}{}%
\begin{description}
\item[Summary:]Prepares aligned columns in two DBs or one DB and a scalar for an array arithmetic operation.
%
\item[Usage:]~%
\begin{lyxcode}%
a\_db = rop(left\_obj, right\_obj, op\_func, op\_id)
%
\end{lyxcode}%
%
\item[Description:]%
If DBs have mismatching columns only the common columns will be kept.
 In any case, the resulting DB columns will be sorted in the order of the
 left-hand-side DB. Array addition (plus), subtraction (minus),
 multiplication (mtimes) and division (rdivide) use this function to
 align columns.
%%
\item[Parameters:]~
\begin{description}%
\item[\texttt{left\_obj, right\_obj}:]
 Operands of the operation. One must be of type tests\_db

and the other can be a scalar or tests\_db.
\item[\texttt{op\_func}:]
 Operation function (e.g., @plus).
\item[\texttt{op\_id}:]
 A string to represent the operation that will show up in the

returned id.
\end{description}%
%
\item[Returns:
]~

   a\_db: The resulting tests\_db.
%
%
\item[See also:]%
\hyperlink{ref_tests_db__plus}{\texttt{tests\_db/plus}}%
\ (p.~\pageref{ref_tests_db__plus})%
\index[funcref]{tests_db@\fidxl{tests\_db}!plus@\fidxl{plus}}%
, \hyperlink{ref_tests_db__minus}{\texttt{tests\_db/minus}}%
\ (p.~\pageref{ref_tests_db__minus})%
\index[funcref]{tests_db@\fidxl{tests\_db}!minus@\fidxl{minus}}%
, \hyperlink{ref_tests_db__mtimes}{\texttt{tests\_db/mtimes}}%
\ (p.~\pageref{ref_tests_db__mtimes})%
\index[funcref]{tests_db@\fidxl{tests\_db}!mtimes@\fidxl{mtimes}}%
, \hyperlink{ref_tests_db__rdivide}{\texttt{tests\_db/rdivide}}%
\ (p.~\pageref{ref_tests_db__rdivide})%
\index[funcref]{tests_db@\fidxl{tests\_db}!rdivide@\fidxl{rdivide}}%
%
\item[Author:]%
Cengiz Gunay <cgunay@emory.edu>, 2007/12/13
%
\end{description}
\methodline%
\subsubsection[Method \texttt{rows2Struct}]{Method \texttt{tests\_db/rows2Struct}}%
\index[funcref]{tests_db@\fidxl{tests\_db}!rows2Struct@\fidxl{rows2Struct}}%
\label{ref_tests_db__rows2Struct}%
\hypertarget{ref_tests_db__rows2Struct}{}%
\begin{description}
\item[Summary:]Convert given rows of database to a structure array.
%
\item[Usage:]~%
\begin{lyxcode}%
s = rows2Struct(db, rows, pages)
%
\end{lyxcode}%
%
%
\item[Parameters:]~
\begin{description}%
\item[\texttt{db}:]
 A tests\_db object.
\item[\texttt{rows}:]
 Indices of rows in db.
\item[\texttt{pages}:]
 Pages of db.
\end{description}%
%
\item[Returns:
]~

	s: A structure of column name and value pairs.
%
%
\item[See also:]%
\hyperlink{ref_tests_db}{\texttt{tests\_db}}%
\ (p.~\pageref{ref_tests_db})%
\index[funcref]{tests_db@\fidxl{tests\_db}}%
%
\item[Author:]%
Cengiz Gunay <cgunay@emory.edu>, 2005/08/17
%
\end{description}
\methodline%
\subsubsection[Method \texttt{set}]{Method \texttt{tests\_db/set}}%
\index[funcref]{tests_db@\fidxl{tests\_db}!set@\fidxl{set}}%
\label{ref_tests_db__set}%
\hypertarget{ref_tests_db__set}{}%
\begin{description}
\item[Summary:]Generic method for setting object attributes.
%
%
%
%
%
%
%
\item[Author:]%
Cengiz Gunay <cgunay@emory.edu>, 2004/10/08
%
\end{description}
\methodline%
\subsubsection[Method \texttt{setProp}]{Method \texttt{tests\_db/setProp}}%
\index[funcref]{tests_db@\fidxl{tests\_db}!setProp@\fidxl{setProp}}%
\label{ref_tests_db__setProp}%
\hypertarget{ref_tests_db__setProp}{}%
\begin{description}
\item[Summary:]Generic method for setting optional object properties.
%
\item[Usage:]~%
\begin{lyxcode}%
obj = setProp(obj, prop1, val1, prop2, val2, ...)
%
\end{lyxcode}%
%
\item[Description:]%
Modifies or adds property values. As many property name-value 
 pairs can be specified.
%%
\item[Parameters:]~
\begin{description}%
\item[\texttt{obj}:]
 Any object that has a props field.
\item[\texttt{attr}:]
 Property name
\item[\texttt{val}:]
 Property value.
\end{description}%
%
\item[Returns:
]~

	obj: The new object with the updated properties.
%
%
\item[See also:]%
%
\item[Author:]%
Cengiz Gunay <cgunay@emory.edu>, 2004/11/22
%
\end{description}
\methodline%
\subsubsection[Method \texttt{setRows}]{Method \texttt{tests\_db/setRows}}%
\index[funcref]{tests_db@\fidxl{tests\_db}!setRows@\fidxl{setRows}}%
\label{ref_tests_db__setRows}%
\hypertarget{ref_tests_db__setRows}{}%
\begin{description}
\item[Summary:]Sets the rows of observations in tests\_db.
%
\item[Usage:]~%
\begin{lyxcode}%
index = setRows(obj, rows)
%
\end{lyxcode}%
%
\item[Description:]%
Sets a new set of observations to the database and returns the new DB.
%%
\item[Parameters:]~
\begin{description}%
\item[\texttt{obj}:]
 A tests\_db object.
\item[\texttt{rows}:]
 A matrix that contains rows for the DB.
\end{description}%
%
\item[Returns:
]~

	obj: The tests\_db object with the new rows.
%
%
\item[See also:]%
\hyperlink{ref_allocateRows}{\texttt{allocateRows}}%
\ (p.~\pageref{ref_allocateRows})%
\index[funcref]{allocateRows@\fidxl{allocateRows}}%
, \hyperlink{ref_addRow}{\texttt{addRow}}%
\ (p.~\pageref{ref_addRow})%
\index[funcref]{addRow@\fidxl{addRow}}%
, \hyperlink{ref_tests_db}{\texttt{tests\_db}}%
\ (p.~\pageref{ref_tests_db})%
\index[funcref]{tests_db@\fidxl{tests\_db}}%
%
\item[Author:]%
Cengiz Gunay <cgunay@emory.edu>, 2004/09/08
%
\end{description}
\methodline%
\subsubsection[Method \texttt{shufflecols}]{Method \texttt{tests\_db/shufflecols}}%
\index[funcref]{tests_db@\fidxl{tests\_db}!shufflecols@\fidxl{shufflecols}}%
\label{ref_tests_db__shufflecols}%
\hypertarget{ref_tests_db__shufflecols}{}%
\begin{description}
\item[Summary:]Returns a db with shuffled columns of given rows. 
%
\item[Usage:]~%
\begin{lyxcode}%
s = shufflecols(db, rows, grouped)
%
\end{lyxcode}%
%
\item[Description:]%
Can be used for shuffle prediction. Basically, shuffle columns of tests and rerun
 high order analyses. 
%%
\item[Parameters:]~
\begin{description}%
\item[\texttt{db}:]
 A tests\_db object.
\item[\texttt{rows}:]
 Rows to shuffle.
\item[\texttt{grouped}:]
 If 1 then apply same shuffling to all rows, 

if 0 shuffle each row independently (default=0).
\end{description}%
%
\item[Returns:
]~

	a\_db: The shuffled db.
%
%
\item[See also:]%
\hyperlink{ref_tests_db}{\texttt{tests\_db}}%
\ (p.~\pageref{ref_tests_db})%
\index[funcref]{tests_db@\fidxl{tests\_db}}%
%
\item[Author:]%
Cengiz Gunay <cgunay@emory.edu>, 2015/11/19
%
\end{description}
\methodline%
\subsubsection[Method \texttt{shufflerows}]{Method \texttt{tests\_db/shufflerows}}%
\index[funcref]{tests_db@\fidxl{tests\_db}!shufflerows@\fidxl{shufflerows}}%
\label{ref_tests_db__shufflerows}%
\hypertarget{ref_tests_db__shufflerows}{}%
\begin{description}
\item[Summary:]Returns a db with shuffled rows of given columns. 
%
\item[Usage:]~%
\begin{lyxcode}%
s = shufflerows(db, tests, grouped)
%
\end{lyxcode}%
%
\item[Description:]%
Can be used for shuffle prediction. Basically, shuffle rows of tests and rerun
 high order analyses. 
%%
\item[Parameters:]~
\begin{description}%
\item[\texttt{db}:]
 A tests\_db object.
\item[\texttt{tests}:]
 Columns to shuffle.
\item[\texttt{grouped}:]
 If 1 then shuffle tests all together, 

if 0 shuffle each test separately (default=0).
\end{description}%
%
\item[Returns:
]~

	a\_db: The shuffled db.
%
%
\item[See also:]%
\hyperlink{ref_tests_db}{\texttt{tests\_db}}%
\ (p.~\pageref{ref_tests_db})%
\index[funcref]{tests_db@\fidxl{tests\_db}}%
%
\item[Author:]%
Cengiz Gunay <cgunay@emory.edu>, 2004/11/10
%
\end{description}
\methodline%
\subsubsection[Method \texttt{sortrows}]{Method \texttt{tests\_db/sortrows}}%
\index[funcref]{tests_db@\fidxl{tests\_db}!sortrows@\fidxl{sortrows}}%
\label{ref_tests_db__sortrows}%
\hypertarget{ref_tests_db__sortrows}{}%
\begin{description}
\item[Summary:]Returns a sorted\_db according to given columns. 
%
\item[Usage:]~%
\begin{lyxcode}%
[sorted\_db, idx] = sortrows(db, cols, props)
%
\end{lyxcode}%
%
\item[Description:]%
WARNING: For multi-page dbs, sorts only the first page and applies the ordering 
 to all other pages which may produce wrong results for some applications.
%%
\item[Parameters:]~
\begin{description}%
\item[\texttt{db}:]
 A tests\_db object.
\item[\texttt{cols}:]
 Columns to use for sorting.
\item[\texttt{props}:]
 Structure of optional parameters.
\begin{description}%
\item[\texttt{reverse}:]
 If exists, sort in reverse order.
\end{description}%
\end{description}%
%
\item[Returns:
]~

	sorted\_db: The sorted tests\_db.
	idx: The row index permutation vector, such that sorted\_db = db(idx, :). 
%
%
\item[See also:]%
\hyperlink{ref_sortrows}{\texttt{sortrows}}%
\ (p.~\pageref{ref_sortrows})%
\index[funcref]{sortrows@\fidxl{sortrows}}%
, \hyperlink{ref_tests_db}{\texttt{tests\_db}}%
\ (p.~\pageref{ref_tests_db})%
\index[funcref]{tests_db@\fidxl{tests\_db}}%
%
\item[Author:]%
Cengiz Gunay <cgunay@emory.edu>, 2004/10/11
%
\end{description}
\methodline%
\subsubsection[Method \texttt{sqrt}]{Method \texttt{tests\_db/sqrt}}%
\index[funcref]{tests_db@\fidxl{tests\_db}!sqrt@\fidxl{sqrt}}%
\label{ref_tests_db__sqrt}%
\hypertarget{ref_tests_db__sqrt}{}%
\begin{description}
\item[Summary:]Takes the square root of a\_db.
%
\item[Usage:]~%
\begin{lyxcode}%
a\_db = sqrt(a\_db)
%
\end{lyxcode}%
%
\item[Description:]%
Overloaded sqrt function.
%%
\item[Parameters:]~
\begin{description}%
\item[\texttt{a\_db}:]
 A tests\_db.
\end{description}%
%
\item[Returns:
]~

   a\_db: The resulting tests\_db.
%
%
\item[See also:]%
\hyperlink{ref_sqrt}{\texttt{sqrt}}%
\ (p.~\pageref{ref_sqrt})%
\index[funcref]{sqrt@\fidxl{sqrt}}%
%
\item[Author:]%
Cengiz Gunay <cgunay@emory.edu>, 2007/12/13
%
\end{description}
\methodline%
\subsubsection[Method \texttt{statsAll}]{Method \texttt{tests\_db/statsAll}}%
\index[funcref]{tests_db@\fidxl{tests\_db}!statsAll@\fidxl{statsAll}}%
\label{ref_tests_db__statsAll}%
\hypertarget{ref_tests_db__statsAll}{}%
\begin{description}
\item[Summary:]Makes a stats\_db with rows of mean, STD, SE, and CV of the tests' distributions in db.
%
\item[Usage:]~%
\begin{lyxcode}%
a\_stats\_db = statsAll(db, tests, props)
%
\end{lyxcode}%
%
%
\item[Parameters:]~
\begin{description}%
\item[\texttt{db}:]
 A tests\_db object.
\item[\texttt{tests}:]
 A selection of tests (see onlyRowsTests).
\item[\texttt{props}:]
 A structure with any optional properties for stats\_db.
\end{description}%
%
\item[Returns:
]~

	a\_stats\_db: A stats\_db object.
%
%
\item[See also:]%
\hyperlink{ref_tests_db}{\texttt{tests\_db}}%
\ (p.~\pageref{ref_tests_db})%
\index[funcref]{tests_db@\fidxl{tests\_db}}%
%
\item[Author:]%
Cengiz Gunay <cgunay@emory.edu>, 2005/08/24
%
\end{description}
\methodline%
\subsubsection[Method \texttt{statsBounds}]{Method \texttt{tests\_db/statsBounds}}%
\index[funcref]{tests_db@\fidxl{tests\_db}!statsBounds@\fidxl{statsBounds}}%
\label{ref_tests_db__statsBounds}%
\hypertarget{ref_tests_db__statsBounds}{}%
\begin{description}
\item[Summary:]Generates a stats\_db object with three rows corresponding to the mean, min, max and number of observations of the tests' distributions. 
%
\item[Usage:]~%
\begin{lyxcode}%
a\_stats\_db = statsBounds(a\_db, tests, props)
%
\end{lyxcode}%
%
\item[Description:]%
A page is generated for each page of data in db.
%%
\item[Parameters:]~
\begin{description}%
\item[\texttt{a\_db}:]
 A tests\_db object.
\item[\texttt{tests}:]
 A selection of tests (see onlyRowsTests).
\item[\texttt{props}:]
 A structure with any optional properties for stats\_db.
\end{description}%
%
\item[Returns:
]~

	a\_stats\_db: A stats\_db object.
%
%
\item[See also:]%
\hyperlink{ref_tests_db}{\texttt{tests\_db}}%
\ (p.~\pageref{ref_tests_db})%
\index[funcref]{tests_db@\fidxl{tests\_db}}%
%
\item[Author:]%
Cengiz Gunay <cgunay@emory.edu>, 2004/10/07
%
\end{description}
\methodline%
\subsubsection[Method \texttt{statsMeanSE}]{Method \texttt{tests\_db/statsMeanSE}}%
\index[funcref]{tests_db@\fidxl{tests\_db}!statsMeanSE@\fidxl{statsMeanSE}}%
\label{ref_tests_db__statsMeanSE}%
\hypertarget{ref_tests_db__statsMeanSE}{}%
\begin{description}
\item[Summary:]Generates a stats\_db object with two rows corresponding to 
		the mean and standard error (SE) of the tests' distributions.
%
\item[Usage:]~%
\begin{lyxcode}%
a\_stats\_db = statsMeanSE(db, tests, props)
%
\end{lyxcode}%
%
%
\item[Parameters:]~
\begin{description}%
\item[\texttt{db}:]
 A tests\_db object.
\item[\texttt{tests}:]
 A selection of tests (see onlyRowsTests).
\item[\texttt{props}:]
 A structure with any optional properties for stats\_db.
\end{description}%
%
\item[Returns:
]~

	a\_stats\_db: A stats\_db object.
%
%
\item[See also:]%
\hyperlink{ref_tests_db}{\texttt{tests\_db}}%
\ (p.~\pageref{ref_tests_db})%
\index[funcref]{tests_db@\fidxl{tests\_db}}%
%
\item[Author:]%
Cengiz Gunay <cgunay@emory.edu>, 2004/10/07
%
\end{description}
\methodline%
\subsubsection[Method \texttt{statsMeanStd}]{Method \texttt{tests\_db/statsMeanStd}}%
\index[funcref]{tests_db@\fidxl{tests\_db}!statsMeanStd@\fidxl{statsMeanStd}}%
\label{ref_tests_db__statsMeanStd}%
\hypertarget{ref_tests_db__statsMeanStd}{}%
\begin{description}
\item[Summary:]Generates a stats\_db object with mean, STD, and number of observations of the tests' distributions.
%
\item[Usage:]~%
\begin{lyxcode}%
a\_stats\_db = statsMeanStd(db, tests, props)
%
\end{lyxcode}%
%
%
\item[Parameters:]~
\begin{description}%
\item[\texttt{db}:]
 A tests\_db object.
\item[\texttt{tests}:]
 A selection of tests (see onlyRowsTests).
\item[\texttt{props}:]
 A structure with any optional properties for stats\_db.
\end{description}%
%
\item[Returns:
]~

	a\_stats\_db: A stats\_db object.
%
%
\item[See also:]%
\hyperlink{ref_tests_db}{\texttt{tests\_db}}%
\ (p.~\pageref{ref_tests_db})%
\index[funcref]{tests_db@\fidxl{tests\_db}}%
%
\item[Author:]%
Cengiz Gunay <cgunay@emory.edu>, 2004/10/07
%
\end{description}
\methodline%
\subsubsection[Method \texttt{std}]{Method \texttt{tests\_db/std}}%
\index[funcref]{tests_db@\fidxl{tests\_db}!std@\fidxl{std}}%
\label{ref_tests_db__std}%
\hypertarget{ref_tests_db__std}{}%
\begin{description}
\item[Summary:]Returns the std of the data matrix of a\_db. Ignores NaN values.
%
\item[Usage:]~%
\begin{lyxcode}%
[a\_db, n] = std(a\_db, sflag, dim)
%
\end{lyxcode}%
%
\item[Description:]%
Does a recursive operation over dimensions in order to remove NaN values.
 This takes considerable amount of time compared with a straightforward std
 operation. 
%%
\item[Parameters:]~
\begin{description}%
\item[\texttt{a\_db}:]
 A tests\_db object.
\item[\texttt{dim}:]
 Work down dimension.
\end{description}%
%
\item[Returns:
]~

	a\_db: The DB with std values.
	n: (Optional) Numbers of non-NaN rows included in calculating each column.
%
%
\item[See also:]%
\hyperlink{ref_std}{\texttt{std}}%
\ (p.~\pageref{ref_std})%
\index[funcref]{std@\fidxl{std}}%
, \hyperlink{ref_tests_db}{\texttt{tests\_db}}%
\ (p.~\pageref{ref_tests_db})%
\index[funcref]{tests_db@\fidxl{tests\_db}}%
%
\item[Author:]%
Cengiz Gunay <cgunay@emory.edu>, 2004/10/06
%
\end{description}
\methodline%
\subsubsection[Method \texttt{subsasgn}]{Method \texttt{tests\_db/subsasgn}}%
\index[funcref]{tests_db@\fidxl{tests\_db}!subsasgn@\fidxl{subsasgn}}%
\label{ref_tests_db__subsasgn}%
\hypertarget{ref_tests_db__subsasgn}{}%
\begin{description}
\item[Summary:]Defines generic index-based assignment for objects.
%
%
%
%
%
%
%
\item[Author:]%
Cengiz Gunay <cgunay@emory.edu>, 2006/02/06
%
\end{description}
\methodline%
\subsubsection[Method \texttt{subsref}]{Method \texttt{tests\_db/subsref}}%
\index[funcref]{tests_db@\fidxl{tests\_db}!subsref@\fidxl{subsref}}%
\label{ref_tests_db__subsref}%
\hypertarget{ref_tests_db__subsref}{}%
\begin{description}
\item[Summary:]Defines indexing for tests\_db objects for () and . operations. 
%
\item[Usage:]~%
\begin{lyxcode}%
obj = obj(rows, tests)
 obj = obj.attribute
%
\end{lyxcode}%
%
\item[Description:]%
Returns attributes or selects the given test columns and rows
 and returns in a new tests\_db object.
%%
\item[Parameters:]~
\begin{description}%
\item[\texttt{obj}:]
 A tests\_db object.
\item[\texttt{rows}:]
 A logical or index vector of rows. If ':', all rows.
\item[\texttt{tests}:]
 Cell array of test names or column indices. If ':', all tests.
\item[\texttt{attribute}:]
 A tests\_db class attribute.
\end{description}%
%
\item[Returns:
]~

	obj: The new tests\_db object.
%
%
\item[See also:]%
\hyperlink{ref_subsref}{\texttt{subsref}}%
\ (p.~\pageref{ref_subsref})%
\index[funcref]{subsref@\fidxl{subsref}}%
, \hyperlink{ref_tests_db}{\texttt{tests\_db}}%
\ (p.~\pageref{ref_tests_db})%
\index[funcref]{tests_db@\fidxl{tests\_db}}%
%
\item[Author:]%
Cengiz Gunay <cgunay@emory.edu>, 2004/09/17
%
\end{description}
\methodline%
\subsubsection[Method \texttt{sum}]{Method \texttt{tests\_db/sum}}%
\index[funcref]{tests_db@\fidxl{tests\_db}!sum@\fidxl{sum}}%
\label{ref_tests_db__sum}%
\hypertarget{ref_tests_db__sum}{}%
\begin{description}
\item[Summary:]Creates a tests\_db by summing all rows.
%
\item[Usage:]~%
\begin{lyxcode}%
a\_db = sum(a\_db, dim, props)
%
\end{lyxcode}%
%
\item[Description:]%
Applies the sum function to whole DB. The resulting DB will have one row.
%%
\item[Parameters:]~
\begin{description}%
\item[\texttt{a\_db}:]
 A tests\_db object.
\item[\texttt{props}:]
 Optional properties.
\end{description}%
%
\item[Returns:
]~

	a\_db: The resulting tests\_db.
%
%
\item[See also:]%
\hyperlink{ref_sum}{\texttt{sum}}%
\ (p.~\pageref{ref_sum})%
\index[funcref]{sum@\fidxl{sum}}%
%
\item[Author:]%
Cengiz Gunay <cgunay@emory.edu>, 2006/05/24
%
\end{description}
\methodline%
\subsubsection[Method \texttt{swapColsPages}]{Method \texttt{tests\_db/swapColsPages}}%
\index[funcref]{tests_db@\fidxl{tests\_db}!swapColsPages@\fidxl{swapColsPages}}%
\label{ref_tests_db__swapColsPages}%
\hypertarget{ref_tests_db__swapColsPages}{}%
\begin{description}
\item[Summary:]Swaps the column dimension with the page dimension of the tests\_db.
%
\item[Usage:]~%
\begin{lyxcode}%
a\_db = swapColsPages(db)
%
\end{lyxcode}%
%
\item[Description:]%
Watered-down version of the tests\_3D\_db/swapColsPages function that
 does not touch column indices. 
%%
\item[Parameters:]~
\begin{description}%
\item[\texttt{db}:]
 A tests\_db object.
\end{description}%
%
\item[Returns:
]~

	a\_db: A tests\_db object.
%
%
\item[See also:]%
\hyperlink{ref_tests_db}{\texttt{tests\_db}}%
\ (p.~\pageref{ref_tests_db})%
\index[funcref]{tests_db@\fidxl{tests\_db}}%
%
\item[Author:]%
Cengiz Gunay <cgunay@emory.edu>, 2017/06/09
%
\end{description}
\methodline%
\subsubsection[Method \texttt{swapRowsPages}]{Method \texttt{tests\_db/swapRowsPages}}%
\index[funcref]{tests_db@\fidxl{tests\_db}!swapRowsPages@\fidxl{swapRowsPages}}%
\label{ref_tests_db__swapRowsPages}%
\hypertarget{ref_tests_db__swapRowsPages}{}%
\begin{description}
\item[Summary:]Swaps the row dimension with the page dimension of the tests\_db.
%
\item[Usage:]~%
\begin{lyxcode}%
a\_db = swapRowsPages(db)
%
\end{lyxcode}%
%
\item[Description:]%
Watered-down version of the tests\_3D\_db/swapRowsPages function that
 does not touch row indices. 
%%
\item[Parameters:]~
\begin{description}%
\item[\texttt{db}:]
 A tests\_db object.
\end{description}%
%
\item[Returns:
]~

	a\_db: A tests\_db object.
%
%
\item[See also:]%
\hyperlink{ref_tests_db}{\texttt{tests\_db}}%
\ (p.~\pageref{ref_tests_db})%
\index[funcref]{tests_db@\fidxl{tests\_db}}%
%
\item[Author:]%
Cengiz Gunay <cgunay@emory.edu>, 2004/10/04
%
\end{description}
\methodline%
\subsubsection[Method \texttt{tests2cols}]{Method \texttt{tests\_db/tests2cols}}%
\index[funcref]{tests_db@\fidxl{tests\_db}!tests2cols@\fidxl{tests2cols}}%
\label{ref_tests_db__tests2cols}%
\hypertarget{ref_tests_db__tests2cols}{}%
\begin{description}
\item[Summary:]Find column numbers from a test names/numbers specification.
%
\item[Usage:]~%
\begin{lyxcode}%
cols = tests2cols(db, tests)
%
\end{lyxcode}%
%
\item[Description:]%
Uses tests2idx.
%%
\item[Parameters:]~
\begin{description}%
\item[\texttt{db}:]
 A tests\_db object.
\item[\texttt{tests}:]
 Either a single or array of column numbers, or a single

test name or a cell array of test names. If ':', all
tests. For name strings, regular expressions are
supported if quoted with slashes (e.g., '/a.*/'). 
See tests2idx for more.
\end{description}%
%
\item[Returns:
]~

	cols: Array of column indices.
%
%
\item[See also:]%
\hyperlink{ref_tests_db}{\texttt{tests\_db}}%
\ (p.~\pageref{ref_tests_db})%
\index[funcref]{tests_db@\fidxl{tests\_db}}%
, \hyperlink{ref_tests2idx}{\texttt{tests2idx}}%
\ (p.~\pageref{ref_tests2idx})%
\index[funcref]{tests2idx@\fidxl{tests2idx}}%
%
\item[Author:]%
Cengiz Gunay <cgunay@emory.edu>, 2004/10/07
%
\end{description}
\methodline%
\subsubsection[Method \texttt{tests2idx}]{Method \texttt{tests\_db/tests2idx}}%
\index[funcref]{tests_db@\fidxl{tests\_db}!tests2idx@\fidxl{tests2idx}}%
\label{ref_tests_db__tests2idx}%
\hypertarget{ref_tests_db__tests2idx}{}%
\begin{description}
\item[Summary:]Find dimension indices from a test names/numbers specification.
%
\item[Usage:]~%
\begin{lyxcode}%
idx = tests2idx(db, dim\_num, tests)
%
\end{lyxcode}%
%
%
\item[Parameters:]~
\begin{description}%
\item[\texttt{db}:]
 A tests\_db object.
\item[\texttt{dim\_num}:]
 Number between 1-3 to choose dimension: row, column, or page.
\item[\texttt{tests}:]
 Either a single or array of column numbers, or a single

test name or a cell array of test names. If ':', all
tests. For name strings, regular expressions are
supported if quoted with slashes (e.g., '/a.*/')
\end{description}%
%
\item[Returns:
]~

	idx: Array of column indices.
%
\item[Example:]~
\begin{lyxcode} >> cols = tests2idx(a\_db, 2, {'col1', '/col2+/'});
\\%
 will return indices of col1 and columns like col2, col22, col22, etc.
\\%
\end{lyxcode}
%
\item[See also:]%
\hyperlink{ref_tests_db}{\texttt{tests\_db}}%
\ (p.~\pageref{ref_tests_db})%
\index[funcref]{tests_db@\fidxl{tests\_db}}%
, \hyperlink{ref_tests2cols}{\texttt{tests2cols}}%
\ (p.~\pageref{ref_tests2cols})%
\index[funcref]{tests2cols@\fidxl{tests2cols}}%
, \hyperlink{ref_regexp}{\texttt{regexp}}%
\ (p.~\pageref{ref_regexp})%
\index[funcref]{regexp@\fidxl{regexp}}%
%
\item[Author:]%
Cengiz Gunay <cgunay@emory.edu>, 2004/10/07
%
\end{description}
\methodline%
\subsubsection[Method \texttt{tests2log}]{Method \texttt{tests\_db/tests2log}}%
\index[funcref]{tests_db@\fidxl{tests\_db}!tests2log@\fidxl{tests2log}}%
\label{ref_tests_db__tests2log}%
\hypertarget{ref_tests_db__tests2log}{}%
\begin{description}
\item[Summary:]Return logical array of indices from a test names/numbers specification.
%
\item[Usage:]~%
\begin{lyxcode}%
a\_log = tests2log(db, dim\_name, tests)
%
\end{lyxcode}%
%
\item[Description:]%
See tests2idx.
%%
\item[Parameters:]~
\begin{description}%
\item[\texttt{db}:]
 A tests\_db object.
\item[\texttt{dim\_num}:]
 Number between 1-3 to choose dimension: row, column, or page.
\item[\texttt{tests}:]
 Either a single or array of column numbers, or a single

test name or a cell array of test names. If ':', all
tests. For name strings, regular expressions are
supported if quoted with slashes (e.g., '/a.*/')
\end{description}%
%
\item[Returns:
]~

	a\_log: Array of column indices.
%
\item[Example:]~
\begin{lyxcode} >> cols = tests2log(a\_db, 'col', {'col1', '/col2+/'});
\\%
 >> stripped\_db = a\_db(:, ~cols)
\\%
 will remove columns col1 and col2, col22, col22, etc. from stripped\_db.
\\%
\end{lyxcode}
%
\item[See also:]%
\hyperlink{ref_tests_db}{\texttt{tests\_db}}%
\ (p.~\pageref{ref_tests_db})%
\index[funcref]{tests_db@\fidxl{tests\_db}}%
, \hyperlink{ref_tests2cols}{\texttt{tests2cols}}%
\ (p.~\pageref{ref_tests2cols})%
\index[funcref]{tests2cols@\fidxl{tests2cols}}%
, \hyperlink{ref_regexp}{\texttt{regexp}}%
\ (p.~\pageref{ref_regexp})%
\index[funcref]{regexp@\fidxl{regexp}}%
%
\item[Author:]%
Cengiz Gunay <cgunay@emory.edu>, 2008/05/27
%
\end{description}
\methodline%
\subsubsection[Method \texttt{testsHists}]{Method \texttt{tests\_db/testsHists}}%
\index[funcref]{tests_db@\fidxl{tests\_db}!testsHists@\fidxl{testsHists}}%
\label{ref_tests_db__testsHists}%
\hypertarget{ref_tests_db__testsHists}{}%
\begin{description}
\item[Summary:]Calculates histograms for all tests.
%
\item[Usage:]~%
\begin{lyxcode}%
t\_hists = testsHists(a\_db, num\_bins)
%
\end{lyxcode}%
%
%
\item[Parameters:]~
\begin{description}%
\item[\texttt{a\_db}:]
 A tests\_db object.
\item[\texttt{num\_bins}:]
 Number of histogram bins (Optional, default=100), or

vector of histogram bin centers.
\end{description}%
%
\item[Returns:
]~

	t\_hists: An array of histograms for each test in a\_db.
%
%
\item[See also:]%
\hyperlink{ref_params_tests_profile}{\texttt{params\_tests\_profile}}%
\ (p.~\pageref{ref_params_tests_profile})%
\index[funcref]{params_tests_profile@\fidxl{params\_tests\_profile}}%
%
\item[Author:]%
Cengiz Gunay <cgunay@emory.edu>, 2005/04/27
%
\end{description}
\methodline%
\subsubsection[Method \texttt{times}]{Method \texttt{tests\_db/times}}%
\index[funcref]{tests_db@\fidxl{tests\_db}!times@\fidxl{times}}%
\label{ref_tests_db__times}%
\hypertarget{ref_tests_db__times}{}%
\begin{description}
%
\item[Usage:]~%
\begin{lyxcode}%
a\_db = mtimes(left\_obj, right\_obj)
%
\end{lyxcode}%
%
%
\item[Parameters:]~
\begin{description}%
\item[\texttt{left\_obj, right\_obj}:]
 Operands of the multiplication. One or more must be of type tests\_db.
\end{description}%
%
\item[Returns:
]~

	a\_db: The resulting tests\_db.
%
%
\item[See also:]%
\hyperlink{ref_tests_db__times}{\texttt{tests\_db/times}}%
\ (p.~\pageref{ref_tests_db__times})%
\index[funcref]{tests_db@\fidxl{tests\_db}!times@\fidxl{times}}%
, \hyperlink{ref_mtimes}{\texttt{mtimes}}%
\ (p.~\pageref{ref_mtimes})%
\index[funcref]{mtimes@\fidxl{mtimes}}%
%
\item[Author:]%
Cengiz Gunay <cgunay@emory.edu>, 2006/05/24
%
\end{description}
\methodline%
\subsubsection[Method \texttt{transpose}]{Method \texttt{tests\_db/transpose}}%
\index[funcref]{tests_db@\fidxl{tests\_db}!transpose@\fidxl{transpose}}%
\label{ref_tests_db__transpose}%
\hypertarget{ref_tests_db__transpose}{}%
\begin{description}
\item[Summary:]Transposes data matrix and swaps row and columns metadata as well.
%
\item[Usage:]~%
\begin{lyxcode}%
a\_db = transpose(a\_db)
%
\end{lyxcode}%
%
%
\item[Parameters:]~
\begin{description}%
\item[\texttt{a\_db}:]
 A tests\_db.
\end{description}%
%
\item[Returns:
]~

	a\_db: The transposed tests\_db.
%
%
\item[See also:]%
\hyperlink{ref_transpose}{\texttt{transpose}}%
\ (p.~\pageref{ref_transpose})%
\index[funcref]{transpose@\fidxl{transpose}}%
%
\item[Author:]%
Cengiz Gunay <cgunay@emory.edu>, 2007/02/07
%
\end{description}
\methodline%
\subsubsection[Method \texttt{uminus}]{Method \texttt{tests\_db/uminus}}%
\index[funcref]{tests_db@\fidxl{tests\_db}!uminus@\fidxl{uminus}}%
\label{ref_tests_db__uminus}%
\hypertarget{ref_tests_db__uminus}{}%
\begin{description}
\item[Summary:]Unary minus or negation.
%
\item[Usage:]~%
\begin{lyxcode}%
a\_db = uminus(left\_obj)
%
\end{lyxcode}%
%
%
\item[Parameters:]~
\begin{description}%
\item[\texttt{left\_obj}:]
 A tests\_db object.
\end{description}%
%
\item[Returns:
]~

	a\_db: The resulting tests\_db.
%
%
\item[See also:]%
\hyperlink{ref_uminus}{\texttt{uminus}}%
\ (p.~\pageref{ref_uminus})%
\index[funcref]{uminus@\fidxl{uminus}}%
, \hyperlink{ref_uop}{\texttt{uop}}%
\ (p.~\pageref{ref_uop})%
\index[funcref]{uop@\fidxl{uop}}%
%
\item[Author:]%
Cengiz Gunay <cgunay@emory.edu>, 2008/01/16
%
\end{description}
\methodline%
\subsubsection[Method \texttt{unique}]{Method \texttt{tests\_db/unique}}%
\index[funcref]{tests_db@\fidxl{tests\_db}!unique@\fidxl{unique}}%
\label{ref_tests_db__unique}%
\hypertarget{ref_tests_db__unique}{}%
\begin{description}
\item[Summary:]Returns DB with unique rows.
%
\item[Usage:]~%
\begin{lyxcode}%
[a\_db idx] = unique(a\_db)
%
\end{lyxcode}%
%
\item[Description:]%
Keeps the original DB order. Uses the uniqueValues function.
%%
\item[Parameters:]~
\begin{description}%
\item[\texttt{a\_db}:]
 tests\_db from which to find uniques.
\end{description}%
%
\item[Returns:
]~

   a\_db: The resulting tests\_db.
   idx: Indices of the unique rows in the original data matrix.
%
%
\item[See also:]%
\hyperlink{ref_uniqueValues}{\texttt{uniqueValues}}%
\ (p.~\pageref{ref_uniqueValues})%
\index[funcref]{uniqueValues@\fidxl{uniqueValues}}%
, \hyperlink{ref_unique}{\texttt{unique}}%
\ (p.~\pageref{ref_unique})%
\index[funcref]{unique@\fidxl{unique}}%
%
\item[Author:]%
Cengiz Gunay <cgunay@emory.edu>, 2007/11/19
%
\end{description}
\methodline%
\subsubsection[Method \texttt{uop}]{Method \texttt{tests\_db/uop}}%
\index[funcref]{tests_db@\fidxl{tests\_db}!uop@\fidxl{uop}}%
\label{ref_tests_db__uop}%
\hypertarget{ref_tests_db__uop}{}%
\begin{description}
\item[Summary:]Unary operation.
%
\item[Usage:]~%
\begin{lyxcode}%
a\_db = uop(left\_obj, op\_func, op\_id)
%
\end{lyxcode}%
%
\item[Description:]%
Applies the operation to the database contents and updates its id
 field. Unary minus (uminus) uses this function.
%%
\item[Parameters:]~
\begin{description}%
\item[\texttt{left\_obj}:]
 Operands of the operation.
\item[\texttt{op\_func}:]
 Operation function (e.g., @plus).
\item[\texttt{op\_id}:]
 A string to represent the operation that will show up in the

returned id.
\end{description}%
%
\item[Returns:
]~

   a\_db: The resulting tests\_db.
%
%
\item[See also:]%
\hyperlink{ref_tests_db__uminus}{\texttt{tests\_db/uminus}}%
\ (p.~\pageref{ref_tests_db__uminus})%
\index[funcref]{tests_db@\fidxl{tests\_db}!uminus@\fidxl{uminus}}%
, \hyperlink{ref_uminus}{\texttt{uminus}}%
\ (p.~\pageref{ref_uminus})%
\index[funcref]{uminus@\fidxl{uminus}}%
%
\item[Author:]%
Cengiz Gunay <cgunay@emory.edu>, 2008/01/16
%
\end{description}
\methodline%
\subsubsection[Method \texttt{vertcat}]{Method \texttt{tests\_db/vertcat}}%
\index[funcref]{tests_db@\fidxl{tests\_db}!vertcat@\fidxl{vertcat}}%
\label{ref_tests_db__vertcat}%
\hypertarget{ref_tests_db__vertcat}{}%
\begin{description}
\item[Summary:]Vertical concatanation [db;with\_db;...] operator.
%
\item[Usage:]~%
\begin{lyxcode}%
a\_db = vertcat(db, with\_db, ...)
%
\end{lyxcode}%
%
\item[Description:]%
Concatanates rows of with\_db to rows of db. Overrides the built-in
 vertcat function that is called when [db;with\_db] is executed. If the 
 first argument is a array of DBs, then this functionality is not needed;
 built-in vertcat is called.
%%
\item[Parameters:]~
\begin{description}%
\item[\texttt{db}:]
 A tests\_db object.
\item[\texttt{with\_db}:]
 A tests\_db object whose rows are concatanated to db.
\end{description}%
%
\item[Returns:
]~

	a\_db: A tests\_db that contains rows of db and with\_db.
%
%
\item[See also:]%
\hyperlink{ref_vertcat}{\texttt{vertcat}}%
\ (p.~\pageref{ref_vertcat})%
\index[funcref]{vertcat@\fidxl{vertcat}}%
, \hyperlink{ref_tests_db}{\texttt{tests\_db}}%
\ (p.~\pageref{ref_tests_db})%
\index[funcref]{tests_db@\fidxl{tests\_db}}%
%
\item[Author:]%
Cengiz Gunay <cgunay@emory.edu>, 2005/01/25
%
\end{description}
\methodline%
\subsection{Class \texttt{trace}}%
\index[funcref]{trace@\fidxl{trace}|boldhyperpage}%
\label{ref_trace}%
\hypertarget{ref_trace}{}%
\subsubsection[Constructor \texttt{trace}]{Constructor \texttt{trace/trace}}%
\index[funcref]{trace@\fidxl{trace}!trace@\fidxl{trace}}%
\label{ref_trace__trace}%
\hypertarget{ref_trace__trace}{}%
\begin{description}
\item[Summary:]Load a generic data trace. It can be membrane voltage, current, etc.
%
\item[Usage:]~%
\begin{lyxcode}%
obj = trace(data\_src, dt, dy, id, props)
%
\end{lyxcode}%
%
\item[Description:]%
This object is designed to recognize most data file formats. See the
 data\_src parameter below. Traces for specific experimental or simulation
 protocols can extend this class for adding new parameters.
%%
\item[Parameters:]~
\begin{description}%
\item[\texttt{data\_src}:]
 Trace data as a column vector OR name of a data file generated by either 

Genesis (.bin, .gbin, .genflac), Neuron, PCDX (.all), Matlab (.mat),
NeuroShare (.nsn, .nev, .stb, .plx, .nex, .map, .son, .smr,
.mcd), ASCII (.txt) or Axoclamp (.abf) files. As last
resort, file will be loaded with Matlab's load() function.
\item[\texttt{dt}:]
 Time resolution in [s], unless specified in HDF5 or NeuroShare

file. For Neuron ASCII files, it is used as the file data units.
\item[\texttt{dy}:]
 y-axis resolution in [ISI (V, A, etc.)], unless specified in HDF5 or NeuroShare file.
\item[\texttt{id}:]
 Identification string
\item[\texttt{props}:]
 A structure with any optional properties.
\begin{description}%
\item[\texttt{scale\_y}:]
 Y-axis scale to be applied to loaded data.
\item[\texttt{offset\_y}:]
 Y-axis offset to be added to loaded and scaled data.
\item[\texttt{unit\_y}:]
 Unit of Y-axis as in 'V' or 'A' (default='V').
\end{description}%
\item[\texttt{y\_label}:]
 String to put on Y-axis of plots.
\begin{description}%
\item[\texttt{trace\_time\_start}:]
 Samples in the beginning to discard [dt]
\item[\texttt{baseline}:]
 Resting potential.
\item[\texttt{channel}:]
 Channel to read from file Genesis, PCDX, NeuroShare or

Neuron file, or column in a data vector.
\end{description}%
\item[\texttt{numTraces}:]
 Divide the single column vector of data into this

many columns by making it a matrix.
\begin{description}%
\item[\texttt{file\_type}:]
 Specify file type instead of guessing from extension:

'genesis': Raw binary files created with Genesis disk\_out method.
'genesis\_flac': Compressed Genesis binary files.
'neuron': Binary files created with Neuron's Vector.vwrite method.
'neuronascii': Ascii files created from Neuron's Vector objects. 
Uses time step in file to scale given dt (Must be in ms).
'pcdx': .ALL data acquisition files from PCDX program.
'matlab': Matlab .MAT binary files with matrix data.
'neuroshare': One of the vendor formats recognized by
NeuroShare Windows DLLs. See above and http://neuroshare.org. A
scale\_y value may need to be supplied to get the correct units.
'abf': AxoClamp .ABF format read with abf2load from
Matlab FileExchange.
'txt': Generic, space-separated ASCII file.
\end{description}%
\item[\texttt{file\_endian}:]
 'l' for little endian and 'b' for big endian

(default='n', for native endian). See machineformat option 
in fopen for more info.
\item[\texttt{traces}:]
 Trace numbers as a numeric array or as a string with

numeric ranges (e.g., '1 2 5-10 28') for PCDX files.
\begin{description}%
\item[\texttt{spike\_finder}:]
 Method of finding spikes 

(1 uses findFilteredSpikes.m, 2 for Li Su's
findspikes, 3 for Alfonso Delgado Reyes's 
findspikes\_old, and 4 for using Matlab's
findpeaks method). Methods 2-4 require a
threshold. For method 4, see additional findpeaks*
props below.
\item[\texttt{threshold}:]
 Spike finding threshold. For the findspikes method,

it is either a scalar, or [thres1 thres2] to define
a range. For findFilteredSpikes it is used on the
filtered data and the default is 2/3 max amplitude
of band-passed data, but with a minimum of
15. For findpeaks, it sets the MinPeakHeight parameter.
\end{description}%
\item[\texttt{downThreshold}:]
 (Only for findFilteredSpikes) Size of the trough

after the spike peak in filtered data (Default=-2).
\begin{description}%
\item[\texttt{minInit2MaxAmp, minMin2MaxAmp}:]
 For spike\_shape elimination,

conditions of minimal allowed values for 
initial point to max point and minimal point to 
max point, respectively (Default=10 for both).
\item[\texttt{init\_Vm\_method}:]
 Method of finding spike thresholds during spike

shape calculation (see spike\_shape/spike\_shape).
\item[\texttt{init\_threshold}:]
 Spike initiation threshold (deriv or accel).

(see above methods and implementation in calcInitVm)
\item[\texttt{init\_lo\_thr, init\_hi\_thr}:]
 Low and high thresholds for slope.
\end{description}%
\item[\texttt{custom\_filter}:]
 Recommended if sampling rate differs appreciably from 10 kHz.

If custom\_filter == 1, a filter with custom lowpass and highpass
cutoffs can be specified. This allows for fast and accurate spike
discrimination. The filter type used is a 2-pole
butterworth, different than the default high-order
Cheby2. Creates new prop called 'butterWorth' to
hold the filter.
\item[\texttt{lowPassFreq}:]
 If set, it sets a new low pass cutoff for custom filter. Default is 3000Hz
\item[\texttt{highPassFreq}:]
 If set it sets a new high pass cutoff for custom filter. Default is 50 Hz
\begin{description}%
\item[\texttt{findpeaksArgs}:]
 Cell array of arguments to pass to findpeaks.
\item[\texttt{findpeaksSign}:]
 Choose -1 to flip the sign of data and look

for negative peaks (default=1).
\item[\texttt{quiet}:]
 If 1, reduces the amount of textual description in plots

and does not add information to id field.
\end{description}%
\end{description}%
%
\item[Returns a structure object with the following fields:
]~

	data: The trace column matrix.
	dt, dy, id, props (see above)
%
%
\item[See also:]%
\hyperlink{ref_spikes}{\texttt{spikes}}%
\ (p.~\pageref{ref_spikes})%
\index[funcref]{spikes@\fidxl{spikes}}%
, \hyperlink{ref_spike_shape}{\texttt{spike\_shape}}%
\ (p.~\pageref{ref_spike_shape})%
\index[funcref]{spike_shape@\fidxl{spike\_shape}}%
, \hyperlink{ref_cip_trace}{\texttt{cip\_trace}}%
\ (p.~\pageref{ref_cip_trace})%
\index[funcref]{cip_trace@\fidxl{cip\_trace}}%
, \hyperlink{ref_period}{\texttt{period}}%
\ (p.~\pageref{ref_period})%
\index[funcref]{period@\fidxl{period}}%
, \hyperlink{ref_findpeaks}{\texttt{findpeaks}}%
\ (p.~\pageref{ref_findpeaks})%
\index[funcref]{findpeaks@\fidxl{findpeaks}}%
, \hyperlink{ref_findFilteredSpikes}{\texttt{findFilteredSpikes}}%
\ (p.~\pageref{ref_findFilteredSpikes})%
\index[funcref]{findFilteredSpikes@\fidxl{findFilteredSpikes}}%
, \hyperlink{ref_findspikes}{\texttt{findspikes}}%
\ (p.~\pageref{ref_findspikes})%
\index[funcref]{findspikes@\fidxl{findspikes}}%
, \hyperlink{ref_findspikes_old}{\texttt{findspikes\_old}}%
\ (p.~\pageref{ref_findspikes_old})%
\index[funcref]{findspikes_old@\fidxl{findspikes\_old}}%
%
\item[Author:]%
Cengiz Gunay <cgunay@emory.edu>, 2004/07/30
%
\end{description}
\methodline%
\subsubsection[Method \texttt{analyzeSpikesInPeriod}]{Method \texttt{trace/analyzeSpikesInPeriod}}%
\index[funcref]{trace@\fidxl{trace}!analyzeSpikesInPeriod@\fidxl{analyzeSpikesInPeriod}}%
\label{ref_trace__analyzeSpikesInPeriod}%
\hypertarget{ref_trace__analyzeSpikesInPeriod}{}%
\begin{description}
\item[Summary:]Returns results and a db of spikes by collecting test results of a trace, analyzing each individual spike.
%
\item[Usage:]~%
\begin{lyxcode}%
[results period\_spikes a\_spikes\_db spikes\_stats\_db spikes\_hists\_dbs] =
      analyzeSpikesInPeriod(a\_trace, a\_spikes, period, prefix\_str)
%
\end{lyxcode}%
%
%
\item[Parameters:]~
\begin{description}%
\item[\texttt{a\_trace}:]
 A trace object.
\item[\texttt{a\_spikes}:]
 A spikes object from the a\_trace object.
\item[\texttt{period}:]
 A period of object of a\_trace object of interest.
\item[\texttt{prefix\_str}:]
 Prefix string to be added to spike shape results.
\end{description}%
%
\item[Returns:
]~

	results: Results structure names prefixed with prefix\_str.
	period\_spikes: Corrected spikes object for this period.
	a\_spikes\_db: A mini spikes database of results from each spike in period. 
	spikes\_stats\_db: Statistics from the mini spikes database.
	spikes\_hists\_dbs: Cell array of histograms from the mini spikes database.
%
%
\item[See also:]%
\hyperlink{ref_trace}{\texttt{trace}}%
\ (p.~\pageref{ref_trace})%
\index[funcref]{trace@\fidxl{trace}}%
, \hyperlink{ref_spikes}{\texttt{spikes}}%
\ (p.~\pageref{ref_spikes})%
\index[funcref]{spikes@\fidxl{spikes}}%
, \hyperlink{ref_period}{\texttt{period}}%
\ (p.~\pageref{ref_period})%
\index[funcref]{period@\fidxl{period}}%
, \hyperlink{ref_spike_shape}{\texttt{spike\_shape}}%
\ (p.~\pageref{ref_spike_shape})%
\index[funcref]{spike_shape@\fidxl{spike\_shape}}%
, \hyperlink{ref_getProfileAllSpikes}{\texttt{getProfileAllSpikes}}%
\ (p.~\pageref{ref_getProfileAllSpikes})%
\index[funcref]{getProfileAllSpikes@\fidxl{getProfileAllSpikes}}%
%
\item[Author:]%
Cengiz Gunay <cgunay@emory.edu>, 2005/05/04
%
\end{description}
\methodline%
\subsubsection[Method \texttt{avgTraces}]{Method \texttt{trace/avgTraces}}%
\index[funcref]{trace@\fidxl{trace}!avgTraces@\fidxl{avgTraces}}%
\label{ref_trace__avgTraces}%
\hypertarget{ref_trace__avgTraces}{}%
\begin{description}
\item[Summary:]Average multiple traces.
%
\item[Usage:]~%
\begin{lyxcode}%
[avg\_tr sd\_tr] = avgTraces(traces, props)
%
\end{lyxcode}%
%
%
\item[Parameters:]~
\begin{description}%
\item[\texttt{traces}:]
 A vector of trace objects.
\item[\texttt{props}:]
 A structure with any optional properties.
\begin{description}%
\item[\texttt{calcSE}:]
 If given, calculate standard error instead of deviation.
\item[\texttt{id}:]
 String to replace the id property of averaged trace. The term

"average" or "SD" will be prepended to it.  By default
it will be lengthy and show the arithmetic done.
\end{description}%
\end{description}%
%
\item[Returns:
]~

	avg\_tr: A trace object that holds the average.
	sd\_tr: A trace object that holds the standard deviation or error.
%
%
\item[See also:]%
\hyperlink{ref_trace}{\texttt{trace}}%
\ (p.~\pageref{ref_trace})%
\index[funcref]{trace@\fidxl{trace}}%
%
\item[Author:]%
Cengiz Gunay <cgunay@emory.edu>, 2010/11/09
%
\end{description}
\methodline%
\subsubsection[Method \texttt{binary\_op}]{Method \texttt{trace/binary\_op}}%
\index[funcref]{trace@\fidxl{trace}!binary_op@\fidxl{binary\_op}}%
\label{ref_trace__binary_op}%
\hypertarget{ref_trace__binary_op}{}%
\begin{description}
\item[Summary:]Generic binary operator applications for trace objects.
%
\item[Usage:]~%
\begin{lyxcode}%
result\_tr = binary\_op(left\_tr, right\_tr, op\_func, op\_id, props)
%
\end{lyxcode}%
%
%
\item[Parameters:]~
\begin{description}%
\item[\texttt{left\_tr, right\_tr}:]
 trace objects.
\item[\texttt{op\_func}:]
 Operation function (e.g., @plus).
\item[\texttt{op\_id}:]
 A string to represent the operation that will show up in the

returned id.
\item[\texttt{props}:]
 A structure with any optional properties.
\end{description}%
%
\item[Returns:
]~

   result\_tr: Resulting trace object.
%
\item[Example:]~
\begin{lyxcode} >> result\_tr = binary\_op(vc1, vc2, @minus, '-')
\\%
\end{lyxcode}
%
\item[See also:]%
\hyperlink{ref_trace}{\texttt{trace}}%
\ (p.~\pageref{ref_trace})%
\index[funcref]{trace@\fidxl{trace}}%
, \hyperlink{ref_plus}{\texttt{plus}}%
\ (p.~\pageref{ref_plus})%
\index[funcref]{plus@\fidxl{plus}}%
, \hyperlink{ref_minus}{\texttt{minus}}%
\ (p.~\pageref{ref_minus})%
\index[funcref]{minus@\fidxl{minus}}%
%
\item[Author:]%
Cengiz Gunay <cgunay@emory.edu>, 2010/05/21
%
\end{description}
\methodline%
\subsubsection[Method \texttt{calcAvg}]{Method \texttt{trace/calcAvg}}%
\index[funcref]{trace@\fidxl{trace}!calcAvg@\fidxl{calcAvg}}%
\label{ref_trace__calcAvg}%
\hypertarget{ref_trace__calcAvg}{}%
\begin{description}
\item[Summary:]Calculates the average value of the given period 
 		of the trace, t. 
%
\item[Usage:]~%
\begin{lyxcode}%
avg\_val = calcAvg(t, period)
%
\end{lyxcode}%
%
%
\item[Parameters:]~
\begin{description}%
\item[\texttt{t}:]
 A trace object.
\item[\texttt{period}:]
 A period object (optional).
\end{description}%
%
%
%
\item[See also:]%
\hyperlink{ref_period}{\texttt{period}}%
\ (p.~\pageref{ref_period})%
\index[funcref]{period@\fidxl{period}}%
, \hyperlink{ref_trace}{\texttt{trace}}%
\ (p.~\pageref{ref_trace})%
\index[funcref]{trace@\fidxl{trace}}%
%
\item[Author:]%
Cengiz Gunay <cgunay@emory.edu>, 2004/07/30
%
\end{description}
\methodline%
\subsubsection[Method \texttt{calcMax}]{Method \texttt{trace/calcMax}}%
\index[funcref]{trace@\fidxl{trace}!calcMax@\fidxl{calcMax}}%
\label{ref_trace__calcMax}%
\hypertarget{ref_trace__calcMax}{}%
\begin{description}
\item[Summary:]Calculates the maximal value of the given period 
 		of the trace, t. 
%
\item[Usage:]~%
\begin{lyxcode}%
[max\_val, max\_idx] = calcMax(t, period)
%
\end{lyxcode}%
%
%
\item[Parameters:]~
\begin{description}%
\item[\texttt{t}:]
 A trace object.
\item[\texttt{period}:]
 A period object (optional).
\end{description}%
%
\item[Returns:
]~

	max\_val: The max value.
	max\_idx: Its index in the trace.
%
%
\item[See also:]%
\hyperlink{ref_period}{\texttt{period}}%
\ (p.~\pageref{ref_period})%
\index[funcref]{period@\fidxl{period}}%
, \hyperlink{ref_trace}{\texttt{trace}}%
\ (p.~\pageref{ref_trace})%
\index[funcref]{trace@\fidxl{trace}}%
%
\item[Author:]%
Cengiz Gunay <cgunay@emory.edu>, 2004/07/30
%
\end{description}
\methodline%
\subsubsection[Method \texttt{calcMin}]{Method \texttt{trace/calcMin}}%
\index[funcref]{trace@\fidxl{trace}!calcMin@\fidxl{calcMin}}%
\label{ref_trace__calcMin}%
\hypertarget{ref_trace__calcMin}{}%
\begin{description}
\item[Summary:]Calculates the minimal value of the given period 
 		of the trace, t. 
%
\item[Usage:]~%
\begin{lyxcode}%
[min\_val, min\_idx] = calcMin(t, a\_period)
%
\end{lyxcode}%
%
%
\item[Parameters:]~
\begin{description}%
\item[\texttt{t}:]
 A trace object.
\item[\texttt{a\_period}:]
 A period object (optional).
\end{description}%
%
\item[Returns:
]~

	min\_val: The min value.
	min\_idx: Its index in the trace.
%
%
\item[See also:]%
\hyperlink{ref_period}{\texttt{period}}%
\ (p.~\pageref{ref_period})%
\index[funcref]{period@\fidxl{period}}%
, \hyperlink{ref_trace}{\texttt{trace}}%
\ (p.~\pageref{ref_trace})%
\index[funcref]{trace@\fidxl{trace}}%
%
\item[Author:]%
Cengiz Gunay <cgunay@emory.edu>, 2004/07/30
%
\end{description}
\methodline%
\subsubsection[Method \texttt{display}]{Method \texttt{trace/display}}%
\index[funcref]{trace@\fidxl{trace}!display@\fidxl{display}}%
\label{ref_trace__display}%
\hypertarget{ref_trace__display}{}%
\begin{description}
%
%
%
%
%
%
%
\item[Author:]%
Cengiz Gunay <cgunay@emory.edu>, 2004/08/04
%
\end{description}
\methodline%
\subsubsection[Method \texttt{findFilteredSpikes}]{Method \texttt{trace/findFilteredSpikes}}%
\index[funcref]{trace@\fidxl{trace}!findFilteredSpikes@\fidxl{findFilteredSpikes}}%
\label{ref_trace__findFilteredSpikes}%
\hypertarget{ref_trace__findFilteredSpikes}{}%
\begin{description}
\item[Summary:]Runs a frequency filter over the data and then finds all peaks using findspikes.
%
\item[Usage:]~%
\begin{lyxcode}%
[times, peaks, n] = 
	findFilteredSpikes(t, a\_period, plotit, minamp)
%
\end{lyxcode}%
%
\item[Description:]%
Runs a 50-300 Hz band-pass filter over the data and then calls findspikes.
   The filter both removes low-frequency offset changes, such as 
   cip period effects, and high-frequency noise that is detected 
   as local peaks by findspikes. The spikes found are 
   post-processed to make sure the rise and fall times are consistent.
   Note: The filter employed only works with data sampled at 10kHz.
%%
\item[Parameters:]~
\begin{description}%
\item[\texttt{t}:]
 Trace object
\item[\texttt{a\_period}:]
 Period of interest.
\item[\texttt{plotit}:]
 Plots the spikes found if 1.

minamp (optional): minimum amplitude above baseline that must be reached.
--> adjust as necessary to discriminate spikes from EPSPs.
\item[\texttt{props}:]
 A structure with any optional properties, such as:
\begin{description}%
\item[\texttt{downThreshold}:]
 Size of trough after spike (default=-2)
\end{description}%
\end{description}%
%
\item[Returns:
]~

   times: The times of spikes [dt].
   peaks: The peaks corresponding to the times of spikes.
   n: The number of spikes.
%
%
\item[See also:]%
\hyperlink{ref_findspikes}{\texttt{findspikes}}%
\ (p.~\pageref{ref_findspikes})%
\index[funcref]{findspikes@\fidxl{findspikes}}%
, \hyperlink{ref_period}{\texttt{period}}%
\ (p.~\pageref{ref_period})%
\index[funcref]{period@\fidxl{period}}%
%
\item[Author:]%
Cengiz Gunay <cgunay@emory.edu>, 2004/03/08
%
\end{description}
\methodline%
\subsubsection[Method \texttt{get}]{Method \texttt{trace/get}}%
\index[funcref]{trace@\fidxl{trace}!get@\fidxl{get}}%
\label{ref_trace__get}%
\hypertarget{ref_trace__get}{}%
\begin{description}
\item[Summary:]Defines generic attribute retrieval for objects.
%
%
%
%
%
%
%
\item[Author:]%
Cengiz Gunay <cgunay@emory.edu>, 2004/09/14
%
\end{description}
\methodline%
\subsubsection[Method \texttt{getDy}]{Method \texttt{trace/getDy}}%
\index[funcref]{trace@\fidxl{trace}!getDy@\fidxl{getDy}}%
\label{ref_trace__getDy}%
\hypertarget{ref_trace__getDy}{}%
\begin{description}
\item[Summary:]Returns dy.
%
\item[Usage:]~%
\begin{lyxcode}%
dy = getDy(t)
%
\end{lyxcode}%
%
%
\item[Parameters:]~
\begin{description}%
\item[\texttt{t}:]
 A trace object.
\end{description}%
%
\item[Returns:
]~

	dy: The dy value.
%
%
\item[See also:]%
\hyperlink{ref_trace}{\texttt{trace}}%
\ (p.~\pageref{ref_trace})%
\index[funcref]{trace@\fidxl{trace}}%
%
\item[Author:]%
Cengiz Gunay <cgunay@emory.edu>, 2004/08/31
%
\end{description}
\methodline%
\subsubsection[Method \texttt{getPotResults}]{Method \texttt{trace/getPotResults}}%
\index[funcref]{trace@\fidxl{trace}!getPotResults@\fidxl{getPotResults}}%
\label{ref_trace__getPotResults}%
\hypertarget{ref_trace__getPotResults}{}%
\begin{description}
%
\item[Usage:]~%
\begin{lyxcode}%
results = getPotResults(t)
%
\end{lyxcode}%
%
%
\item[Parameters:]~
\begin{description}%
\item[\texttt{t}:]
 A trace object.
\end{description}%
%
\item[Returns:
]~

	results: A structure associating potential info names to values in mV as
		 follows:
	   min - minimum potential for the whole trace.
	   avg - average potential for the whole trace.
	   max - maximum potential for the whole trace.
%
%
\item[See also:]%
\hyperlink{ref_spike_shape}{\texttt{spike\_shape}}%
\ (p.~\pageref{ref_spike_shape})%
\index[funcref]{spike_shape@\fidxl{spike\_shape}}%
%
\item[Author:]%
Cengiz Gunay <cgunay@emory.edu>, 2004/09/13
%
\end{description}
\methodline%
\subsubsection[Method \texttt{getProfileAllSpikes}]{Method \texttt{trace/getProfileAllSpikes}}%
\index[funcref]{trace@\fidxl{trace}!getProfileAllSpikes@\fidxl{getProfileAllSpikes}}%
\label{ref_trace__getProfileAllSpikes}%
\hypertarget{ref_trace__getProfileAllSpikes}{}%
\begin{description}
\item[Summary:]Creates a trace\_allspikes\_profile object by collecting test results of a trace, analyzing each individual spike.
%
\item[Usage:]~%
\begin{lyxcode}%
profile\_obj = getProfileAllSpikes(a\_trace)
%
\end{lyxcode}%
%
\item[Description:]%
Analyzes the spontaneous (periodIniSpont), pulse (periodPulse) and the
 recovery (periodRecSpont) periods separately and produces spike shape
 distribution results. Rate and CIP measurements are added to these.
%%
\item[Parameters:]~
\begin{description}%
\item[\texttt{a\_trace}:]
 A trace object.
\end{description}%
%
\item[Returns:
]~

	profile\_obj: A trace\_allspikes\_profile object.
%
%
\item[See also:]%
\hyperlink{ref_trace}{\texttt{trace}}%
\ (p.~\pageref{ref_trace})%
\index[funcref]{trace@\fidxl{trace}}%
, \hyperlink{ref_trace_allspikes_profile}{\texttt{trace\_allspikes\_profile}}%
\ (p.~\pageref{ref_trace_allspikes_profile})%
\index[funcref]{trace_allspikes_profile@\fidxl{trace\_allspikes\_profile}}%
%
\item[Author:]%
Cengiz Gunay <cgunay@emory.edu>, 2005/04/26
%
\end{description}
\methodline%
\subsubsection[Method \texttt{getRateResults}]{Method \texttt{trace/getRateResults}}%
\index[funcref]{trace@\fidxl{trace}!getRateResults@\fidxl{getRateResults}}%
\label{ref_trace__getRateResults}%
\hypertarget{ref_trace__getRateResults}{}%
\begin{description}
\item[Summary:]Calculate test results related to spike rate for the
		   whole spike period.
%
\item[Usage:]~%
\begin{lyxcode}%
results = getRateResults(a\_trace, a\_spikes)
%
\end{lyxcode}%
%
%
\item[Parameters:]~
\begin{description}%
\item[\texttt{a\_trace}:]
 A trace object.
\item[\texttt{a\_spikes}:]
 A spikes object.
\end{description}%
%
\item[Returns:
]~

	results: A structure associating test names with result values.
%
%
\item[See also:]%
\hyperlink{ref_trace}{\texttt{trace}}%
\ (p.~\pageref{ref_trace})%
\index[funcref]{trace@\fidxl{trace}}%
, \hyperlink{ref_spikes}{\texttt{spikes}}%
\ (p.~\pageref{ref_spikes})%
\index[funcref]{spikes@\fidxl{spikes}}%
, \hyperlink{ref_spike_shape}{\texttt{spike\_shape}}%
\ (p.~\pageref{ref_spike_shape})%
\index[funcref]{spike_shape@\fidxl{spike\_shape}}%
%
\item[Author:]%
Cengiz Gunay <cgunay@emory.edu>, 2004/08/30
%
\end{description}
\methodline%
\subsubsection[Method \texttt{getResults}]{Method \texttt{trace/getResults}}%
\index[funcref]{trace@\fidxl{trace}!getResults@\fidxl{getResults}}%
\label{ref_trace__getResults}%
\hypertarget{ref_trace__getResults}{}%
\begin{description}
\item[Summary:]Runs all tests defined by this class and return them in a 
		structure.
%
\item[Usage:]~%
\begin{lyxcode}%
results = getResults(a\_trace, a\_spikes)
%
\end{lyxcode}%
%
%
\item[Parameters:]~
\begin{description}%
\item[\texttt{a\_trace}:]
 A trace object.
\item[\texttt{a\_spikes}:]
 spikes object obtained from the trace object.
\end{description}%
%
\item[Returns:
]~

	results: A structure associating test names to values 
		in ms and mV (or mA).
%
%
\item[See also:]%
\hyperlink{ref_spike_shape}{\texttt{spike\_shape}}%
\ (p.~\pageref{ref_spike_shape})%
\index[funcref]{spike_shape@\fidxl{spike\_shape}}%
, \hyperlink{ref_spikes}{\texttt{spikes}}%
\ (p.~\pageref{ref_spikes})%
\index[funcref]{spikes@\fidxl{spikes}}%
%
\item[Author:]%
Cengiz Gunay <cgunay@emory.edu>, 2004/09/13
%
\end{description}
\methodline%
\subsubsection[Method \texttt{getSpike}]{Method \texttt{trace/getSpike}}%
\index[funcref]{trace@\fidxl{trace}!getSpike@\fidxl{getSpike}}%
\label{ref_trace__getSpike}%
\hypertarget{ref_trace__getSpike}{}%
\begin{description}
\item[Summary:]Convert a spike in the trace to a spike\_shape object.
%
\item[Usage:]~%
\begin{lyxcode}%
obj = getSpike(trace, spikes, spike\_num, props)
%
\end{lyxcode}%
%
\item[Description:]%
Creates a spike\_shape object from desired spike. It is more efficient if
 you already have the spikes object.
%%
\item[Parameters:]~
\begin{description}%
\item[\texttt{trace}:]
 A trace object.
\item[\texttt{spikes}:]
 (Optional) A spikes object obtained from trace, 

calculated automatically if given as [].
\item[\texttt{spike\_num}:]
 The index of spike to extract.
\item[\texttt{props}:]
 A structure with any optional properties.
\begin{description}%
\item[\texttt{spike\_id}:]
 A prefix string added to the spike\_shape object's id.
\end{description}%
\end{description}%
%
%
\item[Example:]~
\begin{lyxcode} This will create an annotated plot of the third spike in my\_trace:
\\%
 >> plotFigure(plotResults(getSpike(my\_trace, [], 3)))
\\%
\end{lyxcode}
%
\item[See also:]%
\hyperlink{ref_spike_shape}{\texttt{spike\_shape}}%
\ (p.~\pageref{ref_spike_shape})%
\index[funcref]{spike_shape@\fidxl{spike\_shape}}%
%
\item[Author:]%
Cengiz Gunay <cgunay@emory.edu>, 2005/04/19
%
\end{description}
\methodline%
\subsubsection[Method \texttt{lowpassfilt}]{Method \texttt{trace/lowpassfilt}}%
\index[funcref]{trace@\fidxl{trace}!lowpassfilt@\fidxl{lowpassfilt}}%
\label{ref_trace__lowpassfilt}%
\hypertarget{ref_trace__lowpassfilt}{}%
\begin{description}
\item[Summary:]Applies a low-pass Butterworth filter to the trace data.
%
\item[Usage:]~%
\begin{lyxcode}%
t = lowpassfilt(t, n, cutoff\_freq)
%
\end{lyxcode}%
%
%
\item[Parameters:]~
\begin{description}%
\item[\texttt{t}:]
 A trace object.
\item[\texttt{n}:]
 Order of the filter (e.g., 2)
\item[\texttt{cutoff\_freq}:]
 Cutoff frequency, max <= sampling rate/2 [Hz].
\end{description}%
%
\item[Returns:
]~

   t: updated trace object.
%
%
\item[See also:]%
\hyperlink{ref_trace}{\texttt{trace}}%
\ (p.~\pageref{ref_trace})%
\index[funcref]{trace@\fidxl{trace}}%
, \hyperlink{ref_butter}{\texttt{butter}}%
\ (p.~\pageref{ref_butter})%
\index[funcref]{butter@\fidxl{butter}}%
, \hyperlink{ref_filter}{\texttt{filter}}%
\ (p.~\pageref{ref_filter})%
\index[funcref]{filter@\fidxl{filter}}%
, \hyperlink{ref_filtfilt}{\texttt{filtfilt}}%
\ (p.~\pageref{ref_filtfilt})%
\index[funcref]{filtfilt@\fidxl{filtfilt}}%
%
\item[Author:]%
Cengiz Gunay <cgunay@emory.edu>, 2010/04/08
%
\end{description}
\methodline%
\subsubsection[Method \texttt{medianfilt}]{Method \texttt{trace/medianfilt}}%
\index[funcref]{trace@\fidxl{trace}!medianfilt@\fidxl{medianfilt}}%
\label{ref_trace__medianfilt}%
\hypertarget{ref_trace__medianfilt}{}%
\begin{description}
\item[Summary:]Applies a median filter to the trace data.
%
\item[Usage:]~%
\begin{lyxcode}%
t = medianfilt(t, window\_size)
%
\end{lyxcode}%
%
%
\item[Parameters:]~
\begin{description}%
\item[\texttt{t}:]
 A trace object.
\item[\texttt{window\_size}:]
 N parameter of medianfilt1 (default=3).
\end{description}%
%
\item[Returns:
]~

   t: updated trace object.
%
%
\item[See also:]%
\hyperlink{ref_trace}{\texttt{trace}}%
\ (p.~\pageref{ref_trace})%
\index[funcref]{trace@\fidxl{trace}}%
, \hyperlink{ref_medianfilt1}{\texttt{medianfilt1}}%
\ (p.~\pageref{ref_medianfilt1})%
\index[funcref]{medianfilt1@\fidxl{medianfilt1}}%
%
\item[Author:]%
Cengiz Gunay <cgunay@emory.edu>, 2010/04/05
%
\end{description}
\methodline%
\subsubsection[Method \texttt{minus}]{Method \texttt{trace/minus}}%
\index[funcref]{trace@\fidxl{trace}!minus@\fidxl{minus}}%
\label{ref_trace__minus}%
\hypertarget{ref_trace__minus}{}%
\begin{description}
\item[Summary:]Subtract trace object right\_tr from left\_tr.
%
\item[Usage:]~%
\begin{lyxcode}%
sub\_tr = minus(left\_tr, right\_tr, props)
%
\end{lyxcode}%
%
%
\item[Parameters:]~
\begin{description}%
\item[\texttt{left\_tr, right\_tr}:]
 trace objects.
\item[\texttt{props}:]
 A structure with any optional properties.
\end{description}%
%
\item[Returns:
]~

   sub\_tr: trace object with subtracted data of left\_tr.
%
\item[Example:]~
\begin{lyxcode} >> sub\_tr = minus(vc1, vc2)
\\%
 OR
\\%
 >> sub\_tr = vc1 - vc2;
\\%
 plot the subtracted voltage clamp
\\%
 >> plot(sub\_tr)
\\%
\end{lyxcode}
%
\item[See also:]%
\hyperlink{ref_trace}{\texttt{trace}}%
\ (p.~\pageref{ref_trace})%
\index[funcref]{trace@\fidxl{trace}}%
, \hyperlink{ref_minus}{\texttt{minus}}%
\ (p.~\pageref{ref_minus})%
\index[funcref]{minus@\fidxl{minus}}%
%
\item[Author:]%
Cengiz Gunay <cgunay@emory.edu>, 2010/03/11
%
\end{description}
\methodline%
\subsubsection[Method \texttt{mtimes}]{Method \texttt{trace/mtimes}}%
\index[funcref]{trace@\fidxl{trace}!mtimes@\fidxl{mtimes}}%
\label{ref_trace__mtimes}%
\hypertarget{ref_trace__mtimes}{}%
\begin{description}
\item[Summary:]Matrix multiply trace object right\_tr with left\_tr.
%
\item[Usage:]~%
\begin{lyxcode}%
res\_tr = mtimes(left\_tr, right\_tr, props)
%
\end{lyxcode}%
%
%
\item[Parameters:]~
\begin{description}%
\item[\texttt{left\_tr, right\_tr}:]
 trace objects.
\item[\texttt{props}:]
 A structure with any optional properties.
\end{description}%
%
\item[Returns:
]~

   res\_tr: resulting trace object
%
\item[Example:]~
\begin{lyxcode} >> res\_tr = mtimes(vc1, vc2)
\\%
 OR
\\%
 >> res\_tr = vc1 * vc2;
\\%
 plot the resulting trace
\\%
 >> plot(res\_tr)
\\%
\end{lyxcode}
%
\item[See also:]%
\hyperlink{ref_trace}{\texttt{trace}}%
\ (p.~\pageref{ref_trace})%
\index[funcref]{trace@\fidxl{trace}}%
, \hyperlink{ref_mtimes}{\texttt{mtimes}}%
\ (p.~\pageref{ref_mtimes})%
\index[funcref]{mtimes@\fidxl{mtimes}}%
%
\item[Author:]%
Cengiz Gunay <cgunay@emory.edu>, 2010/03/11
%
\end{description}
\methodline%
\subsubsection[Method \texttt{periodWhole}]{Method \texttt{trace/periodWhole}}%
\index[funcref]{trace@\fidxl{trace}!periodWhole@\fidxl{periodWhole}}%
\label{ref_trace__periodWhole}%
\hypertarget{ref_trace__periodWhole}{}%
\begin{description}
\item[Summary:]Returns the boundaries of the whole period of trace, t. 
%
\item[Usage:]~%
\begin{lyxcode}%
whole\_period = periodWhole(t)
%
\end{lyxcode}%
%
%
\item[Parameters:]~
\begin{description}%
\item[\texttt{t}:]
 A trace object.
\end{description}%
%
%
%
\item[See also:]%
\hyperlink{ref_period}{\texttt{period}}%
\ (p.~\pageref{ref_period})%
\index[funcref]{period@\fidxl{period}}%
, \hyperlink{ref_trace}{\texttt{trace}}%
\ (p.~\pageref{ref_trace})%
\index[funcref]{trace@\fidxl{trace}}%
%
\item[Author:]%
Cengiz Gunay <cgunay@emory.edu>, 2004/07/30
%
\end{description}
\methodline%
\subsubsection[Method \texttt{plot}]{Method \texttt{trace/plot}}%
\index[funcref]{trace@\fidxl{trace}!plot@\fidxl{plot}}%
\label{ref_trace__plot}%
\hypertarget{ref_trace__plot}{}%
\begin{description}
\item[Summary:]Plots a trace.
%
\item[Usage:]~%
\begin{lyxcode}%
h = plot(t)
%
\end{lyxcode}%
%
%
\item[Parameters:]~
\begin{description}%
\item[\texttt{t}:]
 A trace object.
\item[\texttt{title\_str}:]
 (Optional) String to append to plot title.
\item[\texttt{props}:]
 A structure with any optional properties, passed to plot\_abstract.
\end{description}%
%
\item[Returns:
]~

	h: Handle to figure object.
%
%
\item[See also:]%
\hyperlink{ref_trace}{\texttt{trace}}%
\ (p.~\pageref{ref_trace})%
\index[funcref]{trace@\fidxl{trace}}%
, \hyperlink{ref_plot_abstract}{\texttt{plot\_abstract}}%
\ (p.~\pageref{ref_plot_abstract})%
\index[funcref]{plot_abstract@\fidxl{plot\_abstract}}%
%
\item[Author:]%
Cengiz Gunay <cgunay@emory.edu>, 2004/08/04
%
\end{description}
\methodline%
\subsubsection[Method \texttt{plotData}]{Method \texttt{trace/plotData}}%
\index[funcref]{trace@\fidxl{trace}!plotData@\fidxl{plotData}}%
\label{ref_trace__plotData}%
\hypertarget{ref_trace__plotData}{}%
\begin{description}
\item[Summary:]Plots a trace.
%
\item[Usage:]~%
\begin{lyxcode}%
a\_plot = plotData(t, title\_str, props)
%
\end{lyxcode}%
%
\item[Description:]%
If t is a vector of traces, returns a vector of plot objects.
%%
\item[Parameters:]~
\begin{description}%
\item[\texttt{t}:]
 A trace object.
\item[\texttt{title\_str}:]
 (Optional) String to append to plot title.
\item[\texttt{props}:]
 A structure with any optional properties.
\begin{description}%
\item[\texttt{timeScale}:]
 's' for seconds, or 'ms' for milliseconds.
\item[\texttt{quiet}:]
 If 1, only display given title\_str.

(rest passed to plot\_abstract.)
\end{description}%
\end{description}%
%
\item[Returns:
]~

	a\_plot: A plot\_abstract object that can be visualized.
%
%
\item[See also:]%
\hyperlink{ref_trace}{\texttt{trace}}%
\ (p.~\pageref{ref_trace})%
\index[funcref]{trace@\fidxl{trace}}%
, \hyperlink{ref_trace__plot}{\texttt{trace/plot}}%
\ (p.~\pageref{ref_trace__plot})%
\index[funcref]{trace@\fidxl{trace}!plot@\fidxl{plot}}%
, \hyperlink{ref_plot_abstract}{\texttt{plot\_abstract}}%
\ (p.~\pageref{ref_plot_abstract})%
\index[funcref]{plot_abstract@\fidxl{plot\_abstract}}%
%
\item[Author:]%
Cengiz Gunay <cgunay@emory.edu>, 2004/11/17
%
\end{description}
\methodline%
\subsubsection[Method \texttt{plot\_abstract}]{Method \texttt{trace/plot\_abstract}}%
\index[funcref]{trace@\fidxl{trace}!plot_abstract@\fidxl{plot\_abstract}}%
\label{ref_trace__plot_abstract}%
\hypertarget{ref_trace__plot_abstract}{}%
\begin{description}
\item[Summary:]Plots a trace by calling plotData.
%
\item[Usage:]~%
\begin{lyxcode}%
a\_plot = plot\_abstract(t, title\_str, props)
%
\end{lyxcode}%
%
\item[Description:]%
If t is a vector of traces, returns a vector of plot objects.
%%
\item[Parameters:]~
\begin{description}%
\item[\texttt{t}:]
 A trace object.
\item[\texttt{title\_str}:]
 (Optional) String to append to plot title.
\item[\texttt{props}:]
 A structure with any optional properties.
\begin{description}%
\item[\texttt{timeScale}:]
 's' for seconds, or 'ms' for milliseconds.

(rest passed to plot\_abstract.)
\end{description}%
\end{description}%
%
\item[Returns:
]~

	a\_plot: A plot\_abstract object that can be visualized.
%
%
\item[See also:]%
\hyperlink{ref_trace}{\texttt{trace}}%
\ (p.~\pageref{ref_trace})%
\index[funcref]{trace@\fidxl{trace}}%
, \hyperlink{ref_trace__plot}{\texttt{trace/plot}}%
\ (p.~\pageref{ref_trace__plot})%
\index[funcref]{trace@\fidxl{trace}!plot@\fidxl{plot}}%
, \hyperlink{ref_plot_abstract}{\texttt{plot\_abstract}}%
\ (p.~\pageref{ref_plot_abstract})%
\index[funcref]{plot_abstract@\fidxl{plot\_abstract}}%
%
\item[Author:]%
Cengiz Gunay <cgunay@emory.edu>, 2004/11/17
%
\end{description}
\methodline%
\subsubsection[Method \texttt{plus}]{Method \texttt{trace/plus}}%
\index[funcref]{trace@\fidxl{trace}!plus@\fidxl{plus}}%
\label{ref_trace__plus}%
\hypertarget{ref_trace__plus}{}%
\begin{description}
\item[Summary:]Subtract trace object right\_tr from left\_tr.
%
\item[Usage:]~%
\begin{lyxcode}%
sub\_tr = plus(left\_tr, right\_tr, props)
%
\end{lyxcode}%
%
%
\item[Parameters:]~
\begin{description}%
\item[\texttt{left\_tr, right\_tr}:]
 trace objects.
\item[\texttt{props}:]
 A structure with any optional properties.
\end{description}%
%
\item[Returns:
]~

   sub\_tr: trace object with subtracted data of left\_tr.
%
\item[Example:]~
\begin{lyxcode} >> sub\_tr = plus(vc1, vc2)
\\%
 OR
\\%
 >> sub\_tr = vc1 + vc2;
\\%
 plot the subtracted voltage clamp
\\%
 >> plot(sub\_tr)
\\%
\end{lyxcode}
%
\item[See also:]%
\hyperlink{ref_trace}{\texttt{trace}}%
\ (p.~\pageref{ref_trace})%
\index[funcref]{trace@\fidxl{trace}}%
, \hyperlink{ref_plus}{\texttt{plus}}%
\ (p.~\pageref{ref_plus})%
\index[funcref]{plus@\fidxl{plus}}%
%
\item[Author:]%
Cengiz Gunay <cgunay@emory.edu>, 2010/03/11
%
\end{description}
\methodline%
\subsubsection[Method \texttt{power}]{Method \texttt{trace/power}}%
\index[funcref]{trace@\fidxl{trace}!power@\fidxl{power}}%
\label{ref_trace__power}%
\hypertarget{ref_trace__power}{}%
\begin{description}
\item[Summary:]left\_tr to the power of right\_tr (either can be scalars).
%
\item[Usage:]~%
\begin{lyxcode}%
res\_tr = power(left\_tr, right\_tr, props)
%
\end{lyxcode}%
%
%
\item[Parameters:]~
\begin{description}%
\item[\texttt{left\_tr, right\_tr}:]
 trace objects.
\item[\texttt{props}:]
 A structure with any optional properties.
\end{description}%
%
\item[Returns:
]~

   res\_tr: resulting trace object
%
\item[Example:]~
\begin{lyxcode} >> res\_tr = power(vc1, 3)
\\%
 OR
\\%
 >> res\_tr = vc1 .\textasciicircum{} 2;
\\%
 plot the resulting trace
\\%
 >> plot(res\_tr)
\\%
\end{lyxcode}
%
\item[See also:]%
\hyperlink{ref_trace}{\texttt{trace}}%
\ (p.~\pageref{ref_trace})%
\index[funcref]{trace@\fidxl{trace}}%
, \hyperlink{ref_power}{\texttt{power}}%
\ (p.~\pageref{ref_power})%
\index[funcref]{power@\fidxl{power}}%
%
\item[Author:]%
Cengiz Gunay <cgunay@emory.edu>, 2010/11/09
%
\end{description}
\methodline%
\subsubsection[Method \texttt{rdivide}]{Method \texttt{trace/rdivide}}%
\index[funcref]{trace@\fidxl{trace}!rdivide@\fidxl{rdivide}}%
\label{ref_trace__rdivide}%
\hypertarget{ref_trace__rdivide}{}%
\begin{description}
\item[Summary:]Scalar divide trace object left\_tr by right\_tr.
%
\item[Usage:]~%
\begin{lyxcode}%
res\_tr = rdivide(left\_tr, right\_tr, props)
%
\end{lyxcode}%
%
%
\item[Parameters:]~
\begin{description}%
\item[\texttt{left\_tr, right\_tr}:]
 trace objects.
\item[\texttt{props}:]
 A structure with any optional properties.
\end{description}%
%
\item[Returns:
]~

   res\_tr: resulting trace object
%
\item[Example:]~
\begin{lyxcode} >> res\_tr = rdivide(vc1, vc2)
\\%
 OR
\\%
 >> res\_tr = vc1 ./ vc2;
\\%
 plot the resulting trace
\\%
 >> plot(res\_tr)
\\%
\end{lyxcode}
%
\item[See also:]%
\hyperlink{ref_trace}{\texttt{trace}}%
\ (p.~\pageref{ref_trace})%
\index[funcref]{trace@\fidxl{trace}}%
, \hyperlink{ref_rdivide}{\texttt{rdivide}}%
\ (p.~\pageref{ref_rdivide})%
\index[funcref]{rdivide@\fidxl{rdivide}}%
%
\item[Author:]%
Cengiz Gunay <cgunay@emory.edu>, 2010/11/09
%
\end{description}
\methodline%
\subsubsection[Method \texttt{runAvg}]{Method \texttt{trace/runAvg}}%
\index[funcref]{trace@\fidxl{trace}!runAvg@\fidxl{runAvg}}%
\label{ref_trace__runAvg}%
\hypertarget{ref_trace__runAvg}{}%
\begin{description}
\item[Summary:]Returns a trace which is the running average of this trace.
%
\item[Usage:]~%
\begin{lyxcode}%
avg\_t = runAvg(t)
%
\end{lyxcode}%
%
%
\item[Parameters:]~
\begin{description}%
\item[\texttt{t}:]
 A trace object.
\end{description}%
%
\item[Returns:
]~

	avg\_t: A trace object that contains the running average of this trace.
%
%
\item[See also:]%
\hyperlink{ref_trace}{\texttt{trace}}%
\ (p.~\pageref{ref_trace})%
\index[funcref]{trace@\fidxl{trace}}%
%
\item[Author:]%
Cengiz Gunay <cgunay@emory.edu>, 2008/05/14
%
\end{description}
\methodline%
\subsubsection[Method \texttt{saveAsNeuronVecAscii}]{Method \texttt{trace/saveAsNeuronVecAscii}}%
\index[funcref]{trace@\fidxl{trace}!saveAsNeuronVecAscii@\fidxl{saveAsNeuronVecAscii}}%
\label{ref_trace__saveAsNeuronVecAscii}%
\hypertarget{ref_trace__saveAsNeuronVecAscii}{}%
\begin{description}
\item[Summary:]Writes trace in ASCII file in Neuron simulator Vector format.
%
\item[Usage:]~%
\begin{lyxcode}%
saveAsNeuronVecAscii(trace, filename)
%
\end{lyxcode}%
%
\item[Description:]%
Data converted to Neuron units of nA and mV. Uses writeNeuronVecAscii.
%%
\item[Parameters:]~
\begin{description}%
\item[\texttt{a\_t}:]
 A trace object.
\item[\texttt{filename}:]
 (optional) Full path to Neuron file. If omitted, 

a\_t.id prefixed with 'neuron-vec-' is used as filename in current directory.
\end{description}%
%
\item[Returns:
]~

%
\item[Example:]~
\begin{lyxcode}   saveAsNeuronVecAscii('myvec.dat', data, 1e-4, 1e-3, 'V', 'my membrane voltage');
\\%
\end{lyxcode}
%
%
\item[Author:]%
Cengiz Gunay <cengique@users.sf.net> 2012/03/23
%
\end{description}
\methodline%
\subsubsection[Method \texttt{set}]{Method \texttt{trace/set}}%
\index[funcref]{trace@\fidxl{trace}!set@\fidxl{set}}%
\label{ref_trace__set}%
\hypertarget{ref_trace__set}{}%
\begin{description}
\item[Summary:]Generic method for setting object attributes.
%
%
%
%
%
%
%
\item[Author:]%
Cengiz Gunay <cgunay@emory.edu>, 2004/10/08
%
\end{description}
\methodline%
\subsubsection[Method \texttt{setProp}]{Method \texttt{trace/setProp}}%
\index[funcref]{trace@\fidxl{trace}!setProp@\fidxl{setProp}}%
\label{ref_trace__setProp}%
\hypertarget{ref_trace__setProp}{}%
\begin{description}
\item[Summary:]Generic method for setting optional object properties.
%
\item[Usage:]~%
\begin{lyxcode}%
obj = setProp(obj, prop1, val1, prop2, val2, ...)
%
\end{lyxcode}%
%
\item[Description:]%
Modifies or adds property values. As many property name-value 
 pairs can be specified.
%%
\item[Parameters:]~
\begin{description}%
\item[\texttt{obj}:]
 Any object that has a props field.
\item[\texttt{attr}:]
 Property name
\item[\texttt{val}:]
 Property value.
\end{description}%
%
\item[Returns:
]~

	obj: The new object with the updated properties.
%
%
\item[See also:]%
%
\item[Author:]%
Cengiz Gunay <cgunay@emory.edu>, 2004/11/22
%
\end{description}
\methodline%
\subsubsection[Method \texttt{spikes}]{Method \texttt{trace/spikes}}%
\index[funcref]{trace@\fidxl{trace}!spikes@\fidxl{spikes}}%
\label{ref_trace__spikes}%
\hypertarget{ref_trace__spikes}{}%
\begin{description}
\item[Summary:]Convert trace to spikes object for spike timing calculations.
%
\item[Usage:]~%
\begin{lyxcode}%
obj = spikes(trace, a\_period, plotit, minamp)
%
\end{lyxcode}%
%
\item[Description:]%
Creates a spikes object.
%%
\item[Parameters:]~
\begin{description}%
\item[\texttt{trace}:]
 A trace object.
\item[\texttt{a\_period}:]
 A period object denoting the part of trace of interest 

(optional, if empty vector, taken as wholePeriod).
\item[\texttt{plotit}:]
 If non-zero, a plot is generated for showing spikes found

(optional).
\item[\texttt{minamp}:]
 minimum amplitude that must be reached if using findFilteredSpikes.

--> adjust as needed to discriminate spikes from EPSPs.
(optional)
\end{description}%
%
%
%
\item[See also:]%
\hyperlink{ref_spikes}{\texttt{spikes}}%
\ (p.~\pageref{ref_spikes})%
\index[funcref]{spikes@\fidxl{spikes}}%
%
\item[Author:]%
Cengiz Gunay <cgunay@emory.edu>, 2004/07/30
%
\end{description}
\methodline%
\subsubsection[Method \texttt{spike\_shape}]{Method \texttt{trace/spike\_shape}}%
\index[funcref]{trace@\fidxl{trace}!spike_shape@\fidxl{spike\_shape}}%
\label{ref_trace__spike_shape}%
\hypertarget{ref_trace__spike_shape}{}%
\begin{description}
\item[Summary:]Convert averaged spikes in the trace to a spike\_shape object.
%
\item[Usage:]~%
\begin{lyxcode}%
obj = spike\_shape(trace, spikes, props)
%
\end{lyxcode}%
%
\item[Description:]%
Creates a spike\_shape object from averaged spikes. USE THIS ONLY IF
 YOU WANT TO USE AVERAGED SPIKE SHAPES.
%%
\item[Parameters:]~
\begin{description}%
\item[\texttt{trace}:]
 A trace object.
\item[\texttt{spikes}:]
 A spikes object on trace.
\end{description}%
%
%
%
\item[See also:]%
\hyperlink{ref_spike_shape}{\texttt{spike\_shape}}%
\ (p.~\pageref{ref_spike_shape})%
\index[funcref]{spike_shape@\fidxl{spike\_shape}}%
%
\item[Author:]%
Cengiz Gunay <cgunay@emory.edu>, 2004/08/04
%
\end{description}
\methodline%
\subsubsection[Method \texttt{sqrt}]{Method \texttt{trace/sqrt}}%
\index[funcref]{trace@\fidxl{trace}!sqrt@\fidxl{sqrt}}%
\label{ref_trace__sqrt}%
\hypertarget{ref_trace__sqrt}{}%
\begin{description}
\item[Summary:]Square root of trace object.
%
\item[Usage:]~%
\begin{lyxcode}%
res\_tr = sqrt(a\_tr, props)
%
\end{lyxcode}%
%
%
\item[Parameters:]~
\begin{description}%
\item[\texttt{a\_tr}:]
 A trace object.
\item[\texttt{props}:]
 A structure with any optional properties.
\end{description}%
%
\item[Returns:
]~

   res\_tr: Resulting trace object.
%
\item[Example:]~
\begin{lyxcode} >> a\_tr = sqrt(vc1)
\\%
 plot the result
\\%
 >> plot(a\_tr)
\\%
\end{lyxcode}
%
\item[See also:]%
\hyperlink{ref_trace}{\texttt{trace}}%
\ (p.~\pageref{ref_trace})%
\index[funcref]{trace@\fidxl{trace}}%
, \hyperlink{ref_sqrt}{\texttt{sqrt}}%
\ (p.~\pageref{ref_sqrt})%
\index[funcref]{sqrt@\fidxl{sqrt}}%
%
\item[Author:]%
Cengiz Gunay <cgunay@emory.edu>, 2010/03/11
%
\end{description}
\methodline%
\subsubsection[Method \texttt{subsasgn}]{Method \texttt{trace/subsasgn}}%
\index[funcref]{trace@\fidxl{trace}!subsasgn@\fidxl{subsasgn}}%
\label{ref_trace__subsasgn}%
\hypertarget{ref_trace__subsasgn}{}%
\begin{description}
\item[Summary:]Defines generic index-based assignment for objects.
%
%
%
%
%
%
%
\item[Author:]%
Cengiz Gunay <cgunay@emory.edu>, 2006/02/06
%
\end{description}
\methodline%
\subsubsection[Method \texttt{subsref}]{Method \texttt{trace/subsref}}%
\index[funcref]{trace@\fidxl{trace}!subsref@\fidxl{subsref}}%
\label{ref_trace__subsref}%
\hypertarget{ref_trace__subsref}{}%
\begin{description}
\item[Summary:]Defines generic indexing for objects.
%
%
%
%
%
%
%
\item[Author:]%
Cengiz Gunay <cgunay@emory.edu>, 2004/08/04
%
\end{description}
\methodline%
\subsubsection[Method \texttt{times}]{Method \texttt{trace/times}}%
\index[funcref]{trace@\fidxl{trace}!times@\fidxl{times}}%
\label{ref_trace__times}%
\hypertarget{ref_trace__times}{}%
\begin{description}
\item[Summary:]Scalar multiply trace object right\_tr with left\_tr.
%
\item[Usage:]~%
\begin{lyxcode}%
res\_tr = times(left\_tr, right\_tr, props)
%
\end{lyxcode}%
%
%
\item[Parameters:]~
\begin{description}%
\item[\texttt{left\_tr, right\_tr}:]
 trace objects.
\item[\texttt{props}:]
 A structure with any optional properties.
\end{description}%
%
\item[Returns:
]~

   res\_tr: resulting trace object
%
\item[Example:]~
\begin{lyxcode} >> res\_tr = times(vc1, vc2)
\\%
 OR
\\%
 >> res\_tr = vc1 .* vc2;
\\%
 plot the resulting trace
\\%
 >> plot(res\_tr)
\\%
\end{lyxcode}
%
\item[See also:]%
\hyperlink{ref_trace}{\texttt{trace}}%
\ (p.~\pageref{ref_trace})%
\index[funcref]{trace@\fidxl{trace}}%
, \hyperlink{ref_times}{\texttt{times}}%
\ (p.~\pageref{ref_times})%
\index[funcref]{times@\fidxl{times}}%
%
\item[Author:]%
Cengiz Gunay <cgunay@emory.edu>, 2010/03/11
%
\end{description}
\methodline%
\subsubsection[Method \texttt{uminus}]{Method \texttt{trace/uminus}}%
\index[funcref]{trace@\fidxl{trace}!uminus@\fidxl{uminus}}%
\label{ref_trace__uminus}%
\hypertarget{ref_trace__uminus}{}%
\begin{description}
\item[Summary:]Revert sign of trace object.
%
\item[Usage:]~%
\begin{lyxcode}%
res\_tr = uminus(a\_tr, props)
%
\end{lyxcode}%
%
%
\item[Parameters:]~
\begin{description}%
\item[\texttt{a\_tr}:]
 A trace object.
\item[\texttt{props}:]
 A structure with any optional properties.
\end{description}%
%
\item[Returns:
]~

   res\_tr: Resulting trace object.
%
\item[Example:]~
\begin{lyxcode} >> a\_tr = uminus(vc1)
\\%
 OR
\\%
 >> a\_tr = -vc1;
\\%
 plot the result
\\%
 >> plot(a\_tr)
\\%
\end{lyxcode}
%
\item[See also:]%
\hyperlink{ref_trace}{\texttt{trace}}%
\ (p.~\pageref{ref_trace})%
\index[funcref]{trace@\fidxl{trace}}%
, \hyperlink{ref_uminus}{\texttt{uminus}}%
\ (p.~\pageref{ref_uminus})%
\index[funcref]{uminus@\fidxl{uminus}}%
%
\item[Author:]%
Cengiz Gunay <cgunay@emory.edu>, 2010/03/11
%
\end{description}
\methodline%
\subsubsection[Method \texttt{unary\_op}]{Method \texttt{trace/unary\_op}}%
\index[funcref]{trace@\fidxl{trace}!unary_op@\fidxl{unary\_op}}%
\label{ref_trace__unary_op}%
\hypertarget{ref_trace__unary_op}{}%
\begin{description}
\item[Summary:]Generic unary operator applications for trace objects.
%
\item[Usage:]~%
\begin{lyxcode}%
result\_tr = unary\_op(a\_tr, op\_func, op\_id, props)
%
\end{lyxcode}%
%
%
\item[Parameters:]~
\begin{description}%
\item[\texttt{a\_tr}:]
 A trace object.
\item[\texttt{op\_func}:]
 Unary operation function (e.g., @uminus).
\item[\texttt{op\_id}:]
 A string to represent the operation that will show up in the

returned id.
\item[\texttt{props}:]
 A structure with any optional properties.
\end{description}%
%
\item[Returns:
]~

   result\_tr: Resulting trace object.
%
\item[Example:]~
\begin{lyxcode} >> result\_tr = unary\_op(vc1, @uminus, '-')
\\%
\end{lyxcode}
%
\item[See also:]%
\hyperlink{ref_trace}{\texttt{trace}}%
\ (p.~\pageref{ref_trace})%
\index[funcref]{trace@\fidxl{trace}}%
, \hyperlink{ref_binary_op}{\texttt{binary\_op}}%
\ (p.~\pageref{ref_binary_op})%
\index[funcref]{binary_op@\fidxl{binary\_op}}%
, \hyperlink{ref_uminus}{\texttt{uminus}}%
\ (p.~\pageref{ref_uminus})%
\index[funcref]{uminus@\fidxl{uminus}}%
, \hyperlink{ref_sqrt}{\texttt{sqrt}}%
\ (p.~\pageref{ref_sqrt})%
\index[funcref]{sqrt@\fidxl{sqrt}}%
%
\item[Author:]%
Cengiz Gunay <cgunay@emory.edu>, 2010/11/09
%
\end{description}
\methodline%
\subsubsection[Method \texttt{withinPeriod}]{Method \texttt{trace/withinPeriod}}%
\index[funcref]{trace@\fidxl{trace}!withinPeriod@\fidxl{withinPeriod}}%
\label{ref_trace__withinPeriod}%
\hypertarget{ref_trace__withinPeriod}{}%
\begin{description}
\item[Summary:]Returns a trace object valid only within the given period.
%
\item[Usage:]~%
\begin{lyxcode}%
[obj a\_period] = withinPeriod(t, a\_period, props)
%
\end{lyxcode}%
%
%
\item[Parameters:]~
\begin{description}%
\item[\texttt{t}:]
 A trace object.
\item[\texttt{a\_period}:]
 The desired period
\item[\texttt{props}:]
 A structure with any optional properties.
\begin{description}%
\item[\texttt{useAvailable}:]
 If 1, don't stop if period not available, use maximum available.
\end{description}%
\end{description}%
%
\item[Returns:
]~

	obj: A trace object
	a\_period: The period object, updated if useAvailable is requested.
%
%
\item[See also:]%
\hyperlink{ref_trace}{\texttt{trace}}%
\ (p.~\pageref{ref_trace})%
\index[funcref]{trace@\fidxl{trace}}%
, \hyperlink{ref_period}{\texttt{period}}%
\ (p.~\pageref{ref_period})%
\index[funcref]{period@\fidxl{period}}%
%
\item[Author:]%
Cengiz Gunay <cgunay@emory.edu>, 2004/08/25
%
\end{description}
\methodline%
\subsection{Class \texttt{trace\_allspikes\_profile}}%
\index[funcref]{trace_allspikes_profile@\fidxl{trace\_allspikes\_profile}|boldhyperpage}%
\label{ref_trace_allspikes_profile}%
\hypertarget{ref_trace_allspikes_profile}{}%
\subsubsection[Constructor \texttt{trace\_allspikes\_profile}]{Constructor \texttt{trace\_allspikes\_profile/trace\_allspikes\_profile}}%
\index[funcref]{trace_allspikes_profile@\fidxl{trace\_allspikes\_profile}!trace_allspikes_profile@\fidxl{trace\_allspikes\_profile}}%
\label{ref_trace_allspikes_profile__trace_allspikes_profile}%
\hypertarget{ref_trace_allspikes_profile__trace_allspikes_profile}{}%
\begin{description}
\item[Summary:]Collects individual spike-based test results of a trace.
%
\item[Usage:]~%
\begin{lyxcode}%
obj = 
   trace\_allspikes\_profile(a\_trace, a\_spikes, a\_spikes\_db, results\_obj, props)
%
\end{lyxcode}%
%
\item[Description:]%
This is a subclass of results\_profile. It is made to be used from 
 subclass constructors.
%%
\item[Parameters:]~
\begin{description}%
\item[\texttt{a\_trace}:]
 A trace object.
\item[\texttt{a\_spikes}:]
 A spikes object.
\item[\texttt{spont\_spikes\_db, pulse\_spikes\_db, recov\_spikes\_db}:]
 

tests\_dbs with spontaneous, pulse and recovery period spike info.
\item[\texttt{results\_obj}:]
 A results\_profile object with test results.
\item[\texttt{id}:]
 Identification string.
\item[\texttt{props}:]
 A structure with any optional properties.
\end{description}%
%
\item[Returns a structure object with the following fields:
]~

	trace, spikes, spont\_spikes\_db, 
	pulse\_spikes\_db, recov\_spikes\_db, props
%
%
\item[See also:]%
\hyperlink{ref_trace}{\texttt{trace}}%
\ (p.~\pageref{ref_trace})%
\index[funcref]{trace@\fidxl{trace}}%
, \hyperlink{ref_spikes}{\texttt{spikes}}%
\ (p.~\pageref{ref_spikes})%
\index[funcref]{spikes@\fidxl{spikes}}%
, \hyperlink{ref_tests_db}{\texttt{tests\_db}}%
\ (p.~\pageref{ref_tests_db})%
\index[funcref]{tests_db@\fidxl{tests\_db}}%
%
\item[Author:]%
Cengiz Gunay <cgunay@emory.edu>, 2005/05/04
%
\end{description}
\methodline%
\subsubsection[Method \texttt{display}]{Method \texttt{trace\_allspikes\_profile/display}}%
\index[funcref]{trace_allspikes_profile@\fidxl{trace\_allspikes\_profile}!display@\fidxl{display}}%
\label{ref_trace_allspikes_profile__display}%
\hypertarget{ref_trace_allspikes_profile__display}{}%
\begin{description}
%
%
%
%
%
%
%
\item[Author:]%
Cengiz Gunay <cgunay@emory.edu>, 2004/08/04
%
\end{description}
\methodline%
\subsubsection[Method \texttt{get}]{Method \texttt{trace\_allspikes\_profile/get}}%
\index[funcref]{trace_allspikes_profile@\fidxl{trace\_allspikes\_profile}!get@\fidxl{get}}%
\label{ref_trace_allspikes_profile__get}%
\hypertarget{ref_trace_allspikes_profile__get}{}%
\begin{description}
\item[Summary:]Defines generic attribute retrieval for objects.
%
%
%
%
%
%
%
\item[Author:]%
Cengiz Gunay <cgunay@emory.edu>, 2004/09/14
%
\end{description}
\methodline%
\subsubsection[Method \texttt{set}]{Method \texttt{trace\_allspikes\_profile/set}}%
\index[funcref]{trace_allspikes_profile@\fidxl{trace\_allspikes\_profile}!set@\fidxl{set}}%
\label{ref_trace_allspikes_profile__set}%
\hypertarget{ref_trace_allspikes_profile__set}{}%
\begin{description}
\item[Summary:]Generic method for setting object attributes.
%
%
%
%
%
%
%
\item[Author:]%
Cengiz Gunay <cgunay@emory.edu>, 2004/10/08
%
\end{description}
\methodline%
\subsection{Class \texttt{trace\_profile}}%
\index[funcref]{trace_profile@\fidxl{trace\_profile}|boldhyperpage}%
\label{ref_trace_profile}%
\hypertarget{ref_trace_profile}{}%
\subsubsection[Constructor \texttt{trace\_profile}]{Constructor \texttt{trace\_profile/trace\_profile}}%
\index[funcref]{trace_profile@\fidxl{trace\_profile}!trace_profile@\fidxl{trace\_profile}}%
\label{ref_trace_profile__trace_profile}%
\hypertarget{ref_trace_profile__trace_profile}{}%
\begin{description}
\item[Summary:]Creates and collects test results of a trace.
%
%
\item[Description:]%
The first usage is fully customizable to be used from subclass constructors.
 The second usage generates the spikes and spike\_shape objects, and
 collects some generic test results from them. This usage is only provided
 as an example and is not used practically.
%%
\item[Parameters:]~
\begin{description}%
\item[\texttt{data\_src}:]
 The trace column OR the .MAT filename.
\item[\texttt{dt}:]
 Time resolution [s]
\item[\texttt{dy}:]
 y-axis resolution [ISI (V, A, etc.)]
\item[\texttt{props}:]
 See trace object.
\end{description}%
%
\item[Returns a structure object with the following fields:
]~

	trace, spikes, spike\_shape, results, id, props.
%
%
\item[See also:]%
\hyperlink{ref_trace}{\texttt{trace}}%
\ (p.~\pageref{ref_trace})%
\index[funcref]{trace@\fidxl{trace}}%
, \hyperlink{ref_spikes}{\texttt{spikes}}%
\ (p.~\pageref{ref_spikes})%
\index[funcref]{spikes@\fidxl{spikes}}%
, \hyperlink{ref_spike_shape}{\texttt{spike\_shape}}%
\ (p.~\pageref{ref_spike_shape})%
\index[funcref]{spike_shape@\fidxl{spike\_shape}}%
%
\item[Author:]%
Cengiz Gunay <cgunay@emory.edu>, 2004/09/13
%
\end{description}
\methodline%
\subsubsection[Method \texttt{get}]{Method \texttt{trace\_profile/get}}%
\index[funcref]{trace_profile@\fidxl{trace\_profile}!get@\fidxl{get}}%
\label{ref_trace_profile__get}%
\hypertarget{ref_trace_profile__get}{}%
\begin{description}
\item[Summary:]Defines generic attribute retrieval for objects.
%
%
%
%
%
%
%
\item[Author:]%
Cengiz Gunay <cgunay@emory.edu>, 2004/09/14
%
\end{description}
\methodline%
\subsection{Class \texttt{voltage\_clamp}}%
\index[funcref]{voltage_clamp@\fidxl{voltage\_clamp}|boldhyperpage}%
\label{ref_voltage_clamp}%
\hypertarget{ref_voltage_clamp}{}%
\subsubsection[Constructor \texttt{voltage\_clamp}]{Constructor \texttt{voltage\_clamp/voltage\_clamp}}%
\index[funcref]{voltage_clamp@\fidxl{voltage\_clamp}!voltage_clamp@\fidxl{voltage\_clamp}}%
\label{ref_voltage_clamp__voltage_clamp}%
\hypertarget{ref_voltage_clamp__voltage_clamp}{}%
\begin{description}
\item[Summary:]Voltage clamp object with current and voltage traces.
%
\item[Usage:]~%
\begin{lyxcode}%
a\_vc = voltage\_clamp(data\_i, data\_v, dt, di, dv, id, props)
%
\end{lyxcode}%
%
\item[Description:]%
Uses the generic trace object to store voltage clamp I, V data.
 Inherits the common methods defined in trace.
%%
\item[Parameters:]~
\begin{description}%
\item[\texttt{data\_i,data\_v}:]
 Column vectors of I and V data traces.
\item[\texttt{dt}:]
 Time resolution [s].
\item[\texttt{di,dv}:]
 y-axis resolution for I and V [A and V, resp]
\item[\texttt{id}:]
 Identification string.
\item[\texttt{props}:]
 A structure with any optional properties, such as:
\begin{description}%
\item[\texttt{trace\_time\_start}:]
 Samples in the beginning to discard [dt]

(see trace for more)
\end{description}%
\end{description}%
%
\item[Returns a structure object with the following fields:
]~

   v: Voltage trace object, 
   i: Current trace object,
   time\_steps: Times of voltage steps.
   v\_steps: Mean voltage values before each step (including one after
            last step)
   i\_steps: Mean current values of steady-state before each step.
   trace: A parent trace object to inherit methods from.
%
%
\item[See also:]%
\hyperlink{ref_trace}{\texttt{trace}}%
\ (p.~\pageref{ref_trace})%
\index[funcref]{trace@\fidxl{trace}}%
, \hyperlink{ref_period}{\texttt{period}}%
\ (p.~\pageref{ref_period})%
\index[funcref]{period@\fidxl{period}}%
%
\item[Author:]%
Cengiz Gunay <cgunay@emory.edu>, 2010/02/05
%
\end{description}
\methodline%
\subsubsection[Method \texttt{calcCurPeaks}]{Method \texttt{voltage\_clamp/calcCurPeaks}}%
\index[funcref]{voltage_clamp@\fidxl{voltage\_clamp}!calcCurPeaks@\fidxl{calcCurPeaks}}%
\label{ref_voltage_clamp__calcCurPeaks}%
\hypertarget{ref_voltage_clamp__calcCurPeaks}{}%
\begin{description}
\item[Summary:]Find current peaks during a voltage step.
%
\item[Usage:]~%
\begin{lyxcode}%
a\_vc = calcCurPeaks(a\_vc, step\_num, props)
%
\end{lyxcode}%
%
%
\item[Parameters:]~
\begin{description}%
\item[\texttt{a\_vc}:]
 A voltage clamp object.
\item[\texttt{step\_num}:]
 1 for prestep, 2 for the first step, 3 for next, etc (default=2).
\item[\texttt{props}:]
 A structure with any optional properties.
\begin{description}%
\item[\texttt{pulseRange}:]
 Use this range for finding peaks [dt].
\item[\texttt{pulseStartRel}:]
 Time to start relative to the step beginning (default=+.3)

[ms]. If it has two elements, first one  specifies the voltage step .
\item[\texttt{pulseEndRel}:]
 Time to end relative to the step end (default=-.3) [ms]. Like

pulseStartRel, allows specifying voltage step.
\item[\texttt{avgAroundMs}:]
 If given, after finding a peak, average +/- ms around

it to reduce noise.
\end{description}%
\end{description}%
%
\item[Returns:
]~

   a\_vc: Updated voltage\_clamp object that contains props.iPeaks.
%
\item[Example:]~
\begin{lyxcode} >> a\_vc = calcCurPeaks(a\_vc, 2)
\\%
\end{lyxcode}
%
\item[See also:]%
\hyperlink{ref_voltage_clamp}{\texttt{voltage\_clamp}}%
\ (p.~\pageref{ref_voltage_clamp})%
\index[funcref]{voltage_clamp@\fidxl{voltage\_clamp}}%
, \hyperlink{ref_findSteps}{\texttt{findSteps}}%
\ (p.~\pageref{ref_findSteps})%
\index[funcref]{findSteps@\fidxl{findSteps}}%
%
\item[Author:]%
Cengiz Gunay <cgunay@emory.edu>, 2010/03/30
%
\end{description}
\methodline%
\subsubsection[Method \texttt{get}]{Method \texttt{voltage\_clamp/get}}%
\index[funcref]{voltage_clamp@\fidxl{voltage\_clamp}!get@\fidxl{get}}%
\label{ref_voltage_clamp__get}%
\hypertarget{ref_voltage_clamp__get}{}%
\begin{description}
\item[Summary:]Defines generic attribute retrieval for objects.
%
%
%
%
%
%
%
\item[Author:]%
Cengiz Gunay <cgunay@emory.edu>, 2004/09/14
%
\end{description}
\methodline%
\subsubsection[Method \texttt{getResults}]{Method \texttt{voltage\_clamp/getResults}}%
\index[funcref]{voltage_clamp@\fidxl{voltage\_clamp}!getResults@\fidxl{getResults}}%
\label{ref_voltage_clamp__getResults}%
\hypertarget{ref_voltage_clamp__getResults}{}%
\begin{description}
\item[Summary:]Calculate measurements and observations from this object.
%
\item[Usage:]~%
\begin{lyxcode}%
a\_prof = getResults(a\_vc, props)
%
\end{lyxcode}%
%
%
\item[Parameters:]~
\begin{description}%
\item[\texttt{a\_vc}:]
 A voltage\_clamp object.
\item[\texttt{props}:]
 A structure with any optional properties (inherited from a\_vc).
\begin{description}%
\item[\texttt{skipStep}:]
 Number of voltage step times to skip at the start (default=0).
\end{description}%
\end{description}%
%
\item[Returns:
]~

   a\_prof: A params\_results\_profile object with parameters and results structures.
%
\item[Example:]~
\begin{lyxcode} >> a\_cs = params\_tests\_dataset({md1, md2})
\\%
 >> a\_db = params\_tests\_db(a\_cs) % calls loadItemProfile, which calls getResults
\\%
\end{lyxcode}
%
\item[See also:]%
\hyperlink{ref_params_tests_dataset__loadItemProfile}{\texttt{params\_tests\_dataset/loadItemProfile}}%
\ (p.~\pageref{ref_params_tests_dataset__loadItemProfile})%
\index[funcref]{params_tests_dataset@\fidxl{params\_tests\_dataset}!loadItemProfile@\fidxl{loadItemProfile}}%
, \hyperlink{ref_params_tests_dataset}{\texttt{params\_tests\_dataset}}%
\ (p.~\pageref{ref_params_tests_dataset})%
\index[funcref]{params_tests_dataset@\fidxl{params\_tests\_dataset}}%
%
\item[Author:]%
Cengiz Gunay <cgunay@emory.edu>, 2011/07/08
%
\end{description}
\methodline%
\subsubsection[Method \texttt{getTimeRelStep}]{Method \texttt{voltage\_clamp/getTimeRelStep}}%
\index[funcref]{voltage_clamp@\fidxl{voltage\_clamp}!getTimeRelStep@\fidxl{getTimeRelStep}}%
\label{ref_voltage_clamp__getTimeRelStep}%
\hypertarget{ref_voltage_clamp__getTimeRelStep}{}%
\begin{description}
\item[Summary:]Return time relative to a voltage step.
%
\item[Usage:]~%
\begin{lyxcode}%
rel\_time = getTimeRelStep(a\_vc, step\_num, rel\_time, props)
%
\end{lyxcode}%
%
%
\item[Parameters:]~
\begin{description}%
\item[\texttt{a\_vc}:]
 A voltage\_clamp object.
\item[\texttt{step\_num}:]
 Relative to this voltage step. 1 for start of trace,

2 for first voltage change, and so on.
\item[\texttt{rel\_time}:]
 One or more time values in a vector [ms].
\item[\texttt{props}:]
 A structure with any optional properties.
\end{description}%
%
\item[Returns:
]~

   rel\_time: Time vector from start of trace [dt].
%
\item[Example:]~
\begin{lyxcode} Select [-10, 50] ms range from step 1 into a new VC object.
\\%
 >> new\_vc = withinPeriod(a\_vc, period(getTimeRelStep(a\_vc, 1, [-10 50])))
\\%
\end{lyxcode}
%
\item[See also:]%
\hyperlink{ref_voltage_clamp}{\texttt{voltage\_clamp}}%
\ (p.~\pageref{ref_voltage_clamp})%
\index[funcref]{voltage_clamp@\fidxl{voltage\_clamp}}%
%
\item[Author:]%
Cengiz Gunay <cgunay@emory.edu>, 2010/10/18
%
\end{description}
\methodline%
\subsubsection[Method \texttt{minus}]{Method \texttt{voltage\_clamp/minus}}%
\index[funcref]{voltage_clamp@\fidxl{voltage\_clamp}!minus@\fidxl{minus}}%
\label{ref_voltage_clamp__minus}%
\hypertarget{ref_voltage_clamp__minus}{}%
\begin{description}
\item[Summary:]Subtract current traces of voltage clamp object right\_vc from left\_vc.
%
\item[Usage:]~%
\begin{lyxcode}%
sub\_vc = minus(left\_vc, right\_vc, props)
%
\end{lyxcode}%
%
\item[Description:]%
Also returns the subtracted voltage trace in props.sub\_v for visual
 inspection of match between the two voltage traces.
%%
\item[Parameters:]~
\begin{description}%
\item[\texttt{left\_vc, right\_vc}:]
 voltage\_clamp objects.
\item[\texttt{props}:]
 A structure with any optional properties.
\end{description}%
%
\item[Returns:
]~

   sub\_vc: voltage\_clamp object with subtracted current and 
     voltage of left\_vc.
%
\item[Example:]~
\begin{lyxcode} >> sub\_vc = minus(vc1, vc2)
\\%
 OR
\\%
 >> sub\_vc = vc1 - vc2;
\\%
 plot the subtracted voltage clamp
\\%
 >> plot(sub\_vc)
\\%
 plot the subtracted voltage trace, too
\\%
 >> plot(sub\_vc.props.sub\_v)
\\%
\end{lyxcode}
%
\item[See also:]%
\hyperlink{ref_voltage_clamp}{\texttt{voltage\_clamp}}%
\ (p.~\pageref{ref_voltage_clamp})%
\index[funcref]{voltage_clamp@\fidxl{voltage\_clamp}}%
%
\item[Author:]%
Cengiz Gunay <cgunay@emory.edu>, 2010/03/10
%
\end{description}
\methodline%
\subsubsection[Method \texttt{periodWhole}]{Method \texttt{voltage\_clamp/periodWhole}}%
\index[funcref]{voltage_clamp@\fidxl{voltage\_clamp}!periodWhole@\fidxl{periodWhole}}%
\label{ref_voltage_clamp__periodWhole}%
\hypertarget{ref_voltage_clamp__periodWhole}{}%
\begin{description}
\item[Summary:]Returns the boundaries of the whole period. 
%
\item[Usage:]~%
\begin{lyxcode}%
whole\_period = periodWhole(a\_vc)
%
\end{lyxcode}%
%
%
\item[Parameters:]~
\begin{description}%
\item[\texttt{a\_vc}:]
 A voltage\_clamp object.
\end{description}%
%
%
%
\item[See also:]%
\hyperlink{ref_period}{\texttt{period}}%
\ (p.~\pageref{ref_period})%
\index[funcref]{period@\fidxl{period}}%
, \hyperlink{ref_trace}{\texttt{trace}}%
\ (p.~\pageref{ref_trace})%
\index[funcref]{trace@\fidxl{trace}}%
, \hyperlink{ref_voltage_clamp}{\texttt{voltage\_clamp}}%
\ (p.~\pageref{ref_voltage_clamp})%
\index[funcref]{voltage_clamp@\fidxl{voltage\_clamp}}%
%
\item[Author:]%
Cengiz Gunay <cgunay@emory.edu>, 2010/04/08
%
\end{description}
\methodline%
\subsubsection[Method \texttt{plotAllIVs}]{Method \texttt{voltage\_clamp/plotAllIVs}}%
\index[funcref]{voltage_clamp@\fidxl{voltage\_clamp}!plotAllIVs@\fidxl{plotAllIVs}}%
\label{ref_voltage_clamp__plotAllIVs}%
\hypertarget{ref_voltage_clamp__plotAllIVs}{}%
\begin{description}
\item[Summary:]Plot superposed I/V curves for activation, inactivation and steady-state.
%
\item[Usage:]~%
\begin{lyxcode}%
a\_p = plotAllIVs(a\_vc, title\_str, props)
%
\end{lyxcode}%
%
%
\item[Parameters:]~
\begin{description}%
\item[\texttt{a\_vc}:]
 A voltage\_clamp object.
\item[\texttt{title\_str}:]
 (Optional) Text to appear in the plot title.
\item[\texttt{props}:]
 A structure with any optional properties.
\begin{description}%
\item[\texttt{quiet}:]
 If 1, only use given title\_str.
\item[\texttt{skipStep}:]
 Number of voltage steps to skip at the start (default=0).
\end{description}%
\end{description}%
%
\item[Returns:
]~

   a\_p: A plot\_abstract object.
%
\item[Example:]~
\begin{lyxcode} >> plotFigure(plotAllIVs(data\_vc, 'I/V curves'))
\\%
\end{lyxcode}
%
\item[See also:]%
\hyperlink{ref_model_data_vcs}{\texttt{model\_data\_vcs}}%
\ (p.~\pageref{ref_model_data_vcs})%
\index[funcref]{model_data_vcs@\fidxl{model\_data\_vcs}}%
, \hyperlink{ref_voltage_clamp}{\texttt{voltage\_clamp}}%
\ (p.~\pageref{ref_voltage_clamp})%
\index[funcref]{voltage_clamp@\fidxl{voltage\_clamp}}%
, \hyperlink{ref_plot_abstract}{\texttt{plot\_abstract}}%
\ (p.~\pageref{ref_plot_abstract})%
\index[funcref]{plot_abstract@\fidxl{plot\_abstract}}%
, \hyperlink{ref_plotFigure}{\texttt{plotFigure}}%
\ (p.~\pageref{ref_plotFigure})%
\index[funcref]{plotFigure@\fidxl{plotFigure}}%
%
\item[Author:]%
Cengiz Gunay <cgunay@emory.edu>, 2011/07/16
%
\end{description}
\methodline%
\subsubsection[Method \texttt{plotSimCurrent}]{Method \texttt{voltage\_clamp/plotSimCurrent}}%
\index[funcref]{voltage_clamp@\fidxl{voltage\_clamp}!plotSimCurrent@\fidxl{plotSimCurrent}}%
\label{ref_voltage_clamp__plotSimCurrent}%
\hypertarget{ref_voltage_clamp__plotSimCurrent}{}%
\begin{description}
\item[Summary:]Simulate voltage clamp current on a model channel and superpose on data.
%
\item[Usage:]~%
\begin{lyxcode}%
a\_p = plotSimCurrent(a\_vc, f\_I\_v, props)
%
\end{lyxcode}%
%
%
\item[Parameters:]~
\begin{description}%
\item[\texttt{a\_vc}:]
 A voltage\_clamp object.
\item[\texttt{f\_I\_v}:]
 param\_func object representing the model channel. 
\item[\texttt{props}:]
 A structure with any optional properties.
\begin{description}%
\item[\texttt{delay}:]
 If given, use as voltage clamp delay [ms].
\item[\texttt{levels}:]
 Only plot these voltage level indices.
\end{description}%
\end{description}%
%
\item[Returns:
]~

   a\_p: A plot\_abstract object.
%
\item[Example:]~
\begin{lyxcode} >> plotFigure(plotSimCurrent(a\_vc))
\\%
\end{lyxcode}
%
\item[See also:]%
\hyperlink{ref_param_I_v}{\texttt{param\_I\_v}}%
\ (p.~\pageref{ref_param_I_v})%
\index[funcref]{param_I_v@\fidxl{param\_I\_v}}%
, \hyperlink{ref_param_func}{\texttt{param\_func}}%
\ (p.~\pageref{ref_param_func})%
\index[funcref]{param_func@\fidxl{param\_func}}%
, \hyperlink{ref_plot_abstract}{\texttt{plot\_abstract}}%
\ (p.~\pageref{ref_plot_abstract})%
\index[funcref]{plot_abstract@\fidxl{plot\_abstract}}%
%
\item[Author:]%
Cengiz Gunay <cgunay@emory.edu>, 2010/03/29
%
\end{description}
\methodline%
\subsubsection[Method \texttt{plotSteadyIV}]{Method \texttt{voltage\_clamp/plotSteadyIV}}%
\index[funcref]{voltage_clamp@\fidxl{voltage\_clamp}!plotSteadyIV@\fidxl{plotSteadyIV}}%
\label{ref_voltage_clamp__plotSteadyIV}%
\hypertarget{ref_voltage_clamp__plotSteadyIV}{}%
\begin{description}
\item[Summary:]Plot of the I/V curve at the end of a voltage step.
%
\item[Usage:]~%
\begin{lyxcode}%
a\_p = plotSteadyIV(a\_vc, step\_num, title\_str, props)
%
\end{lyxcode}%
%
\item[Description:]%
Can be superposed with other I/V plot objects (see plot\_superpose).
%%
\item[Parameters:]~
\begin{description}%
\item[\texttt{a\_vc}:]
 A voltage clamp object.
\item[\texttt{step\_num}:]
 1 for prestep, 2 for the first step, 3 for next, etc.
\item[\texttt{title\_str}:]
 (Optional) Text to appear in the plot title.
\item[\texttt{props}:]
 A structure with any optional properties.
\begin{description}%
\item[\texttt{quiet}:]
 If 1, only use given title\_str.
\item[\texttt{curUnit}:]
 Display units for current trace (default='nA').
\item[\texttt{label}:]
 add this as a line label to be used in superposed plots.
\item[\texttt{plotPeaks}:]
 If 1, use the props.iPeaks instead of steady-state.
\item[\texttt{stepRange}:]
 Uses the relative [start end] times in [ms] around 

time of step\_num to calculate the current averages. If vector has
3 items, first one is the step number, which can be different
than step\_num.
\end{description}%
\end{description}%
%
\item[Returns:
]~

   a\_p: A plot\_abstract object.
%
\item[Example:]~
\begin{lyxcode} >> a\_vc = abf\_voltage\_clamp('data-dir/cell-A.abf')
\\%
 >> plotFigure(plotSteadyIV(a\_vc, 2, 'I/V curve'))
\\%
\end{lyxcode}
%
\item[See also:]%
\hyperlink{ref_voltage_clamp}{\texttt{voltage\_clamp}}%
\ (p.~\pageref{ref_voltage_clamp})%
\index[funcref]{voltage_clamp@\fidxl{voltage\_clamp}}%
, \hyperlink{ref_plot_abstract}{\texttt{plot\_abstract}}%
\ (p.~\pageref{ref_plot_abstract})%
\index[funcref]{plot_abstract@\fidxl{plot\_abstract}}%
, \hyperlink{ref_plotFigure}{\texttt{plotFigure}}%
\ (p.~\pageref{ref_plotFigure})%
\index[funcref]{plotFigure@\fidxl{plotFigure}}%
, \hyperlink{ref_plot_superpose}{\texttt{plot\_superpose}}%
\ (p.~\pageref{ref_plot_superpose})%
\index[funcref]{plot_superpose@\fidxl{plot\_superpose}}%
%
\item[Author:]%
Cengiz Gunay <cgunay@emory.edu>, 2010/03/10
%
\end{description}
\methodline%
\subsubsection[Method \texttt{plot\_abstract}]{Method \texttt{voltage\_clamp/plot\_abstract}}%
\index[funcref]{voltage_clamp@\fidxl{voltage\_clamp}!plot_abstract@\fidxl{plot\_abstract}}%
\label{ref_voltage_clamp__plot_abstract}%
\hypertarget{ref_voltage_clamp__plot_abstract}{}%
\begin{description}
\item[Summary:]Plot of the I and V traces of voltage clamp object.
%
\item[Usage:]~%
\begin{lyxcode}%
a\_p = plot\_abstract(a\_vc, title\_str, props)
%
\end{lyxcode}%
%
\item[Description:]%
Can be stacked or superposed with other plot objects.
%%
\item[Parameters:]~
\begin{description}%
\item[\texttt{a\_vc}:]
 A voltage clamp object.
\item[\texttt{title\_str}:]
 (Optional) Text to appear in the plot title.
\item[\texttt{props}:]
 A structure with any optional properties.
\begin{description}%
\item[\texttt{quiet}:]
 If 1, only use given title\_str.
\item[\texttt{vStep}:]
 Index of step with varying voltages (default=2).
\item[\texttt{label}:]
 add this as a line label to be used in superposed plots.
\item[\texttt{onlyPlot}:]
 'i' for current and 'v' for voltage plot.
\item[\texttt{curUnit}:]
 Display units for current trace (default='nA').
\item[\texttt{vColors}:]
 If 1 (default), always use same colors for same voltage levels.
\item[\texttt{vColorsFunc}:]
 Function to get voltage colors (default=@lines)

(rest passed to plot\_stack and plot\_abstract)
\end{description}%
\end{description}%
%
\item[Returns:
]~

   a\_p: A plot\_abstract object.
%
\item[Example:]~
\begin{lyxcode} >> a\_vc = abf2voltage\_clamp('data-dir/cell-A.abf')
\\%
 >> plotFigure(plot\_abstract(a\_vc, 'I/V curve'))
\\%
\end{lyxcode}
%
\item[See also:]%
\hyperlink{ref_plotSteadyIV}{\texttt{plotSteadyIV}}%
\ (p.~\pageref{ref_plotSteadyIV})%
\index[funcref]{plotSteadyIV@\fidxl{plotSteadyIV}}%
, \hyperlink{ref_plot_abstract}{\texttt{plot\_abstract}}%
\ (p.~\pageref{ref_plot_abstract})%
\index[funcref]{plot_abstract@\fidxl{plot\_abstract}}%
, \hyperlink{ref_plotFigure}{\texttt{plotFigure}}%
\ (p.~\pageref{ref_plotFigure})%
\index[funcref]{plotFigure@\fidxl{plotFigure}}%
, \hyperlink{ref_plot_superpose}{\texttt{plot\_superpose}}%
\ (p.~\pageref{ref_plot_superpose})%
\index[funcref]{plot_superpose@\fidxl{plot\_superpose}}%
, \hyperlink{ref_plot_stack}{\texttt{plot\_stack}}%
\ (p.~\pageref{ref_plot_stack})%
\index[funcref]{plot_stack@\fidxl{plot\_stack}}%
%
\item[Author:]%
Cengiz Gunay <cgunay@emory.edu>, 2010/03/11
%
\end{description}
\methodline%
\subsubsection[Method \texttt{saveDataTxt}]{Method \texttt{voltage\_clamp/saveDataTxt}}%
\index[funcref]{voltage_clamp@\fidxl{voltage\_clamp}!saveDataTxt@\fidxl{saveDataTxt}}%
\label{ref_voltage_clamp__saveDataTxt}%
\hypertarget{ref_voltage_clamp__saveDataTxt}{}%
\begin{description}
\item[Summary:]Save time and current into a simple text file as columns.
%
\item[Usage:]~%
\begin{lyxcode}%
saveDataTxt(a\_vc, props)
%
\end{lyxcode}%
%
\item[Description:]%
File will be written to the same directory as the original vc was
 loaded from (using the value of props.filename).
%%
\item[Parameters:]~
\begin{description}%
\item[\texttt{a\_vc}:]
 A voltage\_clamp object.
\item[\texttt{props}:]
 A structure with any optional properties.
\begin{description}%
\item[\texttt{addName}:]
 String to append to file name.
\item[\texttt{saveV}:]
 Save voltage as well.
\end{description}%
\end{description}%
%
\item[Returns:
]~

%
\item[Example:]~
\begin{lyxcode} >> saveDataTxt(a\_vc)
\\%
\end{lyxcode}
%
\item[See also:]%
\hyperlink{ref_voltage_clamp}{\texttt{voltage\_clamp}}%
\ (p.~\pageref{ref_voltage_clamp})%
\index[funcref]{voltage_clamp@\fidxl{voltage\_clamp}}%
%
\item[Author:]%
Cengiz Gunay <cgunay@emory.edu>, 2010/03/29
%
\end{description}
\methodline%
\subsubsection[Method \texttt{scale\_IClCa\_NaP\_sub\_IClCa}]{Method \texttt{voltage\_clamp/scale\_IClCa\_NaP\_sub\_IClCa}}%
\index[funcref]{voltage_clamp@\fidxl{voltage\_clamp}!scale_IClCa_NaP_sub_IClCa@\fidxl{scale\_IClCa\_NaP\_sub\_IClCa}}%
\label{ref_voltage_clamp__scale_IClCa_NaP_sub_IClCa}%
\hypertarget{ref_voltage_clamp__scale_IClCa_NaP_sub_IClCa}{}%
\begin{description}
\item[Summary:]Scale IClCa and steady-state of INaP from voltage-step protocol data and subtract IClCa.
%
\item[Usage:]~%
\begin{lyxcode}%
params = scale\_IClCa\_NaP\_sub\_IClCa(a\_vc, props)
%
\end{lyxcode}%
%
\item[Description:]%
Made for Na recordings from the oocyte. While estimating IClCa, one
 needs to consider INaP since it counteracts IClCa.
%%
\item[Parameters:]~
\begin{description}%
\item[\texttt{a\_vc}:]
 A voltage clamp object.
\item[\texttt{props}:]
 A structure with any optional properties.
\begin{description}%
\item[\texttt{saveData}:]
 If 1, save subtracted data into a new text file (default=0).
\item[\texttt{plotPrepulse}:]
 If 1, show plot of prepulse current change with

voltage (default=0).
\end{description}%
\end{description}%
%
\item[Returns:
]~

   params: Structure with tuned parameters.
%
\item[Example:]~
\begin{lyxcode} >> a\_vc = abf\_voltage\_clamp('data-dir/cell-A.abf')
\\%
 >> params = scale\_IClCa\_NaP\_sub\_IClCa(a\_vc)
\\%
\end{lyxcode}
%
\item[See also:]%
\hyperlink{ref_voltage_clamp}{\texttt{voltage\_clamp}}%
\ (p.~\pageref{ref_voltage_clamp})%
\index[funcref]{voltage_clamp@\fidxl{voltage\_clamp}}%
, \hyperlink{ref_param_I_v}{\texttt{param\_I\_v}}%
\ (p.~\pageref{ref_param_I_v})%
\index[funcref]{param_I_v@\fidxl{param\_I\_v}}%
, \hyperlink{ref_param_func}{\texttt{param\_func}}%
\ (p.~\pageref{ref_param_func})%
\index[funcref]{param_func@\fidxl{param\_func}}%
%
\item[Author:]%
Cengiz Gunay <cgunay@emory.edu>, 2010/02/05
%
\end{description}
\methodline%
\subsubsection[Method \texttt{scale\_sub\_cap\_leak}]{Method \texttt{voltage\_clamp/scale\_sub\_cap\_leak}}%
\index[funcref]{voltage_clamp@\fidxl{voltage\_clamp}!scale_sub_cap_leak@\fidxl{scale\_sub\_cap\_leak}}%
\label{ref_voltage_clamp__scale_sub_cap_leak}%
\hypertarget{ref_voltage_clamp__scale_sub_cap_leak}{}%
\begin{description}
\item[Summary:]Scale capacitance and leak artifacts to subtract them.
%
\item[Usage:]~%
\begin{lyxcode}%
[f\_capleak sub\_vc] = scale\_sub\_cap\_leak(a\_vc, props)
%
\end{lyxcode}%
%
%
\item[Parameters:]~
\begin{description}%
\item[\texttt{a\_vc}:]
 Full path to a\_vc.
\item[\texttt{props}:]
 A structure with any optional properties.
\begin{description}%
\item[\texttt{capLeakModel}:]
 Model object to fit (default obtained from

param\_cap\_leak\_int\_t). Can choose object obtained from another
\item[\texttt{function such as}:]
 param\_Rs\_cap\_leak\_int\_t, param\_cap\_leak\_2comp\_int\_t.
\item[\texttt{fitRange}:]
 Start and end times of range to apply the optimization [ms].
\item[\texttt{fitRangeRel}:]
 Start and end times of range relative to first voltage

step [ms]. Specify any other voltage step as the first element.
\item[\texttt{fitLevels}:]
 Indices of voltage/current levels to use from clamp

data. If empty, not fit is done.
\item[\texttt{dispParams}:]
 If non-zero, display params every once this many iterations.
\item[\texttt{dispPlot}:]
 If non-zero, update a plot of the fit at end of this many iterations.
\item[\texttt{saveData}:]
 If 1, save subtracted data into a new text file (default=0).
\item[\texttt{quiet}:]
 If 1, do not include cell name on title.
\item[\texttt{period}:]
 Limit the subtraction to this period of a\_vc.
\end{description}%
\end{description}%
%
\item[Returns:
]~

   f\_capleak: Updated function with fitted parameters
   sub\_vc: voltage\_clamp object with passive-subtracted I trace.
%
\item[Example:]~
\begin{lyxcode} % set up a function with passive electrode and membrane parameters
\\%
 >> capleakReCe\_f = ...
\\%
    param\_Re\_Ce\_cap\_leak\_int\_t(...
\\%
      struct('Re', 47, 'Ce', 12, 'gL', .56, ...
\\%
             'EL', -67, 'Cm', 270, 'delay', 0.21), ...
\\%
                         ['cap, leak, Re and Ce']);
\\%
 % load ABF file and use above function to fit selected voltage steps
\\%
 >> [capleakReCe\_f sub\_cap\_leak\_vc ] = ...
\\%
    scale\_sub\_cap\_leak(...
\\%
      abf2voltage\_clamp('calcium.abf'), '', ...
\\%
      struct('capLeakModel', capleakReCe\_f, ...
\\%
             'fitRangeRel', [-.2 165], 'fitLevels', 1:5, ...
\\%
             'optimset', struct('Display', 'iter')));
\\%
\end{lyxcode}
%
\item[See also:]%
\hyperlink{ref_param_I_v}{\texttt{param\_I\_v}}%
\ (p.~\pageref{ref_param_I_v})%
\index[funcref]{param_I_v@\fidxl{param\_I\_v}}%
, \hyperlink{ref_param_func}{\texttt{param\_func}}%
\ (p.~\pageref{ref_param_func})%
\index[funcref]{param_func@\fidxl{param\_func}}%
%
\item[Author:]%
Cengiz Gunay <cgunay@emory.edu>, 2010/01/17
%
\end{description}
\methodline%
\subsubsection[Method \texttt{set}]{Method \texttt{voltage\_clamp/set}}%
\index[funcref]{voltage_clamp@\fidxl{voltage\_clamp}!set@\fidxl{set}}%
\label{ref_voltage_clamp__set}%
\hypertarget{ref_voltage_clamp__set}{}%
\begin{description}
\item[Summary:]Generic method for setting object attributes.
%
%
%
%
%
%
%
\item[Author:]%
Cengiz Gunay <cgunay@emory.edu>, 2004/10/08
%
\end{description}
\methodline%
\subsubsection[Method \texttt{setLevels}]{Method \texttt{voltage\_clamp/setLevels}}%
\index[funcref]{voltage_clamp@\fidxl{voltage\_clamp}!setLevels@\fidxl{setLevels}}%
\label{ref_voltage_clamp__setLevels}%
\hypertarget{ref_voltage_clamp__setLevels}{}%
\begin{description}
\item[Summary:]Choose which voltage and current step levels to keep.
%
\item[Usage:]~%
\begin{lyxcode}%
a\_vc = setLevels(a\_vc, levels, props)
%
\end{lyxcode}%
%
%
\item[Parameters:]~
\begin{description}%
\item[\texttt{a\_vc}:]
 A voltage\_clamp object.
\item[\texttt{levels}:]
 Only keep these voltage and current level indices.
\item[\texttt{props}:]
 A structure with any optional properties.
\end{description}%
%
\item[Returns:
]~

   a\_vc: A voltage\_clamp object that contains only the selected levels.
%
\item[Example:]~
\begin{lyxcode} >> a\_vc = setLevels(a\_vc, 1:3) % only select the first few levels
\\%
\end{lyxcode}
%
\item[See also:]%
\hyperlink{ref_voltage_clamp}{\texttt{voltage\_clamp}}%
\ (p.~\pageref{ref_voltage_clamp})%
\index[funcref]{voltage_clamp@\fidxl{voltage\_clamp}}%
%
\item[Author:]%
Cengiz Gunay <cgunay@emory.edu>, 2010/03/30
%
\end{description}
\methodline%
\subsubsection[Method \texttt{simModel}]{Method \texttt{voltage\_clamp/simModel}}%
\index[funcref]{voltage_clamp@\fidxl{voltage\_clamp}!simModel@\fidxl{simModel}}%
\label{ref_voltage_clamp__simModel}%
\hypertarget{ref_voltage_clamp__simModel}{}%
\begin{description}
\item[Summary:]Simulate model channel current using voltage clamp.
%
\item[Usage:]~%
\begin{lyxcode}%
model\_vc = simModel(a\_vc, f\_I\_v, props)
%
\end{lyxcode}%
%
\item[Description:]%
Often the delay is already included in the model, which is better
 because sub-dt precision can be achieved using interpolation.
%%
\item[Parameters:]~
\begin{description}%
\item[\texttt{a\_vc}:]
 A voltage\_clamp object.
\item[\texttt{f\_I\_v}:]
 param\_func object representing the model channel. 
\item[\texttt{props}:]
 A structure with any optional properties.
\begin{description}%
\item[\texttt{delay}:]
 If given, use as voltage clamp delay [ms].
\item[\texttt{levels}:]
 Only simulate these voltage level indices.
\item[\texttt{period}:]
 Limit the simulation to this period of a\_vc.
\item[\texttt{updateVm}:]
 If 1, update v trace from simulation Vm. 
\end{description}%
\end{description}%
%
\item[Returns:
]~

   model\_vc: A voltage\_clamp object with simulated current data and
   	      the original voltage data.
%
\item[Example:]~
\begin{lyxcode} >> I\_Ca = param\_I\_v([1 1 .0077 58], m\_Ca, h\_Ca, 'I\_{Ca}', ...
\\%
              struct('paramRanges', ...
\\%
                     [1 4; 0 1; 0 1e3; 100 200]'))
\\%
 >> model\_vc = simModel(a\_vc, I\_Ca)
\\%
\end{lyxcode}
%
\item[See also:]%
\hyperlink{ref_param_I_v}{\texttt{param\_I\_v}}%
\ (p.~\pageref{ref_param_I_v})%
\index[funcref]{param_I_v@\fidxl{param\_I\_v}}%
, \hyperlink{ref_param_func}{\texttt{param\_func}}%
\ (p.~\pageref{ref_param_func})%
\index[funcref]{param_func@\fidxl{param\_func}}%
, \hyperlink{ref_plot_abstract}{\texttt{plot\_abstract}}%
\ (p.~\pageref{ref_plot_abstract})%
\index[funcref]{plot_abstract@\fidxl{plot\_abstract}}%
%
\item[Author:]%
Cengiz Gunay <cgunay@emory.edu>, 2010/03/29
%
\end{description}
\methodline%
\subsubsection[Method \texttt{subsref}]{Method \texttt{voltage\_clamp/subsref}}%
\index[funcref]{voltage_clamp@\fidxl{voltage\_clamp}!subsref@\fidxl{subsref}}%
\label{ref_voltage_clamp__subsref}%
\hypertarget{ref_voltage_clamp__subsref}{}%
\begin{description}
\item[Summary:]Defines generic indexing for objects.
%
%
%
%
%
%
%
\item[Author:]%
Cengiz Gunay <cgunay@emory.edu>, 2004/08/04
%
\end{description}
\methodline%
\subsubsection[Method \texttt{updateSteps}]{Method \texttt{voltage\_clamp/updateSteps}}%
\index[funcref]{voltage_clamp@\fidxl{voltage\_clamp}!updateSteps@\fidxl{updateSteps}}%
\label{ref_voltage_clamp__updateSteps}%
\hypertarget{ref_voltage_clamp__updateSteps}{}%
\begin{description}
\item[Summary:]Update voltage step time and magnitude info.
%
\item[Usage:]~%
\begin{lyxcode}%
a\_vc = updateSteps(a\_vc, props)
%
\end{lyxcode}%
%
\item[Description:]%
Called by simModel.
%%
\item[Parameters:]~
\begin{description}%
\item[\texttt{a\_vc}:]
 A voltage\_clamp object.
\item[\texttt{props}:]
 A structure with any optional properties.
\end{description}%
%
\item[Returns:
]~

   a\_vc: Updated object.
%
%
\item[See also:]%
\hyperlink{ref_voltage_clamp}{\texttt{voltage\_clamp}}%
\ (p.~\pageref{ref_voltage_clamp})%
\index[funcref]{voltage_clamp@\fidxl{voltage\_clamp}}%
%
\item[Author:]%
Cengiz Gunay <cgunay@emory.edu>, 2010/10/18
%
\end{description}
\methodline%
\subsubsection[Method \texttt{withinPeriod}]{Method \texttt{voltage\_clamp/withinPeriod}}%
\index[funcref]{voltage_clamp@\fidxl{voltage\_clamp}!withinPeriod@\fidxl{withinPeriod}}%
\label{ref_voltage_clamp__withinPeriod}%
\hypertarget{ref_voltage_clamp__withinPeriod}{}%
\begin{description}
\item[Summary:]Returns a voltage\_clamp object valid only within the given period.
%
\item[Usage:]~%
\begin{lyxcode}%
[a\_vc a\_period] = withinPeriod(a\_vc, a\_period, props)
%
\end{lyxcode}%
%
%
\item[Parameters:]~
\begin{description}%
\item[\texttt{a\_vc}:]
 A voltage\_clamp object.
\item[\texttt{a\_period}:]
 The desired period [dt].
\item[\texttt{props}:]
 A structure with any optional properties.
\begin{description}%
\item[\texttt{useAvailable}:]
 If 1, don't stop if period not available, use maximum available.
\end{description}%
\end{description}%
%
\item[Returns:
]~

   a\_vc: A voltage\_clamp object
   a\_period: The period object, updated if useAvailable is requested.
%
%
\item[See also:]%
\hyperlink{ref_trace__withinPeriod}{\texttt{trace/withinPeriod}}%
\ (p.~\pageref{ref_trace__withinPeriod})%
\index[funcref]{trace@\fidxl{trace}!withinPeriod@\fidxl{withinPeriod}}%
, \hyperlink{ref_trace}{\texttt{trace}}%
\ (p.~\pageref{ref_trace})%
\index[funcref]{trace@\fidxl{trace}}%
, \hyperlink{ref_period}{\texttt{period}}%
\ (p.~\pageref{ref_period})%
\index[funcref]{period@\fidxl{period}}%
%
\item[Author:]%
Cengiz Gunay <cgunay@emory.edu>, 2010/03/29
%
\end{description}
\methodline%
\subsection{Utility functions}%
\label{ref_utils}%
\hypertarget{ref_utils}{}%
\subsubsection[Function \texttt{abf2load}]{Function \texttt{functions/abf2load}}%
\index[funcref]{functions@\fidxl{functions}!abf2load@\fidxl{abf2load}}%
\label{ref_functions__abf2load}%
\hypertarget{ref_functions__abf2load}{}%
\begin{description}
%
%
%
%
%
%
%
%
\end{description}
\methodline%
\subsubsection[Function \texttt{abf2voltage\_clamp}]{Function \texttt{functions/abf2voltage\_clamp}}%
\index[funcref]{functions@\fidxl{functions}!abf2voltage_clamp@\fidxl{abf2voltage\_clamp}}%
\label{ref_functions__abf2voltage_clamp}%
\hypertarget{ref_functions__abf2voltage_clamp}{}%
\begin{description}
\item[Summary:]Load I and V traces from an ABF file and make a voltage\_clamp object.
%
\item[Usage:]~%
\begin{lyxcode}%
a\_vc = abf2voltage\_clamp(filename, sup\_id, props)
%
\end{lyxcode}%
%
%
\item[Parameters:]~
\begin{description}%
\item[\texttt{filename}:]
 Full path to filename.
\item[\texttt{sup\_id}:]
 (Optional) Concatenated to cell filename as id of voltage\_clamp object.
\item[\texttt{props}:]
 A structure with any optional properties.
\begin{description}%
\item[\texttt{ichan}:]
 Current channel number or ':' for all channels when

there is more than one  to choose.
\item[\texttt{actualProtocols}:]
 Means current trace is a TTL pulse and its

magnitude is meaningless.
\end{description}%
\end{description}%
%
\item[Returns:
]~

   a\_vc: A voltage\_clamp object.
%
\item[Example:]~
\begin{lyxcode} >> a\_vc = abf2voltage\_clamp('data-dir/cell-A.abf', 'my first cell');
\\%
 >> plot(a\_vc);
\\%
\end{lyxcode}
%
\item[See also:]%
\hyperlink{ref_loadVclampAbf}{\texttt{loadVclampAbf}}%
\ (p.~\pageref{ref_loadVclampAbf})%
\index[funcref]{loadVclampAbf@\fidxl{loadVclampAbf}}%
, \hyperlink{ref_abf2load}{\texttt{abf2load}}%
\ (p.~\pageref{ref_abf2load})%
\index[funcref]{abf2load@\fidxl{abf2load}}%
, \hyperlink{ref_voltage_clamp}{\texttt{voltage\_clamp}}%
\ (p.~\pageref{ref_voltage_clamp})%
\index[funcref]{voltage_clamp@\fidxl{voltage\_clamp}}%
%
\item[Author:]%
Cengiz Gunay <cgunay@emory.edu>, 2010/03/11
%
\end{description}
\methodline%
\subsubsection[Function \texttt{abfload}]{Function \texttt{functions/abfload}}%
\index[funcref]{functions@\fidxl{functions}!abfload@\fidxl{abfload}}%
\label{ref_functions__abfload}%
\hypertarget{ref_functions__abfload}{}%
\begin{description}
%
%
%
%
%
%
%
%
\end{description}
\methodline%
\subsubsection[Function \texttt{array2str}]{Function \texttt{functions/array2str}}%
\index[funcref]{functions@\fidxl{functions}!array2str@\fidxl{array2str}}%
\label{ref_functions__array2str}%
\hypertarget{ref_functions__array2str}{}%
\begin{description}
%
%
%
%
%
\item[Example:]~
\begin{lyxcode}  string = array2str([1 2 3 4 6 8 10 12 13])
\\%
 the output is '[1:4 6:2:12 13]'
\\%
\end{lyxcode}
%
%
%
\end{description}
\methodline%
\subsubsection[Function \texttt{balanceInputProbs}]{Function \texttt{functions/balanceInputProbs}}%
\index[funcref]{functions@\fidxl{functions}!balanceInputProbs@\fidxl{balanceInputProbs}}%
\label{ref_functions__balanceInputProbs}%
\hypertarget{ref_functions__balanceInputProbs}{}%
\begin{description}
\item[Summary:]Balances samples according to prior class probabilities of the outputs.
%
\item[Usage:]~%
\begin{lyxcode}%
[new\_inputs, new\_outputs] = balanceInputProbs(a\_class\_inputs, a\_class\_outputs, balance\_ratio, props)
%
\end{lyxcode}%
%
\item[Description:]%
Uses the method in Lawrence, burns, Back, Tsoi and Lee Giles "Neural
 network classification and prior class probabilities" for
 probabilitic balancing of input and output samples when the number of
 samples in each class is vastly different and causes problems with
 classification without balancing.
%%
\item[Parameters:]~
\begin{description}%
\item[\texttt{a\_class\_inputs, a\_class\_outputs}:]
 Input and output vectors.
\item[\texttt{balance\_ratio}:]
 c\_s, between 0 and 1. If 1, equal samples from

each class if used. If 0, prior class probabilities are followed.
\item[\texttt{props}:]
 A structure with any optional properties.
\begin{description}%
\item[\texttt{repeatSamples}:]
 If 1, repeats the smaller class samples to match

with larger class. Otherwise, takes the minimal number of
samples to avoid repetitions (default=1).
\end{description}%
\end{description}%
%
\item[Returns:
]~

	new\_inputs, new\_outputs: New input and output vectors.
%
%
\item[See also:]%
\hyperlink{ref_approxMappingNNet}{\texttt{approxMappingNNet}}%
\ (p.~\pageref{ref_approxMappingNNet})%
\index[funcref]{approxMappingNNet@\fidxl{approxMappingNNet}}%
, \hyperlink{ref_tests_db}{\texttt{tests\_db}}%
\ (p.~\pageref{ref_tests_db})%
\index[funcref]{tests_db@\fidxl{tests\_db}}%
%
\item[Author:]%
Cengiz Gunay <cgunay@emory.edu>, 2008/01/09
%
\end{description}
\methodline%
\subsubsection[Function \texttt{boxplotp}]{Function \texttt{functions/boxplotp}}%
\index[funcref]{functions@\fidxl{functions}!boxplotp@\fidxl{boxplotp}}%
\label{ref_functions__boxplotp}%
\hypertarget{ref_functions__boxplotp}{}%
\begin{description}
%
%
\item[Description:]%
BOXPLOTP(X,NOTCH,SYM,VERT,WHIS,PROPS) produces a box and whisker plot for
   each column of X. The box has lines at the lower quartile, median, 
   and upper quartile values. The whiskers are lines extending from 
   each end of the box to show the extent of the rest of the data. 
   Outliers are data with values beyond the ends of the whiskers.
%%
%
%
%
%
%
\end{description}
\methodline%
\subsubsection[Function \texttt{boxutilp}]{Function \texttt{functions/boxutilp}}%
\index[funcref]{functions@\fidxl{functions}!boxutilp@\fidxl{boxutilp}}%
\label{ref_functions__boxutilp}%
\hypertarget{ref_functions__boxutilp}{}%
\begin{description}
%
%
\item[Description:]%
BOXUTILP(X) is a utility function for BOXPLOT, which calls
   BOXUTILP once for each column of its first argument. Use
   BOXPLOT rather than BOXUTILP. 
%%
%
%
%
%
%
\end{description}
\methodline%
\subsubsection[Function \texttt{calcGraphNormPtsRatio}]{Function \texttt{functions/calcGraphNormPtsRatio}}%
\index[funcref]{functions@\fidxl{functions}!calcGraphNormPtsRatio@\fidxl{calcGraphNormPtsRatio}}%
\label{ref_functions__calcGraphNormPtsRatio}%
\hypertarget{ref_functions__calcGraphNormPtsRatio}{}%
\begin{description}
\item[Summary:]Return the ratios of normalized to point units for dimensions of axis.
%
\item[Usage:]~%
\begin{lyxcode}%
[ratio\_x, ratio\_y] = calcGraphNormPtsRatio(grfx\_handle)
%
\end{lyxcode}%
%
\item[Description:]%
Used for findind character sizes given the size of an axis. Normally if the plot
 is resized, the characters may take up too much space or may not fit anymore unless
 the spacing is corrected.
%%
\item[Parameters:]~
\begin{description}%
\item[\texttt{grfx\_handle}:]
 A graphics handle to an existing axis or figure.
\end{description}%
%
\item[Returns:
]~

 	ratio\_x, ratio\_y: Normalized to points ratio for axis width and height, respectively.
%
\item[Example:]~
\begin{lyxcode} To find the normalized distance for a 10pt character:
\\%
 >> dx = 10 * calcGraphNormPtsRatio(my\_figure\_handle);
\\%
\end{lyxcode}
%
%
\item[Author:]%
Cengiz Gunay <cgunay@emory.edu>, 2006/03/05
%
\end{description}
\methodline%
\subsubsection[Function \texttt{cell2str}]{Function \texttt{functions/cell2str}}%
\index[funcref]{functions@\fidxl{functions}!cell2str@\fidxl{cell2str}}%
\label{ref_functions__cell2str}%
\hypertarget{ref_functions__cell2str}{}%
\begin{description}
\item[Summary:]Creates a tab-delimited string from the cell array's contents.
%
\item[Usage:]~%
\begin{lyxcode}%
a\_str = cell2str(a\_cell, props)
%
\end{lyxcode}%
%
%
\item[Parameters:]~
\begin{description}%
\item[\texttt{a\_cell}:]
 A cell matrix to be tabularized.
\item[\texttt{props}:]
 A structure with any optional properties.
\end{description}%
%
\item[Returns:
]~

   a\_str: LaTeX formatted table string.
%
%
\item[See also:]%
%
\item[Author:]%
Cengiz Gunay <cgunay@emory.edu>, 2004/12/09
%
\end{description}
\methodline%
\subsubsection[Function \texttt{cell2TeX}]{Function \texttt{functions/cell2TeX}}%
\index[funcref]{functions@\fidxl{functions}!cell2TeX@\fidxl{cell2TeX}}%
\label{ref_functions__cell2TeX}%
\hypertarget{ref_functions__cell2TeX}{}%
\begin{description}
\item[Summary:]Creates LaTeX string of a formatted table with the cell array's contents.
%
\item[Usage:]~%
\begin{lyxcode}%
tex\_string = cell2TeX(a\_cell, props)
%
\end{lyxcode}%
%
%
\item[Parameters:]~
\begin{description}%
\item[\texttt{a\_cell}:]
 A cell matrix to be tabularized.
\item[\texttt{props}:]
 A structure with any optional properties.
\begin{description}%
\item[\texttt{hasTitleRow}:]
 The first row contains titles.
\item[\texttt{titleColWidth}:]
 If specified, makes title cells $\backslash$parbox'es with

the given width.
\item[\texttt{hasTitleCol}:]
 The first column contains titles.
\item[\texttt{numFormat}:]
 Specify a sprintf-style format for displaying numbers.
\end{description}%
\end{description}%
%
\item[Returns:
]~

   tex\_string: LaTeX formatted table string.
%
%
\item[See also:]%
%
\item[Author:]%
Cengiz Gunay <cgunay@emory.edu>, 2004/12/09
%
\end{description}
\methodline%
\subsubsection[Function \texttt{chanTables2DB}]{Function \texttt{functions/chanTables2DB}}%
\index[funcref]{functions@\fidxl{functions}!chanTables2DB@\fidxl{chanTables2DB}}%
\label{ref_functions__chanTables2DB}%
\hypertarget{ref_functions__chanTables2DB}{}%
\begin{description}
\item[Summary:]Creates a DB with channel tables exported from Genesis.
%
\item[Usage:]~%
\begin{lyxcode}%
a\_chans\_db = chanTables2DB(tables, id, props)
%
\end{lyxcode}%
%
%
\item[Parameters:]~
\begin{description}%
\item[\texttt{tables}:]
 Structures returned from the dump files generated by dump\_chans.g.
\item[\texttt{id}:]
 String that identify the source of the tables structure.
\item[\texttt{props}:]
 A structure with any optional properties.

(rest passed to tests\_db.)
\end{description}%
%
\item[Returns:
]~

	a\_chans\_db: A tests\_db object containing channel tables.
%
%
\item[See also:]%
\hyperlink{ref_trace}{\texttt{trace}}%
\ (p.~\pageref{ref_trace})%
\index[funcref]{trace@\fidxl{trace}}%
, \hyperlink{ref_trace__plot}{\texttt{trace/plot}}%
\ (p.~\pageref{ref_trace__plot})%
\index[funcref]{trace@\fidxl{trace}!plot@\fidxl{plot}}%
, \hyperlink{ref_plot_abstract}{\texttt{plot\_abstract}}%
\ (p.~\pageref{ref_plot_abstract})%
\index[funcref]{plot_abstract@\fidxl{plot\_abstract}}%
, \hyperlink{ref_GP__common}{\texttt{GP/common/dump\_chans.g (Genesis)}}%
\ (p.~\pageref{ref_GP__common})%
\index[funcref]{GP@\fidxl{GP}!common@\fidxl{common}}%
%
\item[Author:]%
Cengiz Gunay <cgunay@emory.edu>, 2007/03/07
%
\end{description}
\methodline%
\subsubsection[Function \texttt{collectspikes}]{Function \texttt{functions/collectspikes}}%
\index[funcref]{functions@\fidxl{functions}!collectspikes@\fidxl{collectspikes}}%
\label{ref_functions__collectspikes}%
\hypertarget{ref_functions__collectspikes}{}%
\begin{description}
%
%
%
%
%
%
%
\item[Author:]%
<adelgado@biology.emory.edu>
%
\end{description}
\methodline%
\subsubsection[Function \texttt{colormapBlueCrossRed}]{Function \texttt{functions/colormapBlueCrossRed}}%
\index[funcref]{functions@\fidxl{functions}!colormapBlueCrossRed@\fidxl{colormapBlueCrossRed}}%
\label{ref_functions__colormapBlueCrossRed}%
\hypertarget{ref_functions__colormapBlueCrossRed}{}%
\begin{description}
\item[Summary:]Blue to red crossing colormap, with a black-colored zero-crossing.
%
\item[Usage:]~%
\begin{lyxcode}%
colors = colormapBlueCrossRed(num\_half\_colors)
%
\end{lyxcode}%
%
\item[Description:]%
Colormap contains (2 * num\_half\_colors + 1) colors, where (num\_half\_colors + 1) is the 
 zero crossing.
%%
\item[Parameters:]~
\begin{description}%
\item[\texttt{num\_half\_colors}:]
 Number of colors to generate on one of the red or blue scales.
\item[\texttt{props}:]
 A structure with any optional properties.
\end{description}%
%
\item[Returns:
]~

	colors: RGB color matrix to be passed to colormap function.
%
%
\item[See also:]%
\hyperlink{ref_colormap}{\texttt{colormap}}%
\ (p.~\pageref{ref_colormap})%
\index[funcref]{colormap@\fidxl{colormap}}%
%
\item[Author:]%
Cengiz Gunay <cgunay@emory.edu>, 2006/06/05
%
\end{description}
\methodline%
\subsubsection[Function \texttt{defaultValue}]{Function \texttt{functions/defaultValue}}%
\index[funcref]{functions@\fidxl{functions}!defaultValue@\fidxl{defaultValue}}%
\label{ref_functions__defaultValue}%
\hypertarget{ref_functions__defaultValue}{}%
\begin{description}
\item[Summary:]If variable unset (either nonexistent or empty), assign it
 a default value. Otherwise the variable remains unchanged.
%
\item[Usage:]~%
\begin{lyxcode}%
var = defaultValue(varname, a\_defaultvalue)
%
\end{lyxcode}%
%
\item[Description:]%
If the variable has already been defined, it keeps unchanged. If the
 variable doesn't exist or is an empty matrix, it will be assigned a
 default value to it.
%%
\item[Parameters:]~
\begin{description}%
\item[\texttt{varname}:]
 a string. the name of the variable.
\item[\texttt{a\_defaultvalue}:]
 value for the variable.
\end{description}%
%
%
\item[Example:]~
\begin{lyxcode}   SamplingRate = defaultValue('SamplingRate', 10);
\\%
\end{lyxcode}
%
%
\item[Author:]%
Li, Su
%
\end{description}
\methodline%
\subsubsection[Function \texttt{diff2T}]{Function \texttt{functions/diff2T}}%
\index[funcref]{functions@\fidxl{functions}!diff2T@\fidxl{diff2T}}%
\label{ref_functions__diff2T}%
\hypertarget{ref_functions__diff2T}{}%
\begin{description}
\item[Summary:]Estimate of second derivative using Taylor expansion.
%
\item[Usage:]~%
\begin{lyxcode}%
deriv2 = diff2T(x, dy)
%
\end{lyxcode}%
%
\item[Description:]%
d\textasciicircum{}2 x     - x(k-2) + 16 * x(k-1) - 30 * x(k) + 16 * x(k+1) - x(k+2)
  ------- = -----------------------------------------------------------
   dy\textasciicircum{}2			        12 * dy\textasciicircum{}2
%%
\item[Parameters:]~
\begin{description}%
\item[\texttt{x}:]
 A vector of x = f(y).
\item[\texttt{dy}:]
 The resolution of the discrete points in the vector.
\end{description}%
%
\item[Returns:
]~

 	deriv2: Estimate of the derivative.
%
%
%
\item[Author:]%
Cengiz Gunay <cgunay@emory.edu>, 2004/11/15
%
\end{description}
\methodline%
\subsubsection[Function \texttt{diff2T\_h4}]{Function \texttt{functions/diff2T\_h4}}%
\index[funcref]{functions@\fidxl{functions}!diff2T_h4@\fidxl{diff2T\_h4}}%
\label{ref_functions__diff2T_h4}%
\hypertarget{ref_functions__diff2T_h4}{}%
\begin{description}
\item[Summary:]Estimate of second derivative using Taylor expansion (derived with same method as diffT).
%
\item[Usage:]~%
\begin{lyxcode}%
deriv2 = diff2T\_h4(x, dy)
%
\end{lyxcode}%
%
\item[Description:]%
d\textasciicircum{}2 x     x(k-2) - x(k-1) - x(k+1) + x(k+2)
  ------- = -----------------------------------
   dy\textasciicircum{}2		6 * dy\textasciicircum{}2
%%
\item[Parameters:]~
\begin{description}%
\item[\texttt{x}:]
 A vector of x = f(y).
\item[\texttt{dy}:]
 The resolution of the discrete points in the vector.
\end{description}%
%
\item[Returns:
]~

 	deriv2: Estimate of the derivative.
%
%
%
\item[Author:]%
Cengiz Gunay <cgunay@emory.edu>, 2005/04/15
%
\end{description}
\methodline%
\subsubsection[Function \texttt{diff3T}]{Function \texttt{functions/diff3T}}%
\index[funcref]{functions@\fidxl{functions}!diff3T@\fidxl{diff3T}}%
\label{ref_functions__diff3T}%
\hypertarget{ref_functions__diff3T}{}%
\begin{description}
\item[Summary:]Estimate of third derivative using Taylor expansion.
%
\item[Usage:]~%
\begin{lyxcode}%
deriv3 = diff3T(x, dy)
%
\end{lyxcode}%
%
\item[Description:]%
d\textasciicircum{}3 x     x(k-3) - 8 * x(k-2) + 13 * x(k-1) - 13 * x(k+1) + 8 * x(k+2) - x(k+3)
  ------- = -----------------------------------------------------------------------
   dy\textasciicircum{}3			        8 * dy\textasciicircum{}3
%%
\item[Parameters:]~
\begin{description}%
\item[\texttt{x}:]
 A vector of x = f(y).
\item[\texttt{dy}:]
 The resolution of the discrete points in the vector.
\end{description}%
%
\item[Returns:
]~

 	deriv3: Estimate of the derivative.
%
%
%
\item[Author:]%
Cengiz Gunay <cgunay@emory.edu>, 2004/11/15
%
\end{description}
\methodline%
\subsubsection[Function \texttt{diff3T\_h4}]{Function \texttt{functions/diff3T\_h4}}%
\index[funcref]{functions@\fidxl{functions}!diff3T_h4@\fidxl{diff3T\_h4}}%
\label{ref_functions__diff3T_h4}%
\hypertarget{ref_functions__diff3T_h4}{}%
\begin{description}
\item[Summary:]Estimate of third derivative using Taylor expansion (derived with same method as diffT and diff2T\_h4).
%
\item[Usage:]~%
\begin{lyxcode}%
deriv2 = diff3T\_h4(x, dy)
%
\end{lyxcode}%
%
\item[Description:]%
d\textasciicircum{}3 x     - x(k-2) + 2 * x(k-1) - 2 * x(k+1) + x(k+2)
  ------- = ---------------------------------------------
   dy\textasciicircum{}3			12 * dy\textasciicircum{}3
%%
\item[Parameters:]~
\begin{description}%
\item[\texttt{x}:]
 A vector of x = f(y).
\item[\texttt{dy}:]
 The resolution of the discrete points in the vector.
\end{description}%
%
\item[Returns:
]~

 	deriv2: Estimate of the derivative.
%
%
%
\item[Author:]%
Cengiz Gunay <cgunay@emory.edu>, 2005/04/18
%
\end{description}
\methodline%
\subsubsection[Function \texttt{diffT}]{Function \texttt{functions/diffT}}%
\index[funcref]{functions@\fidxl{functions}!diffT@\fidxl{diffT}}%
\label{ref_functions__diffT}%
\hypertarget{ref_functions__diffT}{}%
\begin{description}
\item[Summary:]Estimate of first derivative using Taylor expansion.
%
\item[Usage:]~%
\begin{lyxcode}%
deriv = diffT(x, dy)
%
\end{lyxcode}%
%
\item[Description:]%
dx     x(k-2) - 8 * x(k-1) + 8 * x(k+1) - x(k+2)
  ---- = ------------------------------------------
   dy			12 * dy
%%
\item[Parameters:]~
\begin{description}%
\item[\texttt{x}:]
 A vector.
\item[\texttt{dy}:]
 The resolution of the discrete points in the vector.
\end{description}%
%
\item[Returns:
]~

 	deriv: Estimate of the first derivative.
%
%
%
\item[Author:]%
Cengiz Gunay <cgunay@emory.edu>, 2004/11/15
%
\end{description}
\methodline%
\subsubsection[Function \texttt{fillederrorbar}]{Function \texttt{functions/fillederrorbar}}%
\index[funcref]{functions@\fidxl{functions}!fillederrorbar@\fidxl{fillederrorbar}}%
\label{ref_functions__fillederrorbar}%
\hypertarget{ref_functions__fillederrorbar}{}%
\begin{description}
\item[Summary:]Plots an errorbar with the middle points filled with the pen color.
%
\item[Usage:]~%
\begin{lyxcode}%
handles = fillederrorbar(...)
%
\end{lyxcode}%
%
%
\item[Parameters:]~

(see errorbar)
%
\item[Returns:
]~

	handles: Handles to graphics objects.
%
%
\item[See also:]%
%
\item[Author:]%
Cengiz Gunay <cgunay@emory.edu>, 2004/10/13
%
\end{description}
\methodline%
\subsubsection[Function \texttt{findspikes}]{Function \texttt{functions/findspikes}}%
\index[funcref]{functions@\fidxl{functions}!findspikes@\fidxl{findspikes}}%
\label{ref_functions__findspikes}%
\hypertarget{ref_functions__findspikes}{}%
\begin{description}
%
%
%
%
%
%
%
\item[Author:]%
Li, Su based on the original of Alfonso Delagado-Reyes
%
\end{description}
\methodline%
\subsubsection[Function \texttt{findspikes\_old}]{Function \texttt{functions/findspikes\_old}}%
\index[funcref]{functions@\fidxl{functions}!findspikes_old@\fidxl{findspikes\_old}}%
\label{ref_functions__findspikes_old}%
\hypertarget{ref_functions__findspikes_old}{}%
\begin{description}
%
%
%
%
%
%
%
\item[Author:]%
<adelgado@biology.emory.edu>, 2003-03-31
%
\end{description}
\methodline%
\subsubsection[Function \texttt{findVectorInMatrix}]{Function \texttt{functions/findVectorInMatrix}}%
\index[funcref]{functions@\fidxl{functions}!findVectorInMatrix@\fidxl{findVectorInMatrix}}%
\label{ref_functions__findVectorInMatrix}%
\hypertarget{ref_functions__findVectorInMatrix}{}%
\begin{description}
\item[Summary:]Finds rows of data that match row.
%
\item[Usage:]~%
\begin{lyxcode}%
idx = findVectorInMatrix(data, row)
%
\end{lyxcode}%
%
\item[Description:]%
Matlab's eq (==) command unfortunately doesn't allow this directly.
%%
\item[Parameters:]~
\begin{description}%
\item[\texttt{data}:]
 A matrix or column vector.
\item[\texttt{row}:]
 A row vector.
\item[\texttt{Returns}:]

\item[\texttt{idx}:]
 Indices of matching rows in the original data matrix.
\end{description}%
%
%
%
\item[See also:]%
%
\item[Author:]%
Cengiz Gunay <cgunay@emory.edu>, 2005/09/1
%
\end{description}
\methodline%
\subsubsection[Function \texttt{getFieldDefault}]{Function \texttt{functions/getFieldDefault}}%
\index[funcref]{functions@\fidxl{functions}!getFieldDefault@\fidxl{getFieldDefault}}%
\label{ref_functions__getFieldDefault}%
\hypertarget{ref_functions__getFieldDefault}{}%
\begin{description}
\item[Summary:]Get a field from a struct and return the default\_value if field does not exist.
%
\item[Usage:]~%
\begin{lyxcode}%
value = getFieldDefault(structure, fieldname, default\_value)
%
\end{lyxcode}%
%
%
\item[Parameters:]~
\begin{description}%
\item[\texttt{fieldname}:]
 Field name.
\item[\texttt{default\_value}:]
 Value to return if field doesn't exist.
\end{description}%
%
%
%
%
\item[Author:]%
Li, Su - 2007
%
\end{description}
\methodline%
\subsubsection[Function \texttt{getfuzzyfield}]{Function \texttt{functions/getfuzzyfield}}%
\index[funcref]{functions@\fidxl{functions}!getfuzzyfield@\fidxl{getfuzzyfield}}%
\label{ref_functions__getfuzzyfield}%
\hypertarget{ref_functions__getfuzzyfield}{}%
\begin{description}
%
%
%
%
%
%
%
\item[Author:]%
Li, Su - 2007
%
\end{description}
\methodline%
\subsubsection[Function \texttt{gettracelist2}]{Function \texttt{functions/gettracelist2}}%
\index[funcref]{functions@\fidxl{functions}!gettracelist2@\fidxl{gettracelist2}}%
\label{ref_functions__gettracelist2}%
\hypertarget{ref_functions__gettracelist2}{}%
\begin{description}
\item[Summary:]Gets a list of the form: '1 3 7-10' and returns a column vector with the traces numbers, and the number of traces.
%
\item[Usage:]~%
\begin{lyxcode}%
[traces, ntraces] = gettracelist2('list');
%
\end{lyxcode}%
%
\item[Description:]%
Please note the space between single traces and the dash for ranges of
traces.
%%
%
%
%
%
\item[Author:]%
<adelgado@biology.emory.edu>
%
\end{description}
\methodline%
\subsubsection[Function \texttt{growRange}]{Function \texttt{functions/growRange}}%
\index[funcref]{functions@\fidxl{functions}!growRange@\fidxl{growRange}}%
\label{ref_functions__growRange}%
\hypertarget{ref_functions__growRange}{}%
\begin{description}
\item[Summary:]Returns the maximal range from rows of axis limits. 
%
\item[Usage:]~%
\begin{lyxcode}%
range = growRange(ranges)
%
\end{lyxcode}%
%
%
\item[Parameters:]~
\begin{description}%
\item[\texttt{ranges}:]
 A matrix where each row is return val of axis func.
\end{description}%
%
\item[Returns:
]~

	range: The maximal range obtained that includes all given axes.
%
%
\item[See also:]%
%
\item[Author:]%
Cengiz Gunay <cgunay@emory.edu>, 2004/10/13
%
\end{description}
\methodline%
\subsubsection[Function \texttt{interpValByIndex}]{Function \texttt{functions/interpValByIndex}}%
\index[funcref]{functions@\fidxl{functions}!interpValByIndex@\fidxl{interpValByIndex}}%
\label{ref_functions__interpValByIndex}%
\hypertarget{ref_functions__interpValByIndex}{}%
\begin{description}
\item[Summary:]Finds the interpolated value by using the real valued index from the data vector.
%
\item[Usage:]~%
\begin{lyxcode}%
val = interpValByIndex(idx, data)
%
\end{lyxcode}%
%
\item[Description:]%
Parameters:
	idx: A real-valued index.
	data: A data vector.
%%
%
\item[Returns:
]~

	val: the value taken from the nearest integer indices of data and interpolated.
%
\item[Example:]~
\begin{lyxcode} >> a= [1 2 3];
\\%
 >> interpValByIndex(1.5, a)
\\%
 ans =
\\%
    1.5000
\\%
\end{lyxcode}
%
\item[See also:]%
\hyperlink{ref_spike_shape}{\texttt{spike\_shape}}%
\ (p.~\pageref{ref_spike_shape})%
\index[funcref]{spike_shape@\fidxl{spike\_shape}}%
%
\item[Author:]%
Cengiz Gunay <cgunay@emory.edu>, 2004/08/02
%
\end{description}
\methodline%
\subsubsection[Function \texttt{loadtraces}]{Function \texttt{functions/loadtraces}}%
\index[funcref]{functions@\fidxl{functions}!loadtraces@\fidxl{loadtraces}}%
\label{ref_functions__loadtraces}%
\hypertarget{ref_functions__loadtraces}{}%
\begin{description}
%
%
%
\item[Parameters:]~
\begin{description}%
\item[\texttt{file}:]
 PCDX file.
\item[\texttt{tracelist}:]
 A string of trace description, such as '1-10'.
\item[\texttt{channel}:]
 Channel to read from.
\item[\texttt{quiet}:]
 (Optional) If 1, produce on print outs.
\end{description}%
%
%
%
%
\item[Author:]%
<adelgado@biology.emory.edu>
%
\end{description}
\methodline%
\subsubsection[Function \texttt{loadVclampAbf}]{Function \texttt{functions/loadVclampAbf}}%
\index[funcref]{functions@\fidxl{functions}!loadVclampAbf@\fidxl{loadVclampAbf}}%
\label{ref_functions__loadVclampAbf}%
\hypertarget{ref_functions__loadVclampAbf}{}%
\begin{description}
\item[Summary:]Load I and V traces from an ABF file.
%
\item[Usage:]~%
\begin{lyxcode}%
[time, dt, data\_i, data\_v, cell\_name] = loadVclampAbf(filename, props)
%
\end{lyxcode}%
%
\item[Description:]%
If filename is wrong or not specified, a dialog will pop up to choose
 file. ABF2 files are not fully supported (see abf2load.m). Time is
 assumed to be in s and converted to ms.
%%
\item[Parameters:]~
\begin{description}%
\item[\texttt{filename}:]
 Full path to filename.
\item[\texttt{props}:]
 A structure with any optional properties.
\begin{description}%
\item[\texttt{scaleI}:]
 multiplier to correct I values to the units specified in file.
\item[\texttt{actualProtocols}:]
 Means current trace is a TTL pulse and its

magnitude is meaningless.
\end{description}%
\end{description}%
%
\item[Returns:
]~

   time: Time vector for measurements [ms],
   dt: Time step [ms],
   data\_i: Current traces (assumed [nA]),
   data\_v: Voltage traces (assumed [mV]),
   cell\_name: Extracted from the file name part of the path.
%
\item[Example:]~
\begin{lyxcode} >> [time, dt, data\_i, data\_v, cell\_name] = ...
\\%
    loadVclampAbf('data-dir/cell-A.abf')
\\%
 >> plotVclampStack(time, data\_i, data\_v, cell\_name);
\\%
\end{lyxcode}
%
\item[See also:]%
\hyperlink{ref_abf2load}{\texttt{abf2load}}%
\ (p.~\pageref{ref_abf2load})%
\index[funcref]{abf2load@\fidxl{abf2load}}%
, \hyperlink{ref_plotVclampAbf}{\texttt{plotVclampAbf}}%
\ (p.~\pageref{ref_plotVclampAbf})%
\index[funcref]{plotVclampAbf@\fidxl{plotVclampAbf}}%
, \hyperlink{ref_plotVclampStack}{\texttt{plotVclampStack}}%
\ (p.~\pageref{ref_plotVclampStack})%
\index[funcref]{plotVclampStack@\fidxl{plotVclampStack}}%
%
\item[Author:]%
Cengiz Gunay <cgunay@emory.edu>, 2009/12/17
%
\end{description}
\methodline%
\subsubsection[Function \texttt{logLevels}]{Function \texttt{functions/logLevels}}%
\index[funcref]{functions@\fidxl{functions}!logLevels@\fidxl{logLevels}}%
\label{ref_functions__logLevels}%
\hypertarget{ref_functions__logLevels}{}%
\begin{description}
\item[Summary:]Returns a logarithmic-scaled series between min\_val and max\_val with num\_levels elements.
%
\item[Usage:]~%
\begin{lyxcode}%
levels = logLevels(min\_val, max\_val, num\_levels)
%
\end{lyxcode}%
%
%
\item[Parameters:]~
\begin{description}%
\item[\texttt{min\_val, max\_val}:]
 The low and high boundaries for the output value.
\item[\texttt{num\_levels}:]
 Number of elements to produce, including the boundaries.
\end{description}%
%
\item[Returns:
]~

 	levels: A column vector of logarithmic series between min\_val and max\_val.
%
%
%
\item[Author:]%
Cengiz Gunay <cgunay@emory.edu>, 2005/04/18
%
\end{description}
\methodline%
\subsubsection[Function \texttt{makeIdealClampV}]{Function \texttt{functions/makeIdealClampV}}%
\index[funcref]{functions@\fidxl{functions}!makeIdealClampV@\fidxl{makeIdealClampV}}%
\label{ref_functions__makeIdealClampV}%
\hypertarget{ref_functions__makeIdealClampV}{}%
\begin{description}
\item[Summary:]Make voltage traces that mimic an ideal voltage clamp.
%
\item[Usage:]~%
\begin{lyxcode}%
[trace\_v] = makeIdealClampV(t\_vals, pre\_v, pulse\_v, post\_v, dt, id, props)
%
\end{lyxcode}%
%
%
\item[Parameters:]~
\begin{description}%
\item[\texttt{t\_vals}:]
 Vector with times of pulse start, end and trace end [ms].
\item[\texttt{pre\_v, post\_v}:]
 Holding and final voltage values [mV].
\item[\texttt{pulse\_v}:]
 Vector of variable voltage steps [mV].
\item[\texttt{dt}:]
 Resolution of time in trace produced [ms].
\item[\texttt{id}:]
 An identifying string.
\item[\texttt{props}:]
 A structure with any optional properties.

(Rest passed to trace)
\end{description}%
%
\item[Returns:
]~

   trace\_v: A trace object with the voltage traces.
%
\item[Example:]~
\begin{lyxcode} >> tr\_v = makeIdealClampV([10 100 110], -90, -80:10:60, -10, 1e-1, ...
\\%
                           'Na chan voltage clamp protocol')
\\%
 >> plot(tr\_v)
\\%
 >> vc\_test = voltage\_clamp(data\_i, tr\_v.data, tr\_v.dt, 1e-9, 'sim Na data')
\\%
\end{lyxcode}
%
\item[See also:]%
\hyperlink{ref_trace}{\texttt{trace}}%
\ (p.~\pageref{ref_trace})%
\index[funcref]{trace@\fidxl{trace}}%
, \hyperlink{ref_voltage_clamp}{\texttt{voltage\_clamp}}%
\ (p.~\pageref{ref_voltage_clamp})%
\index[funcref]{voltage_clamp@\fidxl{voltage\_clamp}}%
%
\item[Author:]%
Cengiz Gunay <cgunay@emory.edu>, 2010/10/07
%
\end{description}
\methodline%
\subsubsection[Function \texttt{makeIdx}]{Function \texttt{functions/makeIdx}}%
\index[funcref]{functions@\fidxl{functions}!makeIdx@\fidxl{makeIdx}}%
\label{ref_functions__makeIdx}%
\hypertarget{ref_functions__makeIdx}{}%
\begin{description}
\item[Summary:]Prepare the idx structure from names.
%
\item[Usage:]~%
\begin{lyxcode}%
idx = makeIdx(names)
%
\end{lyxcode}%
%
\item[Description:]%
Helper function.
%%
\item[Parameters:]~
\begin{description}%
\item[\texttt{names}:]
 Cell array of names for a db dimension.
\end{description}%
%
\item[Returns:
]~

	idx: Structure associating names to array indices.
%
%
\item[See also:]%
\hyperlink{ref_tests_db}{\texttt{tests\_db}}%
\ (p.~\pageref{ref_tests_db})%
\index[funcref]{tests_db@\fidxl{tests\_db}}%
%
\item[Author:]%
Cengiz Gunay <cgunay@emory.edu>, 2004/09/17
%
\end{description}
\methodline%
\subsubsection[Function \texttt{maxima}]{Function \texttt{functions/maxima}}%
\index[funcref]{functions@\fidxl{functions}!maxima@\fidxl{maxima}}%
\label{ref_functions__maxima}%
\hypertarget{ref_functions__maxima}{}%
\begin{description}
\item[Summary:]Find all local maxima.
%
\item[Usage:]~%
\begin{lyxcode}%
x\_idx = maxima(x)
%
\end{lyxcode}%
%
\item[Description:]%
Finds derivative sign-flipping points where the second derivative is 
 less than zero.
%%
\item[Parameters:]~
\begin{description}%
\item[\texttt{x}:]
 A vector.
\end{description}%
%
\item[Returns:
]~

 	x\_idx: Indices of maxima.
%
%
%
\item[Author:]%
Cengiz Gunay <cgunay@emory.edu>, 2005/04/18
%
\end{description}
\methodline%
\subsubsection[Function \texttt{meanSpikeFreq}]{Function \texttt{functions/meanSpikeFreq}}%
\index[funcref]{functions@\fidxl{functions}!meanSpikeFreq@\fidxl{meanSpikeFreq}}%
\label{ref_functions__meanSpikeFreq}%
\hypertarget{ref_functions__meanSpikeFreq}{}%
\begin{description}
\item[Summary:]Returns the mean firing frequency in Hz according to mean $\backslash$
  	    inter-spike-interval of the given spike train and the 
	    time resolution dt.
%
\item[Usage:]~%
\begin{lyxcode}%
meanFreq = meanSpikeFreq( spike\_train, dt, period )
%
\end{lyxcode}%
%
\item[Description:]%
Parameters:
		spike\_train: Spike times returned by findspikes
		dt: Time step size [s]
		period: Duration of the total time period [dt]
%%
%
%
%
\item[See also:]%
%
\item[Author:]%
Cengiz Gunay <cgunay@emory.edu>, 2004/03/08
%
\end{description}
\methodline%
\subsubsection[Function \texttt{mergeStructs}]{Function \texttt{functions/mergeStructs}}%
\index[funcref]{functions@\fidxl{functions}!mergeStructs@\fidxl{mergeStructs}}%
\label{ref_functions__mergeStructs}%
\hypertarget{ref_functions__mergeStructs}{}%
\begin{description}
\item[Summary:]Merges all the structures given as arguments and makes a single structure.
%
\item[Usage:]~%
\begin{lyxcode}%
results = mergeStructs( struct1 [, struct2, ...] )
%
\end{lyxcode}%
%
\item[Description:]%
The fields will in earlier arguments will have priority. So, while merging two
 structs, if there are duplicate fields, the fields in the first will be preserved.
%%
\item[Parameters:]~

struct(n): A structure.
%
\item[Returns:
]~

 	results: The merged structure.
%
\item[Example:]~
\begin{lyxcode} mergeStructs( struct('hello', 1), struct('bye', 2) );
\\%
  => struct('hello', 1, 'bye', 2)
\\%
\end{lyxcode}
%
%
\item[Author:]%
Cengiz Gunay <cgunay@emory.edu>, 2004/09/13
%
\end{description}
\methodline%
\subsubsection[Function \texttt{mergeStructsRecursive}]{Function \texttt{functions/mergeStructsRecursive}}%
\index[funcref]{functions@\fidxl{functions}!mergeStructsRecursive@\fidxl{mergeStructsRecursive}}%
\label{ref_functions__mergeStructsRecursive}%
\hypertarget{ref_functions__mergeStructsRecursive}{}%
\begin{description}
\item[Summary:]Merges given structures into a single structure, merging substructures recursively.
%
\item[Usage:]~%
\begin{lyxcode}%
results = mergeStructsRecursive( struct1 [, struct2, ...] )
%
\end{lyxcode}%
%
\item[Description:]%
The fields will in earlier arguments will have priority. So, while merging two
 structs, if there are duplicate fields, the fields in the first will be
 preserved. If a common field is a structure, then mergeStructsRecursive
 is called to merge their contents.
%%
\item[Parameters:]~

struct(n): A structure.
%
\item[Returns:
]~

 	results: The merged structure.
%
\item[Example:]~
\begin{lyxcode} >> mergeStructsRecursive( struct('hello', struct('a', 1), 
\\%
                            struct('hello', struct('b', 2)) );
\\%
  => struct('hello', struct('a', 1, 'b', 2)
\\%
\end{lyxcode}
%
%
\item[Author:]%
Cengiz Gunay <cgunay@emory.edu>, 2004/09/13
%
\end{description}
\methodline%
\subsubsection[Function \texttt{ns\_CIPlist}]{Function \texttt{functions/ns\_CIPlist}}%
\index[funcref]{functions@\fidxl{functions}!ns_CIPlist@\fidxl{ns\_CIPlist}}%
\label{ref_functions__ns_CIPlist}%
\hypertarget{ref_functions__ns_CIPlist}{}%
\begin{description}
%
%
%
%
%
%
%
\item[Author:]%
Dawid Kurzyniec
%
\end{description}
\methodline%
\subsubsection[Function \texttt{ns\_load\_tracesets}]{Function \texttt{functions/ns\_load\_tracesets}}%
\index[funcref]{functions@\fidxl{functions}!ns_load_tracesets@\fidxl{ns\_load\_tracesets}}%
\label{ref_functions__ns_load_tracesets}%
\hypertarget{ref_functions__ns_load_tracesets}{}%
\begin{description}
\item[Summary:]Return a set of physiol\_cip\_traceset objects loaded from a single NeuroSAGE HDF5 file.
%
\item[Usage:]~%
\begin{lyxcode}%
a\_tss = ns\_load\_tracesets(data\_src, props)
%
\end{lyxcode}%
%
\item[Description:]%
This allows customized loading each NeuroSAGE file separately. Only
 loads traces that has the word 'cip' or 'spont' in the NeuroSAGE sequence name.
 Sample rate, channel gain and dy values are read from the acquisition data.
%%
\item[Parameters:]~
\begin{description}%
\item[\texttt{data\_src}:]
 A pattern for one or more NeuroSAGE filename or structure output of ns\_open\_file.
\item[\texttt{props}:]
 A structure with any optional properties.
\begin{description}%
\item[\texttt{VmChan}:]
 (Optional) If a string, read voltage trace from channel having

this string (e.g., 'Amp1 Vm'). If numeric, use as channel
number. Added to the neuron\_id to distinguish
multiple neurons recorded in same file. If not
specified, the first voltage channel is used.
\item[\texttt{ImChan}:]
 (Optional) Similar to VmChan for reading current

trace. Does not affect neuron\_id.
\item[\texttt{VGain, IGain}:]
 for HDF5 files, these two fields only works as a default value

when the gains are not specified in the file.
\item[\texttt{IncludeSeq}:]
 A string or cell array of strings specifying keywords in

sequence name to look for.
\item[\texttt{ExcludeSeq}:]
 A string or cell array of strings specifying keywords in

sequence name to avoid searching.
\item[\texttt{addTreats}:]
 Structure of default treatment names and their

values for this traceset to keep consistent accross 
tracesets. Use only lowercase in treatment names.
\item[\texttt{fixTreats}:]
 Override wrong treatment information with

these. Same format as addTreats.
\item[\texttt{renameTreats}:]
 Structure with from->to rename pairs.
\item[\texttt{trials}:]
 A vector of trials to load from the file. All others

are skipped.
(All other props are passed to physiol\_cip\_traceset)
\end{description}%
\end{description}%
%
\item[Returns:
]~

	a\_tss: Cell array of physiol\_cip\_traceset objects.
%
%
\item[See also:]%
\hyperlink{ref_physiol_cip_traceset_fileset}{\texttt{physiol\_cip\_traceset\_fileset}}%
\ (p.~\pageref{ref_physiol_cip_traceset_fileset})%
\index[funcref]{physiol_cip_traceset_fileset@\fidxl{physiol\_cip\_traceset\_fileset}}%
, \hyperlink{ref_physiol_cip_traceset}{\texttt{physiol\_cip\_traceset}}%
\ (p.~\pageref{ref_physiol_cip_traceset})%
\index[funcref]{physiol_cip_traceset@\fidxl{physiol\_cip\_traceset}}%
, \hyperlink{ref_params_tests_dataset}{\texttt{params\_tests\_dataset}}%
\ (p.~\pageref{ref_params_tests_dataset})%
\index[funcref]{params_tests_dataset@\fidxl{params\_tests\_dataset}}%
%
\item[Author:]%
Li, Su; Cengiz Gunay <cgunay@emory.edu>; and Jeremy Edgerton, 2007/12/18
%
\end{description}
\methodline%
\subsubsection[Function \texttt{parseFilenameNamesVals}]{Function \texttt{functions/parseFilenameNamesVals}}%
\index[funcref]{functions@\fidxl{functions}!parseFilenameNamesVals@\fidxl{parseFilenameNamesVals}}%
\label{ref_functions__parseFilenameNamesVals}%
\hypertarget{ref_functions__parseFilenameNamesVals}{}%
\begin{description}
\item[Summary:]Parses filename to extract names and values of parameters.
%
\item[Usage:]~%
\begin{lyxcode}%
names\_vals = parseFilenameNamesVals(filename, props)
%
\end{lyxcode}%
%
\item[Description:]%
Parses the given string (e.g., filename) that has names and
 values separated by underscores (\_). 
%%
\item[Parameters:]~
\begin{description}%
\item[\texttt{filename}:]
 file name (no need to exist)
\item[\texttt{props}:]
 Structure with optional properties:
\begin{description}%
\item[\texttt{namesFirst}:]
 If 1, names precede values (default=1).
\item[\texttt{skipNum}:]
 Number of words to skip (default=0). If -1, all

words are skipped until a number is found. this
makes sense when namesFirst=0.
\end{description}%
\end{description}%
%
\item[Returns:
]~

   names\_vals: A two-column cell array with names and values.
   pre\_name: (Optional) Skipped prefix words in the filename.
%
\item[Example:]~
\begin{lyxcode} Names first:
\\%
 >> nv = parseFilenameNamesVals('hello\_boys\_6\_girls\_4.txt', struct('skipNum', 1))
\\%
 nv = 
\\%
    'girls'    [6]
\\%
    'boys'     [4]
\\%
 Same result with values first:
\\%
 >> nv = parseFilenameNamesVals('data/hello\_6\_girls\_4\_boys.txt',
\\%
                struct('namesFirst', 0, 'skipNum', -1))
\\%
\end{lyxcode}
%
%
\item[Author:]%
Cengiz Gunay <cgunay@emory.edu>, 2004/03/10
%
\end{description}
\methodline%
\subsubsection[Function \texttt{parseGenesisFilename}]{Function \texttt{functions/parseGenesisFilename}}%
\index[funcref]{functions@\fidxl{functions}!parseGenesisFilename@\fidxl{parseGenesisFilename}}%
\label{ref_functions__parseGenesisFilename}%
\hypertarget{ref_functions__parseGenesisFilename}{}%
\begin{description}
\item[Summary:](OBSOLETE, see parseFilenameNamesVals) Parses the GENESIS filename to get names and values of simulation parameters.
 Usage:
 names\_vals = parseGenesisFilename(filename)
%
%
\item[Description:]%
Parameters:
		filename: GENESIS filename (no need to exist)
%%
%
\item[Returns:
]~

		names\_vals: A two-column cell array with names and values.
%
%
\item[See also:]%
\hyperlink{ref_parseFilenameNamesVals}{\texttt{parseFilenameNamesVals}}%
\ (p.~\pageref{ref_parseFilenameNamesVals})%
\index[funcref]{parseFilenameNamesVals@\fidxl{parseFilenameNamesVals}}%
%
\item[Author:]%
Cengiz Gunay <cgunay@emory.edu>, 2004/03/10
%
\end{description}
\methodline%
\subsubsection[Function \texttt{plotColormap}]{Function \texttt{functions/plotColormap}}%
\index[funcref]{functions@\fidxl{functions}!plotColormap@\fidxl{plotColormap}}%
\label{ref_functions__plotColormap}%
\hypertarget{ref_functions__plotColormap}{}%
\begin{description}
\item[Summary:]Colormap plot that requires displaying a colorbar.
%
\item[Usage:]~%
\begin{lyxcode}%
h = plotColormap(data, a\_colormap, num\_colors, props)
%
\end{lyxcode}%
%
\item[Description:]%
Mainly serves to format the colorbar, which can only be plotted after
 we prepare the main axis. Thus it cannot be easily integrated into
 plot\_stack. 
%%
\item[Parameters:]~
\begin{description}%
\item[\texttt{data}:]
 2D matrix with image data or cell array to be passed as

arguments to arbitrary plot command (see props).
\item[\texttt{a\_colormap}:]
 Colormap vector, function name or handle to colormap (e.g., 'jet').
\item[\texttt{num\_colors}:]
 Parameter to be passed to the a\_colormap.
\item[\texttt{props}:]
 A structure with any optional properties.
\begin{description}%
\item[\texttt{command}:]
 Plot command to interpret data (default='image').
\item[\texttt{colorbar}:]
 If given, show colorbar on plot.
\item[\texttt{colorbarProps}:]
 Set colorbar axis properties.
\item[\texttt{colorbarLabel}:]
 Set colorbar y-axis label.
\item[\texttt{truncateDecDigits}:]
 Truncate labels to this many decimal digits.
\item[\texttt{minValue,maxValue}:]
 Minimal and maximal values represented by

1, num\_colors to annotate the colorbar, resp.
\item[\texttt{reverseYaxis}:]
 If 1, display y-axis values in reverse (default=0).
\end{description}%
\end{description}%
%
\item[Returns:
]~

   h: Handle to main plot object (e.g., image).
%
%
\item[See also:]%
\hyperlink{ref_colormap}{\texttt{colormap}}%
\ (p.~\pageref{ref_colormap})%
\index[funcref]{colormap@\fidxl{colormap}}%
, \hyperlink{ref_colorbar}{\texttt{colorbar}}%
\ (p.~\pageref{ref_colorbar})%
\index[funcref]{colorbar@\fidxl{colorbar}}%
%
\item[Author:]%
Cengiz Gunay <cgunay@emory.edu>, 2006/06/05
%
\end{description}
\methodline%
\subsubsection[Function \texttt{prefixStruct}]{Function \texttt{functions/prefixStruct}}%
\index[funcref]{functions@\fidxl{functions}!prefixStruct@\fidxl{prefixStruct}}%
\label{ref_functions__prefixStruct}%
\hypertarget{ref_functions__prefixStruct}{}%
\begin{description}
\item[Summary:]Adds the given prefix to each of the field names in the structure.
%
\item[Usage:]~%
\begin{lyxcode}%
new\_struct = prefixStruct(a\_struct, prefix\_str)
%
\end{lyxcode}%
%
%
\item[Parameters:]~
\begin{description}%
\item[\texttt{a\_struct}:]
 A structure.
\item[\texttt{prefix\_str}:]
 A string to be prefixed to each field name.
\end{description}%
%
\item[Returns:
]~

 	new\_struct: The new structure.
%
\item[Example:]~
\begin{lyxcode} prefixStruct( struct('bye', 1), 'hello');
\\%
  => struct('hellobye', 1)
\\%
\end{lyxcode}
%
%
\item[Author:]%
Cengiz Gunay <cgunay@emory.edu>, 2004/12/22
%
\end{description}
\methodline%
\subsubsection[Function \texttt{properAlphaNum}]{Function \texttt{functions/properAlphaNum}}%
\index[funcref]{functions@\fidxl{functions}!properAlphaNum@\fidxl{properAlphaNum}}%
\label{ref_functions__properAlphaNum}%
\hypertarget{ref_functions__properAlphaNum}{}%
\begin{description}
\item[Summary:]Replaces characters in string to make it only alphanumeric.
%
\item[Usage:]~%
\begin{lyxcode}%
a\_label = properAlphaNum( a\_label )
%
\end{lyxcode}%
%
\item[Description:]%
It will only keep the character set 'A-Z a-z 0-9 \_'. It will also
 prepend 'a\_' if the label starts with a number.
%%
\item[Parameters:]~
\begin{description}%
\item[\texttt{a\_label}:]
 A label string.
\end{description}%
%
\item[Returns:
]~

   a\_label: The corrected proper a\_label.
%
\item[Example:]~
\begin{lyxcode} >> a\_label = properAlphaNum('to $\backslash$this \_day+1 and \textasciicircum{}5')
\\%
 ans = 'tothis\_day1and5' 
\\%
\end{lyxcode}
%
%
\item[Author:]%
Cengiz Gunay <cgunay@emory.edu>, 2011/01/19
%
\end{description}
\methodline%
\subsubsection[Function \texttt{properTeXFilename}]{Function \texttt{functions/properTeXFilename}}%
\index[funcref]{functions@\fidxl{functions}!properTeXFilename@\fidxl{properTeXFilename}}%
\label{ref_functions__properTeXFilename}%
\hypertarget{ref_functions__properTeXFilename}{}%
\begin{description}
\item[Summary:]Replaces characters in string to make it a valid filename for inclusion in TeX documents.
%
\item[Usage:]~%
\begin{lyxcode}%
filename = properTeXFilename( filename )
%
\end{lyxcode}%
%
\item[Description:]%
It will replace characters like space, '/', '.', etc.
%%
\item[Parameters:]~
\begin{description}%
\item[\texttt{filename}:]
 An input filename string (without extension!).
\end{description}%
%
\item[Returns:
]~

 	filename: The corrected proper filename.
%
\item[Example:]~
\begin{lyxcode} >> fname = properTeXFilename('hello world/1')
\\%
 ans = 'hello\_world+1' 
\\%
\end{lyxcode}
%
%
\item[Author:]%
Cengiz Gunay <cgunay@emory.edu>, 2005/12/20
%
\end{description}
\methodline%
\subsubsection[Function \texttt{properTeXLabel}]{Function \texttt{functions/properTeXLabel}}%
\index[funcref]{functions@\fidxl{functions}!properTeXLabel@\fidxl{properTeXLabel}}%
\label{ref_functions__properTeXLabel}%
\hypertarget{ref_functions__properTeXLabel}{}%
\begin{description}
\item[Summary:]Replaces characters in string or cell array of strings to make it valid in TeX documents.
%
\item[Usage:]~%
\begin{lyxcode}%
a\_label = properTeXLabel( a\_label )
%
\end{lyxcode}%
%
\item[Description:]%
It will replace characters like space, '/', '.', etc.
%%
\item[Parameters:]~
\begin{description}%
\item[\texttt{a\_label}:]
 A label string.
\end{description}%
%
\item[Returns:
]~

 	a\_label: The corrected proper a\_label.
%
\item[Example:]~
\begin{lyxcode} >> a\_label = properTeXLabel('this\_day')
\\%
 ans = 'this$\backslash$\_day' 
\\%
\end{lyxcode}
%
%
\item[Author:]%
Cengiz Gunay <cgunay@emory.edu>, 2006/01/17
%
\end{description}
\methodline%
\subsubsection[Function \texttt{readgenbin}]{Function \texttt{functions/readgenbin}}%
\index[funcref]{functions@\fidxl{functions}!readgenbin@\fidxl{readgenbin}}%
\label{ref_functions__readgenbin}%
\hypertarget{ref_functions__readgenbin}{}%
\begin{description}
\item[Summary:]Reads a time-range of data from a binary GENESIS file.
%
\item[Usage:]~%
\begin{lyxcode}%
[data, time\_trace] = readgenbin(filename, start\_time, end\_time, endian);
%
\end{lyxcode}%
%
\item[Description:]%
Files should be created by the disk\_out method in the GENESIS neural
 simulator. No checking for binary type is made, so if you want reliability please
 ensure the file is a binary. Files written by GENESIS on big-endian
 machines (like old Mac and Solaris machines with PowerPC architecture)
 must be loaded with the endian='b' option. There are sanity checks to
 flag that the file may be reverse-endian, but this is not automatically
 corrected. Runs faster is you don't request the time\_trace output.
%%
\item[Parameters:]~
\begin{description}%
\item[\texttt{filename}:]
 Path to GENESIS file.
\item[\texttt{start\_time, end\_time}:]
 Time in milliseconds relative to the ACTUAL 

time of the experiment at which data adquisition started 
(if you start gathering data at 200 ms and you specify 0
start time it will not work). If either is [] or NaN, defaults
to beginning and end of trace, respectively. end\_time is 
not inclusive.
\item[\texttt{endian}:]
 (optional) Indicates file format; 'l' for little endian

and 'b' for big endian. See the "machineformat" option 
in fopen for more information. Defaults to the native endian
of this computer.
\end{description}%
%
\item[Returns:
]~

   data: Data vector or matrix read.
   time\_trace: (Optional) Corresponding time range vector (in ms).
%
\item[Example:]~
\begin{lyxcode} Fully read a native-endian file:
\\%
 >> dat = readgenbin('mydir/myfile.bin');
\\%
 Specify a time range:
\\%
 >> dat = readgenbin('mydir/myfile.bin', 100, 1000);
\\%
 Get a time vector back:
\\%
 >> [dat, t] = readgenbin('mydir/myfile.bin', 100, 1000);
\\%
 >> figure; plot(t, dat);
\\%
 Force to fully load big-endian Mac file on PC platform:
\\%
 >> dat = readgenbin('mydir/mymacfile.bin', NaN, NaN, 'b');
\\%
\end{lyxcode}
%
\item[See also:]%
\hyperlink{ref_fopen}{\texttt{fopen}}%
\ (p.~\pageref{ref_fopen})%
\index[funcref]{fopen@\fidxl{fopen}}%
%
\item[Author:]%
Alfonso Delgado-Reyes original version based in open
%
\end{description}
\methodline%
\subsubsection[Function \texttt{readNeuronVecAscii}]{Function \texttt{functions/readNeuronVecAscii}}%
\index[funcref]{functions@\fidxl{functions}!readNeuronVecAscii@\fidxl{readNeuronVecAscii}}%
\label{ref_functions__readNeuronVecAscii}%
\hypertarget{ref_functions__readNeuronVecAscii}{}%
\begin{description}
\item[Summary:]Reads Neuron simulator Vector object data from ascii files.
%
\item[Usage:]~%
\begin{lyxcode}%
[data, label] = readNeuronVecAscii(filename) 
%
\end{lyxcode}%
%
\item[Description:]%
It's one line of code just to read the data: dlmread(filename, '$\backslash$t', 2, 0)
%%
\item[Parameters:]~
\begin{description}%
\item[\texttt{filename}:]
 Full path to Neuron file.
\end{description}%
%
\item[Returns:
]~

   data: Row vector with two columns of data.
   label: String denoting Vector contents.
%
\item[Example:]~
\begin{lyxcode}   data = readNeuronVecAscii('myvec.dat');
\\%
\end{lyxcode}
%
%
\item[Author:]%
Cengiz Gunay <cengique@users.sf.net> 2012/03/02
%
\end{description}
\methodline%
\subsubsection[Function \texttt{readNeuronVecBin}]{Function \texttt{functions/readNeuronVecBin}}%
\index[funcref]{functions@\fidxl{functions}!readNeuronVecBin@\fidxl{readNeuronVecBin}}%
\label{ref_functions__readNeuronVecBin}%
\hypertarget{ref_functions__readNeuronVecBin}{}%
\begin{description}
%
%
%
%
%
%
%
\item[Author:]%
Konstantin Miller <miller@cs.tu-berlin.de>, Aug 09, 2005.
%
\end{description}
\methodline%
\subsubsection[Function \texttt{renameIdx}]{Function \texttt{functions/renameIdx}}%
\index[funcref]{functions@\fidxl{functions}!renameIdx@\fidxl{renameIdx}}%
\label{ref_functions__renameIdx}%
\hypertarget{ref_functions__renameIdx}{}%
\begin{description}
\item[Summary:]Rename one or more items in a database dimension (rows, columns, etc).
%
\item[Usage:]~%
\begin{lyxcode}%
new\_idx = renameIdx(old\_idx, old\_names, new\_names)
%
\end{lyxcode}%
%
\item[Description:]%
Prefer the convenience methods in tests\_db (renameColumns, renameRows) and
 tests\_3D\_db (renamePages). This is a cheap operation than modifies
 meta-data kept in object. For the regular expression renaming, the
 old\_names and new\_names parameters are passed to the regexprep command
 after removing the delimiting slashes (//). At least one grouping
 construct ('()') must be used in the search pattern such that it can be
 used in the replacement pattern (e.g., '\$1'). See example above.
%%
\item[Parameters:]~
\begin{description}%
\item[\texttt{old\_idx}:]
 An indexing structure (a\_db.col\_idx for columns, etc).
\item[\texttt{old\_names}:]
 A cell array of existing names, array of numerical indices, or a regular

expression denoted between slashes (e.g., '/(.*)/').
\item[\texttt{new\_names}:]
 New names to replace existing ones OR regular expression

replace string (no slashes, e.g, '\$1\_test'). See regexprep command.
\end{description}%
%
\item[Returns:
]~

   a\_db: The tests\_db object that includes the new names.
%
\item[Example:]~
\begin{lyxcode} % Renaming a single column:
\\%
 >> a\_db.col\_idx = renameIdx(a\_db.col\_idx, 'PulseIni100msSpikeRateISI\_D40pA', 'Firing\_rate');
\\%
 % Renaming an unnamed column:
\\%
 >> a\_db.col\_idx = renameIdx(a\_db.col\_idx, 1, 'Firing\_rate');
\\%
 % Renaming using regular expressions: add a suffix to all columns
\\%
 >> a\_db.col\_idx = renameIdx(a\_db.col\_idx, '/(.*)/', '\$1\_old');
\\%
 % Renaming multiple columns:
\\%
 >> a\_db.col\_idx = renameIdx(a\_db.col\_idx, {'a', 'b'}, {'c', 'd'});
\\%
 >> a\_db.col\_idx = renameIdx(a\_db.col\_idx, [1, 2], {'c', 'd'});
\\%
\end{lyxcode}
%
\item[See also:]%
\hyperlink{ref_tests_db__renameColumns}{\texttt{tests\_db/renameColumns}}%
\ (p.~\pageref{ref_tests_db__renameColumns})%
\index[funcref]{tests_db@\fidxl{tests\_db}!renameColumns@\fidxl{renameColumns}}%
, \hyperlink{ref_tests_db__renameRows}{\texttt{tests\_db/renameRows}}%
\ (p.~\pageref{ref_tests_db__renameRows})%
\index[funcref]{tests_db@\fidxl{tests\_db}!renameRows@\fidxl{renameRows}}%
, \hyperlink{ref_tests_3D_db__renamePages}{\texttt{tests\_3D\_db/renamePages}}%
\ (p.~\pageref{ref_tests_3D_db__renamePages})%
\index[funcref]{tests_3D_db@\fidxl{tests\_3D\_db}!renamePages@\fidxl{renamePages}}%
, \hyperlink{ref_regexprep}{\texttt{regexprep}}%
\ (p.~\pageref{ref_regexprep})%
\index[funcref]{regexprep@\fidxl{regexprep}}%
, \hyperlink{ref_allocateRows}{\texttt{allocateRows}}%
\ (p.~\pageref{ref_allocateRows})%
\index[funcref]{allocateRows@\fidxl{allocateRows}}%
%
\item[Author:]%
Cengiz Gunay <cgunay@emory.edu>, 2017/06/09
%
\end{description}
\methodline%
\subsubsection[Function \texttt{setAxisNonNaN}]{Function \texttt{functions/setAxisNonNaN}}%
\index[funcref]{functions@\fidxl{functions}!setAxisNonNaN@\fidxl{setAxisNonNaN}}%
\label{ref_functions__setAxisNonNaN}%
\hypertarget{ref_functions__setAxisNonNaN}{}%
\begin{description}
\item[Summary:]Returns the limits of the current axis replaced with the non-NaN elements of the given vector.
%
\item[Usage:]~%
\begin{lyxcode}%
new\_axis = setAxisNonNaN(layout\_axis)
%
\end{lyxcode}%
%
%
\item[Parameters:]~
\begin{description}%
\item[\texttt{layout\_axis}:]
 The axis position to layout this plot. 
\end{description}%
%
\item[Returns:
]~

	new\_axis: Modified axis.
%
\item[Example:]~
\begin{lyxcode} >> axis(setAxisNonNaN([0 100 NaN NaN])) % only change the x-axis limits
\\%
\end{lyxcode}
%
\item[See also:]%
\hyperlink{ref_plot_abstract}{\texttt{plot\_abstract}}%
\ (p.~\pageref{ref_plot_abstract})%
\index[funcref]{plot_abstract@\fidxl{plot\_abstract}}%
%
\item[Author:]%
Cengiz Gunay <cgunay@emory.edu>, 2007/10/29
%
\end{description}
\methodline%
\subsubsection[Function \texttt{sortedUniqueValues}]{Function \texttt{functions/sortedUniqueValues}}%
\index[funcref]{functions@\fidxl{functions}!sortedUniqueValues@\fidxl{sortedUniqueValues}}%
\label{ref_functions__sortedUniqueValues}%
\hypertarget{ref_functions__sortedUniqueValues}{}%
\begin{description}
\item[Summary:]Find unique rows in an already sorted matrix (or column vector).
%
\item[Usage:]~%
\begin{lyxcode}%
[rows, idx] = sortedUniqueValues(data)
%
\end{lyxcode}%
%
\item[Description:]%
Uses the derivation by Matlab diff function method.  Redundant with the
 Matlab function UNIQUE doing the same job:
 [rows, idx]= unique(data, 'rows', 'first'). 
 However, sortedUniqueValues is more efficient if the input data is already
 sorted for some other reason (see usage in tests\_db/invarValues).
%%
\item[Parameters:]~
\begin{description}%
\item[\texttt{data}:]
 A ascending row-sorted matrix or column vector.
\end{description}%
%
\item[Returns:
]~

   rows: A matrix or column vector of unique rows.
   idx: Indices of the unique rows in the original data matrix.
%
%
\item[See also:]%
\hyperlink{ref_uniqueValues}{\texttt{uniqueValues}}%
\ (p.~\pageref{ref_uniqueValues})%
\index[funcref]{uniqueValues@\fidxl{uniqueValues}}%
, \hyperlink{ref_unique}{\texttt{unique}}%
\ (p.~\pageref{ref_unique})%
\index[funcref]{unique@\fidxl{unique}}%
%
\item[Author:]%
Cengiz Gunay <cgunay@emory.edu>, 2004/09/27
%
\end{description}
\methodline%
\subsubsection[Function \texttt{string2File}]{Function \texttt{functions/string2File}}%
\index[funcref]{functions@\fidxl{functions}!string2File@\fidxl{string2File}}%
\label{ref_functions__string2File}%
\hypertarget{ref_functions__string2File}{}%
\begin{description}
\item[Summary:]Writes string verbatim into a file.
%
\item[Usage:]~%
\begin{lyxcode}%
string2File(string, filename, props)
%
\end{lyxcode}%
%
%
\item[Parameters:]~
\begin{description}%
\item[\texttt{string}:]
 To be written into file.
\item[\texttt{filename}:]
 The file to be created.
\item[\texttt{props}:]
 A structure with any optional properties.
\begin{description}%
\item[\texttt{append}:]
 If 1, append to existing file.
\end{description}%
\end{description}%
%
\item[Returns:
]~

%
%
\item[See also:]%
\hyperlink{ref_cell2TeX}{\texttt{cell2TeX}}%
\ (p.~\pageref{ref_cell2TeX})%
\index[funcref]{cell2TeX@\fidxl{cell2TeX}}%
%
\item[Author:]%
Cengiz Gunay <cgunay@emory.edu>, 2004/12/10
%
\end{description}
\methodline%
\subsubsection[Function \texttt{struct2DB}]{Function \texttt{functions/struct2DB}}%
\index[funcref]{functions@\fidxl{functions}!struct2DB@\fidxl{struct2DB}}%
\label{ref_functions__struct2DB}%
\hypertarget{ref_functions__struct2DB}{}%
\begin{description}
\item[Summary:]Converts a structure array to a tests\_db object.
%
\item[Usage:]~%
\begin{lyxcode}%
a\_tests\_db = struct2DB(a\_struct, props)
%
\end{lyxcode}%
%
\item[Description:]%
Field names become column names in the DB.
%%
\item[Parameters:]~
\begin{description}%
\item[\texttt{a\_struct}:]
 A structure to convert.
\item[\texttt{id}:]
 Optional database id string.
\item[\texttt{props}:]
 A structure with any optional properties, passed to tests\_db.
\end{description}%
%
\item[Returns:
]~

	a\_tests\_db: A tests\_db object.
%
%
\item[See also:]%
%
\item[Author:]%
Cengiz Gunay <cgunay@emory.edu>, 2008/01/11
%
\end{description}
\methodline%
\subsubsection[Function \texttt{struct2str}]{Function \texttt{functions/struct2str}}%
\index[funcref]{functions@\fidxl{functions}!struct2str@\fidxl{struct2str}}%
\label{ref_functions__struct2str}%
\hypertarget{ref_functions__struct2str}{}%
\begin{description}
\item[Summary:]Converts numerical structure into a single-line string.
%
\item[Usage:]~%
\begin{lyxcode}%
a\_str = struct2str(a\_struct, props)
%
\end{lyxcode}%
%
%
\item[Parameters:]~
\begin{description}%
\item[\texttt{a\_struct}:]
 Structure that has names pointing to numerical values.
\item[\texttt{props}:]
 A structure with any optional properties.
\end{description}%
%
\item[Returns:
]~

   a\_str: A string that contains structure fields and value like in 'name1\_val1\_name2\_val2\_...'
%
%
\item[See also:]%
\hyperlink{ref_cell2TeX}{\texttt{cell2TeX}}%
\ (p.~\pageref{ref_cell2TeX})%
\index[funcref]{cell2TeX@\fidxl{cell2TeX}}%
%
\item[Author:]%
Cengiz Gunay <cgunay@emory.edu>, 2011/05/17
%
\end{description}
\methodline%
\subsubsection[Function \texttt{subTextLabel}]{Function \texttt{functions/subTextLabel}}%
\index[funcref]{functions@\fidxl{functions}!subTextLabel@\fidxl{subTextLabel}}%
\label{ref_functions__subTextLabel}%
\hypertarget{ref_functions__subTextLabel}{}%
\begin{description}
\item[Summary:]Draws a text label on a plot.
%
\item[Usage:]~%
\begin{lyxcode}%
handle = subTextLabel(x, y, text\_str, props)
%
\end{lyxcode}%
%
%
\item[Parameters:]~
\begin{description}%
\item[\texttt{x, y}:]
 2D coordinates.
\item[\texttt{text\_str}:]
 String to be drawn on plot.
\item[\texttt{props}:]
 A structure with any optional properties.
\begin{description}%
\item[\texttt{Units}:]
 position units for the coordinates (see Units in axes properties).
\end{description}%
\end{description}%
%
\item[Returns:
]~

	handle: Text object handle.
%
%
%
\item[Author:]%
Cengiz Gunay <cgunay@emory.edu>, 2005/04/11
%
\end{description}
\methodline%
\subsubsection[Function \texttt{TeXfloat}]{Function \texttt{functions/TeXfloat}}%
\index[funcref]{functions@\fidxl{functions}!TeXfloat@\fidxl{TeXfloat}}%
\label{ref_functions__TeXfloat}%
\hypertarget{ref_functions__TeXfloat}{}%
\begin{description}
\item[Summary:]Places LaTeX content into a float (e.g., table, figure).
%
\item[Usage:]~%
\begin{lyxcode}%
tex\_string = TeXfloat(contents, caption, props)
%
\end{lyxcode}%
%
\item[Description:]%
Tabular contents can be created from cell arrays using
 cell2TeX. displayRowsTeX calls this function to make the float directly
 from database contents.
%%
\item[Parameters:]~
\begin{description}%
\item[\texttt{contents}:]
 Table contents in LaTeX.
\item[\texttt{caption}:]
 Table caption.
\item[\texttt{props}:]
 A structure with any optional properties.
\begin{description}%
\item[\texttt{rotate}:]
 Degrees to rotate.
\item[\texttt{width}:]
 Resize to this width.
\item[\texttt{height}:]
 Resize to this height
\item[\texttt{center}:]
 Align to center.
\item[\texttt{shortCaption}:]
 Short version of caption to appear at list of tables.
\item[\texttt{floatType}:]
 LaTeX float to use (default='table').
\item[\texttt{label}:]
 Used for internal LaTeX references.
\end{description}%
\end{description}%
%
\item[Returns:
]~

   tex\_string: LaTeX string for float.
%
\item[Example:]~
\begin{lyxcode} >> string2File(TeXfloat(cell2TeX({'a', 1; 'b', 2}), 'a basic table', ...
\\%
                 struct('rotate', 90, 'label', 'simple-table')))
\\%
\end{lyxcode}
%
\item[See also:]%
\hyperlink{ref_cell2TeX}{\texttt{cell2TeX}}%
\ (p.~\pageref{ref_cell2TeX})%
\index[funcref]{cell2TeX@\fidxl{cell2TeX}}%
, \hyperlink{ref_tests_db__displayRowsTeX}{\texttt{tests\_db/displayRowsTeX}}%
\ (p.~\pageref{ref_tests_db__displayRowsTeX})%
\index[funcref]{tests_db@\fidxl{tests\_db}!displayRowsTeX@\fidxl{displayRowsTeX}}%
%
\item[Author:]%
Cengiz Gunay <cgunay@emory.edu>, 2004/12/13
%
\end{description}
\methodline%
\subsubsection[Function \texttt{trace2cc}]{Function \texttt{functions/trace2cc}}%
\index[funcref]{functions@\fidxl{functions}!trace2cc@\fidxl{trace2cc}}%
\label{ref_functions__trace2cc}%
\hypertarget{ref_functions__trace2cc}{}%
\begin{description}
\item[Summary:]Converts a single-column trace vector into a current\_clamp object.
%
\item[Usage:]~%
\begin{lyxcode}%
a\_cc = trace2cc(a\_tr, cip\_times, cip\_vals, props) 
%
\end{lyxcode}%
%
%
\item[Parameters:]~
\begin{description}%
\item[\texttt{a\_tr}:]
 A trace object.
\item[\texttt{cip\_times}:]
 Start and end times of current injection [ms].
\item[\texttt{cip\_vals}:]
 A vector of current injection (CIP) parameter values 

[nA]. The number of the elements in this vector must 
be capable of dividing the length of the trace evenly.
\item[\texttt{props}:]
 A structure with any optional properties.
\begin{description}%
\item[\texttt{Ihold}:]
 [nA] Specifies holding current if different than first step value.
\item[\texttt{dt}:]
 Simulation time step of recorded data [s]. Use dt*nout of XPP (Default=1e-3).
\item[\texttt{paramsVary}:]
 Structure with variable name associated with an

array. If only one value is given cip\_vals, use the
values for this variable for the multiple trials found in file.
(others are passed to current\_clamp)
\end{description}%
\end{description}%
%
\item[Returns:
]~

   a\_cc: A current\_clamp object.
%
\item[Example:]~
\begin{lyxcode} The following creates a current clamp object from the cur\_inj45pA\_t
\\%
 trace object, a current step of holding at -10 to 0 nA at times 50 and
\\%
 500 ms. The props 'threshold' and 'paramsStruct' are passed to current\_clamp.
\\%
 >> a\_cc = trace2cc(cur\_inj45pA\_t, [50 500], 0, ...
\\%
                struct('threshold', 10, 'Ihold', -10, 'paramsStruct', ...
\\%
                       struct('gL\_nS', 7, 'gKs\_nS', 50.1, 'Cm\_pA', 5)));
\\%
\end{lyxcode}
%
\item[See also:]%
\hyperlink{ref_plotXPPparamRanges}{\texttt{plotXPPparamRanges}}%
\ (p.~\pageref{ref_plotXPPparamRanges})%
\index[funcref]{plotXPPparamRanges@\fidxl{plotXPPparamRanges}}%
, \hyperlink{ref_current_clamp}{\texttt{current\_clamp}}%
\ (p.~\pageref{ref_current_clamp})%
\index[funcref]{current_clamp@\fidxl{current\_clamp}}%
%
\item[Author:]%
Cengiz Gunay <cgunay@emory.edu>, 2011/03/04
%
\end{description}
\methodline%
\subsubsection[Function \texttt{uniqueValues}]{Function \texttt{functions/uniqueValues}}%
\index[funcref]{functions@\fidxl{functions}!uniqueValues@\fidxl{uniqueValues}}%
\label{ref_functions__uniqueValues}%
\hypertarget{ref_functions__uniqueValues}{}%
\begin{description}
\item[Summary:]Find unique rows in a matrix (or column vector). 
%
\item[Usage:]~%
\begin{lyxcode}%
[rows, idx] = uniqueValues(data)
%
\end{lyxcode}%
%
\item[Description:]%
Version which makes use of sort and diff. Maintains order of the
 original input.
%%
\item[Parameters:]~
\begin{description}%
\item[\texttt{data}:]
 A matrix or column vector
\end{description}%
%
\item[Returns:
]~

   rows: A matrix or column vector of unique rows.
   idx: Indices of the unique rows in the original data matrix.
%
%
\item[See also:]%
\hyperlink{ref_sortedUniqueValues}{\texttt{sortedUniqueValues}}%
\ (p.~\pageref{ref_sortedUniqueValues})%
\index[funcref]{sortedUniqueValues@\fidxl{sortedUniqueValues}}%
%
\item[Author:]%
Cengiz Gunay <cgunay@emory.edu>, 2004/09/24
%
\end{description}
\methodline%
\subsubsection[Function \texttt{updateErrorBars}]{Function \texttt{functions/updateErrorBars}}%
\index[funcref]{functions@\fidxl{functions}!updateErrorBars@\fidxl{updateErrorBars}}%
\label{ref_functions__updateErrorBars}%
\hypertarget{ref_functions__updateErrorBars}{}%
\begin{description}
%
%
\item[Description:]%
Code covered by the BSD License.
%%
\item[Parameters:]~
\begin{description}%
\item[\texttt{h}:]
 (Optional) Handle to figure or axis where to find errorbars (default=gca).
\item[\texttt{w, xtype}:]
 (Optional) Passed to errorbar\_tick.
\end{description}%
%
%
%
%
%
\end{description}
\methodline%
\subsubsection[Function \texttt{writeNeuronVecAscii}]{Function \texttt{functions/writeNeuronVecAscii}}%
\index[funcref]{functions@\fidxl{functions}!writeNeuronVecAscii@\fidxl{writeNeuronVecAscii}}%
\label{ref_functions__writeNeuronVecAscii}%
\hypertarget{ref_functions__writeNeuronVecAscii}{}%
\begin{description}
\item[Summary:]Writes ascii file to be read by Neuron simulator as Vector object.
%
\item[Usage:]~%
\begin{lyxcode}%
writeNeuronVecAscii(filename, datax, datay, dx, dy, unit\_x, unit\_y, label)
%
\end{lyxcode}%
%
\item[Description:]%
It's one line of code just to write the data: 
 dlmwrite(filename, ...
         [(0:(num\_samples - 1))'*dx*1e3, datay*1e-3], '-append', ...
         'delimiter', '$\backslash$t')
 Data converted to Neuron units of nA, mV, and ms.
%%
\item[Parameters:]~
\begin{description}%
\item[\texttt{filename}:]
 Full path to Neuron file.
\item[\texttt{datax}:]
 (Optional) X-axis points. Give empty vector to skip.
\item[\texttt{datay}:]
 Column or row vector of data.
\item[\texttt{dx}:]
 X-axis resolution in [s] or [V].
\item[\texttt{dy}:]
 y-axis resolution in [A] or [V].
\item[\texttt{unit\_x}:]
 Units of x-axis; 'V' or 's'.
\item[\texttt{unit\_y}:]
 Units of y-axis; 'A', 'V', or 's'.
\item[\texttt{label}:]
 Text label to export to Neuron (spaces will be replaced with '\_')
\end{description}%
%
\item[Returns:
]~

   Nothing.
%
\item[Example:]~
\begin{lyxcode}   writeNeuronVecAscii('myvec.dat', [], datay, 1e-4, 1e-3, 's', 'V', 'my membrane voltage');
\\%
\end{lyxcode}
%
%
\item[Author:]%
Cengiz Gunay <cengique@users.sf.net> 2012/03/23
%
\end{description}
\methodline%

